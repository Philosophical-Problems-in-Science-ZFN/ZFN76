\begin{newrevengenv}{Mateusz Czyżniewski}
	{Are there really any errors in the Austrian theory of welfare?}
	{Are there really any errors in the Austrian theory of welfare?}
	{Are there really any errors in the Austrian theory of welfare?}
	{Gdansk University of Technology\label{czyz-first}}
	{Dawid Megger, \textit{Sprawiedliwość 
	w~Ekonomii Dobrobytu, Liberatarianizm i~Szkoła Austriacka}, Wydawnictwo Naukowe UMK, Toruń, 2021.}
	
	















%Dawid Megger, \textit{Justice in Welfare Economics. Libertarianism and the Austrian School} (in Polish: \textit{Sprawiedliwość 
%w~Ekonomii Dobrobytu, Liberatarianizm i~Szkoła Austriacka}), Wydawnictwo Naukowe UMK, Toruń, 2021. \url{https://doi.org/10.12775/978-83-231-4689-6}









\section{Introduction}

\lettrine[loversize=0.13,lines=2,lraise=-0.03,nindent=0em,findent=0.2pt]%
{O}{}ne of the most controversial domains in economics is welfare (growth) economics, which encompasses both applied and theoretical aspects. Given the normative character of the deliberations conducted within this branch, it is unsurprising that a~number of issues that cut across positive economics, sociology, ethics, philosophy, or political science intersect in a~vast array of possible conclusions 
%\label{ref:RNDT15YZAIlgL}(Blaug, 1998).
\parencite[][]{davis_positive-normative_1998}.%




Supporters of a~free-market economy typically point out that both sides of transactions always benefit from voluntary market exchanges. One can venture even stronger assertions to the effect that voluntary exchanges enhance overall social well-being. Murray Rothbard, one of the most recognised representatives of the Austrian School of Economics (ASE), was the strongest advocate of this view 
%\label{ref:RND281nnabjyi}(Rothbard, 1998; [1956] 2008; [1962] 2009).
\parencites[][]{rothbard_ethics_1998}[][]{rothbard_toward_2008}[][]{rothbard_man_2009}. %
 To put it succinctly, Rothbard argues that it is unfeasible to establish a~universally applicable measure that is based on rigorous scientific principles and can gauge the satisfaction of individuals. Utility rankings \textit{ordinally} reflect the corresponding importance of subjectively framed ends rather than assign \textit{cardinal} numbers representing levels of satisfaction arising from the fulfilment of the said ends. These statements stand in contrast to mainstream economics, which involves optimisation, measurement, comparison, and utility calculation 
%\label{ref:RNDHLQhnxQmSp}(Samuelson, 1971, pp.173–183, 203–256 [orig. 1947]).
\parencite[][pp.~173--183, 203--256 \mbox{[orig. 1947]}]{samuelson_foundations_1971}. %
 As a~result, Rothbard's contribution cannot be overestimated, as ASE related scholars like Gordon 
%\label{ref:RND1AEneN6XM7}(1993),
\parencite*[][]{gordon_toward_1993}, %
 Herbener 
%\label{ref:RND4FYRLEkimm}(1997),
\parencite*[][]{herbener_pareto_1997}, %
 and Hoppe 
%\label{ref:RND1k2IOJNaLZ}([1993] 2006)
\parencite*[][]{hoppe_economics_2006} %
 readily acknowledge. Furthermore, Rothbard developed his own welfare theory, which was heavily based on the Pareto efficiency rule, also referred to as ``unanimity rule'', and the doctrine of demonstrated preference 
%\label{ref:RNDcL1wx8i06I}(Rothbard, 2008 [1956])
\parencite[][]{rothbard_toward_2008}.%
\footnote{Henceforth, his theorems are called Rothbard's Austrian Welfare Economics (RAWE).} The main idea is that, without drawing any moral (normative) conclusions, we can determine, using the Unanimity Rule and the concept of demonstrated preference, that:



\begin{enumerate}

\item 
Free-market (voluntary) transactions always improve society's welfare.

\item 
Government interference can never raise social welfare.

\end{enumerate}

Even if the Rothbardian theorems were regarded by ASE representatives as untroubled for many years, some voices inside the Austro-Libertarian community present critical views on that matter, e.g. 
%\label{ref:RNDZjWkgC15SU}(Prychitko, 1993; Gunning, 2005; Kvasnička, 2008; Wysocki, 2023; Wysocki and Dominiak, 2023).
\parencites[][]{prychitko_formalism_1993}[][]{gunning_did_2005}[][]{kvasnicka_rothbards_2008}[][]{wysocki_how_2023}[][]{wysocki_how_2023}. %
 Dawid Megger joined a~group of opponents of Rothbard's classical approach a~few years ago in his book \textit{Justice in Welfare Economics. Libertarianism and the Austrian School} (in Polish: \textit{Sprawiedliwość w~Ekonomii Dobrobytu, Liberatarianizm i~Szkoła Austriacka})\textit{,} which delves deeply into welfare theory, offers constructive criticism of RAWE, and introduces some original concepts 
%\label{ref:RNDzxWQJS3FB0}(Megger, 2021)
\parencite[][]{megger_sprawiedliwosc_2021}.%
\footnote{The original edition of the book does not contain an English translation of the title, so the author of the review allowed himself to translate the Polish title \textit{Sprawiedliwość w~Ekonomii Dobrobytu, Liberatarianizm i~Szkoła Austriacka} into English literally. Subsequent sections of the review make reference to the book by the abbreviation \textit{Justice} or Megger's book.} This review will concentrate on the previously mentioned work.



The review is divided into the following sections. Section 2 involves a~slight introduction of Megger's scientific achievements and career details. Section 3 presents and gradually discusses the review's most significant observations. Section 4 concludes the discussion. Section 5, which contains the bibliography, concludes.



\section{A~few words about the author and his works}

In 2023, Megger earned his doctoral degree with honours for his dissertation, \textit{The Austrian School of Economics as a~causal-realist research program}. \textit{Methodological investigations}. In 2021, he published a~book entitled \textit{Justice in Welfare Economics. Libertarianism and the Austrian School} 
%\label{ref:RNDPgg0eHMrwD}(Megger, 2021),
\parencite[][]{megger_sprawiedliwosc_2021}, %
 which built on his master's thesis \textit{Austro-Libertarian Welfare Economics and its Aporias} 
%\label{ref:RNDmVBfsGgMWC}(Megger, 2023),
\parencite[][]{megger_austriacka_2023}, %
 defended in 2019. The master's thesis itself was awarded in a~contest conducted within the Faculty of Economic Sciences and Management of Nicolaus Copernicus University in Toruń for the best master's thesis in 2019.



Despite the passage of several years since the publication of this book and the appearance of multiple articles on Megger's account concerning the aforementioned issues of the welfare economy, such as Wysocki and Megger 
%\label{ref:RNDszwEZPmlgM}(2019),
\parencite*[][]{wysocki_austrian_2019}, %
 Wysocki and Megger 
%\label{ref:RNDeCuVOUY7yG}(2020),
\parencite*[][]{wysocki_rejoinder_2020}, %
 and Megger and Wysocki 
%\label{ref:RNDO2bjbyBfx9}(2023),
\parencite*[][]{megger_austriacka_2023}, %
 this review pertains exclusively to the aforementioned book. The author's intention is not to make a~critical cross-section of Megger's achievement or the relationship between the book and the articles, but rather to refer to the theses presented in \textit{Justice}. In fact, during those few years, two critics presented their arguments against him and his collaborator's statements. Both Wiśniewski 
%\label{ref:RNDAtgAhltIH0}(2019)
\parencite*[][]{wisniewski_austrian_2019} %
 and Juszczak 
%\label{ref:RNDq6tcgSI02j}(2021)
\parencite*[][]{juszczak_o_2021} %
 claimed that Megger's doubts put on the classical RAWE misfire and are not properly methodologically justified. However, as the author of this review, my goal is not to confront their statements with Megger's reasoning but rather to present my original assessment of the latter's work.



\section{Substantive assessment of the book content}

This particular part of the text contains the most important remarks that the reviewer wants to emphasise in the context of the book's substantial content.



\subsection{Chapter one: \textit{Introduction}}



Let's start with the first chapter of the book. Section 1.1, called \textit{The Problematic} (Pol: \textit{Problematyka}), aims to outline the ideas and objectives of the work, briefly introducing the reader to the framework and general considerations related to welfare economics, especially from the perspective of ASE (pp. 13–17).



Next, the author decides to describe the research objectives, which he introduces in Section 1.2 called Research \textit{Aims} (Pol: \textit{Cele}), pages 17–19. Megger gives the following statement: ``[…] Our aim is not to completely dismiss the Austro-Libertarian theory of wealth, but simply to demonstrate its inaccuracies and to make selected claims more justified. […]''. More specifically, the author aims to demonstrate that: 1) Rothbard's welfare theory often relies on hidden assumptions and circular reasoning; 2) economic effectiveness does not necessarily hinge on voluntariness; and 3) the praxeological research can incorporate game-theoretical considerations and the expected utility methodology.



The last part of the first chapter, i.e., 1.3, called \textit{Methodology} (Pol: \textit{Metoda}), provides a~description of the methodology that the author applied to the intended research objectives 
%\label{ref:RNDuMoAc83e5t}(Megger, 2021, pp.19–24).
\parencite[][pp.19–24]{megger_sprawiedliwosc_2021}. %
 In a~way that fits with the ASE tradition,. Megger stresses how necessary and important it is to use the praxeological method, which is familiar from von Mises's writings 
%\label{ref:RNDwLb7Cmjn3f}([1957] 1997; [1949] 1998).
\parencites*[][]{mises_theory_1997}[][]{mises_human_1998}. %
 In addition, the author emphasises the need to carry out research using methods characteristic of rationalist philosophical traditions, rejecting positivist-styled empiricism or instrumentalism based on measurements, experiments, or statistical analysis. It is regrettable that at this stage of the book, even if it is only an introduction, Megger did not outline some general methodological aspects, referring, for example, to such works as Robbins 
%\label{ref:RNDi5E1UGDXKm}(1932),
\parencite*[][]{robbins_essay_1932}, %
 Mises 
%\label{ref:RND4TQJANmTK9}(1962),
\parencite*[][]{mises_ultimate_1962}, %
 Lachmann 
%\label{ref:RNDMmVs8CwjzR}(1971),
\parencite*[][]{lachmann_legacy_1971}, %
 Machlup 
%\label{ref:RNDAb6gp52P4b}(1978),
\parencite*[][]{machlup_methodology_1978}, %
 Hausman 
%\label{ref:RNDSdgrz9MvIA}(1995),
\parencite*[][]{hausman_impossibility_1995}, %
 Mises 
%\label{ref:RNDISfzJcCgQt}([1957] 1997),
\parencite*[][]{mises_theory_1997}, %
 Mises 
%\label{ref:RNDvAP10EktzZ}([1949] Mises, 1998),
\parencite*[][]{mises_human_1998}, %
 Hoppe 
%\label{ref:RND4bfC8QYSsF}([1993] 2006),
\parencite*[][]{hoppe_economics_2006}, %
 and O'Driscoll and Rizzo 
%\label{ref:RNDI215IELdbK}(2014).
\parencite*[][]{odriscoll_austrian_2014}.%




\subsection{Chapter two: \textit{Economic efficiency and the issue of rational social order}}



The second chapter makes some state-of-the-art claims and presents some introductory issues. Section 2.1, called \textit{Wealth and Prosperity in the History of Economic Thought} (Pol: \textit{Bogactwo i~Dobrobyt w~Historii Myśli Ekonomicznej})\textit{,} conducts a~selective review of the literature focused on the evolution of welfare economics 
%\label{ref:RND5Iq7Mkx47C}(Megger, 2021, pp.25–34).
\parencite[][pp.25–34]{megger_sprawiedliwosc_2021}. %
 In fact, this review should be broader and more extensive in terms of methodology. Aspects such as the transformation of the neoclassical school into a~mathematical one, the addition or comparison of preferences, and the lack of neutrality in redistributive acts should be highlighted.



Next, the author of the book, in Section 2.2 called \textit{The Austrian Approach: Praxeology} (Pol: \textit{Stanowisko austriackie: Prakseologia}), presents some arguments in favour of the methodology characterizing ASE 
%\label{ref:RNDsKFzlLDSZh}(Megger, 2021, pp.34–43).
\parencite[][pp.34–43]{megger_sprawiedliwosc_2021}. %
 The review itself is appropriate from a~substantive standpoint, and Megger's original comments prove valuable. This section incorporates many of the ASE's key points, and it serves as a~concise summary of the Austrian framework.



Section 2.3, called \textit{The Problem of Socialism} (Pol: \textit{Problem Socjalizmu})\textit{,} describes issues related to the problem of economic calculation. A~special emphasis is put on extreme economic system, i.e., \textit{pure free-market capitalism} and \textit{fully controlled socialism} 
%\label{ref:RNDCUH6Dh1EOu}(Megger, 2021, pp.43–49).
\parencite[][pp.43–49]{megger_sprawiedliwosc_2021}. %
 Even though this section has been described flawlessly and in some way complements the considerations on the welfare economy, I~think this book would do equally well without it, and valuable insights can be transferred to sections 2.2 and 2.4.



The last section of the second chapter, i.e., 2.4, called \textit{Libertarian Solution} (Pol: \textit{Rozwiązanie Libertariańskie}), addresses the libertarian solution to the problem of establishing an efficient economic system 
%\label{ref:RNDjmlXfO3QCk}(Megger, 2021, pp.48–49).
\parencite[][pp.48–49]{megger_sprawiedliwosc_2021}. %
 Megger, after introducing Rothbard's most important works on the libertarian order and its justification 
%\label{ref:RND1bN9BCE02x}(Rothbard, 1998; [1956] 2008),
\parencites[][]{rothbard_ethics_1998}[][]{rothbard_toward_2008}, %
 elucidated the most essential doubts associated with this theory. The author of the book refers to the issues of space, body, and the scarcity of resources (finite amounts), which spells potential conflicts between people 
%\label{ref:RNDL2KFqEO1Qt}(Gordon, 1993; Herbener, 1997; Hoppe, 2006, pp.311–330, 341–345; Wiśniewski, 2019).
\parencites[][]{gordon_toward_1993}[][]{herbener_pareto_1997}[][pp.311–330]{hoppe_economics_2006}[][]{wisniewski_austrian_2019}. %
 To resolve the fundamental issues in terms of property rights, RAWE supporters claim that a~full-free market society is the ultimate solution, allowing efficient economic calculation and then fair and productive goods allocation. Nevertheless, the author of the book, at the very end of the chapter, formulates interesting statements regarding potential errors in RAWE, which he plans to develop in the third chapter.



\section{Chapter three: \textit{The issues of Austro-libertarian welfare economics}}

This whole chapter 
%\label{ref:RNDSp7tSwKvdi}(Megger, 2021, pp.59–102)
\parencite[][pp.59–102]{megger_sprawiedliwosc_2021} %
 presents the most original content of the book, where Megger's critical claims against RAWE are presented.



\subsection{\textit{Ex ante} and \textit{ex post} analysis}



Section 3.1 is entitled \textit{Ex ante and ex post analysis. Voluntary exchange and mutually beneficial exchange} (Pol: \textit{Analiza ex ante i~ex post. Wymiana dobrowolna a~wymiana wzajemnie korzystna}) refers to an attempt to criticise the statement that any voluntary exchange is \textit{ex ante} always mutually beneficial and involuntary exchanges must be considered unfavourable 
%\label{ref:RND7jENv9hWik}(Megger, 2021, pp.59–73).
\parencite[][pp.59–73]{megger_sprawiedliwosc_2021}. %
 Firstly, the text outlined the issues associated with actual physical possession vis-à-vis property rights. Megger believes that in a~voluntary exchange, so that all parties should benefit and no one should lose, it is necessary not so much to transfer property rights as not to violate them 
%\label{ref:RNDE10qKQp035}(Megger, 2021, p.59).
\parencite[][p.59]{megger_sprawiedliwosc_2021}. %
 As long as the ``physical'' transfer of goods can naturally infringe on ownership rights, is this conclusion trivial from the point of view of the RAWE doctrine? The thieves-case serves as an essential example to deliver an appropriate justification. If an exchange among thieves enhances their psychical ``utility'', what about the exchange-related utility of those deprived of property? In their view, \textit{ex ante} transferring goods to others was not a~preferred act, and subsequently, someone stole their property.



If, according to the author of the book, ``[…] this is not irrelevant to the utility function of the people […]'' 
%\label{ref:RNDzMy938dfwt}(Megger, 2021, p.60)
\parencite[][p.60]{megger_sprawiedliwosc_2021}%
\footnote{What does the utility function mean? Is there something in the mathematical framework? If yes, is it a~static (time-invariant) or dynamic (time-related) construct? }, then how do we define the type of ``usefulness'' related to goods acquired voluntarily or stolen? Isn't this sometimes an obvious violation of Pareto's orpimality rule, defined both as in RAWE 
%\label{ref:RNDPTNvsfEhMf}(Rothbard, 2008 [1956])
\parencite[][]{rothbard_toward_2008} %
 and using the mathematical neoclassical framework 
%\label{ref:RNDP7DOtIoRq0}(Samuelson, 1971, pp.173–183, 203–256 [orig. 1947])?
\parencite[][pp.~173--183, 203--256 \mbox{[orig. 1947]}]{samuelson_foundations_1971}? %
 Doesn't this create a~problem of psychologizing and comparing utility if some divergent types of ``exchange`` are considered 
%\label{ref:RNDTahCVDXhGF}(Hausman, 1995)?
\parencite[][]{hausman_impossibility_1995}? %
 What about the temporal aspect? Specifically, does considering an extensive situation with numerous potential outcomes violate the foundations of a~``thought experiment''? In terms of psychological investigations, there is a~potential for specific research related to pragmatically done analysis and thymological issues 
%\label{ref:RNDeUtevmnR0k}(Mises, 1997, pp.264–284, 303–320 [1\textsuperscript{st} ed. 1957]).
\parencite[][pp.~264--284, 303--320 \mbox{[1\textsuperscript{st} ed. 1957]}]{mises_theory_1997}. %
 However, any significant violation of RAWE principles requires a~more detailed discussion. Perhaps we should elaborate on and justify a~clear definition of coercion, contrasting it with direct property right violations.



Certainly, thieves consciously and \textit{voluntarily} carry out their actions with the intention of stealing property from a~person who desires to exchange it and who does not prefer the ``loss`` of this property. The very fact that a~person owns and possesses a~given good and does not give up on it shows that his holding demand (reservation demand and transactional demand) exists in this context, and he perceives the potential uses of that good, no matter whether it is production, consumption, or sale 
%\label{ref:RND3RUv74NHBp}(Rothbard, 2009, pp.137–142).
\parencite[][pp.137–142]{rothbard_man_2009}. %
 Therefore, if there is no mutual exchange (as defined by Böhm-Bawerk)\footnote{To be found in, e.g., Mises 
%\label{ref:RNDg3p4cRirXY}([1949] 1998, pp.213–232, 268–316)
\parencite*[][pp.213–232, 268–316]{mises_human_1998} %
 and Rothbard 
%\label{ref:RNDrHAa56Y8xF}([1962] 2009, pp.95–169).
\parencite*[][pp.95–169]{rothbard_man_2009}. %
 }, it cannot be possible to discuss and quantitatively compare the ``increase'' in utility after theft among thieves and the apparent decrease in utility among the robbed from an \textit{ex ante} perspective.



Furthermore, certain aspects of exchanges between residents and workers require specificity. All these individuals function within a~specific property rights environment, which necessitates their establishment prior to any exchanges. It should be remembered that both workers and residents agree to exchange money for the completion of a~certain current or futures transaction. In this sense, the successive transfer of equipment or belongings to the owner does not interfere in any way with the specified provisions within the context of property rights. Rather, ``exchanges'' occur within the context of certain property rights, allowing them to swap ownership between parties. Regardless of whether such exchanges resolve, deny, or modify the conditions of previous agreements, the utility of these individuals increases. The exchange on the so-called time market 
%\label{ref:RNDOANOtObST5}(Rothbard, 2009, pp.390–410),
\parencite[][pp.390–410]{rothbard_man_2009}, %
 which refers to the temporal structure of exchange between the suppliers of future goods and the present goods, is particularly significant.



Next, the author of the book analyses the problem of blackmail and the productivity of exchange 
%\label{ref:RND47Vnauybpx}(Megger, 2021, pp.62–67).
\parencite[][pp.62–67]{megger_sprawiedliwosc_2021}. %
 In Megger's opinion, blackmail, as a~form of threat, cannot be perceived as potentially mutually beneficial, even if it is \textit{voluntarily} carried out in order to get rid of the blackmailer. The problem is highlighted here: such a~``voluntary exchange'' can lead to negative consequences and have similar effects to actions undertaken under severe threat, i.e., actions of a~non-voluntary and essentially harmful, property-violating nature. Megger gives two examples with a~similar analytical structure to illustrate some issues with RAWE applied to the threat problem 
%\label{ref:RNDoPrUaOluOp}(Megger, 2021, p.63).
\parencite[][p.63]{megger_sprawiedliwosc_2021}. %
 In both cases, you have to pay \$10,000 for successively removing the unwanted influence of the blackmailer and fulfilling the ``obligation'' to the tax collector. However, in the first situation, blackmail does not have a~direct and unavoidable effect linked to violence and further state sanctions, as in the second case of an action undertaken by a~tax collector. In a~sense, the influence of the blackmailer has a~chance to affect the feelings of his victim, but does the attempt to solve this problem not involve psychologizing or a~violation of the principle of \textit{ceteris paribus}? Is this in any way linked to the catallactic properties defined by Mises 
%\label{ref:RNDOba4ohrYMo}(1998, pp.233–257)?
\parencite*[][pp.233–257]{mises_human_1998}? %
 Why is it necessary to immediately associate ``negative feelings`` with the utility changes that occur during the voluntary exchange, as understood in the context of RAWE? Furthermore, what counterfactual scenarios could we regard as possible and not subject to this type of reasoning? While these scenarios are essentially different, the book's author believes that their effects on welfare are similar, necessitating a~correction of Rothbard's considerations. Furthermore, he criticises the attempt to explain this problem on the basis of alternative costs 
%\label{ref:RNDM24cRSJcyW}(Megger, 2021, p.64),
\parencite[][p.64]{megger_sprawiedliwosc_2021}, %
 showing that comparing these scenarios to each other must involve a~utility comparison rather than a~qualitative structural analysis. It also highlights some issues with the concept of demonstrated preference. Here, I~have to agree with Megger that Rothbard's theory leaves some gaps, even significant ones. I~think in this case, referring to alternative costs and not comparing utility in the context of invoking the two scenarios may be a~bit of a~double-edged sword in terms of maintaining the coherence of the argumentation. However, there is a~sort of example where, if there is the possibility of experiencing ``worse'' consequences, people would prefer to pay taxes, perceiving this action as unfavourable but not as bad as government sanctions. Still, there is some preference demonstration, but not as Rothbard would claim, and there is still a~gap to be filled.



Megger proposes to solve this problem by applying the Nozickian 
%\label{ref:RNDc1c2eJc1z9}(Megger, 2021, pp.66–67)
\parencite[][pp.66–67]{megger_sprawiedliwosc_2021} %
 tripartite division of exchanges 
%\label{ref:RND6H1V1cbxGi}(Nozick, 1999, pp.84–87 [orig. 1974]).
\parencite[][pp.84–87 \mbox{[orig. 1974]}]{nozick_anarchy_1999}. %
 I~think that while this methodology presents a~number of differences compared to RAWE, its application to the theory of welfare may have interesting implications, but it requires some additional assumptions and considerations. I~agree with Juszczak 
%\label{ref:RNDTxzlTpfyvL}(2021)
\parencite*[][]{juszczak_o_2021} %
 that those frameworks have some issues when it comes to comparing utility in a~psychological sense. Moreover, other doubts include the inability to clearly define the difference between ``sharp boundary conditions associated with a~given situation'' that can be involved, the development of causal links over time, and, more importantly, deciding what productivity means, what kind of one-sided benefits can really be claimed as advantageous, and what psychological and social factors need to be considered. Nozick's theory offers numerous benefits in specific cases, but it doesn't solve the problem of debunking Rothbard's theory in general.



I~agree with Megger on the statements relating to the continuity or simply conversion of preferences 
%\label{ref:RNDRUa5JAjHdn}(Megger, 2021, pp.68–69).
\parencite[][pp.68–69]{megger_sprawiedliwosc_2021}. %
 That being said, Austrian economists use the value scale or the preference list as a~``model'' to explain what actions are entirely about and identify what people really desire. Thus, it would be wrong to say that there is no mental connection between value scales and the temporal evolution of actions, which involves other preference scales. However, an explanation is necessary to understand the attachment of specific preferences to the ``behavioural'' aspect of action, as well as their appearance and disappearance. But still, preferences cannot exist separately from the actions, because even the acts of thinking, reminding, deliberating, or planning are actions \textit{par excellence} 
%\label{ref:RNDCNhvuedUtU}(Mises, 1998, pp.11–72).
\parencite[][pp.11–72]{mises_human_1998}. %
 I~believe that individuals subconsciously recall past preferences, or entire preference scales, and then apply them during specific action evaluations when a~subjective link between them is present. However, it does not allow for separating some \textit{preferences} from \textit{actions}.



Then, the author of the book starts dealing with the \textit{ex post} analysis 
%\label{ref:RNDvh2Hca1Wtf}(Megger, 2021, pp.70–72).
\parencite[][pp.70–72]{megger_sprawiedliwosc_2021}. %
 On the one hand, Megger agrees with Rothbard's supporters, but he also introduces some comparative aspects between the \textit{ex post} and \textit{ex ante} domains. In my opinion, the author's considerations are correct on the one hand, and, on the other hand, they do not dismiss RAWE completely. While the question of whether it is possible to find genuinely positive aspects of certain ``interactions'' with the violent side is a~valid one, some of the conclusions drawn are overly broad. Furthermore, the general scheme is entangled in certain problems. First, the economic actor's choice of means is always subjective. Hence, the evaluation of the selection's correctness is always rational \textit{ex ante} (even if it is meaningless in strictly physical or conceptual terms)\footnote{See Mises 
%\label{ref:RNDbwPKmBlvy1}([1949] 1998, pp.13–23)
\parencite*[][pp.13–23]{mises_human_1998} %
 and 
%\label{ref:RNDoUAXjbqNek}([1957] 1997, pp.264–271).
\parencite[][pp.264–271]{mises_theory_1997}.%
}. Second, if person A~is doomed to make a~mistake but does not know it, then why does the ``mechanism'' of the thought experiment preclude the implementation of plan modification? Third, doesn't the author himself fall into Rothbard's trap of comparing utility in a~quasi-qualitative manner? Fourth, why are the situation's dynamics so limited? And fifth, why isn't the actor able to act intentionally, meaning that they are fully aware of the potential consequences of committing a~murder but still choose to proceed? In my view, this specific reasoning holds significance in such areas as action support systems known from computer science and control engineering, ergonomics, or game theory concepts, yet it is weakly related to the Rothbardian framework.



\subsection{\itshape Preference and risk}



Section 3.2, entitled \textit{Preference and Risk} (Pol: \textit{Preferencja a~ryzyko),} deals with eponymous preference and risk 
%\label{ref:RND7mRfs39JEr}(Megger, 2021, pp.73–82).
\parencite[][pp.73–82]{megger_sprawiedliwosc_2021}. %
 First, Megger attempts to deal with the issue of the coherence between demonstrated preference and pure choice, which constitute the essence of human action. He claims that ``[…] the more desirable the praxeological statement is to admit that the acting person prefers not the highest valued goal but the highest valued action'' 
%\label{ref:RNDnEVN4enFF2}(Megger, 2021, p.74),
\parencite[][p.74]{megger_sprawiedliwosc_2021}, %
 and then turns into the theory of expected utility. In his opinion, it is possible to make this connection by combining the theory of action with risk assessment methodologies.



The book's author introduces some examples 
%\label{ref:RNDTh1LrdevNn}(Megger, 2021, pp.79–81)
\parencite[][pp.79–81]{megger_sprawiedliwosc_2021} %
 to illustrate his concept, linking risk assessment with time preference 
%\label{ref:RNDsc3DkpPVmA}(Mises, 1998, pp.476–486; Rothbard, 2009, pp.13–17, 49–56).
\parencites[][pp.476–486]{mises_human_1998}[][pp.13–17, 49–56]{rothbard_man_2009}. %
 The two initial examples are almost canonical, and thus no controversy arises as far as the Austrian methodology goes. The following cases, however, show the ``Austrian theory of expected utility'', which substitutes risk assessment for time preference. People with particular risk evaluations will always choose less risky activities when the financial cost and revenue are the same in both cases (\textit{ceteris paribus} clause), so risk assessment ``takes the place of time.'' This is also noncontroversial in terms of the expected utility theorems. The next situation seems to be convincing because, \textit{ceteris paribus}, \$1,000 is associated with a~lower risk, whereas \$10,000 is associated with a~higher level of risk, and thus the risk preference is the key factor in selecting a~particular action. Despite the subjective perception of risk, it's crucial to keep in mind that these assessments, which are merely human projections or models of reality, must have a~strong connection to actual, empirical phenomena.



In fact, even with a~future assessment, the conditions resulting from a~certain risk level could change, and these conditions are not the same as those associated with changes in time preference. It's possible that certain factors could make the second investment less risky. Furthermore, why is it impossible to integrate risk assessments into the straightforward and conventional theory of choice, also known as the Misesian theory of action? People use different methods to attempt to accomplish different goals in different circumstances, so why do we need to distinguish the ``risk category'' as another relevant factor affecting human action? When discussing subjective assessments, we often refer to risk. However, why not consider ``pure uncertainty'' or single-case probability 
%\label{ref:RNDAo19aEu1GW}(Mises, 1998, pp.105–118)?
\parencite[][pp.105–118]{mises_human_1998}?%




Also, what about the cases where the time preference factor would be complementary to the risk assessment perceived in this way? How do we quantitatively evaluate the action's profitability? Is it necessary to make an analytical distinction between those two factors? How can we integrate the deductively described pure time preference into the risk assessment framework, given the numerous tools available for probability calculation?



I~think those attempts at connecting time preference with risk preference sound very interesting and would gain some added value, but this description is not comprehensive and would apply only to some constrained aspects of the theory of action. However, the presented reasoning, which takes into account certain specific assumptions, is coherent and clear.



\subsection{Maximising of utility and social dilemmas}



Section 3.3, entitled \textit{Maximising of utility and social dilemmas} (Pol: \textit{Maksymalizacja użyteczności a~dylematy społeczne}), deals with the issue of maximising utility in specific social contexts 
%\label{ref:RND0oHv0dEgAo}(Megger, 2021, pp.82–94).
\parencite[][pp.82–94]{megger_sprawiedliwosc_2021}. %
 The author of the book, at the very beginning of the chapter 
%\label{ref:RNDnAHaxVnTSo}(Megger, 2021, pp.83–86),
\parencite[][pp.83–86]{megger_sprawiedliwosc_2021}, %
 gives an introduction to the concept of combining praxeology with some elements of game theory. However, as I~properly understood this concept, there is a~chance of making some particular type of game theory framework possible to be interpreted in terms of praxeological reasoning. Even though some ASE representatives had doubts about how this method could be used, Megger says, ``What distinguishes game theory from praxeology will be the fact that it is a~subcategory of the general decision-making theory […], whereas praxeology emphasises »action as such« […]'' 
%\label{ref:RNDtoMuQNkbKu}(Megger, 2021, p.85).
\parencite[][p.85]{megger_sprawiedliwosc_2021}. %
 We have to agree with this because the basic ideas of game theory allow for a~formal comparison or measurement of utility in certain specified, ``constructed'' and controlled scenarios and situations. The proposed ideas suggested replacing cardinal numbers, which describe quantifiable utility values, with order numbers connected to scheduled actions. The author presents two typical types of games, the prisoner's dilemma (pp. 87-88) and the confidence dilemma 
%\label{ref:RNDgNmUnQMtZi}(Megger, 2021, pp.89–91).
\parencite[][pp.89–91]{megger_sprawiedliwosc_2021}.%




The presented praxeological analysis, in my opinion, construes the game theory in a~proper manner, enabling it to transcend certain rigid schemes. However, this interpretation still requires that the phenomena analysed be within a~fixed and persistent context, which shows some limitations in the application of this methodology. Although the explanations based on real scales of values are valuable, it is necessary to treat the game theory as ``lower in hierarchy'' than the general claims of praxeology. Furthermore, Megger controversially says that ``[…] we will use the subjective degree of conviction that certain events can occur to explain why people who »take part in games« will undertake certain actions'' 
%\label{ref:RNDPEsAb0NuHP}(Megger, 2021, p.92).
\parencite[][p.92]{megger_sprawiedliwosc_2021}. %
 How do we understand subjective probability here? Game theory cannot work unless there is a~certain ``top-up coordination'' or a~set of conditions that maintain a~given situation in the game model frames. On the other hand, I~believe this is due to the possibility of players having ``outside'' feelings about the game scheme, additional costs associated with participating in the game, and different valuations in a~strictly axiological sense. Then there's no controversy.



Nevertheless, the inclusion of certain aspects of game theory in the general praxeological framework should be considered successful and merit further development, despite some trouble. In particular, the author of the book emphasises on page 90 that ``compliance'' in the course of the game, which can naturally occur in the real world, ``disrupts'' the game in an ontological way, i.e., its principles do not fit the temporary model of the real situation.



Further on, Megger 
%\label{ref:RNDio21YOWIAt}(2021, pp.91–93)
\parencite*[][pp.91–93]{megger_sprawiedliwosc_2021} %
 refers to recursive problems, where he analyses the prisoner's dilemma for a~finite number of rounds in which players know the amount of iteration and in which they don't. Here too, we must agree with the author's analysis. However, the problem remains that the analyses presented are not as general and important as praxeological theorems.



\subsection{Intellectual property and welfare}



The last subsection 3.4., called \textit{Intellectual Property and Welfare} (Pol: \textit{Własność Intelektualna a~Dobrobyt}), involves aspects of intellectual property rights and welfare 
%\label{ref:RNDuh7Fxc4TQZ}(Megger, 2021, pp.94–101).
\parencite[][pp.94–101]{megger_sprawiedliwosc_2021}. %
 After providing an appropriate introduction to various positions in this field, particularly the libertarian perspective based on RAWE principles\footnote{The libertarian philosophy generally opposes intellectual property rights, viewing them as significant restrictions placed on the economy by the government. Indeed, those regulations have a~vital effect on economic development.}, the discussion shifts to the rate of development, the spread of innovation, and technical implementations. Megger convinces us that he supports Rothbard's claim that patents, copyright, and other similar legislation can greatly divert the monetary streams spent on diverse research, which greatly affects the process of ``real'' innovation emergence 
%\label{ref:RNDX7eyMyCvE0}(Megger, 2021, p.97).
\parencite[][p.97]{megger_sprawiedliwosc_2021}. %
 However, he later expresses some doubts about this claim. On the one hand, it is true that potential intellectual property protection would greatly benefit the particular companies. These institutions, thanks to the greater profits from monopoly rent, were able to accumulate a~greater amount of funds, which they could then dedicate to the development or expansion of their production base. However, comprehending innovation and determining which inventions or concepts to ``test in practice`` in a~developing economy pose challenges.



In fact, the game theory scheme presented 
%\label{ref:RNDmMdDTckMDa}(Megger, 2021, p.98)
\parencite[][p.98]{megger_sprawiedliwosc_2021} %
 is interesting, and the lessons drawn from it are informative, but such methods or algorithms would rather support potential decisions if their mathematical structure were more extensive. Of course, this ``dilemma of innovation implantation'' is correct in certain stable and predictable conditions, but in its current form, it is too simplistic. Perhaps some dynamical game theory tools, as well as some estimation tools associated with risk assessment, would be more appropriate. Furthermore, the implementation of this scheme does not contradict any assertions about a~free-market or regulated economy, meaning that the conclusions drawn from the presented reasoning can be applied to various types of economic environments. Only when certain conditions are satisfied in a~specific manner does a~framework for decision support emerge.



However, from the ASE perspective, the introduction of technical, organisational, and other innovations always requires the appropriate adaptation to the current production structure, both in real and intellectual terms. In order for the ``innovation'' understood in any way to become a~true development, it must be matched to an existing combination of complementary goods closed and coordinated under the given institutional circumstances---for example, in a~production company 
%\label{ref:RNDrdDkngyQbV}(Bylund, 2015).
\parencite[][]{bylund_explaining_2015}. %
 Moreover, the external aspect necessitates adjustment, where the production of new or improved goods consistently faces the pressure of profit-driven economic calculations. In a~market economy, we always verify innovation for goods that are closer to consumption within a~specific production structure, never \textit{in vacuo} 
%\label{ref:RNDVc5l6VCw64}(Rothbard, 2009, pp.509–556 [1962])
\parencite[][pp.509–556 \mbox{[1962]}]{rothbard_man_2009} %
 Only the market mechanism is able to verify real innovations in practice, not just those \textit{on paper}. Applying only some equilibrium-like scheme will not enable a~proper perception of this phenomenon, as will introducing some static comparison between different, completely counterfactual scenarios of dynamical nature.



Moreover, the assessment of differently estimated risk measures is not a~very objective thing because it also depends on various types of information that belong to both social and technical dimensions. Megger undoubtedly highlights crucial aspects of running a~company in a~dynamic economic environment where the need for innovation necessitates the recognition of risk. But is it not a~fundamental aspect of human action, whether or not we operate in a~market economy? Neverlethess, the book's author, correctly invokes Huerta de Soto's, Kirzner's, Mises's, and Schumpeter's claims about entrepreneurship and economic development 
%\label{ref:RNDeSc5hWo4IV}(Megger, 2021, pp.99–100).
\parencite[][pp.99–100]{megger_sprawiedliwosc_2021}. %
 Thus, by aiming for the leading role of an entrepreneur as an innovator, manager, and creative coordinator, it becomes clear how to properly perceive these issues. However, when it comes to safeguarding innovation through specific regulatory frameworks, it's important to acknowledge that in a~free market, businesses have the freedom to handle their intellectual inventions in any manner, such as by securing them as a~technological or organisational secret. Companies can create solutions to make it impossible to ``decode'' their potentially copyable products, sign appropriate contracts with workers, or bind recipients and contractors by appropriate clauses. The ball is still on the free market side.



Even if some companies copy some solutions from other ``innovative and advanced'' firms, consumer tastes can still be associated with different characteristics of goods or companies' activities. It is not a~given that economic agents would view theoretically similar goods with different purchase prices as homogeneous or even substitutable. Keep in mind that the possibility of raising goods' prices or restricting competition, \textit{ceteris paribus}, significantly impacts the ability to allocate a~given amount of funds to alternative purposes. Remember, expenditures for both consumption and production heavily depend on the level of capital accumulation and its forms of \textit{release}, which vary based on time preferences and monetary demand 
%\label{ref:RNDDUJvXbXj2z}(Rothbard, 2009, pp.348–362, 367–420 [1962]).
\parencite[][pp.348–362, 367–420]{rothbard_man_2009}.%




As a~result, protection policies in one sector of the economy can positively influence one company's real productivity but significantly slow down the productiveness (as well as the real value) of other companies, which would not benefit from greater capital accumulation. This illustrates that the argument does not target ``market'' innovations, but rather the dissemination of ``scientifically'' perceived innovations.



\section{Summary}

While the reviewed book represents an improved master thesis of Megger's authorship, his subsequent works present refined and extended aspects of its substantial analysis, establishing and elucidating the content that forms the basis for qualitative research. In fact, even though many of his claims are controversial, not exhaustively described, and reveal some understatements, Megger presented mature arguments against the fundamental theories of Rothbard and other Austro-libertarians. Moreover, we must appreciate Megger's extensive literature review.



In fact, as a~reviewer, I~strongly advocate many of Rothbard's claims on welfare economics and, on the other hand, support various novel arguments by Polish writers in this issue. However, my goal was not to present my substantial position or conduct a~comprehensive survey in this field, but rather to highlight key aspects of the book under review. I~hope that the discussion in the substantial part of the reviewed book delivers resolution to this very intense debate on RAWE.



In conclusion, I~believe that the book \textit{Justice in Welfare Economics. Libertarianism and the Austrian School} was and still is an interesting position in welfare economics, and I~think that the author would significantly improve his statements and prepare very valuable content on the substantial field of welfare economics.



\vspace{15mm}%
{\subsubsectit{\hfill Abstract}}\\
{This text reviews David Megger's 2021 book entitled \textit{Justice in Welfare Economics. Libertarianism and the Austrian School} (in Polish: \textit{Sprawiedliwość w~Ekonomii Dobrobytu, Liberatarianizm i~Szkoła Austriacka}). The review takes a~critical approach, highlighting the most significant aspects of the presented considerations and emphasising their uniqueness and complexity. I~intend to extensively discuss the author's theses concerning the modification of the fundamental claims of Austrian school representatives about justice and welfare, highlighting both their strengths and weaknesses.}\par%
\vspace{2mm}%
{\subsubsectit{\hfill Keywords}}\\
{Austrian school of economics, libertarianism, welfare economics.
}%



\end{newrevengenv}

\label{czyz-last}