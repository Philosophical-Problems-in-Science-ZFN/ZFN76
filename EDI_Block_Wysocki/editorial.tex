\begin{editorialeng2auth}{Walter Block, Igor Wysocki}
	{From a~nitty-gritty debate within economics into the deep waters of philosophy of science.
	Introduction to the special issue of ZFN}
	{From a~nitty-gritty debate within economics into the deep waters\ldots}
	{From a~nitty-gritty debate within economics into the deep waters of philosophy of science.
	Introduction to the special issue of ZFN}
	{%
			{\flushright\subbold{Walter Block}\\\subsubsectit\small{Loyola University New Orleans}\par}%
			{\flushright\subbold{Igor Wysocki}\\\subsubsectit\small{Nicolaus Copernicus University in Toruń}\par}%
	}









It all started in 2021 when we sparked a~rather specific debate within Austrian economics in \textit{Philosophical Problems in Science (Zagadnienia Filozoficzne w~Nauce)} – traditionally abbreviated as \textit{ZFN}. The story unfolded as follows. First, Wysocki submitted a~paper on the concept of indifference, as it is normally understood in the Austrian school of economics. To his astonishment and great relief, this then rising journal (now the one with well-established reputation, its Scopus ranking being as high as Q2 under the rubric of philosophy) accepted the submission in question and published it promptly in ZFN~71
%(2021)
as 
%\label{ref:RNDN6KKr0nHz7}(Wysocki, 2021).
\parencite[][]{wysocki_problem_2021}. %
 It was an honour and a~privilege, especially given the fact that Austrian economics -- with all due respect to its scientific achievements -- is nowadays not a~mainstream economic current, to say the least. Hence, being published in \textit{ZFN} only added to the strength of Wysocki's belief that the journal is clearly unbiased towards any sort of philosophy of science. The very fact that Wysocki published a~paper on indifference in Austrian economics and the fact that he spread the news about \textit{ZFN} being also open to publishing maverick (after all, as already observed, Austrian economics is not a~branch of mainstream economics) papers related to philosophy of science prompted Walter Block (a prominent Austrian economist) to submit to \textit{ZFN} his rejoinder to Wysocki's original paper on indifference, with the said response being ultimately published in ZFN~72
% (2022)
 as 
%\label{ref:RNDZQeCfSsaHd}(Block, 2022).
\parencite[][]{block_response_2022}.%




And thus the debate seemed to have unfolded. Block is well-known for being a~formidable debater and Wysocki knew perfectly well that if he had replied his criticism, Block would have come up with a~successive rejoinder shortly. Although Wysocki sensed that Block's response indeed called for a~rejoinder on the former's part, what prevented Wysocki from writing up his response was his scepticism as to \textit{ZFN}'s willingness to publish a~series of papers focused on nitty-gritty intricacies within the Austrian school of economics (in this case: indifference and how it relates to agents' actual choices). However, it was no less than Mr. Piotr Urbańczyk (an editor of the journal) who took the bull by the horns. The idea he came up with surpassed our wildest expectations. What Piotr suggested was not a~mere permission for us to continue discussing indifference in Austrian economics. Nay, he proposed that we, as guest editors (what an honour!), should dedicate the whole special issue to tackle the philosophical foundations of economics as such, be it Austrian, neoclassical, or what have you.



Obviously, we immediately embarked on the opportunity. The very thought of getting some prominent scholars in the field to contribute to the special issue was riveting. To our delight, we also found it inconceivable that special invitees (we were told we can have as many as three of them) would refuse to contribute their respective papers, given the standing of \textit{ZFN}, definitely one of the best Polish philosophical journals. Surely, it is a~daunting task to pick up three special invitees in the universe of exquisite philosophers or economists. Still, after some deliberation and necessary narrowing down of the said universe, we managed to select the required trio, which was in the end: (1) professor Karl-Friedrich Israel (an economist at the Western Catholic University in Angers, France); (2) professor Alexander Linsbichler (an economist and a~philosopher associated with The~Johannes Kepler University Linz and the University of Vienna) and (3) professor Łukasz Dominiak (a political philosopher at the Nicolaus Copernicus University in Toruń). Amazingly enough, all three of them agreed to contribute a~respective paper to the forthcoming \textit{Special Issue}. At this point, we cannot do better than elaborate on our rationale for selecting this particular set of three special contributors.



First, professor Karl-Friedrich Israel is a~renowned economist, especially well-versed in the Austrian tradition. Without a~doubt, he is one of the most outstanding Austrians in the younger generation. He has recently (i.e. 2023) obtained his habilitation at University Paris 1 Panthéon-Sorbonne, while still being in his thirties, something truly exceptional. His expertise in adverse effects of monetary policies and in inflation is second to none 
%\label{ref:RNDaHgR3d6dBn}(Israel, 2022).
\parencite[][]{israel_monetary_2022}. %
 He also massively contributed to the \textit{Quarterly Journal of Austrian Economics}, a~flagship Austrian journal. His analytic apparatus and overall conceptual grasp are superb. We remember being deeply impressed by his co-authored paper (with Tate Fegley) on the disutility of labour 
%\label{ref:RNDr8oPFzpCQC}(Fegley and Israel, 2020).
\parencite[][]{fegley_disutility_2020}. %
 These two authors boldly went against the Misesian dogma holding the disutility of labour to be an auxiliary empirical proposition in Austrian economics. Somewhat ironically, the authors' critical attempt helped to advance the overall Misesian \textit{a~priori} scientific programme. In the present \textit{Special Issue} professor Karl-Friedrich Isreal joins forces with Tate Fegley yet again, thus producing a~paper \textit{A~Defense of Austrian Welfare Economics} 
%\label{ref:RNDkNpBEcghvM}(Fegley and Israel, 2024),
\parencite[][]{fegley_defense_2024}, %
 wherein the authors respond to a~recent criticism of Rothbardian welfare economics levelled by Wysocki and Dominiak 
%\label{ref:RNDIa7wjfDGjq}(2023).
\parencite*[][]{wysocki_how_2023}. %
 Fegley's and Israel's paper is excellently argued, straightforward and is bound to give Wysocki and Dominiak new headaches. In a~word, it was excellent to have professor Israel on the board.



Second, another excellent economist with a~philosophical bent we could think of was professor Alexander Linsbichler. He is definitely a~force to reckon with. He is insanely versatile. His versatility ranges from the acquaintance with the intellectual history of Vienna through profound conceptual insights within the Austrian school of economics to philosophy of science in general. Professor Linsbichler published in such first-rank journals as \textit{Synthese}, \textit{Journal for General Philosophy of Science} or \textit{Journal of Economic Methodology} 
%\label{ref:RNDiv5pAaUbFE}(Linsbichler, 2021; 2023; Linsbichler and Da Cunha, 2023).
\parencites[][]{linsbichler_austrian_2021}[][]{linsbichler_otto_2023}[][]{linsbichler_otto_2023}. %
 In the present \textit{Special Issue}, professor Linsbichler contributes the paper \textit{What Rothbard could have done but did not do: The merits of Austrian economics without extreme apriorism} 
%\label{ref:RNDWzZPuXlFvM}(2024).
\parencite*[][]{linsbichler_what_2024}. %
 The paper highlights professor Linsbichler at his best: erudite, analytically sharp and argumentatively original.



Additionally, it was no less than professor Łukasz Dominiak who agreed to contribute. Professor Dominiak is Wysocki's friend and mentor (literally a~supervisor of his Ph.D. thesis) at the same time. Professor's Dominiak interests are far-reaching and they include Austrian economics, political philosophy (especially libertarianism) as well as legal and moral philosophy 
%\label{ref:RND7tOtEjykCW}(Dominiak, 2017; 2019; Dominiak and Fegley, 2022).
\parencites[][]{dominiak_libertarianism_2017}[][]{dominiak_must_2019}[][]{dominiak_contract_2022}. %
 He almost single-handedly revolutionized the libertarian theory of justice in such areas as the theory of contract, compossibility of libertarian individual rights, the libertarian methods of property acquisition and what have you. In this \textit{Special Issue} he presents another of his splendid ideas, this time running against the libertarian received view on blackmail. Namely, professor Dominiak contributes the paper \textit{Free Market, Blackmail, and Austro-Libertarianism} 
%\label{ref:RND7iPO8Frudh}(2024),
\parencite*[][]{dominiak_free_2024}, %
 wherein he argues -- in his characteristically unyielding style -- that Austro-libertarians do have a~reason to revise their view on apparent permissibility of blackmail. His finding is all the more impressive, as his argument against libertarians is \textit{internal} in that it does not appeal to any external morality. Rather, professor Dominiak demonstrates that libertarians should favour banning \textit{some} blackmail exchanges, for they constitute frauds, something clearly prohibited in a~free society.



As a~matter of course, this \textit{Special Issue} contains other prominent scholars. Oliva Córdoba, a~renowned scholar, contributed an excellent essay on the philosophy and logic of human action. Interestingly, the author makes use of a~conceptual apparatus of philosophy of action to make sense of the notion of, for example, competition 
%\label{ref:RNDxw8PxTmb3C}(Oliva Córdoba, 2024).
\parencite[][]{oliva_cordoba_philosophy_2024}. %
 Mateusz Machaj, an undeniable Austrian superstar, elucidated the distinction between risk and uncertainty and proposed a~way of modelling uncertainty 
%\label{ref:RNDMDE10oIW23}(Machaj, 2024).
\parencite[][]{machaj_model_2024}. %
 Robert Mcgee, the most brilliant and versatile scholar, indulged us with his reading of Bastiat's view on taxation 
%\label{ref:RNDkZU8zhrHPU}(McGee, 2024).
\parencite[][]{mcgee_taxation_2024}. %
 Krzysztof Turowski analyzed Ludwig Lachmann as an alleged subjectivist institutionalist 
%\label{ref:RNDUAACXxN1gv}(Turowski, 2024).
\parencite[][]{turowski_ludwig_2024}. %
 Ceglarska and Cymbranowicz wrote a~paper on the role of phronesis in knowledge-based economy 
%\label{ref:RNDj7qowYgqeK}(Ceglarska and Cymbranowicz, 2024).
\parencite[][]{ceglarska_role_2024}. %
 Wysocki and Dominiak contributed a~short essay defending the Rothbardian welfare theory against the charges made by, most crucially, Bryan Caplan 
%\label{ref:RNDcAhJDNmSsg}(Wysocki and Dominiak, 2024).
\parencite[][]{wysocki_rejoinder_2024}. %
 Matúš Pošvanc, associated with F.A. Hayek Foundation wrote a~refined essay on the law of diminishing marginal utility 
%\label{ref:RNDs7FXZ4vkWh}(Posvanc, 2024).
\parencite[][]{posvanc_law_2024}. %
 Norbert Slenzok added to the \textit{Special Issue} by writing an essay \textit{Monarchy as Private Property Government. A~Chiefly Methodological Critique} 
%\label{ref:RNDV7H6rqtzrl}(Slenzok, 2024).
\parencite[][]{slenzok_monarchy_2024}. %
 Dawid Megger illuminatingly tackled the problem of demonstrated preference 
%\label{ref:RND2fG8nSblvN}(Megger, 2024).
\parencite[][]{megger_demonstrated_2024}. %
 Paweł Nowakowski critically scrutinized the Rothbardian view on the value of life from a~praxeological perspective 
%\label{ref:RND08YPX5nlKu}(Nowakowski, 2024).
\parencite[][]{nowakowski_praxeology_2024}. %
 Wysocki wrote a~brief rejoinder to Block's rejoinder to the former's original paper on indifference published in \textit{ZFN} 
%\label{ref:RNDYcJ1CcSqnE}(Wysocki, 2024).
\parencite[][]{wysocki_rejoinder_2024}. %
 The said rejoinder was in turn replied by Walter Block, also published in the present issue 
%\label{ref:RNDu4fpSmJ9Cq}(Block, 2024).
\parencite[][]{block_response_2024}. %
 Finally, Mateusz Czyżniewski, a~rising Austrian scholar, contributed a~review of Dawid Megger's book 
%\label{ref:RNDhHI0meumjt}(Czyżniewski, 2024).
\parencite[][]{czyzniewski_are_2024}.%




Eventually, a~word is due on the relevance of the present \textit{Special Issue} to \textit{ZFN}'s programmatic dedication to philosophy of science and its advocacy of interdisciplinarity. As already mentioned, the very inspiration for the whole \textit{Special Issue} came from the debate on the nature of choice vis-à-vis indifference within Austrian economics. But then again, when we were offered a~\textit{Special Issue}, we immediately thought of going beyond Austrianism itself. So, philosophical foundations of economics as such appeared to us to be a~rather apt unifying theme. But even this rather large category would not do justice to a~variety of papers included in this issue. For, what we have here is also epistemology proper (e.g. considerations on the Austrian alleged extreme apriorism), philosophy of action (e.g. reducing the phenomenon of competition or rivalry to certain intentional states of individual agents), some exegetical work (e.g. interpreting the thought of Lachmann, Rothbard and Aristotle himself) or political and legal philosophy combined (e.g. tacking the paradox of blackmail). Even this sample, we believe, satisfies \textit{ZFN}'s unyielding commitment to interdisciplinarity. Needless to say, the present Special Issue, while being dedicated to philosophy of economics, rather effortlessly satisfies \textit{ZFN}'s dedication to philosophy of science, for what is economics if not a~special science.



All in all, editing this issue was quite a~ride, with its joys (networking with exquisite anonymous reviewers, thinking about special invitees etc.) and sorrows (sometimes finding an appropriate reviewer is quite a~daunting task). Still, getting the above scholars to submit their respective work more than compensated for the effort made. We also hope that the prospective readers are going to find the essays included as interesting as we do. And finally, we do believe that this \textit{Special Issue} will give some additional impetus to a~burgeoning field of philosophy of economics.









\end{editorialeng2auth}

