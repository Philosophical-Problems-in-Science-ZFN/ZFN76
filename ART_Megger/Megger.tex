\begin{artengenv}{Dawid Megger}
	{Demonstrated preference in the Austrian economic analysis}
	{Demonstrated preference in the Austrian economic analysis}
	{Demonstrated preference in the Austrian economic analysis}
	{Nicolaus Copernicus University in Toruń\label{megger-first}}
	{This paper is an attempt to clarify the concept of demonstrated preference in the economic analysis of the Austrian School of Economics. It considers several interpretations of this concept: (1) as a~thymological concept which matters in empirical interpretations of concrete human actions; (2) as a~preference expressed in voluntary actions; (3) as the only existing preference; (4) as the only preference which matters in economics; (5) as the only preference which matters in the economy. It is argued that despite the Austrian insistence on (4), the only interpretation resistant to criticism is (5). Unfortunately, it is not sufficient to draw or reinforce the conclusions that Murray N. Rothbard and his followers reach in their considerations on welfare economics, monopoly theory, public goods theory, and game theory. A~number of additional clarifications are also made (e.g., the concepts of ``voluntariness'' and ``psychologizing'' in economics).
	}
	{demonstrated preference; revealed preference; Murray Newton Rothbard; Austrian School of Economics.}








\section{Introduction}

\lettrine[loversize=0.13,lines=2,lraise=-0.03,nindent=0em,findent=0.2pt]%
{T}{}he concept of demonstrated preference (henceforth: DP), named and formulated in 1956 by Murray Newton Rothbard in his essay \textit{Toward a~Reconstruction of Utility and Welfare Economics} 
%\label{ref:RND4dKVflFkMW}(Rothbard, 2011b),
\parencite[][]{rothbard_present_2011}, %
 had a~significant impact on the later research practice of some representatives of the Austrian School of Economics (henceforth: ASE). It allowed Rothbard to draw radical conclusions in welfare economics, monopoly theory, or public goods theory 
%\label{ref:RNDqcMUwJjy2h}(cf. Rothbard, 2009a; 2009b; 2011b).
\parencites[cf.][]{rothbard_man_2009}[][]{rothbard_power_2009}[][]{rothbard_present_2011}. %
 Rothbard's arguments in these areas were adopted by other Austrians 
%\label{ref:RNDlcYtk5IiHh}(e.g., Block, 1983; Herbener, 2008; Hoppe, 1989; 2006; Wiśniewski, 2018).
\parencites[e.g.,][]{block_public_1983}[][]{herbener_defense_2008}[][]{hoppe_theory_1989}[][]{hoppe_economics_2006}[][]{wisniewski_economics_2018}. %
 Roughly speaking, it can be said that according to DP, in economics only those preferences that are demonstrated in the actions of individuals matter. Criticism of the concept 
%\label{ref:RNDn68sMyqofX}(e.g. Nozick, 1977; Caplan, 1999; Kvasnička, 2008),
\parencites[e.g.][]{nozick_austrian_1977}[][]{caplan_austrian_1999}[][]{kvasnicka_rothbards_2008}, %
 and the need for its subsequent defence 
%\label{ref:RNDjDeEz7QZpj}(e.g., Block, 1999; Wysocki, Block and Dominiak, 2019; Gordon, 2022),
\parencites[e.g.,][]{block_austrian_1999}[][]{wysocki_rothbards_2019}[][]{gordon_misunderstanding_2022}, %
 however, suggest that there are some ambiguities related to DP.



The research problem I~take up in this article is: what does it mean that in economics only those preferences that are being demonstrated in actions matter? My goal is to present a~systematic interpretation of DP. This will allow me to fill the research gap in the indicated area and solve an important problem regarding the research practice of some contemporary representatives of ASE. I~will try to solve the problem by using the principle of charity, which involves critiquing certain ideas in their strongest possible variants. In the course of my research, I~will primarily resort to logical analysis and hermeneutics. I~will try to account for the thesis that the most convincing interpretation of DP is not sufficient to draw the conclusions reached by Rothbard and his followers in the field of welfare economics, monopoly theory, public goods theory, or game theory.



In section 2, I~show DP, as laid out by Rothbard. In section 3, I~present DP as a~thymological concept and argue for rejecting this interpretation. In section 4, I~consider whether preferences can be demonstrated in coerced actions and show the consequences of an affirmative and a~negative answer to this question. In section 5, I~turn to an interpretation of DP that seems to best fit the intentions of the Austrians (that only demonstrated preferences matter in economics). I~consider this in two variants: strong, easily exposed to criticism (5.1), and weak, more resistant to criticism (5.2). However, I~submit that both interpretations have similar practical consequences and it is difficult to find a~convincing argument for their acceptance. In section 6, I~present, what I~think is the most appealing understanding of DP (that only demonstrated preferences influence the social processes). Section 7 briefly concludes.



\section{Demonstrated preference -- an outline of the concept}

Rothbard sees traces of DP in the writings of William Stanley Jevons, Irving Fisher, Frank Fetter, and Ludwig von Mises 
%\label{ref:RNDPwRQSLAwT5}(Rothbard, 2011b, p.290).
\parencite[][p.290]{rothbard_present_2011}. %
 Since my considerations are not of a~historical nature, I~will make Rothbard's formulation the starting point for further analysis. The reason for this is that it is his formulation that is at the core of his and his followers' research practice.



Rothbard's motivation for formulating DP is the belief that economics is a~science dealing with human action. Human action, in turn, is a~purposeful behaviour, or a~conscious pursuit of chosen goals by definite means. The main idea behind DP for Rothbard is the belief that action expresses the preferences of the person undertaking it. Or, in his words: ``The concept of demonstrated preference is simply this: that actual choice reveals, or demonstrates, a~man's preferences; that is, that his preferences are deducible from what he has chosen in action'' 
%\label{ref:RNDPa8Rkcv3b1}(Rothbard, 2011b, p.290).
\parencite[][p.290]{rothbard_present_2011}.%




Later in his article, Rothbard rejects Paul A. Samuelson's concept of revealed preferences because, he argues, it inevitably smuggles in the erroneous assumption of the constancy of preferences over time, while, in accordance with the ASE methodology, preferences of economic actors can be constantly changing 
%\label{ref:RNDoGwV4vz8pU}(Rothbard, 2011b, p.294).
\parencite[][p.294]{rothbard_present_2011}.%




Rothbard also rejects two extreme approaches to human action, which he calls psychologizing and behaviourism, respectively. Psychologizing is supposed to consist of (a) speculating about preferences not demonstrated in actions and (b) dealing with psychology's characteristic consideration of why people have certain preferences and how they are formed (as I~will try to show later, (a) and (b) are not the same, so it is difficult to understand why Rothbard writes about these ideas in the same breath). Behaviourism\footnote{More specifically, we could distinguish between methodological and ontological behaviourism. While the former underlies that referring to mental states (such as preferences) does not constitute scientific explanations, the latter says that mental states (such as beliefs and desires) do not exist at all. Although methodological behaviourists need not subscribe to ontological behaviourism, the latter must be recognized as a~sound basis for the former. Rothbard 
%\label{ref:RNDVj6Pz5Ct4f}(2011b, p.297)
\parencite*[][p.297]{rothbard_present_2011} %
 seems to be criticizing both of them. However, he does not make the distinction between them explicitly.}, on the other hand, consists in completely ignoring the mental dimension of human action (goals, preferences, beliefs) and focusing only on its physical dimension. Contrary to behaviourists, Rothbard points out that mental states (preferences) matter in scientific explanations of human actions. However, contrary to ``psychologists'', he argues that in economics, only those preferences that economic actors demonstrate in actions are relevant. Consequently, in practice, he recognizes that the concept of preference makes no sense apart from the actual action. According to DP: ``economics deals only with preference as demonstrated by real action'' 
%\label{ref:RNDYamv9Cdb2R}(Rothbard, 2011b, p.333).
\parencite[][p.333]{rothbard_present_2011}.%




Before proceeding, one further clarification might be regarded as important. More specifically, and as it happens to be unclear in the Austrian literature, it is worth asking what the concept of preference means? Most of all, we need to note that preference is not a~simple intention or desire. Rather, it expresses a~relation between at least two competing wants. In other words, \textit{preference} assumes the existence of at least two alternatives, one of which is valued more highly than the other. It looks consistent with the following quote from Rothbard: ``If a~man chooses to spend an hour at a~concert rather than a~movie, we deduce that the former was preferred, or ranked higher on his value scale. Similarly, if a~man spends five dollars on a~shirt we deduce that he preferred purchasing the shirt to any other uses he could have found for the money'' 
%\label{ref:RNDCLLGrnKjWP}(Rothbard, 2011b, p.290).
\parencite[][p.290]{rothbard_present_2011}.%




It seems that based on the last sentence, we can infer that according to Rothbard, it is not necessary to know what the second-best alternative is. He simply assumes that there was one. Given actual constraints, an agent chooses the course of action that appears to him as the best available. A~probable implicit premise in this reasoning is that human action is based on some kind of deliberation over different possible scenarios. Still, what is important in light of DP is not the content of the second-best alternative but that human action (conscious pursuit of ends) presupposes a~choice between alternatives, and that the first alternative is of particular interest to the sciences of human action. As David Gordon 
%\label{ref:RND1jxuB9YoSD}(2022)
\parencite*[][]{gordon_misunderstanding_2022} %
 put it, DP ``means that choice reveals the chooser's highest preference''.



\section{Demonstrated preference as a~thymological concept}

It is sometimes said that Samuelson's concept of revealed preference was intended to give economic theories empirical significance. By observing consumer behaviour, the assumptions and conclusions of economic theory could be confirmed or falsified. For example, if one of the assumptions of the rational choice theory is that the preferences of economic actors are transitive (someone who prefers A~over B~and B~over C~must prefer A~over C) and if empirical research has shown that sometimes their preferences are intransitive, then the assumption of transitivity could be considered falsified (this type of research is often conducted within the framework of so-called behavioural economics). The question is then: is DP also supposed to give economic theories empirical significance? At this point, I~will try to present the consequences of both possible answers to this question.



Let me begin by recalling the radical division between theory and history introduced by Mises and adapted by Rothbard. According to Mises, there are two branches of the sciences of human action: theoretical and historical. Economics, which is a~part of the theoretical sciences of human action (praxeology), is an axiomatic-deductive science, taking as its basis the axiom of human action as a~conscious pursuit of chosen ends. Economic theories and laws are not subject to empirical verification or falsification. The only way to undermine them is to demonstrate that the said theories and law involve fallacious reasoning 
%\label{ref:RNDGu08IQ9tRb}(cf. Mises, 1998, pp.30–41).
\parencite[cf.][pp.30–41]{mises_human_1998}.%




Mises refers to the science that underlies the historical sciences of human action as thymology. The basic method of thymology is the so-called empathic understanding (or, to use a~term introduced by Max Weber, \textit{Verstehen}). The essence of this method is striving for interpretation of the meaning that other people give to their actions and to natural as well as social phenomena. This method requires the assumption that other people have a~mental structure similar to ours; that they are beings who pursue their own ends by definite means; that they are guided by certain beliefs and desires. Accordingly, if a~literary scholar wants to correctly interpret the work of a~poet, he should get to know his biography, get acquainted with the values he held dear, and deepen his knowledge of the literature he read. If a~historian wants to understand why Henry VIII renounced obedience to the Pope, he should understand his social situation, character traits, and the private and political goals he could have wanted to achieve by entering a~second marriage 
%\label{ref:RNDx3qB6c75PL}(Mises, 2007, pp.264–284).
\parencite[][pp.264–284]{mises_theory_2007}.%




The method of \textit{Verstehen} is also reflected in Austrian subjectivism 
%\label{ref:RNDmlzdy79tSR}(Lachmann, 1971).
\parencite[][]{lachmann_legacy_1971}. %
 The need to refer to this method in empirical research results from the observation that the ``data'' of the social sciences are inherently subjective 
%\label{ref:RNDc5LxReWD8q}(Hayek, 1952).
\parencite[][]{hayek_counter-revolution_1952}.%
Thymological knowledge is never certain. No one has a~direct insight into the minds of other people. Moreover, such data cannot be verified or falsified by any act of measuring external qualities. Thymological interpretations can be made more reliable or plausible by deepening historical, psychological, or sociological knowledge. The humanities and theoretical social sciences can also help here. Historical (thymological) research, however, cannot invalidate the findings of economics (praxeology). They play a~different role: they help to determine when a~given theory can be applied (they can also suggest the direction of development of the theory and be a~source of empirical auxiliary assumptions, cf. e.g., 
%\label{ref:RNDoI7vC6K9p5}(Wiśniewski, 2014)
\parencite[][]{wisniewski_methodology_2014}%
).



Although, as Mises points out, ``thymology has no special relation to praxeology and economics'' 
%\label{ref:RNDD9LBzIPous}(Mises, 2007, p.271),
\parencite[][p.271]{mises_theory_2007}, %
 there is no doubt that he treats these disciplines as complementary to one another. In order to properly interpret historical events, theoretical knowledge derived from economics and praxeology is necessary. In order to find out when a~particular theory is applicable, it is necessary to understand a~specific historical (i.e., taking place in history) situation. One discipline without the other is useless. As Roderick T. Long succinctly puts it: ``Praxeology without thymology is empty; thymology without praxeology is blind'' 
%\label{ref:RNDypJZx3D7pw}(Long, 2008, p.50).
\parencite[][p.50]{long_wittgenstein_2008}.%




At this point, we can restate the initial question: is DP a~praxeological or thymological concept? Is it possible to conclude that, thanks to DP, an observer gets a~certain empirical knowledge about the purposes and preferences of economic actors? By no means. As we have seen, according to the Austrians, such knowledge can never be certain. From the mere observation of external manifestations of someone's actions, it is impossible to draw conclusions about their goals, beliefs, or preferences. As, Hans-Hermann Hoppe 
%\label{ref:RND3g2BJAuNwA}(2005),
\parencite*[][]{hoppe_note_2005}, %
 Rothbard's follower, observes, citing John Searle 
%\label{ref:RNDfXHFdrbbZa}(1984, pp.57–58),
\parencite*[][pp.57–58]{searle_minds_1984}, %
 that human action has two aspects: external (physical, behavioural) and internal (mental, psychological). There is no doubt that only the physical movements of bodies can be observed, not human goals and preferences. Therefore, in order to fully understand someone's actions, it is also necessary to grasp the mental aspect, i.e., the intentions, beliefs, and desires of a~given person.



Consider the example of a~child being baptized in a~church. The goal of both the priest and the parents of this child is to blot out its original sin and include it in the community of the Church. To an outside observer, who has no knowledge of Christianity, however, it will only be an incomprehensible pouring of water onto the child's head and making the sign of the cross on its forehead. Observation of the mere physical movements of bodies, characteristic of behaviourism, does not allow for a~full description of this event. In order to grasp the essence of social situations, it is necessary to understand the goals and beliefs of the people involved in them. Both Mises and Rothbard reject the behaviourist approach 
%\label{ref:RNDglHD9rJKmv}(cf. Mises, 2007; Rothbard, 2011b).
\parencites[cf.][]{mises_theory_2007}[][]{rothbard_present_2011}.%




Nevertheless, there is no shortage of accusations in the literature that Rothbard's DP is essentially behaviourist 
%\label{ref:RNDFiLEXv5NGR}(e.g., Prychitko, 1993, p.574).
\parencite[e.g.,][p.574]{prychitko_formalism_1993}. %
 Bryan Caplan argues in a~similar vein, noting that external manifestations of actions do not allow us to draw conclusions about the mental states of the individuals undertaking them:
\begin{quote}
Indeed, Rothbard could have taken this principle further. When two people sign a~contract, do they actually demonstrate their preference for the terms of the contract? Perhaps they merely demonstrate their preference for writing their name on the piece of paper in front of them. There is no ironclad proof that putting one's name on a~piece of paper is not a~joke or an effort to improve one's penmanship. 
%\label{ref:RNDXqEz04mGRy}(Caplan, 1999, p.833)
\parencite[][p.833]{caplan_austrian_1999}%
\end{quote}
Therefore, it seems that DP cannot be regarded as a~method of interpretation of human actions in empirical reality. Thymological knowledge consists of the proper interpretation of concrete actions. It is based on the method of \textit{Verstehen}. DP does not develop or change anything in this procedure. Accordingly, this concept seems unnecessary at best and misleading at worst in this context. Thus, it seems that the thymological interpretation of DP is also to be rejected.



Furthermore, Rothbard's objection to ``psychologizing'' with regard to ``speculating about preferences not being demonstrated in actions'' also does not seem well-grounded. Rothbard states that non-demonstrated preferences should not be of interest to economics because the assumption of their existence or their specific scale is based on uncertain conjectures 
%\label{ref:RND1N9UizdGaS}(Rothbard, 2011b, pp.296–298).
\parencite[][pp.296–298]{rothbard_present_2011}. %
 But if thymological knowledge can never be certain, does it not mean that the charge of psychologizing can be applied to both ``demonstrated'' and ``non-demonstrated'' preferences? Of course, to say that a~person has a~certain preference, based on the fact that he has undertaken a~definite action, is much better justified than to say that this person has a~preference for something he has never pursued in action (or at least no one has observed it). No external observer, however, can ever say with absolute certainty that the person has demonstrated a~preference for something particular because there is no direct insight into that person's mind (not even that person's self-declaration settles anything -- he could simply lie, after all). Knowledge of mental states is not verifiable. Only an acting individual, through introspection, knows with certainty their own intentions and beliefs.



The considerations presented further in the article are based on the assumption that DP is an analytic tool of economic theory (praxeology), not thymology. Acting individuals demonstrate their goals to themselves or, in an imaginary world invented for didactic purposes, to an economic theorist. An economist who wants to explain an economic phenomenon or explain an economic theory can, after all, describe a~scenario in which he assumes that an economic actor has certain preferences (to the economist, both demonstrated and non-demonstrated preferences can be known with certainty). This is what Rothbard himself does, explaining, for example, the law of diminishing marginal utility 
%\label{ref:RNDN3ojodUZgD}(Rothbard, 2009a, pp.21–33).
\parencite[][pp.21–33]{rothbard_man_2009}.%




Summarizing the above considerations, it is worth emphasizing the following relationship; if DP were a~thymological concept, it would have no direct theoretical consequences. However, if it is a~praxeological concept, then it has no direct empirical consequences (it does not ``improve'' thymological interpretations of observed actions of individuals).



\section{Demonstrated preference as a~preference expressed in voluntary actions}

Another interpretation of DP worth considering is the suggestion that individuals demonstrate their preferences only in voluntary actions. According to this interpretation, if someone undertakes an action under coercion, then it cannot be said that he is demonstrating his preference. For example, if a~robber assaults someone and says: ``Give me your money or I~will kill you'', and as a~result, the victim gives up his money, it cannot be said that the victim has demonstrated a~preference for giving the money to the robber.



At first glance, this interpretation is quite appealing. There is no shortage of suggestions in the literature that it is appropriate. Mateusz Machaj 
%\label{ref:RNDs4RfDqCYhC}(2014),
\parencite*[][]{machaj_murray_2014}, %
 for example, agrees with it, stressing that, according to Rothbard, ``voluntary trade relations within the free market increase the welfare of both parties to the transaction'' 
%\label{ref:RNDvsXYCzcYCU}(Machaj, 2014, pp.10–11, own transl.),
\parencite[][own transl.]{machaj_murray_2014}, %
 and then in a~footnote he adds:



\begin{quote}
The shortcomings in Rothbard's theory stem from the fact that the concept of ``demonstrated preference'' implicitly implies that it is a~demonstrated preference with respect to personal property and what the individual owns. And if so, Rothbard's approach is also not ``value free'' because it presupposes some version of ``justly acquired'' property, which is already a~normative concept. For example, a~tax clerk and a~private building administrator, both of whom apply to someone for a~fee, are praxeologically no different from each other until we introduce additional assumptions about the nature of existing property titles. (
%\label{ref:RND5c6DNWCTHU}(Machaj, 2014, pp.10–11, own transl.)
\parencite[][own transl.]{machaj_murray_2014}%
\end{quote}




As Machaj suggests above, and as other authors also note 
%\label{ref:RNDF7hEgYMRu3}(e.g., Cordato, 1992),
\parencite[e.g.,][]{cordato_welfare_1992}, %
 by voluntary actions Rothbard and his followers mean such actions in which the property rights of the individuals who undertake them are not violated, whereas the concept of ``justly acquired'' property would be inextricably tied with the libertarian theory of justice. This, in turn, would contradict the value freedom postulate (\textit{Wertfreitheit})\footnote{In the economic literature, the normative entanglement of the concept of voluntariness was noticed, e.g., in 
%\label{ref:RNDlY9ztBVk0O}(High, 1985; Hausman and McPherson, 2006).
\parencites[][]{high_is_1985}[][]{hausman_economic_2006}.%
} accepted by the Austrians 
%\label{ref:RND5qUSRuxvJR}(e.g., Mises, 1998; Rothbard, 2011a; Kirzner, 1994; Block, 2005).
\parencites[e.g.,][]{mises_human_1998}[][]{rothbard_praxeology_2011}[][]{kirzner_value-freedom_1994}[][]{block_value_2005}. %
 According to Machaj, however, this interpretation of DP is necessary to defend Rothbardian welfare economics. Without it, his theory would lose its grounds.



The fact that, in their research practice, Rothbard and his followers, before proceeding with economic analysis, assume the concept of ownership, was expressed by them explicitly \footnote{The words of Rothbard himself are especially meaningful here: ``an economist cannot fully analyze the exchange structure of the free market without setting forth the theory of property rights, of justice in property, that would have to obtain in a~free-market society'' 
%\label{ref:RNDX3BoG3ccy3}(Block, 1995; 2000; Hülsmann, 2004; Rothbard, 2009b, p.1047; 2009a).
\parencites[][]{block_ethics_1995}[][]{block_private-property_2000}[][]{hulsmann_priori_2004}[][p.1047]{rothbard_power_2009}[][]{rothbard_man_2009}.%
}. It is worth noting that, as a~consequence, they treat property rights as exogenous to economic theory and \textit{nolens volens} reject the area of research called economic analysis of law 
%\label{ref:RNDrtsNGkPSej}(see: Machaj, 2014).
\parencite[see:][]{machaj_murray_2014}.%




That the Rothbardians assume the libertarian theory of justice in the concept of voluntariness is especially visible in the considerations of Walter E. Block and David Gordon 
%\label{ref:RNDOQGcsX6Deo}(1985).
\parencite*[][]{block_blackmail_1985}. %
 According to these scholars, the proposition ``Give up your money or I~will destroy your reputation'' would not be a~coercive threat. As absurd as it may seem, it would be a~plain offer, categorically not different from the proposal ``If you give me \$2, I~will give you bread''.



Let us compare the types of offers and threats presented so far.

\medskip

\noindent (1) ``Give me your money or I~will kill you''.



\noindent (2) ``Give me your money or I~will destroy your reputation''.



\noindent (3) ``If you give me \$2, I~will give you bread''.

\medskip

In light of the concept of voluntariness adopted by Rothbard and his successors, it should be said that (according to the interpretation of DP considered in this point) an economic actor who, in the case of (1) gives money to the speaker, does not demonstrate a~preference because he acts under a~coercive threat. However, in both cases (2) and (3), he demonstrates his preferences. In both cases, he gives up his money voluntarily.



However counterintuitive case (2) may seem, there is no room for a~thorough analysis of the libertarian theory of justice and the related concept of voluntariness (a thorough analysis of this concept is presented by Igor Wysocki 
%\label{ref:RNDsg7z9jmQpP}(2021)
\parencite*[][]{wysocki_austro-libertarian_2021}%
). However, it is worth considering one more possibility. More specifically, it is possible that by property the Austrians understand not so much the right to property as the physical control over a~resource. The distinction between these concepts was presented by Mises 
%\label{ref:RNDiBtTK66oYm}(1962, pp.37–39).
\parencite*[][pp.37–39]{mises_socialism_1962}. %
 He distinguishes between the legal concept of property (property right) and the physical control over a~resource (``natural ownership'' or ``possession''). Then, the concept of voluntariness adopted by them would not violate the postulate of \textit{Wertfreiheit}. This interpretation is suggested by Jeffrey Herbener 
%\label{ref:RNDzkb18suOad}(1997, p.99).
\parencite*[][p.99]{herbener_pareto_1997}.%




Unfortunately, it seems to be inconsistent with Austrian subjectivism. In accordance with this principle, economics does not deal with the physical world, but rather with the mental states of the acting individuals. This has been emphasized not only by Mises 
%\label{ref:RNDyPzg960f5J}(1998, p.92),
\parencite*[][p.92]{mises_human_1998}, %
 but also by Hayek 
%\label{ref:RND4cYlDCwg9u}(1952)
\parencite*[][]{hayek_counter-revolution_1952} %
 and Rothbard 
%\label{ref:RND37uP1tqqg8}(2011b, p.289).
\parencite*[][p.289]{rothbard-present}.%




Moreover, predictions resulting from the concept of voluntariness based on physical control over a~resource would differ significantly from those resulting from the concept of voluntariness based on the libertarian theory of justice. It is true that both of these concepts would consider giving money in cases (2) and (3) as voluntary actions (since you cannot have physical control over your reputation, and the seller of bread does not violate either the natural ownership or the property right of the buyer). However, let us consider two more cases:



\medskip

\noindent (4) A~lends B~a bike. When A~asks B~to give the bike back, B~says, ``Give me \$10 or I~will destroy the bike''.

\noindent (5) A~steals B's wallet. B~says to A: ``Give my wallet back or I~will hit you''.

\medskip


Would A~give B~the money or wallet voluntarily? In the light of the concept of voluntariness based on the libertarian theory of justice, in case (4) certainly not -- after all, B~violates A's property right (because by lending the bike, A~did not renounce the property title to it, and did not agree to its destruction). In light of the concept of voluntariness based on physical control, however, it should be said that A~gives money to B~voluntarily. Though he did not relinquish his property right to the bike, he lost physical control over it. In accordance with this concept, a~``threat'' of destroying something over which one has no physical control cannot be a~coercive threat! Note, however, that in case (5), it is to the contrary. The heretofore provided examples can be summarized as follows:







\begin{table}[H]
    \centering
    \begin{adjustbox}{max width=\textwidth}
        \begin{tabularx}{\textwidth}{|L{4cm}|Y|Y|}
            \hline
            \textbf{Proposal} & \textbf{Is this coercion, based on the criterion of physical possession violation?} & \textbf{Is this coercion, based on the criterion of property rights violation?} \\ \hline
            (1) B comes to A and says: “Give your money or I will kill you”. & Yes & Yes \\ \hline
            (2) B comes to A and says: “Give your money or I will destroy your reputation”. & No & No \\ \hline
            (3) B comes to A and says: “If you give me \$2, I will give you bread”. & No & No \\ \hline
            (4) A lends B a bike. When A asks B to give the bike back, B says: “Give me \$10 or I will destroy your bike”. & No & Yes \\ \hline
            (5) A steals B’s wallet. B says to A: “Give my wallet back or I will hit you”. & Yes & No \\ \hline
        \end{tabularx}
    \end{adjustbox}
    \caption{Coercive threats and non-coercive propositions.}
\end{table}



As I~have tried to show, the concept of voluntariness, which is adopted by the cited representatives of the ASE, poses serious difficulties for their methodology. To deal with this problem, representatives of the ASE would have to present such a~concept that would be consistent with their methodology. Since it is usually believed that a~sufficient condition for the involuntary nature of an action is to undertake it under coercion, it would be worthwhile to begin such research with a~study of the rich philosophical literature on coercion (a review of the concepts formulated so far can be found in 
%\label{ref:RNDwGx7HSdTFx}(Anderson, 2021);
\parencite[][]{anderson_coercion_2021}; %
 an attempt at finding a~concept of coercion fitting the Austrian methodology can be found in 
%\label{ref:RNDpPzpML5WPU}(Megger and Wysocki, 2023)
\parencite[][]{megger_coercion_2023}%
).



However, even after finding appropriate concepts of coercion and voluntariness, one might still think that linking DP to voluntariness is unjustified. From the perspective of Austrian praxeology, in fact, every action is driven by some preference 
%\label{ref:RNDmb3Dz6GDTo}(Mises, 1998, pp.13–14, 92–98).
\parencite[][pp.13–14, 92–98]{mises_human_1998}. %
 It, therefore, seems quite convincing to say that if a~victim takes an action aimed at giving money to the robber, he demonstrates his preference for the preservation of life over the preservation of money. It is difficult to formulate a~serious objection against such an interpretation of DP. For this reason, it is worth taking a~closer look at the belief found among Rothbard and his followers that in economics only demonstrated preferences matter, and non-demonstrated preferences should not have any theoretical consequences.



\section{``Only demonstrated preferences matter in economics''}

At this point, I~will present two possible interpretations of the assumption that in economics, only demonstrated preferences matter. To the best of my knowledge, these interpretations were presented for the first time by Michal Kvasnička, who writes:



\begin{quote}
Rothbard rejects from analysis everything which is not demonstrated in an actual action, i.e. what goes beyond the scope of the demonstrated preference, as a~vain psychology. We can read this in two ways: 1) we can know nothing that was not demonstrated in an action, or 2) there is nothing more than what was demonstrated in an action. While Rothbard might have the first in mind, he spoke as if he believed the second. 
%\label{ref:RNDLs5MnzUUCg}(Kvasnička, 2008, p.44)
\parencite[][p.44]{kvasnicka_rothbards_2008}%
\end{quote}




As I~will try to prove, the strong version of DP (2) is contrary to common sense and has been effectively challenged by critics of Rothbardian methodology. Next, I~will try to show that a~weaker, more common-sense and criticism-resistant version of DP (1) avoids some accusations but does not have sufficient grounds. I~will also try to prove that both of these versions have similar theoretical consequences.



\subsection{Are there only demonstrated preferences?}



The strong version of DP could be defined as follows:

\medskip

\noindent \textbf{SVDP} (\textit{Strong Version of Demonstrated Preference}): There are only those preferences that are demonstrated in the actions of individuals (or: those that determine their actions).

\medskip

The rationale for such an interpretation of DP can be found in Rothbard's critique of so-called psychologizing 
%\label{ref:RNDdGAPN4T2uc}(Rothbard, 2011b, pp.296–298).
\parencite[][pp.296–298]{rothbard_present_2011}. %
 In turn, according to Kvasnička 
%\label{ref:RNDiTnIGsPOG8}(2008),
\parencite*[][]{kvasnicka_rothbards_2008}, %
 it is indicated primarily by some of Rothbard's arguments.



First, let us deal with the problem of psychologizing. When Rothbard criticizes Samuelson's concept of revealed preference for its implicit assumption of the constancy of preferences over time, he writes:



\begin{quote}
The revealed-preference doctrine is one example of what we may call the fallacy of ``psychologizing,'' the treatment of preference scales as if they existed as separate entities apart from real action. Psychologizing is a~common error in utility analysis. It is based on the assumption that utility analysis is a~kind of ``psychology,'' and that, therefore, economics must enter into psychological analysis in laying the foundations of its theoretical structure. 
%\label{ref:RNDINM3s5bSyQ}(Rothbard, 2011b, p.296)
\parencite[][p.296]{rothbard_present_2011}%
\end{quote}




Here, Rothbard seems to be expressing a~clear doubt about the existence of preferences apart from real actions. However, it is difficult to understand why the assumption of the existence of given preferences apart from actions Rothbard calls psychologizing since according to him psychology deals with the reasons \textit{why} people have certain preferences and how the preferences are formed:



\begin{quote}
Psychology analyzes the \textit{how} and the \textit{why} of people forming values. It treats the concrete content of ends and values. Economics, on the other hand, rests simply on the assumption of the existence of ends, and then deduces its valid theory from such a~general assumption. 
%\label{ref:RNDLcz4ZvPxhf}(Rothbard, 2011b, pp.296–297)
\parencite[][pp.296–297]{rothbard_present_2011}%
\end{quote}




One who assumes that people have preferences not demonstrated in actions does not have to deal with the reasons \textit{why} people have these preferences and \textit{how} these preferences are formed. Therefore, it does not seem that ``psychologizing'' is a~sufficient reason to reject the assumption of the existence of preferences apart from actions.



To understand why Rothbard's research practice seems to be based on the assumption that there are no preferences other than those demonstrated in the actions of individuals, it is worth considering at least two examples of the economist's conduct: indifference and externalities. As can be seen, DP serves Rothbard to eliminate from economic theory the concept of indifference (and the indifference curves known in mainstream economics), supposedly based on ``psychologizing''. As Rothbard states, ``indifference'' may be an important concept in psychology, but not in economics (praxeology). This is because indifference cannot be demonstrated in action. Each action necessarily demonstrates a~strong preference for a~particular state of affairs, whereas: ``Indifference classes are assumed to exist somewhere underlying and apart from action'' 
%\label{ref:RND43VZaZlll7}(Rothbard, 2011b, pp.304–305).
\parencite[][pp.304–305]{rothbard_present_2011}.%




However, as Nozick 
%\label{ref:RNDYCeHNHdhH7}(1977)
\parencite*[][]{nozick_austrian_1977} %
 noted in his famous critique, the Austrians need the concept of indifference to define even such elementary concepts as the supply of goods or the law of diminishing marginal utility, because an economic actor must be indifferent to the units of the same good (each unit of a~particular good must be worth exactly the same to him). Even if we cannot determine who perceives what things as units of the same good, in the economic analysis, we must assume that such a~phenomenon as indifference exists\footnote{Nozick's paper launched a~long-lasting debate on indifference within the Austrian camp 
%\label{ref:RNDZeA2gwNcG7}(see e.g., Block, 1980; 2009; Block and Barnett II, 2010; Hoppe, 2005; 2009; Machaj, 2007; Wysocki, 2016; 2017).
\parencites[see e.g.,][]{block_robert_1980}[][]{block_rejoinder_2009}[][]{block_rejoinder_2010}[][]{hoppe_note_2005}[][]{hoppe_further_2009}[][]{machaj_praxeological_2007}[][]{wysocki_indifference_2016}[][]{wysocki_note_2017}. %
 For a~review of the debate on indifference and a~defence of this concept in ASE, see 
%\label{ref:RNDNIjQ4PVm5l}(Wysocki, 2021).
\parencite[][]{wysocki_austro-libertarian_2021}.%
}.



Rothbard does the same for externalities\footnote{The only negative externalities that Rothbard allows are those that violate someone's property rights.}. In his essay, he cites the example of an envious man who could be worse off due to other people's voluntary actions, and unequivocally rejects this possibility because the envious man cannot demonstrate his preferences:



\begin{quote}
But what about Reder's bogey: the envious man who hates the benefits of others? To the extent that he himself has participated in the market, to that extent he reveals that he likes and benefits from the market. And we are not interested in his opinions about the exchanges made by others, since his preferences are not demonstrated through action and are therefore irrelevant. How do we know that this hypothetical envious one loses in utility because of the exchanges of others? Consulting his verbal opinions does not suffice, for his proclaimed envy might be a~joke or a~literary game or a~deliberate lie. 
%\label{ref:RND4E9xQ1IzLQ}(Rothbard, 2011b, p.320)
\parencite[][p.320]{rothbard_present_2011}%
\end{quote}




According to Kvasnička 
%\label{ref:RNDP8yOStvwYM}(2008),
\parencite*[][]{kvasnicka_rothbards_2008}, %
 the examples of indifference and externalities show the fact that Rothbard in practice treats non-demonstrated preferences as if they did not exist at all (even if he declares that they are simply unknown). However, as Kvasnička argues, it is one thing to say that indifference cannot be demonstrated, and another thing that indifference does not exist 
%\label{ref:RNDdn1motFX7y}(Kvasnička, 2008, p.44)
\parencite[][p.44]{kvasnicka_rothbards_2008}%
\footnote{As he writes: ``The inability to demonstrate indifference is no proof there is no indifference but only that an outside observer cannot observe it, which is quite a~different thing. […] The indifference curves just describe the agent's inner world, in which Rothbard takes no interest, or rather denies its existence altogether.'' 
%\label{ref:RNDR7R6ck0BxE}(Kvasnička, 2008, p.44)
\parencite[][p.44]{kvasnicka_rothbards_2008}%
}. Next, he notes that SVDP suffers from serious drawbacks such as, say, that an individual cannot demonstrate his preferences passively and, therefore, he cannot demonstrate his losses of welfare (frustration of preferences). If, therefore, one gets passively robbed, then (based on the SVDP) it cannot be said that his preferences have been thwarted 
%\label{ref:RNDOsVmC1SN16}(Kvasnička, 2008, pp.45–46).
\parencite[][pp.45–46]{kvasnicka_rothbards_2008}. %
 According to Kvasnička, such an understanding of DP is not only contrary to common sense but also to the principle of Pareto-efficiency accepted by Rothbard and the research practice of this economist.



Nozick comes to a~similar interpretation of DP when he argues that one of the Austrians' theses on preferences states that: ``The notion of preference makes no sense apart from an actual choice made'' 
%\label{ref:RNDaKw2RsSh8U}(Nozick, 1977, p.370).
\parencite[][p.370]{nozick_austrian_1977}. %
 Nozick observes, however, that the Austrians must assume the existence of preferences that are not demonstrated in actions because otherwise they could not define even such an elementary concept as cost (or perhaps: opportunity cost):



\begin{quote}
If we are to speak of the cost of \textit{A}, when there is more than one other alternative rejected, it must \textit{make sense} to speak of preference apart from an actual choice or doing of the preferred alternative. If \textit{that} doesn't make sense, then neither does the notion of the \textit{cost} of the action which was actually chosen. 
%\label{ref:RNDFTI7LdQBmX}(Nozick, 1977, p.373)
\parencite[][p.373]{nozick_austrian_1977}%
\end{quote}




Caplan argues that Rothbard's refusal to acknowledge unobserved preferences is not only extreme behaviourism but also contrary to common sense. The introspective experience of the existence of preferences that are not demonstrated in actions is common. So, even if knowing someone else's mental states is more difficult, it is hard to deny that they exist 
%\label{ref:RNDqyPBWt8nr0}(Caplan, 1999, p.834).
\parencite[][p.834]{caplan_austrian_1999}.%




Finally, it is necessary to emphasize one thing related to preferences and preference scales. In this context, it is worth quoting Mises's remarks, reminiscent of SVDP:



\begin{quote}
one must not forget that the scale of values or wants manifests itself only in the reality of action. These scales have no independent existence apart from the actual behavior of individuals. The only source from which our knowledge concerning these scales is derived is the observation of a~man's actions. Every action is always in perfect agreement with the scale of values or wants because these scales are nothing but an instrument for the interpretation of a~man's acting. 
%\label{ref:RNDFbMuotTpv0}(Mises, 1998, p.95)
\parencite[][p.95]{mises_human_1998}%
\end{quote}




Of course, in the face of the above quote, it cannot be precluded that Mises would also subscribe to SVDP, which, I~argue, should be rejected. However, one more possibility may need to be considered. When Mises speaks of ``scales of values or needs'', he may mean fixed and perfectly ordered preferences of acting individuals (as in the rational choice theory). Undoubtedly, such a~phenomenon does not occur in reality, because human preferences can be constantly changing and, due to the actors' false beliefs, they can be contradictory (for example, one may simultaneously prefer socialism over capitalism and productive allocation of resources over waste, without recognizing that these preferences are mutually exclusive). So, even if speculation about ``fixed and ordered preference scales'' apart from the actions of individuals may be meaningless, preferences as such must exist.



Since common-sense realism can be considered an important element of economic theory in general 
%\label{ref:RNDJCCLQzL8a7}(Mäki, 2008),
\parencite[][]{hausman_realism_2008}, %
 and the Austrians in their concepts and theories necessarily refer to preferences existing apart from actions, there is no doubt that SVDP should be rejected. There is no reason to maintain such a~strong ontological claim that leads to the conclusion that so-called dispositional mental states (that can exist apart from actual awareness, e.g., memories) do not exist at all.



\subsection{Only demonstrated preferences should be taken into account}



The weak version of DP could be defined as follows:

\medskip

\noindent \textbf{WVDP} (\textit{Weak Version of Demonstrated Preference}): Only those preferences that are demonstrated in actions (or: those that determine individuals' actions) can serve as the basis for economic theories.

\medskip

This interpretation of DP seems more common-sensical, more benevolent, and most likely corresponds to Rothbard's intentions. It seems to be exposed in Walter E. Block's reply to Caplan:



\begin{quote}
Rothbard's argument is that only demonstrated preferences are ``genuine'' for economic theory, i.e., related to action. Pie in the sky ``wishes'' that a~person has (e.g., to buy ice cream without money, or to purchase it ``later'') are not preferences at all in the technical sense. Preference is defined, in this technical sense, as the ranking of ends upon which an action is based. This is what makes Caplan's claims about acting on the basis of indifference incorrect. 
%\label{ref:RNDT8mTWYFX3s}(Block, 1999, p.23)
\parencite[][p.23]{block_austrian_1999}%
\end{quote}




Block recognizes that the preferences that are demonstrated in actions as the only ones ``in the technical sense'' can be called preferences because only they lead individuals to actions. Preferences that do not lead to actions are supposed to be irrelevant in economics. Only those preferences that lead to actions are to be of economic importance.



But what is the reason for making such an assumption? The argument used by Rothbard and Block is apparently: ``Because only demonstrated preferences are known with certainty''. Rothbard rejects taking into account non-demonstrated preferences in economic theory because he considers it to be on a~par with ``psychologizing'', something illegitimate in economics. WVDP rejects such ``hypothetical imaginings'':



\begin{quote}
Demonstrated preference, as we remember, eliminates hypothetical imaginings about individual value scales. Welfare economics has until now always considered values as hypothetical valuations of hypothetical ``social states.'' But demonstrated preference only treats values as revealed through chosen action. 
%\label{ref:RNDpcEXa5dfX3}(Rothbard, 2011b, p.320)
\parencite[][p.320]{rothbard_present_2011}%
\end{quote}




Block writes in turn: ``as far as technical economics is concerned, we cannot take cognizance of those of Caplan's wishes which are not objectively revealed or demonstrated in action. How can we, as economists, even know they exist?'' 
%\label{ref:RNDNdMT80rPbQ}(Block, 1999, p.23).
\parencite[][p.23]{block_austrian_1999}.%




The question that arises in the face of the above remarks is: in what sense do we know demonstrated preferences ``objectively'' or ``with certainty''? Certainly not in such a~way that they are known to us in empirical reality since the knowledge of the mental states of others is based on thymology and can never be certain. In addition, thymology also seems to allow other people to have preferences apart from actions (since we recognize this phenomenon introspectively, we can -- as part of \textit{Verstehen} -- ascribe it to other people). Thymologically speaking, knowledge of both demonstrated and non-demonstrated preferences cannot be absolutely certain (see point 2). Since, as I~have tried to show above, the assumption that preferences exist apart from actions is well founded on the basis of common sense and is indispensable in the theories and concepts used in ASE, the presented argument does not seem sufficient to accept WVDP.



There is no doubt, however, that WVDP (in this context SVDP works as well as WVDP) is the foundation of the economic theory of Rothbard and his followers. For example, one of the reasons Rothbard rejects the standard monopoly theory, which allows monopoly prices to occur in the free market 
%\label{ref:RNDaIPjBgTEzd}(which Mises accepted, see: Mises, 1998),
\parencite[whichsee:][]{mises_human_1998}, %
 is that: ``In praxeology we are interested only in preferences that result in, and are therefore demonstrated by, real choices, not in the preferences themselves'' 
%\label{ref:RND78AysLpPTB}(Rothbard, 2009a, p.701).
\parencite[][p.701]{rothbard_man_2009}. %
 The concept of the supply of an economic good, which is the necessary basis for the theory of monopoly prices, is problematic because: ``A good cannot be independently established as such apart from consumer preference on the market'' (ibid.)\footnote{Rothbard also raises other objections, based on the principle of subjectivism. These seem stronger and are not the subject of the presented critique.}.



The Austrian critique of public goods theory also seems implicitly based on WVDP (in this context SVDP works as well as WVDP). According to Rothbard, Block, and Hoppe, any voluntary, free-market (i.e., not resulting from coercion or violence) situation is optimal from the point of view of consumers. And since -- according to the WVDP -- economic actors demonstrate their preferences only in actions, preferences for something they do not pursue in actions are irrelevant in economic theory. For example, all members of a~community would rather breathe clean air and would be willing to heat their houses with slightly more expensive, but more ecological methods. However, they burn garbage only because they are convinced that other members of this community would not change their behaviour, and the expected cost of encouraging them to do so exceeds the expected benefits. According to the WVDP, there is no problem of market failure and, consequently, of public goods: apparently, clean air is not a~sufficiently valued good for members of this community to strive for its acquisition 
%\label{ref:RNDvBRoB07bwD}(Block, 1983; Hoppe, 1989; 2006; Wiśniewski, 2018).
\parencites[][]{block_public_1983}[][]{hoppe_theory_1989}[][]{hoppe_economics_2006}[][]{wisniewski_economics_2018}.%




Mark R. Crovelli proceeds in a~similar way, rejecting game theory, or more precisely, its paradigmatic concept of the prisoner's dilemma. As this scholar argues, even if we can imagine a~hypothetical scenario in which possible choices of given individuals create a~prisoner's dilemma, we can never say with certainty that any prisoner's dilemma ever existed. As Crovelli argues, citing Rothbard and DP, only real actions provide absolutely certain knowledge about human preferences 
%\label{ref:RND4s4oEPXCQK}(Crovelli, 2006).
\parencite[][]{crovelli_trouble_2006}.%




Although Crovelli's argument seems inaccurate in the face of the considerations presented here, it helps us to formulate an important argument against the supporters of WVDP. To paraphrase Kvasnička's argument, it is one thing to say that we can never know with certainty whether and when such situations (monopoly prices, public goods, prisoner's dilemmas) occur, and it is another thing to say that such situations never occur and cannot occur. Undoubtedly, in reality, it can be difficult to determine when such situations occur. It should be noted, however, that this is a~thymological, not a~praxeological, issue, and is valid for all economic theories, not just welfare economics, monopoly theory, public goods theory, or prisoner's dilemmas.



As can be seen, although WVDP protects the Austrian methodology from the objection of a~lack of common sense, it leads to similar practical consequences. In addition, as I~have tried to show, the Austrians give a~dubious argument in favour of WVDP. Thus, rejecting from economic theory any considerations of preferences existing apart from actions seems unconvincing. In the next section, I~will try to present an interpretation of DP that is the only one that does not suffer from any serious objections.



\section{Only demonstrated preferences affect the social processes}

Even if it is difficult to assume that only demonstrated preferences matter in economics, perhaps it can be said that only demonstrated preferences matter in the economy. Undoubtedly, only those preferences that lead economic actors into actions have an impact on the market process. Since the analysis of the market process (as opposed to the analysis of economic equilibrium) is one of the fundamental distinctive features of ASE 
%\label{ref:RNDVcFAHmyH2I}(Rothbard, 2011b; Martin, 2015),
\parencites[][]{rothbard_present_2011}[][]{coyne_austrian_2015}, %
 it seems that the suggested interpretation fits well with the Austrian methodology.



The emphasis on the market process rather than on economic equilibrium dates back to the very beginnings of ASE. It can already be found in Carl Menger. Later, this is clearly visible in the works of Hayek 
%\label{ref:RNDsY5UoChRcm}(1945),
\parencite*[][]{hayek_use_1945}, %
 Mises 
%\label{ref:RNDk1GagtppBt}(1998),
\parencite*[][]{mises_human_1998}, %
 and Israel M. Kirzner 
%\label{ref:RND0jWufa1bsu}(1973).
\parencite*[][]{kirzner_competition_1973}. %
 The Austrians agree that equilibrium analyses do not correspond to the rich and dynamic complexity of market processes. They also reject some assumptions found in neoclassical economic models such as, say, that market participants have complete information. Instead, they show how individuals, entangled in a~specific context of time and place, take actions with incomplete knowledge and subjective, constantly changing preferences, and as a~result, influence other people's actions.



There is no doubt that preferences that do not determine people's actions do not affect the empirical reality and, therefore, cannot affect social processes. The real demand for specific goods arises when a~certain quantity of goods is purchased. Market prices are shaped by concrete transactions. Spontaneous order is an unintended result of human actions, not unrevealed preferences. It seems impossible to deny these statements. Therefore, if DP is to find an important place in ASE, it should mean just that: only actions (in contrast to dispositional mental states alone) are causally relevant for the social and economic processes.



\section{Conclusion}

In this article, my goal was to present a~systematic interpretation of DP based on the available literature on the subject. As I~have argued, the thymological interpretation of DP should be rejected (because it does not develop the method of \textit{Verstehen}). The interpretation linking DP to the condition of voluntariness seems problematic (due to the very concept of voluntariness) and insufficiently well-justified (due to the fact that involuntary actions also express preferences). Then, I~proceeded to an analysis of two variants of the assumption that, in economics, only demonstrated preferences matter. As I~have tried to show, SVDP is contrary to common sense and the research practice of the Austrians, and WVDP does not seem well enough justified. As a~result, I~conclude that the only interpretation of DP that is not exposed to serious objections is as follows: only actions affect the social and economic processes (that is to say, only actions are causally relevant). This interpretation, however, is not sufficient to draw some of the conclusions that Rothbard and his successors reach in the field of welfare economics, monopoly theory, public goods theory, or social dilemmas known from game theory. Other implications of these conclusions remain a~matter of further research.



\end{artengenv}

\label{megger-last}