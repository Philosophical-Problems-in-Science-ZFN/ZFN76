\documentclass[11pt,openany,leqno]{book} % oneside twoside / report book / openany - bibl nie musi rozpoczynać się od nieparzystej
\usepackage[X2,T1]{fontenc}
\usepackage[utf8]{inputenc}
\usepackage{setspace}

\usepackage[gray,table]{xcolor}

\raggedbottom %usuwa przerwy między akapitami, spowodowane przez klasę book
\frenchspacing %usuwa podwójną przerwę po krpoce (zdaniu)

\usepackage{pdfpages}

\usepackage{graphicx}
%\usepackage{svg}

%\usepackage{epic}
%\usepackage{color}
\usepackage{amsmath}
\usepackage{bm}	% \boldsymbol
%\usepackage{amsbsy}	% \boldsymbol
\usepackage{amscd}	% diagramy
\usepackage{amsfonts}	% fonty
\usepackage{mathrsfs}
\usepackage{amssymb}	% dodatek math
\usepackage{amstext}	% \text
\usepackage{amsthm}
\usepackage{amsmath}

\makeatletter
\newcommand{\leqnos}{\tagsleft@true\let\veqno\@@leqno}
\newcommand{\reqnos}{\tagsleft@false\let\veqno\@@eqno}
\makeatother

\usepackage{xfrac} %\sfrac{1}{2}
%\usepackage{nicefrac} %\nicefrac{1}{2}
\usepackage[normalem]{ulem}


\usepackage{longtable}
\usepackage{tabularx}
\renewcommand{\tabularxcolumn}[1]{m{#1}}
\newcolumntype{Y}{>{\centering\arraybackslash}X}
\newcolumntype{s}{>{\hsize=.8\hsize}X}
\renewcommand\arraystretch{1.2}
\usepackage{float} % hard positioning (flag H)
\renewcommand{\floatpagefraction}{.8}% floating on a single page only when figure is 80% of the vertical space
\usepackage{supertabular}


\usepackage{array}
\usepackage{ragged2e}

\newcolumntype{L}[1]{>{\RaggedRight\let\newline\arraybackslash}m{#1}}

\usepackage{adjustbox}


%Equation z marginesami
\usepackage{environ}
\makeatletter
\NewEnviron{widerequation}{%
	\begin{equation*}
	\sbox\z@{\let\label\@gobble$\displaystyle\BODY$}
	\makebox[\textwidth]{%
		\begin{minipage}{\dimexpr\wd\z@+3em}
		\vspace{-\baselineskip}
		\begin{equation*}
		\BODY
		\end{equation*}
		\end{minipage}%
	}
	\end{equation*}
}


\usepackage{emptypage}


\usepackage[protrusion=true,expansion=true]{microtype}

\usepackage{multicol}


% Different font in captions
\newcommand{\captionfonts}{\small} %wydaje sie, ze nie dziala
\usepackage[font={small},singlelinecheck=false]{caption}
\usepackage[font={small},singlelinecheck=false]{subcaption}

%\captionsetup[subfigure]{font=small}


%---------------------------------------------------------

\usepackage{stmaryrd} %dla podwójnego nawiasu w mathmode (przypisanie interpretacji formule/zdaniu) \llbracket \rrbracket
\usepackage{textcomp} %dla podwójnego nawiasu w textmode (przypisanie interpretacji formule/zdaniu) \textlbrackdbl \textrbrackdbl
\usepackage{tikz-cd}
\usetikzlibrary{shapes.geometric,positioning,babel,arrows.meta}
\usepackage[title]{appendix}


\usepackage[hyphens,spaces,obeyspaces]{url}
\usepackage{hyperref} %to nowsza wersja pakietu url, lepsza m.in. robi linki hyperref
\hypersetup{colorlinks=true, linkcolor=black, citecolor=black, urlcolor=black, breaklinks=true, linktocpage}  %to sprawia że nie widać w pdfie aktywnych linków (od GM)
\urlstyle{rm}

%--------------------------------------------------------------------------------

\usepackage[russian,polutonikogreek,greek,german,english,polish]{babel}
\usepackage{textgreek}

\usepackage{geometry}
\geometry{verbose,paperwidth=745mm,paperheight=1005mm,landscape=false,
twoside,top=25mm,bottom=20mm,left=23mm,right=23mm,headheight=14pt}

\usepackage{changepage}

%-----spady---------

%\usepackage[cam,a4,center]{crop}
%\usepackage[width=151mm,height=211mm,center,cam,noinfo]{crop} % konsultuj z dokumentacją
%\usepackage[width=151mm,height=211mm,center,noinfo]{crop} % bex lini cięcia, konsultuj z dokumentacją

%-------------------

\hyphenation{pseudo-recursive pseudo-recursiveness pseudorecur-siveness pseudo-recursion}
\hyphenation{var-iety var-ieties}
\hyphenation{Tisch-ner Tisch-ner-a}
\hyphenation{Sie-ro-to-wicz}
\hyphenation{Nord-haus}
\hyphenation{Flei-schacker}
\hyphenation{Fuku-yama}
\hyphenation{Be-cke-ra Be-cker}
\hyphenation{Schwei-tze-ra}
\hyphenation{me-ta-e-ko-no-mi-czne}
\hyphenation{McCloskey}
\hyphenation{Hu-tchins}
\hyphenation{Kol-mogorov}




\usepackage{enumitem}
\setlist{noitemsep} % lub \setlist{nosep}

\setlistdepth{8}
\newlist{longitemize}{itemize}{8}
\setlist[longitemize,1]{label=$\bullet$}
\setlist[longitemize,2]{label=$\circ$}
\setlist[longitemize,3]{label=$\ast$}
\setlist[longitemize,4]{label=$\bullet$}
\setlist[longitemize,5]{label=$\circ$}
\setlist[longitemize,6]{label=$\ast$}
\setlist[longitemize,7]{label=$\bullet$}
\setlist[longitemize,8]{label=$\bullet$}

\usepackage{scrextend}


%------------------------- MyriadPro ------------------------------------

%\usepackage{Myriad}
\input{glyphtounicode.tex}
\pdfgentounicode=1
\usepackage{inconsolata}

\newcommand{\sansfontype}{\sfdefault}
%\newcommand{\sansfontype}{sourcesanspro}
%\newcommand{\sansfontype}{phv}


\usepackage[medfamily]{MyriadPro} %zainstalowene przy użyciu FontPro https://github.com/sebschub/FontPro

\newenvironment{namesur}{\fontfamily{\sansfontype}\fontseries{m}\fontshape{it}\fontsize{13}{13}\selectfont}{}
\newenvironment{chaptitle}{\fontfamily{\sansfontype}\fontseries{eb}\fontshape{n}\fontsize{18}{19}\selectfont}{}
\newenvironment{chaptitleeng}{\fontfamily{\sansfontype}\fontseries{eb}\fontshape{n}\fontsize{13}{15}\selectfont}{}
\newenvironment{chapsubtit}{\fontfamily{\sansfontype}\fontseries{b}\fontshape{n}\fontsize{13}{16}\selectfont}{}
\newenvironment{pagnum}{\fontfamily{\sansfontype}\fontseries{m}\fontshape{n}\fontsize{10}{12}\selectfont}{}
\newenvironment{sectit}{\fontfamily{\sansfontype}\fontseries{b}\fontshape{n}\fontsize{13}{14}\selectfont}{}
\newenvironment{subsectit}{\fontfamily{\sansfontype}\fontseries{m}\fontshape{n}\fontsize{12}{14}\selectfont}{}
\newenvironment{subsubsectit}{\fontfamily{\sansfontype}\fontseries{l}\fontshape{n}\fontsize{12}{14}\selectfont}{}
\newenvironment{pagname}{\fontfamily{\sansfontype}\fontseries{m}\fontshape{n}\fontsize{9.5}{11.4}\selectfont}{}
\newenvironment{pagtit}{\fontfamily{\sansfontype}\fontseries{m}\fontshape{it}\fontsize{9.5}{11.4}\selectfont}{}
\newenvironment{toctit}{\fontfamily{\sansfontype}\fontseries{b}\fontshape{n}\fontsize{16}{16}\selectfont}{}
\newenvironment{subbold}{\fontfamily{\sansfontype}\fontseries{m}\fontshape{n}\fontsize{12}{14}\selectfont}{}
\newenvironment{parttitle}{\fontfamily{\sansfontype}\fontseries{eb}\fontshape{n}\fontsize{14}{16}\selectfont}{}

\newenvironment{czw}{\fontfamily{\sansfontype}\fontseries{m}\fontshape{n}\fontsize{10}{12}\selectfont}{}
\newenvironment{czwccp}{\fontfamily{\sansfontype}\fontseries{m}\fontshape{n}\fontsize{9}{2}\selectfont}{}
\newenvironment{czwit}{\fontfamily{\sansfontype}\fontseries{m}\fontshape{it}\fontsize{10}{12}\selectfont}{}
\newenvironment{czwad}{\fontfamily{\sansfontype}\fontseries{eb}\fontshape{n}\fontsize{9.5}{11.4}\selectfont}{}


\newenvironment{rectitle}{\fontfamily{\sansfontype}\fontseries{eb}\fontshape{n}\fontsize{16}{15}\selectfont}{}
\newenvironment{reclight}{\fontfamily{\sansfontype}\fontseries{m}\fontshape{n}\fontsize{12}{14}\selectfont}{}
%\newenvironment{reclightit}{\fontfamily{\sansfontype}\fontseries{m}\fontshape{it}\fontsize{12}{14}\selectfont}{}
\newenvironment{recautor}{\fontfamily{\sansfontype}\fontseries{l}\fontshape{it}\fontsize{11}{13}\selectfont}{}

\newenvironment{toclight}{\fontfamily{\sansfontype}\fontseries{l}\fontshape{n}\fontsize{12}{14}\selectfont}{}
\newenvironment{tocbold}{\fontfamily{\sansfontype}\fontseries{b}\fontshape{n}\fontsize{11}{13}\selectfont}{}
\newenvironment{tocname}{\fontfamily{\sansfontype}\fontseries{m}\fontshape{n}\fontsize{10}{12}\selectfont}{}
\newenvironment{toctititem}{\fontfamily{\sansfontype}\fontseries{m}\fontshape{it}\fontsize{10}{12}\selectfont}{}

\newenvironment{bigtitle}{\fontfamily{\sansfontype}\fontseries{ub}\fontshape{n}\fontsize{30}{35}\selectfont}{}

\newenvironment{kolumnada}{\fontfamily{\rmdefault}\fontseries{m}\fontshape{n}\fontsize{9.6}{12.7}\selectfont}{}



%\renewcommand{\rmdefault}{txr}
%\renewcommand{\normalsize}{\fontsize{10.3}{15.2}\selectfont}
%\selectfont
%
%\makeatletter
%\renewcommand\small{%
%	\@setfontsize\small{9.3}{13}%
%	\abovedisplayskip 10\p@ \@plus2\p@ \@minus5\p@
%	\abovedisplayshortskip \z@ \@plus3\p@
%	\belowdisplayshortskip 6\p@ \@plus3\p@ \@minus3\p@
%	\def\@listi{\leftmargin\leftmargini
%		\topsep 6\p@ \@plus2\p@ \@minus2\p@
%		\parsep 3\p@ \@plus2\p@ \@minus\p@
%		\itemsep \parsep}%
%	\belowdisplayskip \abovedisplayskip
%}
%\renewcommand\footnotesize{%
%	\@setfontsize\footnotesize{8.3}{11}%
%	\abovedisplayskip 8\p@ \@plus2\p@ \@minus4\p@
%	\abovedisplayshortskip \z@ \@plus\p@
%	\belowdisplayshortskip 4\p@ \@plus2\p@ \@minus2\p@
%	\def\@listi{\leftmargin\leftmargini
%		\topsep 4\p@ \@plus2\p@ \@minus2\p@
%		\parsep 2\p@ \@plus\p@ \@minus\p@
%		\itemsep \parsep}%
%	\belowdisplayskip \abovedisplayskip
%}
%\makeatother


%%%%%%%%%%%%%%%%%%%

\usepackage{cmap} %polskie znaki przy kopiowaniu do innych edytorów

%----------inicjały-----------------------------------------
\usepackage{lettrine}
\makeatletter
\renewcommand\@makefntext[1]{%
  \noindent\makebox[0.3em][r]{\@makefnmark}#1}
\makeatother
%---------------------------------------------------------

\clubpenalty10000
\widowpenalty10000
%\flushbottom

%\widowpenalties 3 10000 10000 0
\displaywidowpenalties 2 10000 0



\usepackage{fancyhdr}
\pagestyle{fancyplain}
%\renewcommand{\sectionmark}[1]%
%                 {\markright{\thesection\ #1}}
%\renewcommand{\subsectionmark}[1]%
%                 {\markright{\thesubsection\ #1}}
\lhead[\fancyplain{}{\pagnum{\thepage}}]%
      {\fancyplain{}{\pagtit{\rightmark}}}
\rhead[\fancyplain{}{\pagtit{\leftmark}}]%
      {\fancyplain{}{\pagnum{\thepage}}}
\cfoot{}
\sloppy
\headsep10pt
\renewcommand{\sectionmark}[1]{} %usuwa tytuły sekcji z paginy górnej :/




\newcommand{\zfntitle}{Zagadnienia Filozoficzne w~Nauce}
\newcommand{\zfntitleeng}{Philosophical Problems in Science}
\newcommand{\rok}{2023} % lub rok
%\newcommand{\rok}{\the\year} % lub rok
\newcommand{\numerarabski}{75}
\newcommand{\numerrzymski}{LXXV}
\newcommand{\infbiblio}{\zfntitle, nr \numerarabski\ (\rok), ss.~}
\newcommand{\infbiblioeng}{\zfntitle, No \numerarabski\ (\rok), pp.~}

\usepackage{eso-pic}
\usepackage{rotating}
\usepackage[most]{tcolorbox}
\newcommand{\placetextbox}[3]{% \placetextbox{<horizontal pos>}{<vertical pos>}{<stuff>}
	\AddToShipoutPictureFG*{% Add <stuff> to current page foreground
		\put(\LenToUnit{#1},\LenToUnit{#2}){\begin{rotate}{90}\begin{tcolorbox}[hbox,boxsep=0pt,top=3mm,left=23mm,right=10mm, bottom=0mm,arc=0pt,auto outer arc,colback=gray!10,colframe=gray!10]\begin{minipage}[t][23mm][t]{\textwidth}#3\end{minipage}\end{tcolorbox}\end{rotate}}%
	}%
}%

\newcounter{tocnr}\setcounter{tocnr}{0}

%\newcommand{\pasekpl}{%
%\placetextbox{153mm}{-3mm}{\begin{subsubsectit}\infbiblio\pageref*{\tocnr}\textendash\pageref*{e\tocnr}\end{subsubsectit}\par%
%	\vspace{.4mm}%
%	\noindent\raisebox{-1.75pt}{\includegraphics[height=9.5pt]{images/CC-BY-NC-ND}}%
%	\begin{subsubsectit}\begin{small}\enspace CC-BY-NC-ND 4.0\end{small}\end{subsubsectit}}%
%}
%\newcommand{\pasekeng}{%
%	\placetextbox{153mm}{-3mm}{\begin{subsubsectit}\infbiblioeng\pageref*{\tocnr}\textendash\pageref*{e\tocnr}\end{subsubsectit}\par%
%		\vspace{.2mm}%
%		\noindent\raisebox{-1.75pt}{\includegraphics[height=9.5pt]{images/CC-BY-NC-ND}}%
%		\begin{subsubsectit}\begin{small}\enspace CC-BY-NC-ND 4.0\end{small}\end{subsubsectit}}%
%}


\newcommand{\pasekpl}{%
%	\placetextbox{153mm}{-3mm}{\begin{subsubsectit}\begin{small}%
%				\zfntitle\: (\zfntitleeng)%
%		\par\vspace{.1mm}%
%		nr \numerarabski\ (\rok), %
%		ss.~\pageref*{\tocnr}\textendash\pageref*{e\tocnr} \enspace $\bullet$ \enspace\end{small}%
%	\begin{footnotesize}CC-BY-NC-ND 4.0\end{footnotesize}\end{subsubsectit}%
%		\enspace\raisebox{-1.75pt}{\includegraphics[height=9.5pt]{_images/CC-BY-NC-ND}}}%
}
\newcommand{\pasekeng}{%
%	\placetextbox{153mm}{-3mm}{\begin{subsubsectit}\begin{small}%
%				\zfntitleeng\: (\zfntitle)%
%		\par\vspace{.1mm}%
%		No \numerarabski\ (\rok), %
%		pp.~\pageref*{\tocnr}\textendash\pageref*{e\tocnr} \enspace $\bullet$ \enspace\end{small}%
%		\begin{footnotesize}CC-BY-NC-ND 4.0\end{footnotesize}\end{subsubsectit}%
%		\enspace\raisebox{-1.75pt}{\includegraphics[height=9.5pt]{_images/CC-BY-NC-ND}}}%
}


%\newcommand{\pasekpl}{%
%	\placetextbox{153mm}{-3mm}{\begin{subsubsectit}\begin{small}%
%				\zfntitle\: (\zfntitleeng)%
%		\par\vspace{.1mm}%
%		nr \numerarabski\ (\rok), %
%		ss.~\pageref*{\tocnr}\textendash\pageref*{e\tocnr} \enspace $\bullet$ \enspace\end{small}%
%	\begin{footnotesize}CC-BY-NC-ND 4.0\end{footnotesize}\end{subsubsectit}%
%		\enspace\raisebox{-1.75pt}{\includegraphics[height=9.5pt]{_images/CC-BY-NC-ND}}}%
%}
%\newcommand{\pasekeng}{%
%	\placetextbox{153mm}{-3mm}{\begin{subsubsectit}\begin{small}%
%				\zfntitleeng\: (ZFN), \enspace%
%				No \numerarabski\ (\rok), %
%						pp.~\pageref*{\tocnr}\textendash\pageref*{e\tocnr} %
%		\par\vspace{.1mm}%
%			DOI: xxxxxxxxxxxxxxxxx
%		 \enspace $\bullet$ \enspace\end{small}%
%		\begin{footnotesize}CC-BY-NC-ND 4.0\end{footnotesize}\end{subsubsectit}%
%		\enspace\raisebox{-1.75pt}{\includegraphics[height=9.5pt]{_images/CC-BY-NC-ND}}}%
%}



\newenvironment{artplenv}[8]%
{	\setcounter{footnote}{0}%
	\setcounter{equation}{0}%
	\setcounter{section}{0}%
	\setcounter{subsection}{0}%
	\setcounter{equation}{0}%
	\setcounter{table}{0}%
	\setcounter{figure}{0}%
	\cleardoublepage%
	\widowpenalties 3 10000 10000 0%
	\thispagestyle{plain}\addtocounter{tocnr}{1}\newcommand\tocnr{x\arabic{tocnr}}% empty plain
	\phantomsection\label{\tocnr}% 
	\addtocontents{toc}{\noindent{\tocname{#1}}\\}%
	\addtocontents{toc}{{\toctititem#4% \tocname[#6]
	}\dotfill{\subsubsectit\pageref{\tocnr}}}%
	\addtocontents{toc}{\vskip.8\baselineskip}
	\begin{refsection}%
		\pasekpl%
	\begin{raggedleft}%
		{\flushright\chaptitle\color{black!70}{#2\par}}%
		\vspace{7mm}%
		{\flushright\subbold{#1}\\\subsubsectit\small{#5}\par}%
		\vspace{5mm}%
		\selectlanguage{english} %
		{\flushright\chaptitleeng\color{black!50}{#6}}\par%
	\end{raggedleft}%
	\vspace{5mm}%
	{\subsubsectit{\hfill Abstract}}\\{#7}\par%
	\vspace{2mm}%
	{\subsubsectit{\hfill Keywords}}\\{#8}%
	\vspace{5mm}%
	\selectlanguage{polish}%
	\markboth{\pagname\protect{#1}}{\pagtit\protect{#3}}%
	}%
{%
	\section*{Bibliografia}%
	\printbibliography[heading=none]\nopagebreak[4]%
	\label{e\tocnr}%
\end{refsection}%
}

\newcounter{labelhere}
\makeatletter
\newcommand\pagerefhere{%
 \stepcounter{labelhere}%
 \pageref{here\thelabelhere}\label{here\thelabelhere}}


\newenvironment{artengenv}[7]%
{	\setcounter{footnote}{0}%
	\setcounter{equation}{0}%
	\setcounter{section}{0}%
	\setcounter{subsection}{0}%
	\setcounter{equation}{0}%
	\setcounter{table}{0}%
	\setcounter{figure}{0}%
	\cleardoublepage%
	\widowpenalties 3 10000 10000 0%
	\thispagestyle{plain}\addtocounter{tocnr}{1}\newcommand\tocnr{x\arabic{tocnr}}% empty plain
	\phantomsection\label{\tocnr}% 
	\addtocontents{toc}{\noindent{\tocname{#1}}\\}%
	\addtocontents{toc}{{\toctititem#4}\dotfill{\subsubsectit\pageref{\tocnr}}}%
	\addtocontents{toc}{\vskip.8\baselineskip}
	\begin{refsection}%
		\selectlanguage{english}%
		\pasekeng%
		\begin{raggedleft}%
			{\flushright\chaptitle\color{black!70}{#2\par}}%
			\vspace{7mm}%
			{\flushright\subbold{#1}\\\subsubsectit\small{#5}\par}%
		\end{raggedleft}%
		\vspace{5mm}%
		{\subsubsectit{\hfill Abstract}}\\{#6}\par%
		\vspace{2mm}%
		{\subsubsectit{\hfill Keywords}}\\{#7}%
		\vspace{5mm}%
		\markboth{\pagname\protect{#1}}{\pagtit\protect{#3}}%
		}%
	{%	
		\newpage%
		\section*{Bibliography}%
		\printbibliography[heading=none]\nopagebreak[4]%
		\label{e\tocnr}%
	\end{refsection}%
}

%stworzne na potrzeby dwóch autorów, trzrba przemyśleć
\newenvironment{artengenv2auth}[8]%
{	\setcounter{footnote}{0}%
	\setcounter{equation}{0}%
	\setcounter{section}{0}%
	\setcounter{subsection}{0}%
	\setcounter{equation}{0}%
	\setcounter{table}{0}%
	\setcounter{figure}{0}%
	\cleardoublepage%
	\widowpenalties 3 10000 10000 0%
	\thispagestyle{plain}\addtocounter{tocnr}{1}\newcommand\tocnr{x\arabic{tocnr}}% empty plain
	\phantomsection\label{\tocnr}% 
	\addtocontents{toc}{\noindent{\tocname{#1}}\\}%
	\addtocontents{toc}{{\toctititem#4}\dotfill{\subsubsectit\pageref{\tocnr}}}%
	\addtocontents{toc}{\vskip.8\baselineskip}
	\begin{refsection}%
		\selectlanguage{english}%
		\pasekeng%
		\begin{raggedleft}%
			{\flushright\chaptitle\color{black!70}{#2\par}}%
			\vspace{7mm}%
			{#8}%
		\end{raggedleft}%
		\vspace{5mm}%
		{\subsubsectit{\hfill Abstract}}\\{#6}\par%
		\vspace{2mm}%
		{\subsubsectit{\hfill Keywords}}\\{#7}%
		\vspace{5mm}%
		\markboth{\pagname\protect{#1}}{\pagtit\protect{#3}}%
		}%
	{%
		\newpage%
		\section*{Bibliography}%
		\nopagebreak[4]%
		\printbibliography[heading=none]%
		\label{e\tocnr}%
	\end{refsection}%
}

\newenvironment{artplenv2auth}[9]%
{	\setcounter{footnote}{0}%
	\setcounter{equation}{0}%
	\setcounter{section}{0}%
	\setcounter{subsection}{0}%
	\setcounter{equation}{0}%
	\setcounter{table}{0}%
	\setcounter{figure}{0}%
	\cleardoublepage%
	\widowpenalties 3 10000 10000 0%
	\thispagestyle{plain}\addtocounter{tocnr}{1}\newcommand\tocnr{x\arabic{tocnr}}% empty plain
	\phantomsection\label{\tocnr}%
	\addtocontents{toc}{\noindent{\tocname{#1}}\\}%
	\addtocontents{toc}{{\toctititem#4}\dotfill{\subsubsectit\pageref{\tocnr}}}%
	\addtocontents{toc}{\vskip.8\baselineskip}
	\begin{refsection}%
		\pasekpl%
	\begin{raggedleft}%
		{\flushright\chaptitle\color{black!70}{#2\par}}%
		\vspace{7mm}%
		{#9}%
		\vspace{5mm}%
		\selectlanguage{english} %
		{\flushright\chaptitleeng\color{black!50}{#6}}\par%
	\end{raggedleft}%
	\vspace{5mm}%
	{\subsubsectit{\hfill Abstract}}\\{#7}\par%
	\vspace{2mm}%
	{\subsubsectit{\hfill Keywords}}\\{#8}%
	\vspace{5mm}%
	\selectlanguage{polish}%
	\markboth{\pagname\protect{#1}}{\pagtit\protect{#3}}%
	}%
{%
	\section*{Bibliografia}%
	\printbibliography[heading=none]\nopagebreak[4]%
	\label{e\tocnr}%
\end{refsection}%
}



\newenvironment{editorial}[5]%
{	\setcounter{footnote}{0}%
	\setcounter{equation}{0}%
	\setcounter{section}{0}%
	\setcounter{subsection}{0}%
	\setcounter{equation}{0}%
	\setcounter{table}{0}%
	\setcounter{figure}{0}%
	\cleardoublepage%
	\widowpenalties 3 10000 10000 0%
	\thispagestyle{plain}\addtocounter{tocnr}{1}\newcommand\tocnr{x\arabic{tocnr}}% empty plain
	\phantomsection\label{\tocnr}%
	\addtocontents{toc}{\noindent{\tocname{#1}}\\}%
	\addtocontents{toc}{{\toctititem#4}\dotfill{\subsubsectit\pageref{\tocnr}}}%
	\addtocontents{toc}{\vskip.8\baselineskip}
		\pasekpl%
		\begin{raggedleft}%
			{\flushright\chaptitle\color{black!70}{#2\par}}%
%			\vspace{3mm}%
			\selectlanguage{english} %
			{\flushright\chaptitleeng\color{black!50}{#5}}\par%
			\vspace{5mm}%
			{\flushright\subbold{#1}\par}%
			\vspace{5mm}%
		\end{raggedleft}%
		\vspace{5mm}%
		\selectlanguage{polish}%
		\markboth{\pagname\protect{#1}}{\pagtit\protect{#3}}%
	}%
	{%
	\label{e\tocnr}%
	}
	

\newenvironment{editorialeng}[5]%
{	\setcounter{footnote}{0}%
	\setcounter{equation}{0}%
	\setcounter{section}{0}%
	\setcounter{subsection}{0}%
	\setcounter{equation}{0}%
	\setcounter{table}{0}%
	\setcounter{figure}{0}%
	\cleardoublepage%
	\widowpenalties 3 10000 10000 0%
	\thispagestyle{plain}\addtocounter{tocnr}{1}\newcommand\tocnr{x\arabic{tocnr}}% empty plain
	\phantomsection\label{\tocnr}%
	\addtocontents{toc}{\noindent{\tocname{#1}}\\}%
	\addtocontents{toc}{{\toctititem#4}\dotfill{\subsubsectit\pageref{\tocnr}}}%
	\addtocontents{toc}{\vskip.8\baselineskip}
	\begin{refsection}%
		\pasekeng%
		\begin{raggedleft}%
			\selectlanguage{english} %
			{\flushright\chaptitle\color{black!70}{#2\par}}%
%			\vspace{3mm}%
%			{\flushright\chaptitleeng\color{black!50}{#5}}\par%
%			\vspace{5mm}%
			{\flushright\subbold{#1}\\\subsubsectit\small{#5}\par}%
		\end{raggedleft}%
		\vspace{5mm}%
		\selectlanguage{english}%
		\markboth{\pagname\protect{#1}}{\pagtit\protect{#3}}%
}%
{%
		\newpage%
		\section*{Bibliography}%
		\printbibliography[heading=none]\nopagebreak[4]%
		\label{e\tocnr}%
	\end{refsection}%
\label{e\tocnr}%
}



\newenvironment{recplenv}[4]%
{%	
	\setcounter{footnote}{0}%
	\setcounter{equation}{0}%
	\setcounter{section}{0}%
	\setcounter{subsection}{0}%
	\setcounter{equation}{0}%
	\setcounter{table}{0}%
	\setcounter{figure}{0}%
	\clearpage%
	\widowpenalties 2 10000 0%
	\thispagestyle{empty}\addtocounter{tocnr}{1}\newcommand\tocnr{x\arabic{tocnr}}% empty plain
	\phantomsection\label{\tocnr}% 
	\begin{multicols}{2}%
		\raggedcolumns%
		\begin{kolumnada}%
	\begin{refsection}%
		\selectlanguage{polish} %
		\pasekpl%
		\begin{raggedleft}%
			{\flushright\rectitle\color{black!70}{#2\par}}%chaptitle
			\vspace{2mm}%
			{\flushright\reclight\small{#4}\par}%subbold reclight
			\vspace{3mm}%
		\end{raggedleft}%
		%\markboth{\pagname\protect{#1}}{\pagtit\protect{#2}}%
		\fancyhead[C]{\pagtit{Recenzje}}
		\markboth{}{}%
		\addtocontents{toc}{\noindent{\tocname{#1}}\\}%
		\addtocontents{toc}{{\toctititem#3}\dotfill{\subsubsectit\pageref{\tocnr}}}%
		\addtocontents{toc}{\vskip.8\baselineskip}}%
	{%
		\begin{spacing}{0.9}%
			\printbibliography[heading=none]%
		\end{spacing}%
		\label{e\tocnr}%
	\end{refsection}%
		\end{kolumnada}%
	\end{multicols}%
}

\newcommand{\autorrec}[1]{\nopagebreak[4]\medskip\raggedleft{\large\textsc{#1}}\par}
\newcommand{\autorrecaffil}[1]{\nopagebreak[4]\smallskip\raggedleft{#1}\par}
%\newcommand{\autorrec}[1]{\nopagebreak[4]\smallskip\raggedleft{\recautor{#1}}\par}


\newenvironment{recengenv}[4]%
{%	
	\setcounter{footnote}{0}%
	\setcounter{equation}{0}%
	\setcounter{section}{0}%
	\setcounter{subsection}{0}%
	\setcounter{equation}{0}%
	\setcounter{table}{0}%
	\clearpage%
	\widowpenalties 2 10000 0%
	\thispagestyle{empty}\addtocounter{tocnr}{1}\newcommand\tocnr{x\arabic{tocnr}}% empty plain
	\phantomsection\label{\tocnr}% 
	\addtocontents{toc}{\noindent{\tocname{#1}}\\}%
	\addtocontents{toc}{{\toctititem#3}\dotfill{\subsubsectit\pageref{\tocnr}}}%
	\addtocontents{toc}{\vskip.8\baselineskip}
	\begin{multicols}{2}%
		\raggedcolumns%
		\begin{kolumnada}%
			\begin{refsection}%
				\selectlanguage{english} %
				\pasekeng%
				\begin{raggedleft}%
					{\flushright\rectitle\color{black!70}{#2\par}}%chaptitle
					\vspace{2mm}%
					{\flushright\reclight\small{#4}\par}%subbold reclight
					\vspace{3mm}%
				\end{raggedleft}%
				%\markboth{\pagname\protect{#1}}{\pagtit\protect{#2}}%
				\fancyhead[C]{\pagtit{Book reviews}}
				\markboth{}{}%
				}%
			{%
				\begin{spacing}{0.9}%
					\printbibliography[heading=none]%
				\end{spacing}%
				\label{e\tocnr}%
			\end{refsection}%
		\end{kolumnada}%
	\end{multicols}%
}


\newenvironment{newrevplenv}[6]%
{	\setcounter{footnote}{0}%
	\setcounter{equation}{0}%
	\setcounter{section}{0}%
	\setcounter{subsection}{0}%
	\setcounter{equation}{0}%
	\setcounter{table}{0}%
	\setcounter{figure}{0}%
	\cleardoublepage%
	\widowpenalties 3 10000 10000 0%
	\thispagestyle{plain}\addtocounter{tocnr}{1}\newcommand\tocnr{x\arabic{tocnr}}% empty plain
	\phantomsection\label{\tocnr}% 
	\addtocontents{toc}{\noindent{\tocname{#1}}\\}%
	\addtocontents{toc}{{\toctititem#4}\dotfill{\subsubsectit\pageref{\tocnr}}}%
	\addtocontents{toc}{\vskip.8\baselineskip}
	\begin{refsection}%
		\selectlanguage{polish}%
		\pasekpl%
		\begin{raggedleft}%
			{\flushright\chaptitle\color{black!70}{#2\par}}%
			\vspace{5mm}%
			{\flushright\subbold{#1}\\\subsubsectit\small{#5}\par}%
			\vspace{3mm}%
			{\flushright\textsf{\small{#6}}\par}%subbold reclight
		\end{raggedleft}%
		\vspace{7mm}%
		\markboth{\pagname\protect{#1}}{\pagtit\protect{#3}}%
		}%
	{%
		\section*{Bibliografia}%
		\printbibliography[heading=none]\nopagebreak[4]%
		\label{e\tocnr}%
	\end{refsection}%
}

\newenvironment{editorialeng2auth}[5]%
{	\setcounter{footnote}{0}%
	\setcounter{equation}{0}%
	\setcounter{section}{0}%
	\setcounter{subsection}{0}%
	\setcounter{equation}{0}%
	\setcounter{table}{0}%
	\setcounter{figure}{0}%
	\cleardoublepage%
	\widowpenalties 3 10000 10000 0%
	\thispagestyle{plain}\addtocounter{tocnr}{1}\newcommand\tocnr{x\arabic{tocnr}}% empty plain
	\phantomsection\label{\tocnr}%
	\addtocontents{toc}{\noindent{\tocname{#1}}\\}%
	\addtocontents{toc}{{\toctititem#4}\dotfill{\subsubsectit\pageref{\tocnr}}}%
	\addtocontents{toc}{\vskip.8\baselineskip}
	\begin{refsection}%
		\pasekeng%
		\begin{raggedleft}%
			\selectlanguage{english} %
			{\flushright\chaptitle\color{black!70}{#2\par}}%
%			\vspace{3mm}%
%			{\flushright\chaptitleeng\color{black!50}{#5}}\par%
%			\vspace{5mm}%
			{{#5}\par}%
		\end{raggedleft}%
		\vspace{5mm}%
		\selectlanguage{english}%
		\markboth{\pagname\protect{#1}}{\pagtit\protect{#3}}%
}%
{%
		\newpage%
		\section*{Bibliography}%
		\printbibliography[heading=none]\nopagebreak[4]%
		\label{e\tocnr}%
	\end{refsection}%
\label{e\tocnr}%
}


\newenvironment{newrevplenv2auth}[6]%
{	\setcounter{footnote}{0}%
	\setcounter{equation}{0}%
	\setcounter{section}{0}%
	\setcounter{subsection}{0}%
	\setcounter{equation}{0}%
	\setcounter{table}{0}%
	\setcounter{figure}{0}%
	\cleardoublepage%
	\widowpenalties 3 10000 10000 0%
	\thispagestyle{plain}\addtocounter{tocnr}{1}\newcommand\tocnr{x\arabic{tocnr}}% empty plain
	\phantomsection\label{\tocnr}% 
	\addtocontents{toc}{\noindent{\tocname{#1}}\\}%
	\addtocontents{toc}{{\toctititem#4}\dotfill{\subsubsectit\pageref{\tocnr}}}%
	\addtocontents{toc}{\vskip.8\baselineskip}
	\begin{refsection}%
		\selectlanguage{polish}%
		\pasekpl%
		\begin{raggedleft}%
			{\flushright\chaptitle\color{black!70}{#2\par}}%
			\vspace{5mm}%
			{{#5}\par}%
			\vspace{3mm}%
			{\flushright\textsf{\small{#6}}\par}%subbold reclight
		\end{raggedleft}%
		\vspace{7mm}%
		\markboth{\pagname\protect{#1}}{\pagtit\protect{#3}}%
		}%
	{%
		\section*{Bibliografia}%
		\printbibliography[heading=none]\nopagebreak[4]%
		\label{e\tocnr}%
	\end{refsection}%
}


\newenvironment{newrevengenv}[6]%
{	\setcounter{footnote}{0}%
	\setcounter{equation}{0}%
	\setcounter{section}{0}%
	\setcounter{subsection}{0}%
	\setcounter{equation}{0}%
	\setcounter{table}{0}%
	\setcounter{figure}{0}%
	\cleardoublepage%
	\widowpenalties 3 10000 10000 0%
	\thispagestyle{plain}\addtocounter{tocnr}{1}\newcommand\tocnr{x\arabic{tocnr}}% empty plain
	\phantomsection\label{\tocnr}% 
	\addtocontents{toc}{\noindent{\tocname{#1}}\\}%
	\addtocontents{toc}{{\toctititem#4}\dotfill{\subsubsectit\pageref{\tocnr}}}%
	\addtocontents{toc}{\vskip.8\baselineskip}
	\begin{refsection}%
		\selectlanguage{english}%
		\pasekeng%
		\begin{raggedleft}%
			{\flushright\chaptitle\color{black!70}{#2\par}}%
			\vspace{5mm}%
			{\flushright\subbold{#1}\\\subsubsectit\small{#5}\par}%
			\vspace{3mm}%
			{\flushright\textsf{\small{#6}}\par}%subbold reclight
		\end{raggedleft}%
		\vspace{7mm}%
		\markboth{\pagname\protect{#1}}{\pagtit\protect{#3}}%
		}%
	{%
		\newpage%
		\section*{Bibliography}%
		\printbibliography[heading=none]\nopagebreak[4]%
		\label{e\tocnr}%
	\end{refsection}%
}


% ----------------------------------------------------------------------

%\setstretch{1.5}

\renewcommand{\part}[1]{{\par\vskip 2\baselineskip
\flushright\parttitle{#1}\par\nopagebreak[4]\addtocounter{part}{1}}}

\makeatletter
\newcommand{\dodateknumer}{\Roman{section}}
\newcommand{\sekcjanumer}{\arabic{section}}
\renewcommand*\thesection{%
	\ifnum\pdfstrcmp{\@currenvir}{appendices}=0%
	\dodateknumer%
	\else%
	\sekcjanumer%
	\fi
}
\makeatother

%\renewcommand{\section}[1]{{\par\vskip 2\baselineskip
%\flushright\sectit{\thesection.~#1}\par\nopagebreak[4]\vskip 1\baselineskip\addtocounter{section}{1}}}

\usepackage{titlesec}
\makeatletter
\newcommand*{\justifyheading}{\raggedleft}
\newcommand{\setappendix}{Dodatek~\thesection:}
\newcommand{\setsection}{\thesection.}
\titleformat{\section}
{\sectit\justifyheading}{%
\ifnum\pdfstrcmp{\@currenvir}{appendices}=0%
  	\setappendix%
  	\else%
  	\setsection%
  	\fi
}{0.3em}{}
\titlespacing*{\section}{0pt}{2.0\baselineskip}{1.0\baselineskip}[0pt]
%
\titleformat{\subsection}
{\subsectit\justifyheading}{%
\thesubsection}{0.3em}{}
%
\titleformat{\subsubsection}
{\subsubsectit\justifyheading}{%
\thesubsection}{0.3em}{}
\makeatother


%\titleformat{\part}
%  {\parttitle\justifyheading}{\thepart:}{0em}{}
%\titlespacing*{\part}{0pt}{2.0\baselineskip}{1.0\baselineskip}[0pt]



\newcommand{\sectionno}[1]{{\par\vskip 2\baselineskip
\flushright\sectit{#1}\par\nopagebreak[4]\vskip 1\baselineskip}}


%\renewcommand{\subsection}[1]{{\par\vskip 2\baselineskip
%\flushright\subsectit{#1}\par\nopagebreak[4]\vskip 1\baselineskip\addtocounter{subsection}{1}}}

\newcommand{\subsectionno}[1]{{\par\vskip 2\baselineskip
		\flushright\subsectit{#1}\par\nopagebreak[4]\vskip 1\baselineskip}}

%\renewcommand{\subsubsection}[1]{{\par\vskip 2\baselineskip
%\flushright\subsubsectit{#1}\par\nopagebreak[4]\vskip 1\baselineskip\addtocounter{subsubsection}{1}}}

\newcommand{\subsubsectionno}[1]{{\par\vskip 2\baselineskip
		\flushright\subsubsectit{#1}\par\nopagebreak[4]\vskip 1\baselineskip}}



%----------------TOC--------------------
\setcounter{tocdepth}{0}
\makeatletter
\addto\captionsenglish{\renewcommand\contentsname{Table of contents}}
%\renewcommand\tableofcontents{%
%    \vspace*{50\p@}%
%    \section*{\toctit{\contentsname}%
%    \@mkboth{%
%          \czwit\contentsname}{\czwit\contentsname}}%
%          \vskip0.5in%
%    \@starttoc{toc}%
%}
\renewcommand\tableofcontents{%
	\begin{flushright}%
	{\bigtitle\color{black!70}{Philosophical\\Problems\\in Science\par}}%
	\vskip0.35in%
	{\chaptitleeng\color{black!70}{Zagadnienia Filozoficzne w~Nauce\par}}%
	\vskip0.25in%
	\hrule%
	\medskip%
	{\bigtitle\color{black!50}{\numerrzymski\ }%
			\raisebox{1.5pt}{\chaptitle$\blacksquare$}%
			{\ \bigtitle\rok}}%
	\medskip%
	\hrule%
	\end{flushright}%
		\@mkboth{\czwit\contentsname}{\czwit\contentsname}%
	\vskip0.5in%
	\@starttoc{toc}%
}
 \renewcommand*\l@section{\@dottedtocline{1}{0em}{0em}}
 \renewcommand*\l@subsection{\@dottedtocline{1}{0em}{1em}}
% \renewcommand{\cftchapterfont}{\raggedleft}
\makeatother

%\usepackage[titles]{tocloft}
%\renewcommand{\contentsname}{Table of contents}
%\renewcommand{\cftchapfont}{\raggedleft\reclight}% titles in bol

%\usepackage{titletoc}% http://ctan.org/pkg/titletoc
%\titlecontents*{chapter}% <section-type>
%[0pt]% <left>
%{}% <above-code>
%{}% <numbered-entry-format>
%{\reclight}% <numberless-entry-format>
%{}% <filler-page-format>


\newcommand{\sekcja}[2]{%
	\cleardoublepage
	\thispagestyle{empty}
	\vspace*{1.7in}%
	\begin{flushright}
		\chaptitle\color{black!70}{#1}\par
		\vspace*{.5in}%
		\chaptitleeng\color{black!50}{#2}\par
	\end{flushright}
	\vfill
	\cleardoublepage
%	\addcontentsline{toc}{chapter}{#1}%
	\addtocontents{toc}{\vskip0.7\baselineskip}
	\addtocontents{toc}{%
		\hfill%
		{\noindent\toclight{#1}\par}%
		\hfill%
		\mbox{\noindent\toclight\small{#2}\par}%
	}%
		\addtocontents{toc}{\vskip0.5\baselineskip}
}

\newcommand{\sekcjatoconly}[2]{%
	%	\addcontentsline{toc}{chapter}{#1}%
	\addtocontents{toc}{\vskip0.7\baselineskip}
	\addtocontents{toc}{%
		\hfill%
		{\noindent\toclight{#1}\par}%
		\hfill%
		\mbox{\noindent\toclight\small{#2}\par}%
	}%
		\addtocontents{toc}{\vskip0.5\baselineskip}
}



%---------------bibliografia-------------------------

\usepackage[%
style=bath,
citestyle=authoryear-comp,
sorting=nyt,
%bibstyle=numeric,
uniquename=false,
dateabbrev=false,
%sortcites=false,
useprefix=false,
maxbibnames=4,
doi=true,
%autolang=other,
backend=biber]{biblatex}
%uwaga ustew biber jako program do biliografii lub" pdflatex -> biber -> pdflatex -> pdflatex
%readme: ftp://sunsite.icm.edu.pl/pub/CTAN/macros/latex/contrib/biblatex/doc/biblatex.pdf

\setcounter{biburllcpenalty}{7000}
\setcounter{biburlucpenalty}{8000}


\usepackage[autostyle=true]{csquotes}


%----dodaje prefixy nazwisk (von, de) do bibliografii zachowujac sortowanie
%----useprefix musi być ustawione na false
\makeatletter
\AtBeginDocument{\toggletrue{blx@useprefix}}
\makeatother

%---prefixy nazwisk minuskułami
\renewbibmacro*{begentry}{\midsentence}

%\renewbibmacro*{bybookauthor}{%
%  \ifnamesequal{author}{bookauthor}
%    {\usebibmacro{cite:idem}}
%    {\printnames{bookauthor}}}


%\renewbibmacro*{in:}{%
%  \setunit{\addcomma\space}%
%  \printtext{%
%    \bibstring{in}\intitlepunct}}





\addbibresource{ART_Ceglarska/Ceglarska_pu.bib}
\addbibresource{ART_Cordoba/Oliva_pu.bib}
\addbibresource{ART_Dominiak/Dominiak_pu.bib}
\addbibresource{ART_Israel/Israel_pu.bib}
\addbibresource{ART_Linsbichler/Linsbichler_pu.bib}
\addbibresource{ART_Machaj/Machaj_pu.bib}
\addbibresource{ART_McGee/McGee_pu.bib}
\addbibresource{ART_Megger/Megger_pu.bib}
\addbibresource{ART_Nowakowski/Nowakowski_pu.bib}
\addbibresource{ART_Posvanc/Posvanc_pu.bib}
\addbibresource{ART_Slenzok/Slenzok_pu.bib}
\addbibresource{ART_Turowski/lachmann_PU.bib}
\addbibresource{ART_Wysocki_etal/Wysocki_etal_pu.bib}
\addbibresource{DIS_Block/Block_pu.bib}
\addbibresource{DIS_Wysocki/Wysocki_pu.bib}
\addbibresource{REV_Czyzniewski/Czyzniewski_pu.bib}
\addbibresource{EDI_Block_Wysocki/editorial.bib}



\renewcommand*{\nameyeardelim}{\addcomma\space}
\renewcommand{\compcitedelim}{\multicitedelim}
%\assignrefcontextentries[]{*}
\renewcommand*{\bibfont}{\small}
\setlength\bibitemsep{0.1\itemsep}

%Zmienia Wyed. na red.
\usepackage{xpatch}
\NewBibliographyString{series}
\DefineBibliographyStrings{polish}{%
%	byeditor = {red\adddot},%
%	in = {[w\addcolon]},%
	urlseen = {ostatni dostęp\addcolon},%
	online = {Online},%
	urlfrom = {Dostępne na\addcolon},%
	page = {s\adddot~},%
	pages = {ss\adddot~},%
	idem = {Tenże},%
%	part = {część},%
	bytranslator = {{}tłum.},%
	translator = {{}tłum.},%
%	andothers = {i~in\adddot\addcomma},%
	andothers = {i~in\adddot},%
	series = {seria\addcolon\space},%
}

\DefineBibliographyStrings{english}{%
	urlfrom = {Available at:},% w oryginale Available from: moim zdaniem ok
	series = {},%
	online = {Online},%
%	in = {[in\addcolon]},%
	unpublished = {[unpublished]}
}


%--- Journal titles as is

\newbibmacro*{journal}{%
  \iffieldundef{journaltitle}
    {}
    {\printtext[journaltitle]{%
       \printfield[pujournaltits]{journaltitle}%
       \setunit{\subtitlepunct}%
       \printfield[pujournaltits]{journalsubtitle}}}}
       
\DeclareFieldFormat{pujournaltits}{#1}


% print url if no doi
\renewbibmacro*{doi+eprint+url}{%
    \printfield{doi}%
    \newunit\newblock%
    \iftoggle{bbx:eprint}{%
        \usebibmacro{eprint}%
    }{}%
    \newunit\newblock%
    \iffieldundef{doi}{%
        \usebibmacro{url+urldate}}%
        {}%
    }
    
%don't print "[Online]" keyword
\renewbibmacro*{isonline}{}

%Ensuring the name of the translator is not occuring twice in articles
\xpatchbibdriver{article}{
  \usebibmacro{byeditor+others}}{}{}{\wlog{WARNING: biblatex-bath failed to patch article driver}}











%\newcommand*{\doi}[1]{\href{http://dx.doi.org/#1}{doi: #1}}
%\DeclareFieldFormat{doi}{DOI\addcolon\space\url[#1]{https://doi.org/#1}}
\DeclareFieldFormat{doi}{\url{https://doi.org/#1}}
%\DeclareFieldFormat{doi}{\href{https://doi.org/#1}{https://doi.org/#1}}
%\DeclareFieldFormat{doi}{\href{https://doi.org/#1}{#1}}



%-- online no-date without "n.d."

\DeclareLabeldate[online]{%
  \field{date}
  \field{year}
  \field{eventdate}
  \field{origdate}
  \field{urldate}
}


\setlength{\bibhang}{1.5em}

%\usepackage{xparse}
%
%\ExplSyntaxOn
%\NewExpandableDocumentCommand{\mkbibordinale}{m}
% {
%  \tl_if_numeric:nTF { #1 }
%   {
%    \int_compare:nNnTF { #1 } = { 1 }
%     {
%      #1\textsuperscript{st}
%     }
%     {
%      \int_compare:nNnTF { #1 } = { 2 }
%       {
%        #1\textsuperscript{nd}
%       }
%       {
%        \int_compare:nNnTF { #1 } = { 3 }
%         {
%          #1\textsuperscript{rd}
%         }
%         {
%          #1\textsuperscript{th}
%         }
%       }
%     }
%   }
%   {
%    #1 % Just return the input without any superscript if it's not numeric
%   }
% }
%\ExplSyntaxOff
%
%\DeclareFieldFormat{edition}{\mkbibordinale{#1}~edition}


%\DeclareFieldFormat{superedition}{\textsuperscript{#1}}
%%\DeclareFieldFormat{edition}{\thefield{edition}\superedition}




%\DeclareFieldFormat[article]{volume}{{#1}}
%\DeclareFieldFormat{url}{<#1>}
\DeclareFieldFormat{url}{\bibsentence\bibstring{urlfrom}\addcolon\space<\url{#1}>} %dodaje nawiasy trójkątne do url
\DeclareFieldFormat[book,inbook,incollection,collection]{series}{\bibstring{series}\mkbibemph{#1}}
\iflanguage{polish}{\renewcommand*{\finalnamedelim}{\addspace\bibstring{and}~}} %dodaje twardą spację do Autor i~Autor
\renewcommand\bibnamedelimc{\addnbspace}
\renewcommand\bibnamedelimd{\addnbspace}

    
%\renewbibmacro*{cite:idem}{\bibstring[\mkidem]{idem\thefield‌{gender}}\setunit{\p‌​rintdelim{nametitled‌​elim}}}



\DeclareLanguageMapping{polish}{polish-apa}
\DeclareLanguageMapping{english}{british-bath}
\DeclareLanguageMapping{german}{german-apa}




\AtEveryBibitem{%
	\clearfield{titleaddon}%
	\clearfield{month}%
	\clearfield{note}%
	\clearlist{language}%
}
\AtEveryCitekey{\clearfield{month}}


%normalna czcionka przy arXiv
\makeatletter
\DeclareFieldFormat{eprint:arxiv}{%
	arXiv\addcolon\space
	\ifhyperref
	{\href{https://arxiv.org/\abx@arxivpath/#1}{%
			\nolinkurl{#1}%
			\iffieldundef{eprintclass}
			{}
			{\addspace\mkbibbrackets{\thefield{eprintclass}}}}}
	{\nolinkurl{#1}%
		\iffieldundef{eprintclass}
		{}
		{\addspace\mkbibbrackets{\thefield{eprintclass}}}}}
\makeatother


%linkowanie do bib entry
\newcommand{\citelink}[2]{\hyperlink{cite.\therefsection @#1}{#2}}



%\setcounter{biburlnumpenalty}{9000}
%\setcounter{biburllcpenalty}{7000}
%\setcounter{biburlucpenalty}{8000}



%----sieroty i wdowy w bibliografii-----
\usepackage{etoolbox,apptools}
\makeatletter
\AtBeginEnvironment{thebibliography}{%
	\clubpenalty10000
	\@clubpenalty \clubpenalty
	\widowpenalty10000}
\makeatother




%------przypisy----------------------------------------

%\usepackage{bigfoot}
%\DeclareNewFootnote{default}

%\usepackage[splitrule]{footmisc}

\interfootnotelinepenalty=5000


\makeatletter
\renewcommand\@makefntext[1]{\leftskip=0em\hskip0em\@makefnmark#1}
\makeatother

\renewcommand{\footnoterule}{\kern5pt\hrule width1in\kern5pt}

%lekko przedefiniowany przypis (odstęp między znaczkiem przypisu, a~tekstem przypisu)
\let\oldfootnotetext\footnotetext
\renewcommand\footnotetext[1]{\oldfootnotetext{\hspace{0.2em}#1}}

\renewcommand\footnote[1]{\footnotemark\footnotetext{#1}}

%przypis redakcji (oznaczany asteriskiem),
%nie zmienia numeracji przypisów autorskich
\newcommand{\edtfootnote}[1]{\renewcommand{\thefootnote}{*}\footnote{#1}\addtocounter{footnote}{-1}\renewcommand{\thefootnote}{\arabic{footnote}}}

%przypis bez numeru
\newcommand\blfootnote[1]{%
	\begingroup
	\renewcommand\thefootnote{}\footnote{#1}%
	\addtocounter{footnote}{-1}%
	\endgroup
}



\let\oldquote\quote
\let\endoldquote\endquote
\renewenvironment{quote}{\oldquote\small}{\endoldquote}

\let\oldquotation\quotation
\let\endoldquotation\endquotation
\renewenvironment{quotation}{\oldquotation\small}{\endoldquotation}



%\newenvironment{myquote}[1]{\begin{quote}\small} {\par\end{quote}}
%\newenvironment{myQuote}{\begin{quotation}{\small}} {\par\end{quotation}}
%\newcommand{\myquote}[1]{\begin{quote}{\small#1\par}\end{quote}}
%\newcommand{\myQuote}[1]{\begin{quotation}{\small#1\par}\end{quotation}}
\newcommand{\mydots}{\hbox to 1em{.\hss.\hss.}}


\newenvironment{myquoterev}{%
	\list{}{%
		\leftmargin0.3cm   % this is the adjusting screw
		\rightmargin\leftmargin
		\footnotesize
	}
	\item\relax
}
{\endlist\par}







\setlength{\parskip}{0ex plus 0.1ex minus 0.1ex}
\def\chaptername{}

%\makeatletter
%	\newenvironment{indentation}[3]%
%	{\par\setlength{\parindent}{#3}
%	\setlength{\leftmargin}{#1}       \setlength{\rightmargin}{#2}%
%	\advance\linewidth -\leftmargin       \advance\linewidth -\rightmargin%
%	\advance\@totalleftmargin\leftmargin  \@setpar{{\@@par}}%
%	\parshape 1\@totalleftmargin \linewidth\ignorespaces}{\par}%
%\makeatother













\usepackage{wrapfig}

\newlength{\defbaselineskip} % w celu uniknięcia formatowania względnego
\setlength{\defbaselineskip}{\baselineskip}
\newenvironment{adres}%
           {\setlength{\baselineskip}{.7\defbaselineskip}}{}
\newenvironment{adrescc}%
           {\setlength{\baselineskip}{.2\defbaselineskip}}{}           

% Komenda pozwala zmieniać interlinię w dokumencie



\usepackage[shortcuts]{extdash} %hyphens



\addto\captionspolish{\renewcommand{\figurename}{Ilustracja}}




%-----------------mottos-------------------------


\usepackage{epigraph}
\makeatletter
    \newlength\epitextskip
    \pretocmd{\@epitext}{\em}{}{}
    \apptocmd{\@epitext}{\em}{}{}
    \patchcmd{\epigraph}{\@epitext{#1}\\}{\@epitext{#1}\\[\epitextskip]}{}{}
\makeatother
\setlength\epigraphrule{0pt}
\setlength\epitextskip{.1ex}
\setlength\epigraphwidth{.8\textwidth}


\usepackage{mdframed}

\usepackage{mdframed}

% Define a new environment for the epigraph
\newenvironment{customepigraph}
  {% Begin code
   \begin{mdframed}[%
   linewidth=0pt,
   skipabove=\baselineskip,
   skipbelow=\baselineskip,
   innerleftmargin=0.2\textwidth,
   innerrightmargin=0\textwidth]
   \footnotesize
  }
  {% End code
   \end{mdframed}
  }



%%%%%%%%%%%%%%%%%%%%%%%%%%%%%%%%%%%%%%%
%Heathocote
\usepackage{xspace}
\usepackage[mla]{xellipsis}
\newcommand*{\sqrtwo}{\ensuremath{\sqrt{2}}\hspace{0.5ex}\xspace}
\newcommand{\sqrthree}{\ensuremath{\sqrt{3}}\hspace{0.5ex}\xspace}
\newcommand{\Hilb}{\ensuremath{\mathscr{H}}\xspace}
\newcommand{\Hilbtwo}{\ensuremath{\mathscr{H}^{2}}\xspace}
\newcommand{\etc}{\hspace{1pt}\textit{etc.}\@\xspace}
\newcommand{\name}[1]{\hspace{0.5ex}`\textit{#1}\hspace{0.45ex}'\hspace{0.55ex}\xspace}
\newcommand*{\thing}[1]{\hspace{0.75pt}\textit{#1}\hspace{1.5pt}\xspace}
\newcommand{\ksqrtwo}[1]{\hspace{0.4ex}\textit{#1}\ensuremath{\sqrt{2}}\hspace{0.5ex}\xspace}
\newcommand{\isqrtwo}[2]{\hspace{0.4ex}\textit{#1}\ensuremath{\sqrt{#2}}\hspace{0.5ex}\xspace}
\newcommand*{\qv}{q.v.\@}




 \begin{document}

%\includepdf[pages=1]{images/tyt.pdf}
%\newpage
%\cleardoublepage
%\includepdf[pages=2]{images/tyt.pdf}

%\newpage
%\includepdf[pages=1]{images/CT4.pdf}


\thispagestyle{empty}
\vspace*{1.2in}%
\begin{flushright}
\rectitle{Philosophical Problems\\in Science}\par
\vspace*{.5in}%
\chaptitleeng{Zagadnienia Filozoficzne\\w Nauce}\par
\end{flushright}
\vfill
\clearpage

\thispagestyle{empty}



\noindent\begin{czw}© Copernicus Center Foundation \& Authors, \rok\end{czw}

\vfill

\urlstyle{sf}
\begin{adres}
	\begin{pagname}\noindent Except as otherwise noted, the material in this issue is licenced under the Creative Commons BY-NC-ND licence. To view a copy of this licence, visit \url{http://creativecommons.org/licenses/by-nc-nd/4.0}.
	
	\end{pagname}
\end{adres}
\urlstyle{rm}

%\urlstyle{sf}
%\begin{czwccp}
%\linespread{2.0}
%\noindent Except as otherwise noted, the material in this issue is licenced under the Creative Commons BY-NC-ND 4.0 licence. To view a copy of this licence, visit \url{http://creativecommons.org/licenses/by/4.0}.
%\end{czwccp}
%\urlstyle{rm}

\vskip1.5em

%\begin{adres}
%	\begin{pagname}\noindent\begin{czwad}Lead Editor\end{czwad}\\
%			Cory Wright (California State University, Long Beach)
%			
%	\end{pagname}
%\end{adres}
%
%\vfill


\begin{adres}
	\begin{pagname}\noindent\begin{czwad}Editorial Board\end{czwad}\\
%	Guest Editors:  Gordana Dodig-Crnkovic, Roman Krzanowski\\
%		Gordana Dodig-Crnkovic (Lead Editor)\\
%		Roman Krzanowski (Lead Editor)\\
		Paweł Jan Polak (Editor-in-Chief)\\
		Janusz Mączka\\
		Roman Krzanowski\\
		Michał Heller (Honorary Editor)\\
		Piotr Urbańczyk (Editorial Secretary)
		
	\end{pagname}
\end{adres}





\vfill





%\noindent\begin{czw}Technical editor: Artur Figarski\end{czw}

\noindent\begin{czw}Cover design: Mariusz Banachowicz\end{czw}



%\noindent\begin{czw}Proofreading: Roman Krzanowski\end{czw}
%
%%\vskip.5em
%
%\noindent\begin{czw}Adjustment and correction: Artur Figarski\end{czw}
%
%%\vskip.5em
%
%\noindent\begin{czw}Cover design: Mariusz Banachowicz\end{czw}
%
%%\vskip.5em
%
%%\noindent\begin{czw}Technical editor: Artur Figarski\end{czw}
%
%%\vskip.5em
%
%\noindent\begin{czw}Typeset and typographic design: Piotr Urbańczyk\end{czw}
%
%%\vskip.5em
%
%%\noindent\begin{czw}Typeset in \end{czw}\LaTeX




\vskip1.5em

\vfill

\begin{adres}
	\begin{pagname}\noindent ISSN 0867-8286 (print format)\\
		e-ISSN 2451-0602 (electronic format)
	\end{pagname}
\end{adres}

\vskip1.5em

\vfill

\begin{adres}
		\noindent\begin{czwad}Editorial Office\end{czwad}
	
		\noindent\begin{pagname}Philosophical Problems in Science (ZFN)
		
		\noindent Copernicus Center Foundation
		
		\noindent Pl. Szczepański 8, 31-011 Kraków
		
		\noindent POLAND
		
		\vskip.3em
		
		\noindent e-mail: info@zfn.edu.pl
		
		\noindent www.zfn.edu.pl
		
		\end{pagname}
\end{adres}

\vfill

\begin{wrapfigure}{L}{.3\textwidth}
	\noindent\includegraphics[width=.29\textwidth]{_images/ccp.pdf}
%\noindent\includegraphics[width=.27\textwidth]{images/ccp.pdf}
\end{wrapfigure}
\vskip.4em
%\noindent\parbox[t]{6cm}{
\begin{adres}
	\noindent\begin{pagname}Publisher: Copernicus Center Press Sp. z o.o.
		
		\noindent Cholerzyn 501, 32-060 Liszki POLAND
		
		\noindent tel. (+48) 12 448 14 12
		
		\noindent e-mail: zamowienia@ccpress.pl
		
		\noindent www.ccpress.pl\\ \end{pagname}
	
	
\end{adres}
%}




\clearpage

%\input{toctest.tex}
%
%\clearpage

%Spis tresci -------------------------------------------------

{\thispagestyle{plain} \tableofcontents \clearpage}
%-------------------------------------------------------------
\newpage
\thispagestyle{plain}
\cleardoublepage
\thispagestyle{plain}

%-------------------------------------------------------------







%\input{default.tex}


\sekcjatoconly{Editorial}{Od redakcji}
%\sekcja{Editorial}{Od redakcji}
\begin{editorialeng2auth}{Walter Block, Igor Wysocki}
	{From a~nitty-gritty debate within economics into the deep waters of philosophy of science.
	Introduction to the special issue of ZFN}
	{From a~nitty-gritty debate within economics into the deep waters\ldots}
	{From a~nitty-gritty debate within economics into the deep waters\\of philosophy of science.
	Introduction to the special issue of ZFN}
	{%
			{\flushright\subbold{Walter Block}\\\subsubsectit\small{Loyola University New Orleans}\par}%
			{\flushright\subbold{Igor Wysocki}\\\subsubsectit\small{Nicolaus Copernicus University in Toruń}\par}%
	}









It all started in 2021 when we sparked a~rather specific debate within Austrian economics in \textit{Philosophical Problems in Science (Zagadnienia Filozoficzne w~Nauce)} – traditionally abbreviated as \textit{ZFN}. The story unfolded as follows. First, Wysocki submitted a~paper on the concept of indifference, as it is normally understood in the Austrian school of economics. To his astonishment and great relief, this then rising journal (now the one with well-established reputation, its Scopus ranking being as high as Q2 under the rubric of philosophy) accepted the submission in question and published it promptly in ZFN~71
%(2021)
as 
%\label{ref:RNDN6KKr0nHz7}(Wysocki, 2021).
\parencite[][]{wysocki_problem_2021}. %
 It was an honour and a~privilege, especially given the fact that Austrian economics -- with all due respect to its scientific achievements -- is nowadays not a~mainstream economic current, to say the least. Hence, being published in \textit{ZFN} only added to the strength of Wysocki's belief that the journal is clearly unbiased towards any sort of philosophy of science. The very fact that Wysocki published a~paper on indifference in Austrian economics and the fact that he spread the news about \textit{ZFN} being also open to publishing maverick (after all, as already observed, Austrian economics is not a~branch of mainstream economics) papers related to philosophy of science prompted Walter Block (a prominent Austrian economist) to submit to \textit{ZFN} his rejoinder to Wysocki's original paper on indifference, with the said response being ultimately published in ZFN~72
% (2022)
 as 
%\label{ref:RNDZQeCfSsaHd}(Block, 2022).
\parencite[][]{block_response_2022}.%




And thus the debate seemed to have unfolded. Block is well-known for being a~formidable debater and Wysocki knew perfectly well that if he had replied his criticism, Block would have come up with a~successive rejoinder shortly. Although Wysocki sensed that Block's response indeed called for a~rejoinder on the former's part, what prevented Wysocki from writing up his response was his scepticism as to \textit{ZFN}'s willingness to publish a~series of papers focused on nitty-gritty intricacies within the Austrian school of economics (in this case: indifference and how it relates to agents' actual choices). However, it was no less than Mr. Piotr Urbańczyk (an editor of the journal) who took the bull by the horns. The idea he came up with surpassed our wildest expectations. What Piotr suggested was not a~mere permission for us to continue discussing indifference in Austrian economics. Nay, he proposed that we, as guest editors (what an honour!), should dedicate the whole special issue to tackle the philosophical foundations of economics as such, be it Austrian, neoclassical, or what have you.



Obviously, we immediately embarked on the opportunity. The very thought of getting some prominent scholars in the field to contribute to the special issue was riveting. To our delight, we also found it inconceivable that special invitees (we were told we can have as many as three of them) would refuse to contribute their respective papers, given the standing of \textit{ZFN}, definitely one of the best Polish philosophical journals. Surely, it is a~daunting task to pick up three special invitees in the universe of exquisite philosophers or economists. Still, after some deliberation and necessary narrowing down of the said universe, we managed to select the required trio, which was in the end: (1) professor Karl-Friedrich Israel (an economist at the Western Catholic University in Angers, France); (2) professor Alexander Linsbichler (an economist and a~philosopher associated with The~Johannes Kepler University Linz and the University of Vienna) and (3) professor Łukasz Dominiak (a political philosopher at the Nicolaus Copernicus University in Toruń). Amazingly enough, all three of them agreed to contribute a~respective paper to the forthcoming \textit{Special Issue}. At this point, we cannot do better than elaborate on our rationale for selecting this particular set of three special contributors.



First, professor Karl-Friedrich Israel is a~renowned economist, especially well-versed in the Austrian tradition. Without a~doubt, he is one of the most outstanding Austrians in the younger generation. He has recently (i.e. 2023) obtained his habilitation at University Paris 1 Panthéon-Sorbonne, while still being in his thirties, something truly exceptional. His expertise in adverse effects of monetary policies and in inflation is second to none 
%\label{ref:RNDaHgR3d6dBn}(Israel, 2022).
\parencite[][]{israel_monetary_2022}. %
 He also massively contributed to the \textit{Quarterly Journal of Austrian Economics}, a~flagship Austrian journal. His analytic apparatus and overall conceptual grasp are superb. We remember being deeply impressed by his co-authored paper (with Tate Fegley) on the disutility of labour 
%\label{ref:RNDr8oPFzpCQC}(Fegley and Israel, 2020).
\parencite[][]{fegley_disutility_2020}. %
 These two authors boldly went against the Misesian dogma holding the disutility of labour to be an auxiliary empirical proposition in Austrian economics. Somewhat ironically, the authors' critical attempt helped to advance the overall Misesian \textit{a~priori} scientific programme. In the present \textit{Special Issue} professor Karl-Friedrich Isreal joins forces with Tate Fegley yet again, thus producing a~paper \textit{A~Defense of Austrian Welfare Economics} 
%\label{ref:RNDkNpBEcghvM}(Fegley and Israel, 2024),
\parencite[][]{fegley_defense_2024}, %
 wherein the authors respond to a~recent criticism of Rothbardian welfare economics levelled by Wysocki and Dominiak 
%\label{ref:RNDIa7wjfDGjq}(2023).
\parencite*[][]{wysocki_how_2023}. %
 Fegley's and Israel's paper is excellently argued, straightforward and is bound to give Wysocki and Dominiak new headaches. In a~word, it was excellent to have professor Israel on the board.



Second, another excellent economist with a~philosophical bent we could think of was professor Alexander Linsbichler. He is definitely a~force to reckon with. He is insanely versatile. His versatility ranges from the acquaintance with the intellectual history of Vienna through profound conceptual insights within the Austrian school of economics to philosophy of science in general. Professor Linsbichler published in such first-rank journals as \textit{Synthese}, \textit{Journal for General Philosophy of Science} or \textit{Journal of Economic Methodology} 
%\label{ref:RNDiv5pAaUbFE}(Linsbichler, 2021; 2023; Linsbichler and Da Cunha, 2023).
\parencites[][]{linsbichler_austrian_2021}[][]{linsbichler_otto_2023}[][]{linsbichler_otto_2023}. %
 In the present \textit{Special Issue}, professor Linsbichler contributes the paper \textit{What Rothbard could have done but did not do: The merits of Austrian economics without extreme apriorism} 
%\label{ref:RNDWzZPuXlFvM}(2024).
\parencite*[][]{linsbichler_what_2024}. %
 The paper highlights professor Linsbichler at his best: erudite, analytically sharp and argumentatively original.



Additionally, it was no less than professor Łukasz Dominiak who agreed to contribute. Professor Dominiak is Wysocki's friend and mentor (literally a~supervisor of his Ph.D. thesis) at the same time. Professor's Dominiak interests are far-reaching and they include Austrian economics, political philosophy (especially libertarianism) as well as legal and moral philosophy 
%\label{ref:RND7tOtEjykCW}(Dominiak, 2017; 2019; Dominiak and Fegley, 2022).
\parencites[][]{dominiak_libertarianism_2017}[][]{dominiak_must_2019}[][]{dominiak_contract_2022}. %
 He almost single-handedly revolutionized the libertarian theory of justice in such areas as the theory of contract, compossibility of libertarian individual rights, the libertarian methods of property acquisition and what have you. In this \textit{Special Issue} he presents another of his splendid ideas, this time running against the libertarian received view on blackmail. Namely, professor Dominiak contributes the paper \textit{Free Market, Blackmail, and Austro-Libertarianism} 
%\label{ref:RND7iPO8Frudh}(2024),
\parencite*[][]{dominiak_free_2024}, %
 wherein he argues -- in his characteristically unyielding style -- that Austro-libertarians do have a~reason to revise their view on apparent permissibility of blackmail. His finding is all the more impressive, as his argument against libertarians is \textit{internal} in that it does not appeal to any external morality. Rather, professor Dominiak demonstrates that libertarians should favour banning \textit{some} blackmail exchanges, for they constitute frauds, something clearly prohibited in a~free society.



As a~matter of course, this \textit{Special Issue} contains other prominent scholars. Oliva Córdoba, a~renowned scholar, contributed an excellent essay on the philosophy and logic of human action. Interestingly, the author makes use of a~conceptual apparatus of philosophy of action to make sense of the notion of, for example, competition 
%\label{ref:RNDxw8PxTmb3C}(Oliva Córdoba, 2024).
\parencite[][]{oliva_cordoba_philosophy_2024}. %
 Mateusz Machaj, an undeniable Austrian superstar, elucidated the distinction between risk and uncertainty and proposed a~way of modelling uncertainty 
%\label{ref:RNDMDE10oIW23}(Machaj, 2024).
\parencite[][]{machaj_model_2024}. %
 Robert Mcgee, the most brilliant and versatile scholar, indulged us with his reading of Bastiat's view on taxation 
%\label{ref:RNDkZU8zhrHPU}(McGee, 2024).
\parencite[][]{mcgee_taxation_2024}. %
 Krzysztof Turowski analyzed Ludwig Lachmann as an alleged subjectivist institutionalist 
%\label{ref:RNDUAACXxN1gv}(Turowski, 2024).
\parencite[][]{turowski_ludwig_2024}. %
 Ceglarska and Cymbranowicz wrote a~paper on the role of phronesis in knowledge-based economy 
%\label{ref:RNDj7qowYgqeK}(Ceglarska and Cymbranowicz, 2024).
\parencite[][]{ceglarska_role_2024}. %
 Wysocki and Dominiak contributed a~short essay defending the Rothbardian welfare theory against the charges made by, most crucially, Bryan Caplan 
%\label{ref:RNDcAhJDNmSsg}(Wysocki and Dominiak, 2024).
\parencite[][]{wysocki_rejoinder_2024}. %
 Matúš Pošvanc, associated with F.A. Hayek Foundation wrote a~refined essay on the law of diminishing marginal utility 
%\label{ref:RNDs7FXZ4vkWh}(Posvanc, 2024).
\parencite[][]{posvanc_law_2024}. %
 Norbert Slenzok added to the \textit{Special Issue} by writing an essay \textit{Monarchy as Private Property Government. A~Chiefly Methodological Critique} 
%\label{ref:RNDV7H6rqtzrl}(Slenzok, 2024).
\parencite[][]{slenzok_monarchy_2024}. %
 Dawid Megger illuminatingly tackled the problem of demonstrated preference 
%\label{ref:RND2fG8nSblvN}(Megger, 2024).
\parencite[][]{megger_demonstrated_2024}. %
 Paweł Nowakowski critically scrutinized the Rothbardian view on the value of life from a~praxeological perspective 
%\label{ref:RND08YPX5nlKu}(Nowakowski, 2024).
\parencite[][]{nowakowski_praxeology_2024}. %
 Wysocki wrote a~brief rejoinder to Block's rejoinder to the former's original paper on indifference published in \textit{ZFN} 
%\label{ref:RNDYcJ1CcSqnE}(Wysocki, 2024).
\parencite[][]{wysocki_rejoinder_2024}. %
 The said rejoinder was in turn replied by Walter Block, also published in the present issue 
%\label{ref:RNDu4fpSmJ9Cq}(Block, 2024).
\parencite[][]{block_response_2024}. %
 Finally, Mateusz Czyżniewski, a~rising Austrian scholar, contributed a~review of Dawid Megger's book 
%\label{ref:RNDhHI0meumjt}(Czyżniewski, 2024).
\parencite[][]{czyzniewski_are_2024}.%




Eventually, a~word is due on the relevance of the present \textit{Special Issue} to \textit{ZFN}'s programmatic dedication to philosophy of science and its advocacy of interdisciplinarity. As already mentioned, the very inspiration for the whole \textit{Special Issue} came from the debate on the nature of choice vis-à-vis indifference within Austrian economics. But then again, when we were offered a~\textit{Special Issue}, we immediately thought of going beyond Austrianism itself. So, philosophical foundations of economics as such appeared to us to be a~rather apt unifying theme. But even this rather large category would not do justice to a~variety of papers included in this issue. For, what we have here is also epistemology proper (e.g. considerations on the Austrian alleged extreme apriorism), philosophy of action (e.g. reducing the phenomenon of competition or rivalry to certain intentional states of individual agents), some exegetical work (e.g. interpreting the thought of Lachmann, Rothbard and Aristotle himself) or political and legal philosophy combined (e.g. tacking the paradox of blackmail). Even this sample, we believe, satisfies \textit{ZFN}'s unyielding commitment to interdisciplinarity. Needless to say, the present Special Issue, while being dedicated to philosophy of economics, rather effortlessly satisfies \textit{ZFN}'s dedication to philosophy of science, for what is economics if not a~special science.



All in all, editing this issue was quite a~ride, with its joys (networking with exquisite anonymous reviewers, thinking about special invitees etc.) and sorrows (sometimes finding an appropriate reviewer is quite a~daunting task). Still, getting the above scholars to submit their respective work more than compensated for the effort made. We also hope that the prospective readers are going to find the essays included as interesting as we do. And finally, we do believe that this \textit{Special Issue} will give some additional impetus to a~burgeoning field of philosophy of economics.









\end{editorialeng2auth}




\sekcja{Articles}{Artykuły}

%\renewcommand{\theequation}{\arabic{section}.\arabic{equation}}
%\input{ART_Majid/Majid.tex} 
%\renewcommand{\theequation}{\arabic{equation}}


\begin{artengenv2auth}{Tate Fegley, Karl-Firedrich Israel}
	{A defense of Austrian welfare economics}
		{A defense of Austrian welfare economics}
		{A defense of Austrian welfare economics}
	{University of Sydney}
	{Murray N. Rothbard's \textit{Toward a~Reconstruction of Utility and Welfare Economics} is the defining contribution outlining the Austrian school's approach to welfare theory. A~recent attack on this approach is by Wysocki and Dominiak 
	%\label{ref:RNDFpq36Uzdfc}(2023),
	\parencite*[][]{wysocki_how_2023}, %
	 who argue, contra Rothbard, that whether an exchange is welfare-enhancing is not necessarily related to whether that exchange is just, and therefore the Rothbardian framework is wrong. This paper shows that their argument misconceives how Austrians treat the concept of welfare. They also misunderstand the crucial role of the principle of demonstrated preference. Properly conceived, Rothbard's reconstruction remains intact.
		}
		{welfare economics, Austrian economics, Murray N. Rothbard.}
	{%
		{\flushright\subbold{Tate Fegley}\\\subsubsectit\small{Montreat College}\label{israel-first}\par}%
		{\flushright\subbold{Karl-Firedrich Israel}\\\subsubsectit\small{Catholic University of the West}\par}%
	}




\section{Introduction}

\lettrine[loversize=0.13,lines=2,lraise=-0.03,nindent=0em,findent=0.2pt]%
{A}{seminal} %A~seminal
contribution in Austrian welfare economics is Rothbard's 1956 essay \textit{Toward a~Reconstruction of Utility and Welfare Economics} 
%\label{ref:RNDClORnqgdmS}(Rothbard, 2011).
\parencite[][]{rothbard_toward_2011}. %
 As the title suggests, Rothbard aimed to reconstruct welfare economics on solid scientific grounding, avoiding the pitfalls of previous attempts. Although it was a~scientific breakthrough, his argument was not without controversy, but has produced decades of criticism and replies 
%\label{ref:RND9hy2ArGPps}(Block, 1999; Caplan, 1999; Cordato, 1992; Gordon, 1993; Herbener, 1997; 2008; Hülsmann, 1999; Kvasnička, 2008; Prychitko, 1993).
\parencites[][]{block_austrian_1999}[][]{caplan_austrian_1999}[][]{cordato_welfare_1992}[][]{gordon_toward_1993}[][]{herbener_pareto_1997}[][]{herbener_defense_2008}[][]{hulsmann_economic_1999}[][]{kvasnicka_rothbards_2008}[][]{prychitko_formalism_1993}.%




A~recent argument against Rothbard's reconstruction is provided by \href{https://www.zotero.org/google-docs/?CcG5f3}{Wysocki and}\href{https://www.zotero.org/google-docs/?CcG5f3}{ Dominiak }\label{ref:RNDvZAcMn6xM2}\href{https://www.zotero.org/google-docs/?CcG5f3}{(2023)}. The authors argue that the welfare theorems he derives---that free market exchanges always increase social utility and that government intervention can never increase social utility---are false. Our goal in this paper is to defend Rothbard from Wysocki and Dominiak 
%\label{ref:RND0tl7w99Nev}(2023)
\parencite*[][]{wysocki_how_2023} %
 by demonstrating that their criticisms are misplaced and that Rothbard's contribution stands unscathed.



As many critics before them, Wysocki and Dominiak implicitly rely on the assumption of welfare or utility being a~magnitude that can be assessed independently from demonstrated preferences under specific circumstances. This, however, is not the case. We can of course construct all kinds of imaginary scenarios, where all the relevant knowledge about the underlying preferences is assumed into existence, but this does not help us in applications to the real world, where that kind of knowledge remains hidden from us, unless it is demonstrated in voluntary and just interaction.



The paper is structured as follows: the next section summarizes Rothbard's reconstruction of welfare economics. The third section summarizes and replies to the criticism of Wysocki and Dominiak 
%\label{ref:RNDKncprYNjab}(2023),
\parencite*[][]{wysocki_how_2023}, %
 and the fourth section provides a~conclusion and some further reflections on the importance of welfare economics and its relation to moral philosophy.



\section{Rothbard's reconstruction of welfare economics}

Rothbard's reconstruction of welfare economics is firmly based on the theory of subjective value espoused by the Austrian School. As Rothbard explains, welfare theory is utility theory applied to the context of society with the goal of drawing scientific conclusions about the desirability of alternative arrangements:



\begin{quote}
Utility theory analyzes the laws of the values and choices of an individual; welfare theory discusses the relationship between the values of many individuals, and the consequent possibilities of a~scientific conclusion on the ``social'' desirability of various alternatives. 
%\label{ref:RND6HNsiOxKXI}(Rothbard, 2011, p.289)
\parencite[][p.289]{rothbard_toward_2011}%
\end{quote}




To achieve this goal Rothbard invokes two principles: 1) the unanimity rule, and 2) the principle of demonstrated preference. The unanimity rule is better known as the Pareto criterion, which states that social welfare has improved if at least one person is made better off, and nobody is made worse off. Rothbard argues that this rule provides the only way in which we can scientifically speak of an improvement in social welfare. Since value and utility are subjective and we lack an objective unit of measurement, there is no way of comparing the loss in utility for some person with the gain in utility of another person. There is \textit{a~fortiori} no way of determining whether a~loss in utility for some person is outweighed by the gain in utility for another person. But subjectivity is by no means the only problem here. Even if there was an objective unit of measurement, it would still be questionable whether a~benefit for some person can ever outweigh the harm of another.\footnote{Utility can be understood as multi-dimensional, especially if we think of the utility of a~group of people. Social utility, in particular, is not one-dimensional, that is, harm and benefit are not necessarily received by the same people and do thus not occur along the same dimension. They cannot necessarily be lumped together even if they could be quantitatively compared. }



The crucial question is how can we know if somebody gains or loses utility? Here the principle of demonstrated preference comes in. Rothbard argues that we can only know about what a~person prefers, that is, what makes that person better off, from observing their choices and actions. If a~person chooses option A~over an alternative option B~that is also available, we can infer that the person attaches a~higher subjective value to option A~than to option B~and is made better off by choosing option A~(in the \textit{ex ante} sense). The person has demonstrated their preference in action.



We can hypothetically imagine all kinds of preferences of one or more persons and reason through how they would interact in various situations and what outcome would be socially optimal. But the crucial point that Rothbard makes is that we can only know about actual preferences to the extent that they are demonstrated in real action at a~specific point in time under specific circumstances. As Rothbard 
%\label{ref:RND4PCtrODkte}(2011, p.320)
\parencite*[][p.320]{rothbard_toward_2011} %
 describes it:



\begin{quote}
Demonstrated preference […] eliminates hypothetical imaginings about individual value scales. Welfare economics has until now always considered values as hypothetical valuations of hypothetical ``social states.'' But demonstrated preference only treats values as revealed through chosen action.
\end{quote}



Importantly, Rothbard emphasizes that there is no reason to believe that preferences are constant over time. For all we know they can and do change. Preferences as revealed at one point in time by an individual are not necessarily relevant for another point in time.



The assumption of constant preferences is indeed an important feature of Paul Samuelson's theory of revealed preference 
%\label{ref:RNDrzqg9QWabl}(Samuelson, 1938).
\parencite[][]{samuelson_empirical_1938}. %
 Rothbard explicitly distinguishes his own theory from Samuelson's by choosing the term ``demonstrated preference'', admitting that ``revealed preference'' would have been a~very fitting term too. According to Rothbard's principle of demonstrated preference, our limited knowledge of preferences as demonstrated under the specific circumstances of a~given historic situation cannot be extrapolated to other situations. There is no scientific basis for assuming preferences to remain what they have been before. We can know about them only for that specific situation in which they are demonstrated in action, and even then our knowledge about them is never complete.



To make sense of a~given historic situation, interpretive understanding is required and the observer can of course err. If Murray, for example, offers Paul the choice between an apple and a~pear, and Paul picks the pear, we know that Paul did what he preferred to do. But we do not know whether he expected to like the taste of the pear more than the taste of the apple. Maybe Paul just wanted Murray to falsely believe that he likes pears more than apples, although he really prefers apples in general. We only know for certain that Paul attached a~higher expected marginal utility to the option he chose than to the alternatives forgone.



It is important to understand that the principle of demonstrated preference does not allow the economist to make any inference on whether the level of utility of a~person---from a~point in time before the action takes place to a~point in time thereafter---has increased or not. Take the above example. Maybe Paul's utility increased from taking the pear compared to what it was before Murray made his offer. But maybe Murray made his offer to Paul in a~way that made him feel uncomfortable. The penetrating look in his eyes and the sarcastic smile made him tremble with fear, so that Paul really had a~higher level of utility before Murray showed up and made the offer. All of that is possible. So economists can infer nothing about the absolute changes in the level of utility between different points in time---neither for one person nor for a~group of people or society as a~whole. We are not the first ones to make this clarification in response to a~criticism of Rothbard's reconstruction. The same point is explained very well by Herbener 
%\label{ref:RNDM4wCd4Jzed}(Herbener, 2008, p.63)
\parencite[][p.63]{herbener_defense_2008} %
 in his reply to \href{https://www.zotero.org/google-docs/?MN1lOj}{Kvasnička }\label{ref:RNDkFYd3iLdgE}\href{https://www.zotero.org/google-docs/?MN1lOj}{(2008)}. It is worth quoting him at some length:



\begin{quote}
Deducing the effects on social utility from voluntary and involuntary exchanges requires considering each action in turn given the conditions as they are at that point. Nothing can be deduced about the level of utility a~person has at the beginning of a~series of actions compared to the level of utility he has at the end of the series of actions. For example, a~person having dinner with his friends orders steak from the menu. The economist observing him can objectively conclude that, given his options, he selected what he preferred. He is enjoying the conversation when it turns to a~subject he dislikes, but he stays and endures it. The economist observing him, lacking access to what he is experiencing in his mind, can objectively conclude that he prefers to continue dining with his friends. At some point, one of his companions makes a~remark so objectionable to him that he says, ``Anymore such talk and I~shall leave.'' The economist observing him can objectively conclude that he preferred to make this remark. The economist cannot objectively conclude that this line of conversation has lowered the level of his utility. To conclude that would require the economist to make a~judgment about his utility. The economist would have to interpret the meaning of the remark as it relates to his utility. The economist would have to decide whether it was a~serious remark or a~joke and if it was serious did making the remark push his utility up or down. Bullies, after all, like to intimidate others with such remarks. No such judgments are necessary for the economist to conclude that he preferred making the remark. It follows from the objective evidence of his action and the conceptual meaning of action. And so it goes for the rest of the evening with the level of his utility sometimes rising and sometimes falling, but he continues dining with his friends and leaves only after the party breaks up. Is he enjoying a~higher level of utility after the evening is over compared to before it began? Who can objectively say but the person himself? He is the only person with experiential knowledge of his own utility. What another person can objectively deduce is that he preferred what he did each step of the way. 
%\label{ref:RND3vYHZXOYPE}(Herbener, 2008, p.63)
\parencite[][p.63]{herbener_defense_2008}%
\end{quote}
Hence, to say that somebody is made better off as the result of a~voluntary choice involves a~counterfactual comparison between the option chosen (the factual) and the alternative foregone (the counterfactual) at the very same point in time.\footnote{On the counterfactual nature of economic theory in general, see 
%\label{ref:RNDpxxCZtI0tY}(Hülsmann, 2003).
\parencite[][]{hulsmann_facts_2003}. %
 For an interesting critique of Hülsmann, see Machaj 
%\label{ref:RNDKWuHXQB4dq}(2012).
\parencite*[][]{machaj_counterfactuals_2012}. %
 } It does not involve a~comparison between the absolute level of utility before and after the choice. We only know that the expected marginal utility of the option chosen is higher than that of the alternative options not chosen. The actor gains utility relative to the alternative options forgone.



Another important contribution of Rothbard's reconstruction of welfare economics is the clarification of the notion of marginal utility. He explains that the term does not refer to some marginal increment in utility, but rather to the utility of the marginal unit of some good, which is subjective and ordinal. Otherwise, the notion of marginal utility would indeed suggest that utility is something that can be measured and computed mathematically, and that marginal utilities can be added to and subtracted from one another, and that total utility is nothing other than a~sum of marginal utilities. But that is not so. Rothbard 
%\label{ref:RNDcOzUTPrIOp}(2011, p.301)
\parencite*[][p.301]{rothbard_toward_2011} %
 argues that ``there is no such thing as total utility; all utilities are marginal''. And most importantly we can only draw scientific conclusions about welfare and utility on the margin based on demonstrated preferences. People are of course passively affected by all kinds of changes in the environment, including the actions of others. These changes cannot, however, be dealt with scientifically in the realm of welfare economics, because we lack the means of assessing their welfare implications.



All of this imposes radical constraints on what welfare economics can accomplish. But Rothbard argues that despite the fundamental subjectivity of utility, we can at least draw some scientific conclusions. We cannot calculate total utility, but following the unanimity rule, we can in some situations, conclude that overall or social utility has improved, that is, when demonstrated preferences are satisfied. For example, ``welfare economics can make the statement that the free market increases social utility, while still keeping to the framework of the Unanimity Rule'' 
%\label{ref:RNDqptw5GYZaS}(Rothbard, 2011, p.320).
\parencite[][p.320]{rothbard_toward_2011}. %
 The important word here is ``increase'' instead of ``maximize''.\footnote{Rothbard 
%\label{ref:RNDmKyN7PowXv}(2011, p.323)
\parencite*[][p.323]{rothbard_conceived_2011} %
 uses the word ``maximize'' in quotation marks and he makes the following clarification: ``[…] we may conclude that the maintenance of a~free and voluntary market ``maximizes'' social utility (provided we do not interpret ``maximize'' in a~cardinal sense.).'' That is, since the free market is the absence of government intervention, it implies that no voluntary and mutually beneficial exchanges are prevented, thus social utility is ``maximized''.} There is nothing to be maximized, but there are mutually beneficial exchange opportunities which are discovered and exploited within the framework of the free market, leading to improvements in social utility as individuals voluntarily interact without rights violations.



When it comes to government intervention or any rights-violating action by individuals, we can draw no such conclusion. As Rothbard 
%\label{ref:RNDMvZw83RBoQ}(2011, p.322)
\parencite*[][p.322]{rothbard_toward_2011} %
 explains:



\begin{quote}
Suppose that the government prohibits A~and B~from making an exchange they are willing to make. It is clear that the utilities of both A~and B~have been lowered, for they are prevented by threat of violence from making an exchange that they otherwise would have made. On the other hand, there has been a~gain in utility (or at least an anticipated gain) for the government officials imposing this restriction, otherwise they would not have done so. As economists, we can therefore say nothing about social utility in this case, since some individuals have demonstrably gained and some demonstrably lost in utility from the governmental action.
\end{quote}



An analogous explanation can be given in cases where governments do not prevent but enforce a~transaction. In such cases, too, there is a~violation of the unanimity rule and no conclusion can be drawn about whether social utility has improved or not.\footnote{See, in this special issue, Wysocki and Dominiak 
%\label{ref:RNDoZwDIp2f29}(2024)
\parencite*[][]{wysocki_social_2024} %
 on clarifying the dispute over what precisely Rothbard meant by saying ``we can therefore say nothing about social utility in this case…'' In this regard, Rothbard was making a~statement about the epistemological limitations of scientific economics, though elsewhere he allowed for the possibility of knowledge under other disciplines. Regarding the possibility of third parties to a~voluntary exchange being envious, he writes, ``[W]e may know as historians, from interpretive understanding of the hearts and minds of envious neighbors, that they do lose in utility. But we are trying to determine in this paper precisely what scientific economists can say about social utility or can advocate for public policy, and since they must confine themselves to demonstrated preference, they must affirm that social utility has increased'' 
%\label{ref:RNDHfGWvPK7Dv}(Rothbard, 1997, p.89).
\parencite[][p.89]{rothbard_praxeology_1997}.%
} There is no scientific basis for supporting such a~claim if one sticks firmly to the unanimity rule and the subjectivity of utility and value.



Rothbard 
%\label{ref:RNDYs15ehFx8Q}(2024, p.323)
\parencite*[][p.323]{rothbard_toward_2011} %
 then draws two main conclusions that have aroused much criticism among his readers:



\begin{quote}
Economics, therefore, without engaging in any ethical judgment whatever, and following the scientific principles of the Unanimity Rule and Demonstrated Preference, concludes: (1) that the free market always increases social utility; and (2) that no act of government can ever increase social utility. These two propositions are the pillars of the reconstructed welfare economics.
\end{quote}
Some aspects underpinning these claims are not spelled out in detail in Rothbard's reconstruction. But these elements can be provided from the rest of Rothbard's works and the works of his intellectual followers to make his two conclusions whole and defend his analysis from many criticisms.



\section{A~defense against recent critics}

In their recent criticism of Rothbard's reconstruction, Wysocki and Dominiak 
%\label{ref:RNDvqtsRs3Qv8}(2023)
\parencite*[][]{wysocki_how_2023} %
 claim to demonstrate that his two pillars---that the free market always increases social utility and that no act of government can ever increase social utility---are false, and that whether a~particular exchange is welfare-enhancing or welfare-diminishing is a~separate question from whether the exchange is just or unjust.



To show this, Wysocki and Dominiak 
%\label{ref:RNDlGtAmWKEZA}(2023)
\parencite*[][]{wysocki_how_2023} %
 provide counter-examples of exchanges that are alleged exceptions to Rothbard's pillars---one being an example of a~just, that is, property rights respecting, exchange that is not welfare enhancing and the other an example of an unjust, that is, property rights violating, exchange that is welfare enhancing.



\subsection{Just but ``welfare-decreasing'' exchanges}



It is worth noting from the outset that ``welfare-increasing'' and ``welfare-decreasing'' are meant in the ex ante sense of the word. There are of course just exchanges that people regret. They are welfare-decreasing in the \textit{ex post} sense. Nobody would deny their existence. The point of contention is whether there are just and welfare-decreasing exchanges in the \textit{ex ante} sense. Wysocki and Dominiak 
%\label{ref:RNDOFxRjx42jj}(2023)
\parencite*[][]{wysocki_how_2023} %
 think there are.



The supposed exception to the idea that free and voluntary exchange always leads to improved welfare from the \textit{ex ante} perspective of both trading partners is a~blackmail offer. Wysocki and Dominiak 
%\label{ref:RNDuHmJ0QqpIc}(2023, p.22)
\parencite*[][p.22]{wysocki_how_2023} %
 have the reader



\begin{quote}
[S]uppose that a~blackmailer makes the following proposal to the blackmailee:\\
(1) If you pay me \$1.000.000 (demand), I~will let your reputation remain untarnished (relative benefit).\\
(2) If you don't pay me (refusal), I~will gossip about your secrets (threat).
\end{quote}
They argue that the blackmailee, if he accepts the blackmailer's proposal and pays him, demonstrates his preference to have an untarnished reputation and paying \$1 million over the alternative of having a~tarnished reputation but keeping \$1 million, and therefore benefits relative to not paying. However, since he would be better off if the blackmailer had had nothing to do with him at all (since he would then have both his \$1 million and an untarnished reputation), he is not better off in an absolute sense.



But this conception of being better or worse off in an absolute sense is irrelevant to Rothbard's welfare theory as we outlined above, quoting from Herbener's 
%\label{ref:RNDGVme3JlOr4}(2008, p.63)
\parencite*[][p.63]{herbener_defense_2008} %
 excellent exposition. Welfare economics can say nothing about the absolute level of utility. Wysocki and Dominiak 
%\label{ref:RNDlGCmB8Jh2v}(2023, pp.61–62, fn. 12)
\parencite*[][fn. 12]{wysocki_how_2023} %
 appear to fully appreciate this point in a~rather extensive footnote of their article. Given this, it is strange that they pursue this line of argument based on a~different conception of welfare, as if it could provide exceptions to Rothbard's propositions. Rather, the question that is relevant to Rothbard is whether property rights are respected and a~voluntary exchange is made: if so, social welfare increased. Imaginary counterfactuals involving the non-existence or existence of other individuals are irrelevant. Imaginary counterfactuals are very different from the relevant counterfactuals of alternative choices in a~given situation. Only the latter matter. The former do not.



Imagine a~person who voluntarily buys an apple for \$1, but the person would have much rather bought a~banana for \$1. There was no one willing to sell a~banana for \$1. Is it in any way relevant that the apple buyer is made better off, because she prefers an apple over \$1, but would have been still better off if she could have bought a~banana instead? No, given the constraints of the situation in terms of money, time, knowledge, and the rights-respecting actions of others, social welfare has increased because of the exchange made. This is true for the blackmail transaction as for any other free-market transaction.



There is another perspective on the blackmail case. When we consider all of the parties involved in the blackmail transaction, we can more easily see that social welfare increases from the voluntary exchange. That is, unaddressed by Wysocki and Dominiak are the potential beneficiaries of the gossip.\footnote{With blackmail, there is necessarily a~third party. If Friday learns embarrassing information about Robinson Crusoe but they are alone on an island, Friday will not be able to blackmail Crusoe.} What is being traded by the blackmailer is a~property right to decide whether embarrassing information is published or kept secret. The end of the blackmailee to have his reputation untarnished conflicts with the ends of buyers of gossip magazines to read about his secrets. If the blackmailer allows both the blackmailee and publishers of gossip magazines to bid over this property right (that is, the free market is allowed to operate), resources will be allocated to their most highly valued uses and all Pareto-improving transactions that people perceive will be made. In this example, government intervention cannot be demonstrated to lead to a~more preferable allocation of property rights.



We have seen that the distinction between absolute and relative improvements in welfare for one of the two exchange partners is irrelevant. What is relevant from the vantage point of Austrian welfare economics is whether the benchmark for comparison involves a~rights violation or not. The blackmailer threatens to gossip about the blackmailee's secrets, but gossiping is not a~rights violation. He has the right to gossip, although some people might not like it. So the blackmailee who pays and prevents his secrets from being published is made better off relative to a~scenario that involves no rights violation and in which his secrets are made public. Contrast this with a~highwayman who threatens to kill his victim unless she pays money. In that scenario, as Wysocki and Dominiak 
%\label{ref:RNDb1tp3NOZKs}(2023, pp.54–55)
\parencite*[][pp.54–55]{wysocki_how_2023} %
 emphasize, the victim who pays and lives is made better off relative to the alternative of being killed. But that alternative involves a~rights violation and is unjust. The victim is forced into an unjust exchange to protect herself against a~violation of her rights. She has to pay for something that is already rightfully hers---her life. In other words, she has to pay and receives nothing in exchange that is not already hers. And in this sense she is made worse off.



There is indeed a~philosophical discussion to be had as to what constitutes mere gossip and what crosses the demarcation line to libel and should be considered a~rights violation. More generally, a~theory of justice, or in Rothbard's view, a~theory of property rights 
%\label{ref:RNDcpsfNizy6i}(Rothbard, 1998),
\parencite[][]{rothbard_ethics_1998}, %
 is the very foundation that sets the rules according to which people are allowed to demonstrate their preferences and according to which people's choices and actions are allowed to change the environment in which others act. Choices and actions of people do change the conditions under which we act all the time, but as long as their choices and actions do not violate our rights, they are, like the weather, elements of the uncertain environment in which we act according to our own preferences. They sometimes increase and sometimes decrease our level of utility, but we cannot deal with these changes scientifically.



The theory of justice and property rights is independent of welfare economics in the sense that it is its logical prerequisite. It sets the stage for us to engage in welfare economics scientifically. When Rothbard wrote in 1956 that he can draw his welfare economic conclusions without any ethical judgment, he really took the ethics underpinning a~system of free-market exchange for granted. Rothbard realized that, which is why he later worked towards a~broader social philosophy integrating economics and ethics, sometimes referred to as Austro-libertarianism 
%\label{ref:RNDyCjck3pAaa}(Hoppe, 1999).
\parencite[][]{holcombe_murray_1999}.%




\subsection{A~voluntary and welfare-enhancing rights violation}



The second claim of Wysocki and Dominiak 
%\label{ref:RNDZBbOIl2xsP}(2023)
\parencite*[][]{wysocki_how_2023} %
 is that there are rights violations that are welfare-enhancing. Again, this is meant to be the case in the \textit{ex ante} sense. We can all think of scenarios in which a~\textit{prima facie} rights violation turns out to be a~good thing from the perspective of the person whose rights were violated. Take a~drug addict who is forced to have a~cold turkey by a~close relative who locks him in a~room for the time he needs to detox. The addict might later on be grateful for it, although the close relative had no right to lock him up. Wysocki and Dominiak have something else in mind.



To show that unjust exchanges are not necessarily welfare diminishing, Wysocki and Dominiak 
%\label{ref:RNDZi44rrcc2b}(2023)
\parencite*[][]{wysocki_how_2023} %
 offer the example of an individual with a~broken refrigerator in his backyard that he would like to be rid of, but the costs of selling it or hauling it off to the junkyard are deemed too high. However, one day a~thief absconds with the fridge and the owner decides not to interfere, given that his unwanted fridge is being removed for free.



Wysocki and Dominiak argue that this ``exchange'' is unjust because the owner of the fridge never relinquished his ownership rights and he never consented to the fridge being taken, either explicitly or tacitly. They also argue that the owner demonstrated his preference for the fridge being stolen over it remaining in his yard because of his choice not to interfere with the thief. As such, they conclude that he benefited from the theft. Further, the fridge owner benefitted not only in relative terms, but also in absolute terms because if there were no thief, he would still be stuck with the fridge in his backyard.



Does this example show that Rothbard's second pillar---that government intervention can never increase social welfare---is false? No. The primary issue with their argument, from the vantage point of the principle of demonstrated preference, is the limited inference we can make about the fridge owner's preferences based on his action. We can rightfully infer that the owner preferred not to interfere, but we cannot from his act of non-interference infer that he preferred the fridge to be stolen rather than remain in his yard. We could also suspect that he feared that the thief may attack him if he had tried to stop him, or that he would rather enjoy his leisure than have to get up and stop the thief (he was, after all, presumably too lazy to do so little as put a~sign that reads ``FREE'' on the fridge). Therefore, Wysocki and Dominiak do not successfully side-step the ``fallacy of psychologizing'' as they claim since a~real-world equivalent to their thought experiment would require that we are able to analyze the internal thoughts of the fridge owner in order to be able to determine the reason for non-interference, without which we cannot say that he prefers his fridge taken away over remaining in his yard. The fact that we can simply assume all of that in a~thought experiment is completely irrelevant. We emphasize again, as Rothbard 
%\label{ref:RNDYwcNyo7ffn}(2011, p.320)
\parencite*[][p.320]{rothbard_toward_2011} %
 put it, that the principle of demonstrated preference ``eliminates hypothetical imaginings about individual value scales.''



Wysocki and Dominiak 
%\label{ref:RNDEgZGC1fA0G}(2023, pp.63–64)
\parencite*[][pp.63–64]{wysocki_how_2023} %
 further criticize Rothbard's position for assuming that only rights-respecting exchanges can be voluntary. They challenge Rothard's rights-based understanding of voluntariness. They argue that the thief of the fridge is violating the property rights of the fridge owner, but that the fridge owner is agreeing to that rights violation voluntarily. For them, the scenario gives an example of a~voluntary rights-violating exchange and hence of a~welfare-enhancing rights violation. But this is an unsubstantial play with words.\footnote{These quibblings are equally sterile as the debates on the concept of voluntary slavery 
%\label{ref:RNDmVh6cmNpPk}(Block, 2003; Casey, 2011; Dominiak, 2017).
\parencites[][]{block_toward_2003}[][]{casey_can_2011}[][]{dominiak_problem_2017}. %
 Of course we can define our terms in such a~way that ``voluntary slavery'' can exist, but we can do the same for ``married bachelors'' or ``huge midgets.'' It does not help. For more on the concept of voluntariness and rights under Austro-libertarianism according to Dominiak and Wysocki see 
%\label{ref:RNDwGgyIbRXsO}(Dominiak, 2018; 2022; 2023; Dominiak and Fegley, 2022; Megger and Wysocki, 2023; Wysocki, 2020; 2021; Wysocki, Block and Dominiak, 2019; Wysocki and Megger, 2019; 2020).
\parencites[][]{dominiak_libertarianism_2018}[][]{dominiak_contract_2022}[][]{dominiak_proceeds_2023}[][]{dominiak_contract_2022}[][]{megger_coercion_2023}[][]{wysocki_problems_2020}[][]{wysocki_austro-libertarian_2021}[][]{wysocki_austrian_2019}[][]{wysocki_austrian_2019}[][]{wysocki_problems_2020}.%
} Nothing in the thought experiment suggests that the fridge owner's property rights are actually violated. Quite to the contrary, the fridge owner decides to execute his property rights in just the way that allows the thief to freely take the fridge. Economically speaking, the fridge in his backyard is not a~good but a~bad---not an asset, but a~liability. The thief renders a~free service to the fridge owner by removing it, albeit unknowingly.\footnote{For a~general theory of gratuitous goods, see Hülsmann 
%\label{ref:RNDDYslf4v9T4}(2023).
\parencite*[][]{hulsmann_wirtschaft_2023}.%
}



Let us give another example to show that this semantic play is unhelpful. If a~man advances to kiss a~woman, he does not know whether she likes it or not. He has no right to use the woman's lips for his pleasure. She can refuse or reciprocate. If she refuses, but the man forces her, it is an involuntary rights violation. If she instead reciprocates, it must constitute a~voluntary rights violation according to Wysocki and Dominiak 
%\label{ref:RNDk42sOhYRQj}(2023).
\parencite*[][]{wysocki_how_2023}. %
 But the kiss then is always a~rights violation. We can of course define terms in this way, but it does not facilitate or clarify the analysis. And where is the love, if every kiss is a~rights violation?



The difference between the kisser and the thief is that the thief (presumably) assumes that his action is unwelcome and the kisser (presumably) hopes that his advance is welcome. The action of the thief seems like a~rights violation from his own point of view. He does not intend to benefit the fridge owner and is willing to violate his rights, but that seems irrelevant. Sometimes we do not intend to violate anyone's rights, but do, and sometimes we do not violate anyone's rights, although we willingly take the risk of doing so. The intent is not what matters for the welfare economic analysis of the situation.



Interestingly, given that the thief rendered a~welcome service to the fridge owner, he could have charged a~price for it. If he were an honest chap and had asked the owner whether the fridge should be removed, he could have fetched a~better deal for himself. He could have been even better off than from just taking the fridge. From a~welfare economic perspective, it would have been better for the thief himself, if he had intended to respect the fridge owners property rights. He would have benefited absolutely, not just relatively, so to speak.



Wysocki and Dominiak additionally argue that Rothbard is incorrect when he argues that there are two distinct cases that can be made in favor of the free market: the moral and the economic. According to them, it really boils down to only one argument. For if it is the rights-respecting character of an exchange that guarantees mutual benefits and the free market increases welfare by virtue of it being the set of all rights-respecting exchanges that people engage in, then there are no separate moral and economic cases. But this misunderstands Rothbard's argument, for he writes in the passage that Wysocki and Dominiak themselves quote,



\begin{quote}
[i]t so happens that the free-market economy, and the specialization and division of labor it implies, is by far the most productive form of economy known to man, and has been responsible for industrialization and for the modern economy on which civilization has been built […] Even if a~society of despotism and systematic invasion of rights could be shown to be more productive than what Adam Smith called ``the system of natural liberty,'' the libertarian would support this system. 
%\label{ref:RNDcyh3b4SIqg}(Rothbard, 2006, p.48)
\parencite[][p.48]{rothbard_for_2006}%
\end{quote}

We see clearly that for Rothbard, the ``economic case'' for the free market is not synonymous with welfare ``maximization'' based on free exchange. Rather, it is about the production of wealth or material goods and services which widen the possibilities of mutually beneficial exchanges. Material wealth and welfare are distinct, and therefore there really are two separate cases being made, not just one. A~free-market economy does not only respect private property rights and is thus preferable on moral grounds, it also brings about a~greater material abundance and is thus preferable on economic grounds. The potential counterargument that some people might not like material abundance can be discarded, since every person is free to live a~life in poverty amidst an otherwise wealthy society.



\section{Conclusion and some further reflections}

Wysocki and Dominiak 
%\label{ref:RNDFzpr3sgmFD}(2023, pp.58–59)
\parencite*[][pp.58–59]{wysocki_how_2023} %
 anticipate a~counterargument to their fridge example that some readers might believe is similar to ours. They expect that critics might rely on some notion of tacit consent to claim that the thief did not actually violate the fridge owners rights. But this line of argument they say is not available to Austro-libertarians, because they ``repudiate the juridical significance of tacit or implicit consent'' 
%\label{ref:RNDeVR3M4WaAa}(Wysocki and Dominiak, 2023, p.58).
\parencite[][p.58]{wysocki_how_2023}. %
 While it is true that Austro-libertarians reject and sometimes even mock the idea of tacit consent to justify specific state interventions or the institution of the state as such 
%\label{ref:RNDY6LLoBCM6M}(Hoppe, 2006),
\parencite[][]{hoppe_economics_2006}, %
 it is not the case that one has to rely on tacit consent to recognize that the fridge owners rights were not violated. Wysocki and Dominiak give us a~thought experiment after all, and they make it perfectly clear that the owner welcomes the fridge being taken from his yard. In the thought experiment there is nothing implicit about the fridge owner's consent. Wysocki and Dominiak 
%\label{ref:RND6dCZfxt952}(2023, p.58)
\parencite*[][p.58]{wysocki_how_2023} %
 explicitly tell us that ``[o]ne day [the owner] sees, to his delight, a~thief absconding with the fridge. Having realized his fridge is thus being removed for free, he decides not to interfere.''



In a~real-world scenario we could never know. This is why rights violations should not be allowed, neither from a~moral nor welfare-economic point of view. There is no way of demonstrating a~preference for one's own rights to be violated. If you agree to getting smacked in the face, and you get smacked in the face, your rights are not violated. If on the other hand you get smacked in the face without consent, it is still possible that you enjoyed it. You just got lucky. The important point is that if you happen to enjoy such things, the free market allows you to demonstrate your preference for it, for example, by joining a~fight club or a~group of hooligans.



From the example given by Wysocki and Dominiak 
%\label{ref:RNDlgFErvIpA7}(2023)
\parencite*[][]{wysocki_how_2023} %
 it is not clear how government inflicted rights violations could be shown to increase social welfare. One could give an endless number of similar examples:
\begin{itemize}
\item A~student assistant sneaking into the professor's office to correct all of the 250 macroeconomics exams of last semester
\item A~girlfriend taking money out of her boyfriend's wallet to buy groceries to cook his favorite dish
\item A~stranger going into an apartment to clean it up, leaving all of the owner's belongings in their rightful place
\item …
\end{itemize}
In all of these scenarios we can imagine the person whose ``rights were violated'' being perfectly fine with it. A~system of free and voluntary interaction, in which property rights are respected, would allow the persons involved to express these preferences explicitly. The boyfriend could tell his girlfriend that he would appreciate it. The professor could hire the student assistant under the condition that he corrects the exams. And of course anyone could look for free cleaning services. None of these examples is sufficient to disprove Rothbard's second pillar of welfare economics---that ``no act of government can ever increase social utility'' 
%\label{ref:RNDpi9y7PXvjj}(Rothbard, 2011, p.323).
\parencite[][p.323]{rothbard_toward_2011}.%




Rothbard's formulation would have been more on point if he had used the word ``state'' instead of ``government.'' We can imagine forms of government that do not involve rights violations, that is, governments to which everyone affected consents, but that is decidedly not the case for the modern state. By virtue of it being financed through coercive taxation it violates by its very nature the unanimity rule. Its actions therefore cannot increase social utility if one accepts that rule.



Now, one could imagine a~fictitious world in which every single citizen pays ``taxes'' voluntarily, believing that what their respective ``state'' does is necessary and welfare-enhancing. This would be a~world of implicit consent. Rothbard would probably have loved to live in such a~world. But in the real world, institutions would have to radically change for us to know whether we are in such an admirable state. Institutions would have to change in such a~way that implicit consent can be made explicit. This would mean among many other things the end of coercive taxation.


\end{artengenv2auth}

\label{israel-last}
\begin{artengenv}{Alexander Linsbichler}
	{What Rothbard could have done but did not do: The merits of Austrian economics without extreme apriorism}
	{What Rothbard could have done but did not do\ldots}
	{What Rothbard could have done but did not do: The merits of Austrian economics without extreme apriorism}
	{Johannes Kepler University Linz}
	{Austrian economics emphasizes a~priori components of social scientific theory. Most emphatically, Ludwig Mises and Murray Rothbard champion praxeology, a~methodology often criticized as extremely aprioristic. Among the numerous justifications and interpretations of praxeology to be found in the primary and secondary literature, conventionalism avoids the charge of extreme apriorism by construing the fundamental axiom of praxeology as analytic instead of synthetic. This paper (1) explicates the tentative structure of the fundamental axiom, (2) clarifies some aspects of a~conventionalist defense of praxeology, and (3) appraises conventionalist praxeology according to Rothbardian criteria. While Rothbard provides an essentialist justification of praxeology and embraces extreme apriorism, a~mildly aprioristic conventionalist defense of praxeology fares better on Rothbard's own criteria and is much more compatible to other contemporary methodological positions and economic theories.
	}
	{Austrian economics, praxeology, conventionalism, apriorism, analyticity, Ludwig Mises, Murray N. Rothbard.}







\section{Apriorism and Praxeology}

\lettrine[loversize=0.13,lines=2,lraise=-0.03,nindent=0em,findent=0.2pt]%
{P}{}roponents of Austrian Economics have emphasized a~priori aspects of economic theorizing ever since the publication the Austrian School's ``founding document'', Carl Menger's \textit{Principles of Economics} 
%\label{ref:RNDnSZZ3nuwtm}(1871).
\parencite*[][]{menger_grundsatze_1871}. %
 Yet, the extent to which members of the Austrian School theoretically endorsed and practically applied apriorism varies considerably between different scholars and perhaps even between different writings of a~single author. One of the most famous Austrian economists, F. A. Hayek, radically changed his stance towards apriorism once or twice---at least according to some but not all of his discordant interpreters.\footnote{See Caldwell 
%\label{ref:RNDCrv4IytRnq}(2009)
\parencite*[][]{caldwell_skirmish_2009} %
 and Scheall 
%\label{ref:RND69HaOrw7iT}(2015).
\parencite*[][]{scheall_hayek_2015}.%
} Sympathizers and critics alike identify the praxeological branch of Austrian economics as the most extremely apriorist.



Praxeology as a~methodology for the social sciences was introduced by Ludwig Mises and most famously continued by Murray Rothbard. Although these two most prominent champions of praxeology justify their position with different arguments, the basic idea is the same. We will merely sketch it in two steps here, referring readers to more extensive expositions, reconstructions, and discussions in the literature.\footnote{See e.g. Linsbichler 
%\label{ref:RNDG22bqR6a2k}(2017; 2021a),
\parencites*[][]{linsbichler_was_2017}[][]{linsbichler_austrian_2021}, %
 Long 
%\label{ref:RNDfShiK5kH0W}(2008),
\parencite*[][]{long_wittgenstein_2008}, %
 Mises 
%\label{ref:RNDQXgDZPa7KF}(1940; 2003; 2007; 1962; 2012).
\parencites*[][]{mises_nationalokonomie_1940}[][]{mises_epistemological_2003}[][]{mises_theory_2007}[][]{mises_ultimate_1962}[][]{mises_ultimate_2012}.%
}



As Step One, the praxeologist has to prove that the fundamental axiom of praxeology, ``man acts'' 
%\label{ref:RNDuxK6gL3UCe}(see, e.g. Mises, 2012, p.4),
\parencite[see, e.g.][p.4]{mises_ultimate_2012}, %
 is an a~priori true starting point. Explications of the overly short ``man acts'' identify its content along the following lines: human individuals and only human individuals (as opposed to viruses, planets, or social classes) at least sometimes behave purposefully, i.e. they choose goals and apply means they subjectively consider expedient to attain these goals on the basis of their beliefs. Strictly speaking, the way Mises and other Austrian economists apply the fundamental axiom only suggests that human individuals act and none of the other known types of objects act. In case we encounter intelligent aliens, praxeologists might reconsider the ``and only human individuals'' clause.\footnote{Compare Rothbard's related take on aliens or animals having rights 
%\label{ref:RNDYRhTmAkccB}(Rothbard, 1998, pp.155–157)
\parencite[][pp.155–157]{rothbard_ethics_1998} %
 and children having rights 
%\label{ref:RNDiRjeHyq9o0}(Rothbard, 1998, pp.97–112).
\parencite[][pp.97–112]{rothbard_ethics_1998}.%
}



Note that our explication of the fundamental axiom presupposes an independent characterization of ``human beings'' in advance but does not state that human beings exist. When a~social scientist ascribes goals and beliefs to each human being (and only to human beings), she has finished her job instantaneously if there are no human beings in the universe of discourse. \textit{If (and only if) x~is a~human being, then x~acts.} This is vacuously true, if there are no human beings.\footnote{One reviewer invited us to consider that there might not have been men if a~meteorite wiped out not just dinosaurs but Mother Earth too. Can we nevertheless uphold the analytic truth of ``man acts''? We hope that our further explication of the fundamental axiom addresses this worry---without engaging in discussions on the relationship between necessity and analyticity.} Strictly speaking, the fundamental axiom is of no help in ascertaining whether a~certain human behavior is merely behavior or an action either. According to Mises, this assessment is neither praxeological nor natural scientific but a~thymological matter, i.e. obtained by the ``method'' of \textit{Verstehen} (understanding) and a~posteriori. We hope that the following tentative structure of the fundamental axiom---the first explication of its kind---will facilitate further clarifications of its content:



\begin{enumerate}[label=\alph*)]
\item \textit{For all x: If and only if x~is a~human individual, then certain theoretical entities t}\textit{\textsubscript{1x}}\textit{, t}\textit{\textsubscript{2x}}\textit{, t}\textit{\textsubscript{3x}}\textit{, … (goals, preferences, beliefs, interpretations, …) exist, such that $\varphi $(x, t}\textit{\textsubscript{1x}}\textit{, t}\textit{\textsubscript{2x}}\textit{, t}\textit{\textsubscript{3x}}\textit{, …).}
\item \textit{For all x, y: If y~is behavior of a~human individual x, then: (If and only if y~is an action then $\psi $ (x, y, t}\textit{\textsubscript{1x}}\textit{, t}\textit{\textsubscript{2x}}\textit{, t}\textit{\textsubscript{3x}}\textit{, …)).}
\end{enumerate}

In Step Two, together with auxiliary hypotheses and empirical claims including claims about the content of the actors' preferences and beliefs, economic theorizing proceeds in a~purely deductive manner. Hence, the praxeological ``economist need not displace himself; he can, in spite of all sneers, like the logician and the mathematician, accomplish his job in an armchair'' 
%\label{ref:RND3t5KuzxSEk}(Mises, 2012, p.78).
\parencite[][p.78]{mises_ultimate_2012}. %
 Since deduction preserves truth and aprioricity, all logical consequences of the fundamental axiom are a~priori true---provided Step One was successful.\footnote{Strictly speaking, given an a~priori true axiom f, a~posteriori auxiliary hypotheses h1 and h2, and the a~posteriori thymological statement t, the a~priori praxeological theorems could at best have the form ‘((h1 \& h2 \& t) -{\textgreater} x)'. Typically, the statement ‘x' will not be a~priori true. For a~similar analysis of mathematics and potential ensuing problems, see Carnap 
%\label{ref:RND2bBReCHa6a}(1955; 2000)
\parencites*[][]{carnap_foundations_1955}[][]{carnap_untersuchungen_2000} %
 and Jeffreys 
%\label{ref:RNDzLzRvRGf9m}(1938)
\parencite*[][]{jeffreys_nature_1938}%
(1938) respectively.} The wonderful result would be an a~priori true economic theory, immune to empirical criticisms.



Advancing the intricate methodological and epistemological discussions regarding praxeology and economic principles in general, this paper reviews two types of problems with praxeology raised in the literature (section 2), sketches how conventionalist praxeology aims to circumvent and solve these problems (section 3), clarifies misunderstood aspects of conventionalist praxeology (section 4), contrasts conventionalism with Rothbard's essentialist defense of praxeology (section 5), presents Rothbard's criteria for the acceptability of fundamental axioms (section 6), and appraises whether his own arguments (section 7) or conventionalist proposals (section 8) succeed in meeting the criteria, and finally indicates one of many open problems for future research (section 9).



Before concluding this introduction, a~clarification regarding ‘apriorism' will be expedient for the remainder. The concept of ‘apriorism' as discussed in the Austrian School literature and beyond comprises three distinct, yet not always sufficiently distinguished, constituents: First, and foremost for our purposes, apriorism is an epistemological notion referring to those elements of economic theory that are not subject to falsification, verification, confirmation, corroboration, or challenge on an empirical basis. It would be a~category mistake to employ experience as a~critical standard for an a~priori statement.\footnote{Some Austrian economists including Rothbard reject sensory experience as a~critical standard for economic theory but highlight the justificatory role of inner experience (intuition, introspection). We will return to this in sections 5-8.} Second, experience as well as interpretative understanding are enabled and directed by a~theory and interpretational standpoints. Thirdly, experience is not the source or origin of ideas for theories. This final component is contested even within the Austrian School, especially if ‘experience' is meant to encompass inner, intuitive experience. Yet, for the purposes of this paper, we are mainly concerned with justifications and criticisms of praxeology, not with its psychological origins.



\section{Two Types of Criticisms of Praxeology}

Arguably, almost every scientific research program contains implicit a~priori elements and perhaps Austrian economists merely tend to be explicit and reflective about their presuppositions. Having said that, extreme apriorism which immunizes large parts of theory from empirical criticisms, has become highly suspect in the development of philosophy of science and, with some time lag, also among economists 
%\label{ref:RND8Gl3x8WMa4}(Scheall and Linsbichler, 2023).
\parencite[][]{scheall_rise_2023}. %
 Accordingly, and since the standard view maintains that praxeology depends on extreme apriorism, philosophers and economists have condemned praxeological methodology as well as economic claims based on praxeological research.\footnote{See e.g. the quotes in Linsbichler 
%\label{ref:RNDUeu9GiRQmL}(2021a, p.3360).
\parencite*[][p.3360]{linsbichler_austrian_2021}.%
}



While some Austrian economists, most prominently Rothbard, embrace extreme apriorism, others challenged the standard interpretation of Mises's justification of praxeology.\footnote{See e.g. 
%\label{ref:RNDy2Gi0QQiqH}(Zanotti, Borella and Cachanosky, 2023).
\parencite[][]{zanotti_hermeneutics_2023}.%
} They tried to argue that, rightly understood, Mises's position is not extremely aprioristic after all. Scheall 
%\label{ref:RNDj9VW96RfwL}(Scheall, 2017)
\parencite[][]{scheall_review_2017} %
 clarified these debates by explicating the notion of ``extreme apriorism'' as involving three dimensions. Unlike some overblown statements, especially in popularized portrayals of Austrian economics and in uncharitable criticisms, the \textit{extent} of apriorism in Mises's (and Rothbard's) account of praxeology is not extreme after all. Only the fundamental axiom is a~priori and very little is implied by the fundamental axiom without additional premises. Yet, the \textit{kind of justification} given for the fundamental axiom and its purported \textit{certainty} are indeed extreme on almost all accounts of Misesean epistemology because they invoke introspection, intuition, or some other form of inner experience as, possibly infallible, justification.



Partly motivated by the attempt to gauge the extremeness of Mises's apriorism, partly for its own sake, a~considerable bulk of secondary literature has emerged that engages in exegetical discussions concerning Mises's justification of praxeology. A~radical but convincingly argued assessment of the state of research by Scheall 
%\label{ref:RNDPKolnKTXsG}(2023)
\parencite*[][]{scheall_rise_2023} %
 maintains that Mises's own writings are so incoherent that a~wide range of epistemological position can be ascribed to him.\footnote{Zilian 
%\label{ref:RNDSF4T7VBSae}(1990)
\parencite*[][]{zilian_klarheit_1990} %
 also identifies indications of inconsistencies in Mises's epistemological and methodological writings.}



Praxeology in Mises's tradition faces two problems: (i) if it is extremely aprioristic as most interpretations hold, then it is considered untenable in light of contemporary philosophy of science; (ii) Mises's writings seem to allow for radically different interpretations as to how he attempts to justify praxeology and consequently how extreme his apriorism is.



\section{The Conventionalist Turn: a~Few Clarifications}

A~recent proposal by Linsbichler 
%\label{ref:RNDlW1QIwTJo3}(Linsbichler, 2017; 2021a)
\parencites[][]{linsbichler_was_2017}[][]{linsbichler_austrian_2021} %
 addresses the problem of extreme apriorism and circumvents the problem of Mises exegesis. Instead of engaging in the exegetical debates, Linsbichler proposes a~defense of praxeology that is supported by some passages in Mises's writings but, more importantly, aims at ``dispelling charges according to which praxeology is untenable because it relies on extreme apriorism'' 
%\label{ref:RNDXZ0MfOu4Mw}(Linsbichler, 2021b, p.204)
\parencite[][p.204]{linsbichler_otto_2021}%
---independently of whether Mises defended praxeology in this manner or not.\footnote{This reformist and constructive agenda was already present in the first presentation 
%\label{ref:RNDn49BghLwDD}(Linsbichler, 2017, see e~.g. p.124)
\parencite[][see e~.g. p.124]{linsbichler_was_2017} %
 but is more accentuated in 
%\label{ref:RNDx6k5IXKjmn}(Linsbichler, 2021a; 2021b).
\parencites[][]{linsbichler_austrian_2021}[][]{linsbichler_otto_2021}.%
}



Other justifications of praxeology which avoid extreme apriorism are not precluded, but Linsbichler 
%\label{ref:RND55SBTeNGJR}(2017)
\parencite*[][]{linsbichler_was_2017} %
 proposes a~conventionalist defense of analytic praxeology, first embedded in a~broader reconstruction of Mises's methodological views and later more focused and detailed on conventionalist praxeology 
%\label{ref:RNDe7HDimJLVQ}(Linsbichler, 2021a).
\parencite[][]{linsbichler_austrian_2021}.%




The vital step is to construe the fundamental axiom as analytic instead of synthetic a~priori. This shift is prompted by the insight that, contrary to claims by many praxeologists, it is perfectly conceivable to explain human behavior employing alternatives to the fundamental axiom. Neither direct observation nor intuition nor introspection can rule out these alternatives conclusively. This is a~challenge to the interpretation of the fundamental axiom as a~Kantian synthetic a~priori, which would preclude the existence of any alternatives.



An analytic sentence is true in virtue of the definitions and semantic rules of the language in which it is formulated (and logical rules of inference). Hence, an analytic fundamental axiom would also be a~priori.\footnote{See Kripke 
%\label{ref:RNDuEnwiZcTQR}(Kripke, 1972; 1980, pp.122–123)
\parencites[][]{kripke_naming_1972}[][pp.122–123]{kripke_naming_1980} %
 for potential complications that can, however, be avoided by choosing a~suitable semantic theory and adequate definitions. } This is in line with Oliva Córdoba's presentation of ``Analytic Praxeology'' according to which ``it is conceded on all sides that being analytic is sufficient for being a~priori'' 
%\label{ref:RNDONbynCZeey}(Oliva Córdoba, 2017, p.528; see also 523).
\parencites[][p.528]{oliva_cordoba_uneasiness_2017}[see also 523,][]{oliva_cordoba_uneasiness_2017}.%




Having said that, there seems to be a~subtle difference between Linsbichler's and Oliva Córdoba's versions of analytic praxeology.\footnote{Linsbichler also separates his approach from Oliva Córdoba's while commending the ``logical and explicatory aspect'' of the latter 
%\label{ref:RNDYoACIP99Qz}(Linsbichler, 2021a, p.3374).
\parencite[][p.3374]{linsbichler_austrian_2021}.%
} On the one hand, Linsbichler stresses that definitions of terms and rules of a~language can in principle be set at will, as long as they are consistent with each other. Which definitions and rules of language to adopt is a~matter of choice. Definitions are true in virtue of being definitions and, more generally, analytic truths are true by convention.\footnote{One reviewer objects that analytic truths being true by convention ``is a~controversial and ultimately untenable position in the philosophy of language and logic''. While an encompassing defense would go far beyond the scope of this paper, we submit, first, that our conceptions of analyticity and conventionalism as well as the internal/external distinction sketched below are particularly amenable to deriving the conclusion that analytic truths are truths by convention. Second, although there is indeed controversy about the open problems of this account, it is by no means outlandish. For a~contemporary defense of conventionalism in logic and mathematics in natural and formalized languages, see Warren 
%\label{ref:RNDMjzcmUgHCQ}(2020).
\parencite*[][]{warren_shadows_2020}. %
 } Introspection or intuition play a~minor role at best, and the resulting approach is only mildly aprioristic with respect to the kind of justification. On the other hand, Oliva Córdoba 
%\label{ref:RNDIeI4jD99FV}(2017, p.527)
\parencite*[][p.527]{oliva_cordoba_uneasiness_2017} %
 states that ``what accounts for the truth of the conceptual explications […] is nothing over and above a~proper grasp of the concepts involved''. This approach seems to assume the existence of concepts, independent of and prior to language. Apparently, these concepts can be ``grasped'' and ``explicated'' more or less properly. Oliva Córdoba 
%\label{ref:RNDLJ4qokZIWz}(2017)
\parencite*[][]{oliva_cordoba_uneasiness_2017} %
 gives partial ``conceptual explications'' or ``conceptual clarifications'' of the concepts of uneasiness, action, and scarcity, which he then skillfully employs as premises in proofs. While Linsbichler would likely construe these starting points of the proofs as partial definitions and thus analytically true by convention (and probably scientifically fruitful and in broad agreement with everyday language to boot), Oliva Córdoba's notion of analyticity seems to require more. These starting points have to reflect a~``proper grasp of the concepts involved'' 
%\label{ref:RND91xBTCkiZW}(Oliva Córdoba, 2017, p.527)
\parencite[][p.527]{oliva_cordoba_uneasiness_2017} %
 to yield analytic truths. Arguably, the judgment of whether such a~grasp is indeed proper or not involves some sort of intuition. Hence the resulting research program is more extremely aprioristic than Linsbichler's.\footnote{See also footnote 24 on essentialist conceptual analysis.} \textit{If} there is an ultimate standard to assess the correctness of logical rules, semantic rules, and definitions of terms, then analytic praxeology is not conventional.


\begin{table}[h!]
    \centering
    \begin{adjustbox}{max width=\textwidth}
        \begin{tabular}{|>{\centering\arraybackslash}m{4cm}|>{\centering\arraybackslash}m{4cm}|>{\centering\arraybackslash}m{4cm}|}
            \hline
            & \textbf{fundamental axiom claimed to be analytic} & \textbf{fundamental axiom claimed to be synthetic} \\ \hline
            \textbf{conventionalist justification} & Linsbichler & \footnotemark \\ \hline
            \textbf{non-conventionalist justification} & Oliva Córdoba & standard interpretation of Mises and Rothbard \\ \hline
        \end{tabular}
    \end{adjustbox}
    %\caption{Comparison of Justifications for Fundamental Axioms}
\end{table}
\footnotetext{In principle, there could also be a~conventionalist defense of a~synthetic fundamental axiom. It would have to ignore certain unpleasant empirical findings though. See also footnote 32.}

Once the fundamental axiom is understood analytically and the existence of different linguistic systems is acknowledged, conventionalism stands to reason. Unless one invokes a~strongly essentialist philosophy of language aiming for ``the one correct notion of action,'' the choice of linguistic systems is guided by pragmatic arguments. Arguably, linguistic systems in which ``man acts'' is analytic are quite close to natural language and also fruitful for social scientific investigations.



Given an analytic fundamental axiom and provided the deductions are valid, there is a~limited theoretical core of analytic statements that potentially facilitates more fruitful theorizing about economic phenomena.\footnote{Cf. Linsbichler's 
%\label{ref:RNDVSDfIfhjYT}(2023a)
\parencite*[][]{linsbichler_ultra-refined_2023} %
 reconstruction of Aumann's position in the philosophy of game theory, according to which game theory is an ultra-refined, analytic grammar for talking and thinking about interactions.} Some clarifications of the notion of conventionalism might be helpful.



Quite different epistemological positions have been labeled ``conventionalism'' in the history of philosophy of science and beyond. Linsbichler discusses some of the problems of many variants and immunizing strategies of conventionalism. While he deployed a~specific Popperian notion of conventionalism first 
%\label{ref:RND7DERgtGqG2}(Linsbichler, 2017; Popper, 2009, pp.367–511),
\parencites[][]{linsbichler_was_2017}[][pp.367–511]{popper_two_2009}, %
 he extended and generalized the approach later 
%\label{ref:RNDBcx9r4Apec}(Linsbichler, 2021a)
\parencite[][]{linsbichler_austrian_2021} %
 by highlighting two necessary conditions for a~methodology to be conventionalist. Arguably, these conditions---which we will also adopt in this paper---encompass almost all positions labelled as ``conventionalist'':



\begin{enumerate}[label=(\Alph*)]
\item The conventions could in principle have been chosen differently, i.e. alternative theories or research programs are possible.
\item The conventions are not justified by observation or intuition, but by pragmatic arguments for the superior expediency of the resulting theory or research program. 
%\label{ref:RNDXB8Ph8FEJ2}(Linsbichler, 2021a, p.3371)
\parencite[][p.3371]{linsbichler_austrian_2021}%
\end{enumerate}

While Linsbichler 
%\label{ref:RNDwYmkRVug3B}(2021a)
\parencite*[][]{linsbichler_austrian_2021} %
 indirectly hints at the Carnapian inspiration throughout the paper, it should perhaps have been stated more clearly that the formulations of (A) and (B) draw on a~distinction between internal questions to be solved within a~linguistic framework on the one hand and external questions about such frameworks to be discussed in a~meta-language on the other 
%\label{ref:RNDYtqYstHR21}(cf. Carnap, 1950).
\parencite[cf.][]{carnap_empiricism_1950}. %
 Once definitions and the rules of language are postulated, such that a~version of the fundamental axiom is analytic, there are no alternatives to the fundamental axiom \textit{in this framework.} The axiom is analytically true and it cannot be otherwise \textit{internally}. The alternatives mentioned in (A) exist in different frameworks which are visible from the meta-perspective outside the framework only. By contrast, a~non-conventionalist Kantian reading of Mises holds that the fundamental axiom is a~synthetic statement a~priori and as humans, equipped with a~particular structure of mind, we are not capable of imagining or experiencing the world in a~manner that would contradict the fundamental axiom, so there are no alternatives to it, full stop.



The second condition, (B), can be elucidated by the framework approach as well. Internal questions have to be distinguished from external questions again. Within a~linguistic framework, the justification of an analytic fundamental axiom rests solely on the definitions and rules of language of that framework. Note that being true in virtue of definitions does not imply triviality. The respective proofs can be highly intricate. The formulation of (B), stating that conventions are justified by pragmatic arguments, refers to the \textit{external} question how to set up a~framework or which framework to choose. Empirical evidence and intuitions play a~role in such pragmatic arguments and in the decisions they inform but so do evaluative elements. We expand on this point at some length to emphasize that the choice of frameworks and the choice of conventions is arbitrary only in the limited sense that it is usually not determined but requires a~decision. The choice of conventions is eminently \textit{not} arbitrary in the sense that it is a~purely subjective matter of taste 
%\label{ref:RNDstA1J070U7}(see also Linsbichler, 2021a, p.3379; 2024).
\parencites[see also][p.3379]{linsbichler_austrian_2021}[][]{dambock_factvalue_2024}.%




\section{Dissipating Further Worries About Praxeology Without Extreme Apriorism}

An explication of praxeology with an analytic fundamental axiom and with limited a~priori scope is only mildly aprioristic. Accordingly, it is much more amenable to empiricist and other contemporary positions in philosophy of science and methodology of economics. Results of praxeological investigations thus cannot be dismissed off-hand but should and can be discussed constructively instead of dogmatically between Austrian and non-Austrian economists.



Yet, perhaps conventionalist praxeology is only praxeology by name, whereas in substance it is completely detached from Mises's and Rothbard's original approach. Surely not any theory of human action should be subsumed under praxeology in the sense of Austrian economics. We offer a~fivefold response to this worry: First, the originator of praxeology, Mises, proclaims remarks and arguments that contain at least traces of the idea of an analytic fundamental axiom and arguably even of conventionalism 
%\label{ref:RNDeJPPfGeeyZ}(Linsbichler, 2021a, pp.3376–3378).
\parencite[][pp.3376–3378]{linsbichler_austrian_2021}.%




Second, aprioricity is the crucial property of the fundamental axiom and of praxeology that Mises, Rothbard, virtually all praxeologists, and critics of praxeology stress again and again as quintessential to the approach. Since analyticity implies aprioricity, this requirement is unequivocally fulfilled. By contrast, Lipski 
%\label{ref:RNDz8nrHmtDro}(2021)
\parencite*[][]{lipski_austrian_2021} %
 proposes a~more radical reform of praxeology by explicitly adding empirical hypotheses as axioms to obtain directly testable predictions. Thereby, he drops aprioricity. Lipski's diligently argued venture might well be advisable to promote the explanatory power of theories of human action. However, for better or worse, without aprioricity it ceases to qualify as praxeology in the Misesean tradition in our assessment.



Third, analogously to aprioricity, other epistemological or methodological traits might be considered indispensable from a~praxeological or Austrian School perspective. Although there is no clear consensus on the details, some forms of realism and of anti-instrumentalism are often regarded as a~trademark of the philosophy of Austrian economics 
%\label{ref:RNDSNfPPAFDPk}(Linsbichler, 2021c; 2021d).
\parencites[][]{heilmann_philosophy_2021}[][]{linsbichler_philosophy_2021}. %
 If conventionalism contradicted realism or anti-instrumentalism, it might be unpalatable to many Austrian School methodologists. Yet, Linsbichler 
%\label{ref:RNDWepo4LZD4u}(2021a, pp.3380–3383)
\parencite*[][pp.3380–3383]{linsbichler_austrian_2021} %
 substantiates at length why his variant of conventionalism is anti-instrumentalist as well as compatible with many versions of both realism and anti-realism.



Fourth, prima facie the conventionalist proposal concerns the \textit{justification} of praxeology, not its \textit{content}.\footnote{Compare the related assessment that Mises's praxeological research program can be reconstructed in Lakatosian terms with a~hard core that is de facto barred from empirical tests---but that this does not imply that Mises was a~conventionalist 
%\label{ref:RND9a1JWIwMHm}(see Zanotti, Borella and Cachanosky, 2022).
\parencite[see][]{zanotti_can_2022}. %
 Different justifications are in principle available as to why certain statements should be barred from empirical testing and these immunized statements need not even be analytic.} If successful, working Austrian economist can ignore the methodological disputes and by and large continue to use the fundamental axiom as before, i.e. draw logical conclusions from it and not submit it to direct empirical tests.\footnote{Linsbichler's 
%\label{ref:RNDxidXuxOg8p}(2017; 2021a)
\parencites*[][]{linsbichler_was_2017}[][]{linsbichler_austrian_2021} %
 conventionalist defense of praxeology hinges on a~construal of the fundamental axiom as analytic, though. To be sure, it seems plausible to assume that such an explication of the axiom's exact content is in line with the meaning of the fundamental axiom described by Mises and Rothbard and employed by praxeologists, as well as quite coherent with natural language usage. Yet, future more detailed logical inquiries might reveal that in order to be kept analytic, the exact specification of the fundamental axiom would have to be altered to an extent unacceptable to working economists. Only in this unlikely case, a~conventionalist defense of praxeology would trigger a~substantial change in the content of praxeological theory.}



Fifth, in contrast to Mises's justification of praxeology, Rothbard's does not contain any traces of conventionalism. At one point, he explicitly dismisses the idea that the law expressed in the fundamental axiom is a~disguised definition 
%\label{ref:RNDWu4c6k8o1q}(Rothbard, 1957, p.318).
\parencite[][p.318]{rothbard_defense_1957}. %
 Consequently, an analytic praxeology with a~conventionalist justification seems to be out of the question for him.



Having said that, his methodological writings allow for the reconstruction of four requirements which a~praxeological fundamental axiom must fulfill. Linsbichler 
%\label{ref:RNDJokiZDcpL2}(2021a, p.3376)
\parencite*[][p.3376]{linsbichler_austrian_2021} %
 bluntly states that ``these requirements pose more severe difficulties to Rothbard's own arguments […] than to a~conventionalist justification of praxeology'' but does not exhibit an argument. Sections 6 to 8 in the paper at hand address that lacuna. If an analytic fundamental axiom meets the Rothbardian criteria, we corroborate that conventionalist praxeology is perfectly compatible with the philosophy of the Austrian School of economics.



One final type of worry casts doubt on almost the entire debate on praxeological epistemology. Aspects of those doubts can be traced back to Rothbard who wonders whether ``epistemological pigeon-holing'' into a~priori and a~posteriori, analytic and synthetic, or empirical and theoretical, might not be a~waste of time. After all, is not the only relevant point that the fundamental axiom is self-evidently true 
%\label{ref:RNDkh8WQl609w}(see e.g. Rothbard, 1957, p.318)?
\parencite[see e.g.][p.318]{rothbard_defense_1957}? %
 Working economists might indeed be well advised not to spend too much time with methodology. However, for an analysis of the logical structure of the arguments those economists make, of the consequences of the fundamental axiom, and of the questionable claim regarding self-evidence---for these and other questions, economic methodologists employ specialized technical terms. They are picked out of the conceptual toolbox of philosophy. It is inconsequential whether Mises, Rothbard, or other scholars themselves used these concepts. They are analytical tools which---if well-defined---might facilitate the analysis of the ideas of these authors.



A~priori, analytic, and the other concepts mentioned above are applicable to sentences. So, if praxeology does not consist of sentences and the fundamental axiom is not a~sentence, this would indeed constitute a~deep problem for the entire enterprise. In this spirit, Bylund 
%\label{ref:RNDfETRhfiEw1}(2023)
\parencite*[][]{bylund_alexander_2023} %
 demurs that the term ``fundamental axiom'' was only coined by Rothbard and the very idea of axiomatizing a~set of sentences was foreign to Mises. The latter's conception of praxeology rather targets at delimiting a~realm of study, according to Bylund. Given the vital role that Mises and virtually all praxeologists attribute to truth and to deduction, we strongly suggest sticking with the standard interpretation of praxeological theory as a~set of sentences. The predicate ‘truth' is only applicable to sentences; and it is sentences that are governed by deductive rules of inference as well 
%\label{ref:RNDV0PuVjA7oy}(Linsbichler, 2023a).
\parencite[][]{linsbichler_ultra-refined_2023}.%




\section{Rothbard's Essentialist Defense of Praxeology}

Mises's significance as the originator and principal proponent of praxeology notwithstanding, the secondary literature has perhaps overly focused on his contributions---at the expense of more profound examination of his main successor. An analysis of the defenses of praxeology must take into account the view of Rothbard. After all, his first monograph on human action 
%\label{ref:RNDt9O9iJznjf}(Rothbard, 1962a; 2009)
\parencites[][]{rothbard_man_1962}[][]{rothbard_man_2009} %
 is considered the most important theoretical work of the praxeological branch of the Austrian School, which formed as a~self-conscious group from the 1970s onwards, not least through Rothbard's initiatives, articles, and personal conversations 
%\label{ref:RND4QBFH93Gtt}(Gordon, 2007, pp.122–124; Holcombe, 2014; Rockwell Jr., 2010; White, 1977; 2003, pp.26–27).
\parencites[][pp.122–124]{gordon_essential_2007}[][]{holcombe_advanced_2014}[][]{rockwell__jr_murray_2010}[][]{white_methodology_1977}[][pp.26–27]{white_methodology_2003}. %
 In reconstructing his defense of praxeology below, we will place particular emphasis on deviations from Mises. This should not obscure the fact that Rothbard rightly sees his work as a~continuation of Mises's work and always writes most admiringly of his mentor. Conversely, Mises expressly praised the contents of the sections of the monograph presented first: ``I would subscribe to every word Rothbard has written in his study.'' 
%\label{ref:RNDTvH7rmxIeU}(see Mises, 1976, p.158; cf. also Gordon, 2008, p.2).
\parencites[see][p.158]{mises_my_1976}[cf. also][p.2]{gordon_who_2008}.%




Rothbard, like Mises, considers the deduction of economic theorems from the fundamental axiom to be the task of economics. He refines and expands on the elaboration of praxeological theory formation in relation to economics. In doing so, he identifies the a~posteriori auxiliary axioms and discusses their role in derivations more clearly than his predecessors. As with Mises, the extension of praxeology to all human activity remains largely programmatic. The economic sphere may be somewhat broader than in the mainstream of economics but the theorems of the ``general, formal theory of human action'' 
%\label{ref:RNDMCM5jCH3iW}(Rothbard, 1951b, p.945)
\parencite[][p.945]{rothbard_praxeology_1951} %
 rarely stray far from the sphere of catallactics. Rothbard does indeed mention ``largely unexplored areas'' of praxeology: the theory of war,\footnote{Compare Taghizadegan \& Otto 
%\label{ref:RNDYdcziWWUMQ}(2015).
\parencite*[][]{taghizadegan_praxeology_2015}.%
} game theory,\footnote{Mises 
%\label{ref:RND0MPTgvQsyY}(1998, pp.116–117)
\parencite*[][pp.116–117]{mises_human_1998} %
 does not concede the slightest connection between game theory and praxeology. A~similar remark can also be found later 
%\label{ref:RNDwU45DyMn50}(Mises, 2012, p.135).
\parencite[][p.135]{mises_ultimate_2012}. %
 However, the more detailed passage on the relationship between Morgenstern's and Neumann's work and praxeology could also be read as indicating a~shift of opinion 
%\label{ref:RNDl5uTDlT5gW}(Mises, 2012, pp.89–90).
\parencite[][pp.89–90]{mises_ultimate_2012}. %
 It is difficult to assess the extent to which substantive reasons were responsible for this, such as the advances within game theory, which was now able to deliver results beyond zero-sum games. A~certain distancing of parts of the Austrian school from game theory, which still exists today, could be explained by the general aversion to formalization.} and ``unknown'' 
%\label{ref:RND2YaaLT3CDn}(Rothbard, 1951b, p.946).
\parencite[][p.946]{rothbard_praxeology_1951}.%




Rothbard goes into much more detail than Mises regarding the structure of social scientific explanations and predictions. The reconstruction of praxeological explanations by Linsbichler 
%\label{ref:RND8itbPl9s4E}(2017, pp.52–55; see also Gordon, 1999)
\parencites*[][pp.52–55]{linsbichler_was_2017}[see also][]{gordon_economics_1999} %
 draws on Mises's and Rothbard's expositions and are coherent with a~conventionalist methodology: ``If his prediction proves erroneous, it is not praxeology that has failed, but his judgement of the future behavior of the elements in the praxeological theorem. Praxeology is indispensable, but it does not provide omniscience'' 
%\label{ref:RNDC5iIntOb8F}(Rothbard, 1951b, p.945).
\parencite[][p.945]{rothbard_praxeology_1951}. %
 In the interplay of praxeology and thymology, Rothbard transfers the entire empirical content to the boundary conditions. However, Rothbard 
%\label{ref:RNDjUUbQjJzFr}(1989)
\parencite*[][]{rothbard_hermeneutical_1989} %
 not only describes the application of praxeology and distinguishes thymology from hermeneutics, but applied these conceptual insights practically. While Mises largely confines himself to economics, philosophy of science and, in his later work, some social philosophy, Rothbard, a~student of mathematics and economics, tries to pursue and combine economics, philosophy, political theory, ethics, natural law, history, and the history of ideas. In addition to this already versatile oeuvre, he published a~(not entirely serious) drama. The interplay between praxeology and thymology is consciously applied in many of Rothbard's historical works 
%\label{ref:RND08J9vCIZE1}(Rothbard, 1962b; 1963; 2000; 1975; 2011a; 1996; 2012; 2020).
\parencites[][]{rothbard_panic_1962}[][]{rothbard_americas_1963}[][]{rothbard_americas_2000}[][]{rothbard_conceived_1975}[][]{rothbard_conceived_2011}[][]{rothbard_origins_1996}[][]{rothbard_war_2012}[][]{rothbard_history_2020}. %
 His methodological approach is in line with Mises' and Hayek's ideas of the interplay between theory and history 
%\label{ref:RNDwoQdRh3wsS}(White, 1977; 2003, p.26).
\parencites[][]{white_methodology_1977}[][p.26]{white_methodology_2003}. %
 In the context of this paper, however, we are primarily interested in the epistemological status of the fundamental axiom in Rothbard's approach. How does he try to justify the truth of praxeology and how successful is he? We will explore these questions in sections 5-7.



Rothbard phrases the fundamental axiom in some places, as Mises does, with ``man acts'', but more often with ``human action exists''. If one tries to explicate the content of these vague phrases from the explanations as well as from the use in the deductions, one arrives at very similar results as in Mises and as sketched in section 1.\footnote{We deem the apparent form as an existential proposition, which Rothbard's short form of the fundamental axiom assumes, of no particular importance.}



Let us next turn to attempts to establish the truth or the necessary truth of the fundamental axiom, and thus to Rothbard's entire perspective on praxeology. He prefaces his arguments with an extremely revealing remark. Rothbard reminds the reader, almost apologetically, that the undertaking is difficult and, in a~sense, useless. He quotes Toohey and the choice of words is indicative of Rothbard's view of social scientific knowledge:



\begin{quote}
Proving means making evident something which is not evident. If a~truth or proposition is self-evident, it is useless to attempt to prove it, to attempt to prove it would be to attempt to make evident which is already evident 
%\label{ref:RNDKK0YiIOE3V}(Rothbard, 1976, p.28).
\parencite[][p.28]{rothbard_praxeology:_1976}.%
\end{quote}




Both Rothbard's and Mises's defense of praxeology are dealing with ``proofs'', i.e. the establishment of certain knowledge. Likewise, Rothbard, in his anticipation of arguments of Hoppe's discourse ethics, writes of attempts at a~``refutation'' 
%\label{ref:RNDjJ746CI5yD}(Rothbard, 1976, pp.28–29).
\parencite[][pp.28–29]{rothbard_praxeology:_1976}. %
 Of course, in Rothbard's work, too, the economist is a~fallible human being, and critical debate is the key to scientific progress. For Rothbard, however, current social science knowledge does not contain bold, fallible hypotheses that provide good explanations at the moment and have corroborated their worth, but established truths.\footnote{Similarly, Mises 
%\label{ref:RNDSQ8XGlNzEJ}(1998, p.68)
\parencite*[][p.68]{mises_human_1998} %
 draws a~sharp line between tentative laws in the natural sciences and praxeology: ``Praxeology---and consequently economics too---is a~deductive system. It draws its strength from the starting point of its deductions, from the category of action. No economic theorem can be considered sound that is not solidly fastened upon its foundation by an irrefutable chain of reasoning. A~statement proclaimed without such a~connection is arbitrary and floats in midair.''} They might only be overturned if some researcher made a~mistake. This conception of science as a~search for certainty may be partly responsible, at least psychologically, for the vehemence with which Rothbard and many other Austrians advocate their economic and oftentimes also their political positions.



Rothbard's epistemological and methodological writings hardly suffer from the tensions discerned in Mises's. He offers a~straightforward, strongly essentialist justification of the fundamental axiom.



In the Aristotelian and Thomistic tradition, Rothbard does not want to deal primarily with isolated sensory impressions, atomistic units, or superficial economic quantities. The goal is rather to holistically uncover the essences of phenomena by means of a~cognitive synthesis: ``The empiricism is broad and qualitative, stemming from the essence of human experience'' 
%\label{ref:RNDM4nDuoFAW8}(Rothbard, 2007, p.xvi).
\parencite[][p.xvi]{mises_preface_2007}. %
 The grasp of potentialities and essences is purportedly possible through targeted attention. Rothbard controversially attributes a~similar position to Alfred Schütz 
%\label{ref:RNDrFuJHBOSbB}(Rothbard, 2011b [1973]; 1976),
\parencites[][]{rothbard_praxeology_2011}[][]{rothbard_praxeology:_1976}, %
 arguably these essentialist tenets rather have Austrian precursors in Spann, Wieser, and Mayer 
%\label{ref:RNDT68jV0GE7t}(Milford and Rosner, 1997; Linsbichler, 2022).
\parencites[][]{milford_abkopplung_1997}[][]{linsbichler_viel_2022}. %
 How the synthetic, holistic grasp of certain, intersubjectively verifiable truths is to proceed is not even hinted at, let alone precise methodological regulations given. Anyways, Rothbard insists that introspection, without any inductive steps, warrants the necessary and universal truth of ``man acts'' with certainty.



\section{Rothbard's Requirements for a~Fundamental Axiom }

From Rothbard's methodological and epistemological writings, four criteria for the acceptability of a~fundamental axiom for the social sciences can be reconstructed. To be sure, Rothbard's deliberations on ``man acts'' are much more exhaustive. Since our goal is to evaluate \textit{justifications} of a~fundamental axiom, we only include pertinent statements, though. Among other things, statements about the context of discovery are not included in the criteria. For instance, Rothbard obtains the fundamental axiom via introspection and uses alleged attributes of introspection to argue for the a~priori truth of the axiom. We deem a~priori truth to be the desired goal, the criterion which a~fundamental axiom must meet. Introspection is merely a~means to this end and thus not a~necessary requirement.



The four Rothbardian criteria, on the basis of which we will assess Rothbard's essentialist and Linsbichler's conventionalist defense of ``man acts,'' are that a~fundamental axiom of praxeology must have the following four properties: (I) Its falsification is inconceivable. (II) It is empirically meaningful. (III) It is a~priori with respect to complex historical events. (IV) It is absolutely true. We will now explicate and briefly discuss these four claims in turn 
%\label{ref:RNDU8yJRItL0L}(Rothbard, 1957, pp.314, 317–319; 1976, p.25).
\parencites[][pp.314, 317–319]{rothbard_defense_1957}[][p.25]{rothbard_praxeology:_1976}.%
\footnote{Note, however, that this paper commits neither to an endorsement nor to a~criticism of these criteria for social scientific research in general. They serve an instrumental purpose only. Given Rothbard's eminent status within the praxeological branch of Austrian economics, it is plausible that his criteria are or should be important for many praxeologists. Hence the Rothbardian requirements are a~prime candidate for the intended ``undogmatic methodological critique'' 
%\label{ref:RNDslLhUTEKvP}(Caldwell, 1984, p.129),
\parencite[][p.129]{caldwell_praxeology_1984}, %
 i.e. an appraisal of (praxeological) Austrian School claims and arguments from the perspective of (praxeological) Austrian economics.}

\medskip

\noindent (I) A~falsification of the fundamental axiom is inconceivable 
%\label{ref:RNDvvIS51VpZd}(Rothbard, 1957, p.318).
\parencite[][p.318]{rothbard_defense_1957}. %
 While Rothbard classifies the auxiliary axioms of praxeology as obviously true in our world, a~counterfactual scenario in which they are false can be thought of without contradiction, so they are not necessarily true. As for the fundamental axiom however, the fact that human individuals have some goals and pursue them by any means must apply in every possible world in which there are human individuals 
%\label{ref:RND4io5TrPEIB}(Rothbard, 1957, pp.314–315):
\parencite[][pp.314–315]{rothbard_defense_1957}:%




\begin{quote}
In short, we can imagine a~world where resources are not diverse, but not one where people exist but don't act. We have seen that the other postulates, although ``empirical'', are so obvious and acceptable that they can hardly be called ``falsifiable'' in the usual empiricist sense. How much more true this is true of the axiom, which is not even conceivably falsifiable! 
%\label{ref:RNDaZ2L4VueKN}(Rothbard, 1957, p.317)
\parencite[][p.317]{rothbard_defense_1957}%
\end{quote}




Quoting Toohey, Rothbard 
%\label{ref:RNDUmKxgZZmiV}(1976, p.28)
\parencite*[][p.28]{rothbard_praxeology:_1976} %
 provides another illustrative example of a~proposition, the falsification of which is inconceivable. He asserts that one cannot think that one has seen a~round square. Although Rothbard phrases this and similar remarks in terms of impossible thought processes, he can plausibly be interpreted in line with the approach in this paper. Since Rothbard disclaims impositionist views, he arguably holds that the justification of a~priori statements is not concerned with the limitations of the human cognitive apparatus but with conceptual analysis. Having said that, conceptual analysis is not a~purely analytic method for Rothbard but involves intuitive access to essences.\footnote{For analytic conceptual analysis, see Linsbichler 
%\label{ref:RNDisDzAtvzGk}(2017, pp.81–83).
\parencite*[][pp.81–83]{linsbichler_was_2017}. %
 For another variant of essentialist conceptual analysis, see Wieser 
%\label{ref:RNDF6FogNfTlc}(1884),
\parencite*[][]{wieser_uber_1884}, %
 Linsbichler 
%\label{ref:RNDlY3HvkkXUB}(2021e; 2023b),
\parencites*[][]{linsbichler_sprachgeist_2021}[][]{linsbichler_case_2023}, %
 Schweinzer 
%\label{ref:RNDmFVAxCGos6}(2000),
\parencite*[][]{schweinzer_two_2000}, %
 Tokumaru 
%\label{ref:RNDSwFdzfo5P1}(2015).
\parencite*[][]{tokumaru_wiesers_2015}.%
}

\medskip

\noindent (II) The fundamental axiom is ``empirically meaningful'' 
%\label{ref:RNDQnYdJDqgyr}(Rothbard, 1957, p.318).
\parencite[][p.318]{rothbard_defense_1957}. %
 Without clearly distinguishing the two, Rothbard situates Mises's epistemology sometimes in a~Kantian framework 
%\label{ref:RND7CBMgHMBtO}(Rothbard, 2011b, p.33),
\parencite[][p.33]{rothbard_praxeology_2011}, %
 and sometimes in neo-Kantian one 
%\label{ref:RNDsph6QajynC}(Rothbard, 1957, pp.317–318).
\parencite[][pp.317–318]{rothbard_defense_1957}. %
 The salient point is that, according to Rothbard, Mises considers the fundamental axiom to be a~``law of thought'' 
%\label{ref:RND7dSPtxX13P}(Rothbard, 1957, p.318),
\parencite[][p.318]{rothbard_defense_1957}, %
 a~categorical truth a~priori to all experience, and apodictically true.



Rothbard 
%\label{ref:RNDpQYrIs96gq}(2011b, pp.33–34)
\parencite*[][pp.33–34]{rothbard_praxeology_2011} %
 asserts that most praxeologists, like himself and in contrast to Mises, interpret the fundamental axiom \textit{empirically}, albeit apodictically true nonetheless. However, such references to experience or to ‘the real world' are a~far cry from modern conceptions of empiricism, as Rothbard 
%\label{ref:RNDqtVsM41HHy}(1957, p.318)
\parencite*[][p.318]{rothbard_defense_1957} %
 himself acknowledges. For Rothbard, in order to be ``empirically meaningful'' some indirect, possibly vague relationship between the terms of praxeological theory and phenomena in the physical world suffices. In particular, note that in more contemporary terminology, his idiosyncratic use of ``empirically meaningful'' neither implies falsifiability 
%\label{ref:RNDY9Mukvlib6}(Rothbard, 1976, p.25),
\parencite[][p.25]{rothbard_praxeology:_1976}, %
 nor testability, nor does it establish intersubjective experience as a~critical standard for the truth of the statement. With a~criterion of meaning or of empirical significance, such as that discussed in the Vienna Circle and in today's philosophy of science, Rothbard's demand has little more in common than the name.



An upshot of Rothbard's view why the fundamental axiom counts as empirically meaningful is that it is neither a~law of thought nor a~psychological theory about the capacity of the human sensory and cognitive apparatus. For Rothbard, 'human action exists.' purports to make an assertion about the world outside the human cognitive apparatus, not merely about the human limits of the possibility of perceiving this world. Thus, an interpretation of the fundamental axiom as a~genetic or psychological a~priori can be ruled out for Rothbard's defense of praxeology.\footnote{Mises's late work 
%\label{ref:RNDDNpFD88wmg}(Mises, 1962; 2012)
\parencites[][]{mises_ultimate_1962}[][]{mises_ultimate_2012} %
 is not entirely clear on this point.} When Rothbard calls the fundamental axiom ``empirically meaningful'', he excludes not only the genetic a~priori but also other interpretations: it cannot be a~methodological principle because such a~principle would be a~normative rule and not a~descriptive assertion. Moreover, we can conclude from Rothbard's demand for empirical meaning that an explication of the fundamental axiom should not be understood as an uninterpreted axiom system or as a~group of pseudo-propositions. Instead, the meanings of the terms contained, for example 'human', are at least partially fixed independently of the fundamental axiom. Some of Rothbard's objections to the mathematization of economics underpin this reading of '''empirically meaningful'' as well.\footnote{See 
%\label{ref:RND9l9D44emh0}(Rothbard, 1976, pp.21–24)
\parencite[][pp.21–24]{rothbard_praxeology:_1976} %
 and also Linsbichler 
%\label{ref:RNDC4XyuHZvSQ}(2021e; 2023b).
\parencites*[][]{linsbichler_sprachgeist_2021}[][]{linsbichler_case_2023}. %
 }

\medskip

\noindent (III) The fundamental axiom is a~priori with respect to complex historical events 
%\label{ref:RNDjIaa2dS4Sn}(Rothbard, 1957, p.318; 1976, p.25).
\parencites[][p.318]{rothbard_defense_1957}[][p.25]{rothbard_praxeology:_1976}. %
 Rothbard's engagement with the complexity of social scientific situations (in alleged contrast to less complex natural scientific situations) is typical, if relatively extreme, for Austrian economics. He describes the fundamental axiom not only as ``empirically meaningful'' but even as ``radically empirical'' 
%\label{ref:RNDQK4p3xnGRM}(Rothbard, 1976, p.24).
\parencite[][p.24]{rothbard_praxeology:_1976}. %
 We explained above that these statements are not to be misunderstood in a~post-Humean sense of empiricism. According to Rothbard, complex historical events can only illustrate conclusions from the fundamental axiom. They are not suitable as proof or test 
%\label{ref:RND03nZtGseIi}(Rothbard, 1951a, p.181; 1951b, pp.944–945).
\parencites[][p.181]{rothbard_mises_1951}[][pp.944–945]{rothbard_praxeology_1951}. %
 If, like Rothbard, one understands forms of introspection or reflection as a~form of experience, the fundamental axiom regarding this specific inner experience is a~posteriori. In any case, it is a~priori with regard to complex historical events. External experience is not a~critical standard by which praxeological sentences are measured.

\medskip

\noindent (IV) The fundamental axiom is ``absolutely true'' 
%\label{ref:RNDUQtTEj9JT8}(Rothbard, 1957, pp.314, 317).
\parencite[][pp.314]{rothbard_defense_1957}. %
 What distinguishes truth from absolute truth in Rothbard's nomenclature is not entirely clear. The formulation can be read as an expression of the lack of differentiation between truth and certainty. In other passages, Rothbard seems to have in mind truth without exception in our world or the much stronger truth in all possible worlds, i.e. necessary truth.

			
In any case, it is crucial for praxeology that the fundamental axiom be true. Only if the starting point of the deductive chains is true, this desired truth value is transferred to all conclusions. This fourth and final requirement that Rothbard makes of the fundamental axiom is therefore the most important for the project to justify general social scientific laws, i.e. to solve the problem of induction in the theoretical social sciences by means of praxeology.\footnote{For praxeology as a~solution to the problem of induction, see Linsbichler 
%\label{ref:RND8GgGs1SToS}(2017)
\parencite*[][]{linsbichler_was_2017} %
 and Tokumaru 
%\label{ref:RNDGfDZvmYmFZ}(2009).
\parencite*[][]{tokumaru_poppers_2009}. %
 Note that the essentialist Rothbard explicitly contends, quoting John Elliott Cairnes, that no process of induction is necessary for the discovery of praxeological knowledge because strictly general knowledge can be obtained directly by turning attention to our consciousness 
%\label{ref:RND834S1QkBrv}(Rothbard, 2011b, pp.65–68).
\parencite[][pp.65–68]{rothbard_praxeology_2011}.%
} As we shall see, however, Rothbard's argument is quite problematic.



First of all, it is striking that he tries to establish the truth of the fundamental axiom with considerations that can be assigned to the context of discovery instead of the context of justification.\footnote{See Reichenbach 
%\label{ref:RNDWWO1OcjzKc}(1938)
\parencite*[][]{reichenbach_experience_1938} %
 and also Hoyningen-Huene 
%\label{ref:RNDT7VlkQJ1Bb}(1987)
\parencite*[][]{hoyningen-huene_context_1987} %
 for complications with the distinction.} According to Rothbard, the fundamental axiom as well as its truth are grounded in ``universal inner experience, and not simply on external experience, that is, its evidence is reflective rather than physical'' 
%\label{ref:RND8OlJCT90lz}(Rothbard, 1957, p.318).
\parencite[][p.318]{rothbard_defense_1957}.%
\footnote{Elsewhere, Rothbard 
%\label{ref:RNDojeceSLUM1}(2011b, pp.33–34)
\parencite*[][pp.33–34]{rothbard_praxeology_2011} %
 also emphasizes introspection. Yet, the basis of knowledge about human action is not always solely universal inner or reflective experience but external physical experience is additionally invoked.} The special character of this holistic introspection as a~source of knowledge is supposed to prove the universality of the fundamental axiom:



\begin{quote}
However, although the axioms are \textit{a~priori} to history, they are a~posteriori to the universal observations of the logical structure of the human mind and human action. The axioms are therefore open to the test of observation in the sense that, once postulated, they are universally recognized as true. Such recognition may be accused of being ‘introspective‘, but it is nonetheless scientific, since it is an introspection that can command the agreement of all. 
%\label{ref:RND12Z6Ef9JUu}(Rothbard, 1951a, p.181)
\parencite[][p.181]{rothbard_mises_1951}%
\footnote{See also Rothbard 
%\label{ref:RNDir9I42DiTe}(1957, pp.317–318).
\parencite*[][pp.317–318]{rothbard_defense_1957}.%
}
\end{quote}



Rothbard, like Mises, considers intersubjectivity to be a~hallmark of scientificity. With a~wide variety of formulations, he tries to suggest that the specific view of the nature of human action provides and guarantees intersubjectivity in addition to truth and certainty: The fundamental axiom is allegedly evident to anyone who contemplates it---just as evident as sense experience 
%\label{ref:RNDI8K1rKWhUU}(Rothbard, 2011b, p.35).
\parencite[][p.35]{rothbard_praxeology_2011}.%
\footnote{Note that standard empiricist epistemologies which Rothbard apparently aims to emulate here do not accept reports of sensory data as infallible guarantees of certainty.} Every individual, in the face of a~reflection on the axiom of action, must agree to its truth and to its importance for the social sciences 
%\label{ref:RNDPqCeqb8cZy}(Rothbard, 1951b, p.943).
\parencite[][p.943]{rothbard_praxeology_1951}. %
 A~person could, of course, claim to deny the existence of these self-evident principles. You can say whatever you want; but there are limits to thinking and doing 
%\label{ref:RNDXB4yXj5XdJ}(Rothbard, 1976, p.28).
\parencite[][p.28]{rothbard_praxeology:_1976}. %
 For logical reasons, for example, no one can imagine a~round square.\footnote{So far, so good, but Rothbard does not merely reject claims involving inconsistent concepts. Going a~decisive step further, he maintains that certain reports of alleged logical or empirical findings have ``no epistemological validity'' either. If data contradict ``established truths of the real world'', they can and should be ignored in Rothbard's methodology 
%\label{ref:RNDC3iv5tDkqp}(1976, p.28).
\parencite*[][p.28]{rothbard_praxeology:_1976}. %
 Such ad hoc immunizing strategies are also characteristic of some variants of conventionalism but decidedly not what Linsbichler 
%\label{ref:RNDQvwBT7B5xP}(2021a, p.3370)
\parencite*[][p.3370]{linsbichler_austrian_2021} %
 suggests, not least because they ``facilitate [...] dogmatic tendencies''.}



\section{Appraisal of Rothbard's Account According to His Own Criteria}


\begin{flushright}
\textit{``You can't always get what you want.''} (Keith Richards, Mick Jagger)
\end{flushright}






Given the discussion of Rothbard's position in the previous two sections, we can put on record that an essentialist account is able to render the fundamental axiom ``man acts'' empirically meaningful in Rothbard's weak sense (II) and a~priori to complex historical events (III).



Requirement (IV), absolute truth, turns out more questionable. Rothbard invokes a~special form of introspection as a~source of knowledge and approves of it as a~criterion of truth. The postulation of truth criteria is extremely problematic in the context of a~fallibilistic conception of science, even if only sentences about one's own consciousness would be affected. For someone with only the slightest empiricist inclination (in the modern sense), the description of an empirical fact, such as a~personal psychic experience, can only ever be a~hypothesis, not a~certified truth.\footnote{The only exception to the epistemological impossibility of truth criteria may be some formal systems with no reference to experience or an external world. Ironically, the early Mises 
%\label{ref:RND0tUEGbPQAB}(1940, p.18)
\parencite*[][p.18]{mises_nationalokonomie_1940} %
 characterizes experience, including inner experience, as yielding findings that always could have been expected differently and infers explicitly that neither outer nor inner experience can justify the universal propositions of praxeology.} The compelling conviction that Rothbard obviously feels, and which perhaps many humans feel regarding some specific inner or outer experience, does not guarantee that the sentence describing the content of the compelling experience is true. A~mental conviction of truth, no matter how intense, is not proof of the validity of the content of a~sentence or of a~chain of deductive steps.



Yet, for the sake of argument let us concede to Rothbard that he has intuited, with necessary truth, that he himself has goals and uses means to achieve them. The main difficulties for establishing the fundamental axiom in a~Rothbardian manner arise when one tries to infer statements about the minds of other people from inner experience. How is it possible to draw necessary conclusions about other people from the exploration of one's own consciousness?



Since Rothbard requires and considers the fundamental axiom to be empirically meaningful, the term 'human' is at least partly interpreted, i.e. at least for many paradigmatic cases it is determined which physical objects are in the extension of ‘human' and which are not. Suppose m~is one such human individual and suppose it turns out that m~does not act. Then, for Rothbard, the potential immunization strategy of simply not calling everything that does not act as a~human being is blocked.



Rothbard seems to be aware of the problem and the respective rejection of inner experience as a~reliable source of knowledge in Mises's earlier writings 
%\label{ref:RNDk6cICUoXyQ}(e.g. Mises, 1940, pp.17–19).
\parencite[e.g.][pp.17–19]{mises_nationalokonomie_1940}.%
\footnote{Cf also the following criticism of Spann's essentialist intuitive universalism by Mises, which would incidentally be applicable to Rothbard as well: ``However, what Spann has in mind when he declares the a~priori method to be the only one appropriate for sociology as he conceives it is not at all a~priori reasoning, but intuitive insight into a~whole'' 
%\label{ref:RNDslCDspFDYH}(Mises, 2003, p.46).
\parencite[][p.46]{mises_epistemological_2003}.%
} Thus, Rothbard struggles to demonstrate why this particular form of intuition is not tantamount to ``the arbitrariness of intuitive flights of fancy'' 
%\label{ref:RND3EVK1VqwC5}(Mises, 2003, p.52)
\parencite[][p.52]{mises_epistemological_2003} %
 but would necessarily command universal intersubjective agreement. Without providing new arguments, he repeats and reformulates the claim that it is so, sometimes quoting supposed authorities like Aristotle, Thomas Aquinas, Say, Cairnes, Toohey, Schütz, and Knight.\footnote{See section 6, (IV) above.}



Let us suppose that every person states that introspection made her realize that she is acting. This is insufficient for intersubjectivity, though. Intersubjectivity would require several people to be able to focus their attention on the consciousness of the same person p. Then, for the time being, consensus could possibly be reached on the truth value of the statement 'M acts.' Inner experience, however, does not allow us to explore the consciousness of other people---at least not without analogical conclusions. Such an analogy inference involves induction. According to Rothbard, however, inductive methods are not possible or necessary in the sphere of human action. By Rothbard's own standards, not even the proposition 'M acts' is intersubjectively verifiable. How much more problematic is the demand that the fundamental axiom 'All people act' can established as true.



Furthermore, it is dubious how Rothbard's account can show that a~falsification of the fundamental axiom is inconceivable. This criterion (I) does not concern falsifiability in a~Popperian sense but negates the existence of a~consistent alternative. Yet, behaviorism and the ontology of what Mises calls ``primitive man'' contradict the fundamental axiom ``man and only man acts'' by attributing purpose, goals, desires, and beliefs to no objects at all (behaviorism) or to more objects than human individuals respectively (``primitive man'' speaks of angry rivers and sad clouds and their intentions).\footnote{Cf. the discussions of these alternatives and their acknowledgement by Mises in Linsbichler 
%\label{ref:RNDoaDrm74RGo}(2017; 2021a).
\parencites*[][]{linsbichler_was_2017}[][]{linsbichler_austrian_2021}.%
} Behaviorist monism may be rejected for pragmatic reasons, as conventionalist praxeologists and arguably Mises do. But theories in which other people merely behave instead of acting purposefully can be conceived and formulated without special problems. Indeed, some radical post-Humean empiricists call for people to be treated in social scientific theories in the same way as animals, plants, crystals, buildings, swamps, rain, rivers, cities, X-rays, and the Milky Way 
%\label{ref:RNDiEouYpi0iB}(see Neurath, 1944; 1970).
\parencites[see][]{neurath_foundations_1944}[][]{neurath_foundations_1970}.%




Rothbard's attempt to establish certainty, intersubjectivity, and truth and for the fundamental axiom is on shaky ground.\footnote{Since Tarski's work, the conceptual distinction between certainty and truth can be made without epistemological concerns. ``Once this is noted, it is obvious that truth is distinct from certainty and that the supposed unattainability of the latter does not undermine the legitimacy or utility of the former'' 
%\label{ref:RND9FFZlBGahH}(Soames, 1999, p.32).
\parencite[][p.32]{soames_understanding_1999}.%
} Vague references to a~specific source of knowledge cannot close the gaps in the arguments needed. A~more precise specification of procedures and methods of application of introspection would most likely reveal its inductive character. Moreover, the object of cognition---the category of one's own actions---is not accessible to others.



\section{Appraisal of Linsbichler's Conventionalist Praxeology According to Rothbard's Criteria}


\begin{flushright}
\textit{``You can get it if you really want.``} (Jimmy Cliff)
\end{flushright}






In sections 1,3, and 4, we outlined the conventionalist research program with an analytic fundamental axiom as a~starting point. Despite Rothbard's deprecating stance towards such a~project, we now investigate whether it meets the four criteria he staunchly upholds.

\medskip

\noindent (I) A~falsification of the fundamental axiom is inconceivable. This is the most challenging hurdle to overcome. While conventionalism fares better than essentialism, neither approach fully meets the requirement.



Although Rothbard does not refer to the standard notion of falsifiability, note that such a~Popperian falsification is hard to achieve for common versions of the fundamental axiom. Which observational statements would contradict that human individuals and only human individuals behave purposefully, i.e. act. The goals, preferences, and knowledge which according to praxeology play a~crucial role in acting are not directly observable. Maybe humans do not have goals, maybe door handles do, but how could we experience this? Following Mises 
%\label{ref:RNDmwS0e6eH6O}(1940, p.85),
\parencite*[][p.85]{mises_nationalokonomie_1940}, %
 one might consider future improved neurophysiological aids for falsifying the fundamental axiom. Such prospective methods would identify observable physical and chemical processes in the brain with the very content of specific thoughts. Such a~decision between behaviorism and praxeology would, among other things, depend on non-trivial theories of translation, though.



The very fact that---except for debatable future neurophysiology---no potentially observable states of affairs are excluded by the fundamental axiom motivated the very idea to construe it as analytic and renders the variant of conventionalism regarding the fundamental axiom ``one of the least uncontroversial versions'' 
%\label{ref:RNDc1cn3lAhzP}(Linsbichler, 2021a, p.3371).
\parencite[][p.3371]{linsbichler_austrian_2021}.%




Reverting back to Rothbard's demand, alternatives to the fundamental axiom remain conceivable, no matter what essentialist or conventionalist arguments are brought forward. Even when granting the validity of introspection for one's own mental states, it remains possible and conceivable---in principle---that all other human individuals do not act but merely behave. Metaphysical speculation cannot definitively decide the prima facie logical tie between praxeology and behaviorism either, as Mises actually acknowledges at one point 
%\label{ref:RNDbJIGryZ1C1}(Mises, 1940, pp.84–86).
\parencite[][pp.84–86]{mises_nationalokonomie_1940}.%




What conventionalism---in contrast to essentialism---can provide is an approximation to meeting criterion (I). Once a~specific version of the fundamental axiom is construed as analytic, praxeological reasoning proceeds in a~framework in which it is true by definition. Thus, within the conceptual scheme of this framework, a~negation of the fundamental axiom is analytically false after all. In this limited sense, criterion (I) is almost fulfilled, as long as the economist stays within her linguistic framework. Of course, she can step out of her linguistic framework, abandon her research program, and conceive of behaviorism in a~meta-language. These remaining objections to Rothbard's criterion (I) are ultimately unavoidable.

\medskip

\noindent (II) The fundamental axiom is empirically meaningful. In the conventionalist research program, the term 'human individuals' is intended to designate objects in the physical world,\footnote{Strictly speaking, theories are not interpreted in ``the physical world'' or ``reality'' but in the model(s) which serve as a~proxy for the ``real world'' 
%\label{ref:RNDa93nusHe6z}(cf. Linsbichler, 2023a; Przelecki, 1969).
\parencites[cf.][]{linsbichler_ultra-refined_2023}[][]{przelecki_logic_1969}.%
} even though some borderline cases might be left undecided. On top of that, in all likelihood, the fundamental axiom can be ascribed the truth value 'true' in all situations without changing the observable extension of ‘human' and by merely tinkering with the structure of theoretical terms if need be. The fundamental axiom is thus clearly empirically meaningful in the weak sense demanded by Rothbard.

\medskip

\noindent (III) The fundamental axiom is a~priori with regard to complex historical events. The fundamental axiom is analytic, and therefore a~priori to any experience, both in the sense of epistemology and in the sense of primacy. Capturing complex historical events is not possible without praxeology or a~comparable other theory.



\begin{quote}
It is only with the aid of a~theory that we can determine what the facts are. Even a~complete stranger to scientific thinking, who naively believes in being nothing if not ``practical,'' has a~definite theoretical conception of what he is doing. Without a~``theory'' he could not speak about his action at all, he could not think about it 
%\label{ref:RNDz23fAOhBEi}(Mises, 2003, p.29).
\parencite[][p.29]{mises_epistemological_2003}.%
\end{quote}


\medskip

\noindent (IV) The fundamental axiom is absolutely true. In the conventionalist approach, the fundamental axiom is always true, namely true per conventionem. This result is already an improvement over the arguments available to the essentialist. Yet, Rothbard's criterion requires ``absolute'' truth (whatever that exactly amounts to). If we interpret this as being true in all linguistic frameworks, no matter how the terms are defined in them, any justification must obviously fail. The sentence ‘Murray is a~libertarian' is true if the terms have their usual meaning, but we can easily render the sentence false by changing the meaning of ‘Murray' or of ‘libertarian'. And to ask for the ‘truth' of a~sentence, \textit{independently} of a~framework which assigns meanings to the sentence, is unintelligibly with standard notions of truth. No sentence, considered as a~purely syntactic string of signs, is true independently of the meaning attached to it.



It is neither clear whether Rothbard demands necessary truth for the fundamental axiom, nor which notion of necessity such a~demand would draw on. We therefore suspend judgement on whether a~conventionalist justification substantiates the necessary truth of the fundamental axiom. However, Linsbichler's conventionalist praxeology does offer more than mere plain truth. The fundamental axiom is \textit{analytically} true, and thus can be plausibly interpreted as fulfilling (IV). Our analysis results in the following summarizing chart:\footnote{The quotation marks serve as a~reminder that Rothbard uses these terms with idiosyncratic meaning.}






\begin{table}[H]
    \centering
    \begin{adjustbox}{max width=\textwidth}
        \begin{tabularx}{\textwidth}{|L{4.5cm}|Y|Y|}
            \hline
            & \textbf{Rothbardian Essentialism} & \textbf{Conven\-tion\-alism} \\ \hline
            \textbf{not conceivably ``falsifiable''} & \ding{55} & (\ding{51}) \\ \hline
            \textbf{``empirically meaningful''} & \ding{51} & \ding{51} \\ \hline
            \textbf{\textit{a priori} to complex historical events} & \ding{51} & \ding{51} \\ \hline
            \textbf{absolutely true} & \ding{55} & \ding{51} \\ \hline
        \end{tabularx}
    \end{adjustbox}
%    \caption{Comparison between Rothbardian Essentialism and Conventionalism}
\end{table}

Rothbard does not provide a~conventionalist defense of praxeology at all. His methodological and epistemological writings do not even contain the traces of this idea which we find in Mises. Given the problems Rothbard's own justification of praxeology faces in light of his self-imposed criteria, perhaps he should have considered conventionalism after all.



\section{Outlook: What is to be done?}

This paper, hopefully, clarified some details of Linsbichler's conventionalist defense of praxeology and compared its merits with Rothbard's essentialist arguments. Whereas previous work 
%\label{ref:RNDhnBqp7f94K}(Linsbichler, 2017; 2021a; Lipski, 2021; Scheall, 2017; Tokumaru, 2018)
\parencites[][]{linsbichler_was_2017}[][]{linsbichler_austrian_2021}[][]{lipski_austrian_2021}[][]{scheall_review_2017}[][]{tokumaru_review_2018} %
 mainly emphasized the superiority of a~conventionalist defense from the external stance of empirically minded contemporary philosophy of science, the paper at hand takes a~different approach. Using Rothbard's methodological and epistemological writings as the source, we reconstructed four desired properties of a~fundamental axiom which its justification should be able to establish. These four are by no means the only internal Austrian criteria by which conventionalism can be evaluated but at least it passed this first test more successfully than intuitive universalism.



One---but certainly not the only--- major open problem that remains to be addressed is the exact formulation of the fundamental axiom, ideally both in natural language and in a~formal language.\footnote{The structure of the fundamental axiom explicated in section 1 of this paper as well as the partial definitions and proofs in Oliva Córdoba 
%\label{ref:RNDHxH5dUvjj8}(2017)
\parencite*[][]{oliva_cordoba_uneasiness_2017} %
 are first steps in that direction. See also Linsbichler 
%\label{ref:RNDATKCNmFn9u}(2023b).
\parencite*[][]{linsbichler_case_2023}.%
} This task is vital for all praxeologists because it enables an assessment whether certain intended consequences are actually derivable deductively. If the fundamental axiom is construed as analytic, the issue is even more pressing. It would be desirable to instantiate that such a~construal is possible.



\paragraph{Acknowledgments:}
I~am grateful to two anonymous reviewers, Agustina Borella, Daniel Eckert, David Mayer, Karl Milford, Berthold Molden, William Peden, Scott Scheall, Lukas Starchl, Richard Sturn, Igor Wysocki, and Gabriel Zanotti for helpful discussions, questions, and comments.



\paragraph{Funding information:} This research was funded in whole or in part by the Austrian Science Fund (FWF) [grant DOI {10.55776/ESP206}]. For open access purposes, the author has applied a~CC BY public copyright license to any author-accepted manuscript version arising from this submission.

\end{artengenv}


\begin{artengenv}{Łukasz Dominiak}
	{Free market, blackmail, and Austro-libertarianism\edtfootnote{This research was funded in whole or in part by the National Science Centre, Poland, grant number 2020/39/B/HS5/00610.}}
	{Free market, blackmail, and Austro-libertarianism}
	{Free market, blackmail, and Austro-libertarianism}
	{Nicolaus Copernicus University in Toruń\label{dominiak-first}}
	{In the present paper we examine the standard Austro-libertarian account of blackmail according to which blackmail should be legal as it does not coerce the blackmailee to part with his property and so cannot be subsumed under extortion. Against this account we put forth a~preliminary argument or a~hypothesis, if you will, that even if blackmail cannot be subsumed under extortion, it still does not follow that it should be legal, for it might be subsumed under fraud. Indeed, the hypothesis we would like to offer for consideration is that blackmail is fraud, at least under some circumstances. To wit, we claim that even if the blackmailer does not coerce the blackmailee, in cases in which the blackmailer does not have an intention to execute his otherwise legal threats, he nonetheless deceives the blackmailee, thereby inducing him to part with his property. This is fraud and it renders the blackmailee's property transfer involuntary and invalid. As fraud should be illegal under Austro-libertarianism, so should blackmail.
	}
	{blackmail, fraud, coercion, Austro-libertarianism, Walter Block.}






\section{Introduction}

\lettrine[loversize=0.13,lines=2,lraise=-0.03,nindent=0em,findent=0.2pt]%
{T}{}he standard Austro-libertarian view on blackmail is that blackmail would be legal on the free market due to the fact that exchanges effectuated under blackmail are not coerced and thus free or voluntary.\footnote{Writes Rothbard 
%\label{ref:RNDjUbFoVONYE}(2009, p.183):
\parencite*[][p.183]{rothbard_man_2009}: %
 ``[\textit{B}]\textit{lackmail} would not be illegal in the free society. For blackmail is the receipt of money in exchange for the service of not publicizing certain information about the other person. No violence or threat of violence to person or property is involved.'' } Since the free market is nothing else than the entirety of voluntary exchanges,\footnote{Again, writes Rothbard 
%\label{ref:RNDp1HOXIHIjd}(2011, p.320):
\parencite*[][p.320]{rothbard_toward_2011}: %
 ``The free market is the name for the array of all the voluntary exchanges that take place in the world.'' } blackmail would be part and parcel thereof. Certainly, blackmail is immoral, but there are many immoral things taking place on the free market.\footnote{On immoral although legitimate free market practices see 
%\label{ref:RNDLcJ4HJkm6P}(Block, 2018).
\parencite[][]{block_defending_2018}.%
} What is crucial, is not whether it is moral or immoral, but whether it is free or unfree, voluntary or involuntary. Similarly, it can be argued that blackmail is a~threat rather than an offer, but then again, it is inconsequential if it is a~threat or an offer.\footnote{Walter Block sometimes calls blackmail a~threat, sometimes an offer. See, for example, Block and Anderson 
%\label{ref:RNDP2L8VyRXtU}(2000, p.546),
\parencite*[][p.546]{block_blackmail_2000}, %
 Block 
%\label{ref:RNDRCqHeE2EGG}(1998a, p.218).
\parencite*[][p.218]{block_libertarian_1998}.%
} On the free market there are many offers that cannot be refused and many threats that can be withstood. What matters is whether it coerces another to part with his goods or services. Since it does not, it is voluntary and would therefore be legal on the free market. Or so argue Austro-libertarians.



In the present paper we supplement this argument with an observation concerning another dimension of blackmail. More specifically, we put forth a~preliminary argument or a~hypothesis, if you will, that even if one accepts the Austro-libertarian premise that blackmail proposals do not coerce, it does not follow that exchanges induced by such proposals are free or voluntary\footnote{It might be viewed as a~bit clumsy to use ‘free' and ‘voluntary' synonymously. We are aware of that. Regardless, we take our liberty to do so because Austro-libertarians themselves do so. } and so would find their place on the free market as it is understood by Austro-libertarians. The reason for which the non-coercive character of blackmail proposals does not entail voluntariness of the resultant exchanges is that there are two ways in which voluntariness of human actions can be vitiated and coercion is only one of them. Another is ignorance. As we try to argue and explain, at least some blackmail proposals deceive the blackmailee and thus render the resultant exchange involuntary due to the blackmailee's induced ignorance. In other words, the blackmailer, at least in some cases, defrauds the blackmailee. Thus, the hypothesis we would like to offer for consideration is that blackmail is oftentimes fraud. Now since Austro-libertarians strongly believe that fraud ought to be illegal, they should reject their current view on blackmail as inconsistent with this strong belief and instead embrace the view that \textit{qua} fraud, blackmail should also be illegal.



The present paper is organized in the following way. Section 2 offers an in-depth analysis of the standard Austro-libertarian account of blackmail. Section 3 argues against this account and puts forth what can be called a~preliminary revisionist Austro-libertarian account of blackmail. The crucial aspect of this account is that it tries to subsume some types of blackmail under fraud. Section 4 elucidates reasons for which fraud, and thus blackmail, should be illegal under Austro-libertarianism. Section 5 concludes.



\section{The standard Austro-libertarian account of blackmail}


The standard Austro-libertarian account of blackmail begins with the distinction between blackmail and extortion. According to this account, extortion---regardless of how this term is actually used in existing legal systems\footnote{For example, Glanville Williams 
%\label{ref:RND0ml2CeCOEc}(1983, p.838)
\parencite*[][p.838]{williams_textbook_1983} %
 reminds us that as far as English law is concerned, ``[i]t is the offence of extortion at common law for a~public officer to take, by colour of his office, any money or thing that is not due to him.'' In turn, \textit{California Penal Code (2006)}, as reported by Sanford H. Kadish, Stephen J. Schulhofer and Carol S. Steiker 
%\label{ref:RNDjkPItdvcdA}(2007, p.941),
\parencite*[][p.941]{kadish_criminal_2007}, %
 defines extortion under Section 518 in the following way: ``Extortion is the obtaining of property from another, with his consent, or the obtaining of an official act of a~public officer, induced by a~wrongful use of force or fear, or under color of official right.''}---consists in obtaining another's goods or services by coercion or effective threats of property rights violation. A~typical case of extortion would be the proverbial highwayman threatening a~traveler with his ‘Your money or your life' proposal or John Locke's 
%\label{ref:RNDwB8yHTPBMU}(2003, p.385 [1698, II, Chap. XVI, §176])
\parencite[p.~385\ \mbox{[1698, II, Chap.~XVI, §176]}]{locke_two_2003}\nocite{locke_two_1689} %
 robber who ``break[s] into my House, and with a~Dagger at my Throat make[s] me seal Deeds to convey my Estate to him.'' What renders extortion illegal according to the standard Austro-libertarian account is that the transfer of money to the highwayman is involuntary. As a~result, although the money physically travels to the highwayman, title thereto stays with the victim. In other words, victim's waiver is void or what comes to the same thing, his consent is invalid. Now the fact that the highwayman gets hold of the victim's property without having title thereto results in a~broadly construed theft, appropriation of another's property\footnote{On the crime of appropriation see George P. Fletcher 
%\label{ref:RNDjFzcnT8z5P}(2000, pp.7–22).
\parencite*[][pp.7–22]{fletcher_rethinking_2000}.%
} or what Murray Rothbard 
%\label{ref:RNDaFslS9qDov}(1998, p.77)
\parencite*[][p.77]{rothbard_ethics_1998} %
 calls an implicit theft. Since theft should be illegal on the free market, so should extortion.



The crucial step in the above argument concerns the reason for which extortive exchanges are involuntary and thus result in invalid title transfers. Generally speaking, this reason can be identified as acting under duress or coercion. It is the fact that the victim acts under duress or coercion that renders his actions involuntary and in consequence invalidates his title transfers. However, and here a~peculiarity of the Austro-libertarian account comes to the fore, it is not (only?) due to the fact that the extortion victim's will is overborne, fettered or somehow influenced by threats that the victim's actions are involuntary. Rather, it is (also?) a~function of the content of the extortive proposal that renders these actions involuntary. To wit, it is because the extortive proposal threatens the victim with \textit{rights violation} that makes the victim's actions involuntary. To use Richard Epstein's 
%\label{ref:RND5kqTzR0p6J}(1975, p.296)
\parencite*[][p.296]{epstein_unconscionability_1975} %
 pertinent words, the victim's actions are involuntary and his title transfer invalid because ``in the case of duress by the threat of force, B~has required A~to abandon one of his rights to protect another.'' Thus, as explained by Robert Nozick\footnote{Note, however, that Nozick cannot be classified as a~representative of the standard Austro-libertarian account of blackmail and this is so not only for the reason that he was not Austrian in his economic thinking. More importantly, in his discussion of blackmail, Nozick focuses mainly on the question of productivity of blackmail exchanges rather than on the question of rights. Moreover, exactly due to its unproductivity, he is quite critical of blackmail legalization.} 
%\label{ref:RNDjkXGhJnNF4}(1974, p.262),
\parencite*[][p.262]{nozick_anarchy_1974}, %
 under libertarianism:



\begin{quote}
Whether a~person's actions are voluntary depends on what it is that limits his alternatives. If facts of nature do so, the actions are voluntary. (I may voluntarily walk to someplace I~would prefer to fly to unaided.) Other peoples' actions place limits on one's available opportunities. Whether this makes one's resulting actions non-voluntary depends upon whether these others had the right to act as they did.
\end{quote}



And this view is further confirmed by Rothbard who also believes that whether an action or an exchange is free depends on the question of property rights. After all, for Rothbard freedom as such is defined in terms of property rights. As he 
%\label{ref:RNDglNd2aabKy}(2006, p.50)
\parencite*[][p.50]{rothbard_for_2006} %
 puts it, ``[f]reedom is a~condition in which a~person's ownership rights in his own body and his legitimate material property are \textit{not} invaded.'' Thus, for example, Rothbard 
%\label{ref:RNDuo4SagrLXZ}(2009, pp.182–183)
\parencite*[][pp.182–183]{rothbard_man_2009} %
 ``completely overthrows the basis for a~law of defamation'' because ``a man has no such objective property'' in his reputation. Rather, ``[h]is reputation is simply what others think of him, i.e., it is purely a~function of the subjective thoughts of others. But a~man cannot own the minds or thoughts of others. Therefore, I~cannot invade a~man's property by criticizing him publicly.''



Indeed, analyzing extortion, Walter Block and Gary M. Anderson 
%\label{ref:RND2yxPgVFm9F}(2000, p.546)
\parencite*[][p.546]{block_blackmail_2000} %
 point out that in the case of extortion ``there is no voluntary exchange'' since ``the victim's rights are violated, in that he must give up something to which he was legally entitled.'' And further they 
%\label{ref:RNDu2jLgJ1uqk}(2000, p.546)
\parencite*[][p.546]{block_blackmail_2000} %
 elaborate that ``[w]hen someone extorts money from you with the statement ‘your money or your life!' and you give up the former, you are wronged since you own both.'' Thus, for Block and Anderson 
%\label{ref:RNDAPVS2dq6nW}(2000, p.545)
\parencite*[][p.545]{block_blackmail_2000} %
 the highwayman's proposal ``would not constitute a~voluntary contract'' because regarding the threatened consequence, his ``right does not exist, since we have no right to murder other people.'' Clearly, one does not have a~right to kill, rape, maim or rob another and since it is and should be ``illegal to murder or rape, it should also be a~criminal act to threaten such acts.'' 
%\label{ref:RNDXx6n6esyNb}(Block and Anderson, 2000, p.543)
\parencite[][p.543]{block_blackmail_2000}%




Now Block and Anderson 
%\label{ref:RNDXdT6GoRXhI}(2000, p.544)
\parencite*[][p.544]{block_blackmail_2000} %
 draw a~very sharp distinction ``between blackmail and extortion, and argues that the former does, under all circumstances, represent an entirely voluntary transaction.'' Or as they 
%\label{ref:RNDTW9FyHRo1s}(2000, p.560)
\parencite*[][p.560]{block_blackmail_2000} %
 put it in slightly different terms, ``\textit{blackmail} per se, the exchange of silence for cash, is an uncomplicated voluntary act between consenting adults.'' And when they 
%\label{ref:RNDi1ti0nn07X}(2000, p.546)
\parencite*[][p.546]{block_blackmail_2000} %
 identify the fact that ``the victim's rights are violated'' as the reason for which ``there is no voluntary exchange'' in extortion, they 
%\label{ref:RND11m9wqpmlV}(2000, p.546)
\parencite*[][p.546]{block_blackmail_2000} %
 in turn point out that in the case of blackmail, ``[i]n sharp contrast, when someone threatens ‘Give me money or I~reveal your secret,' you are not wronged since you do not have title to both.'' More specifically, you do not have title to the blackmailer's forbearance to exercise his freedom of speech. After all, as pointed out by Block and Anderson 
%\label{ref:RNDEbtQt944Ua}(2000, p.546),
\parencite*[][p.546]{block_blackmail_2000}, %
 while ``extortion is the threat to do something which should be illegal (murder, rape, pillage)…, in blackmail the offer is to commit the paradigm lawful act (i.e. engage in free speech or gossip about secrets which embarrass or humiliate other people).'' Thus, ultimately, for Block 
%\label{ref:RNDjP7dzJegg1}(1998a, p.281)
\parencite*[][p.281]{block_libertarian_1998} %
 the difference between voluntary blackmail and involuntary extortion stems from the fact that although ``[i]n both cases, a~threat is made, coupled with a~demand (usually for money, but it might include sexual or other services, etc.) But in the former case, as we have seen, the threat is to do something licit; e.g., indulge in free speech. In the latter, the threat is anything but legal.''\footnote{Note, for example, that Rothbard concurs with this analysis. As he 
%\label{ref:RNDR9pph6KYMZ}(1998, p.124)
\parencite*[][p.124]{rothbard_ethics_1998} %
 points out, ``Smith has the right to ‘blackmail' Jones. As in all voluntary exchanges, both parties benefit from such an exchange. Smith receives money, and Jones obtains the service of Smith's not disseminating information about him which Jones does not wish to see others possess. The right to blackmail is deducible from the general property right in one's person and knowledge and the right to disseminate or not disseminate that knowledge.''}



Hence, the standard Austro-libertarian account of blackmail can be summarized in the following way. Even though it can be viewed as a~threat, a~blackmail proposal does not coerce the blackmailee to part with his goods or services. It does not coerce the blackmailee because the threat it involves is legitimate, that is, it threatens the blackmailee with something he does not have a~right against. Or from a~different angle, it threatens the blackmailee with something that the blackmailer has a~right to. Now since the blackmail proposal does not coerce the blackmailee, his parting with his goods or services is voluntary. In consequence, the blackmailee's consent or waiver or title transfer, if you will, is valid and so the blackmailer acquires not only the blackmailee's goods or services, but also the rights thereto. Accordingly, the blackmailer cannot be considered liable for theft, be it implicit, explicit or attempted. In sharp contrast, an extortive proposal involves an illegitimate threat (of something that the victim has a~right against and the offender does not have a~right to) and so coerces the victim, rendering his actions involuntary and thus invalidating his consent, waivers or title transfers. In consequence, the perpetrator of extortion acquires the victim's goods or services without having the rights thereto and so becomes liable for an implicit theft (or attempted one if his actions are not carried out to completion).



The standard Austro-libertarian account of blackmail can therefore be reduced to the following reasoning:



\begin{enumerate}

\item Since blackmail proposals are legitimate (they do not threaten with rights violations), they do not coerce.

\item Since blackmail proposals do not coerce, the blackmailee's actions are voluntary.

\item Since the blackmailee's actions are voluntary, the blackmailee's waivers are valid.

\item Since the blackmailee's waivers are valid, the blackmailer acquires rights to blackmailee's goods and services.

\item Since the blackmailer acquires rights to blackmailee's goods and services, blackmail is not an implicit, explicit or attempted theft.

\item Since blackmail is not a~theft (implicit, explicit or attempted), it is legitimate itself.

\end{enumerate}

We assume the truth of the first premise for the sake of discussion. We also believe that if the second premise were true, all the steps from 3 to 6 would be true as well. However, we submit that the second premise is false. Hence, it is to the second premise that we now turn.



\section{The revisionist Austro-libertarian account of blackmail}

The standard Austro-libertarian account of blackmail boils down to the claim that since blackmail proposals are not extortive, that is, they do not coerce the blackmailee to part with his property or, what comes to the same thing, they do not threaten the blackmailee with rights violation so that he has to give up one of his rights, they are legitimate. This claim can be debunked in two different ways which yet in the end come to the same thing. The first approach is to argue that even if blackmail is not extortive, it is still illegitimate under a~different heading. The second approach is to submit that even if blackmail does not coerce, it still renders the blackmailee's actions involuntary via a~different route. In this section we take the first approach. In the next one, the second.



Consider Block and Anderson's 
%\label{ref:RNDPjQtihxvLc}(2000, p.546)
\parencite*[][p.546]{block_blackmail_2000} %
 aforementioned typical blackmail formula: ``Give me money or I~reveal your secret.'' Clearly, this typical formula covers almost infinite number of blackmail instances (for example, ‘Give me your money or I~reveal your affair to your wife,' ‘Give me your money or I~enter this year's music competition' etc.), so if we show that it can be illegitimate, we will show---\textit{pace} Block and Rothbard---that indeed innumerable cases of blackmail can be illegitimate as well. Now Block and Anderson 
%\label{ref:RNDxGonCBUCEN}(2000, p.546)
\parencite*[][p.546]{block_blackmail_2000} %
 believe that it is legitimate to make a~proposal of this type because ``you are not wronged since you do not have title to both'' and it is only an ``offer to commit the paradigm lawful act (i.e. engage in free speech or gossip about secrets which embarrass or humiliate other people).'' In other words, it is legitimate because it is not extortive, where ``extortion is the threat to do something which should be illegal (murder, rape, pillage).''



However, assume that the blackmailer does not want to ``reveal your secret.'' The only thing he wants, quite typically as it seems, is your money. So, he leverages the fact that he knows your secret which you do not want to be revealed to induce you to pay---similarly to Nozick's 
%\label{ref:RNDJFK7EyBV16}(1974, pp.84–85)
\parencite*[][pp.84–85]{nozick_anarchy_1974} %
 architectonic monstrosity case in which the blackmailer ``has no desire to erect the structure on the land; he formulates his plan and informs you of it solely in order to sell you his abstention from it.'' In such cases, nothing changes as far as the extortion/blackmail distinction is concerned, for the blackmailer still proposes to commit what Block and Anderson 
%\label{ref:RNDZRNh01V6a4}(2000, p.546)
\parencite*[][p.546]{block_blackmail_2000} %
 call ``the paradigm lawful act'' of engaging in free speech (and free speech clearly comprises speaking as well as abstaining from speaking) or building on his own land (and private property rights to land equally clearly comprise rights to build as well as to abstain from building on the land). And indeed, Block and David Gordon 
%\label{ref:RNDqXAF8bXe26}(1985, p.49)
\parencite*[][p.49]{block_blackmail_1985} %
 admit that ``[i]t is difficult to see… why ‘unproductive' exchanges, in this sense, ought to be prohibited or singled out for special regulations.'' Alas, there is a~pretty straightforward Austro-libertarian reason why they ought to.



For note that in such cases the blackmailer deceives the blackmailee about his intentions. Even though he proposes to reveal a~secret or to build a~monstrosity, he does not intend to do so. He intentionally misrepresents crucial facts about his plans, purposes or, if you will, mental states (desires and intentions to reveal a~secret etc.) in order to deprive the blackmailee of his money and to acquire it himself. This is already an attempted fraud. And if the blackmailer successfully induces by such an intentional misrepresentation the blackmailee to part with his property, it is a~completed crime of fraud, period. There is (1) the \textit{actus reus} of fraud in the shape of the blackmailer making a~false representation of the blackmailer's mental state, thereby deceiving the blackmailee about the said mental state and inducing or causing him to part with his property, accompanied by (2) the \textit{mens rea} of fraud in the shape of making the false representation \textit{knowingly} (re its falsehood) while \textit{intending} to deceive the blackmailee and to deprive him of his property.\footnote{Compare, for example, 
%\label{ref:RNDw0ylt9eLIE}(Regina v. Théroux, [1993])
\parencite[][]{noauthor_regina_1993} %
 2 S.C.R. 5, where the court identifies the \textit{actus reus} of fraud as an act of ``deceit, falsehood, or some other dishonest act'' which ``consequence is depriving another of what is or should be his'' while the accompanying ``\textit{mens rea} would then consist in the subjective awareness that one was undertaking a~prohibited act (the deceit, falsehood or other dishonest act) which could cause deprivation in the sense of depriving another of property or putting that property at risk.~If this is shown, the crime is complete.'' Even more pertinently, compare the 
%\label{ref:RNDDkVV8diKx7}(Fraud Act 2006)
\parencite[][]{noauthor_fraud_nodate} %
 of the Parliament of the United Kingdom, section 2 (Fraud by false representation):
s\begin{enumerate}
\item A~person is in breach of this section if he---
\begin{enumerate}
\item dishonestly makes a~false representation, and
\item intends, by making the representation---
\begin{enumerate}
\item to make a~gain for himself or another, or
\item to cause loss to another or to expose another to a~risk of loss.
\end{enumerate}
\end{enumerate}
\item A~representation is false if---
\begin{enumerate}
\item it is untrue or misleading, and
\item the person making it knows that it is, or might be, untrue or misleading.
\end{enumerate}
\item ``Representation'' means any representation as to fact or law, including a~representation as to the state of mind of---
\begin{enumerate}
\item the person making the representation, or
\item any other person.
\end{enumerate}
\end{enumerate}
} Hence, we can conclude, in a~nutshell, that any blackmailer (such as, for example, Nozick's monstrosity builder) who says anything that falls under Block and Anderson's 
%\label{ref:RNDwL0ap4dqCo}(2000, p.546)
\parencite*[][p.546]{block_blackmail_2000} %
 generic formula ``Give me money or I~reveal your secret'' while not being keen on executing his threats, commits a~crime of fraud.\footnote{An anonymous referee of this journal put some strain on our present argument by inviting us to consider a~scenario in which a~car dealer makes the following proposal to his potential customer: ``Pay me \$20,000 and I~will give you a~car. Don't pay me \$20,000 and I~will not give you the car.'' The customer decides to not pay the \$20,000, but the car dealer gives him the car anyway (and planned to do so regardless if the customer paid him or not). Is this an attempted fraud since the car dealer gave the car when he said he would not do so? And further: Should the customer be free to accept the car? Should the car dealer not be free to give the car for free since he said he would not? By the same token, should the blackmailer who never intended to reveal secrets be punished? Should the blackmailee be free to accept the silence for free? Should the blackmailer not be free to not reveal secrets for free since he said he would not? Now this ingenious thought experiment of the referee is supposed to provide a~\textit{reductio ad absurdum} of our argument, for we have a~strong intuition that the car dealer does not do anything wrong. However, if he commits no wrong, neither does the blackmailer and our argument is debunked. One response to this challenge is to point out that in normal circumstances the customer is not caused to pay \$20,000 for the car simply by the car dealer saying that otherwise he will not give it to the customer and thus there is no \textit{actus reus} of fraud. But since the customer is not normally caused to part with his \$20,000 by the car dealer simply saying this, then saying this does not seem to constitute coming to a~dangerous proximity of causing such a~deprivation and so is not sufficient for the \textit{actus reus} of attempted fraud either. Once this is established, answers to other questions follow quite straightforwardly.}



Now Austro-libertarians are as much against fraud as they are against extortion, for they both are kinds of implicit (attempted or completed) theft. Thus, for example, Rothbard 
%\label{ref:RNDpog49veGns}(1998, p.77)
\parencite*[][p.77]{rothbard_ethics_1998} %
 argues that invasion of private property ``may include two corollaries to actual physical aggression: \textit{intimidation}, or a~direct threat of physical violence; and \textit{fraud}, which involves the appropriation of someone else's property without his consent, and is therefore ‘implicit theft'.'' And further he asks: ``Under our proposed theory, would fraud be actionable at law? Yes, because fraud is failure to fulfill a~voluntarily agreed upon transfer of property, and is therefore implicit theft.'' 
%\label{ref:RNDdbbgVxxDkt}(Rothbard, 1998, p.143)
\parencite[][p.143]{rothbard_ethics_1998} %
 Also in other places Rothbard expresses a~similar view about fraud, for he 
%\label{ref:RNDAYaE8wgbTP}(2009, p.803)
\parencite*[][p.803]{rothbard_man_2009} %
 believes that ``[t]he purely free market is, by definition, one where theft and fraud (implicit theft) are illegal and do not exist.'' After all, as he 
%\label{ref:RNDz1YiDhftrP}(Rothbard, 2011, p.216)
\parencite[][p.216]{rothbard_toward_2011} %
 explains, ``a ‘free market' necessarily implies total respect for and protection of private property… This implies not only a~cracking down on assault and murder, but also on all forms of theft and fraud.'' Thus, Rothbard 
%\label{ref:RNDjDEEgcxCzp}(2009, p.184)
\parencite*[][p.184]{rothbard_man_2009} %
 contends that we should ``exclude both explicit violence and the implicit violence of fraud from our definition of the free market---the pattern of voluntary interpersonal exchanges.''



By the same token, Block 
%\label{ref:RNDlLcBMgbD1O}(1998a, p.294)
\parencite*[][p.294]{block_libertarian_1998} %
 also claims that ``fraud is equivalent to theft'' and to this effect references the above quoted passages from Rothbard 
%\label{ref:RNDV2NpI9vXyu}(1998, pp.77–78).
\parencite*[][pp.77–78]{rothbard_ethics_1998}. %
 According to Block, this is so both in special cases of fraud such as, for example, counterfeiting or false witness, and in fraud \textit{per se}, regardless of its specific subject-matter. Thus, when he 
%\label{ref:RNDBWAuFWZnWU}(2015, p.38)
\parencite*[][p.38]{block_natural_2015} %
 probes the relation between the Ten Commandments and the libertarian Non-Aggression Principle (NAP), Block intimates that the biblical prohibition of bearing false witness could also find its place in the libertarian penal code \textit{qua} fraud prohibition. As he 
%\label{ref:RNDkVl6MVnksO}(2015, p.38)
\parencite*[][p.38]{block_natural_2015} %
 puts it, ``[m]urder, stealing, and false witness (fraud) are explicitly prohibited by libertarian law.'' In turn writing about counterfeiting, he 
%\label{ref:RND10eCuCDrbl}(2018, p.99)
\parencite*[][p.99]{block_defending_2018} %
 submits that ``counterfeiting is a~special case of fraud… This special case of fraud constitutes theft, just as fraud in general does.'' Now of course this anti-fraud stance stems from Block's 
%\label{ref:RNDMNzFGU0omm}(2004, p.275)
\parencite*[][p.275]{block_libertarianism_2004} %
 belief that ``libertarianism is a~deontological theory of law… [where] [p]roper legal enactments are these that support this basic premise (e.g. prohibitions of murder, rape, theft, fraud, etc.)''. Or as he 
%\label{ref:RNDSf4xIsahP8}(1998b, p.1889)
\parencite*[][p.1889]{block_environmentalism_1998} %
 explains it in a~different place, in ``libertarianism… the only improper human activity is the initiation of threat or force against another or his property'' while ``[t]o prevent murder, theft, rape, trespass, fraud, arson, etc., and all other such invasions is the only proper function of legal enactments.''



It is therefore clear that Austro-libertarians believe that fraud should be illegal. However, when juxtaposed with the above analysis of blackmail, this belief puts them in the following predicament. The blackmailer who does not intend to execute his otherwise legal threats and is only after the blackmailee's money does not commit extortion, but he does commit fraud. Since fraud is illegal under libertarian law, so should any blackmail that is perpetrated without intention to execute its otherwise legal threats. To put it as transparently as possible:


\medskip

\noindent P\textsubscript{1}: Fraud is illegal under libertarian law.



\noindent P\textsubscript{2}: Blackmail (without intention to execute its threats) is fraud.



\noindent C: Blackmail (without intention to execute its threats) is illegal under libertarian law.

\medskip

Yet, Austro-libertarians, including the most prominent ones, that is, Rothbard and Block, want to ``legalize blackmail.'' 
%\label{ref:RNDkv37M7UvWp}(see Block, 2013)
\parencite[see][]{block_legalize_2013} %
 This position, as far as it pertains to blackmail without intention to execute its threats, clearly fails to account for the possibility of blackmail being fraud and so to cohere with their own stance on fraud. That they do not see it can only be explained by what Judith Jarvis Thomson 
%\label{ref:RNDdPAoLHVhWW}(1990, pp.25–33)
\parencite*[][pp.25–33]{thomson_realm_1990} %
 called ``failing to connect.''\footnote{Thomson quotes here Edward Morgan Forster 
%\label{ref:RNDflGmxLKR1K}(1941)
\parencite*[][]{forster_howards_1941} %
 as the author of the term.} Thus, once presented with the proper connection between blackmail and fraud, they should withdraw their support for the legalization of (this sort of) blackmail, for the legalization of fraud as such would have much more profound and far-reaching consequences for the libertarian theory of justice than opposing legalization of blackmail and since there is no third way, they should oppose legalization of blackmail. Hence, they should embrace what we called a~revisionist Austro-libertarian account of blackmail. It is revisionist because it proposes that (a) blackmail without intention to execute its threats should be illegal and that (b) this sort of blackmail is better subsumed under fraud than extortion. It is nonetheless Austro-libertarian because it acknowledges that (c) blackmail proposals do not coerce and that (d) fraud should be illegal. It basically connects these dots, as the standard Austro-libertarian account fails to do.



\section{The logic of the Austro-libertarian ban on fraud}

Now let us turn to the second way of debunking the standard Austro-libertarian account of blackmail, that is, to the claim that even if blackmail does not coerce, it still renders the blackmailee's actions involuntary via a~different route. In other words, let us try to demonstrate that from the fact (assumed for the sake of discussion) that the blackmailer does not (due to the legitimate nature of his threats) coerce the blackmailee, it does not follow that the blackmailee's actions are voluntary.



As pointed out by Michael S. Moore 
%\label{ref:RNDSpzIWMKdwv}(1984, p.85),
\parencite*[][p.85]{moore_law_1984}, %
 beginning with Aristotle's \textit{Nicomachean Ethics}, human actions have always been deemed ``involuntary when they are performed (\textit{a}) under compulsion, (\textit{b}) as the result of ignorance.'' 
%\label{ref:RND187ZTKVeln}(Aristotle, 1955, p.77 [Book III, Chap. I, 1110a]).
\parencite[p.~77\ \mbox{[Book III, Chap.~I, 1110a]}]{aristotle_nicomachean_1955}. %
 Compulsion assumes either a~form of necessity when a~natural threat of, say, a~sudden tempest compels a~captain to jettison cargo in order to save the ship or a~form of duress or coercion, if you will, when a~human threat of, for example, death compels a~man to hand his money to a~robber. One peculiarity of Austro-libertarianism is that it rejects the claim that necessity compels in a~way that can justify or excuse property rights violations or invalidate title transfers. Another peculiarity of Austro-libertarianism is that it believes that only illegal threats, that is, proposals of rights violations, compel in a~way that can invalidate consent (although it is not clear whether such illegal threats can also excuse or justify violations of the third party's property rights). As we saw above, it is ultimately for this reason that Austro-libertarianism contends that only extortion should be prohibited whereas blackmail should be legal.



However, as we also saw above, Austro-libertarians believe that fraud should be illegal despite the fact that there is no illegal threat involved in it. For instance, Rothbard 
%\label{ref:RNDVNkJ4EFOvN}(1998, p.77)
\parencite*[][p.77]{rothbard_ethics_1998} %
 explicitly distinguishes a~``threat of physical violence; and \textit{fraud}, which involves the appropriation of someone else's property without his consent'' and Block 
%\label{ref:RND2qyymZt9jb}(2015, p.38)
\parencite*[][p.38]{block_natural_2015} %
 links the biblical prohibition of bearing false witness with the libertarian prohibition of fraud. Thus, it stands to reason to say that fraud must affect consent in some other way than via threat or coercion and that this way has something to do with the falsehood of the representation made by the offender. Indeed, as pointed out by Hillel Steiner 
%\label{ref:RNDmcS2x9KCmI}(2019, p.100),
\parencite*[][p.100]{steiner_asymmetric_2019}, %
 it is most natural for libertarians to try to oppose fraud by taking the second Aristotelian route, that is, the route of ignorance or mistake. As Steiner 
%\label{ref:RND5zIYzBqMnb}(2019, p.100)
\parencite*[][p.100]{steiner_asymmetric_2019} %
 puts it, for an exchange to be valid, there must be a~title transfer between the parties and ``[f]or that waiver-generated transfer to be normatively valid---for the waiver to effect the transfer of the right in question---it is necessary that it be done \textit{voluntarily}.'' Since coercion is here beside the point, it is therefore sufficient for the preservation of this voluntariness condition that the transferee, to put it in Steiner's 
%\label{ref:RNDq0NEK35t7s}(2019, p.100)
\parencite*[][p.100]{steiner_asymmetric_2019} %
 own words, ``is not falsely informed, or what I'll simply call \textit{ignorant}…. The buyer's waiver, to be normally valid, must also be performed non-ignorantly. And the duplicity of the fraudulent seller is held to defeat that condition.'' Hence, it is the ignorance of the defrauded party that accounts for the fact that his consent is invalid or as Rothbard 
%\label{ref:RND3wj8xukl2f}(1998, p.77)
\parencite*[][p.77]{rothbard_ethics_1998} %
 puts it, that ``\textit{fraud}… involves the appropriation of someone else's property without his consent.''



Certainly, if the defrauded party knew that, for example, a~car he was buying was a~lemon, he would not have bought it. It is only because he thought that the car is in good condition that he decided to purchase it. Unfortunately, he was deceived and thus ignorant about the crucial fact, that is, the car's poor condition. Accordingly, he did not know what he was really buying. He thought he was purchasing a~good car while what he was getting was a~lemon. Therefore, if he consented to anything at all, it was to exchange his money for a~different car than the one he actually got. For the latter, he did not consent to pay. Hence, now the other party has his money without his consent. This is an implicit theft, for although the offender did not take the money himself, it was handed to him without the title travelling therewith and so he now has the money without any rights thereto.



By the same token, if the blackmailee knew that the blackmailer had no intention to reveal his secrets, he could have decided not to pay him. It is only because he thought that the blackmailer would reveal his secrets that he chose to pay him. In a~sense, the blackmailee paid for what he already had. If he knew that he was paying for what he already had, he most likely would not have paid for it. Or still in other words, it was a~crucial fact for the blackmailee that the blackmailer was willing to reveal his secrets. As it turned out, he was deceived and so mistaken about this crucial fact. Thus, he did not know what he was paying for. He thought he was paying for \textit{x} while what he was getting was \textit{y}. Hence, if he voluntarily and validly consented to anything at all, it was to paying for \textit{x}, not for \textit{y}. He decidedly did not consent to exchange his money for \textit{y}. For this purpose, his waiver-generated title transfer was invalid. Accordingly, the blackmailer got the blackmailee's money without his consent. This is an implicit theft, for even though the offender did not take the money himself, it was transferred to him without proper waiver-generated title transfer and so he now enjoys the money without having any title thereto.



We can therefore see that it is not true---\textit{contra} what the standard Austro-libertarian account of blackmail claims in the second step of its case for the legalization of blackmail---that:
\begin{enumerate}[label=(\arabic*), start=2]
\item Since blackmail proposals do not coerce, the blackmailee's actions are voluntary.
\end{enumerate}
It is not true because even though the blackmailee is indeed (by assumption) not coerced, his actions are nonetheless involuntary due to his---induced by the blackmailer---mistake or ignorance. In consequence, nothing that follows from (2) can be true either. Thus, it is likewise not the case that:
\begin{enumerate}[label=(\arabic*), start=3]
\item Since the blackmailee's actions are voluntary, the blackmailee's waivers are valid.
\item Since the blackmailee's waivers are valid, the blackmailer acquires rights to blackmailee's goods and services.
\item Since the blackmailer acquires rights to blackmailee's goods and services, blackmail is not an implicit, explicit or attempted theft.
\item Since blackmail is not a~theft (implicit, explicit or attempted), it is legitimate itself.
\end{enumerate}
No, blackmail (without intention to execute its threats) is fraud and as such it is neither legitimate nor should it be legalized.



\section{Conclusions}

In the present paper we examined the standard Austro-libertarian account of blackmail. According to this account, blackmail should be legal because the blackmailer's threat---in contradistinction to the extortionist's threat---is in itself legal and so does not coerce the blackmailee. In consequence, the blackmailee's property transfer is voluntary and valid and the blackmailer does not commit any theft by acquiring it. Against this account we argued that even if the blackmailer's threat does not coerce, it does not follow that the blackmailee's property transfer is voluntary and valid. Or in other words, even if blackmail cannot be subsumed under extortion, it does not follow that it cannot be subsumed under some other crime. Indeed, as we demonstrated, in the case of blackmail which is not accompanied by the blackmailer's intention to execute his threats, the blackmailee is deceived by the blackmailer about the latter's mental state and thus ignorant about the crucial fact regarding the service he is buying. Accordingly, even though the blackmailee is not coerced to pay, his title transfer is involuntary and invalid due to ignorance. Likewise and for the same reason, blackmail which is not accompanied by the blackmailer's intention to execute his threats can be viewed as the one in which the blackmailer intentionally deceives the blackmailee about his mental state and induces him by this intentional misrepresentation to part with his property. This is fraud. Thus, even if blackmail cannot be subsumed under extortion, it can nonetheless be subsumed under fraud. As such, it is illegitimate even by Austro-libertarians' own lights and so should not be legalized.



\end{artengenv}

\label{dominiak-last}
\setcounter{secnumdepth}{1}





\title{Szablon-EN}

\begin{document}

On the philosophy and logic of human action: A~Neo-Austrian Contribution to the Methodology of the Social Sciences





Dr. Michael Oliva Córdoba



University of Hamburg



Department of Philosophy



Von-Melle-Park 6



20146 Hamburg



Germany



\href{mailto:michael.oliva-cordoba@uni-hamburg.de}{\textstyleInternetLink{\textit{michael.oliva-cordoba@uni-hamburg.de}}}



\url{https://orcid.org/0000-0001-7299-3070}



Abstract



Philosophical action theory seems to be in pretty good shape. The same may not be true for the study of human action in economics. Famous is the rant that the study of human action in economics gives reason to tremble for the reputation of the subject. But how does this come about? Since economic action is about action, the broader study must surely have a~strong impact on the more specific field. The paper sets out, from the ground up, how an essential concept in economic theory–the concept of competition–can fundamentally benefit from insights derived exclusively from analytical action theory broadly conceived. In doing so, the paper delivers on an old Austrian promise: it is sometimes claimed that Austrian economists understand competition better than most economists. This may be a~bold claim, since Austrian economists have neither traced the understanding of subjectivity to its very origin (the theory of intentionality), nor have they traced their sympathy for methodological individualism in relation to market processes to its very ground (the theory of (human) action). This paper aims to fill this gap. Moreover, by grounding an Austrian view of competition in analytic action theory, it succeeds in avoiding the serious problems of the dominant equilibrium approach. By explaining competition as rivalry, the paper draws on the philosophy and logic of human action to bring the (economic) agent back into play. In this way, a~case is made for an integrated view of Austrian theory as an amalgam of Austrian economics and analytic action theory.



Key words:



Competition • Rivalry • Equilibrium Theory • Action Theory • Subjectivism • Ludwig von Mises



\section{Introduction}

In the last century, much attention has been paid to the philosophy and logic of human action. Milestones in its development were Anscombe's \textit{Intention} 
%\label{ref:RNDwpmhr1p7kL}(1957),
\parencite*[][]{}, %
 Davidson's ``Actions, reasons and causes'' 
%\label{ref:RNDrARiCC0I40}(1963)
\parencite*[][]{} %
 and von Wright's \textit{Explanation and understanding} 
%\label{ref:RNDgx1JdbCQ0v}(1971).
\parencite*[][]{}. %
 Anscombe sought to highlight the knowledge basis that must be invoked when attributing an action to someone. Davidson defended the claim that action explanations are a~kind of causal explanations. Von Wright pointed out that explanations in history and the social sciences take very different forms. These studies arguably shaped the form of the philosophical discipline now known as action theory. They triggered a~multitude of philosophical contributions that eventually broadened the perspective on the philosophy and logic of human action to encompass approaches as diverse as critical reviews of ancient problems 
%\label{ref:RNDqWaAMGRskc}(such as the problem of weakness of will, cf., e.g., Mele, 2010; Walker, 1989; Davidson, 2001 [1970])
\parencites[such as the problem of weakness of will, cf., e.g.,][]{mele_weakness_2010}[][]{walker_problem_1989}[][]{davidson_actions_2001} %
 and contemporary concerns about normative aspects of reason-based approaches 
%\label{ref:RNDTM6NYuMMlD}(such as patient autonomy in medical ethics and related problems, cf., e.g., Zambrano, 2017; Flanigan, 2016; Jennings, 2009).
\parencites[such as patient autonomy in medical ethics and related problems, cf., e.g.,][]{zambrano_patient_2017}[][]{flanigan_obstetric_2016}[][]{jennings_agency_2009}. %
 Thus, the stream became a~river, and the river became an ocean. Today, there is no denying that action theory is in pretty good shape. Of course, there are controversies and difficulties in action theory, as in all other scientific disciplines. But there is a~solid consensus on the phenomena to be explained, there are paradigmatic theories that are referred to again and again, and there are classic contributions that offer points of contact for old insights and new debates. Although there are specialists in the field, philosophical action theory is by no means marginalised. Even theorists who do not specialise in action theory acknowledge its relevance for practical disciplines without hesitation. Philosophers of any provenance also usually have more than a~hunch that the relevance of action theory must somehow spill over into the social sciences themselves. And last but not least: Being a~philosopher of action is neither leftist, centrist or rightist. It has no hidden or obvious implications for your ideology, political and moral views or creed. So, it is safe to say that as a~scientific discipline, theory of action is a~decent, well-established and worthwhile subject to study.



In the social sciences, and especially in economics, this seems to be different. Apart from occasional lip service, the study of human action does not seem to have a~high priority in economics. This is especially true for \textit{praxeology}, the most comprehensive and complete economic approach towards the study of human action, which emerged from the Austrian school of economics. Praxeology has antedated philosophical action theory by about a~quarter of a~century. Unlike philosophical action theory, however, praxeology was not particularly well-received. One gets the impression that the study of praxeology is seen as a~trivial, partisan, dogmatic or shadowy endeavour. Some economists openly toy with the idea that praxeology is not a~scientific enterprise at all. The picture is emerging that the study of human agency in economics is considered to be a~serious threat to the respectability of economic theory. But how can the study of human action in economics, as \textit{Paul Samuelson} once put it 
%\label{ref:RNDPF4y5zZI10}(1964, p.736),
\parencite*[][p.736]{}, %
 give ``reason to tremble for the reputation'' of the subject, when economics is, as \textit{Alfred Marshall} 
%\label{ref:RNDkSaOWa77Q4}(1890, p.1)
\parencite*[][p.1]{marshall_principles_1890} %
 famously observed, ``a study of mankind in the ordinary business of life'' and a~study of ``individual and social action''?



The present paper is intended to help to resolve this tension and to make a~new attempt at justifying the importance that the study of human action can have for the social sciences and for economics in particular. This will be done by tracing economic problems, especially the problem of competition, back to their action-theoretical foundations. A~welcome side effect will be a~belated rehabilitation of the research programme that has unjustly brought Ludwig von Mises and the Austrian School of Economics into disrepute in the social sciences: If Mises' praxeology is ultimately interpreted as merely an early variant of what analytical action theory does in philosophy, then there is no reason to worry about the foundations of economics, quite the opposite.



\section{The need for a~better understanding}

The contrasting views of Marshall and Samuelson make it clear that something is fundamentally wrong with economists' understanding of the basics of their science. What has gone wrong? As always, the explanation is complex. I~can only hint at a~few elements. Certainly, the rise of socialism to scientific respectability in the early 20\textsuperscript{th} century played a~role. It raised hopes of the feasibility of a~supposedly superior system of objective central planning, freed from the arbitrariness of consideration for the individual. The same applies to the missionary impetus of the Vienna Circle. Even if some of its members liked to think they could keep their scientific work separate from their political goals, \textit{Otto Neurath} being a~prominent exception 
%\label{ref:RND3PGecJj6L9}(cf. Richardson, 2009, p.23; Carnap, 1963, p.23),
\parencites[cf.][p.23]{richardson_left_2009}[][p.23]{schilpp_intellectual_1963}, %
 the strong socialist undercurrent ensured a~remarkable anti-individualist tendency. Thus, the positivist view of science triggered by the Vienna Circle and the astonishing advances in the natural sciences led to a~view that was incompatible with a~subjectivist and individualist understanding of society and its sciences. No wonder many thinkers were tempted to align the social sciences with the natural sciences and mathematics. They still are today. The last element, but not the least, was the triumphant emergence of equilibrium theory. It gradually led to a~transformation of economic theory as a~whole. Contrary to its original intention, positive economics ultimately developed into a~normative enterprise. And as positivism, naturalism and normativism gained more and more influence, a~tendency towards objectivism seemed more and more inevitable.



All these issues have been discussed elsewhere. They have contributed significantly to the diminished importance of the study of human action in economics. Consequently, they led to the marginalisation of Austrian economics to the point where it was declared dead and mentioned only in historical retrospect. On this occasion, however, I~do not want to go into this research. The reason is that it is not entirely clear whether the study of action in economics is really best placed in the mainstream of the Austrian school of economics, at least in its present state. To be sure, there is no doubt that the Austrian school of economics openly professes \textit{subjectivism}, the central element in explaining human action. In the words of one of its most important representatives, \textit{Israel Kirzner}, ,,the Austrian school is usually and quite correctly identified with subjectivism. Subjectivism in economics means that Austrian economists are convinced that the regularities in economic life […] can be understood only by focusing analytical attention on individual actions`` of the human agent 
%\label{ref:RNDvEwaYbAVxW}(Kirzner, 2016, p.2:12).
\parencite[][p.2:12]{kirzner_history_2016}. %
 But this concession seems half-hearted in more ways than one.



First, what Kirzner calls the ``modern version of subjectivism'' aims to find a~middle ground between the ``flawed subjectivism of Menger'' and the ``nihilistic conclusions'' of the Shackle-Lachmann view 
%\label{ref:RNDUwquUHZxP1}(Kirzner, 1995, pp.14, 19; cf. Lachmann, 1982).
\parencites[][pp.14]{}[cf.][]{kirzner_ludwig_1982}. %
 This modern Austrian view thus rejects both Menger's ``heritage'' of perfect knowledge\footnote{
%\label{ref:RND18zwZr9hbq}(Kirzner, 1995, pp.14 \& 16)
\parencite[][pp.14 \& 16]{meijer_subjectivism_1995} %
 seems to assume such a~``legacy of perfect knowledge'' and I~will not dispute that: ``We have seen the central subjectivist thrust of Menger's vision. And we have seen the incompleteness of that vision (in its assumption of the normalcy of perfect knowledge.). […] We shall [steer] clear of […] the incompleteness in Menger's view (which led to the death of subjectivism in mainstream microeconomics).''} \textit{and} the idea of the radical spontaneity of choice. But while the first rejection is fully justified, the second is not. Denying the ``radical spontaneity of choice'' comes dangerously close to denying the essential autonomy of the agent. From the point of view of action theory, then, it remains a~mystery how the individual actions of the acting individual can be given the full weight they deserve without accepting much of what Kirzner calls ``nihilistic conclusions''. Therefore, one would really hope that ``Lachmann's influence on modern Austrian economics'' would be ``underappreciated'' and that his positions ``especially [on] subjectivism'' would be ``the dominant positions within the school'' 
%\label{ref:RNDVrtpYazpkd}(Storr, 2019, p.63).
\parencite[][p.63]{storr_ludwig_2019}. %
 Unfortunately, however, this may be an overly optimistic assessment.



Second, and more importantly, Kirzner's Austrian commitment to subjectivism underlines the importance of subjectivism in economics and the economisation of human action without really analysing subjectivism and human action in sufficient detail. Kirzner's ``modern Austrian subjectivism'' proceeds as if the \textit{differentia specifica} can be understood without understanding the \textit{genus proximum}. It treats subjectivism in economics and the economisation of human action as simple concepts whose meanings do not need to be broken down into their conceptual components. It seeks to redeem Ludwig von Mises' claim that economics is grounded in the theory of (human) action, but shies away from going beyond the boundaries of economic theory. And since it deals only with subjectivism in economics, it is silent on the nature of subjectivism itself.



Mises' assertion that economics is grounded in action theory was naturally quite disturbing to his fellow economists, even to some Austrians. One can understand why: reductive claims of this magnitude are rarely met with enthusiasm, especially by those whose discipline is subsumed under another. Consider the resistance that the positivist credo of the unity of the sciences met with in some natural sciences. Chemists and biologists usually pay lip service at best to the assumption that they are really concerned with physics. However, Mises' rallying cry found at least some support. The sociologist Alfred Schütz, a~long-time member of Mises' private seminar in Vienna 
%\label{ref:RNDaa9uzQdiX0}(Prendergast, 1986, p.5ff),
\parencite[][p.5ff]{prendergast_alfred_1986}, %
 echoed it: ‘\textit{All} social phenomena can be traced back to actions of agents in the social world, which in turn can be observed by social scientists' 
%\label{ref:RND58rPTjK0YJ}(Schütz, 1996, p.96; cf. 1953, p.26; Kurrild-Klitgaard, 2001, p.122).
\parencites[][p.96]{schutz_political_1996}[cf. 1953, p.26,][]{}[][p.122]{kurrild-klitgaard_rationality_2001}. %
 In order to give more substance to the claim that the social sciences, especially economics, are based on action theory, this paper will focus on the two aspects that have not yet received all the attention they deserve. We will focus on a~more general and thorough understanding of subjectivism and human agency. Subjectivism in economics and the economic aspects of human action will then emerge only as special cases.



It is clear that these investigations must be carried out independently of what they are later applied to. The impatient reader may therefore get the impression of a~somewhat lengthy diversions. However, since this is a~paper on proper foundations, there is no alternative to starting from scratch. Our reward will be a~picture of what the study of human action can contribute to the study of the social sciences. A~systematic and integrated approach will be outlined, showing what the philosophy and logic of human action can contribute to the social sciences at large and economics in particular. It will also show that it can contribute in this way without compromising the rigour, richness, and seriousness it deserves as the decent, well-established and worthwhile field of study that it is.



\section{The subjective and the objective: A~fundamental distinction}

\footnotetext{ For a~more detailed discussion of the following c\textit{f.} my forthcoming paper ``Subjectivity and objectivity. Intentional inexistence and the independence of the mind''.}

We can only understand subjectivism if we understand the subjective. Starting from scratch means going beyond economics and social sciences. Therefore, a~more fundamental science, \textit{i.e.}, philosophy, will be our guide. There, the distinction between the subjective and the objective has a~very long tradition. The terms go back to \textit{Aristotle's} Categories. In his translation, \textit{Boethius} 
%\label{ref:RNDL2rsHFWOjq}(cf. Minio-Paluello, 1961, 5:22; Aristotle, 1938, \textit{Cat.} 1a20)
\parencites[cf.][]{}[][]{aristotle_categories_1938} %
 uses the Latin word \textit{subiectum} as a~counterpart of the original Greek \textit{$\text{\textgreek{<u}}\pi o\kappa \varepsilon \text{\textgreek{'i}}\mu \varepsilon \nu o\nu $} (\textit{hypokeímenon}, the ``underlying thing''). However, our modern understanding of these terms dates back only to the early modern period. The distinction they mark as a~pair of opposites is usually described as a~kind of \textit{mind-(in)dependence}. As mathematician and logician \textit{Gottlob Frege} put it:



If we say ``The North Sea is 10,000 square miles in extent'' then neither by ``North Sea'' nor by ``10,000'' do we refer to any state of or process in our minds: on the contrary, we assert something quite objective, which is independent of our ideas and everything of the sort. 
%\label{ref:RNDXp0nrd47LI}(Frege, 1953, p.34)
\parencite[][p.34]{frege_foundations_1953}%




This understanding it is echoed time and again:



An element in some subject-matter conceptions of objectivity is \textit{mind independence}: an objective subject matter is a~subject matter that is constitutively mind-independent. […] By contrast, minds, beliefs, feelings, […] are not constitutively mind-independent, and hence not objective, in this sense 
%\label{ref:RNDf5sKcDXSwR}(Burge, 2010, p.46).
\parencite[][p.46]{burge_origins_2010}.%




So, according to the common view, the objective is objective insofar as it is independent of the mind, and the subjective is subjective insofar as it is not. But what exactly are the elements that make the subjective and the objective independent or dependent?



There are two paths open to us, the \textit{cognitive} and the \textit{attitudinal}. The cognitive way describes the element as a~\textit{perspective} or a~\textit{point of view}. To take a~subjective attitude towards something would be to look at it from a~particular perspective: the individual perspective of the subject. To take an objective attitude towards something would be not to look at it from a~particular perspective. In this way, it has become popular to distinguish the \textit{view from somewhere} against the \textit{view from nowhere} 
%\label{ref:RNDGi21Uuu7kQ}(cf. Nagel, 1979).
\parencite[cf.][]{nagel_subjective_1979}. %
 The most important metaphor of the cognitive path is the metaphor of the eye and what and how it sees. A~powerful metaphor indeed, but ultimately not a~very helpful metaphor: surely there can be no looking from nowhere. Therefore, we had better explore the other path, \textit{i.e.}, the path of attitude. In doing so, we implicitly acknowledge the importance of the \textit{intentional}. This is what the Austro-German philosopher \textit{Franz Brentano} considered to be the very characteristic of the mental 
%\label{ref:RND8DCAKIYKPd}(see, e.g., Crane, 1998; 2001; 2013).
\parencites[see, e.g.,][]{ohear_intentionality_1998}[][]{crane_elements_2001}[][]{crane_objects_2013}. %
 Brentano's much quoted illustration reads:



Every mental phenomenon is characterized by what the Scholastics of the Middle Ages called the \textit{intentional} (or mental) \textit{inexistence} of an object, and what we might call, though not wholly unambiguously, reference to a~content, direction toward an object (which is not to be understood here as meaning a~[real] thing), or immanent objectivity. Every mental phenomenon includes something as object within itself, although they do not all do so in the same way. In presentation something is presented, in judgement something is affirmed or denied, in love loved, in hate hated, in desire desired and so on. […] \textit{This intentional inexistence is characteristic exclusively of mental phenomena.} No physical phenomenon exhibits anything like it. We can, therefore, define mental phenomena by saying that they are those phenomena which contain an object intentionally within themselves. 
%\label{ref:RNDI7VRH284tM}(Brentano, 2009, p.68; orig. 1874; emphasis added)
\parencites[][p.68]{brentano_psychology_2009}[orig. 1874,][]{}[emphasis added,][]{}%




It is this passage where Brentano rediscovers the intentional. Eventually, this discovery led to the development of the \textit{theory of propositional attitudes}. This is because in natural language we are familiar with a~common feature that pretty much shows what Brentano took to be the defining feature of the mental: We recall that in natural language we very often attribute propositional attitudes to persons: We say, for example, that Tom \textit{believes} that the earth is flat, or that Dick \textit{wants} the man in the doorway to stop staring at him, or that little Harry \textit{hopes} that Father Christmas will come to visit next Christmas. Believing, wanting and hoping (and others) are \textit{propositional attitudes}; they are \textit{mental states} or \textit{events} attributed by reference to a~person experiencing the mental state or event and described by (the \textit{nominalisation} of) a~sentence within the \textit{scope} of an appropriate \textit{attitude verb}. That's a~lot of new vocabulary to learn, of course, but despite the new jargon it is not sophistry. It is a~natural feature of humans to have propositional attitudes, and it is a~natural feature of language that they can be expressed in natural language. Propositional attitudes are not sophisticated theoretical gimmicks, but part of the cognitive toolbox with which humans encounter the world. And, very importantly in this context, propositional attitudes have that very important feature of intentionality. This is the link to Brentano. For as the examples illustrate, someone can be in such a~state of mind that it can be correct to attribute a~certain propositional attitude to him, even if the object given in the attitude does not exist or is not as the subject imagines it. The earth is not flat, there is no Father Christmas, and sometimes we mistake a~reflection of ourselves for something or someone else. Nevertheless, Tom can believe that the earth is flat, Harry can hope that Father Christmas will come to visit next Christmas, and Dick can want the man at the door to stop staring at him. So, attitudes can have a~``real'' object, but they don't have to. You could say they provide an ``internal'' or ``intentional'' object. Or, as philosophers choose to express it, intentional objects are \textit{inexistent}, (propositional) attitudes display \textit{intentionality.}\footnotetext{ For the sake of simplicity, I~will refrain from adding ``propositional'' in the following where no misunderstandings are to be expected. In general, however, I~have no other attitudes in mind in this work than propositional ones. Moreover, for what could theoretically be called \textit{subpropositional} attitudes (like, \textit{e.g.}, \textit{making reference} to an object) I~would argue that these are only partial aspects in which one can regard ``fully-fledged'' or ``complete'' propositional attitudes and not a~separate category of attitudes in their own right.}





The intentional inexistence of objects and, by extension, that of attitudes is what best illustrates the attitude's intentionality 
%\label{ref:RND5TUDCgB8CW}(cf. Simons, 2009, p.xvi).
\parencite[cf.][p.xvi]{brentano_introduction_2009}. %
 It is also what constitutes the subjective. By providing an intentional object, attitudes bring out the subjective view of the individual who holds the attitude. Put differently: By describing attitudes, we describe \textit{the peculiar view} that Tom, Dick and Harry have of the earth, the man in the door and next Christmas. \textit{We are describing their subjective perspective.} Thus, we have an explanation of subjectivity that both makes the metaphor of the eye superfluous and is able to incorporate it: The cognitively subjective is subjective if and insofar as it is grounded in the attitudinally subjective. The mind-dependence that explains the subjective turns out to be a~dependence on the attitudes of the individual. The objective is thus objective because it is independent of the attitudes of the individual, and the subjective is subjective because it is not. So, all's well that ends well: The cognitive path leads to the attitudinal path, and the attitudinal path leads to the correct understanding of the matter.



In closing, let us illustrate the specificity of both subjectivity and individuality in a~more formal way. To do this, we use the basic language of modern attitudinal logic along the lines proposed in 
%\label{ref:RNDLzegm7vLXI}(Hintikka, 1962)
\parencite[][]{hintikka_knowledge_1962} %
 and explained, for example, in 
%\label{ref:RNDT3NCNO62Ji}(Ditmarsch et al., 2015, p.7).
\parencite[][p.7]{ditmarsch_handbook_2015}. %
 Let us extend it to apply to attitudes in general, using ``$\Delta $\textit{\textsubscript{x}}'' as a~proxy for any adequate form of an attitude operator, \textit{e.g}., ``B\textit{\textsubscript{x}}'' for ``\textit{x} believes that'', ``F\textit{\textsubscript{x}}'' for ``\textit{x} fears that'', and so on. Note that what ``$\Delta $\textit{\textsubscript{x}}'' is representative of involves the expression of an attitude subject and takes an indicative sentence as an argument (\textit{p}). We can now express that \textit{subjectivity} lies in the following fact of mutual non-entailment:



(Subjectivity)



(i) \textit{p} ${\nvdash}$ $\Delta $\textit{\textsubscript{x}} \textit{p}



(ii) $\Delta $\textit{\textsubscript{x}} \textit{p} ${\nvdash}$ \textit{p}



Thus, from the fact that Columbus discovered America (\textit{p}), it does not follow (${\nvdash}$) that he believed he discovered America (B\textit{\textsubscript{x}} \textit{p}). Nor does it follow (${\nvdash}$) from the fact that George VI did not want to follow his brother to the throne (W\textit{\textsubscript{x}} \textit{p}) that he did not follow his brother to the throne (\textit{p}). No special knowledge of early modern or modern history is needed to see this. It is already analytically contained in our understanding of behavioural verbs. By extension, we can characterise \textit{individuality} by the following fact of intrapersonal non-entailment (for \textit{x} ${\neq}$ \textit{y}, of course):



(Individuality)



(i) $\Delta $\textit{\textsubscript{x}} \textit{p} ${\nvdash}$ $\Delta $\textit{\textsubscript{y}} \textit{p}



(ii) $\Delta $\textit{\textsubscript{y}} \textit{p} ${\nvdash}$ $\Delta $\textit{\textsubscript{x}} \textit{p}



From the fact that Cleopatra (\textit{x}) feared being brought to Rome and paraded in the streets as part of Octavian's triumphal procession (F\textit{\textsubscript{x}} \textit{p}), it does not follow (${\nvdash}$) that Octavian (\textit{y}) feared bringing Cleopatra to Rome and parading her in the streets as part of his triumphal procession (F\textit{\textsubscript{y}} \textit{p}). Nor does it follow from the fact that Odysseus hoped that the Trojans would drag the wooden horse to their city (H\textit{\textsubscript{y}} \textit{p}) that Laocoon hoped this (H\textit{\textsubscript{x}} \textit{p}). Again, all that is required is a~proper understanding of the corresponding verbs. So, in the end, mind-independence amounts to mutual non-entailment.\footnote{In my ``Subjectivity and objectivity'' I~argue it is even stronger and comprises \textit{causal independence} as well.}



Let us summarise: One's attitudes are independent of both the world at stake and the attitudes of others. We happen to have stumbled upon the fact that the subjective-objective gap is, from a~certain point of view, simply the gap between mind and world. What is subjective is subjective because it depends on someone's attitudes. What is objective is objective because it does not depend on anyone's attitudes. Certainly, more could be said about the subjective, the objective and their distinction. But none of what has been said could be a~sound insight into the matter if it were not ultimately based on this. So basically, we have just based the subjective-objective distinction on the unique mental feature of intentionality, \textit{i.e}., intentional inexistence. We must leave it at that, however, because we need to move on quickly to the next topic, the topic of (human) action. To this I~turn now.



\section{Foundations of action theory}

\footnotetext{ For more detailed discussions, please refer to my book \textit{Analytical Action Theory, Fundamentals and Applications} [in German], forthcoming from Academia-Verlag, Baden-Baden.}

We have understood what the subjective is: it is what we understand to be dependent on a~person's attitude. Now we need to understand what action is. Our everyday talk about our actions will serve as a~guide. Using the long-established \textit{method of variation}, we can identify the underlying basic categories of action in what the average person would regard as accounts of action. This sort of corpus analysis is basically best practice among logicians, semanticists and linguists. They all use this method when defining basic categories via \textit{distribution}, even if they apply it to different domains 
%\label{ref:RNDYd59Ns4Wnw}(see, e.g., Burton-Roberts, 2016, p.46; Tallerman, 2015, p.34; Lewis, 1970, p.20ff; Lyons, 1968, p.147; Ajdukiewicz, 1935, p.3; Husserl, 1913, p.242; all anticipated by Frege, 1891; Engl. transl. 1960, p.189; and Plato, 1921 [Sophist 261d-262e]).
\parencites[see, e.g.,][p.46]{burton-roberts_analysing_2016}[][p.34]{tallerman_understanding_2015}[][p.20ff]{lewis_general_1970}[][p.147]{lyons_introduction_1968}[][p.3]{ajdukiewicz_syntaktische_1935}[][p.242]{husserl_prolegomena_1913}[all anticipated by][]{frege_function_1891}[p.189,][]{}[and][]{}.%




Our starting point is that accounts of action, when properly ordered, are \textit{substitution instances} of each other. This is true across contexts, styles and registers. So



(1) Peter eases the jib because he thinks that will stop the main from backing (and he wants it to)\footnote{Natural language is quite economical, \textit{cf.} 
%\label{ref:RNDfl51DVSlzk}(Davidson, 1963, 6f.):
\parencite[][]{}: %
 ``[I]t is generally otiose to mention both, If you tell me you are easing the jib because you think that will stop the main from backing, I~don't need to be told that you \textit{want} to stop the main from backing; and if you say you are biting your thumb at me because you want to insult me, there is no point in adding that you \textit{think} that by biting your thumb at me you will insult me'' (emphasis added).}



and



(2) Oedipus married Jocasta because he wanted to ascend the throne of Thebes (and thought he would if he did)



can be understood as resulting from each other by substitution \textit{salva congruitate}. That means that the substitution of an appropriate non-logical part of speech with a~categorically equivalent one does not transform an account of action into something that would not count as such. Of course, substituting ``Oedipus'' in (2) with ``Peter'' from (1) or ``wanted to ascend the throne of Thebes'' in (2) with ``wants to stop the main from backing'' in (1), \textit{etc.}, may turn a~correct action report into one that is most likely false. However, since we are not concerned with truth, but only with logical form, conceptual structure, and, ultimately, understanding, this difference does not matter. On the contrary, it gives us the \textit{canonical form of action reports} (A):



(A) \textit{x} \textit{$\varphi $-s} because \textit{x} wants that \textit{p} \& \textit{x} believes that \textit{x~$\varphi $-s} \ding{213} \textit{p}



This rendering now brings our logico-linguistic approach to fruition. For (A) manifests, understood distributively, the basic categories of action. We can thus distinguish the formal categories of \textit{agent}, \textit{doing}, \textit{wanting} and \textit{believing} in the following way: We take an \textit{agent} to be whatever is made reference to by an appropriate substitution instance \textit{salva congruitate} in the argument place indicated by [231C?]\textit{x}[231D?]; we take a~\textit{doing} to be whatever is made reference to by an appropriate substitution instance \textit{salva congruitate} in the argument place indicated [231C?]\textit{$\varphi $-s}[231D?]; and we proceed in the same way with regard to the remaining categories. When done correctly, we arrive at something closely resembling the classic Davidson \textit{belief desire model} of human action, where acting would be doing something for a~reason. This is the general model favoured by Anscombe 
%\label{ref:RNDw4miDnJrB8}(1957),
\parencite*[][]{anscombe_intention_1957}, %
 Davidson 
%\label{ref:RNDartTj7xlUB}(1963)
\parencite*[][]{} %
 and Wright 
%\label{ref:RND49dalMUE3q}(1971)
\parencite*[][]{} %
 in their respective versions, and it is probably fair to say that it is generally accepted nowadays. However, we arrive at our version of this model in a~purely formal, purely descriptive way, with the fewest possible theoretical presuppositions and without unwanted ballast. This spares us a~whole series of substantial and often controversial theoretical assumptions that are common in action theory today.\footnote{Is acting a~kind of doing and doing a~kind of bodily movement? But then how about \textit{mental actions}? 
%\label{ref:RNDi0JtkevpFW}(cf., e.g., O'Brien and Soteriou, 2009).
\parencite[cf., e.g.,][]{obrien_mental_2009}. %
 And are all doings extended in time? 
%\label{ref:RNDjrdhQNlYvw}(Frankfurt, 1978, p.158)
\parencite[][p.158]{frankfurt_problem_1978} %
 But then how about \textit{point actions} like, \textit{e.g.}, finishing a~paper or taking Mary to be your lawfully wedded wife? Other questions in this context would be whether there is a~causal sense of ``because'' that ensures that action explanations are causal explanations 
%\label{ref:RNDgpo0HfVbSy}(cf. Davidson, 1963)
\parencite[cf.][]{} %
 and, frankly, even whether the agents must necessarily be human beings. We need not go into all these thorny issues here: They only arise if one adds substantial assumptions to our minimalist explanation of action, which is not at all necessary at this point. }



The formal understanding we have arrived at also rewards us with a~formal understanding of what reasons for action (also known as ``motivating reasons'') are. Recall that it is common to call anything that starts with the connective ``because'' in response to a~``why?'' question a~\textit{reason}. Why is four even? \textit{Because it is divisible by two}. Why did the dinosaurs become extinct? \textit{Because the Chicxulub asteroid hit the Gulf of Mexico some 65 million years ago}. In relation to (1) and (2), the reason for Peter's manœuvre and Oedipus' marriage to Jocasta is what is given in response to a~corresponding question in (1) and (2) respectively. Reasons for action are thus hybrid. They are given by the combination of two particular attitudes: Peter's (or Oedipus') \textit{wanting} that \textit{p} in conjunction with his (or Oedipus') \textit{believing} that if he \textit{$\varphi $-}s, then \textit{p}.



This fits in well with our findings from the previous section. Given that reasons for action are described by complex attitudes, it is clear that motivating reasons are (i) \textit{subjective} and (ii) \textit{individual} in the following ways: (i) someone's reason is neither implied nor otherwise determined by how things are, nor does it imply or otherwise determine how things are; (ii) a~reason for one need not be a~reason for the other. Moreover, because of the intentionality of attitudes, the reason of the agent can, but need not, collide with reality. It can lead to failure. But that is just grist to our mill because, surely, an unsuccessful action is still an action. On closer examination, this raises an even more interesting question: If the reasons for action must necessarily be seen as subjective and individual, what about the talk of objective reasons that is so prominent today? Indeed, the essential subjectivity and individuality of motivation bears a~striking resemblance to the \textit{eye of the needle} in Matthew 19:24: ``And again I~say unto you, It is easier for a~camel to go through the eye of a~needle, than for a~rich man to enter into the kingdom of God.'' Since objective reasons are neither necessary nor sufficient for an agent's actions, but his subjective reasons are, it seems that objective reasons are like the rich man in the Gospel. Like him, who would have to divest himself of his wealth in order to enter the kingdom of God, objective reasons would have to divest themselves of their objectivity and instead become subjective in order to truly motivate. Thus, in order to truly explain human action, one cannot ultimately abstract from the individual agent and his subjective reasons.



If we take stock now, we see that to act is to do something for a~reason. For most people this is just a~platitude. But the way we have derived it has unlocked the foundations of action theory. And since we started from scratch, we now know exactly the theoretical presuppositions we encountered. In particular, we see that in our approach they are minimal and purely descriptive. Interestingly, human action is also seen as necessarily subjective and individual in the Austrian School of Economics. This is what Austrian subjectivism boils down to, or at least it should be based on. Our brief examination of the philosophy and logic of human action, however, was conducted independently of any economic and social science presuppositions. Frankly, it was independent of all questions of practical disciplines, including moral philosophy, political theory, law and economics and so on. Our subjectivism is thus based on nothing other than a~fundamental understanding of intentionality and a~distributional analysis of action reports. As a~result, it is much more comprehensive than the surrogate discussed in economic methodology or the social sciences at large. Subjectivism in economics or the social sciences now appears only as a~special case.



It should be noted in passing that those branches of philosophy that usually come into play when economists discuss the foundations of their subject\textit{, i.e.}, Kantianism, positivism, sometimes even phenomenology or hermeneutics, were neither necessary nor helpful. To dispel a~common misunderstanding about the ultimate foundation of economic science, it must also be pointed out that our enquiry was by no means \textit{epistemological} either 
%\label{ref:RNDtyxHyOgZWT}(\textit{pace} Mises, 1962).
\parencite*[][]{}. %
 Thus, since the ultimate foundation of economic science is the philosophy and logic of human action---just as (young) Mises rightly said, Austrians should like to assume, and as was demonstrated in the previous reasoning---what (old) Mises arrived at at the \textit{end} of his intellectual development, namely that these ultimate foundations were epistemological, cannot also be true. It is not: Action theory is \textit{not} epistemology; it has nothing essential in common with it. To assume otherwise is simply to commit an error in judgement.\footnote{Unfortunately, it will hardly help to make the Kantian point that ``in some sense epistemology is the basis of all the sciences''. At the end of the day, that is quite a~strong statement. It presupposes its own truth and, lamentably, proves nothing. Following 
%\label{ref:RNDaNem1NNmgA}(Fumerton, 2017, p.3)
\parencite[][p.3]{fumerton_epistemology_2017} %
 one could complain that proponents of such a~view are ``simply trying to legislate a~meaning for the term ‘[science]' a~meaning that has little bearing on how the term is actually used.'' \textit{Kant}, of course, thought otherwise. But the history of philosophy has not been kind to this kind of epistemological imperialism. Kantian idealism is in part excused, however, since Kant planted his flag well before the advent of modern-day logic and formal semantics. But it is fair to say that advances in philosophical reasoning, particularly in logic and semantics, have ensured that the idealist stance in philosophy has not aged well. It may well be that the present foundational stance in philosophy, adopted by the Vienna Circle and acknowledged by Fumerton 
%\label{ref:RNDznyFyvfiQv}(2017, p.14),
\parencite*[][p.14]{fumerton_epistemology_2017}, %
 is an exaggeration too: ``All philosophy is a~‘critique of language''' 
%\label{ref:RNDpQECsk5Coz}(Wittgenstein, 1922; 2013, 4.0031).
\parencites[][]{wittgenstein_tractatus_1922}[][]{}. %
 But surely, \textit{that} is a~different kind of exaggeration. One that places logic and semantics at the heart of science. Not epistemology.} This should be a~serious warning to all those Austrians who are in the habit of saying that there is an epistemological problem at the core of economics 
%\label{ref:RNDa61qO57Q7R}(cf. Condic and Morefield, 2021; Rajagopalan and Rizzo, 2019, p.94; Knudsen, 2004; Yeager, 1994; Ebeling, 1993, p.63f.; Boettke, 1990, p.23ff.; Lavoie, 2015, p.50; Hayek, 1945; 1948, p.33; Schütz, 1996, 98f).
\parencites[cf.][]{condic_hayek_2021}[][p.94]{rajagopalan_austrian_2019}[][]{knudsen_alfred_2004}[][]{yeager_mises_1994}[][p.63f.]{herbener_economic_1993}[][p.23ff.]{boettke_political_1990}[][p.50]{lavoie_rivalry_2015}[][]{hayek_use_1945}[][p.33]{hayek_economics_1948}[][]{}. %
 More importantly, however, we have seen the sketch of a~sound and solid philosophical basis for the study of human action. As the study of human action in economics is quite often accused of resting ``on a~weak philosophical foundation'' 
%\label{ref:RNDMbwknyPHli}(cf. Barrotta, 1996, p.65),
\parencite[cf.][p.65]{barrotta_neo-kantian_1996}, %
 it is almost vital to be able to show that we are not like the foolish man in Matthew 7:27 who built his house on sand, and it rained, and the flood came, and the winds blew and beat against the house, and it fell. And that is precisely what we have shown.



\section{The study of human action in economics}

We have now acquired a~sufficiently thorough and solid understanding of the concept of human agency and the phenomenon of subjectivity. If economics is really a~part of the ``study of mankind in the ordinary business of life'' and an examination ``of individual and social action'' (Marshall 1890, 1), we should expect these insights to bear fruit in relation to essential economic questions. In fact, first steps in this direction have already been taken when, with the help of analytic action theory, it was shown that two cornerstones of praxeology, the \textit{Uneasiness Theorem} and the \textit{Scarcity Theorem}, are analytic, hence not synthetic, but nevertheless a~priori 
%\label{ref:RNDM3l39q66MF}(Oliva Córdoba, 2017).
\parencite[][]{oliva_cordoba_uneasiness_2017}. %
 The Uneasiness Theorem, which states that the incentive to act is always uneasiness 
%\label{ref:RNDFSRgIQxEo3}(Mises, 1998, p.13),
\parencite[][p.13]{mises_human_1998}, %
 and the Scarcity Theorem, which states that action is the manifestation of scarcity 
%\label{ref:RNDWK5Y8nAnaZ}(Mises, 1998, p.70),
\parencite[][p.70]{mises_human_1998}, %
 are at the centre of Mises' programme to ground economic theory in action theory. Given the controversial nature of this programme even among Austrian economists, it seems that this justification of the proper study of human action in economics was far too subtle to leave a~more lasting impression on economists. But, as the saying goes, a~house is built by wisdom and erected by understanding; fools tear it down with impatience. Having demonstrated the purity and soundness of its foundations, we can now take the study of human action in economics a~step further and address a~subject that must certainly be classified as essential in both theory and practice: the problem of competition.



Competition is both an ancient phenomenon and a~central concept in economics. With the increasing importance of welfare economics for policy advice, competition has acquired an increasingly important role as the main criterion for assessing the so-called efficiency of actual markets 
%\label{ref:RNDvbNc3Twdn2}(e.g., Motta, 2004; Armentano, 1972, p.31ff.).
\parencites[e.g.,][]{motta_competition_2004}[][p.31ff.]{armentano_myths_1972}. %
 This importance stands in stark contrast to the still inadequate understanding of the phenomenon and the insufficient understanding of the concept. It is true that the development of the theory of perfect competition, a~centrepiece of general equilibrium theory,\footnote{Thanks to an anonymous reviewer for pointing out that this is true only relative to the nature of your approach to equilibrium theory. Thus, while the starting point of Walras 1874–1877/1896---entirely in line with his conviction that ``economic theory is essentially the theory of the determination of prices in a~hypothetical regime of perfectly free competition'' 
%\label{ref:RNDWn1v8aNxVK}(Walras, 2019, VIII)
\parencite[][]{walras_lewalras_2019}%
---, and also the classic works 
%\label{ref:RNDd1ZqS9HHR1}(Edgeworth, 1881; Marshall, 1890; Arrow and Debreu, 1954; McKenzie, 1954)
\parencites[][]{edgeworth_mathematical_1881}[][]{marshall_principles_1890}[][]{arrow_existence_1954}[][]{mckenzie_equilibrium_1954} %
 do seem to make this essential connection, this is less obvious in the case of, say, 
%\label{ref:RNDAgpWJk7EqI}(Wald, 1935; Samuelson, 1947; Mas-Colell, 1974).
\parencites[][]{wald_uber_1935}[][]{samuelson_foundations_1947}[][]{mas-colell_equilibrium_1974}. %
 \textit{Cf.} also 
%\label{ref:RNDaO6Djas59W}(McKenzie, 1981; Weintraub, 2011).
\parencites[][]{mckenzie_classical_1981}[][]{weintraub_retrospectives_2011}. %
 Nevertheless, some importance must be attached to the fact that nowadays there still seems to be a~widespread belief ``that GE theory describes with sufficient approximation the result of the unfettered working of competitive markets'' 
%\label{ref:RNDflMWzgksHc}(Petri and Hahn, 2003, p.8).
\parencite[][p.8]{petri_general_2003}.%
} was hoped to improve understanding; and today's mainstream economic theory seems more or less satisfied on this issue. However, as we shall see in a~moment, there are even more serious difficulties with the equilibrium approach to competition, precisely because it aims to explain competition in terms of that perfectly realised market structure it describes.



This market structure is criticised even within mainstream economic theory 
%\label{ref:RND39XlTt3fQn}(cf., e.g., Ackerman and Nadal, 2004; Petri and Hahn, 2003).
\parencites[cf., e.g.,][]{ackerman_flawed_2004}[][]{petri_general_2003}. %
 Completely unimpressed, however, economics textbooks reiterate \textit{ad nauseam}, that it exists when (i) the number of suppliers is very large and (ii) the goods traded are homogeneous 
%\label{ref:RNDVRCibbzDF0}(see, e.g., Mankiw, 2020, p.62).
\parencite[see, e.g.,][p.62]{mankiw_principles_2020}. %
 As a~rule, the requirements are also added, at least implicitly, that in a~perfectly competitive market (iii) transaction costs or other obstacles to free and direct exchange and (iv) knowledge differences between market participants are negligible. These provisions are intended to ensure that under conditions of perfect competition sellers have no influence on market prices and thus take prices as given. Perfect competition, according to mainstream textbooks, ``is the world of price takers'' 
%\label{ref:RNDxIXBN3eF1z}(Samuelson and Nordhaus, 2009, p.150).
\parencite[][p.150]{samuelson_economics_2009}. %
 However, from a~more general point of view, since there is no clear distinction between buyers and sellers, there is no difference in principle between the case in which Dick trades his goat for Tom's sheep and the case in which he trades it for Tom's \$40. Consequently, we cannot say in principle who is the buyer and who is the seller apart from saying that both are both:



The buyer of a~thing is the seller of what he gives in exchange. The seller of a~thing is the buyer of what he receives in exchange for it. In other words, every exchange of two things, one for the other, is composed of a~double purchase and a~double sale. 
%\label{ref:RNDaHrTjQRQwA}(Walras, 2019, p.42 [orig. 1896])
\parencite[][p.42 [orig. 1896]]{walras_lewalras_2019}%




To simplify matters, what we can say is that both Dick and Tom are economic subjects, individual participants in the economy, or, if you will, \textit{traders}. So, the idea of a~world of price-takers has to be formulated more generally. What the perfect competition provisions are really meant to ensure is ``the fundamental competitive assumption that agents cannot influence market prices'' 
%\label{ref:RND7Y3KfXIxK6}(Safra, 1989, p.225; cf. Khan, 2008).
\parencites[][p.225]{safra_strategic_1989}[cf.][]{palgrave_macmillan_perfect_2008}. %
 The economist's basic perspective is thus to ensure that ``the influence of an individual participant on the economy […] be mathematically negligible'' 
%\label{ref:RNDSJdd6Ir68W}(Aumann, 1964, p.39).
\parencite[][p.39]{aumann_markets_1964}. %
 This can best be achieved, as Aumann has shown, by representing the ideal infinity of economic agents as a~single continuum. Since the circumstances in which individual economic agents are economically negligible are precisely the circumstances in which they are numerically negligible 
%\label{ref:RNDdiE0hhlLXI}(Bryant, 2010, p.332),
\parencite[][p.332]{bryant_general_2010}, %
 this formally amounts to the introduction of a~single entity, \textit{the all-trader}, as the single unit of economic exchange. The assumption that traded goods are homogeneous also serves a~similar function. It abstracts from the differences between goods, so it is about product differentiation. It is assumed that under perfect competition it makes no significant difference whether the traded goods are, for example, slightly heavier or smell slightly different: ``A perfectly competitive [trader] sells a~homogeneous product (one identical to the product sold by others in the industry)'' 
%\label{ref:RNDcc8FxLpUuj}(Samuelson and Nordhaus, 2009, p.150).
\parencite[][p.150]{samuelson_economics_2009}. %
 The homogeneity assumption on the side of the goods and the continuum assumption regarding traders are thus two sides of the same coin: both serve the purpose of mathematical integration. They are supported in this by the third stipulation that there are no transaction costs or other obstacles to free and immediate exchange. This ensures the uniqueness of the allocation. Thus, from a~\textit{logical} point of view (and this analysis is \textit{not} anticipated by economists) the following picture emerges: In perfect competition \textit{the all-trader is uniquely mapped onto the all-good}. The fact that the all-trader then also knows everything there is to know is only a~trivial consequence. The triviality of perfect knowledge. So now we are almost in a~position to understand what is deeply problematic about the equilibrium picture of perfect competition. It is not primarily what Friedman intended to defend, namely the lack of realism of the assumptions 
%\label{ref:RNDULGGYHxb8s}(Friedman, 1966),
\parencite[][]{friedman_methodology_1966}, %
 although the assumptions are of course very strong and highly unrealistic. Also, it is not what economists usually criticise from within economic theory 
%\label{ref:RND15H1gooK6T}(cf., e.g., Ackerman and Nadal, 2004; Petri and Hahn, 2003),
\parencites[cf., e.g.,][]{ackerman_flawed_2004}[][]{petri_general_2003}, %
 although these are often points of criticism that very much deserve attention. What really speaks against this picture is ultimately something else.



To see this more clearly, we first need to look at the standard response that is used to dismiss all inconsistencies that arise from the picture of perfect competition. Inconsistencies with real markets and real competition are usually answered by saying that perfect competition is only an ideal. For example, perfect competition is routinely compared to the idea of frictionless surfaces 
%\label{ref:RNDuiB9zgmW3r}(Samuelson, 1947; Friedman, 1966 [1953]; Aumann, 1964; Khan, 2008, \textit{etc.}).
\parencites[][]{samuelson_foundations_1947}[][]{friedman_methodology_1966}[][]{aumann_markets_1964}[][]{palgrave_macmillan_perfect_2008}. %
 The argument goes something like this: Frictionless surfaces cannot exist, but progress towards this ideal helps to reduce friction on real surfaces. This is what makes frictionless surfaces an ideal in the first place. In the case of perfect competition, unfortunately, the opposite is true. Here, every step towards perfection contributes to a~reduction in competition. Take (product) differentiation, for example. Decried in applied equilibrium theory as an unfair barrier to entry to the detriment of pure competition, in real life it is more a~function of consumer acceptance. In an effort to secure business, every supplier or producer will try to attract consumers to his product or service. He will strive to make his product or service as unique from the point of view of his potential customers as they will honour by buying it. As competition increases, we will therefore expect more rather than less differentiation. If need be, not in the product itself, but in the service, in the transaction costs or elsewhere in the economic sphere: ``In a~\textit{free} market individualism is to be expected on the part of the consumers and firms; the goods produced, therefore, will be differentiated to the extent and degree that consumers reward differentiation'' 
%\label{ref:RND7m63Rpm27u}(Armentano, 1972, p.33).
\parencite[][p.33]{armentano_myths_1972}. %
 Differentiation, \textit{i.e.}, making a~difference, is of the very essence of real competition. Remove this feature, abstract from all remaining differences, and what you are looking at is really something else. Seen in the light of day, then, the idea of perfect competition is not at all an ideal that enhances competition or that gives us a~better understanding of it, but quite the opposite. It is a~false, mock or anti-ideal. The pursuit of this ideal leads to a~gradual elimination of competition to the point where there is none at all. The idea of perfect competition thus tempts us to misunderstand the nature of competition. Instead, it paints an irretrievably distorted picture. Perhaps the most charitable thing to say would be that perfect competition is about perfection, not competition. A~perfection that is admittedly neither achievable nor desirable in the real world. A~perfection that is guaranteed by successive steps of logical abstraction. But that is precisely what has got us into trouble.



The logical analysis we have arrived at ultimately reveals the following: we are dealing with a~neat mathematical representation of a~quasi-Parmenidean idea of an almost all-encompassing monism: \textit{The all-trader is uniquely mapped onto the all-good.} No wonder there is neither change nor waste in such a~metaphysical picture. As a~result, there is Pareto optimality and even a~Nash equilibrium, great. But this is merely due to stipulation. A~nice little sleight of hand. And look what it costs: There is no competition either. That is why the immense intellectual effort invested in this idea has always led to resistance. What has not been taken into account, and what could instead help us to better understand competition, is the individual economic agent with all his subjective attitudes. It is to him that we must turn next.



\section{Competition as rivalry}

The idea of pure competition arose in an effort to understand more precisely the ultimate ground of truth of two very popular and plausible classical theses. One was \textit{Adam Smith's} assertion that the greater the number of sellers, the lower the price 
%\label{ref:RND2wfOaQk58A}(Smith, 1776, pp.68–69),
\parencite[][pp.68–69]{smith_inquiry_1776}, %
 the other was \textit{John Stuart Mill's} assumption that there can be only one price in the market 
%\label{ref:RNDncWBEiqSuh}(Mill, 1848, p.291).
\parencite[][p.291]{mill_principles_1848}. %
 The aim of the fathers of general equilibrium theory was to prove these assumptions truisms in a~mathematically convenient way. The imprecise understanding that economists sought to refine (and eventually inadvertently replaced) related to the behaviour of people: ``‘Competition' entered economics from common discourse, and for long it connoted only the independent rivalry of two or more persons'' 
%\label{ref:RNDIpkeBGOokO}(Stigler, 1957, p.1).
\parencite[][p.1]{stigler_perfect_1957}. %
 Today, when the economic mainstream understands competition almost exclusively in terms of perfect competition, the original understanding of competition as rivalry is nevertheless taken for granted. It is consistently implicit in mainstream textbooks 
%\label{ref:RNDfJqUvg9oJD}(cf. Acemoglu, Laibson and List, 2016, p.357; Pindyck and Rubinfeld, 2013, p.281 et passim; Samuelson and Nordhaus, 2009, 172f. Stiglitz and Walsh, 2006, p.241 et passim; among others).
\parencites[cf.][p.357]{acemoglu_microeconomics_2016}[][p.281 et passim]{pindyck_microeconomics_2013}[][p.241 et passim]{}[among others,][]{}. %
 Sometimes it is also stated very clearly: ``Competition exists when two or more firms are rivals for customers'' 
%\label{ref:RNDGhoH3v9Jju}(Mankiw and Taylor, 2014, p.42).
\parencite[][p.42]{mankiw_economics_2014}.%




Underlying all these characterisations is the concession that competition is essentially due to human behaviour. However, the concept of competition as rivalry is then usually explained from equilibrium theory and not the other way around. In contrast to the economic mainstream, the Austrian School of Economics has long recognised that this reverse order of explanation puts the cart before the horse. In his \textit{Rivalry and central planning}, Austrian economist \textit{Don Lavoie} argued that the information function of rivalry is fundamental to understanding the market process. ``Markets are inherently rivalrous, […] they work only as a~consequence of a~competitive struggle among incompatible plans'' 
%\label{ref:RNDHLp0gR5MoM}(Lavoie, 2015, p.180 [orig. 1985]).
\parencite[][p.180 [orig. 1985]]{lavoie_rivalry_2015}. %
 But like other Austrian approaches, Lavoie's account is full of strong assumptions and, more importantly, it does not provide us with an action-theoretic explanation either. Rather, we are offered an inherently economic explanation that invokes assumed ``market forces''. This explanation may or may not be plausible, but it is certainly not fundamental in the sense we are exploring in this paper. So how can we make sense of the idea that competition is essentially rivalry without introducing strong assumptions or economic presuppositions on our part? This is where the minimalist philosophy and the logic of action outlined in the first two sections will make the difference.



We will use (and have already been using) a~simplified, slightly extended variant of first-order predicate logic with logical connectives, variables and the usual quantifiers. Connectives are ``¬'', ``\&'', ``v'', ``\ding{213}'', and ``[27F7?]'', which correspond to their natural language equivalents ``not'', ``and'', ``or'', ``if ... then'', and ``if and only if … then …''. Standard single variables are ``\textit{x}'', ``\textit{y}'', ``\textit{z}'', \textit{etc.}, which can be replaced by proper names (or expressions of the same logical type) such as ``Tom'', ``Dick'' and ``Harry''. Standard variables that take a~predicate position are ``\textit{$\varphi $}'', ``\textit{$\psi $}'', ``\textit{$\chi $}'', \textit{etc.}, which can be replaced by predicates such as ``sleeps'', ``dropped out of high school'' and ``will join the military''. Standard propositional variables are ``\textit{p}'', ``\textit{q}'', ``\textit{r}'', \textit{etc.}, which can be replaced by full declarative sentences such as ``Tom will join the military'', ``Dick is asleep'' and ``Harry dropped out of high school''. The essential point about variables is that they can be bound, thus there are the quantifiers ``${\exists}$'' and ``${\forall}$'', the latter symbol often omitted, which correspond to their natural language equivalents ``at least one (is such that)'' and ``all (are such that)'', so that we can render formulae like ``(${\exists}$\textit{x}) (\textit{x} is asleep)'' as approximately ``Someone is asleep'' or ``(\textit{x}) (${\exists}$\textit{$\varphi $}) (\textit{$\varphi $x})'' as approximately ``Everyone is somehow'' or ``(\textit{p}) (Harry says that \textit{p} \ding{213} \textit{p})'' as approximately ``Everything is as Harry says''. The final step, already introduced in Section 2 above, is the addition of attitude operators ``B\textit{\textsubscript{x}}'', ``W\textit{\textsubscript{x}}'', ``F\textit{\textsubscript{x}}'' and ``H\textit{\textsubscript{x}}'', which correspond to their natural language equivalents ``\textit{x} believes that'', ``\textit{x} wants that'', ``\textit{x} fears that'', and ``\textit{x} hopes that'', so that we can reproduce formulae such as ``B\textit{\textsubscript{x}} \textit{r}'' which can be expanded to ``Tom believes that Harry dropped out of high school'', ``W\textit{\textsubscript{y}} \textit{q}'' to ``Harry wants Dick to sleep'', ``F\textit{\textsubscript{x}} \textit{p}'' to ``Tom fears that nothing is as Harry says'', and ``H\textit{\textsubscript{z}} \textit{r}'' to ``Harry hopes that someone will join the military''. So much for a~brief sketch of the apparatus involved.\footnote{Should readers miss an easily accessible introduction to logic at this point, I~refer them to the classic Lemmon 
%\label{ref:RNDlnC2fBZ9lD}(1965),
\parencite*[][]{lemmon_begining_1965}, %
 for example.}



The next step is to imagine a~simple exchange, such as Dick trading his goat for Tom's sheep. This involves at least the following:



(a) Tom gives Dick his sheep



(b) Dick gives Tom his goat



(c) Tom wants Dick to give him his goat



(d) Dick wants Tom to give him his sheep



(e) Tom thinks that if he gives Dick his sheep, Dick will give Tom his goat



(f) Dick thinks that if he gives Tom his goat, Tom will give him his sheep.



But that is not all. Tom gives Dick his sheep and Dick gives Tom his goat \textit{because} they want what they want and believe what they believe:



(TD*)(a) \& (b) because ((c) \& (d)) \& ((e)\&(f)).



So, we have a~case of intertwined, one could also say \textit{reciprocal}, action, for the above is nothing but a~notational variant for a~plural case of our familiar canonical form of action-reports (A):



(A) \textit{x} \textit{$\varphi $-s} because \textit{x} wants that \textit{p} \& \textit{x} believes that \textit{x~$\varphi $-s} \ding{213} \textit{p}



According to simple formal language described, Tom and Dick's exchange would have to be rendered more perspicuously as follows:



(TD) \textit{$\varphi $xy} \& \textit{$\psi $yx} because W\textit{\textsubscript{x}} \textit{$\psi $yx} \& W\textit{\textsubscript{y}} \textit{$\varphi $xy} \& B\textit{\textsubscript{x}} (\textit{$\varphi $xy} \ding{213} \textit{$\psi $yx}) \& B\textit{\textsubscript{y}} (\textit{$\psi $yx} \ding{213} \textit{$\varphi $xy})



with ``\textit{$\varphi $}'' = ``gives his sheep to'', ``\textit{x}'' = ``Tom'', ``\textit{y}'' = ``Dick'', ``\textit{$\psi $}'' = ``gives his goat to'', ``W\textit{\textsubscript{x}}'' = ``Tom wants that'', ``W\textit{\textsubscript{y}}'' = ``Dick wants that'', ``B\textit{\textsubscript{x}}'' = ``Tom believes that'', and ``B\textit{\textsubscript{y}}'' = ``Dick believes that''. I~will admit that this may look a~bit cryptic indeed. But remember that this is only applying the previously explained and innocuous stipulations. (TD) may be complex, but it is not complicated. Note also that (TD) is just an action-theoretic account of a~reciprocal doing, a~rendering of what sometimes is referred to by the Latin phrase \textit{do ut des}. There is nothing particularly economic about it, or to put it another way, an economic exchange would be nothing but a~special case of (TD).



Now the rivalry only comes into play when we add another participant to the scene. So, let's imagine a~different situation. Tom is still willing to trade with Dick, but now we are counting on another possible trader, Harry. Nothing has happened yet, but in this alternative situation it is conceivable that Tom will trade his sheep for Harry's llama. In strict analogy to (TD), but with suitable substitutions, this would yield (TH):



(TH) \textit{$\varphi $xz} \& \textit{$\psi $zx} because W\textit{\textsubscript{x}} \textit{$\psi $zx} \& W\textit{\textsubscript{z}} \textit{$\varphi $xz} \& B\textit{\textsubscript{x}} (\textit{$\varphi $xz} \ding{213} \textit{$\psi $zx}) \& B\textit{\textsubscript{z}} (\textit{$\psi $zx} \ding{213} \textit{$\varphi $xz})



In order to give an action-theoretic explanation of rivalry, we need to put these parts together in the right way. The essential step we need to add comes from the theory of intentionality: we need to take into account the attitudes Dick and Harry have towards the possibilities (TD) and (TH). This is what makes them rivals in the first place.



The realisation that the introduction of an intentional element is essential to explaining rivalry is almost a~truism. What causes two runners to be in a~race with each other is not that they are moving fast in the same direction. So many people do that every day. Rather, it is the fact that one wants to outdo the other. So, of course, they have to have a~certain attitude towards each other. This introduces an intentional, \textit{i.e.}, subjective, characteristic as an essential element. Since the role of human beings in general equilibrium theory is not really different from the role of ``atoms of the rare gas in my balloon'' 
%\label{ref:RND5O7KkbLpz1}(Samuelson, 1966, p.1411),
\parencite[][p.1411]{samuelson_modern_1966}, %
 we cannot be surprised that this essential element of competition must be absent from the equilibrium picture of perfect competition. However, with the help of the philosophy and logic of human action, it is not difficult to reinsert this element. The essential step is that Dick \textit{hopes} to make the deal but \textit{fears} that Harry might make it instead, and vice versa. This means that they see each other as rivals, and that if they act accordingly, they will be rivals. So, the next step is to establish that if and only if



(PR) H\textit{\textsubscript{y}} (TD) \& F\textit{\textsubscript{y}} (TH) \& H\textit{\textsubscript{z}} (TH) \& F\textit{\textsubscript{z}} (TD)



Dick and Harry \textit{perceive} each other as rivals. They \textit{are} rivals if and only if they act on this perception:



(AR) \textit{$\gamma $y} because W\textit{\textsubscript{y}} ((TD) \& ¬ (TH)) \& B\textit{\textsubscript{y}} \textit{$\gamma $y} \ding{213} ((TD) \& ¬ (TH))



\&



\textit{$\lambda $z} because W\textit{\textsubscript{z}} ((TH) \& ¬ (TD)) \& B\textit{\textsubscript{z}} \textit{$\lambda $z} \ding{213} ((TH) \& ¬ (TD))



where ``\textit{$\gamma $}'' and ``\textit{$\lambda $}'' are representative of what Dick and Harry do to outdo the other. What might that be? Well, Dick might offer Tom a~discount or some other perk, Harry might offer Tom special trade relations or immediate delivery. If this is what they do to secure the deal (and prevent the other from making it), this is their respective rivalrous behaviour. For each agent that involves an individual complex attitude, though. But through simple conjunction elimination in (PR) and (AR) we can uncover the subjective and individual perspective of the respective agent:



(PR\textit{\textsubscript{y}}) H\textit{\textsubscript{y}} (TD) \& F\textit{\textsubscript{y}} (TH)



and



(AR\textit{\textsubscript{y}}) \textit{$\gamma $y} because W\textit{\textsubscript{y}} ((TD) \& ¬ (TH))



such that now we can describe his rivalrous behaviour:



(R)\textit{y} acts \textit{rivalrously} [27F7?] (AR\textit{\textsubscript{y}}) because (PR\textit{\textsubscript{y}})



Rivalry, thus, is when an agent acts rivalrously because he perceives another to be a~rival. And, lo and behold, I~hear some people scoff and say that this is exactly what we had to hear from the philosophers. But anyone who reacts in this way misses an important, indeed crucial, point: in any serious scientific discussion, success is not measured by the conclusion you reach, but by \textit{the way you derive it}. This is precisely the reason why we talk about the scientific \textit{method}. Science without method is not science. It may well be that authors like \textit{Sebastian De Haro} are right and that the interaction between the empirical, natural and social sciences on the one hand and philosophy on the other is characterised by a~kind of ``love-hate relationship'' 
%\label{ref:RNDc0Q1y7MyqS}(De Haro, 2020).
\parencite[][]{de_haro_science_2020}. %
 Nevertheless, the ``analytical function of philosophy'' 
%\label{ref:RNDRLzch5O5FC}(De Haro, 2020, p.304f.)
\parencite[][p.304f.]{de_haro_science_2020} %
 is undeniable in any case. So let us not forget that there are good arguments with a~true conclusion and good arguments with a~false conclusion; there are bad arguments with a~true conclusion and bad arguments with a~false conclusion. Hence, it is not the truth or falsity of a~conclusion that determines whether an argument is good or bad. It must be something else. Philosophers would say: the plausibility of the premisses and the extent to which they lead to the conclusion. But as already mentioned, some economists, most likely under the spell of Friedmann's methodology 
%\label{ref:RNDpgwLITrysJ}(Friedman, 1966, 14f.),
\parencite[][]{},%
care little about the so-called ``reality of assumptions''. This only means, though, that they sometimes and to a~certain extent do not care whether they have a~good or a~bad argument in front of them. Ultimately, however, this cannot stand. And where the foundations of praxeology are at stake, we are well advised not to allow it to.



Let us therefore continue on our chosen path and see that the seemingly trivial (R) leads to our last step, explaining competition to be present when there is rivalrous behaviour, \textit{i.e.}, if and only if there is at least one acting rivalrously:



(C) Competition exists [27F7?] (${\exists}$\textit{x}) (\textit{x} acts rivalrously)



Again, this may be complex when expanded, but it is not complicated. More importantly, we can trace this understanding of competition back to its familiar origins in the theory of action and intentionality, \textit{i.e.}, Sections 2 and 3, so that we are left with nothing but the parsimonious and innocuous assumptions we made there and the assumptions that belong to our variant of first-order predicate logic, which are in any case essential to any reasonable argument.



\section{Conclusion and a~glimpse beyond}

It is sometimes said that ``Austrian economists understand competition better than most economists'' 
%\label{ref:RNDZ0RHWjuQFn}(Nell, 2010, p.142).
\parencite[][p.142]{nell_competition_2010}. %
 Perhaps this is so, but the fact remains that Austrian economists have not traced their understanding of subjectivity to its origin, the theory of intentionality, nor have they traced their sympathy for agent-based modelling of market processes to its foundation, the theory of action. So, they struggled to establish what makes their contribution to economic theory so unique: the philosophy and logic of human action. Looking back at our explanation of competition as rivalry, one might be tempted to say that the conclusion we reached is hardly surprising. And it is true, I~never meant to doubt that economists were aware of the \textit{truth} of this conclusion.\footnote{Remember, however, the lesson from the previous section above.} But what some did not know, or others could not trace back to its root cause, was that there was no need or place in this understanding for anything remotely resembling an equilibrium picture of perfect competition. (C) even makes it clear that explaining competition as rivalry cannot be done within the framework of its market structure approach. Its market structure leaves out what is essential, namely the individual with his subjective attitudes. The explanation of competition as rivalry, on the other hand, avoids the pitfalls of the equilibrium picture. It can also give us a~good idea of what the study of human action can contribute to the study of social sciences in general and economics in particular.



So, what else can the action-theoretic approach contribute to economic theory besides a~solid foundation? For reasons of space, I~can only give an outline here:



\begin{enumerate}

\item \textit{The Coase presumption:} Competition without competitors. \textit{Ronald Coase} 
%\label{ref:RNDKZ0g3nS4yx}(1972)
\parencite*[][]{coase_durability_1972} %
 famously posited that even a~monopolist can only charge competitive prices in the long run. This conjecture helped to explain real phenomena, \textit{e.g.}, why OPEC did not arbitrarily raise oil prices even when it had a~(near) monopoly. Our approach can explain these results without making extravagant assumptions (such as Coase's assumption of competition with a~future self). According to (C), it is sufficient for competitive behaviour that an agent perceives someone as a~rival and acts accordingly. This perception may be erroneous. It may merely be an anticipation of possible future behaviour. Since, in our view, the rival is merely the intentional object of the agent's attitudes, he may or may not be as the agent imagines him, he may even not exist at all (see Section 2);

\item \textit{Risk, Uncertainty and Profit.} In action theory it is a~commonplace that an agent neither strives for what he (really) believes to be impossible, nor for what he (really) believes to be already achieved. Motivation can therefore only be located in the realm of the uncertain. But it is only where the agent acts that the meaning of all competitive behaviour, namely profit, can lie. Thus, we can underline a~result advocated by \textit{Frank Knight} 
%\label{ref:RNDyHIG4LqLPb}(1921),
\parencite*[][]{knight_risk_1921}, %
 and we need only resort to insights gained with the help of the philosophy and logic of human action;

\item \textit{Market failure and antitrust.} Competition does not presuppose the existence of any kind of equilibrium. On the contrary, if there were such an equilibrium, there would be no competition. Consequently, there is also no market failure that manifests itself in competitive behaviour such as (product) differentiation, mergers and acquisitions. This undermines the conceptual basis of most antitrust laws 
%\label{ref:RNDpr6wCvCelM}(cf. Armentano, 1972).
\parencite[cf.][]{armentano_myths_1972}. %
 What drives competition is intentional and therefore subjective: it is the fear of losing business and the hope of somehow still getting it. On action-theoretic grounds, then, it is difficult to find any justification at all for state intervention into the market.

\end{enumerate}

As I~said earlier, Mises held that economic science is based on action theory. This was a~claim that many found too disturbing to defend. He also believed that the theory of human action was ultimately grounded in epistemology, and in his last book he even referred to epistemology as the very foundation of economic science 
%\label{ref:RNDLSWzhOnvon}(Mises, 1962).
\parencite[][]{mises_ultimate_1962}. %
 On this latter point, Mises was mistaken. There is nothing epistemological about action theory or the theory of intentionality. We have proven this by omission. It is more important, however, that we found considerable support for Mises' former point. What has been shown here is evidence for something closely akin to Mises' original claim: The basis of economic science is analytic action theory. To make this very clear: The point here is not to accuse Mises of not having seriously attempted to ground economics in the theory of action. Mises did this like no other. And with considerable success. However, Mises was arguably the only Austrian who was really prepared to go beyond the confines of economic theory---which you have to do if you want to anchor it in another discipline. In this respect, support for Mises within the Austrian community was half-hearted at best. And it did not help that (old) Mises turned on his \textit{alter ego} and endorsed the mistaken claim about epistemology, which so many have repeated ever since. But this unforced error can be corrected, and in part this is what the present paper has done. Thus, as has been suggested elsewhere before 
%\label{ref:RND82RhAudMiD}(cf. Oliva Córdoba, 2017),
\parencite[cf.][]{oliva_cordoba_uneasiness_2017}, %
 praxeology can be well aligned with analytic action theory, retaining the spirit but not the letter of Mises' original approach. The prospects for an integrated approach to Austrian theory as a~fusion of Austrian economics and analytical action theory thus seem good. But even if Austrian economists were to abstain, we should not overlook the fact that in the course of this enquiry we have never had to compromise the rigour, richness and soundness of analytical action theory and the theory of propositional attitudes. If these are decent, well-established and worthwhile fields of study, then recourse to them has most likely added to, rather than detracted from, economic theory. And if this way of studying human action has made a~valuable contribution to explaining competition, it shows not only that the philosophy and logic of human action is useful in the social sciences, but also that it is, or should be, central to economic theory.



\section{Acknowledgments}

Versions of this paper were presented at the one-day conference ``Perspectives of Integrated Austrian Theory'' at the University of Hamburg on Wednesday, 4 October 2017, and at the Libertarian Scholar''s Conference at King's College, New York, on 20 October 2018. I~am indebted to the audiences of these lectures, especially \textit{Joseph T. Salerno} in New York and my commentator \textit{Stefan Kooths} in Hamburg, for helpful comments. Thanks are also due to my students and the participants in \textit{Rolf W. Puster's} research colloquium at the University of Hamburg, which I~have had the honour of co-hosting for twelve years. I~would also like to thank an anonymous reviewer for his valuable suggestions. My special thanks go to \textit{Randall G. Holcombe} and \textit{Rolf W. Puster}, who kindly read earlier versions of the manuscript and improved it in many respects. Any remaining errors are, of course, mine alone.



\section{References}

Acemoglu, D., Laibson, D.I. and List, J.A., 2016. \textit{Microeconomics}. Pearson series in economics. Boston: Pearson.



Ackerman, F. and Nadal, A. eds., 2004. \textit{The Flawed Foundations of General Equilibrium: Critical Essays on Economic Theory}. Routledge frontiers of political economy. London: Routledge.



Ajdukiewicz, K., 1935. Die syntaktische Konnexität. \textit{Studia Philosophica}, 1, pp.1–27.



Anscombe, G.E.M., 1957. \textit{Intention}. 1\textsuperscript{st} ed. Cambridge, MA: Harvard University Press.



Aristotle, 1938. \textit{Categories. On Interpretation. Prior Analytics}. [online] Translated by H.P. Cooke and H. Tredennick Cambridge, MA: Harvard University Press. Available at: {\textless}https://www.loebclassics.com/view/LCL325/1938/volume.xml{\textgreater} [Accessed 14 October 2024].



Armentano, D.T., 1972. \textit{The Myths of Antitrust: Economic Theory and Legal Cases}. New Rochelle, N.Y: Arlington House.



Arrow, K.J. and Debreu, G., 1954. Existence of an Equilibrium for a~Competitive Economy. \textit{Econometrica}, [online] 22(3), pp.265–290. https://doi.org/10.2307/1907353.



Aumann, R.J., 1964. Markets with a~Continuum of Traders. \textit{Econometrica}, [online] 32(1/2), pp.39–50. https://doi.org/10.2307/1913732.



Barrotta, P., 1996. A~Neo-Kantian Critique of Von Mises's Epistemology. \textit{Economics \& Philosophy}, [online] 12(1), pp.51–66. https://doi.org/10.1017/S0266267100003710.



Boettke, P.J., 1990. \textit{The Political Economy of Soviet Socialism: The Formative Years, 1918–1928}. [online] Dordrecht: Springer Netherlands. https://doi.org/10.1007/978-94-017-3433-2.



Brentano, F., 1874. \textit{Psychologie vom empirischen Standpunkte}. [online] Leipzig: Verlag von Duncker \& Humblot. Available at: {\textless}https://archive.org/details/psychologievome02brengoog{\textgreater}.



Brentano, F., 2009. \textit{Psychology from an Empirical Standpoint}. International Library of Philosophy. Translated by A.C. Rancurello, D.B. Terrell and L.L. McAlister Taylor \& Francis.



Bryant, W.D.A. ed., 2010. \textit{General Equilibrium: Theory and Evidence}. Singapore; Hackensack, NJ: World Scientific Pub. Co.



Burge, T., 2010. \textit{Origins of Objectivity}. Oxford: Oxford University Press. https://doi.org/10.1093/acprof:oso/9780199581405.001.0001.



Burton-Roberts, N., 2016. \textit{Analysing Sentences: An Introduction to English Syntax}. 4\textsuperscript{th} ed. Learning about Language. London; New York: Routledge.



Carnap, R., 1963. Intellectual Autobiography. In: \textit{The Philosophy of Rudolf Carnap}, The Library of living philosophers. La Salle, IL: Open Court. pp.3–84.



Coase, R.H., 1972. Durability and Monopoly. \textit{The Journal of Law and Economics}, [online] 15(1), pp.143–149. https://doi.org/10.1086/466731.



Condic, S.B. and Morefield, R., 2021. Hayek on the essential dispersion of market knowledge. \textit{The Review of Austrian Economics}, [online] 34(4), pp.449–463. https://doi.org/10.1007/s11138-019-00487-4.



Crane, T., 1998. Intentionality as the Mark of the Mental. In: A. O'Hear, ed. \textit{Contemporary Issues in the Philosophy of Mind}, Royal Institute of Philosophy Supplements. [online] Cambridge: Cambridge University Press. pp.229–252. https://doi.org/10.1017/CBO9780511563744.013.



Crane, T., 2001. \textit{Elements of Mind: An Introduction to the Philosophy of Mind}. Oxford, New York: Oxford University Press.



Crane, T., 2013. \textit{The Objects of Thought}. Oxford, New York: Oxford University Press.



Davidson, D., 1963. Actions, Reasons, and Causes. \textit{The Journal of Philosophy}, [online] 60(23), pp.685–700. https://doi.org/10.2307/2023177.



Davidson, D., 2001. How is weakness of the Will possible? [1970]. In: \textit{Essays on Actions and Events}, Philosophical Essays of Donald Davidson, 2\textsuperscript{nd} ed. Oxford, New York: Oxford University Press. pp.21–42.



De Haro, S., 2020. Science and Philosophy: A~Love–Hate Relationship. \textit{Foundations of Science}, [online] 25(2), pp.297–314. https://doi.org/10.1007/s10699-019-09619-2.



Ditmarsch, H. van, Halpern, J.Y., Hoek, W. van der and Kooi, B.P. eds., 2015. \textit{Handbook of Epistemic Logic}. London: College publications.



Ebeling, R.M., 1993. Economic Calculation Under Socialism: Ludwig von Mises and His Predecessors. In: J.M. Herbener, ed. \textit{The Meaning of Ludwig von Mises}. [online] Dordrecht: Kluwer Academic Publishers. pp.56–101. https://doi.org/10.1007/978-94-011-2176-7\_3.



Edgeworth, F., Y., 1881. \textit{Mathematical Psychics: An Essay on the Application of Mathematics to the Moral Sciences}. [online] London: Kegan Paul. Available at: {\textless}http://historyofeconomicthought.mcmaster.ca/edgeworth/mathpsychics.pdf{\textgreater} [Accessed 2 September 2019].



Flanigan, J., 2016. Obstetric Autonomy and Informed Consent. \textit{Ethical Theory and Moral Practice}, [online] 19(1), pp.225–244. https://doi.org/10.1007/s10677-015-9610-8.



Frankfurt, H.G., 1978. The Problem of Action. \textit{American Philosophical Quarterly}, [online] 15(2), pp.157–162. Available at: {\textless}https://www.jstor.org/stable/20009708{\textgreater} [Accessed 14 October 2024].



Frege, G., 1891. \textit{Function und Begriff: Vortrag, gehalten in der Sitzung vom 9. Januar 1891 der Jenaischen Gesellschaft für Medicin und Naturwissenschaft}. [online] Jena: Pohle. Available at: {\textless}https://gdz.sub.uni-goettingen.de/id/PPN64299370X{\textgreater} [Accessed 16 October 2024].



Frege, G., 1953. \textit{The Foundations of Arithmetic: A~Logico-Mathematical Enquiry into the Concept of Number}. Rev. ed. New York: Harper and Brothers.



Frege, G., 1960. Function and concept. Translated by P.T. Geach and M. Black. In: \textit{Translations from the Philosophical Writings}, 2\textsuperscript{nd} ed. Oxford: Basil Blackwell. pp.21–41.



Friedman, M., 1966. The Methodology of Positive Economics. In: \textit{Essays in Positive Economics}. Chicago; London: Chicago University Press. pp.3–43.



Fumerton, R., 2017. Epistemology and Science: Some Metaphilosophical Reflections. \textit{Philosophical Topics}, [online] 45(1), pp.1–16. Available at: {\textless}https://www.jstor.org/stable/26529422{\textgreater} [Accessed 15 October 2024].



Hayek, F.A., 1948. Economics and Knowledge. In: \textit{Individualism and Economic Order}. London: Routledge. pp.33–56.



Hayek, F.A. von, 1945. The Use of Knowledge in Society. \textit{The American Economic Review}, [online] 35(4), pp.519–530. Available at: {\textless}https://www.kysq.org/docs/Hayek\_45.pdf{\textgreater} [Accessed 20 September 2024].



Hintikka, J., 1962. \textit{Knowledge and Belief: An Introduction to the Logic of the Two Notions}. Contemporary philosophy. Ithaca, NY: Cornell University Press.



Husserl, E., 1913. \textit{Prolegomena zur reinen Logik}. Halle: Max Niemeyer.



Jennings, B., 2009. Agency and Moral Relationship in Dementia. \textit{Metaphilosophy}, [online] 40(3–4), pp.425–437. https://doi.org/10.1111/j.1467-9973.2009.01591.x.



Khan, M.A., 2008. Perfect Competition. In: Palgrave Macmillan, ed. \textit{The New Palgrave Dictionary of Economics}. [online] London: Palgrave Macmillan UK. pp.1–15. https://doi.org/10.1057/978-1-349-95121-5\_1633-2.



Kirzner, I.M., 1995. The Subjectivism of Austrian Economics. In: G. Meijer, ed. \textit{New Perspectives on Austrian Economics}. London: Routledge. pp.11–22.



Kirzner, I.M., 2016. \textit{The History and Importance of the Austrian Theory of the Market Process}. [online] 2016 Advanced Austrian Seminar, Mercatus Center Academic \& Student Programs. Available at: {\textless}https://www.youtube.com/watch?v=GvE4zEfrv0k{\textgreater} [Accessed 15 October 2024].



Knight, F.H., 1921. \textit{Risk, Uncertainty and Profit}. Boston; New York: Houghton Mifflin Company.



Knudsen, C., 2004. Alfred schutz, Austrian Economists and the Knowledge Problem. \textit{Rationality and Society}, [online] 16(1), pp.45–89. https://doi.org/10.1177/1043463104036622.



Kurrild-Klitgaard, P., 2001. On Rationality, Ideal Types and Economics: Alfred Schüutz and the Austrian School. \textit{The Review of Austrian Economics}, [online] 14(2), pp.119–143. https://doi.org/10.1023/A:1011199831428.



Lachmann, L.M., 1982. Ludwig von Mises and the Extension of Subjectivism. In: I.M. Kirzner, ed. \textit{Method, Process, and Austrian Economics: Essays in Honor of Ludwig Von Mises}. Lexington, MA: Lexington Books. pp.31–40.



Lavoie, D., 2015. \textit{Rivalry and Central Planning: The Socialist Calculation Debate Reconsidered}. Advanced studies in political economy. Arlington, Virginia: Mercatus Center, George Mason University.



Lemmon, E.J., 1965. \textit{Begining Logic}. London: Thomas Nelson and Sons Limited.



Lewis, D., 1970. General semantics. \textit{Synthese}, [online] 22(1), pp.18–67. https://doi.org/10.1007/BF00413598.



Lyons, J., 1968. \textit{Introduction to Theoretical Linguistics}. 1\textsuperscript{st} ed. [online] Cambridge: Cambridge University Press. https://doi.org/10.1017/CBO9781139165570.



Mankiw, N.G., 2020. \textit{Principles of Economics}. 9\textsuperscript{th} ed. Boston, MA: Cengage Learning, Inc.



Mankiw, N.G. and Taylor, M.P., 2014. \textit{Economics}. 3\textsuperscript{rd} ed. Andover: Cengage Learning.



Marshall, A., 1890. \textit{Principles of Economics}. London; New York: Macmillan and Company.



Mas-Colell, A., 1974. An equilibrium existence theorem without complete or transitive preferences. \textit{Journal of Mathematical Economics}, [online] 1(3), pp.237–246. https://doi.org/10.1016/0304-4068(74)90015-9.



McKenzie, L.W., 1954. On Equilibrium in Graham's Model of World Trade and Other Competitive Systems. \textit{Econometrica}, 22(2), pp.147–161. https://doi.org/10.2307/1907539.



McKenzie, L.W., 1981. The Classical Theorem on Existence of Competitive Equilibrium. \textit{Econometrica}, [online] 49(4), pp.819–841. https://doi.org/10.2307/1912505.



Mele, A., 2010. Weakness of will and akrasia. \textit{Philosophical Studies}, [online] 150(3), pp.391–404. https://doi.org/10.1007/s11098-009-9418-2.



Mill, J.S., 1848. \textit{Principles of Political Economy with Some of Their Applications to Social Philosophy}. Boston: Charles C. Little \& John Brown.



Minio-Paluello, L. ed., 1961. \textit{Aristoteles latinus. I:1-5 Categoriae vel Praedicamenta}. Corpus philosophorum medii aevi. Leiden: Brill.



Mises, L. von, 1962. \textit{The Ultimate of Foundation of Economic Science: An Essay on Method}. William Vilker Fund Series in the Humane Studies. Princeton, NJ: Van Nostrand Comp.



Mises, L. von, 1998. \textit{Human Action: A~Treatise on Economics}. Scholar's ed. ed. Auburn AL: Ludwig von Mises Institute.



Motta, M., 2004. \textit{Competition Policy: Theory and Practice}. [online] Cambridge: Cambridge University Press. https://doi.org/10.1017/CBO9780511804038.



Nagel, T., 1979. Subjective and Objective. In: \textit{Mortal Questions}. Cambridge: Cambridge University Press. pp.196–213.



Nell, G.L., 2010. Competition as market progress: An Austrian rationale for agent-based modeling. \textit{The Review of Austrian Economics}, [online] 23(2), pp.127–145. https://doi.org/10.1007/s11138-009-0088-2.



O'Brien, E. by L. and Soteriou, M. eds., 2009. \textit{Mental Actions}. Oxford, New York: Oxford University Press.



Oliva Córdoba, M., 2017. Uneasiness and Scarcity: An Analytic Approach Towards Ludwig von Mises's Praxeology. \textit{Axiomathes}, [online] 27(5), pp.521–529. https://doi.org/10.1007/s10516-017-9352-4.



Petri, F. and Hahn, F. eds., 2003. \textit{General Equilibrium: Problems and Prospects}. Routledge Siena studies in political economy. London: Routledge.



Pindyck, R.S. and Rubinfeld, D.L., 2013. \textit{Microeconomics}. 8\textsuperscript{th} ed. Boston: Pearson.



Plato, 1921. \textit{Plato in Twelve Volumes: Vol. 12: Theaetetus}. Loeb classical library. Translated by H.N. Fowler London; Cambridge: W. Heinemann; Harvard University Press.



Prendergast, C., 1986. Alfred Schutz and the Austrian School of Economics. \textit{American Journal of Sociology}, [online] 92(1), pp.1–26. https://doi.org/10.1086/228461.



Rajagopalan, S. and Rizzo, M.J., 2019. Austrian Perspectives in Law and Economics. In: A. Marciano and G.B. Ramello, eds. \textit{Encyclopedia of Law and Economics}. [online] New York, NY: Springer. pp.92–99. https://doi.org/10.1007/978-1-4614-7753-2\_621.



Richardson, S.S., 2009. The Left Vienna Circle, Part 1. Carnap, Neurath, and the Left Vienna Circle thesis. \textit{Studies in History and Philosophy of Science Part A}, [online] 40(1), pp.14–24. https://doi.org/10.1016/j.shpsa.2008.12.002.



Safra, Z., 1989. Strategic Reallocation of Endowments. In: J. Eatwell, M. Milgate and P. Newman, eds. \textit{Game Theory}. [online] London: Palgrave Macmillan UK. pp.225–231. https://doi.org/10.1007/978-1-349-20181-5\_27.



Samuelson, P.A., 1947. \textit{Foundations of Economic Analysis}. Harvard economic studies. Cambridge: Harvard University Press.



Samuelson, P.A., 1964. Theory and Realism: A~Reply. \textit{The American Economic Review}, [online] 54(5), pp.736–739. Available at: {\textless}https://www.jstor.org/stable/1818572{\textgreater} [Accessed 15 October 2024].



Samuelson, P.A., 1966. Modern Economic Realities and Individualism. In: J.E. Stiglitz, ed. \textit{The Collected Scientific Papers of Paul a. Samuelson. Vol. 2}. Cambridge: MIT Press. pp.1407–1418.



Samuelson, P.A. and Nordhaus, W.D., 2009. \textit{Economics}. 19. ed ed. The McGraw-Hill series economics. Boston, MA: McGraw-Hill.



Schütz, A., 1953. Common-Sense and Scientific Interpretation of Human Action. \textit{Philosophy and Phenomenological Research}, [online] 14(1), pp.1–38. https://doi.org/10.2307/2104013.



Schütz, A., 1996. Political Economy: Human Conduct in Social Life. In: \textit{Collected Papers: Volume IV}. [online] Dordrecht: Springer Netherlands. pp.93–105. https://doi.org/10.1007/978-94-017-1077-0\_10.



Simons, P., 2009. Introduction to the second edition. Translated by A.C. Rancurello, D.B. Terrell and L.L. McAlister. In: O. Kraus and L.L. McAlister, eds. \textit{Psychology from an Empirical Standpoint}, International Library of Philosophy. Taylor \& Francis. pp.xiii–xx.



Smith, A., 1776. \textit{An Inquiry into the Nature and Causes of the Wealth of Nations. Vol. 1}. [online] London: W. Strahan \& T. Cadell. Available at: {\textless}https://books.google.pl/books?id=jRNDAAAAcAAJ{\textgreater} [Accessed 16 October 2024].



Stigler, G.J., 1957. Perfect Competition, Historically Contemplated. \textit{Journal of Political Economy}, [online] 65(1), pp.1–17. https://doi.org/10.1086/257878.



Stiglitz, J.E. and Walsh, C.E., 2006. \textit{Economics}. 4\textsuperscript{th} ed. New York: W.W. Norton.



Storr, V.H., 2019. Ludwig Lachmann's peculiar status within Austrian economics. \textit{The Review of Austrian Economics}, [online] 32(1), pp.63–75. https://doi.org/10.1007/s11138-017-0403-2.



Tallerman, M., 2015. \textit{Understanding Syntax}. 4\textsuperscript{th} ed. London: Routledge. https://doi.org/10.4324/9781315758084.



Wald, A., 1935. Über die eindeutige positive Lösbarkeit der neuen Productionsgleichungen. \textit{Ergebnisse eines Mathematischen Kolloquiums}, 6, pp.12–20.



Walker, A.F., 1989. The Problem of Weakness of Will. \textit{Noûs}, [online] 23(5), pp.653–676. https://doi.org/10.2307/2216006.



Walras, L., 2019. \textit{Léon Walras, Elements of Theoretical Economics: Or the Theory of Social Wealth}. Cambridge: Cambridge University Press.



Weintraub, E.R., 2011. Retrospectives: Lionel W. McKenzie and the Proof of the Existence of a~Competitive Equilibrium. \textit{Journal of Economic Perspectives}, [online] 25(2), pp.199–215. https://doi.org/10.1257/jep.25.2.199.



Wittgenstein, L., 1922. \textit{Tractatus logico-philosophicus}. International library of psychology, philosophy and scientific method. London: Kegan Paul, Trench, Trubner.



Wittgenstein, L., 2013. \textit{Tractatus Logico-Philosophicus}. Routledge great minds. Translated by D. Pears and B. McGuinness London: Routledge.



Wright, G.H. von, 1971. \textit{Explanation and Understanding}. Contemporary philosophy. Ithaca, NY: Cornell University Press.



Yeager, L.B., 1994. Mises and Hayek on calculation and knowledge. \textit{The Review of Austrian Economics}, [online] 7(2), pp.93–109. https://doi.org/10.1007/BF01101944.



Zambrano, A., 2017. Patient Autonomy and the Family Veto Problem in Organ Procurement. \textit{Social Theory and Practice}, [online] 43(1), pp.180–200. Available at: {\textless}https://www.jstor.org/stable/24871373{\textgreater} [Accessed 15 October 2024].

\end{document}


\setcounter{secnumdepth}{1}



\title{Szablon-EN}

\begin{document}

Model Uncertainty: when modeling risk leads to pretense of knowledge





Mateusz Machaj (University of New York in Prague)



Abstract: The main purpose of the paper is to develop a~concept of \textit{model uncertainty} as opposed to the existing and well-established concept of model risk. Up to date the broad literature on probability not only developed complete probability systems, but also correctly noticed limitations of probability calculus. Despite the acknowledgement of such probability restrictions, drawbacks of modeling are often related to model risk. We present an argument here to distinguish a~feature limiting models even further: model uncertainty. The tenets of it already exist in the literature on probability, but were not properly emphasized while the idea of model risk was developed. Our plan it to start with a~broad overview of the existing knowledge about probability in order to start with fundamental principles. From them we are deriving a~new concept of model uncertainty.



Keywords: Uncertainty, Frank Knight, model risk, model uncertainty



JEL: B40, B41, D5, C00, C18



\section{Introduction}

Probability theoreticians from all disciplines have recognized for a~long time that calculation of probability has its limitations, especially when one applies it to describe existing reality, or even predict future events. Probability has wide variety of applications in various scientific fields, ranging from hard natural physical sciences, through biological sciences, to social sciences, including economics, sociology, and especially policy making. Even though, while building philosophical foundations of probability, experts virtually always recognize its shortcomings, these are often brushed aside in application to practical aspects of social sciences. There is a~notion of model risk used especially in finance regarding models used for pricing and decision making -- which is an attempt to infer the potential mistakes from wrong parameters used in the model. Yet it may suffer a~similar limiting feature as the model itself, for it has to assume something about knowing the underlying parameters (or kind of meta-parameters).



In analyzing probability two separate concepts were developed: the Knightian distinction between uncertainty and risk, which happens to parallel the Misesian distinction between case probability and class probability. Class probability (risk) is commonly associated with the traditional approach in statistics, and refers to the probability of an event based on a~long-run frequency within a~well-defined reference class. It is therefore applicable to situations where events are repeatable and strictly homogeneous, whereas case probability (uncertainty), or specific event probability, applies to unique, non-repeatable events and is based on subjective judgment rather than empirical frequency. Following this line of distinction we suggest to create a~concept of model uncertainty being parallel to already existing concept of model risk (uncertainty would here mean that we have an undetermined component which influences the outcomes and is not subjected to probability calculus). In order to arrive at it we start off with the basic principles of probability.



The first section describes the subjective perspective on the nature of probability. The second section discusses the limits of probability calculus, mostly due to Knightian uncertainty. The third section defends the notion that even under pure uncertainty there exist regularities in economies, hence economic laws. The fourth section explains how flawed probability models can lead to pretense of knowledge, thus increasing economic ignorance rather than enhancing knowledge. The fifth section discusses model risk as opposed to our notion of model uncertainty. The last section offers concluding comments.



\section{Probability as a~Solution to Ignorance}

Probability is an indispensible scientific concept. Repeated analyses of numerous events under varying circumstances do not always lead to deterministic recognition of the variables. The future of any observed system, both in the social and natural sciences, is not entirely foreseeable. Despite such lack of knowledge, we might recognize patterns of possible outcomes. Under reasonable assumptions scientists can create probability distributions of likely scenarios. The absence of full knowledge leads to partial knowledge. From this perspective probability analysis can be seen as a~partial solution to ignorance. Probability analysis produces knowledge about ignorance that helps us identify the boundaries of knowing and predicting.



Let us use the example of coin flipping by person A. Person B~is asked whether the result is going to be heads or tails. To give a~correct answer she would have to know all the relevant conditions and factors that might influence the result, including very specific circumstances of small particles and forces affecting the coin flipping. This would have to include knowledge of magnetic forces, atoms, electrons, and their relation to each other, plus of course a~perfect simulation of person A's hand throwing a~coin into the air. In other words, one would need to have a~complete model of the part of the universe inside the room to make a~correct prediction. The model would be complete, and it would be an equilibrium model of reality. Probability distributions would be worthless. Strictly speaking there would exist only two probabilities: 1 or 0. Something was sure to happen, or not to happen.



Equipped with this knowledge person B~would become like Laplace's demon, capable of giving an ultimate and complete description of the world. But human beings are not capable of creating a~complete model of the whole universe, and there are a~priori and empirical reasons to believe they will never be capable of doing so. It seems that statements about reality contain probabilities ranging from 0 and 1 because our knowledge of causal relation is necessarily deficient. Our ignorance becomes the reason for probability substituting for the unattainable ideal of full knowledge. If we knew more about the specific state of the coin, then probabilities might have been different 
%\label{ref:RNDduNU5wnOj5}(Reeves, 1988, pp.179–180).
\parencite[][pp.179–180]{reeves_theory_1988}.%




Probability limits the strictness of scientific laws. Nonetheless the recognition of limits for exact laws in physics does not justify scientific nihilism. One cannot answer with certainty whether the coin will land heads or tails up; but this does not mean one cannot say anything about the coin flipping. The role of science is to allow people to minimize their ignorance and yield information even about cases where full prediction is impossible. Even though one is not able to gather all the individual pieces of knowledge and predict the result of coin flipping, it is possible to learn something about this event (or these types of events). Observed and systematized studies on the distribution of results in such cases can increase our knowledge, although it is still partial knowledge 
%\label{ref:RNDuELnrfCUNf}(Kyburg, 1966, p.254).
\parencite[][p.254]{kyburg_probability_1966}. %
 The analysis would tell us whether something is more or less likely to follow. Assuming the analyzed event can be repeatedly observed, this ``more or less likely'' is captured more technically in the mathematical operations known as probability calculus. The principle of maximum likelihood selects preferred statistical theories 
%\label{ref:RNDvBK4kwPLQe}(Swinburne, 1971, p.328).
\parencite[][p.328]{swinburne_probability_1971}. %
 Because we recognize limits to our understanding, though, we accept the fact that a~full, Laplacean model of the universe and perfect predictability is unattainable.



Assuming a~probabilistic view of the world does not prohibit our assuming a~more general metaphysical determinism. Only one world exists, the one we are experiencing, ``and it never occurs twice in exactly the same state'' 
%\label{ref:RNDC4jJxRfgku}(Bricmont, 2002, p.4).
\parencite[][p.4]{bricmont_determinism_2002}. %
 Every event occurring in the world represents some characteristic feature of this world 
%\label{ref:RNDLxiY1xDjOx}(Fetzer, 1977, p.397).
\parencite[][p.397]{fetzer_world_1977}. %
 Yet determinism broadly understood as the rule that every effect has a~specifically related, exclusive set of causes is not the same as predictability. As Bricmont argues, just because we can lock up a~clock in a~drawer on an unattainable mountain and make its movement become unpredictable to us, does not mean that the movement itself is undetermined. Something can be unknown and unpredictable to us, but still determined by a~strict set of laws. Physics and metaphysics are not against each other in this respect. It might be the case that outside of the physical perspective Laplace's demon, or God, can describe the universe in a~more fundamental manner than probability theory does. Probability theory is merely a~specific type of theory that allows us to gather partial empirical knowledge that is better than complete ignorance. Under (the impossible ideal of) full knowledge the concept of probability would not be needed. In other words, we study probabilities because of epistemic indeterminism, not ontic indeterminism 
%\label{ref:RNDlEDVmbvBMM}(Fetzer, 1983, pp.371–372).
\parencite[][pp.371–372]{fetzer_probability_1983}.%
\footnote{Although the probabilistic view does not rule out an underlying determinism, it doesn't require it either.}



Put differently, probabilities need not really be ``out there'' in the universe. They are inherently linked to our existence in empirical reality and represent the relationship of our mind to that reality.\footnote{Bricmont 
%\label{ref:RNDO0lPY3mpu6}(2002)
\parencite*[][]{} %
 argues that even physical determinism leading to the rejection of the neo-indeterminist approach might come one day. The case of micro world and quantum laws is more complex and controversial. Probabilities can be seen as ``casual tendencies'' 
%\label{ref:RNDdAXEn8iWn8}(Shanks, 1993, p.295).
\parencite[][p.295]{shanks_time_1993}. %
 In those cases we appear to deal with irreducibly probabilistic behavior of molecules not being disrupted by additional forces 
%\label{ref:RNDsouCPNySdR}(Fetzer, 1983, p.372).
\parencite[][p.372]{fetzer_probability_1983}. %
 Yet just because the current state of knowledge does not allow us to point to any secondary factors, it does not mean that they are not there 
%\label{ref:RNDoZ8DmgRN4v}(Fetzer, 1983, p.373).
\parencite[][p.373]{fetzer_probability_1983}. %
 On the general level, Max Planck, commented, similarly to Albert Einstein, ``determinism is to be preferred over indeterminism under all circumstances, simply for the reason that determinate (\textit{bestimmte}) answer to a~question is always more valuable than an undeterminate (\textit{unbestimmte}) one'' 
%\label{ref:RNDntMzpY6NaW}(quoted in Krüger, 1986, p.281).
\parencite[quoted in][p.281]{kruger_probability_1986}.%
} Probability statements reflect the ``relation between a~body of evidence and propositions'' 
%\label{ref:RNDBDpZOOu9tg}(Moser, 1988, p.232).
\parencite[][p.232]{moser_foundations_1988}.%




\section{Limits of Probability Calculus: From Calculus to Judgment}

Ironically, the application of probability models might be risky. There is one important reason for that, which is to be found in the answer to the question, ``What is probability?'' Mathematics is not an empirical science---it is a~reflection of the mind (corresponding in some loose way to real objects). For that reason, relating mathematics to the real world is always challenging. The theory of probability, being a~mathematical science, is not different in that respect. In order to make sure that observations of real events comply with computed probability distributions, a~methodological leap is needed. If probabilities were just mathematical functions, written and worked out on computers, then they would have to be limited to mere mental gymnastics.



One of the most important probability theorists in history was the great Austrian mathematician Richard von Mises, who offered strong support for the frequency interpretation of probability.\footnote{The concept of frequency probability was of course developed much earlier than Richard von Mises. Its traces can be found in Aristotle, while, among others, Gauss, Laplace, Poisson were well aware of it (I thank anonymous referee for this point).} 
%\label{ref:RNDyHPAk8RZiV}(Modern, mainstream axiomatic foundations were built by Andrey Kolmogorov; see Howson, 1995, pp.17–18).
\parencites[mainstream axiomatic foundations were built by][]{}[see][pp.17–18]{howson_theories_1995}. %
 According to him, probabilities, understood as mathematical functions, need to be applied to certain collectives, which one can subject to repeated trials. We cannot talk of probabilities of single events, but only about classes of events constituting a~collective. Hence one cannot say there is a~90 percent probability that a~certain candidate will win the presidential elections in 2012 because it is a~one-off event. For that claim to be true one would need a~number of that type of elections, and only then could one venture probability distributions. Particular events have to be classified in terms of truly homogenous collectives (like the number of coins flipped) to be subjected to a~probability calculus.



Richard von Mises's argument was that a~collective needs to satisfy two essential conditions: relative frequencies need to tend to fixed limits and they have to be random 
%\label{ref:RND5Az0sxp6wG}(Mises, 1957, pp.28–29).
\parencite[][pp.28–29]{mises_probability_1957}. %
 If the coin is perfect, then the probability is 50 percent each for heads and tails. This does not mean, however, that for every ten throws the result will be 5 of each. What it means is that an infinite amount of throws will lead to a~distribution in which 50 percent of them will land heads up and 50 percent will land tails up. Randomness also means that if I~decide to register only every seventh flip of a~coin, probabilities would still tend to the same fixed limits---that is, for every seventh throw until infinity, probabilities would also tend to 50 percent.



Despite mathematical clarity there is an obvious problem here, since it is never possible to engage in infinite trials to identify true and ``certain'' probabilities of empirical regularities. One would have to rely instead on approximations and experiments. For Richard von Mises, a~methodological positivist, probabilities are out there in the world, existing objectively, and a~sufficient amount of controlled experiments should allow us to establish them in purely mathematical form. Thus the experiment under controlled conditions is a~bridge between pure mathematical function and reality. In this sense probability comes from experience with large elements from collectives. Estimations, however, are usually not set and fixed links between reality and mathematical formulas because they are derived from prior experiences, subjected to some unpredicted changes waiting to happen in the future (possible exception is naturally probability in fundamental physical models).



Richard's brother Ludwig offered what is probably a~better solution to this problem. (At least it is more empirical than relying on the concept of limiting frequencies.) Instead of a~criterion of randomness and limiting frequencies, where one needs infinite trials, it would be better to state that we do not know anything specific about particular elements of a~class except that they are members of that class 
%\label{ref:RNDs2p10QEehj}(Mises, 1966, p.109).
\parencite[][p.109]{mises_human_1966}. %
 Ludwig's improvement on his brother's theory, though, not only clarified probability assumptions and thereby made it better suited for empirical science, but also dramatically shifted away in a~philosophical approach. According to Ludwig, probability is not out there in the world, but comes from our reflections upon reality. Probabilities are proxies used to tame full ignorance. They are neither purely subjective, nor do they completely describe objective reality. They are not wishful thinking, and they are based on empirical evidence 
%\label{ref:RNDTjWllbRSJi}(Moser, 1988, p.233).
\parencite[][p.233]{moser_foundations_1988}.%




Because of his quasi-subjectivist approach, Ludwig von Mises noticed another form of probability, which is not applied to repeatable and homogenous events, that he called case probability. By this he referred to events, in particular related to humans, where conditions and circumstances are so specific that repeatable trials are impossible. It is questionable to use the word probability in those cases, since the class cannot be identified through experience. We cannot, for example, say what the probability is that Bill Gates will earn \$10 million next month in the same way that we may say a~coin flip lands 50 percent tails up and 50 percent heads up. The meaning is radically different. How so? Because the former event is unique and we know something more and something less about it. We cannot find an analytically useful class and yet it is not true that we know nothing distinctive about this event. Hence when we say that Trump has a~10 percent chance of winning the election, we merely express our qualitative judgment. We cannot and will not have ten identical elections, leading to ten parallel worlds, one of which would have Trump as president in it. We have only one really existing world. Moreover, the judgment of 10\% does not mean the event would not happen in this one existing world. Neither it means there was some necessary fundamental flaw in the reasoning.



The case-probability notion relates closely to Frank Knight's concept of uncertainty 
%\label{ref:RNDA60I125HTb}(Knight, 1971, pp.226–232).
\parencite[][pp.226–232]{knight_risk_1971}. %
 Knight pointed to the unknown, let us even say accidental, element in everyday life.\footnote{It seems that John Maynard Keynes 
%\label{ref:RNDarfxW2Lo4K}(1921)
\parencite*[][]{keynes_treatise_1921} %
 would also adhere to this view (though he subscribed to a~logical-relationist theory of probability). See also a~comparison between Mises and Keynes 
%\label{ref:RNDPT46rFN4ff}(Hauwe, 2007).
\parencite[][]{hauwe_john_2007}. %
 Van den Hauwe makes a~compelling case to demonstrate that both of those thinkers were subjective probability theorists. For a~comprehensible comparison between Knight and Ludwig von Mises, see 
%\label{ref:RND5lbM7j7Kef}(Hoppe, 2007).
\parencite[][]{hoppe_limits_2007}.%
} This element cannot be applied to a~probability calculus because it concerns unique events.\footnote{Although there is a~difference between stating that class probabilities do not exist and that class probabilities are not known. Economists sometimes understand ``uncertainty'' in the much narrower sense as not knowing the really existing probability distribution, or not knowing the exact position in the probability distribution.} Hence the radical conclusion that we cannot perfectly model human beings and their economic choices as elements of probability distributions. Yet this neither stops us from stating economic laws, nor from using a~probability calculus.



\section{True Uncertainty and Economic Laws}

Given true, or Knightian, uncertainty, are there truly any social universals? The impossibility of inference under uncertainty may lead one to scientific skepticism, the rejection of universal social laws, and what Lachmann 
%\label{ref:RNDOaTptNEBjN}(1976)
\parencite*[][]{lachmann_mises_1976} %
 saw a~kaleidoscopic view of the world. Every single decision, with its distinctive, unrepeatable features, reshapes existing and dynamic social reality, moving it to a~new disequilibrium. In fact very soon this disequilibrium is again disturbed by another unrepeatable and unique event. Hence an economist trying to answer the question about regularities in the economy is fooling himself. That seems to be an implication of rejecting determinism and probability determinism.



George Shackle, skeptical of the neoclassical approach, took this observation to its radical extreme: there are no strict economic laws 
%\label{ref:RNDdVdmTXNIBR}(Shackle, 1972, p.427).
\parencite[][p.427]{shackle_epistemics_1972}. %
 Uncertainty pervades everyday choices, which hence cannot be subjected to formalization. Such a~criticism refers to the deterministic approach in the form of simple marginal calculus, but also with the same strength to the probability calculus, since such calculus requires the economic world to be varying yet unchanging 
%\label{ref:RNDYrPxHniUv9}(Shackle, 1972, p.381).
\parencite[][p.381]{shackle_epistemics_1972}.%




Despite the fact that mainstream economists are not extreme Shackleans, they seem to implicitly agree with Shackle's point of view. Mathematical models are all we have and without them nothing is left. Nassim Taleb, criticizing the naïve class-probability approach, has this attitude 
%\label{ref:RNDW0K1oL5NLf}(Taleb, 2007, p.276).
\parencite[][p.276]{taleb_black_2007}. %
 One could echo here Keynes's comment on Tinbergen's works, which back in the 1930s took a~step towards greater mathematization of economics: ``I have a~feeling that Prof. Tinbergen may agree with much of my comment, but that his reaction will be to engage another ten computors [sic] and drown his sorrows in arithmetic'' 
%\label{ref:RNDfHedplYQzH}(Keynes, 1939, p.568).
\parencite[][p.568]{keynes_professor_1939}.%




Economists from the mainstream recognize this problem, no matter which school of thought they represent. Robert Lucas, brilliant pioneer of New Classical macroeconomics, recognized that under true, Knightian uncertainty neoclassical theory is not useful: ``In situations of risk, the hypothesis of rational behavior may be explainable in terms of economic theory. […] In cases of uncertainty, \textit{economic reasoning will be of} [sic] \textit{no value}`` 
%\label{ref:RNDXX155bY6t4}(Lucas, 1977, p.15, emphasis added).
\parencite[][emphasis added]{lucas_understanding_1977}. %
 Paul Samuelson, the godfather of the neoclassical synthesis and the Keynesian interpretation of business cycles, commenting on utility analysis, expressed the same opinion: ``[We should] never forget that economics can at no time become an \textit{exact} science for the reason that actual economic history is not ever what mathematicians call a~‘stationary probability distribution'. There are thus no exact simple rules to learn how to benefit from knowledge of the past. \textit{None at all}'' 
%\label{ref:RNDCodD7sRVc8}(Samuelson, 2008, pp.113–114, emphasis added).
\parencite[][emphasis added]{samuelson_asymmetric_2008}. %
 Let us notice that even though Lucas and Samuelson, both Nobel Prize winners, radically differ on macroeconomic policies and their effectiveness, in this they reach the same conclusion: Knightian uncertainty endangers their economic theories and pushes them towards Shackle's kaleidics.



The Shacklean approach to the validity of economic laws is defensible, however, only if economics exclusively relies on human beings' motives, ideas, and psychological states. If the economic subject dwells only on preference functions and expectations, then an ultrasubjectivist rejection of economic laws may seem reasonable. Naturally economics is about choices, while all choices are unique and specific events. We can grasp them individually in separation of other choices and discuss them using \textit{Verstehen}---historical understanding of specific circumstances 
%\label{ref:RNDcFKo6Atyl7}(Tucker, 1965).
\parencite[][]{tucker_max_1965}. %
 Yet economics is not about exact and particular choices, which form our historical experience. Economics is about the broad range of choices made in objective reality bounded by observable constraints. It is true that people can adopt completely randomly their subjective preferences (and also do not engage in mental gymnastics with indifference curves). But since the objects of their choices exist objectively, outside of their minds, there is at least the possibility of conditions limiting the power of humans to shape economic reality. Hence, economics can illustrate the connections between subjectively chosen ends and objectively existing means no matter what those ends are.



One example concerns expectations and budget constraints. People form expectations subjectively. No economic modeling could create a~complete description of them that could lead to full predictability of actions. If economics were only about expectations (individual human perceptions and motives), then there would be no universal economic laws, since expectations are always unique. But, as Garrison 
%\label{ref:RNDRLCa3Fxbmt}(Garrison, 2001, p.9)
\parencite[][p.9]{garrison_time_2001} %
 commented, we cannot spend our expectations. We form our expectations based on budget constraints and resource scarcity, money supplies, asset ownership and levels of debt, etc.--- the markets and institutions that limit our choices.



Since we are capable of analyzing those constraints and their effects, we are also capable of identifying economic laws, even given Knightian uncertainty. Examples (possibly debatable) include the following: ``Price controls lead to discoordination (surpluses and shortages),'' or ``The central bank cannot permanently keep the interest rates below the market level,'' or ``Increases of the money supply lead to redistribution effects.'' Economic laws do not lead to perfect forecasts. But science is not synonymous with prediction.



\section{Illusions of Certainty: Probability as a~Pretense of Knowledge}

There are certain pillars of probability calculus. As Shackle stresses, the analyzed system neither is inherently evolutionary, nor has a~``tendency to explode.'' Thus the system needs to be stable and modeled with constant features, so it is to be one particular thing 
%\label{ref:RNDWEkT6TNZ96}(Shackle, 1972, p.381).
\parencite[][p.381]{shackle_epistemics_1972}. %
 For example, regarding dice throwing, the dices need to be solid, with sufficient material strength. They cannot collapse with each and every throw because a~sufficiently large number of observations is needed for the proper modeling of probabilities 
%\label{ref:RNDIERiE0y73J}(Salmon, 1967, p.91).
\parencite[][p.91]{salmon_foundations_1967}. %
 An intermediate goal is to find out when the number becomes large enough that we can be sure that the system is not evolving and stays relatively ``stable''. An important part of this process is classification of events into classes, which is not simple 
%\label{ref:RNDe4EfpyVBJi}(Swinburne, 1971, pp.337–338).
\parencite[][pp.337–338]{swinburne_probability_1971}.%




Robert Higgs described in his article the notion of ``regime uncertainty'' 
%\label{ref:RNDFsza52V3nA}(Higgs, 1997).
\parencite[][]{higgs_regime_1997}. %
 Persistent change of economic policies increases uncertainty, causing capital and other economic resources to become idle. In a~similar manner, bad probability assignments due to government regulations can lead to illusions of certainty. They can create an impression of economic safety and push investors into malinvestments, which may end in capital consumption. Under specific conditions and assumptions probability analysis is a~solution to ignorance. But it is useful only if certain conditions are met. If those conditions do not exist it might be that probability calculus in response to government policy is not only not a~solution to ignorance, but even worse: it itself is a~source of pretense of knowledge, especially when it is based on past data, which are ``explosive'' in the Shacklean sense. Data may represent a~stable past trend, but at the same time an unsustainable, destructive trend for the future that is not visible in the past data.



Consider a~simple metaphorical Nassim Taleb's example of an owner feeding his turkey each day 
%\label{ref:RND9c8o9LePpr}(Taleb, 2007).
\parencite[][]{taleb_black_2007}. %
 Based on past behavior, the pattern of feeding hours might help us establish a~probability distribution: the likeliness that during the day the owner will show up to feed the turkey. But the probability calculus is based on important assumptions. Richard von Mises's argument was that one needs randomness and an infinite amount of trials; otherwise the calculus is just an approximation that may fail. Since that is never perfectly the case, the truer statement is that of his brother Ludwig: we do not know the individual characteristics of actions, but we know they are part of one class. Hence we start from this fact: we do not know something and yet we can accurately group and interpret historical data.



When the owner finally kills the turkey, we might conclude that we failed in our probability modeling and were surprised by a~black swan, or rather a~dead turkey. One possible conclusion is that it was a~completely unpredictable event. This would be a~comfortable explanation for the failed positivist model. Perhaps the overall prediction was fine, and we lacked sufficient experience to recognize the mistake. Hence in that case, black swans would be Shacklean demons attacking existing economic frameworks.



In the above case, calling the outcome a~black swan rests on an important assumption we made before our probability calculus: we do not know anything particular about events apart from that they are members of the same class. We can gain knowledge for feeding in the future only from past trends. But from a~different perspective, the opposite may be the case, for perhaps we do know something more, something qualitative, about the particular event than just that it is a~member of a~class. We know the purpose of the owner---why he kills the turkey. Focus on the repetitive homogenous data can lead to neglect of qualitative analysis, and create a~quasi-certainty. Repetition of that data may falsely suggest there is inherent stability in the non-evolving system.



In the late 1990's an investment fund Long-Term Capital Management (LTCM) believed itself to have found an El Dorado of business investments. It inferred from specific assumptions that it could flawlessly arbitrage government bonds. To Nobel Prize winners Robert Merton and Myron Scholes, both involved in the LTCM case, the crisis of 1997–1998 came like a~dead turkey, or black swan. We could give similar examples from the recent financial crisis. Additionally conducting an empirical analysis of past events, and constructing models built upon it with sophisticated RiskMetrics and Creditscoring programs, one would not have foreseen the Great Recession.



Such an approach is based on the assumption that we do not know more about price movements than that they are members of the same class. We do not see the potentially explosive aspect, and choose to hide important factors behind the notion of randomness. Yet there is something more to be found than randomness. From Knut Wicksell we know that the central bank cannot permanently reduce interest rates below market levels; and from Hayek (and for the mainstream, from Phelps and Friedman) we know that endogenous market forces counteract the interest-rate reduction, leading to a~recession and a~market correction\footnote{See: Wicksell 1962, Friedman 1968, Phelps 1967, Hayek 1969.}.



Similarly, in 1929 Irving Fisher famously declared that the values of stock market assets were too low. There are analyses still trying to prove him right 
%\label{ref:RNDoC4k3KYpQw}(see, for example, McGrattan and Prescott, 2003).
\parencite[see, for example,][]{mcgrattan_1929_2003}. %
 These analyses are based on the assumption that one can extrapolate future trends from historical data. Fisher's hypothesis is based on an analysis of data from 1921 to 1929 and the postulate that prices of assets, interest rates, and other significant variables were in equilibrium---that there was no tendency for the economic system to explode in the Shacklean sense.



Hence, as in our discussion of probability, one could assume that one does not know about particular variables, and construct a~seemingly viable model based on past data. It turns out, nevertheless, that one could know something about the variables, particularly about artificially low interest rates during 1921–1929, that caused an asset bubble. In that case the assets were not priced correctly; hence the extension of the trends from 1929 prices was also incorrect. If something was ``wrong'' with the data, it could mean that this ``wrongness'' could not go on forever. But this can be recognized only if we go beyond extrapolations from statistics.



The same is the case with debt and credit creation in the United States from the 1980s, which intensified after 2001. Existence of previous data is a~fact, but an extrapolation of it into the future through a~probability calculus assumes that underlying conditions are sustainable. Contrary to this we argue that we can know much more about existing cases than just that they are members of the same class. We know economic laws, which do not assume away Knightian uncertainty. Otherwise, if they do, and substitute it for objective probability calculus, they might lead to a~mere pretense of knowledge. They may become truly a~source of ignorance, rather than knowledge. For an example of how deceptive this probabilistic theory might be, consider Stiglitz's analysis of the mortgage market in 2002:



Specifically, historical data were used to create millions of potential future scenarios. […] These results regarding the risk-based capital standard are striking: They suggest that on the basis of historical experience, the risk to the government from a~potential default on GSE debt is \textit{effectively zero}. […] The first potential shortcoming is that the risk-based capital standard, while based on a~hypothetical economic shock significantly more severe than anything that the economy has actually experienced over the past forty years, may fail to reflect the probability of another Great Depression-like scenario. Fundamentally, the extremely rare events located in the tail of a~distribution are often quite difficult to analyze accurately. Interestingly, however, the Office of Management and Budget tested Fannie Mae's and Freddie Mac's capital adequacy in the early 1990s by subjecting their business activities to a~ten-year stress test that simulated the financial and economic conditions of the Great Depression. The test showed that if a~Depression lasted ten years, given 1990 levels of capital, both Fannie Mae and Freddie Mac had sufficient capital to survive. This result led OMB to conclude that in the event of a~severe nationwide economic downturn, the probability of either Fannie Mae or Freddie Mac defaulting would be ``\textit{close to zero}.'' 
%\label{ref:RNDoemulejcPj}(Stiglitz, Orszag and Orszag, 2002, p.5, emphasis added)
\parencite[][emphasis added]{stiglitz_implications_2002}%




In the foreword to this paper, Arne Christenson, senior vice president for regulatory policy of Fannie Mae, commented that probability of a~default was ``effectively zero.'' His approach was based on econometric analysis and simulations, which in their nature are the same as the equations used in the banking system, based on Basel regulatory rules. Under those regulations banks are supposed to measure risk and assign it to particular assets in order to protect themselves by raising a~sufficient amount of capital. The cornerstone of the problem lies in the fact that there is no universal probability calculus that one could apply to an economy subjected to credit expansions;\footnote{On a~more general level, take the case of the so called ``operational risk'': the risk that the employer will steal the funds from the institution. It begs Richard von Mises's question, how could such an incidental and unique event be assigned with an objective probability measure?} hence those ``probabilities'' and ``risks'' should be called what they truly are: subjective probabilities or judgments based on historical data. Most American banks before the crisis of 2008 achieved relatively high capital ratios, and yet they severely suffered from the crisis. Even though, on paper, risk was properly measured and secured by capital, the crisis strongly hit financial institutions. Apparently, regulations reduced uncertainty. In reality, uncertainty was hidden under the veil of pretense of knowledge and this led to capital consumption and property misusage.\footnote{See on this Jabłecki and Machaj 2009.}



\section{Model Risk versus Model Uncertainty}

One way to curtail uncertainty about grouping economic events into class probabilities is to subject the act of calculating itself to calculus. This is called model risk. To use again the coin-flipping example, it is possible that the coin may not be perfect---and the model builder may be wrong in assuming some probability distribution. The model-risk approach subjects model building itself to probability calculus to do away with the model's uncertainty. As we can see, this shifts the problem to another level, for now the act of model building itself has to be a~member of some homogenous class, and we would need repeatable experience. Sufficiently large numbers would produce statistics. Hence for example we could speculate that a~model of 50 percent probability of heads has a~90 percent chance of working, but other models, though less likely to succeed, are still possibly correct.\footnote{After the 2008 BASEL regulators introduced the necessity for capital reserves associated with ``model risk'' 
%\label{ref:RNDdETPOr7vSk}(Alexander and Sarabia, 2012, p.1295).
\parencite[][p.1295]{alexander_quantile_2012}.%
}



The above considerations are particularly relevant for financial markets. Derivatives markets depend heavily on theoretical models traders use in their transactions 
%\label{ref:RNDJYAW6LfBEh}(Green and Figlewski, 1999, p.1466).
\parencite[][p.1466]{green_market_1999}. %
 There are problems in estimating and specifying various types of risks. Since the risks are not easily verified, assessing how reliable the model is can be problematic. In proceeding to model model risk we could compare the model used to some imaginary ``correct'' model, or take one model and compare it to a~bunch of other models as though they were of one class 
%\label{ref:RNDlCi2uRJnbj}(Kerkhof, Melenberg and Schumacher, 2010, p.268).
\parencite[][p.268]{kerkhof_model_2010}. %
 Under the assumption that the ``correct model'' is still an idealized one and works as virtual reality with well-established parameters and structure\footnote{I~thank an anonymous referee for this point.}.



This comparison can help shield the company from risk, especially if it leads the company to create an additional capital reserve. But does this actually do away with uncertainty to such an extent as to potentially secure the economy from macroeconomic crises? It all depends on the underlying theory and how we treat the explosive aspect of credit expansions. Part of the problem with model risk comes with identifying the distribution in the tails---assigning probabilities to ``random'' and less predictable events (see Nassim Taleb)---or the existence of markets that are not ``perfect'' such that any arbitrage may take place. Another issue relates to the positivist reliance on observable data. Not all input parameters are observable 
%\label{ref:RNDgqQAZcAvNn}(Green and Figlewski, 1999, p.1467):
\parencite[][p.1467]{green_market_1999}:%




In particular, even if one has a~correctly specified model, using it requires knowledge of the volatility of the underlying asset over the entire lifetime of the contract. This creates a~formidable forecasting problem, for which neither the {\textquotedbl}best{\textquotedbl} estimation procedure nor the model risk characteristics of the resulting theoretical option values are known.



Futhermore according to Green and Figlewski three known sources of model risk are (1) tail distribution, (2) with non-observable input parameters (and hence wrong estimations resulting from the impossibility to have future knowledge of all asset changes) and also (3) non-continuous markets (and hence arbitrage is not working infinitely long to equilibrate). Yet their ``quantitative impact is not known.'' Therefore as Alexander and Sarbia 
%\label{ref:RNDlH8ND5Fwkx}(2012, pp.1295–1296)
\parencite*[][pp.1295–1296]{alexander_quantile_2012} %
 comment:



Outside of a~simulation environment, the concept of a~``true'' model against which one might assess model risk is meaningless. All we have is some observable data and our beliefs about the conditional and/or unconditional distribution of the random variable in question. As a~result, model risk can only be assessed relative to some benchmark model, which itself is a~matter for subjective choice.…



[O]utside of an experimental or simulation environment, we never know the ``true'' model for sure. In practice, all we can observe are realizations of the data generation processes for the random variables in our model. It is futile to propose the existence of a~unique and measurable ``true'' process because such an exercise is beyond our realm of knowledge. (emphasis added)



As Alexander and Sarabia discuss further, parameter uncertainty and model choice condition model risk; therefore model risk cannot assume true uncertainty away.\footnote{Especially in the case of exotic instruments 
%\label{ref:RNDOe9woNEgaw}(Hull and Suo, 2002, p.298).
\parencite[][p.298]{hull_methodology_2002}.%
} For those reasons the term ``model risk'' does not adequately apply to the nature of the problem inferred. The more proper name would be \textit{model uncertainty}, which would be a~qualitative margin of stating: there may be \textit{something} wrong with the model; \textit{something} that cannot be quantitatively expressed and compared to an imaginary perfect scenario.



Although data analyses can no doubt be helpful in risk assessments, exclusive focus on past data is not sufficient for good model choice. Same applies to a~meta approach of generating model risk with assigned probabilities of being successful with the chosen models. What we need is a~proper economic theory that helps to go beyond visible data, and allows us to notice the Shacklean epistemic probability, Misesian case probability, or Knightian uncertainty associated with something outside of empirically witnessed computation numbers.



\section{Conclusion}

The theory of probability is a~significant scientific tool that should not be underemphasized. Its utility, however, is based on correct recognition of its limits. Only then will probability analysis increase our knowledge and capability of prediction. If, on the other hand, we apply probability calculus to instances where it should not be, especially as we may in economics and regulatory policies, we can get erroneous results and cause even more ignorance than it aspires to reduce. Probability theorist are well aware of model limitations, therefore they try to develop notions of ``model risk'', which is in a~way extension of a~traditional approach to risk based on the notion of probability distributions.



Yet as we have seen, it cannot fully solve the problem of true Knightian uncertainty, reflecting the challenges of Misesian case probability. Therefore even if model risk may be helpful in tackling parametric mistakes, there still remains a~possibility that models do not capture some things that cannot be modelled. Literature of probability concepts created a~distinction between risk and uncertainty, hence it would be appropriate to use the term ``model uncertainty'' parallel to model risk, since some aspects cannot be parametrized under the notion of probability measurements. In other words, there is some non-measurable element in choosing correct and incorrect economic models and this also applies to meta considerations of inter-model comparisons. An element of ``true model uncertainty'', which is not subjected to similar calculus as model risk is. The main benefit of such an approach is to extend economic interpretations of true uncertainty and apply them also to broader model considerations.



\section{References}

Alexander, C. and Sarabia, J.M., 2012. Quantile Uncertainty and Value-at-Risk Model Risk. \textit{Risk Analysis}, [online] 32(8), pp.1293–1308. https://doi.org/10.1111/j.1539-6924.2012.01824.x.



Bricmont, J., 2002. \textit{Determinism, Chaos and Quantum Mechanics.} Available at: {\textless}https://www.dogma.lu/txt/JB-Determinism.pdf{\textgreater} [Accessed 8 October 2024].



Fetzer, J.H., 1977. A~world of dispositions. \textit{Synthese}, [online] 34(4), pp.397–421. https://doi.org/10.1007/BF00485648.



Fetzer, J.H., 1983. Probability and objectivity in deterministic and indeterministic situations. \textit{Synthese}, [online] 57(3), pp.367–386. https://doi.org/10.1007/BF01064703.



Garrison, R.W., 2001. \textit{Time and Money: The Macroeconomics of Capital Structure}. Foundations of the market economy. London; New York: Routledge.



Green, T.C. and Figlewski, S., 1999. Market Risk and Model Risk for a~Financial Institution Writing Options. \textit{The Journal of Finance}, [online] 54(4), pp.1465–1499. https://doi.org/10.1111/0022-1082.00152.



Hauwe, L. van den, 2007. \textit{John Maynard Keynes and Ludwig von Mises on Probability}. MPRA Paper, No. 6965. [online] MPRA - Munich Personal RePEc Archive. pp.1–46. Available at: {\textless}https://publicacion-digital.procesosdemercado.com/index.php/inicio/article/view/315{\textgreater} [Accessed 8 October 2024].



Higgs, R., 1997. Regime Uncertainty: Why the Great Depression Lasted So Long and Why Prosperity Resumed after the War. \textit{The Independent Review}, [online] 1(4), pp.561–590. Available at: {\textless}https://www.jstor.org/stable/24560785{\textgreater} [Accessed 8 October 2024].



Hoppe, H.-H., 2007. The Limits of Numerical Probability: Frank H. Knight and Ludwig von Mises and The Frequency Interpretation. \textit{The Quarterly Journal of Austrian Economics}, [online] 10(1), pp.1–20. https://doi.org/10.1007/s12113-007-9005-3.



Howson, C., 1995. Theories of Probability. \textit{The British Journal for the Philosophy of Science}, [online] 46(1), pp.1–32. https://doi.org/10.1093/bjps/46.1.1.



Hull, J. and Suo, W., 2002. A~Methodology for Assessing Model Risk and Its Application to the Implied Volatility Function Model. \textit{The Journal of Financial and Quantitative Analysis}, [online] 37(2), p.297. https://doi.org/10.2307/3595007.



Kerkhof, J., Melenberg, B. and Schumacher, H., 2010. Model risk and capital reserves. \textit{Journal of Banking \& Finance}, [online] 34(1), pp.267–279. https://doi.org/10.1016/j.jbankfin.2009.07.025.



Keynes, J.M., 1921. \textit{A~Treatise on Probability}. [online] London: Macmillan and Co. Available at: {\textless}https://catalog.hathitrust.org/api/volumes/oclc/182544.html{\textgreater} [Accessed 8 October 2024].



Keynes, J.M., 1939. Professor Tinbergen's method. \textit{The Economic Journal}, [online] 49(195), pp.558–577. https://doi.org/10.1093/ej/49.195.558.



Knight, F.H., 1971. \textit{Risk, Uncertainty and Profit}. Chicago: University of Chicago Press.



Krüger, L., 1986. Probability as a~Theoretical Concept in Physics. \textit{PSA: Proceedings of the Biennial Meeting of the Philosophy of Science Association}, [online] 1986(2), pp.273–287. https://doi.org/10.1086/psaprocbienmeetp.1986.2.192806.



Kyburg, H.E., 1966. Probability and Decision. \textit{Philosophy of Science}, [online] 33(3), pp.250–261. https://doi.org/10.1086/288097.



Lachmann, L.M., 1976. From Mises to Shackle: An Essay on Austrian Economics and the Kaleidic Society. \textit{Journal of Economic Literature}, [online] 14(1), pp.54–62. Available at: {\textless}https://www.jstor.org/stable/2722803{\textgreater} [Accessed 8 October 2024].



Lucas, R.E., 1977. Understanding business cycles. \textit{Carnegie-Rochester Conference Series on Public Policy}, [online] 5, pp.7–29. https://doi.org/10.1016/0167-2231(77)90002-1.



McGrattan, E.R. and Prescott, E.C., 2003. \textit{The 1929 Stock Market: Irving Fisher Was Right}. Research Department Staff Report No. 294. [online] Minneapolis: Federal Reserve Bank of Minneapolis. Available at: {\textless}https://core.ac.uk/download/pdf/6717612.pdf{\textgreater} [Accessed 8 October 2024].



Mises, L. von, 1966. \textit{Human Action: A~Treatise on Economics}. 3., rev. ed ed. Chicago: Regnery.



Mises, R. von, 1957. \textit{Probability, Statistics and Truth}. 2d rev. English ed. ed. New York: Macmillan Co.



Moser, P.K., 1988. The foundations of epistemological probability. \textit{Erkenntnis}, [online] 28(2), pp.231–251. https://doi.org/10.1007/BF00166444.



Reeves, T.V., 1988. A~Theory of Probability. \textit{The British Journal for the Philosophy of Science}, [online] 39(2), pp.161–182. https://doi.org/10.1093/bjps/39.2.161.



Salmon, W.C., 1967. \textit{The Foundations of Scientific Inference}. [online] Pittsburgh: University of Pittsburgh Press. https://doi.org/10.2307/j.ctt5hjqm2.



Samuelson, P.A., 2008. Asymmetric or symmetric time preference and discounting in many facets of economic theory: A~miscellany. \textit{Journal of Risk and Uncertainty}, [online] 37(2–3), pp.107–114. https://doi.org/10.1007/s11166-008-9047-8.



Shackle, G.L.S., 1972. \textit{Epistemics \& Economics: A~Critique of Economic Doctrines}. Cambridge: Cambrdige University Press.



Shanks, N., 1993. Time and the propensity interpretation of probability. \textit{Journal for General Philosophy of Science}, [online] 24(2), pp.293–302. https://doi.org/10.1007/BF00764391.



Stiglitz, J.E., Orszag, J.M. and Orszag, P.R., 2002. Implications of the New Fannie Mae and Freddie Mac Risk-based Capital Standard. \textit{Fannie Mae Papers}, 1(2), pp.1–10.



Swinburne, R.G., 1971. The Probability of Particular Events. \textit{Philosophy of Science}, [online] 38(3), pp.327–343. https://doi.org/10.1086/288374.



Taleb, N.N., 2007. \textit{The Black Swan: The Impact of the Highly Improbable}. New York: Random House.



Tucker, W.T., 1965. Max Weber's \textit{Verstehen}. \textit{The Sociological Quarterly}, 6(2), pp.157–164. https://doi.org/10.1111/j.1533-8525.1965.tb01649.x.

\end{document}


\begin{artengenv}{Robert W. McGee}
	{Taxation and the philosophy of Frédéric Bastiat}
	{Taxation and the philosophy of Frédéric Bastiat}
	{Taxation and the philosophy of Frédéric Bastiat}
	{Fayetteville State University\label{mcgee-first}}
	{Frederic Bastiat (1801-1850) was an economist and journalist. A~member of the French Liberal School, he is best known for his free trade ideas and his philosophy of law. Mark Blaug ranks him as one of the 100 greatest economists before Keynes. Schumpeter called him a~brilliant economic journalist. Haney devoted a~chapter of his History of Economic Thought to Bastiat.
	
	
	
	Although Bastiat is known for his work on free trade and the philosophy of law, he also wrote on other topics. To date, almost no one has examined his views on taxation. The purpose of this paper is to fill that gap in the literature.
	}
	{taxation, utilitarian, rights theory, public finance, French liberalism, Frédéric Bastiat.}






\section{Introduction}

\lettrine[loversize=0.13,lines=2,lraise=-0.03,nindent=0em,findent=0.2pt]%
{F}{}rédéric Bastiat (1801-1850) was born in France and spent most of his life there, although he traveled to England to visit with Cobden and Bright and fully supported their free trade movement. Much of their correspondence was later published as an entire volume of his seven-volume \textit{Oeuvres Complètes} (1864). He died in Rome on December 24, 1850 of tuberculosis.



Bastiat wore many hats. He was an economic journalist and philosopher of law. He was a~gentleman farmer. He was a~justice of the peace and later served in the French Chamber of Deputies in Paris. He was a~husband and father 
%\label{ref:RNDgF5xD6mtga}(Bidet, 1906; Bastiat, 1889; Haney, 1949; Imbert, 1913; Nouvion, 1905; Roche, 1971; 1993; Ronce, 1905; Russell, 1959; 1969).
\parencites[][]{bidet_frederic_1906}[][]{bastiat_f_1889}[][]{haney_history_1949}[][]{imbert_frederic_1913}[][]{nouvion_frederic_1905}[][]{roche_frederic_1971}[][]{roche_free_1993}[][]{ronce_frederic_1905}[][]{russell_frederic_1959}[][]{russell_frederic_1969}. %
 For six years, starting at age seventeen, he worked in his uncle's counting house, which exposed him to accounting 
%\label{ref:RND1ayKh01aSu}(Hazlitt, 1964, p.xi).
\parencite[][p.xi]{bastiat_introduction_1964}.%




Richman 
%\label{ref:RNDXkS7VtnW7W}(1998, p.ix)
\parencite*[][p.ix]{bastiat_foreward_1998} %
 stated that he was a~legal philosopher of the first rank. Skousen 
%\label{ref:RNDjsccfn9meM}(2001, p.59)
\parencite*[][p.59]{skousen_making_2001} %
 compared him to Benjamin Franklin or Voltaire for his integrity and the purity and elegance of his writing style. Hébert 
%\label{ref:RNDnEQ1zpcgQh}(1987, p.205)
\parencite*[][p.205]{hebert_bastiat_2016} %
 considers him to be unrivaled in the way he exposed fallacies 
%\label{ref:RNDakURdQpQoB}(Skousen, 2001, p.59).
\parencite[][p.59]{skousen_making_2001}. %
 Schumpeter 
%\label{ref:RNDTZENZLsD9f}(1954, p.500)
\parencite*[][p.500]{schumpeter_history_1954} %
 called him one of the most brilliant economic journalists who ever lived, although he did not consider him to be a~first-rate theorist. Blaug 
%\label{ref:RNDBOLdGCRMkR}(1986)
\parencite*[][]{blaug_great_1986} %
 ranks Bastiat as one of the 100 greatest economists before Keynes. According to some historians of economic thought, political economy reached its apogée with Bastiat. ``After Bastiat, Reybaud could state that work in political economy had almost been exhausted and that there was nothing else to discover.'' 
%\label{ref:RND31otxCAyvR}(Screpanti and Zamagni, 1993, p.2).
\parencite[][p.2]{screpanti_outline_1993}. %
 That turned out not to be true, of course, but that was the thinking at the time.



Although classified as a~member of the French Liberal School or Optimist School 
%\label{ref:RNDOb4ofnxmaQ}(Cossa, 1893, pp.376–382; Gide and Rist, 1948, pp.329–354),
\parencites[][pp.376–382]{cossa_introduction_1893}[][pp.329–354]{gide_history_1948}, %
 he is also considered to be a~forerunner of the Austrian School of Economics 
%\label{ref:RNDGILdpqE6fv}(DiLorenzo, 1999)
\parencite[][]{holcombe_frederic_1999} %
 because of the similarity of his methodology to theirs. Some of his essays applied the theory of opportunity cost, which was unusual at the time (1840s), since the theory of opportunity costs was not fully developed until Carl Menger, the founder of the Austrian School of Economics 
%\label{ref:RND7VqZ5QNcJ3}(Menger, 1871).
\parencite[][]{menger_grundsatze_1871}.%




Although the concept of opportunity cost is attributed to the Austrian School of Economics 
%\label{ref:RNDjjr8PxKcAo}(Buchanan, 1973, p.14; Haney, 1949, p.895; Schumpeter, 1954, p.917),
\parencites[][p.14]{buchanan1973introduction}[][p.895]{haney_history_1949}[][p.917]{schumpeter_history_1954}, %
 its origins can be traced back to Cantillon's \textit{Essai sur la Nature du Commerce en Général} 
%\label{ref:RNDOG85niwr8C}(1755)
\parencite*[][]{cantillon_essai_1755} %
 as well as the work of Turgot 
%\label{ref:RNDY6fKBB5xss}(Rothbard, 1995, p.391; 1999, pp.34, 40)
\parencites[][p.391]{rothbard_economic_1995}[][pp.34, 40]{holcombe_rj_1999} %
 and Bastiat 
%\label{ref:RND0FWA8W2Ggo}(DiLorenzo, 1999, pp.62–63),
\parencite[][pp.62–63]{holcombe_frederic_1999}, %
 all of whom were French economists.



The classic exposition of Bastiat's application of opportunity cost is in his essay, \textit{What Is Seen and What Is Not Seen} 
%\label{ref:RND7O08FnOYll}(Bastiat, 1850; 1964c, pp.1–50; 2007, U: 1-48).
\parencites[][]{bastiat_ce_1850}[][pp.1–50]{bastiat_selected_1964}[][U:~1-48]{bastiat_bastiat_2007}. %
 In this essay, Bastiat applies the theory of opportunity cost to a~number of issues, including destruction of property, military demobilization, taxes, theater and the fine arts, public works, middlemen, restraints on trade, machinery, credit and several other topics.



Opportunity cost might be defined as ``the sacrifice of the utility of those other things which we could have had from the resources that went into the one we did produce.'' 
%\label{ref:RNDYbz6nKKUhn}(Schumpeter, 1954, p.917).
\parencite[][p.917]{schumpeter_history_1954}. %
 Stated more simply, ``Opportunity cost is income of a~foregone opportunity.'' 
%\label{ref:RND1sysiFarAm}(Magni, 2009).
\parencite[][]{magni_splitting_2009}. %
 Friedrich von Wieser (1851-1926) is credited with inventing the term 
%\label{ref:RNDTXqHiZZ4HX}(Skousen, 2001, p.184),
\parencite[][p.184]{skousen_making_2001}, %
 although Bastiat actually applied the concept before von Wieser was born.



The importance of applying opportunity cost to public policy issues cannot be overstated. Much public policy debate ignores the issue of opportunity cost. Economists, politicians and the media almost uniformly ignore some affected groups when they try to determine public policy positions. Bastiat's methodology makes a~serious effort to include all affected groups. Thus, his essay, \textit{What Is Seen and What Is Not Seen}, remains an important, if neglected, piece of literature.



He was a~vehement opponent of protectionism and socialism and much of his writing attacked one or the other. His book, \textit{The Law} 
%\label{ref:RND7KqnjDq9J4}(1998)
\parencite*[][]{bastiat_foreward_1998} %
 is required reading in some Tea Party circles 
%\label{ref:RNDj6ymMN9gqr}(Zernike, 2010).
\parencite[][]{zernike_shaping_2010}.%




He equated socialism with a~government that goes beyond its role of protecting life, liberty and property and ventures into the realm of redistribution. He debated the socialists of his time, most notably Proudhon, with whom he exchanged a~series of letters 
%\label{ref:RNDLvnnxp63dC}(Bastiat, 1873b).
\parencite[][]{bastiat_sophismes_1873}. %
 Unfortunately, that debate has not been discussed in the English literature to any great extent, although Imbert 
%\label{ref:RNDa8BYpwRzuE}(1913, pp.57–66)
\parencite*[][pp.57–66]{imbert_frederic_1913} %
 and de Nouvion 
%\label{ref:RNDAZp8LP3T0U}(1905, pp.256–269)
\parencite*[][pp.256–269]{nouvion_frederic_1905} %
 discussed it in French and Mülberger 
%\label{ref:RNDl2y5LdxKKF}(1896)
\parencite*[][]{mulberger_kapital_1896} %
 wrote about it extensively in German.



Much of his work, in the original French, is now available on the internet 
%\label{ref:RNDlLvm30yr33}(Bastiat, 1850; 1861; 1862b; 1862a; 1864; 1870; 1873a; 1873b).
\parencites[][]{bastiat_ce_1850}[][]{bastiat_essais_1861}[][]{bastiat_libre-echange_1862}[][]{bastiat_correspondance_1862}[][]{bastiat_cobden_1864}[][]{bastiat_harmonies_1870}[][]{bastiat_ce_1873}[][]{bastiat_sophismes_1873}. %
 About one-third of his works have been translated into English 
%\label{ref:RND0rdHKvAWlE}(Bastiat, 1926; 1964c; 1964b; 1964a; 1991; 1998; 2007).
\parencites[][]{bastiat_bastiat_1926}[][]{bastiat_selected_1964}[][]{bastiat_economic_1964}[][]{bastiat_economic_1964-1}[][]{bastiat_providence_1991}[][]{bastiat_law_1998}[][]{bastiat_bastiat_2007}. %
 The person most responsible for introducing Bastiat to the English speaking world is Dean Russell, who wrote a~dissertation 
%\label{ref:RNDqSGjNLI42a}(Russell, 1959)
\parencite[][]{russell_frederic_1959} %
 and two books 
%\label{ref:RNDTtLQiD9G7W}(Russell, 1969; 1985)
\parencites[][]{russell_frederic_1969}[][]{russell_government_1985} %
 about Bastiat and his work. Hendrick 
%\label{ref:RNDOeWtxbhLkH}(1987)
\parencite*[][]{hendrick_frederic_1987} %
 also wrote a~doctoral dissertation devoted to Bastiat's work, although he did not publish any portion of it. Buccino 
%\label{ref:RND3G8yKECCfw}(1990)
\parencite*[][]{buccino_examination_1990} %
 discussed some of Bastiat's philosophy in her study of other classical political economists in her doctoral dissertation.



George Charles Roche, III, an American historian, wrote two books about Bastiat 
%\label{ref:RNDiZsZB1Uk3m}(Roche, 1971; 1993).
\parencites[][]{roche_frederic_1971}[][]{roche_free_1993}. %
 Bidet 
%\label{ref:RNDl44mvan1N9}(1906),
\parencite*[][]{bidet_frederic_1906}, %
 DeFoville 
%\label{ref:RNDNVpLIQhS4V}(1889),
\parencite*[][]{defoville1889oeuvres}, %
 Imbert 
%\label{ref:RNDQTGSchVWka}(1913),
\parencite*[][]{imbert_frederic_1913}, %
 de Nouvion 
%\label{ref:RND2VF7mLIo3y}(1905)
\parencite*[][]{nouvion_frederic_1905} %
 and Ronce 
%\label{ref:RND7Xfpta7olv}(1905)
\parencite*[][]{ronce_frederic_1905} %
 wrote books about him in French. Henry Hazlitt 
%\label{ref:RNDvzMjwg5kfT}(1946; 1979)
\parencites*[][]{hazlitt_economics_1946}[][]{hazlitt_economics_1979} %
 applied Bastiat's methodology to a~number of economic policy issues in the mid- twentieth century. Russell 
%\label{ref:RNDcxTN2E0NDd}(1985)
\parencite*[][]{russell_government_1985} %
 took a~similar approach in the mid-1980s.



Although best known for his work in trade and the philosophy of law, he wrote on other topics as well. To date, no one has examined his views on taxation. The purpose of this paper is to fill that gap in the literature.



\section{Two philosophical approaches}

Bastiat was both a~utilitarian 
%\label{ref:RND9IzYgkpHFM}(Bastiat, 1850; 1862b; 1864; 1870; 1873a; 1873b; 1964b; 1964a; 1964c; 2007; 2010)
\parencites[][]{bastiat_ce_1850}[][]{bastiat_libre-echange_1862}[][]{bastiat_cobden_1864}[][]{bastiat_harmonies_1870}[][]{bastiat_ce_1873}[][]{bastiat_sophismes_1873}[][]{bastiat_economic_1964}[][]{bastiat_economic_1964-1}[][]{bastiat_selected_1964}[][]{bastiat_bastiat_2007}[][]{bastiat_collected_2010} %
 and a~rights theorist 
%\label{ref:RNDQ5flWHttis}(Bastiat, 1873a, pp.342–393; 1998; 2007, I: 49-94).
\parencites[][pp.342–393]{bastiat_ce_1873}[][]{bastiat_law_1998}[][I:~49-94]{bastiat_bastiat_2007}. %
 In order to more fully understand his views on taxation it is necessary to take a~few minutes to discuss his two philosophical approaches to various public policy issues.



\subsection{Rights theory }



The most comprehensive presentation of his rights theory is contained in \textit{The Law} 
%\label{ref:RND2T3VAzAC6i}(Bastiat, 1873a, pp.342–393; 1998; 2007, I: 49-94).
\parencites[][pp.342–393]{bastiat_ce_1873}[][]{bastiat_law_1998}[][I:~49-94]{bastiat_bastiat_2007}. %
 In this essay, first published as a~pamphlet in 1850, Bastiat outlines his basic legal philosophy, which is similar to that of Locke 
%\label{ref:RNDZcjNPFwZHM}(1689)
\parencite*[][]{locke_two_1689} %
 and Nozick 
%\label{ref:RNDwFQBa6BJf4}(1974)
\parencite*[][]{nozick_anarchy_1974} %
 in many ways. All three believed that government should be limited to the defense of life, liberty and property, which could be labeled a~night watchman state.



Justice reigns when the law is confined to these three functions. When the law goes beyond these three basic functions and into the realm of the redistribution of wealth, the result is injustice.



Bastiat believed that liberty and property existed prior to government. Indeed, the reason governments were formed was to protect life, liberty and property. His position rejects the view of legal positivists, who believe that there is no such thing as inherent rights and that all rights come from government 
%\label{ref:RNDjOtOHRJPAi}(Bentham, 1843; Austin, 1869; Fuller, 1969; Kramer, 1999; Marmor, 2001; Waldron, 1987).
\parencites[][]{bentham_anarchical_1843}[][]{austin_lectures_1869}[][]{fuller_morality_1969}[][]{kramer_defense_1999}[][]{marmor_positive_2001}[][]{waldron_nonsense_1987}.%




In this regard, his view is similar to that of Locke 
%\label{ref:RNDE9eKcDlf3d}(1689)
\parencite*[][]{locke_two_1689} %
 and Nozick 
%\label{ref:RNDFz5IReSonN}(1974),
\parencite*[][]{nozick_anarchy_1974}, %
 who also reject legal positivism. Although Bastiat mentions ``God'' several times in his writings, he does not take the position that property rights are part of God's plan to make the world a~better place. Such a~position would be closer to that of Locke, who was a~natural law theorist in the Protestant tradition. It could fairly be said that Bastiat was a~secular rights theorist, since his views could be accepted and applied by individuals of any religion, or no religion, although Meredith 
%\label{ref:RNDzsfFbLsrga}(2009)
\parencite*[][]{meredith_taxation_2009} %
 places him in the Christian tradition of normative natural law along the lines of Augustine and Aquinas.



He is against entitlements, such as the right to government health care, government pensions, government subsidies, government enforced minimum wages, government provided welfare, protective tariffs, government education, and so forth. These entitlements are examples of positive law, where the right is not inherent, but rather comes from government. In order for one individual to have a~positive right to something, a~negative right (such as the right to property or the right to contract) of someone else must be violated. In these cases, one person lives at the expense of another.



\begin{quote}
L'État, c'est la grande fiction à travers laquelle tout le monde s'efforce de vivre aux dépens de tout le monde. 
%\label{ref:RNDOvRPCU5ONu}(Bastiat, 1873a, p.332)
\parencite[][p.332]{bastiat_ce_1873} %
 [The state is that great fiction through which everyone tries to live at the expense of everyone else.]
\end{quote}



Governments may not legitimately engage in charity. Engaging in government charity is false philanthropy. True philanthropy involves the giving of one's own property for some worthy cause. False philanthropy involves taking one person's property and giving it to another for what some politician or bureaucrat considers to be a~worthy cause.



Individuals have the right to defend their rights to life, liberty and property. That being the case, it follows logically that groups of individuals can band together to defend these individual rights. Forming defense organizations to defend these rights often makes sense, since it increases efficiency. These defense organizations are often governments, but can also be private defense organizations.



These governments or private defense organizations possess no rights that the individuals who formed them do not possess. Just as an individual has no right to steal, neither does a~government have a~right to steal. If an individual forcibly takes someone else's property, it is theft, which Bastiat refers to as illegal plunder. When a~government takes one person's property and gives it to another person, Bastiat calls it illegal plunder.



\begin{quote}
Under the pretense of organization, regulation, protection, or encouragement, the law takes property from one person and gives it to another; the law takes the wealth of all and gives it to a~few -- whether farmers, manufacturers, shipowners, artists, or comedians. 
%\label{ref:RNDXBZSHuGBWQ}(Bastiat, 1998, p.13)
\parencite[][p.13]{bastiat_law_1998}%
\end{quote}




For Bastiat, government expenditures are just and legitimate only if made for the general welfare, such as the protection of life, liberty and property. Justice, defense and public works expenditures may be justified, but not much else 
%\label{ref:RNDcUGp619bR8}(Braun and Blanco, 2011).
\parencite[][]{braun_bastiat_2011}. %
 Expenditures for special interests such as farmers, manufacturers, artists, students or anyone else are illegitimate because the government takes property from some and gives it to others. All special interest legislation constitutes a~form of legal plunder and any legislation that does not benefit the vast majority constitutes special interest legislation.



\begin{quote}
But how is this legal plunder to be identified? Quite simply. See if the law takes from some persons what belongs to them, and gives it to other persons to whom it does not belong. See if the law benefits one citizen at the expense of another by doing what the citizen himself cannot do without committing a~crime.



Then abolish this law without delay, for it is not only an evil itself, but also it is a~fertile source for further evils because it invites reprisals. If such a~law -- which may be an isolated case -- is not abolished immediately, it will spread, multiply, and develop into a~system. 
%\label{ref:RNDUMFlokbDKY}(Bastiat, 1998, p.17)
\parencite[][p.17]{bastiat_law_1998}%
\end{quote}


\begin{quote}
When a~portion of wealth is transferred from the person who owns it -- without his consent and without compensation, and whether by force or by fraud -- to anyone who does not own it, then I~say that property is violated; that an act of plunder is committed. 
%\label{ref:RNDWq8TJ2zRot}(Bastiat, 1998, p.22)
\parencite[][p.22]{bastiat_law_1998}%
\end{quote}




\subsection{Utilitarianism}



Bastiat was also a~utilitarian. What distinguishes Bastiat's version of utilitarianism from some other versions is that Bastiat made a~sincere attempt to determine the effect a~policy would have on all groups in both the long-run and the short-run. On the first page of his \textit{Selected Essays on Political Economy} 
%\label{ref:RND8PcUAYIFfM}(Bastiat, 1964c, p.1; 2007, I, p.1)
\parencites[][p.1]{bastiat_selected_1964}[][I, p.1]{bastiat_bastiat_2007} %
 he states that:



\begin{quote}
There is only one difference between a~bad economist and a~good one: the bad economist confines himself to the \textit{visible} effect; the good economist takes into account both the effect that can be seen and those effects that must be \textit{foreseen}. 
%\label{ref:RNDJMRpVTcWNE}(Bastiat, 1964c, p.1)
\parencite[][p.1]{bastiat_selected_1964}%
\end{quote}




He goes on the elaborate on this methodology and provides examples in several of his works 
%\label{ref:RNDW4BCke9c9t}(Bastiat, 1850; 1870; 1873a; 1873b; 1964b; 1964a; 1964c; 2007).
\parencites[][]{bastiat_ce_1850}[][]{bastiat_harmonies_1870}[][]{bastiat_ce_1873}[][]{bastiat_sophismes_1873}[][]{bastiat_economic_1964}[][]{bastiat_economic_1964-1}[][]{bastiat_selected_1964}[][]{bastiat_bastiat_2007}.%




\section{Views on taxation and public finance}

\begin{flushright}
``The state can give nothing to the citizens that it has not first taken from them.'' 
%\label{ref:RNDO7WWnl5Pkl}(Bastiat, 1964c, p.183)
\parencite[][p.183]{bastiat_selected_1964}%
\end{flushright}




According to one journalist, Bastiat ``argues that governments are essentially stealing when they tax their citizens to spend on welfare, infrastructure or public education 
%\label{ref:RNDVR77wY4A9G}(Zernike, 2010).
\parencite[][]{zernike_shaping_2010}. %
 However, this statement is not quite accurate. Bastiat was not against spending for some public works 
%\label{ref:RNDhoySKuedbl}(Bastiat, 1964b, p.46)
\parencite[][p.46]{bastiat_economic_1964} %
 and he thought that spending for national defense and justice were acceptable uses of tax funds 
%\label{ref:RNDouvGYzHrci}(Bastiat, 1964c, p.184).
\parencite[][p.184]{bastiat_selected_1964}.%




While taxes may be raised for the defense of life, liberty and property, they may not be raised for redistributive purposes. A~redistributive tax system is inherently unjust because it uses force to take property from its rightful owners and distributes it to those who have no just claim on it. Redistributive taxation is a~form of legal plunder.



However, Bastiat was not against all forms of taxation. Taxes were justified if the people whose salaries they paid rendered services to the taxpayers that were equal to what they were paid, in other words, if the people got their moneysworth from their taxes 
%\label{ref:RNDS9qEUqX5PK}(Bastiat, 1964c, p.182).
\parencite[][p.182]{bastiat_selected_1964}. %
 Economists who are familiar with the relative costs and benefits of privatization might be quick to assert that the citizenry seldom, if ever, gets its moneysworth from government, since the private sector can do just about anything faster, cheaper and better than government 
%\label{ref:RNDmwcdkh1Cdz}(Finley, 1989; Ohashi and Roth, 1980; Pirie, 1988; Savas, 1982),
\parencites[][]{finley_public_1989}[][]{ohashi_privatization_nodate}[][]{pirie_privatization_1988}[][]{savas_privatizing_1991}, %
 but Bastiat did not raise that question, since privatization was not an issue in the 1840s, probably because the state was relatively small at the time in terms of the institutions and infrastructure that it owned that could be transferred to the private sector.



Then there is the question of whether it can be determined whether the people actually got their moneysworth from government services, since value is a~subjective thing. It's probably true to say that there is no way to determine whether every individual received equal value for government services rendered, but it can be assumed that some individuals received more in services than what they paid in taxes while others received less than what they paid. That being the case, it would be impossible to determine whether taxes could be justified.



\subsection{Progressive taxation}



Bastiat identified progressive taxation as a~form of plunder 
%\label{ref:RNDMMlIQpes1v}(Bastiat, 1998, pp.18, 27).
\parencite[][pp.18]{bastiat_foreward_1998}. %
 Presumably, he would approve of a~flat tax, provided the funds spent were limited to the defense of life, liberty and property. He strongly opposed the Marxist concept, ``From each according to his ability, to each according to his needs.'' 
%\label{ref:RNDjDbswalsUx}(Marx, 1875)
\parencite[][]{marx_kritik_1875} %
 Marx and Engels advocated both a~heavy, progressive income tax and a~100 percent inheritance tax 
%\label{ref:RNDem3WxeKsrN}(Marx and Engels, 1848).
\parencite[][]{marx_manifest_1848}. %
 Unless Bastiat could read German, we can be sure that he did not read \textit{The Communist Manifesto} 
%\label{ref:RNDyyv1lKuX9y}(Marx and Engels, 1848)
\parencite[][]{marx_manifest_1848} %
 because the French and English translations did not appear until after his death 
%\label{ref:RND8kmSshTBjS}(see Marx and Engels, 2010),
\parencite[see][]{marx_communist_2010}, %
 but the ideas Marx and Engels espoused in that document were circulating in Europe during Bastiat's lifetime.



\subsection{Using taxes as a~means of equalizing wealth}



Bastiat viewed the use of the tax system as a~means of equalizing wealth as communism 
%\label{ref:RNDIc1TXkx8IV}(Bastiat, 1964c, p.111).
\parencite[][p.111]{bastiat_selected_1964}. %
 He was against the notion that disparities of wealth should be reduced through the tax system.



\subsection{Gift taxes}



Bastiat viewed gift taxes as a~violation of property rights:



\begin{quote}
Exchange, like property, is a~natural right. Every citizen who has produced or acquired a~product should have the option of applying it immediately to his own use or of giving it to whoever on the face of the earth consents to give him in exchange the object of his desires. To deprive him of this faculty, when he has committed no act contrary to public order and good morals, and solely to satisfy the convenience of another citizen, is to legitimize an act of plunder and to violate the law of justice. 
%\label{ref:RNDzjuKu0y4K5}(Bastiat, 1964c, p.112)
\parencite[][p.112]{bastiat_selected_1964}%
\end{quote}




\subsection{Inheritance taxes}



Bastiat was against inheritance taxes, which he regarded as a~violation of property rights. Property comes into existence as the result of labor. It is the fruit of one's labor, which can be passed on to others 
%\label{ref:RNDRITReuo711}(Bastiat, 1964c, pp.188–193).
\parencite[][pp.188–193]{bastiat_selected_1964}.%




\begin{quote}
The right of inheritance, against which so much has been objected of late, is one of the forms of gift, and assuredly the most natural of all. That which a~man has produced, he may consume, exchange, or give. What can be more natural than that he should give it to his children? It is this power, more than any other, that inspires him with the drive to labor and to save. Do you know why the principle of right of inheritance is thus called in question? Because it is imagined that the property thus transmitted is plundered from the masses. This is a~fatal error. 
%\label{ref:RNDL1Xg5XbOj9}(Bastiat, 2007, I, p.142)
\parencite[][p.142]{bastiat_bastiat_2007}%
\end{quote}




\subsection{Using taxes to stimulate economic activity}



Bastiat opposed the use of tax money to stimulate the economy for two reasons: (1) it was a~form of redistribution of wealth, and therefore legalized plunder, and (2) it did not work. He did not oppose the use of taxes to provide legitimate services, whatever they may be (i.e. services that benefitted the vast majority of the people), but he did oppose using taxes to prime the pump, so to speak, as Keynesian economists advocate. 
%\label{ref:RNDb4zD0FTgTR}(Bastiat, 1964c, pp.8–9, 16).
\parencite[][pp.8–9, 16]{bastiat_selected_1964}. %
 Every hundred sous (a French monetary unit at the time) a~Frenchman gives to support the salary of some government bureaucrat is 100 sous that he cannot spend himself 
%\label{ref:RND2E1glbqoo2}(Bastiat, 1964c, p.8).
\parencite[][p.8]{bastiat_selected_1964}. %
 The transfer is merely from one person's pocket to that of another. Total spending and total economic activity do not increase.



While this example may seem to be so obvious that it is hardly worthy of mention, the Keynesian multiplier theory 
%\label{ref:RNDy06zMPLaFN}(Keynes, 1936)
\parencite[][]{keynes_general_1936} %
 is based on the belief that increasing government spending results in a~multiplier effect that increases total economic activity. In fact, increased government spending results in less private sector spending. If the additional funds are raised in the form of borrowing rather than taxes, the result does not change. A~detailed examination of this phenomenon is beyond the scope of the present paper, but this topic has been covered in depth elsewhere 
%\label{ref:RND5ErxIxZb8j}(Ahiakpor, 2000; Dimand, 1997; 2000; Hazlitt, 1946; 1959; 1960; 1979; Hegeland, 1954; Hutt, 1963; 1979; Skousen, 1992; Terborgh, 1968).
\parencites[][]{ahiakpor_hawtrey_2000}[][]{dimand_hawtrey_1997}[][]{dimand_hawtrey_2000}[][]{hazlitt_economics_1946}[][]{hazlitt_failure_1959}[][]{hazlitt_critics_1960}[][]{hazlitt_economics_1979}[][]{hegeland_multiplier_1954}[][]{hutt_keynesianism--retrospect_1963}[][]{hutt_keynesian_1979}[][]{skousen_dissent_1992}[][]{terborgh_new_1968}.%




In Bastiat's time the argument was made that a~troop demobilization would result in increased unemployment. What would happen to the troops if they were demobilized? The reply was that they would become unemployed. As they returned to their home towns they would depress labor rates.



The opposite side of the coin is that French taxpayers would be relieved of paying a~hundred million francs. But the army consumes bread, wine, clothes and weapons, and such purchases spread throughout the economy. All this commercial activity would come to an end if the soldiers went home. Thus, the army must be maintained for economic reasons, even though the soldiers are not needed, or so the argument goes 
%\label{ref:RNDn74wZdSbIx}(Bastiat, 1964c, pp.4–5).
\parencite[][pp.4–5]{bastiat_selected_1964}.%




As Bastiat would say, what is seen is 100,000 soldiers who live well and who provide a~living for their suppliers. What is not seen is the fact that the hundred million francs used to support those soldiers cannot be used to support the taxpayers who are providing the funds.



If the soldiers return to their home towns, what is seen is 100,000 unemployed men being dumped into the labor market, causing wages to become depressed and deepening unemployment. What is not seen is the hundred million francs that are now free to hire those unemployed soldiers. Since the taxpayers are no longer being taxed to support soldiers who create no products or services, they are free to employ those soldiers, who will now be able to produce something. Overall production will increase because the soldiers, who were paid to march back and force, will now be producing something. All of society will benefit 
%\label{ref:RNDoqCIEYBTNp}(Bastiat, 1964c, pp.6–7).
\parencite[][pp.6–7]{bastiat_selected_1964}.%




\subsection{Tariffs}



Tariffs are a~form of taxation, in the sense that they raise revenue for governments. Prior to the adoption of the Sixteenth Amendment to the U.S. Constitution in 1913, tariffs were one of the major sources of revenue in the United States and were the major source of revenue in many other countries prior to the income tax 
%\label{ref:RND1clfVc4Zud}(Webber and Wildavsky, 1986, pp.269–270).
\parencite[][pp.269–270]{webber_history_1986}. %
 However, governments often use tariffs for a~more sinister reason: to protect domestic industry from foreign competition. This use (abuse) of tariffs has been present in history ever since tariffs were first imposed 
%\label{ref:RNDdGkZc37UKS}(Webber and Wildavsky, 1986).
\parencite[][]{webber_history_1986}.%




Tariffs are a~form of subsidy, special interest legislation, since they help one small segment of the domestic population (domestic producers) at the expense of the general public. Bastiat was against all tariffs because he regarded them as legalized plunder 
%\label{ref:RNDhyUtB62p9V}(Bastiat, 1861; 1862b; 1862a; 1864; 1873a; 1873b; 1964b; 1964a; 1964c; 1998; 2007).
\parencites[][]{bastiat_essais_1861}[][]{bastiat_libre-echange_1862}[][]{bastiat_correspondance_1862}[][]{bastiat_cobden_1864}[][]{bastiat_ce_1873}[][]{bastiat_sophismes_1873}[][]{bastiat_economic_1964}[][]{bastiat_economic_1964-1}[][]{bastiat_selected_1964}[][]{bastiat_law_1998}[][]{bastiat_bastiat_2007}.%




\begin{quote}
There are two potential causes of revolution in the United States: slavery and the high protective tariff…In regard to the tariff question the law says: ``I shall create an armed force, at the citizens' expense, not to make sure that their transactions are free, but to make sure that they are not free, to impair the equivalence of services, so that one citizen may have the liberty of two, and that another may have none at all.'' 
%\label{ref:RND9Bjc9YQet1}(Bastiat, 1964a, p.462)
\parencite[][p.462]{bastiat_economic_1964-1}%
\end{quote}




Bastiat's perception was correct. The reason Fort Sumter was attacked by Confederate forces on April 12, 1861, thus starting the American Civil War, was because the fort was being used as a~collection point for the tariff. The fort had no military significance 
%\label{ref:RNDmFxIUMOrZ7}(Adams, 2000, pp.17–33).
\parencite[][pp.17–33]{adams_when_2000}. %
 The reason the southern states wanted to secede from the Union was because of northern hegemony, part of which included the high tariff.



\begin{quote}
At the time Lincoln was pushing his high tariff through the Congress, the Southerners were doing just the opposite. Their new constitution was adopted … with a~unique provision banning high import taxation… Jefferson Davis, the first president of the Confederacy, justified secession in his inaugural address by making reference to the Declaration of Independence, then emphasizing the import tax issue…With low duties the trade of North America would shift from New York, Boston, and Philadelphia to Savannah, Charleston, and New Orleans…This would spell disaster for the Northern industrialists. Secession offered the South not only freedom from Northern tax bondage but also an opportunity to turn from the oppressed into the oppressor… 
%\label{ref:RNDNcxL4T6tok}(Adams, 1993, p.332).
\parencite[][p.332]{adams_for_1993}.%
\end{quote}




Adams 
%\label{ref:RNDzvtbcx1EWl}(Adams, 1993, pp.332–333)
\parencite[][pp.332–333]{adams_for_1993} %
 goes on to state that the main cause of the Civil War was the tariff, not slavery, which was a~secure institution in the South, and which Lincoln promised not to change in the territories where it already existed. Bastiat 
%\label{ref:RNDGeL1m0uVQp}(1964a, p.462)
\parencite*[][p.462]{bastiat_economic_1964-1} %
 was able to see that the high U.S. tariff could lead to war during the 1840s. DiLorenzo 
%\label{ref:RNDa4ppuAQk7U}(2002, p.63)
\parencite*[][p.63]{dilorenzo_real_2002} %
 points out that the high tariff triggered a~constitutional crisis when some South Carolina politicians suggested refusing to collect the tariff at the Charleston, South Carolina port.



As a~general rule, Bastiat viewed tariffs, or customs duties, as a~violation of property rights because the purpose is to protect domestic producers from foreign competition. The tariff constitutes special interest legislation because it benefits a~small group at the expense of the general public. However, if the funds are used for the common expense, the tax is legitimate 
%\label{ref:RNDxODiCazUnT}(Bastiat, 1964c, pp.111–112).
\parencite[][pp.111–112]{bastiat_selected_1964}.%




\subsection{Taxes on capital}



Bastiat believed that the proletariat can be freed only by increases in capital accumulation. When the amount of capital increases more rapidly than the increase in population, two things happen: lower prices and higher wages. Both of these things improve the lot of the worker. He was against what he referred to as the \textit{war on capital}, the taxing of capital for reasons other than to raise the revenue necessary to perform the legitimate functions of government. Capital that is not secure hides or flees. When that happens there is less money available to employ people. The result is unemployment for some and lower wages for others 
%\label{ref:RNDSAmFwdnUqh}(Bastiat, 1964c, pp.184–185).
\parencite[][pp.184–185]{bastiat_selected_1964}.%




\subsection{Tax burden}



When a~nation is burdened with taxes, nothing is more impossible than to levy them equally. The tax burden is shifted onto the rich. When government expenditures expand beyond what is needed to pay for its legitimate functions, the state produces more poverty than it cures. When it is an accepted principle that the function of government is to distribute wealth, the tax burden expands beyond its just limits. The amount taken in taxes should be no more than what is needed to protect the people from violence and fraud. Bastiat proposed a~single tax that is proportional to the amount of property owned 
%\label{ref:RNDcF0ePYPYxa}(Bastiat, 1964c, pp.125–126).
\parencite[][pp.125–126]{bastiat_selected_1964}.%




\subsection{School taxes}



Bastiat opposed forcing some people to pay for the education of other people's children. In Bastiat's time, the government supported the major religions. As a~result, Catholics were forced to support Jewish organizations and Jews were forced to support Catholic organizations. Some religious organizations had their own schools.



Bastiat believed that parents should be responsible for the education of their own children. He also believed that government should not have a~monopoly on education. He disapproved of the top-down, government monopoly on university curriculum, which was based on a~study of the classics. He disapproved of a~classic education because classical scholars glorified plunder and socialism. He did not believe that taxpayers should be forced to pay for indoctrinating the younger generation with such false knowledge 
%\label{ref:RNDRW1O0QjT3Q}(Bastiat, 1964c, pp.278–283).
\parencite[][pp.278–283]{bastiat_selected_1964}.%




\subsection{Subsidizing the arts}



Bastiat begins his discussion with the question, ``Should the state subsidize the arts?'' 
%\label{ref:RNDCCxtmorCrS}(Bastiat, 1964c, p.11)
\parencite[][p.11]{bastiat_selected_1964} %
 It could be argued that the arts broaden and elevate the soul of the nation. Furthermore, French culture is the envy of the world. Should this modest assessment on the citizens of France be stopped?



He goes on to point out that the issue is really a~question of distributive justice. Do the rights of the legislator allow him to reach into the pocket of the workers to supplement the income of the artist? He also asks whether subsidizing the arts results in the progress of the arts. One might point out that in totalitarian regimes such as those in Nazi Germany 
%\label{ref:RND4kJX8s8Nxs}(Fürstenau, 2020),
\parencite[][]{furstenau_how_2020}, %
 Stalinist Russia 
%\label{ref:RNDU3ezK1zX6G}(Beale, 2019)
\parencite[][]{beale_history_2019} %
 or Maoist China 
%\label{ref:RNDEygDrr4cu9}(Burgess, 2018)
\parencite[][]{burgess_art_2018} %
 the arts were used as propaganda tools. Using art as propaganda has a~long, if undistinguished history 
%\label{ref:RNDxDVwqNLgFZ}(Levy, 2021; Weissman, 2023).
\parencites[][]{levy_art_2017}[][]{weissman_how_2023}.%




But getting back to the question of opportunity cost, what is seen is the effect of subsidizing certain arts. What is not seen is what would have happened if those funds had instead been spent by the taxpayers who earned that income. Bastiat believes that the decision as to where the funds should be spent should come from below, not from above.



He later points out that any francs the government spends on the arts creates employment in that field, but only at the expense of employment in the fields where the taxpayers otherwise would have spent their wages. He concludes that the government cannot create jobs but only shift them from one sector of the economy to another.



\subsection{Public works}



Another example in his essay, \textit{What Is Seen and What Is Not Seen} 
%\label{ref:RND4rWwZ7LFq1}(Bastiat, 1964c)
\parencite[][]{bastiat_selected_1964}%
\textit{,} addresses the question of public works. Whenever the state opens a~road, builds a~palace digs a~canal or repairs a~street it provides jobs for certain workers. That is what is seen. But what is not seen is the workers who are deprived of jobs because the funds that are used for those public works cannot be used to hire their services.



He goes on to say that where the expenditure has utility, such as building a~bridge that is needed, there is not a~problem. Problems result when the state engages in public works projects for the purpose of creating employment. Such a~goal might be used to justify the most prodigal enterprises 
%\label{ref:RNDnxpuZdjmYZ}(Bastiat, 1964c, p.17).
\parencite[][p.17]{bastiat_selected_1964}.%




\begin{quote}
The great Napoleon, it is said, thought he was doing philanthropic work when he had ditches dug and then filled in. He also said: ``What difference does the result make? All we need is to see wealth spread among the laboring classes.'' 
%\label{ref:RND0k8NGv0ief}(Bastiat, 1964c, p.18)
\parencite[][p.18]{bastiat_selected_1964}%
\end{quote}




Later in his essay he goes on to say that public expenditures must be evaluated on their own merits because the effect of any public expenditure is not to create jobs but to divert them. Furthermore, reallocating jobs displaces workers, which disturbs the natural laws that govern the distribution of population over the earth 
%\label{ref:RNDeuVFcwTHlJ}(Bastiat, 1964c, p.41).
\parencite[][p.41]{bastiat_selected_1964}. %
 There is also the danger that the public expenditure will create less useful jobs than the jobs that are prevented from coming into existence, since the latter are created by the wants and demands of the people who have earned the money whereas the former are created by bureaucrats, who are creating jobs just for the sake of expanding employment without regard to the wants and needs of the citizenry.



\section{Conclusion}

The contributions Bastiat has made to the economic and philosophical literature are substantial 
%\label{ref:RNDrAsjj3NeS5}(McGee, 2014c).
\parencite[][]{mcgee_relevance_2014}. %
 He saw the market economy as a~harmony of interests rather than a~struggle between classes 
%\label{ref:RNDHwqgMy5613}(Braun and Blanco, 2011).
\parencite[][]{braun_bastiat_2011}. %
 He opposed government intervention in the economy, since intervention would cause more harm than good 
%\label{ref:RNDywko6BORQM}(Hülsmann, 2001).
\parencite[][]{hulsmann_bastiats_2001}. %
 He anticipated and refuted the Keynesian multiplier theory more than a~generation before Keynes (1883-1946) was born 
%\label{ref:RNDeodsB6C4Fy}(McGee, 2014b).
\parencite[][]{mcgee_keynes_2014}. %
 He is one of the few economic philosophers whose essays have lived on more than 150 years after his death. His view of free trade and protectionism is unsurpassed 
%\label{ref:RNDhwKIapjMx0}(McGee, 2014a).
\parencite[][]{mcgee_economic_2014}.%




His contributions to public finance are discussed in the current paper. His view was that of a~utilitarian classical liberal who believed that taxation could be justified only in cases where the tax funds were spent on projects that benefitted the vast majority of the population. Such expenditures included programs that would protect life, liberty and property. Tax funds spent for any other purpose constituted redistribution, and were therefore illegitimate. His philosophy of public finance is as relevant today as it was in the 1840s, when he wrote on this topic.




\end{artengenv}
\label{mcgee-last}

\begin{artengenv}{Krzysztof Turowski}
	{Ludwig Lachmann: A~subjectivist institutionalist, but~not a~nihilist}
	{Ludwig Lachmann: A~subjectivist institutionalist, but~not a~nihilist}
	{Ludwig Lachmann: A~subjectivist institutionalist, but~not\\a~nihilist}
	{Jagiellonian University\label{turowski-first}}
	{The legacy of Ludwig Lachmann within the Austrian School of Economics is subject to several interpretations in the literature: though he clearly considered himself a~member of the school and he influenced many Austrian economists, his particular methodological claims prompted Murray Rothbard to disavow him as a~nihilist.
	
	
	
	In this article, we defend Lachmann by arguing that in order to defend his methodological stance he invoked extra-Austrian influences (Max Weber, G.L.S. Shackle). This way, he championed subjectivist institutionalism consistently both in theory and in practice. His approach leaves a~peculiar, unorthodox, yet positive legacy for contemporary Austrian economics, not so far from the orthodox Misesian stance as it is broadly understood.
	}
	{subjectivism, institutionalism, methodology of economics, financial markets, Austrian School of Economics, Ludwig Lachmann}





\section{Introduction}

\lettrine[loversize=0.13,lines=2,lraise=-0.03,nindent=0em,findent=0.2pt]%
{L}{}udwig Lachmann (1906--1990) is definitely one of the most controversial figures within the Austrian School of Economics.
He came across the writings of Menger while he was studying in Berlin with Werner Sombart, the leader of the last generation of the German Historical School\footnote{Actually, his tutor in Berlin was Emil Kauder, another disciple of Sombart who got interested in Austrian economics \parencite[111]{wasserman-kauder}.}.
He had become so interested in this approach to economics that he eventually went to study at the London School of Economics in the 1930s.
There he witnessed first-hand the Austrian-Keynesian debates on capital and trade cycle, and the eclipse of Austrian economics by Keynesianism in the UK, as he remained the only young adherent of the thought of Mises and Hayek at the LSE.
Then, for many years he taught in South Africa, which put him somewhat at a~distance from the center of gravity of the Austrian School, that moved to the USA after the Second World War. Despite that, he was still active e.g. with publishing his book \emph{Capital and its Structure} in 1956.
Ultimately, he came to the forefront when he was invited as one of the three main speakers at the first big post-war Austrian School meeting at the South Royalton Conference in 1974, alongside Murray Rothbard and Israel Kirzner, the most prominent post-war students of Mises.
Later, throughout the 1970s and 1980s, he was invited every year to the New York University and George Mason University, two important centers of Austrian economics, thus gaining prominence among younger generations of economists gathered there\footnote{For a~biographical sketch of Lachmann, see \textcite{mittenmaier}, \textcite{lewin-life}. For personal reminiscences of his students and acquaintances see e.g. \textcite{reminiscences,caldwell,boehm,boehm2000professor}.}.

Lachmann is widely praised by many economists for his works e.g. on capital \parencite{rothbard-present,lewin-life}, entrepreneurship \parencite{endres2013wresting,horwitz-entrepreneurship}, and institutions \parencite{foss2007institutions}. His assessment of the Hayek-Sraffa debate on the theory of cycles also stands out as a~lucid restatement of the crux of the dispute, either missed or deliberately obscured by both sides \parencite{gordon-other}.
He is also perceived as a~harsh critic of the dominant general equilibrium paradigm, which he viewed as inferior to the market process perspective, espoused by Mises and his followers.

Yet, the Austrianness of Lachmann is only one side of his work.
It is also true that he held John Maynard Keynes in much higher regard than any other Austrian School economist. He was also highly influenced by a~radical Keynesian George Lennox Sharman Shackle, who is best known for stressing the importance of all-pervasive uncertainty in the world of human affairs for economics.
Although Lachmann was keen on subscribing himself to the Austrian School of Economics as a~follower of Menger, Mises, and Hayek, he was also very eager to look for fruitful interactions with institutionalists or post-Keynesians, both perceived as having congenial insights that could be assimilated to form a~broader common approach to the studies of the markets \parencite[8]{lavoie-introduction}.
Moreover, he was widely known to have strong methodological pronouncements that led Murray Rothbard to disavow him as a~nihilist \parencite*[52--53]{rothbard-present}.

His peculiar intellectual perspective combined with the influence on many representatives of the Austrian School, especially in the 1980s \parencite[139--140]{vaughn1998austrian}, incited various opinions of his legacy, from highly negative \parencite[82]{rothbard-present} to overwhelmingly positive ones \parencite[1]{lavoie-introduction}.
This divergence itself raises a~question of the proper assessment of the place Ludwig Lachmann occupies within Austrian economics, both for his methodological position and for its relevance to economic practice.

The aim of this article is to argue contra Rothbard that Ludwig Lachmann indeed offered a~fruitful positive program for economic research in line with Austrian tradition.
However, we also recognize a~grain of truth in Rothbard's assertion that Lachmann, thanks to his extra-Austrian influences, put strong emphasis on institutions and distanced himself from high theory, and pursued a~different route than the people steeped in Austrian economics in the traditions of Mises, Hayek, and Rothbard---and therefore prone to other kinds of challenges.
Overall, it is best to treat Lachmann as supplementing the main corpus of economic knowledge rather than superseding the praxeological paradigm.

Our line of reasoning proceeds in three steps.
First, in section~\ref{sec:tenets}, we reconstruct the main tenets of Lachmann's methodology: his radical subjectivism and the primacy of institutions as the guiding posts for actions in the rapidly changing economic world.
Next, in section~\ref{sec:inspirations} we outline his general view of economics as a~science, and we follow up with a~section discussing the differences separating him from Mises, Hayek, and Rothbard. We also summarize the intellectual indebtedness of Lachmann to Max Weber and G.L.S. Shackle, and argue that these extra-Austrian influences are both clearly recognizable at the core of his subjectivist-institutionalist methodology, and they are reasons why Rothbard accused Lachmann of being a~nihilist and anti-economist.
Finally, in section~\ref{sec:finance} we show, contra Rothbard, Lachmann's method in action in his analysis of financial markets. This helps us assess both the strengths and weaknesses of the Lachmannian approach, and its relevance to the main (Misesian-Rothbardian) Austrian economics paradigm.

\section{Main methodological tenets}
\label{sec:tenets}

Although Lachmann had the same inclination as many other Austrian economists to outline a~broader vision of doing social science, extending beyond economics, he did not delve deep into systematic philosophical and anthropological considerations, unlike Mises or Hayek. Neither did he publish a~comprehensive definitive pronouncement of his methodological views as did Mises and Menger.
It is important to remember that Lachmann started as a~capital theorist, and it was the motivation to clear misunderstandings and get rid of flawed approaches mostly in this area that eventually drew his methodological efforts to the forefront \parencite[215]{prychitko-review}.

His most prominent methodological analyses can be found in a~series of articles ranging from the 1940s up to his death, gathered mostly in four books:
\begin{itemize}
\item \emph{Legacy of Max Weber}, 1970,
\item \emph{The Market as an Economic Process}, 1986,
\item \emph{Capital, Expectations, and the Market Process}, 1977, a~collection of articles from the 1940s to the 1970s, edited by Walter Grinder,
\item \emph{Expectations and the Meaning of Institutions}, 2005, a~collection of articles from the 1930s to the 1980s, edited by Don Lavoie.
\end{itemize}
Unfortunately, in these works many of his particular insights appear only in passing, for example when he is commenting on works of other economists, such as Mises or Shackle. And in more programmatic publications he does not repeat some crucial insights or reservations that nuance his line of reasoning.
Still, we believe that even though Lachmann did not write any work devoted solely to outlining his methodological stance, it is possible to reconstruct the main principles of his methodological stance from these works. Indeed two themes come to the forefront throughout his career: subjectivism and institutions. Let us look at each of them in turn.

\subsection{Subjectivism}


As it was observed in the literature, if there is one particular stance that can be associated with Lachmann throughout his whole career, it is his often-repeated commitment to subjectivism \parencite[3]{grinder-introduction}. He defined it as
\begin{quote}
[t]he postulate that all economic and social phenomena have to be made intelligible by explaining them in terms of human choices and decisions \parencite[10]{lachmann1973macro}.
\end{quote}
This includes also uncovering the purpose and the general design of the plan behind observable actions \parencite[71--72]{lachmann-expectations}.

Subjectivism is for Lachmann the principle of explanation of social sciences, and he tried to push it as far as possible with his agenda of radical subjectivism.
He quoted several times with approval Hayek's remark that ``every important advance in economic theory during the last hundred years was a~further step in the consistent application of subjectivism'' (\cite[155]{lachmann-individualism}; \citeyear[23]{lachmann1986market}; \citeyear[3]{lachmann-shackle-place}, originally in \cite[31]{hayek-counterrevolution}).
The task of the economist in this view (dubbed ``the market process approach''), is ``to make human action intelligible, to let us understand the nature of the logical structures called `plans', to exhibit the successive modes of thought which give rise to successive modes of action'' \parencite[417]{lachmann-ha}, or simply ``to understand [\ldots] what men do in markets'' \parencite[3]{lachmann1986market}.
He states that the access to intelligible meaning as social causes gives social scientists an advantage compared to the natural sciences, which he assumes to be confined only to observable uniformities \parencite[90]{lachmann-shackle-time}.

This conceptualization of human action as the subject matter of economics has several further consequences. First, as Lachmann points out, ``each plan is a~logical structure in which·means and ends are coordinated by a~directing and controlling mind'' \parencite[418]{lachmann-ha}.
This dispenses with the possibility that actions in the real world can be considered passive reactions to external incentives, as these would effectively mean abolishing planning altogether.
Next, since the plans are meaningful, this means that not only social scientists but also other agents can understand them and use them in their plans:
\begin{quote}
At any moment the actor's mind takes its orientation from (but does not permit its acts to be dictated by) surrounding facts as seen from its perspective, and in the light of this assessment decides on action, making and carrying out plans marked by the distinction between means and ends. [\ldots] [W]hat men adjust their plans to are not observable events as such, but their own interpretations of them and their changing expectations about them \parencite[4]{lachmann1986market}.
\end{quote}
This, in turn, leads him to point out that there are no objective data such as ``tastes'' that can be separated from resources and technological knowledge as an independent exogenous variable (\cite[24]{lachmann1986market}; see also \cite[35]{lachmann-crisis}).

As Lachmann points out, plans are conceived with certain background knowledge about the environment, including both the physical world and the actions of other agents.
He notes that events happening in the world in virtue of their being observable and understandable may affect our knowledge.
Our previous actions, and especially our assessment of their success, also influence our current planning.
However, knowledge has very peculiar properties. As he notes: ``[c]hanges in the constellation of knowledge are an inevitable concomitant of the passing of time'' \parencite[200]{lachmann-hayek}.
Thus, models assuming a~fixed stock of knowledge of agents are essentially timeless.

The subjective character of knowledge acquisition by a~human mind implies heterogeneity of knowledge among agents.
Particular results depend on countless factors e.g. on attitude towards the future (optimists vs. pessimists, bulls vs. bears), and ``no recipe for turning information into knowledge can exist'' \parencite[51]{lachmann1986market}.
The whole concept of a~market of homogeneous units of information flow is deeply flawed and cannot be sustained in this view.
Any attempt to incorporate a~formal rule of learning is self-defeating since Lachmann argues that such an approach would undermine free will as a~plausible working hypothesis concerning human action in general:
\begin{quote}
[f]or how otherwise could they take part in discussions without regarding themselves as mere human gramophones emitting strange but irrelevant noises, and how could they ever hope to ``convince'' anybody else? \parencite[167]{lachmann-science}
\end{quote}
In essence, this argument anticipates\footnote{Thus, \textcite[38]{hoppe} is not right in his attribution of primacy to Popper. And though one can see this argument also in \textcite[104]{shackle-time}, it seems that Lachmann got it right first.} \textcite[10]{popper-poverty} by pointing out that if there were causal laws determining human learning and actions, then it would mean the world devoid of meanings and arguments, so only with passive reactions and not actions in the proper sense of the word.

Lachmann as early as 1943 underlined that not only the current knowledge matters for agents, but even more importantly, their expectations concerning the world and other agents:
\begin{quote}
Expectations, it is true, are largely a~response to events experienced in the past, but the \emph{modus operandi} of the response is not the same in all cases even of the same experience. This experience, before being transformed into expectations, has, so to speak, to pass through a~``filter'' in the human mind, and the undefinable character of this process makes the outcome of it unpredictable \ldots It follows that they [expectations] have to be regarded as economically indeterminate and cannot be treated as ``variables which it is our task to explain'' \parencite[67]{lachmann-role-expectations}.
\end{quote}
The heterogeneity and subjectivity of expectations induced him to reject an attempt by Oskar Lange to find an objective measure of the degree of uncertainty of price expectations (\cite[120]{lachmann-expectations}; see also \cite[422]{van-zijp}).

In a~broader perspective, Lachmann viewed the modern history of economic thought as a~battlefield between two approaches: subjectivist and formalist---the first exemplified by Austrians, but also by Post-Keynesians, the second associated typically with general equilibrium framework and Neo-Ricardians \parencite[22--23, 164]{lachmann1986market}.
The problem with the formalist approach is, as he points out, an assumption of constant relationships, mathematical tractability, and measurability, which dispenses with the real causal force of human action, both subjective and changing.
In Lachmann's own words ``expectations, and other subjective elements, constitute an alien body within the organism of formal model analysis'' \parencite[249]{lachmann-hicks}.

These two views are tied to two meanings of economics distinguished by John Hicks: \emph{plutology}, the science of wealth, and \emph{catallactics}, the science of exchange \parencite[215]{hicks}.
Although there is no one-to-one correspondence, the former is often framed in formalist language, and the latter is more congenial to the subjectivist approach.
Lachmann accepts this distinction, yet he sees a~paradox: neoclassical theory of growth, a~contemporary example of plutology, requires capital homogeneity as one of its assumptions, and in doing so it relies on a~catallactic framework of general equilibrium of Walras and Pareto \parencite[25--26]{lachmann1986market}.

The emphasis on subjectivity with regard to production plans made Lachmann the harshest critic of all equilibrium approaches among all representatives of the Austrian School.
He argued that the only meaningful sense of equilibrium in modern economics obtains when an individual (household, firm) acts rationally and exhausts all the gains from removing the inconsistencies between his various plans (\cite[15]{lachmann1973macro}; \citeyear[141]{lachmann1986market}). However, for the whole economy, he seems to be taking exactly the opposite view:
\begin{quote}
In a~kaleidic society the equilibrating forces, operating slowly, especially where much of the capital equipment is durable and specific, are always overtaken by unexpected change before they have done their work. [\ldots] Equilibrium of the economic system as a~whole will thus never be reached \parencite[60--61]{lachmann-kaleidic}.
\end{quote}
His argument was simple: for an individual person (household, firm) we can talk of equilibrium as rationality, consistency of concurrent plans because there is a~single organizing unit of agency\footnote{Lachmann actually broadens the legitimate use of equilibrium to include single organized markets or even single industries, as did Marshall (\cite[37]{lachmann-crisis}; \citeyear[149--150]{lachmann-individualism}). However, he does not provide any examples of good and bad uses of the concept in these areas, so it is hard to assess these claims.}.
However, with a~multiplicity of agents, it is a~brute fact of life that there is no such unified perspective.
All capital goods gain meaning only within some production plan and such plans are divergent since they are undertaken by different people. It directly follows that there is no God-like macroeconomic perspective or an objective measure in terms of some appropriately defined quantity that allows the amalgamation of heterogeneous capital goods into a~single blob like the macroeconomic $K$ (\cite[175--177]{lachmann-salvage}; \citeyear[194]{lachmann-hayek}; see also \cite{garzarelli}).

One could expect that Lachmann would be more sympathetic to the neoclassical microeconomic theory since it is concentrated on a~single decision unit. However, this is not the case. He identifies a~pernicious influence of formalism in the assumption of ``independent variables'' of tastes, resources, and technical knowledge, completely unrealistic and removing the true objective of the study of human action from the picture \parencite[217--220]{lachmann-vicissitudes}.
In the indifference curve approach, a~complete scale of preference is assumed, thus action follows by inference. But, as Lachmann asserts, real acting people making genuine choices have limited imagination and they can conceive only several alternative courses of action \parencite[216]{lachmann-vicissitudes}. For similar reasons he suggests that a~concept of production function is useless in the world of perpetual change, requiring entrepreneurs to devise and execute their plans \parencite[312]{lachmann-market-distribution}.

\subsection{Importance of institutions}


Lachmann, in line with his upbringing under Werner Sombart, a~leader of the last generation of the German Historical School, and with his lasting admiration for Max Weber was always inclined to emphasize the institutional aspect of the economy. As he said late in his life:
\begin{quote}
Few economists will deny that the market operates within a~framework of legal and other institutions, that its modus operandi may be helped or hindered by the varying modes of this framework, and that the outcome of market processes will not be unaffected by changes in it. [\ldots] our world is far more complex than was that of the classical economists and [\ldots] there is evidently a~good case for having another look at the relationship between the market economy of our days and its institutional basis \parencite[249--250]{lachmann-legislation}
\end{quote}
At one point he defines institutions as ``certain superindividual schemes of thought [\ldots] to which the schemes of thought of the first order, the plans, must be oriented'' \parencite[62]{lachmann-significance} and comments that ``designed institutions can be regarded as successful plans which have crystallized into institutions through widespread imitation'' \parencite[81, 89]{lachmann-mises-process}.
This functional description indicates that he does not want to limit his analysis to organized or legal institutions \parencite[62--63]{lachmann-weber}. Rather there would fall all kinds of associations and norms under this category, just as in the popular contemporary new institutionalist approach \parencite[7--8]{alvesson}.

Institutions perform a~very important function within the subjectivist framework: they provide people with means of orientation towards their goals in a~more effective way since they ``enable us to rely on the actions of thousands of anonymous others about whose individual purposes and plans we can know nothing'' \parencite[49--50]{lachmann-weber}. In other words,
\begin{quote}
[i]nstitutions reduce uncertainty by circumscribing the range of action of different groups of actors, buyers and sellers, creditors and debtors, employers and employees. We understand how they work by grasping the meaning of the orientation of these groups towards them \parencite[277]{lachmann-hermeneutic}.
\end{quote}
For example, we just have to know what a~post office does (delivers letters), and we do not need to grasp the plans of any managers or postmen to use this idea to our advantage in our plans\footnote{Although Lachmann would probably say those insights into plan patterns of such people are of course crucial when we try to explain why post offices work in general, or why some are more effective than others.}.

Clearly, institutions do not have an objective character to be inferred e.g. from their physical characteristics, but they are intrinsically intersubjective, and perceived individually by each agent. Thus, the orientation they give, as any other knowledge ``cannot be regarded as a~`function' of anything else'' and ``does not fit into a~world of `function-maximizing' agents'' \parencite[277]{lachmann-hermeneutic}.

In Lachmann's view, one of the main research problems is the investigation of institutional change. Institutions are good indicators of other people's actions if they are stable and predictable.
However, omnipresent uncertainty and continuous change require that effective institutions have to be also flexible, to adapt to new circumstances. As he writes,
\begin{quote}
All institutions are subject to historical change. In the due course, they may on the one hand acquire new functions, while old functions become obsolete [\ldots] it may happen that what was originally quite a~sound institution may turn out to become most unsound, or (though I~would not know of one example!) it may happen the other way round. \parencite[177]{lachmann1962cost}.
\end{quote}

However, in writing he distances himself from an institutionalist charge against neoclassical economics that the latter is neglecting institutions (\cite[275]{lachmann-hermeneutic}; see also \cite[499]{udehn} comparing Austrians with general equilibrium theorists on this point).
As he points out, after all, markets are institutions too, and there are at least rudimentary theories of property, contract, banking, and finance assumed.
For example, Lachmann as a~market process theorist champions a~view that
\begin{quote}
[I]n a~world of continuous change prices are no longer in all circumstances a~safe guide to action [\ldots] nevertheless even here price changes do transmit information, though now incomplete information [\ldots] such information, therefore, requires interpretation (the messages have to be ``decoded'') in order to be transformed into knowledge, and all such knowledge is bound to be imperfect knowledge. In a~market economy success depends largely on the degree of refinement of one's instruments of interpretation \parencite[22]{lachmann1956capital}.
\end{quote}

In fact, despite all his criticism of the Walrasian paradigm Lachmann conceded that it also cannot be accused of an institutionless approach.
The only problem is that the ideal types of institutions may be and indeed are ill-designed in their case.
For example, he even concedes that it may be useful to rely on an auctioneer as an ideal type provided that it is supplemented with comparative studies of real markets in comparison to this ideal type---but to his disappointment, there was no research in this field \parencite[40--41]{lachmann1986market}.
Instead, neoclassical economists focus solely on this assumption as a~tractable, mathematically convenient axiom for building formal theories. However, as Lachmann notes by doing so, they had to dispense with practical relevance for many important questions about the real world \parencite[142]{lachmann1986market}.

\section{Economics as a~subjectivist-institutionalist science}
\label{sec:inspirations}

Given these two major themes underlying Lachmann's methodology throughout his whole career, we can coin the phrase ``subjective institutionalism'' to describe this overall outlook.
In short, it would suggest a~research program that would be interested in understanding the economic phenomena in terms of subjective plans of agents, with the emphasis on how they are shaped by particular institutions as perceived by these agents.

Lachmann argues that both the subject matter and the aim of theoretical and historical social sciences are identical since both are concerned with causal explanations of phenomena of the social world (both intended and unintended) in terms of action guided by plans as their causes.
Their only difference lies in the guiding methodological principles.
However, he does not support the Neo-Kantian division between nomothetic and idiographic sciences but rather opts for pure versus applied sciences as the proper way of framing both types of sciences \parencite[173--175]{lachmann-science}.

The tasks of economists and historians are in this view highly complementary.
Theoreticians contribute analytical schemes of interpretations at different levels of abstractions \parencite[179]{lachmann-science}. In economics they are concerned with (social) causation, thus they have to be constructed according to the ``compositive'' method, i.e. ``analyzing complex phenomena into their simplest elements'' \parencite[172]{lachmann-science}, in this case, individual actions guided by plans.
Interestingly, he adds that it is hardly an accident that
\begin{quote}
has more nearly approached the ideal of a~closed theoretical system in which all propositions are linked to each other and the number of fundamental hypotheses reduced to a~bare minimum than any other social science \parencite[179]{lachmann-science}.
\end{quote}
Theoretical models do not provide predictions. However, Lachmann allows for negative prediction in the sense that certain policies could be uncovered as internally inconsistent and thus will be doomed to failure (\cite[89]{lachmann-shackle-time}; see also \cite[7--8]{lachmann-shackle-place}).

A~historian, on the other hand, ``endeavors to render his narrative intelligible by means of causal imputation'' \parencite[178]{lachmann-science} i.e. ``to `fill in' the descriptive signs between the logical signs, to tell us what ends by what means men in a~given situation pursued'' \parencite[175]{lachmann-science}.

This perspective also comes to the forefront when Lachmann downplays the importance of \emph{a~prioristic} praxeology as purely analytical:
\begin{quote}
[O]ur network of means and ends, precisely by virtue of the logical necessity inherent in it, is impotent to engender empirical generalizations. Its truth is purely abstract and formal. The means and ends it connects are abstract entities. In the real world the concrete means used sought are ever-changing as knowledge changes and what seemed worthwhile yesterday no longer seems so today. We appeal in vain to the logic of means and ends to provide us with support for empirical generalizations of the kind mentioned \parencite[31]{lachmann1986market}.
\end{quote}

However, most of the Lachmannian scorn is directed again towards models, which abandoned the pursuit of describing and accentuating particular significant traits of reality in favor of devising a~set of mathematical equations describing some observable relations with parameters as regression coefficients in statistical time series \parencite[35]{lachmann1986market}.
In passing, he also dispelled the myth that the accumulation of statistical data made much impact on economics, as he noted that there were no recurring patterns of observable variables detected \parencite[177]{lachmann-science}.

The unknowability of the future posits a~problem for empirical generalizations, but he did not preclude their existence for some past events, even if only with a~narrow scope and character. To quote Lachmann:
\begin{quote}
every action depends on the state of knowledge of the agent at the point in time of the action, which is not predictable at the point in time of the formulation of the theory \parencite[61]{lachmann-significance}.
\end{quote}
Moreover, while it is possible to trace the consequences of action in the sphere of production and wealth accumulation e.g. due to coin debasements or tariffs, it is far more questionable to trace the effects of changing the technology, tastes, or available resources on prices and quantities, as the latter have to be understood as taken in equilibrium \parencite[32--33]{lachmann1986market}.

Lachmann did not present a~single unified systematic perspective, nor did he believe in one. Rather he assumes that there are different goals for economic sciences, and while some of them may be unreachable because of some inherent limitations within the subject matter (e.g. prediction), the Austrian market process perspective is only one of the possibilities, adequate for some explanations, but maybe not universal \parencite[41]{lachmann1986market}.

It is the task of a~historian to look at alternative models provided by theoreticians and choose the proper ones according to his understanding of a~situation or a~process under study \parencite[179]{lachmann-science}. There are no rules for applying models by historians, they can be misapplied in various ways, but the ultimate test is the fruitfulness of research \parencite[175]{lachmann-science}.
For example, he criticizes Hayek for calling Frank Knight's concept of capital ``mythological'', because:
\begin{quote}
In assessing the merits of our two perspectives we have to judge by the facts on which they cast light and by the significance of these facts to us. If we are interested in certain facts, which is one of the perspectives are either abstracted from or given low status, we shall of course not adopt it, but this gives us no right to condemn it as an analytical device \parencite[175]{lachmann-salvage}.
\end{quote}
In particular, in some circumstances he suggests concentrating on particular traits of phenomena and the differences they bring to the table:
\begin{quote}
Markets differ in many ways that do not matter to the purpose of understanding the constellation, the entirety, of market forces. These differences become relevant only when they affect the character of human action in markets. But when they do, they must not be abstracted from, for in such cases talk of ``the market'' is as likely to mislead as to enlighten \parencite[271]{lachmann-speculative-markets}.
\end{quote}

Similarly, in the context of the Austrian Business Cycle Theory he wrote:
\begin{quote}
Once we admit that people learn from experience, the cycle cannot be reproduced time after time [\ldots]
[I]t may be better to give up the doubtful quest for a~model of the business cycle and to regard phenomena such as cyclical fluctuations in output and prices simply as phenomena of history [\ldots] with the events of each successive cycle requiring different, though often similar, explanations \parencite[30--31]{lachmann1986market}.
\end{quote}
Overall, Lachmann seems to be critical of any unified approach:
\begin{quote}
logic certainly is immanent in all human action. But this alone does not mean that the logic of success, which depends upon means and ends, is also the logic governing all action. Conceivably another kind of logic, one employing other categories, might be applicable here \parencite[59]{lachmann-significance}.
\end{quote}

Given this, it should not surprise anyone that Lachmann was happy when he noted similarities between the reformulated Hicksian and the Austrian capital and growth theory \parencite[253--254,258,264--265]{lachmann-hicks}. At the same time he did not dispute the validity of the Keynesian one \parencite[106]{lachmann-ha}. Indeed at one point, he stated that in his view Great Depression was an example of a~crisis of underconsumption \parencite[111]{lachmann-ha}.

Lachmann gives historians practical advice not to give in to the temptation of reducing the number of causal factors as it often leads to perceiving a~historical process under consideration as a~response of an individual or a~social group to a~quasi-external cause.
Such ``cause'', e.g. Hegelian group-spirit, is substituted for plans of individuals, and it is in his opinion mythology, not history, ``reminiscent of the Olympian interventions in the struggles of the Homeric heroes \parencite[175]{lachmann-science}.
As an example of failed endeavors, Lachmann points to explanations of the period 1815--1914 solely in terms of the ``process of industrializations'', as they abstract from crucially significant dissimilarities between countries or industries that this frame of reference cannot capture \parencite[176]{lachmann-science}.

It is informative to have a~short glance at Lachmann's interpretation of two important debates in the history of economic thought: between Hayek and Keynes on the Great Depression, and between Hayek and Sraffa on business cycles.
For the first one, Lachmann does not really challenge the validity of their respective theories. He rather resolved the issue by claiming that both sides were talking past each other. Interestingly enough, he noticed that there was a~lot of common ground since Keynes and Hayek were both committed to a~similar subjectivist methodology, and they put aside their political differences in the course of the debate \parencite[183]{lachmann-keynes}. The true difference arose at the level of interpretation of contemporary facts: Keynes assumed that it was only in the case of financial markets that prices were fixed largely independent of expectations, whereas Hayek pointed out that it is rather the case for ordinary market, but not for financial assets with prices set by banks \parencite[183--184]{lachmann-keynes}.
However, for Lachmann such conflicts are hard to avoid, because:
\begin{quote}
whenever we confront very large numbers of facts, it is in any case impossible to know all of them and we have to `stylize' what we regard as a~representative selection of them\footnote{Compare \textcite[304]{mises-theory}, who was convinced that ``scientists may disagree about theories. They never lastingly disagree about the establishment of what is called pure facts''.} \parencite[190]{lachmann-keynes}.
\end{quote}
Lachmann notes this is an exact application of the principle of subjectivity to the social sciences (after all, products of human activity themselves) that leads us to dispense with the idea of objective facts in economic history that could be subjected to a~universal intersubjective agreement e.g. via testing.

However, when he discusses the attack Sraffa launched on the Hayekian theory of business cycles, he gives him credit on several points, but not for having an alternative sound theory. On the contrary, he accuses him of having the wrong theory, based on improper usage of equilibrium, inconsistent with subjectivism.
Even worse, since Sraffa knew that his (neo-Ricardian) stance was highly disputed, he was deliberately concealing it to discredit Hayek in the eyes of fellow subjectivist Keynesians or formalist general equilibrium theory adherents \parencite[144--145]{lachmann-hayek-sraffa}.

\section{Lachmann \emph{versus} the Austrian School}

For a~reader familiar with the works of the major economists of the Austrian School, that is, Mises, Hayek, and Rothbard, the above view on the tasks of economists might sound like a~mixed bag.
Clearly, Lachmann with his insistence on subjectivism and acting men is perfectly in line with their own pronouncements, and it was acknowledged at least by Israel Kirzner, another major Austrian economist:
\begin{quote}
Lachmann, similarly, instructed us that when we deal with broader questions, with institutions and regularities in economic affairs, we have not completed our task if we have not called attention to the purposes and motives and interests that underlie these phenomena \parencite[46]{kirzner-method}.
\end{quote}
His agenda is in clear agreement with the works of the Austrian School, and Lachmann acknowledged the connection, hailing his predecessors as champions of subjectivism \parencite[28]{lachmann-crisis}.
Indeed, he seems to be strongly influenced by the seminal works of Hayek on knowledge and its dissemination in society (primarily \emph{Economics and Knowledge}, 1937, and \emph{The Use of Knowledge in Society}, 1945), and he recognizes congeniality of the main claims from the Mises' \emph{opus magnum} \emph{Human Action} to his own research program \parencite[56--57]{lachmann-kaleidic}.

Kirzner also agrees with the two main tasks of economics outlined by Lachmann: ``to make the world around us intelligible in terms of human action and the pursuit of plans [\ldots] [and] to trace the unintended consequences of such action (\cite[41]{kirzner-method}; see also \cite[261--262]{lachmann-hicks-neo}), and directly relate them to the writings of Carl Menger.\footnote{Interestingly, both Kirzner and \textcite[66--67]{rothbard-praxeology} pointed out the insufficiency of Hayek's position in this respect, preferred to emphasize only the latter task.}

In addition, Lachmann is also eager to defend methodological dualism of natural and social sciences \parencite[167--168]{lachmann-science}, independence of the theoretical social sciences, in particular economics \parencite[59]{lachmann-significance}---both points heavily emphasized by modern Austrian economists.
He also agrees with his predecessors that the validity of economic theories is warranted solely by logic, and not by experience (\cite[58]{lachmann-significance}; see also \cite[41]{mises-ha}, \cite[21, 31--32]{rothbard-praxeology}).
His insistence that ``actions certainly are events in space and time and, as such, are observable. But observation alone cannot reveal meaning'' \parencite[58]{lachmann-significance} is highly reminiscent of the respective pronouncements e.g. by Mises (\citeyear[26]{mises-ha}, \citeyear[245]{mises-theory}), or \textcite[63--64]{hoppe} when they rebuke behaviorism and positivism.

Paradoxically, Lachmann is closer to Mises than to his teacher Hayek, as the latter used intertemporal general equilibrium as his basic tool of macroeconomic analysis\footnote{Note that even Hayek was conscious that ``to make full use of the equilibrium concept we must abandon the pretence that it refers to something real'' \parencite[23]{hayek-pure}.} (\cite[190]{lachmann-hayek}; see also \cite{lachmann-hayek-sraffa}).
Mises and Lachmann were known to be uncompromising in rejecting any kind of macroeconomic reasoning in terms of equilibrium terms as meaningless for the real economy.
Lachmann happily endorsed Misesian restriction of general equilibrium constructs to hypothetical ones and replaced them with the concept of the market process (\cite[183]{lachmann-mises-process}; see also \cite[230--231]{mises-planning}).\footnote{See \textcite{salerno-equilibrium} for different kinds of equilibrium used by Austrian economists. Check also \textcite{cowen-ere} for a~critique of inconsistent use of evenly rotating economy auxiliary construct by Mises and Rothbard.}

There are also many similarities between Lachmann and other Austrians in the economics proper, e.g. in the theories of business cycles and entrepreneurship.
Lachmann acknowledges Mises as the champion of the market process approach (\cite[182--183]{lachmann-mises-process}; \citeyear[60]{lachmann-kaleidic}), and he praises Misesian dynamic theory of entrepreneurship \parencite[102]{lachmann-ha}. Similarly, he points to Hayek as the one who raised fatal charges against the neoclassical notion of a~homogeneous capital already in the 1930s\footnote{As \textcite[lxii-lxiii]{hulsmann} notes, this point was observed even earlier by Mises in his 1933 essay \emph{Inconvertible Capital}, and only then developed in detail by Hayek.}, but unfortunately, their insights were completely ignored by the mainstream, though preserved in the Austrian School e.g. in the works of Kirzner \parencite[195--198]{lachmann-hayek}.
And while he notes that there was also later, independent, but far more famous criticism espoused by so-called Cambridge UK Keynesians, at the same time he points out that they rely on the Ricardian and formalist framework instead of Keynes' subjectivism---which makes them wrong in other respects (\cite[21, 51--52]{lachmann1973macro}; see also \cite[33]{lachmann-crisis}).

\subsection{Points of divergence: subjectivism and institutions}


On the other hand, there are in Lachmann some pronouncements that distanced him from his fellow Austrian economists. They were mostly concerned with the two main topics of his methodological thought, subjectivism, and institutions.

First, Lachmann contended that the Austrians were not radical enough in their subjectivism.
To prove the case, he distinguished three stages of the development of subjectivism. The first one, appearing in the 1870s and presented most consistently in the works of Carl Menger, was concerned with the consumer as a~source of value in economics, and stressed the subjectivity of wants.
However, as Lachmann rightly pointed out, Menger's subjectivism was limited in that he believed in distinctions between real and imaginary goods, and he postulated the existence of the objective hierarchy of wants \parencite[57]{lachmann-menger}.

The next step was done by Mises, who first recognized these limitations in Menger's work \parencite[192]{mises-epe}, and improved on him by introducing subjectivism of means and ends. In doing so he argued that uncertainty and change in the world imply the appraisal of means. Still, in Lachmann's view, Mises did not pay enough attention to the role of changing knowledge and expectations (\cite[57]{lachmann-expectations}; \citeyear[37]{lachmann-extension}; see also \cite[65--66]{koppl}). Later in his life, Lachmann expressed his concern that Mises assumed the aims of individuals as fixed, thus neglecting the importance of mind choosing and changing goals \parencite[6]{lachmann-shackle-place}.

Hayek, though still, for Lachmann, remained an incomplete subjectivist, is credited with going beyond Mises at least on two occasions.
Already in 1933 in his Copenhagen lecture, he explicitly mentioned expectations in the context of his trade cycle theory (\cite{hayek-1933}; see also \cite[259]{lachmann-hicks-neo}). Moreover, in his famous 1948 article \emph{Economics and Knowledge}, he claimed that the logic of choice is far from sufficient, and for economics to be empirical it has to study patterns of knowledge acquisition and dissemination \parencite[33]{hayek-knowledge}.
Still, as it was mentioned above, in Lachmann's view even Hayek did not pursue this route consistently because he considered the general equilibrium model as his starting point, and for a~while, he was captured by an idea that there is a~``strong tendency towards general equilibrium as a~real phenomenon of the market economy'' \parencite[60]{lachmann-kaleidic}.

For Lachmann, the final stage comes with an acknowledgment of the subjectivity of expectations. He praised Keynes, Knight, and the Swedish disciples of Wicksell (mainly Lindahl and Myrdal) for introducing expectations in their economic theories in the 1930s (\cite[141]{lachmann-notes}; \citeyear[157--158]{lachmann-individualism}; \citeyear[5]{lachmann-shackle-place}).
He viewed that this move was partly responsible for the Keynesian victory over the Austrians in the 1930s \parencite[see][10]{mittenmaier}.
However, he notes that the usage of expectations in \emph{General Theory} was inconsistent, and later Keynesians disposed of them when they formalized the dominant neoclassical synthesis paradigm, so the radical subjectivist parts of Keynes' work remained unnoticed (\cite[141--142]{lachmann-notes}; see also \cite[221]{lachmann-vicissitudes}).
On this point, he also criticized Mises and Hayek for not noticing expectations as a~fellow subjectivist topic that should be embraced and analyzed using a~market process approach they were developing \parencite[5]{lachmann-shackle-place}.
But it was only when the dominant neoclassical paradigm started to be challenged in the 1960s that the issue was slowly reintroduced in the economic discussion.

Indeed, it was a~long-time friend of Lachmann and a~fellow student at the LSE, George Shackle, who was credited by Lachmann for carrying forward the ideas of close links between time and knowledge, subjectivism of expectations, and finally the notion of the kaleidic world.
Following Shackle, Lachmann also endorsed the subjectivist reading of \emph{General Theory}, according to which there is an internal tension in the book between the formalist, equilibrium way of presenting a~large part of his arguments, and his subjectivist leanings visible e.g. in his treatment of expectations, leading him to regard Keynes as even more subjectivist than Austrians \parencite[281]{lachmann-hermeneutic}.
Clearly, with this perspective at hand, both Shackle and Lachmann were extremely critical of what was preserved from Keynes in the post-war neoclassical synthesis, i.e. the ``hydraulic approach'', in particular including the multiplier-accelerator mechanism (\cite[188]{lachmann-keynes}; \citeyear[149]{lachmann-hayek-sraffa}).

No wonder Lachmann was against any notion of ``lagged responses'' or ``adaptive expectations'', which reduced actions to reactions to antecedent events and denied creativity on the part of the economic agents.
With changing knowledge and without a~deterministic dependency between knowledge and expectations he contends after Shackle that the world of human action is \emph{kaleidic}, that is, changing rapidly like in a~kaleidoscope, forming new, ever-changing patterns as time passes \parencite[28--29]{lachmann1986market}.

The second point of divergence between Lachmann and the Austrians is the embrace of the Weberian method of understanding (\emph{Verstehen}) as ``the `natural' method of rendering an intelligible account of the manifestations of the human mind'' (\cite[17--18]{lachmann-weber}; see also \cite[47]{lachmann-significance}), which
\begin{quote}
is nothing less than the traditional method of scholarship that scholars have used throughout the ages whenever they were concerned with the interpretation of texts. Whenever one is in doubt about the meaning of a~passage one tries to establish what the author ``meant by it''.
[\ldots] It is evidently possible to extend this classical method of scholarship to human acts other than writings \parencite[10]{lachmann-weber}.
\end{quote}

Applications of \emph{Verstehen} result, following Weber, in the formation of the ideal types.
They are not distillations from historical experience, but rather figments of our imagination, and there is no universal algorithm for their construction, as they depend on the events under consideration.
They abstract from a~mass of unnecessary detail but accentuate the features we wish to study (\cite[26--27]{lachmann-weber}; see also \cite[90]{weber}). They ``serve us as criteria of classification of real events [but] we must not confuse them with reality'' \parencite[254]{lachmann-legislation}.

Contrary to what Mises wrote in distinguishing real and ideal types e.g. in the context of entrepreneurship \parencite[59--64, 252--256]{mises-ha}, and restricting the usage of ideal types to history, Lachmann considers praxeology as providing historians with ideal-typical conceptual classification schemes \parencite[34--35]{lachmann1986market}.
In fact, his perspective on ideal and real types is almost the opposite of what we can see in Mises: real types here serve as proxies for masses of particular historical facts, obtained by inductive generalization. Either facts themselves or real types are compared with ``theoretical'' ideal types to gain insights into particular causal processes and to obtain explanations expressed in terms of the plans of individuals.

Lachmann takes the considerations on ideal types further by arguing that what makes the general equilibrium framework problematic is not exactly its assumptions such as setting all producers as price takers, as this can be seen as an accentuation of the situation of the real-world consumers, where they cannot alter the prices.
The real problem lies in mistaking the ideal type for a~``normal'' or ``higher'' reality that real events may deviate from \parencite[37]{lachmann1986market}. Moreover, there is a~question of of what use could be such a~model since its inbuilt stability can only accommodate a~very narrow group of adjustment processes.

Lachmann believes that although Weber himself was reluctant to search for wider generalizations, it is possible to develop a~general dynamic theory of institutions based on Weber's work.
Using subjectivist insights that every plan has to include expectations of plans of others, we saw Lachmann introducing institutions as points of orientation for acting people. Then, he believes there can be developed a~rudimentary general theory that can capture issues e.g. of elasticity of institutions and cohesion of orders \parencite[8]{lachmann-weber}.

\subsection{Was Lachmann a~nihilist?}


The comments on the subjectivism, ideal types, and the relative neglect of the \emph{a~priori} theory present in Lachmann's works were likely the cause of Rothbard's ire.
Although Rothbard was happily endorsing \emph{Capital and its structure} by Lachmann as a~work in the Misesian paradigm, he stated that by the mid-1970s there was a~significant break in Lachmann's thought related to his ``conversion to Shackleinism'' (\cite[53]{rothbard-present}; see also \cite{barbieri2021lachmann}) leading to his `` crusade to bring the blessings of randomness and abandonment of theory to Austrian economics'' \parencite[56--57]{rothbard-hermeneutic}. To quote Rothbard at length:
\begin{quote}
Lachmannian Man knows no economic law, no law of cause and effect, qualitative or quantitative. In fact, he can have no \emph{Verstehen} into patterns that are likely to occur in the future. At every moment of succeeding time, Lachmannian Man steps into a~trackless void [\ldots] Money? Prices? They can have no relation to the future, qualitative or quantitative, which means they are not causally related at all \parencite[52]{rothbard-present}.\footnote{Curiously, twenty years earlier \textcite[50]{rothbard-praxeology-method} quoted Lachmann approvingly in that ``the Austrians were endeavoring to construct a~`verstehende social science', the same ideal that Max Weber was later to uphold''.}
\end{quote}
In short, Rothbard adds that by assuming the radical uncertainty of the future Lachmann confined himself to the studies of the past.
Then, we can pose a~simple dilemma: either we have causal theories in social science, and thus the future is somewhat (even though imperfectly) knowable, or we do not have ones---but then there appears a~problem with how we can interpret the past. And Lachmann by discarding the former case has to embrace the untenable second one \parencite[53--54]{rothbard-present}.

However, note that even late in his life \textcite[140]{lachmann1986market} did not consider himself a~nihilist. Rather he called nihilists those looking for mechanical causation in social sciences, despite all the problems that subjectivists raise against this line of research.
He still believed that with all their limitations economists can render useful services to society in a~kaleidic world \parencite[7]{lachmann-shackle-place} and stressed that
\begin{quote}
if we accept that we have to seek the causes of human action in ends pursued and the constraints operating in such pursuit, causal analysis in terms of the orientation of the various actors at various points of time during a~course of action appears quite possible \parencite[200]{lachmann-hayek}.
\end{quote}

On several points, Rothbard's criticism sounds too harsh and not justified enough.
For example, he asserted that ``the past is, in principle, absolutely knowable; the future is absolutely unknowable'' \parencite[52]{rothbard-present}, but he forgot to add that Lachmann qualified it by saying that the future is not unimaginable (\cite[194]{lachmann-hayek}; \citeyear[215]{lachmann-vicissitudes}; \citeyear[265]{lachmann-speculative-markets}).
And it was already Mises who in his \emph{Theory and History} pronounced that ``one of the fundamental conditions of man's existence and action is the fact that he does not know what will happen in the future'' \parencite[180]{mises-theory} and ``what a~man can say about the future is always merely speculative anticipation'' \parencite[203]{mises-theory}.
This is in complete agreement with Lachmann's own words that ``a world of uncertainty is not a~world of chaos'' and our condition compels us to make forecasts about the success of our actions, but we just cannot have any scientific ones \parencite[139]{lachmann1986market}.
In this view, he rather restricts the uncertainty problem to a~lack of exact predictions, while still allowing for informed guesswork in ordinary action based on \emph{Verstehen} \parencite{lewin-life}.
Curiously, even in the case of the subjectivism of expectations, one can find in the writings of Mises thoughts congenial to Lachmann:
\begin{quote}
There is neither constancy nor continuity in the valuations and in the formation of exchange ratios between various commodities. Every new datum brings about a~reshuffling of the whole price structure. Understanding, by trying to grasp what is going on in the minds of the men concerned, can approach the problem of forecasting future conditions. We may call its methods unsatisfactory and the positivists may arrogantly scorn it. But such arbitrary judgments must not and cannot obscure the fact that understanding is the only appropriate method of dealing with the uncertainty of future conditions \parencite[118]{mises-ha}.
\end{quote}

Similarly, when Rothbard \parencite*[57]{rothbard-present} declares that ``by tossing out equilibrium concepts altogether, and in concentrating only on market processes, Lachmannians and other non-Misesian Austrians fail to realize that they thereby give up any chance of understanding those processes themselves,'' it is not directed against Lachmann, as he was declaring that ``equilibrium analysis is a~necessary first step on our way to causal explanation, a~means towards an end'' \parencite[198]{lachmann-hayek}.

And when Rothbard wrote that
\begin{quote}
In value theory, the non-Misesians, especially the Lachmannians, neglect or deny the objective fact that physical objects are being produced, exchanged, and evaluated, albeit that they are subjectively evaluated by acting individuals \parencite[50]{rothbard-present},
\end{quote}
he clearly forgot that it was his teacher Mises who pointed out that ``Economics is not about goods and services; it is about human choice and action [\ldots] The sole task of economics is analysis of the actions of men, is the analysis of processes'' \parencite[354]{mises-ha}.
That said, Lachmann would never deny that plans in the sphere of production determine the uses of capital goods, i.e. stocks of material resources \parencite[for example in][10--11]{lachmann1956capital}.

Moreover, it is too far-fetched to identify Lachmann's views with Shackle. For example, in his early review of Shackle, he rightly notes that the kaleidic claim, if it was taken literally, would imply that ``there could be no testing the success of plans, no plan revision, and no comparison between \emph{ex ante} and \emph{ex post}'' \parencite[84]{lachmann-shackle-time}. Therefore, he postulates a~clear delineation, allowing for intertemporal comparisons concerning knowledge of relations between means and ends while admitting discontinuities of human ends.
Certainly, Rothbard pointed to the change that occurred somewhere until the mid-1970s, but it can be easily interpreted as a~change of emphasis.
For example, the late Lachmann was still known to convince Shackle later in his life to admit the role of institutions as points of orientation for agents in the uncertain world \parencite[in][31]{dekker-lachmann}.
And while discussing kaleidic markets he still throws an off-hand remark that ''Marshallian markets for individual goods may for a~time find their respective equilibria'' \parencite[61]{lachmann-kaleidic}.

Overall, general denigration of \emph{a~priori} theory by Lachmann is not limited to his later years, and bears resemblance to the comments Hayek formulated against the pure logic of choice, cited favorably by \textcite[57]{lachmann-significance}. And by Hayek's own admission, this was directed also against the Misesian approach to economic theory:
\begin{quote}
[M]y 1937 article on the economics of knowledge [\ldots] was an attempt to persuade Mises himself that when he asserted that the market theory was a~priori, he was wrong; that what was a~priori was only the logic of particular action, but the moment that you passed from this to the interaction of many people, you entered into the empirical field (\cite[72]{hayek-on-hayek}; see also \cite[Lachmann quoted in][35]{selgin}).
\end{quote}
Thus, the real point of contention is that Hayek and Lachmann relied in the latter context on the considerations about knowledge, where there can be no definite laws.
This combines well on the one hand with his criticism of behaviorism and purely observational language in economics, but on the other hand with his negative remarks about any talking of stable dispositions as inoperative as they change over time, sometimes very rapidly \parencite[11]{lachmann-weber}. He includes preferences, plans, knowledge, and expectations as the central notions of analysis, but only as terms denoting momentary dispositions.

Mises and Rothbard distinguished formal (universal) and material (contingent) aspects of actions by restricting theory only to an inquiry into the formal side (see e.g. \cite[31--32]{mises-ha}; \cite[83]{rothbard-mes}). Therefore, they can be easily seen as more interested in isolating certain singular causal processes in the social world under \emph{ceteris paribus} clause or using counterfactual reasoning. This is exactly why they developed the Austrian theory of growth based on the analysis of singular changes in time preference or the Austrian theory of a~business cycle based on tracing a~single injection of new money substitutes into a~credit market.
This, contra Hayek and Lachmann, could be a~case for \emph{a~priori} laws in the sphere of catallactics---however to argue for the full-blown theories of growth and business cycle we also need to trace the subsequent changes, and they clearly would proceed differently depending on the particular pattern of knowledge dissemination, which indeed complicated the picture.
And contra Menger who claimed that the laws of economics are as exact as in natural sciences, Lachmann was the first to correctly object that such determinism would contradict freedom of the human will \parencite[59]{lachmann-menger}.
In short, Lachmann would not even need to dispute if the claim ``that if the money supply increases and the people's demand for money remains the same, prices will rise'' (cited in \cite[52]{rothbard-present}) is an absolutely true, apodictic praxeological law, but he could just complain that one of the antecedents (constant demand for money) is virtually never true and thus hardly relevant to the real world.\footnote{See also similar doubts about the quantitatively determinable law of demand in the absence of error and ignorance in \textcite[58]{lachmann-menger}.}

Additionally, a~Weberian economic sociologist could be much more interested in the totality of social causation, including secondary chains contingent on particular characteristics of an epoch, a~market, etc. For Mises it would not count as praxeology, but rather thymology, a~purely historical discipline\footnote{As one of the commentators noted, ``from Mises's perspective, Lachmann is interested in the methods of history, not those of economics \parencite[37]{parsons}.} \parencite[272--274]{mises-theory}.
And indeed later in his life, Lachmann called for the ``economic sociology'', general theory of institutions along Weberian lines \parencite[277--278,282]{lachmann-hermeneutic}.

Seemingly, often Lachmann had none of these subtleties in mind e.g. when he claimed that in \emph{Human Action} ``it is the work of Max Weber that is being carried on'' \parencite[95]{lachmann-ha}, and when he downplayed the Misesian distinction between \emph{Verstehen} and \emph{Begreifen} as the methods of historical and theoretical inquiry, respectively \parencite[49]{lachmann-significance}.
Of course, Mises acknowledged his intellectual debt to Weber \parencite[79]{mises-epe}, but unlike Lachmann it was not done without serious qualifications.
And many Austrians, contra Lachmann, would rather frame it in a~way that leaves out the necessity apodictic, yet open to counter-operation of some other contingent causes or limited to the cases when entities in question (such as humans, society, money) exist \parencite[57]{rothbard-present}.

However, there is also one common point between Lachmann, Shackle, and Weber, separating them from the Misesian paradigm.
It is the case that all agreed on the importance of more particular studies, and constructing theory in a~bottom-up fashion instead of searching for large, comprehensive theoretical systems.
Interestingly, it is clearly in line with the famous phrase of Joan Robinson, borrowed by another idiosyncratic Austrian Joseph Schumpeter, that ``economic theory is a~box of tools'' \parencite[15]{schumpeter-history}, that neatly described the approach that is dominating in the mainstream since the post-war period \parencite{morgan,rodrik}.

In scarce remarks on a~general concept of science in his earlier writings, Lachmann defines science as ``systematic generalizations about observable
phenomena'' \parencite[166]{lachmann-science} and he argues for the similarity between scientists forming working hypotheses and businessmen forming their expectations, picking the right concepts for the problem at hand\footnote{Note the striking similarity to the quote ``It is not enough for the statesman, the politician, the general, or the entrepreneur to know all the factors that can possibly contribute to the determination of a~future event. In order to anticipate correctly they must also anticipate correctly the quantity as it were of each factor's contribution and the instant at which its contribution will become effective. And later the historians will have to face the same difficulty in analyzing and understanding the case in retrospect'' in \textcite[314--315]{mises-theory}.} (\cite[90, 93]{lachmann-shackle-time}). If theories of social sciences differ from commonsense generalizations only by a~degree of systematicity and prudence involved in their formation, then there is no reason to state such hard distinctions.
In fact, Lachmann seems to be leaning toward this view when he mentions that the proper understanding of the past taking into account nuances of subjective interpretation helps to recognize e.g. which current problems are the most urgent \parencite[240]{lachmann-hermeneutic}.

This leads us to a~problem that was identified by Rothbard when he wrote that ``they could be called ``historians'' except they do very little actual historical work'' \parencite[53]{rothbard-present}.
One can justifiably ask: is there any lasting value for example to Lachmann's comments about differences between fixprice and flexprice markets, or a~division of processes into intra-market, inter-market, and macroeconomic ones? \parencite[6--14]{lachmann1986market}
Awkward silence on this issue by younger generations of Austrian economists inspired by Lachmann can serve as evidence that ultimately this did not bring anything important to the table.

\section{The method applied: financial markets}
\label{sec:finance}

In our view a~defense of the Lachmannian subjectivist-institutionalist project would be incomplete if it were concluded on the philosophical plane.
And probably the best way to prove the fruitfulness of methodological pronouncements is to put them into practice.
Fortunately, with Lachmann we can find examples that show his adherence to the professed method in his economic works---so let us concentrate on one, often overlooked example of his research interests, that is, the topic of financial markets.

As Lachmann notes, ``in the real world there are markets and markets'', and abstracting from their differences can easily lead one astray \parencite[263--264]{lachmann-speculative-markets}.
And it is clear that if there is one institution that distinguishes capitalism from other economic systems, it is the capital market.
Lachmann agrees with this claim completely when he states that
\begin{quote}
[m]arkets of course may exist in a~centrally administered economy [\ldots] but markets for capital assets, and thus for financial assets, cannot exist in a~socialist economy. [\ldots]
asset markets, and in particular a~Stock Exchange embedded in a~network of financial asset markets, form the core of a~market economy: they are in fact its central markets\footnote{Interestingly, Mises too once said to Rothbard that ``a stock market is crucial to the existence of capitalism and private property'', and it serves as the criterion to distinguish capitalism from socialism \parencite[426]{rothbard-stock}.} \parencite[255]{lachmann-monetary}.
\end{quote}

Lachmann distinguished two classes of agents in intertemporal markets: hedgers and speculators (\cite[10]{lachmann1986market}; \citeyear[264--265]{lachmann-speculative-markets}). The first typically want to ``cover a~position they for other reasons have to take up, for example, to protect stock they hold against depreciation through fall in price, or to ascertain their ability to buy future input into production processes under their control'', whereas the second just wants to earn profits from intertemporal price changes.
Note that speculators are not exactly arbitrageurs, because they do not secure their position by buying and selling the same good at the same time \parencite[10]{lachmann1986market}.

He, however, often stresses another property of financial markets, that is, their speculative nature:
\begin{quote}
without divergence of expectations there can be no market at all, we can say that this divergence provides the substrate upon which the market price rests \parencite[161]{lachmann-model}.
\end{quote}
Note that this claim can sound problematic to Austrian economists in the tradition of Mises and Rothbard: although it is true that the real-world financial markets exhibit a~high divergence of expectations, its existence is not necessary for the transactions to occur. For example, such economists could claim that financial markets are ultimately just capital markets, where people trade not only because of existing uncertainty but also because of differences in their time preferences \parencite[376--378]{rothbard-mes}. Both functions are important: first serves as a~selection process of people with better entrepreneurial skills, who are rewarded with monetary profit; second allows for adjustment of investment to the interest rate as a~social expression of individual time preferences \parencite{klein-entrepreneurship}.

Lachmann often underlines that since financial markets are speculative, they have a~peculiar quality that agents can far more easily switch sides of transactions compared to the more traditional commodity markets, which in turn leads to the peculiar volatility of asset markets (\cite[42]{lachmann1986market}; \citeyear[267]{lachmann-speculative-markets}).
In comparison, ``ordinary'' markets have stable underlying patterns of supply and demand, ``which provides all participants a~common point of orientation'' for expectation convergence \parencite[264]{lachmann-speculative-markets}.
Furthermore, in his 1976 article, Lachmann claims that
\begin{quote}
[i]n an asset market in which the whole stock always is potentially on sale and in which everybody can easily choose or change sides, we find an element of volatility that is absent from the product market. Such asset markets are inherently ``restless'', and equilibrium prices established in them reflect nothing, but the daily balance of expectations. In the cotton market, for example, it is likely that expectations about the probable price in July 1976 will tend to converge as this date draws nearer. But this cannot happen in the Stock Exchange, since what is being traded there are titles to (in principle) permanent income streams, which have no ``date'' that could ``move nearer''. All we get is a~succession of market-day equilibria determined by a~balance of expectations tilting from one day to the next as the flow of the news turns bulls into bears and vice versa. There is here no question of a~gradual approach towards long-run equilibrium (\cite[60]{lachmann-kaleidic}, see also \cite[202]{lachmann-hayek}; \citeyear[161--162]{lachmann-individualism}; \citeyear[264]{lachmann-speculative-markets}).
\end{quote}

However, under this description stock market is equivalent to some kind of organized betting on some purely random events.
However, one may ask a~very simple question: what could be a~rationale for such a~market to systematically support coordination?
Clearly, betting markets help people with divergent expectations concerning such events to meet and engage in transactions, but despite realizing the double coincidence of wants e.g. stemming from the pure joy of betting it is hard to find any reason to call such markets ``coordination institutions''.
Unfortunately, Lachmann does not provide us with any indication what could be the difference between a~stock exchange and a~casino.
He is embracing the idea of the volatility of financial markets, as marked by the following quote:
\begin{quote}
It is a~typical feature of volatile speculative markets that strong price movements will attract outsiders to them so that either bulls or bears are continuously reinforced and a~given price trend is maintained. In such circumstances, market forces tending towards a~balance of bullish and bearish expectations may remain weak \parencite[259]{lachmann-monetary}.
\end{quote}
All this does raise a~question of why any follower of this argument should not agree with the famous comparison by Keynes:
\begin{quote}
It is usually agreed that casinos should, in the public interest, be inaccessible and expensive. And perhaps the same is true of stock exchanges \parencite[159]{keynes-gt}.
\end{quote}
If the profit and loss mechanism is not in place, then in volatile financial markets it is more plausible that the expectations often function as self-fulfilling prophecies that destabilize production structure e.g. as described in Financial Instability Hypothesis by another Post-Keynesian economist, Hyman Minsky \parencite*{minsky}.

In his earlier works, he clearly stated coordination forces on the market stem from knowledge
transmission through the price system, aligning the expectations of people and the structure of production (\cite[103]{lachmann-ha}; \citeyear[62]{lachmann1956capital}).
For example, he follows Mises that the market is
\begin{quote}
[a] process of redistribution of wealth [\ldots] not prompted by a~concatenation of hazards. Those who participate in it are not playing a~game of chance, but a~game of skill. This process, like all real dynamic processes, reflects the transmission of knowledge from mind to mind. It is possible only because some people have knowledge that others have not yet acquired because knowledge of change and its implications spread gradually and unevenly throughout society \parencite[313]{lachmann-market-distribution}.
\end{quote}
Similarly, for assets markets
\begin{quote}
the sources of income streams are revalued every day in accordance with the prevailing balance of expectations, giving capital gains to some, and inflicting capital losses upon others. What reason is there to believe that interference with this market process is any less detrimental than interference with the production and exchange of goods and services? Those who believe that such a~reason does exist (and most of our contemporary ``welfare economists'' do!) must assume that asset holders, like Ricardian landlords, somehow stand outside all market processes and ``get rich in their sleep'' \parencite[163]{lachmann-model}.
\end{quote}
And for Lachmann this ``continued redistribution of wealth in a~market economy'' \parencite[202]{lachmann-hayek} has an important function:
\begin{quote}
Stock Exchange ``monitors'' the performance of managers. [\ldots] The shareholder watches these prices and draws his conclusions. When he disapproves of some action by his managers he `votes with his feet'---he sells. [\ldots] Owners and managers, so far from being `separated' from each other, are linked together indirectly through the market \parencite[249]{lachmann-legislation}.
\end{quote}
This way he sounds like Mises, who declared ``the more profits a~man earns, the greater his wealth consequently becomes, the more influential does he become in the conduct of business affairs'' \parencite[23]{mises-profit}.

So, did Lachmann in the later years change his mind on the price system and its function?
It is clear that as late as 1967 he contended that
\begin{quote}
while it is true that in an uncertain world present prices cannot offer entrepreneurs more than a~basis of orientation for their plans, it is also true that the disappearance of this basis must constitute a~serious loss \parencite[300]{lachmann-causes}.
\end{quote}
And although in later works Lachmann did not return to this issue, it is not clear if he repudiated them in any form or just shifted his attention to other aspects.
As long as he remained in agreement with Mises on this point, this provided the missing puzzle, which does not allow to equate markets with games of chance \parencite[221]{manish}.

However, there is another puzzling statement about the expectations:
\begin{quote}
[e]ach one of us catches a~different glimpse. The wider the range of divergence the greater the possibility that somebody's expectation will turn out to be right \parencite[59]{lachmann-kaleidic}.
\end{quote}
This claim is obviously true, but on a~closer look, it has no explanatory power. The success of a~single plan among the masses of failures would not give any hint of the apparent functioning of capital markets, as admitted even by many critics of capitalism.

Although Lachmann did not say it directly, he probably would appreciate an intrinsic advantage of asset markets stemming from their network character. Capital goods can change hands more easily, which is especially important for durable ones that were created with some plan in mind, which turned out to be inconsistent. And some other people may bid on them, to use them in their plans.

Overall, it can be said that Lachmann was only emphasizing the problems that were neglected by some Austrians, especially the ones sympathetic towards some kind of general equilibrium perspective\footnote{See \textcite{salerno-place,salerno-wieser} for a~parallel view of two traditions in Austrian economics, one causal-realist, more in line with the market process approach, and another relying on a~verbal general equilibrium analysis.}, while repeating after Mises the essential functions of capital markets continued well into the 1970s and 1980s.

\section{Conclusion}

As with other Austrian economists, assessing Lachmann's deep philosophical influences has to include the fact that he was not interested in philosophy for its own sake, but rather to develop a~useful alternative to formal neoclassical models of production with their mathematically convenient assumptions. In the beginning, he was trying to make the point to his fellow economists about the importance of commonsensical characteristics of capital goods, such as their heterogeneity or limited specificity, and their dependence on the use of knowledge and expectations in society.

Over the years, he refined his methodological views along the lines of subjectivist institutionalism, taking inspiration from Max Weber (institutions) and G.L.S. Shackle (subjectivism), and ultimately arrived at the stance that appeared out of line with the orthodox approach represented e.g. by Mises, Hayek, and Rothbard.
This prompted Rothbard to criticize Lachmann as ``opposed to even the possibility of economic theory'', ``no longer economists at all'', or even ``professional anti-economists and meta-historians, expending their energies denouncing economics and urging other economists to act as historians'' \parencite[53]{rothbard-present}.
Unfortunately, this criticism largely stemmed from a~misunderstanding of Lachmann as a~traitor of the Austrian banner\footnote{The is somewhat understandable, because some of his particular insights appear only in passing, for example when he is commenting on works of other economists, such as Mises or Shackle. Instead, in more programmatic publications he does not repeat some crucial insights or reservations that nuance his line of reasoning.}, rather than a~heavily Austrian-influenced institutionalist with a~decisive subjectivist bent with a~modest, eclectic, and ecumenical approach.

At the same time, one is under a~clear impression that Lachmann is deliberately trying to emphasize similarities between them while downplaying the fundamental differences e.g. between Austrians and subjectivist Keynesians (e.g. in \cite[184]{lachmann-keynes}).
He is always eager to praise subjectivist and institutionalist endeavors of such non-Austrian thinkers as John Hicks (\cite[218]{lachmann-vicissitudes}; \citeyear[184]{lachmann-keynes}), Luigi Pasinetti \parencite[164]{lachmann-salvage}, or Paul Davidson \parencite[166]{lachmann-salvage}, and calls for brokers of ideas, who could assimilate ideas stemming from different paradigms \parencite[282]{lachmann-hermeneutic}.
Even when Lachmann credits Mises, Hayek, and their disciples as the ones ``concerned with meaningful action'' and emphasizing institutional aspects of the economy, he does so in one breath with a~mention of ordoliberals and disciples of Weber \parencite[251--253]{lachmann-legislation}.

Unfortunately, this approach obscures some theoretical problems, for example, the completely different price and entrepreneurship theories, which lie at the heart of understanding the market process and its main institutions such as financial markets, probably the most important institutions in developed capitalist economies.
However, this does not mean that the particular analysis or insights could not be transferred between schools or paradigms.

Although Lachmann is rightly viewed as guilty by Rothbard for accepting at least \emph{prima facie} on equal footing different theories as possible explanations\footnote{Interestingly, Rothbard and Lachmann agree on one point: both are skeptical of theories emphasizing biological evolution, inspired by Hayek and popular for some time in Austrian circles (\cite[81]{rothbard-present}; \cite[Lachmann quoted in][26]{dekker-lachmann}).}, one should not be too quick in dismissing the whole Lachmannian enterprise as a~completely useless lesson for Austrians.
Of course, in such lines of research there is always the risk of wasting time developing distinctions with no lasting relevance. At the same time, no matter how powerful we judge the praxological theory to be, there is always a~huge room for purely historical research and it cannot be reduced to a~simple application of ready-made theorems.

First, as we have seen above, the program pursued by Lachmann was far from being anti-economics, but in practice allowed for some non-trivial insights into particular properties of financial markets. Second, many of his particular results could be directly assimilated by any Austrian economist stemming from the Misesian paradigm.
In doing so, one does not have to reject praxeology or extreme apriorism \parencite{rothbard-defense}, but one can fully embrace this line of research as thymological.

Austrian economists should remind themselves that they do not have only a~particular approach to studying human action but also developed a~system of theories according to this methodology.
When one is confronted with a~subjectivist approach from another strand of thought it may be not the case that these are just different models based on different stylized facts, capturing different aspects of price phenomena.
It may be also the case that our theory in question is indeed universal and immune to such external subjective objections, though by rethinking it we can understand better its strength or refine it. For example, it would be very instructive to check the core Austrian theories (e.g. price or money theory) and point out where other Austrian economists made unwarranted steps and went astray in their analyses along the similar lines as Lachmann, who tried to raise some issues concerning expectations and learning in his comments on the Austrian theory of the business cycle \parencite[123--124]{lachmann-expectations}.\footnote{See e.g. excellent work by \textcite{machaj-postkeynesian}, when an Austrian economist confronts Post-Keynesian arguments for mark-up pricing and shows that indeed rightly understood Austrian price theory including B\"ohm-Bawerk's law of costs is compatible with these arguments, and thus immune to a~valid criticism directed towards the neoclassical price theory.}

Finally, Lachmann's remarks may be helpful as a~guide for some Austrian economists more interested in developing theories of particular markets or providing some case studies.
However, in the area of history the proof of the pudding is in eating---and as Lachmann himself noted, any progress in this area has to be judged \emph{ex post} by the value of particular insights, not by merely being faithful to the right pronouncements.
After all, in economics, it is not the plausibility of Austrian methodology that is the major argument for endorsing it---but rather the fact that this method can be indeed used to develop a~large body of useful theories and relevant explanations.\footnote{Interestingly, one of the students of Lachmann assessed that his methodological ideas like kaleidic world would not stand the test of time, unlike some of his contributions to the economics proper \parencite[388]{boehm2000professor}.}








\end{artengenv}

\label{turowski-last}
\begin{artengenv2auth}{Anna Ceglarska, Katarzyna Cymbranowicz}
	{The role of \textit{phronesis} in knowledge-based Economy}
		{The role of \textit{phronesis} in knowledge-based Economy}
		{The role of \textit{phronesis} in knowledge-based Economy}
	{University of Sydney}
	{The aim of this paper is to reflect on contemporary understanding of ``knowledge'' within the Knowledge-Based Economy. Since the pursuit of knowledge has been a~longstanding focus of European culture since Greek philosophy, we employ original ancient terminology. Applying the hermeneutics of ancient texts along with critical and comparative analysis can aid in differentiating between ``knowledge'' and ``wisdom'', often linked in modern theories, while also connecting this issue to the Aristotelian concept of \textit{phronesis}. The authors argue that since human relations impact social (and so---economic) spheres, the issue of phronesis, a~relational type of knowledge, should not go unexamined. The idea that application of knowledge (rather than its mere acquisition), crucial for the Knowledge-Based Economy, was embedded in the Greek term \textit{oikonomiké}, which provides a~basis for considering oneself a~\textit{phronimos}. Our aim is to demonstrate the value of phronesis particularly in the fields of management and the philosophical foundations of economics, as the skills encompassed within it have the potential to aid in educating not only a~``sage'' but also an active member of the community, capable of acting in a~manner that benefits both themselves and the society.
		}
		{wisdom, knowledge, knowledge-based economy, phronesis, Aristotle, prudence, entrepreneurship.}
	{%
		{\flushright\subbold{Anna Ceglarska}\\\subsubsectit\small{Jagiellonian University}\par}%
		{\flushright\subbold{Katarzyna Cymbranowicz}\\\subsubsectit\small{Krakow University of Economics}\par}%
	}




\section{Introduction}

\lettrine[loversize=0.13,lines=2,lraise=-0.03,nindent=0em,findent=0.2pt]%
{T}{}he inspiration for this text was the question posed by David Rooney and Bernard McKenna in their article: ``Should the Knowledge-based Economy be a~Savant or a~Sage?'' 
%\label{ref:RNDHcH6MjXY2B}(2005).
\parencite*[][]{rooney_should_2005}. %
 The answer we propose is---neither. The problem that Rooney and McKenna pose is: should we not be demanding an economy based on wisdom, i.e. that decision-makers should not only acquire knowledge but also the wisdom that results from it? On the basis of ancient considerations, we put up for discussion the fact that replacing knowledge with wisdom (or making the Knowledge-Based Economy a~Sage) may not be as desirable as it seems. Yet the alternative seems to be the economy based on the knowledge of technocratic experts, who could measure its development with indicators, treating it like another material that can be measured and formed by some higher authority. This opening question firmly establishes a~distinction between a~fully ethical and almost altruistic economy (that of the sages), or a~technocratic one, focused on the goal, expressed in sets of indicators (that of the savants). And given the importance of the Knowledge-Based Economy nowadays, the first alternative is quite tempting. The idea that wisdom should govern our lives in all aspects: political, economic, social, etc. is not new. As Alfred North Whitehead famously said: ``The safest general characterization of the European philosophical tradition is that it consists of a~series of footnotes to Plato'' 
%\label{ref:RNDh1GxWPqkx7}(Whitehead, 2010, p.39).
\parencite[][p.39]{whitehead_process_2010}. %
 And yet, according to Karl Popper, among others, Plato's ideal state, ruled by wise men (we deliberately avoid the term ``philosophers'' here), is seen as a~pre-totalitarian one, denying its citizens most of the freedoms we cherish today.



The question that we see poised before us is not the one that obliges us to choose between savant, sage, or a~fool or an \textit{astute ignoramus}, as this terms are also used by Rooney and McKenna 
%\label{ref:RNDqeDmHPcgRi}(2005, p.315),
\parencite*[][p.315]{rooney_should_2005}, %
 but rather between a~sage, a~savant, and a~prudent man, who, as we argue, contrary to the aforementioned authors and following the footsteps of Aristotle and Greek philosophy, is not the same as the wise-man. Since ancient times, knowledge has had many different names, referring to its different aspects and qualities: \textit{episteme} (scientific knowledge), \textit{techne} (technical knowledge) and \textit{phronesis} (practical knowledge, prudence), while \textit{sophia} has been associated with wisdom 
%\label{ref:RNDulINJNZRMa}(Aristotle, 1934, VI.6.3).
\parencite[][]{rackham_nicomachean_1934}. %
 The latter has also been perceived for centuries as the ultimate goal of man and has been the object of interest and analysis since ancient times. The wise-man knows everything and can therefore make the best decisions and give the best advice. In Plato's ideal state the philosophers---those who love wisdom---should be the decision-makers. Instead, we would propose to understand the ``knowledge'' in the Knowledge-Based Economy as the Aristotelian concept of \textit{phronesis}, usually translated as ``practical knowledge'' or ``prudence''. Therefore, our first goal is to establish the difference between \textit{sophia} and \textit{phronesis} in the present-day world. We wish to reflect on today's understanding of ``knowledge'' within the Knowledge-Based Economy and, by referring to ancient terminology, evaluate if there has been an unjustifiable association of ``knowledge'' with the notion of ``wisdom''. Furthermore, we would like to consider whether this ``knowledge'' is not treated similarly to all other resources, such as labour and capital. Finally, we shall emphasise the importance of the relational nature of knowledge, which involves not only acquiring it but also the ability to apply it in practical situations. That would lead to our second goal: to see whether, and if so, how, the concept of \textit{phronesis} can be found useful in current interpretations and what can still be learnt from it. Given the specificity of the \textit{phronesis} itself, as it relates to individuals and their relations with the community, we aim to demonstrate the value of \textit{phronesis} for leaders and managers\footnote{Since we are basing our discussion on Aristotle's concepts, we use the terms ``manager'', ``leader'', and ``phronimos'' interchangeably, as Aristotle viewed these spheres of life (economic, social, and political as well) as deeply interconnected. Further elaboration on Aristotle's thoughts regarding the relationship between the economic and political spheres is provided in the subsequent chapter.}, as it pertains to the relational aspects of human society. Our argument is that the ability to function in relationships cannot be left unexamined, as the ``human factor'' significantly affects both economic and social spheres.



In order to achieve these goals, we have mostly employed a~desk research method combined with hermeneutics of ancient texts, enriched with the elements of the critical analysis, developed by CLS. The comparative analysis was undertaken to confront the meaning of different types of knowledge in our source material (Aristotle) and modern interpretations. As we were primarily interested in exploring the philosophical foundations of economics, we deliberately avoided researching specific economic trends that would require research methods and techniques characteristic for the discipline of economics. First, we will discuss some basic ancient concepts, starting with the \textit{oikonomiké} (\textit{oeconomica}) itself, as it shall, along with its differentiation from \textit{chrematistiké}, will lead us to an important distinction between ``action'' and ``accumulation''. The ability to ``act'' we see as a~fundamental value that leads to the attainment of knowledge, a~good life and general progress. The next part shall be devoted to establishing some general meaning of the Knowledge-Based Economy and within this we will consider whether, despite the lexical distinction, we have not made an unauthorised identification of knowledge with wisdom in contemporary discourse.



These two concepts have been distinguished since ancient times. Today, however, most theories focus on what we possess, be it wealth, capital or knowledge, rather than on what we do with this good (or how we do it). We therefore remain mostly in the realm of \textit{chrematistiké}, and---transferring it to the philosophical realm---\textit{vita contemplativa}. A~true sage who lives such a~life, becomes more and more immersed in himself through contemplation, striving to see the Good and the Truth---universal and unchanging ideas\footnote{At this point, it should be noted that the terms Good and Truth, written with a~capital letter in the text, denote to these values understood as the highest ones, Ideas in the Platonic sense. ``Ordinary'' good and truth (in lower case), which appear e.g. in the context of the Aristotelian ``good life'', are already relativised. Hence, they are not Platonic Ideas, but only exist in the realm of opinion (\textit{doxa}).}---and thus alienates himself from other people. On the other hand, \textit{vita activa}, like \textit{oeconomica}, is concerned with human affairs, relationships and actions that arise from ``the fact that men, not Man, live on the earth'' 
%\label{ref:RNDe9sDCL8Id7}(Arendt, 1998, p.7).
\parencite[][p.7]{arendt_human_1998}. %
 It belongs to the realm of \textit{praxis}, which includes all kinds of active engagement with the things of this world. Thus, politics, like economics, is about people and is only realised in relationships with others and in action.



\section{From \textit{oikos} to economy}

The term ``economy'' itself comes from the Greek \textit{oikos} ($o\text{\textgreek{>~i}}\kappa o\varsigma $), which is often translated simply as ``household'', but it is worth noting that its use in Greek is often broader and also changes depending on the context 
%\label{ref:RNDgh8e6xnQbO}(Roy, 1999, p.2).
\parencite[][p.2]{roy_polis_1999}. %
 For \textit{oikos} is not only a~mere ``house'' in the sense of building, but also ``home'', understood as a~community living in the same house (in the narrowest sense---family, however it could also include more distant relatives, servants or slaves), but it could also be used to mean the ``king's house'', as the whole dynasty 
%\label{ref:RND9lTeKUR8P5}(Herodotus, 1920, V.31, VI.9)
\parencite[][V.31, VI.9]{herodotus_histories_1920} %
 and in material terms was akin to the \textit{patrimonium}, the inheritance running through the whole lineage, and extending beyond a~single generation. Thus, while \textit{oikos} was a~basic social unit, it was not limited to what Cheryl Anne Cox 
%\label{ref:RND4DhHfzL5Yj}(1998)
\parencite*[][]{cox_household_1998} %
 calls the ``nuclear family'', since the activities associated with \textit{oikos} could be, in today's sense, strictly private ones (concerning only the family unit), but also extended to public affairs and people outside the particular \textit{oikos} 
%\label{ref:RNDMrQ2gLXGjK}(Martin, 2016).
\parencite[][]{martin_urban_2016}.%




The inherent relationship between these two spheres of life---the private and the public---is a~point of emphasis for Werner Jaeger: man leads ``besides his private life a~sort of second life, his \textit{bios politikos}. Now, every citizen belongs to two orders of existence'' 
%\label{ref:RNDGM9BYZCb6p}(Jaeger, 1946, p.111).
\parencite[][p.111]{jaeger_paideia_1946}. %
 Indeed, there were important and unmistakable links between the \textit{oikos} and the \textit{polis} (state). For Aristotle 
%\label{ref:RNDKq7obwJ3DY}(Aristotle, 1934, 1253b.1),
\parencite[][1253b.1]{rackham_nicomachean_1934}, %
 the \textit{oikos} is the basis for the functioning of society and the state, the smallest unit of the human community. Because of the primacy of the community prevailing in ancient Greece, actions within the \textit{oikos}, while remaining private, had political significance for the \textit{polis} \label{ref:RNDgDEuVBJb3a}\textit{(Roy, 1999, p.4)}. Therefore, even if treaties devoted to ``economy''\footnote{From which the most famous ones were the dialogue of Xenophon (\textit{Oeconomicus}) and a~treatise attributed to Aristotle (\textit{Oeconomica}).} dealt mainly with the household management, some scholars emphasize that they could also be seen as a~guide to the successful management of state. Aristotle 
%\label{ref:RND6ONKIbu2Pz}(1920, p.1345b)
\parencite*[][p.1345b]{aristotle_oeconomica_1920} %
 for example distinguishes the type of ``political economy'', which seems to be the closest to our current understanding\footnote{Especially since, due to changes in the political arena, other types distinguished by the philosopher (namely royal and satrapic) have been absorbed into it, as the coinage, exports, imports or taxes are now also the areas of state regulation and action.}.



Naturally, the direct application of the Aristotelian framework of concepts and definitions raises a~number of questions today. It has been argued more than once in the literature that the Stagirite used a~different concept of the economy, focusing primarily on domestic issues. However, as Ricardo Crespo 
%\label{ref:RNDn6ZuWTwuj2}(2010)
\parencite*[][]{crespo_aristotle_2010} %
 rightly points out, this does not imply that Aristotelian thought is completely irrelevant, nor that it lacks links or foundations for today's thinking about the economy, its goals and the rules that govern it. However, our intention is not to analyse the Aristotelian concept of the economy, as this has already been done quite well by many scholars 
%\label{ref:RNDbG68dQ89fZ}(e.g., Crespo, 2006; 2010; Meikle, 1995; Soudek, 1952; Pack, 2008; Finley, 1970).
\parencites[e.g.,][]{crespo_ontology_2006}[][]{crespo_aristotle_2010}[][]{meikle_aristotles_1995}[][]{soudek_aristotles_1952}[][]{pack_aristotles_2008}[][]{finley_aristotle_1970}.%




In the context of our analysis, it is worth noting Aristotle's distinction between \textit{oikonomiké} and \textit{chrematistiké}. It is worth reflecting on the implications of this division. In \textit{Politics} Aristotle poses the question: is the art of the acquisition of wealth ($\chi \rho \eta \mu \alpha \tau \iota \sigma \tau \iota \kappa \text{\textgreek{'h}}$) the same as the art of the management of the household ($o\text{\textgreek{>i}}\kappa o\nu o\mu \iota \kappa \text{\textgreek{~h|}}$). The answer he gives is negative 
%\label{ref:RNDlYCRRM81fB}(Aristotle, 1944, 1.1256a).
\parencite[][1.1256a]{aristotle_politics_1944}. %
 Whereas \textit{chrematistiké} is concerned with the acquisition of wealth, \textit{oikonomiké} focuses on its use. Moreover, if the accumulation of wealth (\textit{chrematistiké}) consists solely in the pursuit of capital accumulation and the expansion of one's wealth, rather than in the acquisition of things necessary to live, or to love well, it becomes, for Aristotle, something contrary to nature:



Consequently some people suppose that it is the function of household management to increase property, and they are continually under the idea that it is their duty to be either safeguarding their substance in money or increasing it to an unlimited amount. The cause of this state of mind is that their interests are set upon life but not upon the good life 
%\label{ref:RND4UP3Blb0Hi}(Aristotle, 1944, 1.1257b).
\parencite[][1.1257b]{aristotle_politics_1944}.%




Given the fact that the ``good life'' was the main objective of the \textit{polis} and that moderation was one of the most important virtues not only for Aristotle but for other philosophers as well, we can clearly see that the mere ``accumulation'' of wealth is not a~suitable way to live and manage the household, i.e., to participate in the broadly understood economy. The natural, proper acquisition of wealth is when the goods are being used not to obtain more goods, but for a~good life. Therefore, the basic factor that distinguishes between the natural, useful way of accumulating wealth from the unnatural, and therefore requiring the introduction of certain restrictions, is the way in which the accumulated wealth is used, or more precisely, how it is acted upon. Economics presupposes precisely the use and therefore the action on and with the goods. As stated by R. Crespo 
%\label{ref:RNDPkfW5EhU4e}(2006, p.772),
\parencite*[][p.772]{crespo_ontology_2006}, %
 ``\textit{Oikonomiké} is an action of using, in Greek, \textit{chresasthai}''.



This focus on action and human activity, particularly within a~community like the \textit{oikos} or \textit{polis}, appears to be fundamental in Aristotle's philosophy. It also applies to other areas of life and, in our opinion, can and should be used when interpreting the fundamental goals and objectives of the Knowledge-Based Economy.



\section{What is Knowledge-Based Economy?}

However, as the initial definitions of economics placed a~stronger emphasis on wealth, the notion of the Knowledge-Based Economy presents a~substantial breakthrough. This concept deviates significantly from these early endeavours in defining the economy and its corresponding regulations and customs, given that knowledge---rather than material goods, their value and distribution---now plays a~crucial role. In the 21\textsuperscript{st} century, knowledge became the crucial element at the heart of management theory and the pursuit of economic achievement. And so, ``Knowledge-Based Economy'' can be included into the rich catalogue of modern economic schools that focus on sustainable development 
%\label{ref:RNDqB6xoBbNAW}(e.g., Rogall, 2010; Shmelev, 2012; Raworth, 2017; Govender, 2021).
\parencites[e.g.,][]{rogall_ekonomia_2010}[][]{shmelev_ecological_2012}[][]{raworth_doughnut_2017}[][]{govender_rise_2021}. %
 It strives for development through the rational use of human resources (information, knowledge), not natural resources (land, raw materials) or financial resources (capital). Therefore, according to its assumptions, if we want to develop, we should invest in human capital. This leads to the conclusion that knowledge as a~resource plays an important role in shaping the socio-economic reality and effective harnessing of knowledge potential, including human intellectual potential, science, and research and development sphere, are the strategic factors that determine the pace and extent of socio-economic development today 
%\label{ref:RND8qWhMOHbcZ}(Skrzypek, 2012, p.193).
\parencite[][p.193]{skrzypek_gow_2012}. %
 In light of the above, it can be concluded that the continuous creation and use of knowledge is a~source of innovation and provides innovative solutions that are the basis for the creation of Knowledge-Based Economy 
%\label{ref:RNDgzfW31E5S6}(Zienkowski, 2003, p.15).
\parencite[][p.15]{zienkowski_gospodarka_2003}. %
 While traditional factors such as land, natural resources, labour or capital continue to impact socio-economic development opportunities, knowledge plays a~crucial role as it not only acts as a~new factor of production, but also coordinates others. The significance of knowledge is continuously rising, rendering it the primary factor of production and the key source of wealth. Elżbieta Skrzypek's statement that knowledge is the ``raw material of the future'' and the currency of the ``new economy'' is fitting in this context 
%\label{ref:RNDXIB6aQVPXn}(Skrzypek, 2018, p.20).
\parencite[][p.20]{skrzypek_gow_2018}.%




The phrase ``Knowledge-Based Economy'' is a~fairly recent addition to the world's literature, but it is receiving increasing attention. The first academic study to define the ``knowledge-based economies'' as ``economies which are directly based on the production, distribution and use of knowledge and information'' was a~research report produced by the OECD, entitled \textit{The Knowledge-Based Economy} 
%\label{ref:RNDsk8yzt3tgB}(OECD, 1996, p.7).
\parencite[][p.7]{oecd_knowledge-based_1996}. %
 Since then, however, the term ``Knowledge-Based Economy'' has not received a~single universally accepted definition, although it should be noted that the vast majority of proposed definitions are based on an attempt to list the characteristics that characterise it 
%\label{ref:RNDTTTY3NLs4T}(Chojnacki, 2001, p.80).
\parencite[][p.80]{kuklinski_wiedza_2001}. %
 However, this poses a~significant problem, which was observed \textit{nota bene} already in ancient times. In the Platonic dialogue \textit{Meno}, Socrates points out that providing numerous characteristics or instances of a~defined concept does not significantly aid our comprehension but rather poses additional issues. Using the case of colours, Socrates argues to Meno that attempting to define colour by giving an example or even listing all possible colours does not bring us any closer to a~general definition of what colour itself is. Furthermore, encountering a~new phenomenon poses a~significant problem in attributing it. A~better approach is to seek identifying features that they share in common 
%\label{ref:RNDtTciZTgOzk}(Plato, 1967a, 74c-77a).
\parencite[][74c-77a]{plato_plato_1967}. %
 Studies and analyses concerning the Knowledge-Based Economy encounter a~similar challenge. While emphasising the importance of information, knowledge, and intellectual capital in modern society and economy, they tend to expand this list to include other elements such as data, experience and wisdom. This catalogue of attributes can be almost endless, but paradoxically, it can divert attention from the fundamental concept. This can be clearly observed in the literature on the subject, where the general term ``Knowledge-Based Economy'' itself has several significant ``competitors'' to claim the title of the most precise definition of contemporary socio-economic reality. These include, among others: ``new economy'', ``digital economy'', ``knowledge-driven economy'', ``post-industrial economy'' or ``post-industrial society'', ``post-capitalist economy'', ``network economy'', ``third wave economy'' (or, again, society), ``service economy'', ``intangible resource economy'', ``information age'', ``knowledge society'' and several others. All of them describe the same reality, but it is impossible to definitively determine which set of characteristics accurately represents the current state of affairs.



Hence, without attempting a~universal definition, yet emulating Socrates in search of a~shared feature among them, enabling us to affirm that they ``all have one common character'' 
%\label{ref:RND3zSfnyjVk9}(Plato, 1967a, 72c),
\parencite[][72c]{plato_plato_1967}, %
 our attention centres on the realm of knowledge, with the aim of highlighting some fundamental problems.



\section{Between Knowledge and Wisdom}

Various definitions of knowledge can be found in the literature. This presents a~terminological challenge, which persists in a~Knowledge-Based Economy 
%\label{ref:RNDtkuJnEpVTX}(Winter, 1987; OECD, 2000).
\parencites[][]{teece_knowledge_1987}[][]{oecd_knowledge_2000}. %
 This is due to the fact that knowledge is an elusive resource that is complex to define, measure and apply in practice, given the limited conceptual resources, methods and techniques that are available in the current, post-industrial era 
%\label{ref:RNDFkXjpIAIPC}(Strojny, 2000, p.20).
\parencite[][p.20]{strojny_zarzadzanie_2000}.%




Peter F. Drucker 
%\label{ref:RNDuYUbdmITLw}(2013, p.7)
\parencite*[][p.7]{drucker_post-capitalist_2013} %
 emphasises that: ``[…] the basic economic resource---‘the means of production' to use the economist's term---is no longer capital, nor natural resources (the economist's ‘land') nor ‘labour' […] Value is now created by productivity' and innovation', both applications of knowledge to work. The leading social groups of the knowledge society will be ‘knowledge workers'---knowledge executives who know how to allocate knowledge to productive use''. Marcin Kłak 
%\label{ref:RNDR8f84tvwst}(2010, p.42)
\parencite*[][p.42]{klak_zarzadzanie_2010} %
 highlights that this unique situation is a~result of knowledge's indeterminate nature and the need for constant renewal, updating and modification. Only knowledge that is applied has any value as it serves progress, development and change, in other words, it is useful.



The pursuit of systematisation and the effort to create reasonably uniform yet comprehensive definitions of scientific concepts have generated several definitions of knowledge. According to Thomas H. Davenport and Laurence Prusak 
%\label{ref:RND6Vr58dCvG9}(1998),
\parencite*[][]{davenport_working_1998}, %
 knowledge, in contrast to data and information, is produced, developed and consolidated in the human mind as a~result of accumulated experience and learning---it is, so to speak, a~``product'' of the human mind, therefore it can be classified as either conscious (acquired systematically and intentionally through education) or unconscious (acquired unsystematically and unintentionally). In light of the above, it can be argued, in line with Michael Polanyi's thinking, that individuals are not always conscious of the knowledge they possess, and therefore also unaware of its worth 
%\label{ref:RND98uFE4rc5d}(Polanyi, 1966, p.37).
\parencite[][p.37]{polanyi_tacit_1966}. %
 Thus, data and information form the foundation of knowledge, which only becomes knowledge after it has been analysed. It is noteworthy to mention in this context the definition of knowledge proposed by Wiesław M. Grudzewski and Irena K. Hajduk, who differentiate between the concept of knowledge, understood as the application of information in practice, and wisdom, which is a~combination of knowledge, intuition and experience 
%\label{ref:RNDh20Yn52Jk1}(Grudzewski and Hejduk, 2004, p.73).
\parencite[][p.73]{grudzewski_zarzadzanie_2004}. %
 The distinction between these two elements, knowledge and wisdom, also dates back to antiquity. The philosopher, according to Plato, is defined as the one who ``loved wisdom'' (\textit{sophia}), and the acquisition of it constitutes his ultimate goal and desire. The famous metaphor of the cave depicted in the book VII of the \textit{Republic}, portrays \textit{sophia} as the sun, the source of pure, primal light that is, however, unattainable for most individuals 
%\label{ref:RNDGXaj6zfN6D}(Plato, 1969, VII, 514-516).
\parencite[][VII, 514-516]{plato_plato_1969}. %
 They sit in the cave observing only shadows, which are imperfect representations of the true object. The philosopher, however, can liberate themselves from constraints and, upon exiting the cave, step out into the sun and see the Truth.



The association of wisdom with the Truth holds significance in this context. An average individual typically possesses mere opinions (\textit{doxa}). Such opinions can have varying degrees of accuracy (or inaccuracy), lack the quality of certainty and completeness. In Plato's view, opinion is starkly contrasting to truth 
%\label{ref:RNDu3fhsJne72}(Arendt, 2005, pp.7–8).
\parencite[][pp.7–8]{arendt_promise_2005}. %
 Thus, while the multiplicity of opinions allows for discourse and persuasion, which are, after all, the foundation of Athenian politics, Truth is not subject to doubt or criticism. Furthermore, someone who has attained knowledge of the Truth through this intuition often chooses to retreat into \textit{vita contemplativa}, instead of taking action in a~social field, for they are incapable of describing the ``light'' to people mired in darkness. Such person does not receive understanding or attention from society, and he himself above all wishes to see more, to know more. The ancient \textit{sophia}, the knowledge of the sage, was the knowledge of the observer who merely watches the truth without interacting with it\footnote{This matter is also connected to the notion of ``theory''. The \textit{theoroi} were special envoys who observed customs and rituals in other \textit{poleis} (without engaging in them) and then reported their observations, enriching the knowledge of their homeland. For a~more comprehensive analysis of the role of observation in Greek culture, see 
%\label{ref:RND2JhLJamaGJ}(Ceglarska, 2022).
\parencite[][]{ceglarska_od_2022}.%
}. Such wisdom is absolute, but only few are able to possess it. Socrates' renowned statement, ``I know that I~know nothing'', arose from the fact that, unlike others, he was aware of his own limitations.



The paradox of modern understanding is that we expect the ``wise'' to possess full knowledge while being capable of challenging it. The wise person understands the workings of various social domains, including political, economic, and cultural aspects. Based on this understanding, they can accurately predict behaviours, consequences, and changes. However, their opinions may face criticism. As wise-men, they should be capable of defending their viewpoint and persuading others of its validity. Thus, they are, firstly, closer to Socrates, who walked among people, questioned and taught them, than to Plato, who preferred to observe. Secondly, they should possess the ability to accomplish what the archetypal philosopher, Socrates, failed to do, namely persuade others to adopt their viewpoint.



It was actually Plato's disciple, Aristotle, who adopted an approach that aligned more closely with Socrates' beliefs. For Aristotle, relationships play a~fundamental role in the human world, where practical knowledge, rather than wisdom, reigns supreme. For him prudence involves above all the ability to act---and after all, proclaiming one's position, teaching and persuading is an action. It is called \textit{phronesis}.



\section{Practical knowledge in Knowledge-Based Economy}

Based on Aristotle's \textit{Nicomachean Ethics,} it is indicated that the Greeks distinguished between three types of knowledge: \textit{episteme} (scientific knowledge), \textit{techne} (technical, manufacturing knowledge) and \textit{phronesis} (practical knowledge, although perhaps it should rather be called knowledge of action and its consequences, also known as ``prudence'' thanks to Cicero's translation). Notably, the concept of ``wisdom'' is absent from this framework. This is because wisdom is not simply knowledge, but rather something that can only be attained through it, in addition to some other essential elements, as defined earlier by W.M. Grudzewski and I.K. Hajduk. Aristotle believed that one of the most crucial elements is \textit{nous}, which translates to intuitive thinking or intuition. Thanks to \textit{nous}, individuals can discover the initial premises that form the foundation of knowledge, even if they are often indescribable. Although a~child may not be able to articulate the laws of physics, they intuitively comprehend the concept of gravitation to a~certain degree; their intuition informs them that objects fall. The possession of this intuition enables further exploration and acquisition of knowledge, however, not everyone possessing it, nor even those who specialise in a~particular field, count as a~``sage''. Socrates raises this matter somewhat mischievously in Plato's \textit{Republic}: ``Is it then owing to the science of her carpenters that a~city is to be called wise and well advised?'', to which his interlocutor replies: ``By no means for that, but rather mistress of the arts of building.'' 
%\label{ref:RNDTn2EF3lYv9}(Plato, 1969, IV, 428b-c).
\parencite[][IV, 428b-c]{plato_plato_1969}.%




``Master'' (of some craft) does not equate to being a~``sage''. This does not disregard the importance of craftsmen and their role in the state, which was considered the optimal community by Greeks. They are essential to meeting the needs of citizens, as are farmers, merchants or warriors (although Plato had reservations with poets). Nevertheless, they lack ``true'' wisdom and only possess the knowledge of a~producer, focused on the goal. It is worth noting that the philosophers included the sophists in this group of specialists in \textit{techne}. According to them, the sophists did not strive to attain \textit{sophia---}wisdom, contrary to their name. Instead, they used their skilful manipulation of language as a~tool to influence, shape, and convince their listeners of their own reasoning, just as a~craftsman skilfully shapes wood to obtain the desired piece of furniture. This is also the foundation of the sophists' teachings: refining the ability to use eloquence in a~competent manner, craftsman-like, rather than seeking the Truth and wisdom. Sophists were not truly ``sages'' but rather ``experts in craft''.



Nowadays, experts are widely respected. Dating back to the era of Saint-Simon, they have been viewed as the individuals who set goals for and direct global development. As it was already stated by Friedrich von Hayek 
%\label{ref:RNDE1fOlW6GAK}(1945, p.521),
\parencite*[][p.521]{hayek_use_1945}, %
 the kind of knowledge we ``expect to find in the possession of an authority made up of suitably chosen experts […] occupies now so prominent a~place in public imagination that we tend to forget that it is not the only kind that is relevant.'' This knowledge of the experts---``scientific knowledge''---is seen as an organized system that encompasses all knowledge and can help define development objectives. It should be noted, however, that this kind of knowledge is not one of the ``sages''. Rather it belongs to the ``savants'' who utilise their accumulated knowledge as a~tool to mould their surrounding reality, similar to how ancient sophists used words. The emergence of a~new \textit{techne} required new craftsmen and tools. As a~result, this kind of knowledge was enclosed within sets of parameters or indicators. The mainstream economy, with a~focus on promoting economic and social development, has embraced GDP per capita as the key indicator of progress. Later on, various alternative measures, including the Human Development Index (HDI) and the Genuine Progress Indicator (GPI), have emerged. However, while indicators can measure progress, they fail to address the fundamental questions: how to achieve balanced development and well-being. How to act? The social (and so---economic) sphere is treated as a~material to work on, ``design'', and the quality of this design is evaluated solely through established indicators. Those appear to be practical, but only in the sense of \textit{techne}, which focuses on a~goal, expressed through said indicators. Yet, while they aid in influencing societies and governments to reach established goals, they yet they do not provide any information regarding actual progress, values, consequences, nor on the human actions. The ``savant economy'' can be called a~``technocratic one'', which was defined by Howard Scott 
%\label{ref:RNDAiGwiOTBZ5}(1965, p.10)
\parencite*[][p.10]{scott_history_1965} %
 in the following words: ``Technocracy has proposed the design of almost every component of a~large scale social system.'' It is also a~knowledge of the experts but intended not to uncover the truth, but instead to manage the unpredictable reality within precise bounds of indicators that give the impression of command over the rapidly changing environment.



Therefore, we are still consequently stuck in the dichotomy between savants and sages. Aristotle, however, leaves a~caveat. While wisdom is the highest value, those who wish to engage with worldly matters, to immerse themselves in \textit{vita activa}, ought to pursue \textit{phronesis}---practical knowledge. Although this pursuit does not result in becoming a~philosopher, it can help one be a~good leader, ruler, or, in modern times, manager, without succumbing to mere ``technical'' or rather ``technocratic'' approach. This \textit{phronetic} knowledge pertains to interpersonal relationships and facilitates a~community's functioning, with the goal of ensuring a~``good life'' for the general public.



In this regard, the goals of ancient philosophers and the Knowledge-Based Economy share a~commonality. They both assume the establishment of a~community, whether political or economic, founded on knowledge. However, this knowledge is not to be understood in abstract, as wisdom or truth (finally, even Plato deemed this impossible to achieve in ``real'' world). It is also not merely a~``technique'' used to control the reality or to reach a~certain goal. Rather, it refers to knowledge that has practical applications and therefore enables peaceful coexistence and development. This particular type of knowledge is called \textit{phronesis} by Aristotle, and the individual who possesses it is called \textit{phronimos}.



As mentioned, Knowledge-Based Economy has mostly integrated the concept of \textit{phronesis} through the work of Ikujiro Nonaka, Ryoko Toyama and Toru Hirata, entitled \textit{Managing Flow. A~Process Theory of the Knowledge-Based Firm}. The Japanese researchers define \textit{phronesis} as a~type of tacit knowledge, ``the ability to grasp the essence of a~situation in process and take the action necessary to create change.'' 
%\label{ref:RNDYTLHA5e0YQ}(Nonaka et al., 2008, p.4).
\parencite[][p.4]{nonaka_managing_2008}. %
 It is a~unique attribute of leaders who strive to benefit the collective interests of the enterprise they manage. According to them, ``\textit{phronesis} synthesizes ``knowing why'' as in scientific theory, and ``knowing how'' as in practical skill, with ``knowing what'' as a~goal to be realized.'' 
%\label{ref:RNDsvtANIHNiA}(Nonaka et al., 2008, pp.14–15).
\parencite[][pp.14–15]{nonaka_managing_2008}. %
 This concept aligns with the economic definition of knowledge put forward by the OECD. The organization has introduced a~classification system that divides knowledge into four distinct categories:
\begin{enumerate}
\item know-what, descriptive-informational knowledge---this is a~normative knowledge based on experience, context and common sense; it refers to fundamental knowledge used in everyday functioning; its meaning is very close to information and it is easily communicated and passed on;

\item know-how, practical-technological knowledge---it refers to people's skills and capabilities and means the ability to do something; it is instrumental, contextual and related to experience;

\item know-why, exploratory-prognostic knowledge---this is universal and theoretical knowledge; it explains the principles and laws of nature and is closest to what we would call ``scientific knowledge'';

\item know-who, descriptive-informational knowledge---this knowledge mostly refers to information about social relationships, such as who knows whom and what they know. This type of knowledge is becoming increasingly important due to the growing level of specialisation and constant changes 
%\label{ref:RNDqJ8XSdtaFB}(OECD, 1996, p.12; Clarke, 2001, p.190).
\parencites[][p.12]{oecd_knowledge-based_1996}[][p.190]{clarke_knowledge_2001}.%
\end{enumerate}

Considering both the OECD classification of knowledge and Nonaka, Toyama and Hirata's definition of practical knowledge, we can observe that \textit{phronesis} appears to be a~kind of ``super-knowledge'' that combines elements of various knowledge types listed by the OECD. It encompasses both ``knowing why/how/what'' and so is not limited to the Aristotelian concept of the ability to ``calculate well'', but is akin to the all-encompassing ``full knowledge'' of the world that only a~``good manager'' can possess. Meanwhile, as indicated above, in Greek thought there already is an appropriate term for ``certain'' and ``full'' knowledge, namely wisdom (\textit{sofia}). \textit{Phronesis} is not so much knowledge \textit{per se}, but the ability to act. According to Nonaka, Toyama and Hirata 
%\label{ref:RNDRUclcghpN5}(2008, p.53),
\parencite*[][p.53]{nonaka_managing_2008}, %
 it involves ``the ability to determine and undertake the best action in a~specific situation to serve the common good''. Aristotle provides a~seemingly similar definition. In the \textit{Nicomachean Ethics}, he defines \textit{phronesis} as ability to ``deliberate well about what is good and advantageous for himself, not in some one department […] but what is advantageous as a~means to the good life in general'' and also ``rational quality, concerned with action in relation to things that are good and bad for human beings.'' Furthermore, he completes his definition with an example: the one deemed prudent was Pericles, since he was one of the men able to judge ``what things are good for themselves and for mankind'' 
%\label{ref:RNDkorQPLdq55}(Aristotle, 1934, VI.5).
\parencite[][]{rackham_nicomachean_1934}.%




There is a~fundamental difference between these definitions. Whereas Aristotle's definition focuses on the action itself (``to deliberate well'', ``action in relation to things''), later definitions refer to the effects of that action (``action […] to serve common good''). Moreover, in modern definitions \textit{phronimos} is the one who HAS \textit{phronesis}---possesses this ability or skill. In Aristotle, one is CONSIDERED to be a~\textit{phronimos}. Thus, for the Stagirite, the emphasis was on the relationship between the \textit{phronimos} and the community. It was the community, which, judging one's actions, could recognise them as the possessor of practical knowledge, and therefore---deem him a~\textit{phronimos}. It was not a~given quality, but one that depended on the judgement of others. This element of judgement firstly established the relationship between the leader and his followers as a~mutual one, and secondly, while allowing the leader to act for his own benefit, it also ensured concern for the benefit of others. However, in later times, \textit{phronesis} came to be identified with one of the many qualities that a~leader is entitled to, that he should acquire and possess as an attribute---another sceptre that he can show to his subjects (or subordinates) to gain their obedience. The Aristotelian \textit{phronesis} was shifted to either \textit{episteme} or \textit{techne}.



This first aspect, the identification of prudence (\textit{phronesis}) with knowledge (\textit{episteme}), is a~particular merit of Christian doctrine. As St Thomas Aquinas notes, Augustine ascribes to prudence ``the avoidance of ambushes'', thus associating it not only with knowledge but also with the most common colloquial understanding of it: the ability to avoid unnecessary risk. St Thomas himself emphasises the ``commanding'' aspect of prudence, since it establishes order and applies the previous judgement, thus restraining the will and ensuring one's proper conduct. It does not allow any action, but only the ``proper'' one---those who sin voluntarily do not possess prudence, since they lack the right reason. Prudence that is ``both true and perfect, […] commands aright in respect of the good end of man's whole life'' 
%\label{ref:RNDDyfEVRPnTI}(Thomas Aquinas, 1947, II-II, q.47 a.8,13).
\parencite[][q.47 a.8,13]{thomas_aquinas_summa_1947}. %
 An important implication follows from this---in Christian thought, the one who has prudence has knowledge of right conduct. Therefore, he does not need recognition from his subordinates; on the contrary, as in Plato's ideal state, they should give him unconditional obedience. St Thomas makes this argument directly in relation to political power---the best system would be a~monarchy, because one person is better able to govern, without having to consult with others and listen to their opinions. For the whole may not be as reasonable and wise as the chosen individual, especially since Aquinas's doctrine assumes that the earthly monarch is a~reflection of the one God, so that the community will be best if it comes as close as possible to the ideal of a~community subject to a~single, eminently wise ruler who most resembles God 
%\label{ref:RNDtHUyoNpZRj}(Thomas Aquinas, 1949, I.2-3).
\parencite[][]{thomas_aquinas_kingship_1949}. %
 The possibilities of opposing the will of this monarch, on the other hand, are relatively limited and concern the situation in which he goes against the word of God---de facto manifesting a~lack of \textit{episteme}, knowledge of higher matters and first premises.



On the other side of the spectrum \textit{phronesis} is placed by Niccolo Machiavelli. For the Florentine philosopher, it becomes identical to \textit{techne}. A~prudent leader is one who knows the secrets of the art of governing and is able to use them to achieve specific goals. In Machiavelli's political theory, the ultimate goal is to raise the state from decline, and so a~good leader needs to ``differentiate between the lion and the fox'' 
%\label{ref:RND7aL8bA6ZR9}(Machiavelli, 2003, p.96),
\parencite[][p.96]{machiavelli_prince_2003}, %
 and so possess a~certain knowledge---not of the highest premises, but a~knowledge of the craft. For a~change, he will not resemble a~Platonic philosopher, but an ancient sophist who, thanks to his knowledge of the art of eloquence, argumentation and rhetorical techniques, will be able to shape the audience to agree with his position and concede the point\footnote{It is worth noting that the Machiavellian prince first and foremost acts for the good of the state, to develop it or save it from decline, not just to pursue his own ends, no matter the consequences.}.



In both cases, the understanding of prudence differs from that proposed by Aristotle. First of all, it is directed towards an end---be it salvation or the survival of the state---rather than being an activity in itself. Moreover, it is treated as a~kind of virtue that only a~select few possess. The general public should submit to them and listen to them, accepting their wisdom, and if they do not do so---this only shows the stupidity of the general public, and does not undermine the virtue of the \textit{phronimos}. For Aristotle, however, it was precisely in the eyes of the public that the \textit{phronimos} had to prove himself. Notice the wording: Pericles ``is deemed'' \textit{phronimos}, about Thales people ``say'' he is not. Thus, the recognition of a~leader's prudence is something that depends on the community in which he functions---Pericles does not ``is'', but ``is recognised as'', by a~particular group, in specific situations. Moreover, his prudence is not a~fixed and unquestionable virtue. Thucydides, describing the activities of Pericles in the \textit{Peloponnesian War}, in addition to emphasising his merits, also repeatedly refers to the criticism or opposition of the citizens of Athens, who constantly comment on, praise or blame the actions of their leader 
%\label{ref:RNDYsmWDTByye}(Thucydides, 2009, II.21).
\parencite[][]{hammond_peloponnesian_2009}. %
 Their opinion is not always correct, but it is what positions Pericles in relation to the community. He is aware that his actions are being watched and evaluated. His leadership role also depends on this assessment---he can be re-elected or removed. Pericles does not act from the height of infallible authority, nor is he a~simple manipulator. He strives to ensure that his actions benefit Athens as well as himself, because the interests of the community are not separable from the interests of the individual. A~good leader is one who cares about the group he leads, be it a~state, an organisation or a~company, but at the same time expects (and has the right to expect) certain benefits for himself---respect and recognition, another term of office, remuneration.



Contemporary conceptions, on the other hand, emphasise mostly the aspect of looking only after the good of the community. Machiavelli's image of the leader has taken on a~negative connotation, in which the leader is concerned only with himself and pursues only his own interests, using the community for this purpose\footnote{This is, as we have said, a~fundamental distortion of Machiavelli's concept, but because of the different leitmotif of this text, we do not attempt to rehabilitate the Florentine's theory here.}. A~good leader should therefore become someone close to the image presented by Plato or Aquinas. In both Nonaka, Toyama and Hirata's theory and in the quoted text by Rooney and MacKenna, \textit{phronesis} is something that should lead to the common good, while the interests of the individual are overlooked or seen as merely an additional outcome of concern for the whole. Moreover, \textit{phronesis} actually becomes a~tool for transforming knowledge into wisdom 
%\label{ref:RNDLUEbgHqXo4}(Nonaka et al., 2008, p.67).
\parencite[][p.67]{nonaka_managing_2008}. %
 As stated by Germán Scalzo and Guillermo Fariñas 
%\label{ref:RNDAPHuyaU7iA}(2018, p.30):
\parencite*[][p.30]{scalzo_aristotelian_2018}: %
 ``an original interest in knowledge, with the idea of \textit{phronesis}, clearly evolved into a~more ambitious purpose: wisdom''.



It may appear that the concept of ``dispersed knowledge'' proposed by F. von Hayek 
%\label{ref:RNDIyXOLbhV7v}(1945)
\parencite*[][]{hayek_use_1945} %
 is closest to the original meaning of phronesis. He strongly emphasized that no one has complete and perfect knowledge---there are no Platonic sages. Furthermore, knowledge itself never exists in a~concentrated form but rather is scattered, with multiple individuals possessing bits and pieces of it. In contrast to the aforementioned ``experts' knowledge'', Hayek acknowledged that individuals' ``dispersed knowledge'' is frequently marginalised. Meanwhile, this type of knowledge relates to specific temporal and spatial circumstances and therefore requires (and promotes) quick adaptation to new circumstances. As knowledge is not evenly spread, those who hold the presently relevant portion of ``dispersed knowledge'' are best equipped to make informed decisions, based on the possessed premises.



It would seem that this is the knowledge of \textit{phronimos}, who is able to consider the context of a~situation and its various possible developments, adapting and modifying the undertaken actions accordingly to effectively achieve their goals in given circumstances. This individual does not need to be aware of all circumstances or their consequences, but should be capable of adjusting their behaviour as necessary in response to the situation. However, a~crucial difference exists that prevents us from classifying Hayek's possessor of ``dispersed knowledge'' as a~\textit{phronimos}, and that is the postulated lack of deliberation. Hayek, to affirm his point, quotes A. Whitehead: ``Civilization advances by extending the number of important operations which we can perform without thinking about them'' 
%\label{ref:RNDChwhy8Zp7N}(Hayek, 1945, p.528).
\parencite[][p.528]{hayek_use_1945}. %
 Hayek's man operates intuitively, activating the knowledge he possesses subconsciously. However, this is not true in the case of \textit{phronimos,} who not only thinks, but thinks well and thoroughly. Aristotle defined humans as being that are not only the \textit{zoon politikon}, but also possess the ``rational principle'' 
%\label{ref:RNDFqbpXPH2vU}(Aristotle, 1934, I.13)
\parencite[][]{rackham_nicomachean_1934} %
 which distinguishes them from other animals. We have the ability not just to think, but to think rationally. Relying solely on intuition when interpreting ``formulas, symbols, and rules whose meaning we do not understand'', as Hayek 
%\label{ref:RND8KqGZz1V43}(1945, p.528)
\parencite*[][p.528]{hayek_use_1945} %
 stated, limits our knowledge solely to \textit{nous}---Aristotelian ``intuitive thinking''. And \textit{nous} is only the initial phase of acquiring knowledge. \textit{Phronimos}, besides \textit{nous}, must also possess the knowledge of time and place, which cannot be merely gained through intuition. These are indeed the elements of ``dispersed knowledge'', but to undertake successful action, these circumstances must be acknowledged and analysed. This notion is present in Hayek's considerations, despite his subsequent affirmation of intuitive thinking. In order to plan and act, one must use, exchange, and put one's knowledge into action. Hayek provides an example: ``All that the users of tin need to know is that some of the tin they used to consume is now more profitably employed elsewhere and that, in consequence, they must economize tin.'' 
%\label{ref:RNDgIdHHe1DHi}(Hayek, 1945, p.526).
\parencite[][p.526]{hayek_use_1945}. %
 It is not essential for them to possess a~complete understanding of all the circumstances that have contributed to this situation, nor is it necessary for them to gain more knowledge. However, they must draw on their existing, ``dispersed'' knowledge to take action that would be most beneficial to them, and so act consciously rather than intuitively.



The action itself is the core of Aristotelian concept\footnote{Not gaining the full knowledge (which would mean gaining episteme and so becoming a~sage) nor reaching some goal or level of indicator---that falls into the realm of techne.}. To better illustrate this aspect, let us return to Aristotle's distinction between \textit{chrematistiké} and \textit{oikonomiké}. The former is the pursuit of the accumulation of goods, the latter the use of goods. An important aspect of ``use'' is a~certain possibility of its evaluation---one can use something well or badly, more or less carefully, achieving the intended goal or not. However, our predictions and expectations may be wrong or not accurate enough, we may lack certain information for various reasons, or we may succumb to bad advice. Following Hayek's example: we economised tin, only for the sudden demand for it to stop abruptly. As a~result, we were left with loads of now useless and worthless tin. According to Aristotle, those who have practical knowledge are able to analyse all the conditions they know of in order to take what they consider to be the best action. Nevertheless, its effect remains uncertain. Moreover, this action is judged \textit{post factum}, and in the case of the leader or manager, not only by him, but also by the whole community, which then will be able to give him (or take away) the title of \textit{phronimos}. Mere \textit{chrematistiké} (economization or accumulation of goods, e.g. tin) is not enough to obtain it. \textit{Phronesis} is the ability to adapt to changing situations, based on the dispersed knowledge possessed, but not absolute and infallible because it is about what can be different.



It seems somewhat ironic that in theoretical approach, Knowledge-Based Economy is more about \textit{chrematistiké} than \textit{oikonomiké}, since the emphasis is on acquiring, deepening and developing knowledge; of course, knowledge that can be used practically, but the latter aspect arouses much less interest. There seems to be an unspoken assumption that someone who acquires this knowledge (and we mean the various types of knowledge mentioned above) will also know how to use it correctly. This knowledge should come from experience, but since our leader, after accumulating \textit{chrematistiké}, always acted properly---for the common good, it is virtually impossible to point out when and where they could have gained such experience. Meanwhile, the ancient concept leads to a~disturbing implication---even with theoretical knowledge and experience, our actions are always uncertain and subject to evaluation. Conditions, people, premises change, purely subjective or emotional factors come to the fore. \textit{Phronesis}, then, is not so much the ability to act effectively towards a~specific goal, as it is the ability to take the risk of action---action that the \textit{phronimos}, on the basis of their knowledge, believes to be effective, but the effect of which is not a~foregone conclusion.



\textit{Phronesis} is thus the most social of all the dispositions that Aristotle writes about, and it can only be realised within a~community, be it a~state, a~company, or any other kind of society. It is shaped not so much by experience as by relationships and constant evaluation, as the case of Pericles shows: the Athenian made a~series of decisions that were evaluated both positively and negatively by the citizens, and he was able to adapt his behaviour to the situation, not only because of the influence of external factors (e.g. by changing his strategy), but precisely because of opinion. He was able to negotiate and persuade to such an extent that he ``was recognized'' as a~\textit{phronimos}. An experienced \textit{phronimos} evaluates and draws conclusions from it not only from the perspective of the results achieved, but also taking into the way his behaviour is evaluated by others, while this evaluation (feedback) should not so much set new or different goals for him, but show the possibility of change at the level of behaviour. Moreover, the Aristotelian \textit{phronimos} is not obliged to act altruistically only for the benefit of the community. Nowadays, it is the sphere of the ``common good'' that is most emphasized. Nonaka, Toyama and Hirata see profit as a~side effect, resulting from pursuing the standards of excellence, rather than ultimate goal. The aim is to produce an almost infallible leader who will always make the right decisions, of course, ``right'' in the sense of ``virtuous''. This is the result of the Thomistic transformation of \textit{phronesis}, which became another of the virtues. Therefore, the prudence expected of a~leader is prudence understood as caution in activities, impartiality, virtue, and action for the benefit of society. Aristotle, on the other hand, allows \textit{phronimos} to concentrate also on what is good for himself.



\textit{Phronesis} does not require the sacrifice of one's own interests on the altar of the common good, but precisely the kind of reflection that makes it possible to achieve both the particular interests of the individual (whether they be benefits or, for example, recognition and respect) and the interests of the whole community (the benefits achieved will affect the whole company, a~good manager will lead to greater trust on the part of contractors, etc.). It can therefore be understood as making the ``right'' decisions and actions, but these are not just virtuous ones, they are also beneficial to the person who makes them, to the community, and to whom it is dedicated. It is by no means strictly utilitarian, though it is not strictly virtuous either. Moreover, it is emphasized that the leader should act for the common good, without trying to think about how to do it (since he somehow already has the knowledge of how to do it), focusing on the fact that they should strive for development, success, improvement in quality and what is good for everyone. Here, too, the focus is on the objectives to be achieved, rather than on the value of the action itself.



Returning to Nonaka, Toyama and Hirata, the authors illustrate their concept with a~vivid comparison to constructing a~car. ``If \textit{techne} is the knowledge of how to make a~car well, \textit{phronesis} is the knowledge of what a~‘good' car is (value judgment) and how to build such a~car (realize the value judgment).'' 
%\label{ref:RNDrL5MveplTy}(Nonaka et al., 2008, p.54).
\parencite[][p.54]{nonaka_managing_2008}. %
 However, with reference to Aristotle's definitions, we would consider it more reasonable to combine \textit{techne} with the knowledge of ``how to make a~car'', and \textit{phronesis} is not so much the knowledge what a~``good'' car is, since this aspect fits more with \textit{episteme}, the scientific knowledge of things. The authors suggest that the \textit{episteme} cannot answer the question of what is a~``good'' car, because the question is subjective. No doubt, but if our understanding of a~``good car'' completely subjectively, what about the concept of ``common good'', that a~prudent manager should pursue? And since the ``common good'' is presented as a~kind of superior one, in order to maintain consistency, a~truly ``good'' car should also refer to some superior values, and thus try to match as many elements of the ``ideal'' car as possible. Therefore, either the ``common good'' (and, consequently, a~``good car'') is subjective and thus depends on the will of the manager, or it has to fulfil additional premises. From the previous arguments, we can conclude that it is the latter, since the ``common good'' common good of the company means pursuing the interests of employees, shareholders or customers, and so many different and sometimes conflicting ones.



The difference here lies both in what can be judged as a~``good life'' or ``common good'' and in the actions of a~leader. For Aristotle, the \textit{phronimos} acts to achieve the ``good life'', which is defined rather vaguely as self-sufficiency. \textit{Phronimos} is not expected to achieve the ``perfect life'', because that is impossible---it would be achievable in Plato's world of ideas. \textit{Phronimos} has to act to make the normal earthly life as good as possible. To use the car analogy, is a~``good car'' a~safe car in the sense that it guarantees survival in the event of an accident, or should it prevent injury or even be automated enough to avoid accidents? For Aristotle, each of these goals is important, but what matters is what the designer or builder actually does. If he wants only and at all costs a~car that will never allow an accident---which is the realisation of the ``highest'' common good, namely safety---and for this reason does not take any measures to i.e. increase the chances of survival during an accident, he will not be deemed a~\textit{phronimos}, although he will strive to achieve a~good cause. In this case he will become like the philosopher Thales, who, dealing with the affairs of the universe, did not notice the well on his way.



This example is cited by both Plato and Aristotle. Plato gives the following anecdote: ``While he [Thales] was studying the stars and looking upwards, he fell into a~pit, and a~neat, witty Thracian servant girl jeered at him, they say, because he was so eager to know the things in the sky that he could not see what was there before him at his very feet.'' 
%\label{ref:RNDqdQRWiqqrJ}(Plato, 1921, 174a).
\parencite[][174a]{plato_plato_1921}. %
 Aristotle, referring to this anecdote, claims that people like Thales can be attributed theoretical wisdom (\textit{episteme} and \textit{sophia})\footnote{For Plato, this particular anecdote is also the story of all the philosophers who study fundamental and universal things. They are like a~wise-man who, blinded by the light, returns to the cave to share his knowledge with the rest of the people there, but since his eyes are no longer accustomed to the darkness, he is unable to move smoothly in it and thus exposes himself to ridicule.} but not practical knowledge (\textit{phronesis}), because ``these sages do not seek to know the things that are good for human beings.'' 
%\label{ref:RNDFkMVLEhROp}(Aristotle, 1934, VI.7.5).
\parencite[][]{rackham_nicomachean_1934}. %
 However the same Thales in \textit{Politics} displays some practical knowledge. When his fellow citizens reproached him for the uselessness of philosophy, Thales, on the basis of his knowledge of astrology, predicted an extraordinarily rich olive harvest for the coming year. Then he rented out all the olive presses in advance for a~pittance. When his theory was confirmed and the harvest was indeed bountiful, everyone had to turn to him for the use of the presses, and then Thales---as the current monopolist---could set any rental price. In this way he made a~fortune, but the purpose of his activities was not to get rich, but to show that ``it is easy for philosophers to be rich if they choose, but this is not what they care about'' 
%\label{ref:RNDDW1TNkf2r9}(Aristotle, 1944, 1.1259a).
\parencite[][1.1259a]{aristotle_politics_1944}.%




An attempt to reconcile these two images of Thales, the sage and the \textit{phronimos}, leads to a~simple conclusion---a true philosopher has both theoretical and practical knowledge, and is able to forge one into the other. At the same time, \textit{sophia} is more important to him, and therefore he often does not do what is useful to him (for example, live in poverty or endure the ridicule of his fellow citizens), because above all he wants to finally achieve wisdom. This does not mean, however, that he could not act and use his knowledge if he wanted to. The problem, from a~practical point of view, is that it he does not want to. Paradoxically, full knowledge encourages neither action nor risk. Thales, who bought olive presses, did not turn out to be a~good manager and \textit{phronimos}---his behaviour did not bring much benefit to the community, unless we consider as such a~greater respect for philosophy.



Similarly, our creator of the car may be a~brilliant inventor, a~sage, but has no prudence, because the knowledge he accumulates is not applicable. Not only does it not benefit society (e.g. by slightly increasing safety), but it also does not benefit the owner himself, who, locked in his studio, leads a~kind of \textit{vita} \textit{contemplativa}, searching for the final, ideal solution. Meanwhile, the Aristotle \textit{phronetic} leader uses what he has gathered (\textit{chrematistiké}) to act on the accumulated goods---knowledge, experience, knowledge of the craft. He acts with the awareness that his action is subject to the risk of lack of success, but at the same time, basing on his knowledge, he considers it good and beneficial. This is because only action allows him to verify this knowledge. The postulate of achieving a~more ambitious goal---wisdom---and making Knowledge-Based Economy a~Sage, as in Rooney and McKenna's text, paradoxically leads to the inhibition of its development. For a~true sage, having perceived the whole truth, does not feel the need to interact, to engage in human, less important matters, because ``human affairs do not deserve to be given great importance'', as Plato 
%\label{ref:RNDPFfnbqlyWg}(Plato, 1967b, 803b)
\parencite[][803b]{plato_plato_1967-1} %
 wrote. And as Hannah Arendt 
%\label{ref:RNDBcZPvH1J3X}(2005, p.32)
\parencite*[][p.32]{arendt_promise_2005} %
 notes, the philosopher devotes himself entirely to the \textit{vita contemplativa}. He participates in community life solely because that community may be an obstacle to his complete engagement in philosophy.



The true \textit{phronimos} is the one who predicts---but does not know. Nonetheless, he is willing to take some risks. He assumes some possibilities of development, but he takes the risk that his decision is flawed. He introduces a~fresh invention on the assumption that it will be successful---but people accustomed to the old methods may decline to employ it. However, progress in development and the accumulation of new knowledge are only possible due to uncertain activities that bear the risk of error.



\section{Conclusions}

It should be noted that modern research often dismisses the significance of ancient ideas or interprets aspects of the ancient world using contemporary terminology. Scott Mielke stresses that ``The ‘modernist' view is that the ancient economy is to be understood as an early restricted version of what we are familiar with today'' 
%\label{ref:RNDfgHrtKZO66}(Meikle, 1995, p.2).
\parencite[][p.2]{meikle_aristotles_1995}. %
 Hence, there is emerging criticism regarding the relevance of Aristotle's theories in contemporary research, as the Stagirite addressed a~significantly different economy. On the other hand, there is an attempt to adapt past phenomena and events to modern schemes. That unfortunately results in the loss of historical context. Ancient situations or myths are described without reference to their contemporary background, which included different values and concepts (like the role of fate, concept of justice, and punishment)\footnote{A~good example can be also found in Mielke, where there is an attempt to describe Prometheus' trick at Mecone as ``an example of a~pure isolated distribution where two parties meet on an equal footing and negotiate the division of a~joint asset'' 
%\label{ref:RND5o3ixQ3XwU}(Meikle, 1995, p.178).
\parencite[][p.178]{meikle_aristotles_1995}. %
 While this may be adequate in economic terms, it fails to present the complexity behind the myth and, more importantly, does not address its main purpose. The myth was meant to explain sacrificial customs as well as the reason why mankind is plagued by troubles, illnesses and sorrow. What is worth noting in this context is that Zeus and Prometheus were certainly not ``on an equal footing'' and ``negotiating'', as it is clear that one party (Prometheus), aware of the other's (Zeus) superiority, attempted to cheat in order to reach the desired outcome. Additionally, in one of the earliest descriptions of the myth, Hesiod suggests that Zeus was aware of the deception, but gave in to it, since Fate demanded so.}.



Therefore, our aim was not to reinterpret ancient theories in contemporary terminology, nor to shoehorn modern theories into the ancient conceptual framework. Rather, by drawing on the wealth of philosophical ideas, our objective was to highlight the potential relevance of the ancient Greek notion of ``knowledge'' and its associated elements in present-day analyses. Knowledge played a~crucial role in ancient thinking, regarded both as an intrinsic value and a~means to attain virtue. It served as the foundation for many aspects of life, including political, cultural, and economic spheres. And Greeks understood quite well the different types of this knowledge, including not only \textit{episteme}, (pure knowledge) and \textit{techne} (knowledge of the craft) but also knowledge of human relations, that influences the community in which we live and work - the \textit{phronetic} one.



Considering the volatile nature of the modern world, including the rapidly changing social and economic relations, we believe that the concept of phronesis remains relevant in updating the prevailing perception of the Knowledge-Based Economy and contemporary management theories.



Above all, we advocate for the prioritisation of the acquisition and application of knowledge (\textit{oikonomiké}) over its mere accumulation and possession (\textit{chrematistiké}) as the fundamental principle of the Knowledge-Based Economy. In numerous instances, attempts to characterise Knowledge-Based Economy focus on the stage where knowledge is already possessed or assume that its acquisition occurs during the learning process, therefore seeking to streamline this process by minimizing errors, introducing indicators and forming recommendations to enable the largest number of people to acquire knowledge. Unfortunately, in this manner, we only elevate the level of \textit{chrematistiké} and delve deeper into the ``savant economy'', quantifying our attained knowledge through grades, diplomas, or certificates, without due consideration of how to apply it. This aspect, the significant role of education in the Knowledge-Based Economy was highlighted by the International Commission on Education for the 21\textsuperscript{st} Century chaired by Jacques Delors and by Benjamin R. Barber, referring to the infantilisation of knowledge and education\footnote{Barber explains this phenomenon by referencing three dichotomous pairs of concepts: the dominance of ``easy over hard'', ``simple over complex'', and ``fast over slow'' 
%\label{ref:RNDwOmI94reKx}(Barber, 2008, pp.85–107).
\parencite[][pp.85–107]{barber_consumed_2008}.%
}. In the midst of these complex issues, it may be worthwhile to follow Hayek's advice and perceive the idea of knowledge as ``dispersed'', while preserving the Aristotelian elements of risk and action, which we deem particularly valuable. Progress can be achieved not by attaining higher levels of indicators, but by equipping future leaders\footnote{Managers, business leaders, political ones, etc.} in various social fields with competencies that empower them to apply their knowledge while being mindful of potential risks. This requires acting with due consideration and not only as a~leader, but also as a~team member because \textit{phronesis} can only be achieved through communal relationships. As Aristotle previously explained, it is necessary to possess a~certain level of adaptability in a~constantly changing reality. Rather than having complete control through certainty and expertise (\textit{episteme}), both the ability to think and act are required, accepting the possibility of failure and receiving criticism from others involved in the interaction. \textit{Phronetic} knowledge is not an unequivocal or definitive knowledge, given once and for all. It evolves, adapts and moulds itself to suit the various types and requirements of human societies. Therefore, the endeavour to assign the ``knowledge'' only to ``sages'', as in the question posed in the introduction, automatically reduces its complexity. As Nonaka, Toyama and Hirata 
%\label{ref:RNDzCMnFBtgWi}(2008, p.242)
\parencite*[][p.242]{nonaka_managing_2008} %
 accurately note, ``knowledge is created by human beings in relationships, knowledge-based theory of the firm has to broaden its perspective from the static, atomistic, substance-based worldview typical of conventional economic theory, to a~view of the firm as a~dynamic entity in flow.''



The competences included within \textit{phronesis} might lead to very (\textit{nomen omen}) practical recommendations. The concept in its original, Aristotelian meaning appears worthwhile for implementation, particularly in management theories, since \textit{phronesis} pertains to an individual's knowledge expressed through action. Hence, it can only be realized in situations that necessitate interactions among diverse actors and not merely in the theoretical sphere. Thus, our aim should not be to create a~know-it-all Platonic philosopher, but rather an Aristotelian \textit{phronetic} leader who is willing to take action, make errors, and receive feedback from others. Who is not afraid to act or make difficult but deliberate decisions that influence the whole society, with their (and his own) best interests in mind. This requires focusing not only on the desired qualities of the manager in terms of their character and skills but also providing them with tools from both \textit{techne} and \textit{episteme}---abilities related to managing stress, decision-making, holding challenging conversations or negotiations, thinking creatively or out-of-the-box, and mentoring. Those could assist in educating a~conscious, mindful individual, able to use their particular, individual knowledge to operate and interact within the dynamic domain of social relationships in a~manner that would benefit both themselves and the surrounding community.



\end{artengenv2auth}



\setcounter{secnumdepth}{0}





\begin{document}

Igor Wysocki\footnote{Igor Wysocki, M.A. is a~doctoral student at The Interdisciplinary Doctoral School of Social Sciences, Nicolaus Copernicus University in Toruń, Poland. He is a~fan of jazz and chess aficionado. }, M.A.



Interdisciplinary Doctoral School of Social Sciences



Nicolaus Copernicus University in Toruń, Poland



ORCID: 0000-0002-4926-4010



igorwysocki82@wp.pl



Institutional e-mail address: wysocki@doktorant.umk.pl



phone: 48~503~476~136



Łukasz Dominiak\footnote{Dr. habil. Łukasz Dominiak is an Associate Professor at the Department of Social Philosophy, Institute of Philosophy, Faculty of Philosophy and Social Sciences, Nicolaus Copernicus University in Toruń, Poland and a~Fellow of the Ludwig von Mises Institute, Auburn, Alabama, United States.}, Ph.D.



Department of Social Philosophy



Faculty of Philosophy and Social Sciences



Nicolaus Copernicus University in Toruń, Poland



ORCID: 0000-0001-6192-8468



e-mail: lukasdominiak80@gmail.com, cogito1@umk.pl



phone: 48~533~572~619



Social Welfare, Interventionism, and Indeterminacy:





\textbf{In Defense of Rothbard}\footnotetext{ This research was funded in whole or in part by the National Science Centre, Poland, grant number 2020/39/B/HS5/00610. For the purpose of Open Access, the author has applied a~CC-BY public copyright licence to any Author Accepted Manuscript (AAM) version arising from this submission. }







Abstract



The present paper argues that Rothbard's economic case against the state is more robust than suggested by his critics. The charge that it might be anemic is based on the suggestion that we can say literally nothing about the way governmental acts bear on social utility. Contra this supposition we submit that Rothbard's critics missed the fact that the effects of governmental interventions might be actually indeterminate in two ways: weakly or strongly. If the indeterminacy involved in his welfare theory is weak, then his economic criticism of the state is more robust than envisaged by these authors. To the effect that this indeterminacy is indeed weak we advance the following reasons: Rothbard's understanding of the Unanimity Rule; the avoidance of the contradiction allegedly committed by Rothbard over one and the same page of his famous essay; his economic criticism of interventionism being better aligned with his overall ethical anti-governmental stance; the principle of charitable reading, which cuts across all of the previously stated reasons. If our arguments count for something, then we are warranted in claiming that Rothbard is indeed able to say something about social utility under interventionism. And if so, then his criticism of interventionism should be viewed as robust rather than anemic.



\textbf{Keywords:} indeterminacy, interventionism, social welfare, Rothbard, welfare



\section{Introduction}

Bryan Caplan 
%\label{ref:RNDKcMbzp7U9L}(1999, p.834)
\parencite*[][p.834]{caplan_austrian_1999} %
 claims that Murray Rothbard's welfare theory provides only a~weak basis for the criticism of governmental interventions. Specifically, Caplan argues that what Rothbard can at most establish is that these interventions have indeterminate effects on social utility. It is true, as demonstrated by Joseph Salerno 
%\label{ref:RNDwyjoeKjIF6}(1993, p.131),
\parencite*[][p.131]{salerno_mises_1993}, %
 that Rothbard does not show that governmental interventions decrease social welfare and so this fact might have prompted Caplan to make the said charge. However, there are still two possible sorts of indeterminacies left to consider, given Rothbard's anti-governmental stance. For the impact of governmental interventions on social welfare might be indeterminate in a~strong or a~weak sense. In the strong sense, we cannot say whether these interventions increase, decrease or leave social utility unaffected. By contrast, in the weak sense, we cannot say only whether they decrease or leave social utility as it was although what we can say is that they never increase it. Now if Rothbard's criticism of governmental intervention were to involve the strong indeterminacy, then it would indeed be anemic. If, on the other hand, the indeterminacy appealed to in his welfare theory were to be weak, then his criticism of the government would be much more radical than suggested by Caplan.



In the present paper we argue that the Rothbardian welfare economics\footnote{An anonymous reviewer rightly noted that it is not very clear whether in this paper we defend Rothbard himself or his welfare theory. What we can offer as a~reply is that this paper is meant to be primarily theoretical (even if interpretive at times). Therefore, its main focus is to defend Rothbard's welfare \textit{theory} rather than its author. However, by defending the theory, we, \textit{nolens volens}, defend its author. Given this, irrespective of whether we speak of ``the Rothbardian welfare theory'' or indeed of ``Rothbard's welfare theory'', it is \textit{always} the theory itself that we intend to defend. } should be interpreted as claiming that the effects governmental interventions have on social welfare are indeterminate\footnote{A~compelling case can be made that according to Rothbard's welfare economics it does not make sense to call effects of state interventions ‘indeterminate' to start with. Besides the fact that Rothbard himself does not call them ‘indeterminate', the idea that they could be indeterminate presupposes that interpersonal comparisons of utility can be made, although the result of such comparisons is, sometimes, indeterminate. However, Rothbard rejected the very possibility of making such comparisons. It is therefore better to say that Rothbard's conclusion that state interventions cannot increase social utility simply and trivially follows from his premise---afforded by his doctrine of demonstrated preferences---that interpersonal comparisons of utility are impossible than to say that the effects of state interventions are indeterminate. Nevertheless, Rothbard's critics base their argument on the concept of indeterminacy. Thus, our ambition in this paper is to meet them on their own grounds and show that even if one accepts their problematic conceptual framework, Rothbard still comes out victorious.} only in the weak sense, that is, that they can never increase it and that the only indeterminacy they involve reduces to whether they decrease or leave social utility unaffected. Hence, we believe that Rothbard's critique of governmental interventions should be viewed as much more radical than Caplan contends. We posit that unless we construed the concept of indeterminacy in the weak way, we would have to conclude that Rothbard contradicts himself over one and the same page of his paper \textit{Toward a~Reconstruction of Utility and Welfare Economics}, which is a~highly unlikely diagnosis and an extremely uncharitable thing to say. On the other hand, once we interpret the indeterminacy involved as the weak one, no contradiction ensues and the Rothbardian welfare theory is then unproblematically coherent. Moreover, this interpretation tallies better with both what we argue is the proper Rothbardian understanding of Pareto-Superiority and with his overall anti-governmental anarcho-capitalist stance\footnote{Rothbard's commitment to anarcho-capitalism is probably most plainly laid down in Rothbard 
%\label{ref:RNDIHdWgif7zQ}(2006 [1973]; 2009 [1970]; 2002 [1982]).
\parencites*[][]{rothbard_for_2006}[][]{rothbard_man_2009}[][]{rothbard_ethics_2002}. %
 For an excellent exposition of the Rothbardian \textit{moral} argument for the free market, see also Juruś 
%\label{ref:RNDP4gfdk63tg}(2012).
\parencite*[][]{jurus_w_2012}. %
 } and therefore with the broader Austro-libertarian framework adopted by this author.\footnote{One of the anonymous referees wondered why it is all important to revisit the debate over the Rothardian welfare economics. First of all, we believe that we (at least to some extent) contribute to showing that the free market---as opposed to governmental interventions---bears positively on social utility not only \textit{ex ante} but also \textit{ex post}. Granted, for libertarians, the defense of the free market is primarily of \textit{moral} nature. However, as acknowledged by Hausman and McPherson 
%\label{ref:RNDwO9MjrW0yK}(2006, p.172),
\parencite*[][p.172]{hausman_economic_2006}, %
 ``libertarians would like it to be the case that protecting freedom also makes people better off.'' After all, it is precisely the task of providing a~purely \textit{economic} argument in favor of the free market regime that Rothbard set himself in his paper \textit{Toward a~Reconstruction of Utility and Welfare Economics}. And we believe that our paper to some degree fills in the lacuna between the free market (understood as a~totality of \textit{rights}{-respecting exchanges) and its beneficial }\textit{economic} consequences. Second, we submit that the present paper also sheds more light on the Paretian Unanimity Rule, not only a~central tenet of the Austrian welfare economics in its Rothbardian version but also an important device adopted in mainstream economics. }



The present paper proceeds in the following fashion. Section 2 introduces the distinction between weak and strong indeterminacy, in terms of which the Rothbardian conception of the impact of governmental interventions on social utility can be analyzed. Section 3 illuminates the relation between the kind of indeterminacy and the strength of his economic criticism of the state. Section 4 argues that the weakly indeterminate character of state interventions into economy follows as a~corollary from Rothbard's commitment to the Paretian Unanimity Rule. Section 5 addresses the challenge levelled at the Rothbardian welfare theory to the effect that he contradicts himself in his assessment of the effects of governmental interventions upon social utility. Section 6 undertakes the problem of coherence of Rothbard's overall theoretical system under alternative interpretations of indeterminacy. Section 7 concludes.



\section{Strong \textit{vs} Weak Indeterminacy}

It is incontrovertible that Rothbard does not say that governmental interventions necessarily decrease social utility. As he himself points out 
%\label{ref:RNDxs4sSCJa1z}(Rothbard, 1976, p.100),
\parencite[][p.100]{rothbard_praxeology_1976}, %
 ``we cannot say that any action of the State \textit{decreases} social utility.'' This fact is further confirmed by Salerno 
%\label{ref:RND777DlroeQv}(1993, p.131),
\parencite*[][p.131]{salerno_mises_1993}, %
 who says that, contrary to his own ``more radical conclusion'' which is indeed ``able to completely discount any gains, in terms of direct utility or exchangeable goods, that accrue to the interveners and their beneficiaries,'' what Rothbard ``has ably demonstrated on purely scientific grounds'' was only that governmental interventions never ``increase social welfare.'' 
%\label{ref:RNDPAKoLkrqJz}(Salerno, 1993, p.131)
\parencite[][p.131]{salerno_mises_1993} %
 This is also acknowledged by Caplan 
%\label{ref:RNDS1VrnMwyxj}(1999, p.833)
\parencite*[][p.833]{caplan_austrian_1999} %
 saying that Salerno's argument to the effect that the government does reduce social welfare is ``stronger than Rothbard's.'' Likewise, Kvasnička's 
%\label{ref:RND2kuyp0Q2YA}(2008, p.49)
\parencite*[][p.49]{kvasnicka_rothbards_2008} %
 criticism of Herbener 
%\label{ref:RND8zCYNKLPxm}(1997, pp.103–104)
\parencite*[][pp.103–104]{herbener_pareto_1997} %
 allegedly getting it wrong that ``involuntary interaction [is] ‘Pareto-Inferior''' implies that ``Rothbard says it correctly'' when he submits that ``it is only indeterminate.'' 
%\label{ref:RND4BySAMMWx8}(Kvasnička, 2008, p.49)
\parencite[][p.49]{kvasnicka_rothbards_2008}%




It seems that the fact that Rothbard does not claim that state interventions necessarily decrease social utility prompted some of the above authors to make a~charge against Rothbard that his economic criticism of state interventions is anemic. Most notably, Caplan 
%\label{ref:RNDgEXNTiPuEC}(1999, p.834)
\parencite*[][p.834]{caplan_austrian_1999} %
 argued that:



Rothbard could only claim the welfare effect of government intervention upon social utility is indeterminate. This is an important point because it shows that Rothbard's welfare economics provides a~much weaker defense of laissez-faire than usually assumed. In particular, Rothbard's own theory strips him of the ability to call any act of government inefficient. By denying others the ability to endorse state action in the name of efficiency, Rothbard also implicitly denies his own ability to reject state action in the name of efficiency. His welfare criterion justifies agnosticism about---not denial of---the benefits of state.



There are other authors making a~similar point. For instance, Kvasnička 
%\label{ref:RNDhdryjHewnm}(2008, p.49)
\parencite*[][p.49]{kvasnicka_rothbards_2008} %
 concurs with Caplan to the effect that ``even if Rothbard's welfare theory was correct (which it is not), it would be a~very weak basis for a~critique of governmental meddling with the economy'' because governmental interventions, as any involuntary interactions, instead of being Pareto-Inferior are ``only indeterminate.'' Moreover, Prychitko 
%\label{ref:RNDaaIZjicZr8}(1993, p.576)
\parencite*[][p.576]{prychitko_formalism_1993} %
 maintains that, according to Rothbard, ``we must remain agnostic: we simply don't know'' what the effects of state interventions are. All these charges find some additional support in Rothbard 
%\label{ref:RNDmLfvHOPtXI}(2008, p.252)
\parencite*[][p.252]{rothbard_toward_2008} %
 himself saying that ``[a]s economists, we can therefore say nothing about social utility in this case, since some individuals have demonstrably gained and some demonstrably lost in utility from the governmental action.''



There are, however, two ways in which the effects of governmental interventions on social utility can be indeterminate. The first way in which they might be indeterminate is that we cannot say whether social utility decreases, increases or is left unaffected by governmental interventions. This sort of indeterminacy we label \textit{strong indeterminacy}. Note that if the impact of governmental interventions on social utility were strongly indeterminate, Rothbard would be right saying that ``we cannot say that any action of the State \textit{decreases} social utility.'' Indeed, we would not be able to say that because we would not be able to say anything, that is, whether these interventions increase, decrease or leave social welfare unaffected.



Now the second way in which governmental interventions might have indeterminate influence on social utility is that we cannot say whether social utility decreases or is left unaffected, even though what we can say for sure is that it never increases as a~result of such interventions. This kind of indeterminacy we label \textit{weak indeterminacy}. Note again that if the influence of governmental interventions on social utility were to be weakly indeterminate, Rothbard could neither say ``that any action of the State \textit{decreases} social utility'' because he would not be able to say whether state interventions decrease or leave social welfare unaffected. Therefore, more specifically, even though he would be justified in saying that state interventions never increase social utility, he would not be able to determine whether they decrease or leave it unaffected and so, he would not be prepared to state with certainty that they decrease it.



\section{Indeterminacy and Economic Criticism of the Government}

As we mentioned above, Caplan and other authors criticize Rothbard for making a~very anemic economic case against the state. The reason they cite for this criticism is that, according to Rothbard, the effects of governmental interventions on social welfare are indeterminate. However, they are not specific enough about the kind of indeterminacy involved in Rothbard's welfare theory. After all, as we saw above, there are two possible types of such indeterminacy and we submit that the Rothbardian criticism of the state would indeed be anemic, as the above-mentioned authors claim, only if the indeterminacy involved in his theory were \textit{strong indeterminacy}. By contrast, his criticism would by no means be anemic if the indeterminacy he talks about were \textit{weak indeterminacy}. For, if the indeterminacy in question were weak, Rothbard would indeed be able to say that governmental interventions can never increase social utility. And that does not seem to be a~weak criticism of the state at all.



What is yet due at this point is a~word of more precise explanation of why the criticism of governmental interventions following from the adoption of weak indeterminacy would be robust indeed. Note that if the state were an institution which is inherently powerless to increase social utility, there would be no welfare-related point of having it in the first place. Additionally, it would be possible for the state to decrease social welfare although it must be granted that one cannot say with apodictic certainty whether the state would do so in any particular case of its intervention. Given the fact that under this interpretation the state could not increase social welfare and might indeed even decrease it, the Rothbardian criticism appears to be almost as robust as it can get. After all, if showing that a~given institution is structurally unable to ever improve social utility does not amount to a~robust criticism of it, then almost nothing does.



Now note that Caplan and those other authors do not provide a~single reason to prefer strong indeterminacy as the proper way of interpreting the Rothbardian welfare theory. This should come as no surprise because they do not even draw the very distinction between strong and weak indeterminacy. Thus, even if their criticism of Rothbard's economic case against the government happened to be true, it would nonetheless be unjustified as far as their argument goes. For, as we already made clear, the anemic character of the economic criticism of the government does not follow from the indeterminate nature of its impact on social utility. It would only follow if the indeterminacy in question were to be weak---but this, however, was not established. Moreover, we submit that there are actually four reasons to believe that the indeterminacy in question should be construed as \textit{weak indeterminacy}. First of all, it follows from the way Rothbard understands the Unanimity Rule, a~crucial element of his welfare economics. Second of all, it is only weak indeterminacy that would save Rothbard from contradicting himself within the confines of one and the same page of his seminal essay \textit{Toward a~Reconstruction of Utility and Welfare Economics}. On the other hand, assuming strong indeterminacy would enmesh him in the contradiction. Certainly, it would be uncharitable to maintain that this author makes two mutually exclusive claims over one and the same page, especially when there is an interpretation available that can easily block making such an improbable charge. Third, weak indeterminacy translates into more robust economic criticism of the state and therefore it best aligns with his anti-governmental ethical stance, thus rendering Rothbard's overall position more coherent. Finally, as already suggested while presenting the second reason, interpreting Rothbard's welfare economics in terms of weak indeterminacy would abide by the principle of charity.



\section{Rothbardian Understating of the Unanimity Rule }

We submit that the fact that Rothbard adopts the Unanimity Rule as a~criterion of welfare-enhancing exchanges provides a~reason to believe that the indeterminacy involved in his theory about the impact of governmental interventions on social utility is weak (and hence, that his criticism of the state is robust rather than anemic). How Rothbard conceives of the said rule is evinced by the following lengthy quote:



This Rule runs as follows: We can only say that ``social welfare'' (or better, ``social utility'') has increased due to a~change, if no individual is worse off because of the change (and at least one is better off). If one individual is worse off, the fact that interpersonal utilities cannot be added or subtracted prevents economics from saying anything about social utility. Any statement about social utility would, in the absence of unanimity, imply an ethical interpersonal comparison between the gainers and the losers from a~change. If X~number of individuals gain, and Y~number lose, from a~change, any weighting to sum up in a~``social'' conclusion would necessarily imply an ethical judgment on the relative importance of the two groups. 
%\label{ref:RNDI7LWQSSyob}(Rothbard, 2008, pp.244–245)
\parencite[][pp.244–245]{rothbard_toward_2008}%




Note that, according to Rothbard, there is only one sort of change after the occurrence of which an increase in social utility can be justifiably predicated and that is the situation wherein at least one party benefits and nobody loses. By contrast, in case in which one party gains while the other loses, that is, ``in the absence of unanimity,'' we must be left with an indeterminate verdict as to the impact of such changes on social utility. Now the question arises: is the verdict under consideration strongly or weakly indeterminate?



We claim that the corollary of Rothbard's contention to the effect that ``we can only say that ‘social welfare' [...] has \textit{increased} [...], if no individual is worse off because of the change (and at least one is better off)'' is the weak indeterminacy interpretation of the way governmental interventions influence social utility. After all, if ``we can only say'' that social welfare increases if nobody loses utility and at least one person gains it, then in the situation wherein there are both utility gainers and losers it must be the case---by way of contraposition---that what we cannot say is precisely one thing only: that social welfare was enhanced. And since we cannot say that it was enhanced, we are justified in saying that it was not enhanced. This in turn leaves us with indeterminacy only about two things, that is, whether (a) social utility diminished or (b) remained at the same level. But this is exactly the weakly indeterminate reading of the way Rothbard conceives of governmental acts vis-à-vis\textstyleEmphasis{\textbf{\textup{\textcolor[rgb]{0.37254903,0.3882353,0.40784314}{~}}}}social utility. For indeed, it is the weak indeterminacy interpretation that has it that we are warranted in being agnostic only about whether governmental interventions decrease social utility or leave it unaffected.



To make our point even clearer, note that what Rothbard claims is that ‘We can only say that social welfare increases if no one loses in utility' (and at least one person gains). We contend that what it means is that only then it is true that social welfare increased. Now by contraposition it must be the case that ‘If someone loses in utility, then we cannot say that social welfare increases.' Again, we submit that what it means is that it is false that social utility increases in such a~case.\footnote{But why do we claim so? Does not Rothbard say that ``[i]f one individual is worse off, the fact that interpersonal utilities cannot be added or subtracted prevents economics from saying anything about social utility'' rather than it prevents economics from saying that social welfare increases? He does, but then he adds that we should ``conclude therefore that \textit{no government interference with exchanges can ever increase social utility}.'' Thus, the point is that it is up for debate how to understand Rothbard's stance on what is going on when someone loses in utility. Our claim is that it is better to understand him as saying that it is false that social utility increases in such a~case than that we cannot say absolutely anything about it. Why? For one thing, because it avoids what Prychitko called ``a careless self-contradiction'' in Rothbard (see section 5 below). Second, because opting for the agnostic reading renders Rothbard's second welfare theorem---that ``no act of government can ever increase social utility''---disappointingly uninformative. Of course, ``no act of government can ever increase social utility'' if no act of government can ever decrease it, increase it or leave it as it is (due to impossibility of interpersonal comparisons of utility). To be sure, then Rothbard's second welfare theorem follows as a~matter of logic, but it follows vacuously, due to the antecedent being false. Finally, the agnostic reading gives rise to the question of why, if we cannot say absolutely anything about social utility in the case of governmental intervention, Rothbard is so keen on saying that therefore ``no act of government can ever increase social utility'' rather than that no act of government can ever \textit{decrease} social utility? We are equally in the dark about both of these effects. Would then honesty not require that an economist use less prejudicial language in expressing his agnosticism about the effects of state interventions? Our reading of Rothbard avoids these and other problems. Or so it seems to us.} However, if it is false that social utility increases, then it must be true that it does not increase. But does it mean that, therefore, social utility decreases? This does not follow. For even though social utility does not increase, it is still not clear whether it decreases or stays at the same level. This, of course, means that social utility is indeterminate but only in the weak sense, that is, only between two possibilities of decreasing or remaining constant. As to the third possibility, it is determined: ``no act of government can ever increase social utility.'' 
%\label{ref:RNDsMoEH91qUU}(Rothbard, 2008, p.253)
\parencite[][p.253]{rothbard_toward_2008} %
 Therefore, it seems that the weakly indeterminate character of the governmental bearing on social utility also follows from the Rothbardian understanding and commitment to the Unanimity Rule.



\section{The Contradiction Problem}

But why assuming \textit{strong indeterminacy} would portray Rothbard as committing simple contradiction? For on the very same page he says that: ``[a]s economists, we can therefore say nothing about social utility in this case, since some individuals have demonstrably gained and some demonstrably lost in utility from the governmental action.'' 
%\label{ref:RNDK8mEHJNkz8}(Rothbard, 2008, p.252)
\parencite[][p.252]{rothbard_toward_2008} %
 And right after it, he states that: ``[w]e conclude therefore that \textit{no government interference with exchanges can ever increase social utility}… Given the fact that coercion is used for taxes, therefore, and since all government actions rest on its taxing power, we deduce that: \textit{no act of government whatever can increase social utility}.'' 
%\label{ref:RNDkRnf4XTlLi}(Rothbard, 2008, p.252)
\parencite[][p.252]{rothbard_toward_2008} %
 Now if the indeterminacy were to be strong, the latter passage would be inconsistent with the former because the former would exclude the possibility of knowing that governmental interventions never increase social utility. After all, strong indeterminacy implies not knowing whether social welfare diminished, stayed unchanged or increased.



Indeed, this was perspicuously noted by Prychitko 
%\label{ref:RND2IJpLBXfkC}(1993, p.575),
\parencite*[][p.575]{prychitko_formalism_1993}, %
 who contends that ``the additional claim Rothbard makes about social welfare under interventionism---specifically, that no state intervention can ever increase social utility---is a~careless self-contradiction.'' This author goes on to indicate that ``Rothbard argues, ‘economics can say nothing about social utility in this case. Again. We must remain agnostic: we simply don't know.'' In the very next paragraph, Prychitko 
%\label{ref:RNDiuxAubW4Qw}(1993, p.576)
\parencite*[][p.576]{prychitko_formalism_1993} %
 additionally notes that:



Yet his next sentence reads: ``We conclude therefore that \textit{no government interference with exchanges can ever increase social utility}.'' In fact, he goes so far as to proclaim that ``since some lose by the existence of taxes, therefore, and since all government actions rest on its taxing power, we deduce that: \textit{no act of government whatever can increase social utility}.'' Somehow Rothbard has leapt from agnosticism to certainty: the state definitely cannot increase social utility. His italics suggest we take his claim seriously, as an apodictic truth. But it's more apoplectic than apodictic.



Granted, as we pointed out above, at least \textit{prima facie} there seems to be a~tension between Rothbard's prior assertion to the effect that ``economics can say nothing about social utility'' in case of state interventions and his apparently bolder statement which has it that ``no act of government whatever can increase social utility.'' Clearly, if it is literally \textit{nothing} that economics can say about the impact of governmental interventions upon social welfare, then this statement warrants greater skepticism than his more informative assertion to the effect that it is only increases in social utility that the state cannot bring about. In other words, Rothbard's first assertion does not seem to rule out \textit{any} effect of governmental acts on social welfare, whereas his subsequent statement explicitly rules out the possibility of governmental interventions ever increasing social utility.



And yet, there is a~neat way out of this seeming contradiction. A~solution appears to hinge on the way we interpret the Rothbardian contention as to the alleged inability of economics to issue \textit{any} welfare-related verdicts concerning the impact of governmental acts on social utility. We posit that if only we construe the first skeptical assertion by Rothbard along the lines of weak indeterminacy, then the contradiction between his two statements disappears. After all, if \textit{nothing} that economics can say about social utility in case of governmental interventions is only weakly indeterminate \textit{nothing}, then the proposition expressed by Rothbard's first pronouncement is identical with the one expressed by his next sentence. But, most certainly, if the relation between two statements is that of propositional identity, then they cannot contradict one another by any means. Still in other words, if the indeterminacy is interpreted as weak, then it only means that we cannot say whether social utility decreased or stayed unchanged, something perfectly consistent with saying that it necessarily did not increase. By contrast, if we were to conceive of the first Rothbardian assertion in terms of \textit{strong indeterminacy}, then the contradiction would inevitably ensue, for Rothbard would be effectively saying two inconsistent things at the same time, that is, (a) that we cannot say literally anything about the way governmental acts impact social utility and (b) that whatever the effect of state's intervention upon social welfare is, one thing we know for certain is that the state is powerless to increase social utility.



Now given that it would be most uncharitable to attribute to Rothbard self-contradiction within the space of one and the same page of his essay; taking into consideration the fact that the hypothesis according to which Rothbard contradicted himself over one and the same page is highly unlikely; and, most importantly, having at one's disposal an alternative hypothesis that easily explains away the alleged contradiction and coheres better with the rest of Rothbard's theory, we claim that the most plausible interpretation of \textit{nothing} that economics can say about the influence of state's intervention on social welfare is only weakly indeterminate \textit{nothing}, that is, such that is indeed informative, for it rules out the possibility of governmental acts ever increasing social utility.



\section{Coherence of Rothbard's Economic and Ethical Criticisms of the State}

Now Caplan and other authors suggest that there is something wrong with a~putative fact that Rothbard's economic criticism of the government is anemic. If they had not thought so, they would not have made a~charge of it in the first place. Allegedly, it has something to do with his overall anti-governmental stance, for, on the one hand, he is an adamant enemy of the state as far as ethics is concerned while he is presumably only a~weak critic of the government on economic grounds on the other. Besides this fact suggesting that the Rothbardian system might be incoherent across these two branches, it also does not tally well with what Rothbard says about ``a fortunate utilitarian result of the free market'', which is ``by far the most productive form of economy known to man''.\footnote{It is well-worth stressing that, according to Rothbard, it is not only \textit{ex ante} but also \textit{ex post} that the free market is economically more efficient than interventionism. Says Rothbard 
%\label{ref:RNDAli5XD0Vd0}(2009, p.891):
\parencite*[][p.891]{rothbard_man_2009}: %
 ``[T]he free market has a~smooth, efficient mechanism to bring anticipated, \textit{ex ante} utility into the realization and fruition of \textit{ex post}. The free market always maximizes \textit{ex ante} social utility; it always tends to maximize \textit{ex post} social utility as well.'' More, he goes on saying that ``the divergence in \textit{ex post} results between free market and intervention is even greater than in \textit{ex ante}, anticipated utility.'' Upon saying it, Rothbard brilliantly illustrates how the state's interventions prove to be counter-productive. For example, the imposition of a~\textit{maximum} price set \textit{below} a~market-clearing price (i.e. one of the two types of effective price control) inevitably leads to the creation of an artificial shortage. Hence, however benevolently motivated and however beneficial \textit{in expectation}, price control policies fail spectacularly \textit{ex post}. By contrast, as demonstrated by Rothbard, free market is a~self-correcting system. It is losses that allow for weeding out those entrepreneurs that do not serve their customers well and it is continual profits that constitute a~signal that given entrepreneurs do increase the consumers' utility \textit{ex post}. Granted, there is no guarantee that \textit{each} market exchange is going to be mutually beneficial \textit{ex post}. However, as perspicuously observed by Rothbard 
%\label{ref:RNDsYLfxrjoa4}(2009, p.885),
\parencite*[][p.885]{rothbard_man_2009}, %
 ``[p]rofits and losses spur rapid adjustment to consumer demands''. All in all, as far as the \textit{ex post} welfare goes, the market still performs better than interventionism. } 
%\label{ref:RND6GODaJ8Ghj}(Rothbard, 2006, p.48)
\parencite[][p.48]{rothbard_for_2006}%




However, we contend that the apparent incoherence cited above would be attenuated or would disappear completely if the indeterminacy of state's interventions were to be interpreted as weak. The reason is that then Rothbard's economic criticism would be more robust, proving that state's interventions cannot ever increase social utility and thus calling into question the very economic \textit{raison d'être} of the state. After all, then the state would transpire to be at least redundant since it would be economically indifferent at best and harmful at worst. This, of course, would tally much better with Rothbard's otherwise well-known vehement criticism of the state and with his overall anarcho-capitalist stance.



It should be clear that coherence is a~virtue of any theoretical system. So, whenever possible, we should strive for it either \textit{via} theoretical revisions or reinterpretations that allow us to achieve it. Because our distinction between weak and strong indeterminacy, and especially the appeal to the former, bolsters coherence within the Rothbardian system whereas its critics' indiscriminate idea of indeterminacy threatens it, this very fact speaks in favor of supporting our reading of Rothbard's welfare economics. Besides, interpreting his utility theory in a~way that suggests incoherence in his overall system would run against the principle of charity, particularly when there is an alternative interpretation easily avoiding it. Finally, because Caplan and other critics believe, as we pointed out above, that the alleged weakness of Rothbard's economic case against the state and the incoherence it engenders constitute a~vice in his general theoretical system, these authors too should conceive of our distinction as preferable to their own indiscriminate idea of indeterminacy, for it would enable them to get rid of what they themselves consider a~vice.



\section{Conclusion }

The aim of this paper was to argue that---contrary to what some critics maintain---the Rothbardian theory of social utility under interventionism is by no means anemic. That is, the verdicts it reaches are more informative, and therefore less indeterminate, than its critics believe them to be. Specifically, we posit that Rothbard's welfare theory should be indeed construed as saying that there is one thing that we can say for certain; namely, that governmental acts are powerless to increase social utility.



The reasons we provided for the above contention are four-fold. First of all, in his welfare economics, Rothbard explicitly adopts the Paretian Unanimity Rule as the determinant of welfare-enhancing exchanges. What clearly follows as a~corollary from the way Rothbard interprets the said rule is only weakly rather than strongly indeterminate character of state interventions into economy. This in turn means that the best the government can do is to leave social utility unaffected, which calls into question this very institution at least as far social welfare is concerned. It should be noted that such a~conclusion reached by the Rothbardian welfare theory does not even remotely resemble supposedly agnostic conclusions attributed to it by its critics. Second of all, we argued that unless we construed the concept of indeterminacy in the weak way, we would have to conclude that Rothbard contradicts himself over one and the same page of his famous paper \textit{Toward a~Reconstruction of Utility and Welfare Economics}. On the other hand, if we interpret the indeterminacy involved as the weak one, no contradiction ensues and the Rothbardian welfare theory is then rendered consistent. Third, it is only under weak indeterminacy interpretation that Rothbard's overall theoretical system achieves coherence. And fourth, we pointed to the principle of charity, which would be obeyed only if we stick to our fine-grained distinction between weak and strong indeterminacy. All these reasons operating \textit{via} the discrimination between weak and strong indeterminacy support the final conclusion that Rothbard's economic criticism of the state is much more radical than his critics believe it to be.



\section{References}

Caplan, B., 1999. The Austrian Search for Realistic Foundations. \textit{Southern Economic Journal}, [online] 65(4), pp.823–838. https://doi.org/10.2307/1061278.



Hausman, D.M. and McPherson, M.S., 2006. \textit{Economic Analysis, Moral Philosophy, and Public Policy}. 2\textsuperscript{nd} ed ed. Cambridge [etc.]: Cambridge University Press.



Herbener, J.M., 1997. The pareto rule and welfare economics. \textit{The Review of Austrian Economics}, [online] 10(1), pp.79–106. https://doi.org/10.1007/BF02538144.



Juruś, D., 2012. \textit{W~poszukiwaniu podstaw libertarianizmu: w~perspektywie rothbardowskiej koncepcji własności}. Kraków: Księgarnia Akademicka.



Kvasnička, M., 2008. Rothbard's Welfare Theory: A~Critique. \textit{New Perspectives on Political Economy}, 4(1), pp.41–52.



Prychitko, D.L., 1993. Formalism in Austrian-school welfare economics: Another pretense of knowledge? \textit{Critical Review}, [online] 7(4), pp.567–592. https://doi.org/10.1080/08913819308443319.



Rothbard, M., 2006. \textit{For a~New Liberty: The Libertarian Manifesto}. Auburn AL: Ludwig von Mises Institute.



Rothbard, M., 2008. Toward a~Reconstruction of Utility and Welfare Economics. In: M. Sennholz, ed. \textit{On Freedom and Free Enterprise}. Auburn AL: Ludwig von Mises Institute. pp.224–262.



Rothbard, M.N., 1976. Praxeology: The methodology of Austrian economics. In: E. Dolan, ed. \textit{The Foundations of Modern Austrian Economics}. Kansas City: Sheed \& Ward. pp.58–77.



Rothbard, M.N., 2002. \textit{The Ethics of Liberty}. New York; London: New York University Press.



Rothbard, M.N., 2009. \textit{Man, Economy, and State: A~Treatise on Economic Principles; with Power and Market: Government and the Economy}. 2\textsuperscript{nd} ed. Auburn AL: Ludwig von Mises Institute.



Salerno, J.T., 1993. Mises and Hayek dehomogenized. \textit{The Review of Austrian Economics}, [online] 6(2), pp.113–146. https://doi.org/10.1007/BF00842707.

\end{document}


\setcounter{secnumdepth}{0}





\begin{document}

The law of diminishing marginal utility as law of mental order-ness





\section{Abstract}

Nozick 
%\label{ref:RNDcVxqy2Ek3g}(1977)
\parencite*[][]{} %
 formulated a~challenge to Austrians related to the application of the Law of diminishing marginal utility in the context of notion of indifference. To be able to claim that the value or attributed utility of the subsequent units of goods decreases, we must compare comparables, even if deliberate choice means that we have chosen a~particular as being value-different. This causes a~logical paradox. One cannot be indifferent and demonstrate a~particular preference at the same time. It is mutually exclusive.



The paper discusses a~critique of Wysocki 
%\label{ref:RNDRq4kLlZm7t}(2021),
\parencite*[][]{}, %
 who proposes a~solution to the paradox in terms of a~counterfactual perception of the Law. The critique points to the essence of why neo-Misesians cannot resolve the paradox, which lies in the interpretation of the origin of valuation within the particular value scale.



The paper offers an alternative solution based on Hayek's concept of mental order-ness with the implication of the general applicability of the Law to any order in reality.



\section{Keywords}

indifference, choice, homogeneity, Nozick's challenge, orderness.



\section*{The law of diminishing marginal utility as law of mental order-ness}

\footnotetext{ I~would like to thank Walter Block for his helpful comments and remarks regarding an earlier version of this paper and also two anonymous reviewers for their excellent comments and insights, which significantly improved and refined the arguments presented in this thesis. All errors and inaccuracies are mine alone.}

\subsection*{1. Introduction }



This paper is partly a~reply to Wysocki 
%\label{ref:RNDYRga3rIv6O}(2021),
\parencite*[][]{}, %
 but my intention is much broader. Wysocki follows the discussion related to Nozick's 
%\label{ref:RNDDzKXv9kKmZ}(1977)
\parencite*[][]{} %
 challenge to Austrians about the concepts of indifference, choice, homogeneity, and the law of diminishing marginal utility (henceforth the law). Austrians generally don't regard the concept of indifference as a~relevant insight into the economic description of reality. This is due to the fact that related action is always choice-based. There is simply no way to demonstrate, or reveal, indifference that supposedly occurs during economic activity.



However, Nozick 
%\label{ref:RNDLcpecUIYOE}(1977)
\parencite*[][]{} %
 correctly claims that the validity of the law requires the concept of indifference. This condition is a~significant problem in the interpretation of economic phenomena for Austrians in particular. The reason is simple. The choice associated with an agent's economic action implies that the agent chooses goods in a~strictly value-heterogeneous way, i.e., a~given good or action is chosen because it is strictly preferred to something else. What is chosen cannot be value-homogeneous but heterogeneous, otherwise what we choose wouldn't be preferred to something else. But the law must be applied to something which is homogeneous (so far there has been an effort to define homogenous class of goods), otherwise we couldn't make the assumption of diminishing marginal utility associated with an additional unit of a~goods.



Consider the following. Smith drinks one beer after another. The first beer ``Pilsner Urquell'' is good A, the second beer ``Pilsner Urquell'' is good B, and the third one is good C, given that the choice is always specific. However, from the point of view of the law, it is necessary to view the beers in question as homogeneous (e.g., as a~value class of goods or apply the law to something that the homogeneity element contains), otherwise it couldn't be argued that the second and third beers in the order confer a~continually lower utility. From the standpoint of choice, we look at the action in question as three distinct goods: e.g., Beer 1, Beer 2, Beer 3. The reader should be warned that physical sameness or similarity doesn't play much of a~role in the interpretation, since in economics we are concerned with the views of agents as to values.



Thus, it doesn't matter whether we are speculating about the sequence of first to third beer or a~sequence of a~pear, an apple, and a~lemon. The law and choice view both as valid for human action. This constitutes an apparent logical paradox. The paradox attracts what Block 
%\label{ref:RND6rbAd3BgIQ}(1980)
\parencite*[][]{} %
 calls one of the greatest challenges to the Austrian school of thought.\footnote{The reader can follow the debate as starting point of Rothbard 
%\label{ref:RNDD69c6OP65R}(1997)
\parencite*[][]{} %
 and then from Nozick 
%\label{ref:RND2obbZhDtLR}(1977),
\parencite*[][]{}, %
 Block 
%\label{ref:RNDuWubuLrpLQ}(1980; 2009; 2012),
\parencites*[][]{}[][]{}[][]{}, %
 Block and Barnett 
%\label{ref:RND4QVzZ8Rivl}(2010),
\parencite*[][]{}, %
 Hoppe 
%\label{ref:RNDLHBKNzOGku}(2005; 2009),
\parencites*[][]{}[][]{}, %
 Hudik 
%\label{ref:RNDVMNPrwWPjn}(2011),
\parencite*[][]{}, %
 Machaj 
%\label{ref:RNDbJjuY6iIg8}(2007),
\parencite*[][]{}, %
 O'Neill 
%\label{ref:RNDJTEtJ7bafa}(2010),
\parencite*[][]{}, %
 Wysocki 
%\label{ref:RNDQU8lGNPJe0}(2016; 2021),
\parencites*[][]{}[][]{}, %
 Wysocki and Block 
%\label{ref:RNDCAIxj3g1sP}(2018; 2019).
\parencites*[][]{}[][]{}.%
}



It won't be the purpose of this paper to describe the entire related debate. Rather, I~begin with a~critical response to Wysocki 
%\label{ref:RNDoqPQ3VMLhC}(2021).
\parencite*[][]{}. %
 While this author consistently interprets the problem in the neo-Misesian tradition, the conclusions of his paper lead to a~methodological problem in the form of a~shift in the perception of the law to the counterfactual domain 
%\label{ref:RNDvwPL4Uvkwb}(Wysocki, 2021, pp.41–42).
\parencite[][pp.41–42]{}. %
 I~argue that Wysocki 
%\label{ref:RNDG9EMDfAYkA}(2021)
\parencite*[][]{} %
 provides (unconsciously, as can be seen from the context of his work) evidence of interpretive limits of the tradition. As the reader will see, these interpretive limits are primarily related to today's view of the neo-Misesian interpretation of the value scale.



At the same time, I~maintain that it is possible to explain the logical paradox mentioned above using a~different, equally Austrian interpretation. This is based on the work of F. A. Hayek (post-1937 research) and his efforts to explain economic phenomena in terms of order-ness 
%\label{ref:RNDFGLgBfEpnu}(see e.g., Caldwell, 2014; Lewis and Lewin, 2015).
\parencites[see e.g.,][]{}[][]{}. %
 The use of this interpretative tool applied to the valuation process will make it possible to explain the above-mentioned paradox and to show that the law is also compatible with the Austrian view of action, such as preferring and setting aside, not indifference.



I~proceed as follows. In section II, an analysis and critique of the solution provided by Wysocki 
%\label{ref:RNDHFR1pyC9Ln}(2021)
\parencite*[][]{} %
 is offered. Section III names the main problem why neo-Misesians cannot answer Nozick's challenge, suggesting interpretive limits to their approach. In section IV the focus is on a~sketch of the solution based on the Hayek's Model-Map analogy of mind. The conclusion of the thesis will constitute new challenges to explore the problem of the unit of utility (util) and application of the law more generally as one of the laws of any order-ness.



\subsection*{2. Wysocki's (2021) proposal and its criticism}



Wysocki starts his analysis by recognizing that the law requires some kind of homogeneity; simply put, we must compare {\textquotedbl}apples to apples{\textquotedbl} in the context of the law. He also realizes that basing homogeneity on the physical similarity of goods is economically improper. Economics deals with the attribution of valuation and not to the physical nature of goods. The economic actor is the master of valuation and its attribution. Wysocki's interpretation implies that one can, theoretically, subjectively view the Panzer tank, the apple, and the song as economically homogenous goods from the perspective of human evaluation process. He writes 
%\label{ref:RNDIzAOucdPjB}(Wysocki, 2021, p.14):
\parencite[][p.14]{}:%




[…] economists are concerned with only this subset of things, which are economic goods. And for a~thing to constitute an economic good, what it takes is at least one economic actor that believes (falsely or not) that the physical object in question is able to satisfy at least one of his actual needs. Incidentally, note that given Austrian extreme subjectivism, no case can be made for any entailment between physical sameness and indifference (economic sameness).



This is a~combination of radical subjectivism and relativism and I, mildly so far, disagree with \textit{radicalism.}\footnote{This radical subjectivism is possible in terms of interpretation, but man is also constrained by the structural character of reality. A~Panzer tank, an apple, and a~song may be regarded as the same class of goods, but their physical properties \textit{also} ``predetermine'' them in the context of how we deal with them in terms of value. Which in turn leads to whether we evaluate our actions as erroneous or successful. Simply put, a~panzer tank cannot be crunched by hand like an apple and isn't a~sonata. They can be combined to satisfy some defined need but if it were true that classes of goods can be composed subjectively \textit{anyway}, there would be no concept of economic error 
%\label{ref:RNDv58MXnjgoG}(see Pošvanc, 2021a to demonstrate problem).
\parencite[see][]{}. %
 The human Spirit would fall into a~relativistic self-satisfaction delusion, which would be determined by the fact that \textit{every} \textit{subjectively-motivated} decision is correct from the individual's point of view. The concept of error would be non-existent, which equally implies the impossibility of learning from mistakes and non-existence of rational economic development. At the same time, this isn't a~denial of subjectivism. In other words, \textit{even the belief} in the satisfaction of individual needs has its regularities and is based on human knowledge, when the purpose of knowledge is to eliminate the false belief in any value-economic causality, when, at the same time, radical subjectivism is still valid in the sense that man has the right to be mistaken in his beliefs. As one reviewer correctly points out, this is a~\textit{modus tollens} argument. By the argument I~implicate that the radical Austrian position of subjectivity and decision-making has its limits. } However, what is important is that Wysocki recognizes the absolute necessity of a~value-centered interpretation and, within this view, to define indifference as such 
%\label{ref:RNDleFtc5OE0d}(see also Machaj, 2007).
\parencite[see also][]{}.%




Wysocki follows with the definition of the concept of sameness of goods 
%\label{ref:RNDoXylSNct0x}(Wysocki, 2021, pp.16, 20–21).
\parencite[][pp.16]{}. %
 He shows that Austrians consider a~homogeneous group of goods to be such to which the law can be applied\footnote{Austrians also proceed in the same way in the case of the application of time preference; the time preferences are applied to a~value-homogeneous group of goods to ensure a~comparison of value over time. The time preferences issue is interconnected with the problem of interest; a~critique of the concept of interest and time preferences see in Pošvanc 
%\label{ref:RNDEtwG4qif43}(2019).
\parencite*[][]{}.%
}, which causes a~logical problem because the law cannot by applied to other than homogenous units of goods; in other words, we define value-homogenous units of goods based on the law and the law is based on the notion of value-homogenous units of goods. He argues that unless we independently define the notion of homogeneity of a~goods, the law would be a~pure tautology.



Next, he focuses on a~critique of Block's solution to of the problem of indifference and choice, which Block 
%\label{ref:RNDLpmZenxvOI}(1980)
\parencite*[][]{} %
 describes using the example of a~vendor selling the 72\textsuperscript{nd} unit of butter out of a~stock of 100 ounces. Block 
%\label{ref:RNDbRvWS76jyX}(2009)
\parencite*[][]{} %
 claims that 100 ounces of butter should be considered psychologically (apart from human action, as a~thymological concept within the psychological-historical realm) unless we actually engage in choosing. That is to say, before an actual choice is made the owner of this stock of butter is indifferent to all of them. However, once he picks one of these ounces to sell, he can no longer be indifferent among them all.



According to Wysocki, Block 
%\label{ref:RNDiyv1N3QuGL}(1980)
\parencite*[][]{} %
 can be interpreted in two ways. (1) Block's solution can be viewed as the choice of the 72\textsuperscript{nd} unit being the breaking point between a~value-homogenized view of some class of goods (100 ounces) and the subsequent division of that (thymologically) perceived class into two parts---the singleton, the chosen unit in question (72\textsuperscript{nd}), and the remainder of the class (99 units), which becomes heterogenous with the previous set. Using choice as the criterion for the determination of the definition of a~good, however, Block is still faced with the question of why vendor selected the 72\textsuperscript{nd} unit when he previously perceived the given class of goods as homogenized, implying the impossibility of choice. And if Block is claiming that it is \textit{the} given 72\textsuperscript{nd} ounce which somehow \textit{exactly} fits one's preference by virtue of an extensively defined particular state of affairs, then he can't talk about the concept of the same commodity needed for the application of the law. (2) If we view Block in terms of the claim that we have chosen any unit of the good because they can all serve a~given end equally, we fall into the tautological view of the problem mentioned above. So, Wysocki rejects Block's solution entirely.



He then turns to Hoppe 
%\label{ref:RNDGbJIsCfBCD}(2005),
\parencite*[][]{}, %
 who applies indifference to a~different domain. Hoppe can be interpreted as saying that when we act, we choose strictly, being indifferent to something or everything else (the example of T-shirts and sweaters). Wysocki goes on to remind the reader that indifference has to do with how we interpret what is happening and how we interpret the action itself, which he argues is no ad-hoc defense against Nozick's challenge 
%\label{ref:RNDyQIVl9adEf}(Wysocki, 2021, p.30, see footnote 27).
\parencite[][see footnote 27]{}. %
 Wysocki argues that we have no way of avoiding Block's having chosen \textit{the} 72\textsuperscript{nd} unit while at the same time perceiving the given class of butter as homogeneous; once realized, the choice apodictically implies, the absence of indifference.



The Following is a~description of Hoppe's example of drowning children, only one of whom is saved by their mother, Wysocki maintains that the \textit{context} is the proper way to view the act in question. That is, in choosing to save Peter not Paul, the mother doesn't strictly prefer saving Peter over Paul. Rather, she prefers saving the child without preferring Peter over Paul. Indeed, the mother's choice to save Peter as a~strict preference over Paul was the position of Block. This would imply that the mother was \textit{not} indifferent between Peter and Paul. It cannot be denied that she loves both children and she made a~choice to save one of them, not the other, as a~matter of the fact. However, Wysocki opines that the act can be interpreted as a~non-intentional choice, where the mother was merely \textit{authorizing} the rescue of Peter when she acted in terms of her \textit{maxims} (which could be, e.g., morally based). He implies the existence of a~maxim, which is part of agency, where we don't \textit{intentionally} decide a~given act but automatically carry it out. Wysocki 
%\label{ref:RNDKhOt5Am9VM}(2021, p.36, emphasis his):
\parencite*[][emphasis his]{}:%




[…] whether A~or B~is employed \textit{cannot make a~difference} to the actual maxim we are acting on. If our maxim (preferred description of an action) is to save a~child, it simply follows that any child would do equally well. The mother cannot be rendered worse off when Peter (or Paul for that matter) is saved simply because both of these scenarios count as the satisfaction of the very same policy of ours. And that is the reason these two (only seemingly distinct) goods are actually the same economic good and it is precisely for the very same reason that we don't choose between them.



Wysocki concludes that choice under indifference is absolutely impossible, and, to face Nozick's challenge, all the Austrians \textit{have to do} is to use the concept of indifference in \textit{conceptualizing the supply of the same goods} in a~way that two units of the same commodity 
%\label{ref:RNDDRxT9mZuqH}(Wysocki, 2021, p.37):
\parencite[][p.37]{}: %
 ``shall never figure in a~description of one and the same action. In other words, once any two items represent the same economic good, there is no choice between them.''



To sum it up: Wysocki claims that Hoppean position is unscathed once we put action and law based on a~homogenous supply of goods against each other. This is because Hoppe claims that once we act, we choose a~particular good over the other, so it cannot be a~part of the homogenous supply, and we are, therefore, by definition, indifferent to something else.



It follows that, while it \textit{could} seem to vendor and consumer before action that the chosen good is a~part of the same supply of goods (e.g., 1 ounce of butter which was before a~choice in the stock of 100 ounces), it isn't; it is a~part of a~different supply for the actor (1 ounce of butter and 99 ounces of butter) because the choice of the chooser informs us about this difference.



Peter and Paul are the same ``economic good'', however, mother has also some independent notion of ``children'' (so to speak). Mother loves/values Peter and Paul equally and she can imagine protecting them as Peter and Paul, but once they were drowning coincidentally at the same time, she jumped to save a~child (e.g., Peter) and she didn't choose him particularly as Peter but universally as her child (not as Peter) because she did it based on the maxim to save child.



According to him, Block's position is out of consideration because Block would force us to claim that mother saved Peter as a~particular child. Wysocki considers this as inappropriate because mother was indifferent to both Peter and Paul (but not to the notion of saving a~child) before she jumped to water to save her child (Peter).



It sounds quite strange once we subscribe to the Hoppean account of choice/action but, let's say, so far so good and let's look at the solution provided by Wysocki.



Wysocki introduces the notion of double-time indexation; this focuses on the fact that time elapses between conceptual consideration of some supply of a~homogenized units/class of goods and an actual choice thereupon made. It means that 
%\label{ref:RNDCXJ8r6oqrE}(Wysocki, 2021, p.39):
\parencite[][p.39]{}: %
 ``the actor believes that he can swap these units at any time in the future without any loss of utility'', at least when he thinks about the units of goods in question. He gives the example of eggs, which we perceive as suitable for fulfilling different ends (e.g., throwing them at an enemy's window, or eating them hard- or soft-boiled). The passage of time allows one to define a~homogenized supply of a~good (eggs), where one can speculatively apply the law in terms of what one can do with a~good (eggs) as a~homogenized supply, of course unless the man acts.



However, in principle, this is again just Block's solution (to consider before the action ounces of butter to be homogenized units of goods usable for different purposes), of a~more vital nature, since Wysocki is working directly with a~mental environment\footnote{Via speculation on how to use eggs based on our preferences, motives or needs.}, which Block implies in his solution. However, at the end of the process, Wysocki arrives at a~choice anyway, which is a~turning point, exactly the same as in Block's interpretation, only in the mental environment. The difference, compared to Block, is that the choice in the Wysocki's solution could be, following the Hoppean account, anything; meaning here that either it could be something contextual to what we were thinking about (e.g., eggs), or literally anything else. However, I~see no good reason why Wysocki shouldn't apply the same criticism he applied to Block to his own solution; he has, as well as Block, first a~notion of indifference, then a~choice.



The whole interpretation of how we apply the law in terms of considering eventualities, or a~kind of preparatory phase before action, or as a~decision is being made, then leads Wysocki to a~crucial problematic proposition 
%\label{ref:RND2zSPtaMw45}(Wysocki, 2021, pp.41–42):
\parencite[][pp.41–42]{}: %
 ``we claim that the law of diminishing marginal utility (in a~truly Austrian spirit) doesn't depend on the actual employment of our eggs. Rather, the law should be conceived of counterfactually.'' This is a~consistent conclusion in the context of a~neo-Misesian interpretation of action. At the same time, however, this reasoning leads us to a~problematic conclusion. The law should be viewed only counterfactually. Why is this a~fundamental methodological problem?



Every law is regular, repeats itself, and can be interpreted both factually and counterfactually. There are two basic interpretive traditions explaining the relationship between cause and effect which constitute law or regularity. One finding the forces behind causation (e.g., some mechanism\footnote{Protagonists in this field are, e.g, Glennan 
%\label{ref:RNDhqoeeOtnC9}(1996),
\parencite*[][]{}, %
 Machamer, Darden and Craver 
%\label{ref:RNDuhwJ4TCyr1}(2000),
\parencite*[][]{}, %
 Bechtel and Abrahamsen 
%\label{ref:RNDQHzZ3M8vTZ}(2005)
\parencite*[][]{} %
 and many others. }); the other---counterfactual---focused on what causes the fundamental difference that determines the nature of the cause. Ioannidis and Psillos 
%\label{ref:RNDkqgiyUTfaN}(2018, p.144)
\parencite*[][p.144]{} %
 write on behalf of contrafactual account\footnote{Ioannidis and Psillos don't deal with the topic of indifference. Using them is a~methodological attack on Wysocki.}: ``… a~causal claim of the form ‘A caused B' would be understood as implying: if A~hadn't happened, B~wouldn't have happened either. It is in this sense that A~actually makes a~difference for B.'' This is the principle that is always the case 
%\label{ref:RNDsu29FDH3aU}(except for a~once-existing or irregular mechanism, see Ioannidis and Psillos, 2018, p.153).
\parencite[except for a~once-existing or irregular mechanism, see][p.153]{}. %
 This conclusion implies the very concept of regularity which is based on it being a~recurrent phenomenon. If Wysocki then claims that the law of diminishing marginal utility can be perceived \textit{only} counterfactually, then it isn't true that it is a~law or regularity; by definition.



To put it in other words, any regularity in its factual form (as a~regularity) can also be interpreted counterfactually. Applying it to Wysocki, he argues that one of the most important economic laws should only be perceived counterfactually, while the factual aspect (the regularity per se) is logically just the action itself, within which, as we have seen above, he states that we \textit{must} follow strict choice. Applying this on his example of speculation about what is possible to do with 3 eggs and 3 ends provides us, according to Wysocki, with 9 possible scenarios where we are indifferent concerning the supply of 3 eggs, the factual part of the law is either non-existent or could be literally anything, e.g., saving Peter. Mother could days and nights think about protection of her children but once they are drowning, she \textit{can} start to play with eggs and we, following this interpretation, must consider that as a~correct interpretation of action. Based on this we would use the law for quite a~deep thinking and preparation to make a~decision, but once the ``action or preference demonstration comes on the scene'', we put everything behind. This cannot hold as an explanation of a~decision-making process.



Nozick's claim related to the law is correct. We need to compare the comparable over time, even when it comes to valuation. Block 
%\label{ref:RNDhAEC6XecsI}(1980; 2009)
\parencites*[][]{}[][]{} %
 vitally replies that indifference is related to the perception of indifference before the action, and action is already particular. Hoppe 
%\label{ref:RNDlderj4s2kZ}(2005)
\parencite*[][]{} %
 disagrees and argues that we cannot be indifferent to a~supply of goods and then choose one unit of it; it doesn't make sense. Thus, both agree that choice is specific and both work with indifference. Block, however, sees indifference before action as some historical (\textit{enduring}) fact and Hoppe understands it as being indifferent to something else. Wysocki 
%\label{ref:RNDmGAPUiLEeZ}(2021),
\parencite*[][]{}, %
 while following Hoppe, slides, in my opinion, into a~similar solution as Block, and falls into a~methodological trap. It seems to me that it is a~limit of the neo-Misesian interpretation. They cannot overcome the limit without a~change of interpretation, because the law is correct.



\subsection*{3o. The crux of the problem}



As Wysocki 
%\label{ref:RNDvRRPlMsx25}(Wysocki, 2021, p.37, footnote 30)
\parencite[][footnote 30]{} %
 correctly writes, the crux of the problem is in the interpretation of the action 
%\label{ref:RNDba1OTlqSWr}(see also Hudik, 2011).
\parencite[see also][]{}. %
 I~argue that the problem lies in the characterization of the valuation process via a~scale, i.e., the process that takes place before we perform the act---whether intentional or automatic. Present interpretation is based on Rothbard, 
%\label{ref:RNDip0iILW7cP}(Rothbard, 2009, pp.5–6)
\parencite[][pp.5–6]{} %
 who ranks needs that are satisfied by goods. The valuation is a~trilateral relation between the most important need, contrafactual needs, and the human subject 
%\label{ref:RNDurpdQ3DNwY}(Biľo, 2004).
\parencite[][]{}. %
 It means that we rank needs and the most valued one is then the subject of action. Immediately thereafter, the ``whole scale is discarded'' and built anew in a~new economic context; this state of a~completely praxeologically perceived beginning isn't uncommon within the neo-Misesian interpretation, where it is argued that, praxeologically speaking, we are a~completely different person after each action.



The nature of this interpretation must inevitably lead to the Nozick's paradox. We have to have a~new scale after each action, which prevents universal continuity and brings \textit{only} particularity. This is why the Nozick's challenge is still present. In other words, it is because we interpret action based on the one-time-existing, very particular value scale, and we didn't elaborate the notion of an enduring indifference concept which is necessary for the application of the law as an integral part of the decision-making process. It could be said that \textit{interpretation} of the decision-making process is based too much on ``particulars'' (non-applicable over time) without relying on the description of ``universals'' (applicable over time).



In order to resolve the paradox, we need an interpretation that allows us to preserve something at the moment of strict choice\footnote{This is already implicit in 
%\label{ref:RND9fcomjinpr}(Block, 1980; 2009).
\parencites[][]{}[][]{}.%
}; something that persists and something that is only actualized into a~new state by the choice in question. But, at the same time, this should have the same (formal-logical) character, i.e., remains the same over time, in order to be able to apply the law to it. All that remains, then, is to change the interpretation, to make it conditional to the law.



As usual in science, it isn't so easy. To provide a~solution, first, I~have to mention one, barely recognized, problem of the neo-Misesian interpretation. Menger 
%\label{ref:RNDaQlBbq5Hb2}(2007, p.52)
\parencite*[][p.52]{} %
 teaches us that things become goods if there is: a) existence of a~need, b) existence of properties that render the thing capable of being brought into a~causal connection with the satisfaction of this need, c) \textit{human knowledge} of this causal connection and d) the ability to control goods. Humans attribute value based on the importance of goods in question. Value as a~subjectively assigned importance is described by Mises .
%\label{ref:RNDrZMeMmloeR}(Mises, 2014, p.160)
\parencite[][p.160]{} %
 as a~value scale based on the ranking of goods. Mises 
%\label{ref:RNDYXzAVmCrQP}(Mises, 1998, pp.94–95)
\parencite[][pp.94–95]{} %
 describes that a~person chooses between alternatives and chooses the most useful one; the value scale is only implied from action. Rothbard changed the focus on scaling of needs as the source of value. But Mises 
%\label{ref:RNDgvS0QIP8G4}(Mises, 1998, p.92)
\parencite[][p.92]{} %
 states that ``the thinking man sees the serviceableness of things, i.e., their ability to minister to his ends, and acting man makes them means.'' It follows that within ranking of needs \textit{we must already presume its satisfaction by means and only then we choose some things from reality to make them goods}. This is also supported by Wysocki and Block 
%\label{ref:RNDYSPUH7j6J5}(2019)
\parencite*[][]{} %
 who point out that it makes a~difference whether a~need, e.g., N1 is defined as going to \textit{some} cinema, with \textit{some} wife, and \textit{some} way, or whether a~need, e.g., N2 is defined as going to a~\textit{particular} cinema, with a~\textit{particular} wife, and using a~\textit{particular} way.



In other words, it isn't enough for the interpretation to claim that a~man ranks needs and then he chooses goods to eliminate the most anticipated uneasiness.\footnote{It is anticipated uneasiness because for already present-felt uneasiness it would be too late to eliminate it by action 
%\label{ref:RNDY5ZsaBEn3W}(Shackle, 1992; Biľo, 2004).
\parencites[][]{}[][]{}.%
} It is necessary to implicate, already when thinking of needs, the knowledge of means which man has as the ability to use/combine means to satisfy more and more new/novel combinations of needs. The ranking process, therefore, should be based on the rank of needs already interconnected with means. This is correct because there is no mental need that we can conceive of without having a~mental mean for its satisfaction - this causal link is a~dichotomy and it is unbreakable.\footnote{Possible opposition to the claim about the dichotomy of needs and means is provided by Hülsmann (2002 pp. 86-87) who, between needs and means looks for a~concept of interest. He claims that if we had the possibility to choose only the satisfaction of ends, we would do it. He writes (the emphasis is mine): ``Here it is all-important to stress the somewhat trivial point that the purpose of employing a~means can only be to attain an end. The end is what really counts for the acting person, whereas the means is merely the thing or the action that is in between his present state of affairs and the state of affairs in which his end is realized. … For it follows from this fact that, by their very nature, ends have, in the eyes of the acting person, a~higher value than the corresponding means. Clearly, \textit{if an acting person could choose between either having his end realized or having the means to attain it, he would choose the end}.'' This assumption is however incorrect, because once means aren't needed, we wouldn't think about needs because we wouldn't have them; they would be already satisfied with absence of the process of satisfaction. For different criticism of Hülsmann (2002), see Biľo (2004).}



This leads us to creating a~kind of ladder-like character of the (mental) valuation scale, where left side of the ``ladder'' are needs, the right are means, and between them there is a~kind of rung that connects them, thus creating a~value scale (ladder). However, this (I claim consistent neo-Misesian) interpretation would lead us to a~petitio principii error which would be present within this kind of modified interpretation (but a~correct one as I~explain above). Let's say that this is a~very simple way to draw the scale of needs and associated means:



Illustration no.1 -- Consistent neo-Misesian Value scale



{\centering \includegraphics{Lawofdiminishingaslawoforderafter2ndroundofreviewsCopyeditedPP-img001.jpg} }

The illustration describes the \textit{connection} of the need N1 with the means M1 and it is this connection which is ranked on the scale. Once man (mentally) decides about the most valuable one, he decides accordingly to choose some things in reality and make them economic goods. We cannot have only a~scale of needs on the left side of the ``value ladder'' and non-connected means on the right side.\footnote{It is Biľo 
%\label{ref:RNDOs6YYKfDB6}(2004)
\parencite*[][]{} %
 who tried at least to divide the evaluation process into ex-ante evaluation of ends and then in-action evaluation when we select the most appropriate state of affairs. For a~critique of this approach, see 
%\label{ref:RNDEONR80u8bo}(Pošvanc, 2019).
\parencite[][]{}.%
} The scaling of an N-M interconnection is necessary. It is important to repeat that it is different if the N1 is defined as going to some cinema, with some wife, and some way, or whether the N2 is defined as going to a~particular cinema, with a~particular wife, and using a~particular way. The interconnection in question, so to speak, defines the need and corresponding means. This isn't a~purely empirical, somehow objectively given, interconnection. It is derived from the agent's knowledge. It is mentally constructed by him to compare his factual and contrafactual \textit{value} possibilities to determine the most important one. However, this would cause a~circularity in the argumentation. Neo-Misesian interpretation namely claims that the value is derived from ranking, but the rank of Ns-Ms already implies a~value link.



The consistent neo-Misesian interpretation of value emergence based on the value scale would be, therefore, problematic, and its circularity is already in our interpretative language. Consider when we say that something is valued more relative to something that is also valued, but just less. We rank within the same class in question (value class) to be able to compare something as more or less valuable. Similarly, in terms of the definition of the concept of cost, we speak of the sacrificed opportunity as the second most valuable alternative 
%\label{ref:RND2AzKBVAaQp}(or ``the next most urgent want'' Mises, 2003, p.174),
\parencite[or ``the next most urgent want''][p.174]{}, %
 which implies that the second alternative is already valued before the scale creation and it is put as second in the order. Neo-Misesians basically describe a~\textit{ranking} of \textit{attribution of use value} to means/goods to define the \textit{most valuable} one, which manifests itself precisely in particular action. However, the process of arriving at a~given decision and the question of value essence is clearly more complicated 
%\label{ref:RNDyadcewONt9}(see also O'Driscoll and Rizzo, 1996, pp.45–46; Grassl, 2017).
\parencites[see also][pp.45–46]{}[][]{}.%




\subsection*{4. Proposal to resolve a~paradox }



To resolve the paradox, we need a~new interpretative paradigm of the decision-making process behind valuation and choice. I~claim that we mustn't apply the law to goods per se (which are always particularly chosen), but to the mental order concerning our state of well-being and its marginal changes. This isn't just a~practical solution to avoid particularity of goods. The decision-making process is phenomenal in its nature. The human mind interprets surrounding reality only phenomenally.\footnote{Although the interpretation conducted here is based on Hayek 
%\label{ref:RNDnnzYjuT5mS}(1952),
\parencite*[][]{}, %
 in my view, there is a~quite vital possibility to connect it to the phenomenological branch of the value theory related to mental states developed by authors such as Brentano, Ehrenfels, Marty, Meinong, Witasek, and others 
%\label{ref:RNDS5MU0XqU1r}(see Smith, 1994; Grassl, 2017).
\parencites[see][]{}[][]{}.%
}



We should start with Pošvanc's 
%\label{ref:RND3ZvY09RO1G}(2021b)
\parencite*[][]{} %
 attempt to deal with the paradox which provides background for here-presented interpretation. Pošvanc addresses Nozick's challenge by accepting the impossibility of indifference associated with particular choice defended by Austrians. Every action is particularistic and hence so is the value attributed to chosen goods! However, Pošvanc claims that we act in the context of some desired state of ordering of things (portfolio), which provides us with an (admittedly dynamic but still referential) state of indifference. Basically, man would love to have such a~combination of goods which provides him such a~state of affairs that he wouldn't be forced, by felt uneasiness, to intervene by action.



Each choice aimed at acquiring a~good is particularistic and changes the structuring of the portfolio of goods, while the homogenized enduring element of interpretation is the portfolio itself (the whole) and its marginal changes. The more the portfolio is structured in such a~way that we are better able to react with it to the potential removal of the anticipated uneasiness, the more satisfied we are and vice versa. The perceived decrease in the utility of an additional unit of X~is derived from how appropriately/inappropriately the addition of unit of X~in question changes the structuring of the portfolio; basically, the second and third unit of X~causes less and less relevant changes compared to changes made by the first unit of X. At the same time, the disposal of already possessed Y~from a~portfolio as a~good needed to acquire the X, not necessarily as part of the exchange of X~and Y, is part of the interpretation.



It is clear that the portfolio is a~thought construction. Any grouping of goods that we can call a~portfolio would be just a~bunch of things as parts of reality without the agent's mind and his view. It is the agent that ascribes meaning and context to a~given structuration. A~portfolio is thus a~reflection of the agent's Idea of economic orientation where the Idea is the \textit{essence} of the portfolio, which manifests itself in the concrete combination of real goods, which we can call the \textit{substance-based} structuring of reality according to the agent's insight.



The interpretation in Pošvanc 
%\label{ref:RNDx7babpHire}(2021b)
\parencite*[][]{} %
 is only substance-based in its character. It is an interpreted consequence, but not the cause of the phenomena in question. Some mental-phenomenological-level interpretation must be implied. This is where Hayek 
%\label{ref:RNDULiSzgdAcM}(1952)
\parencite*[][]{} %
 and especially his analogy of the Map and the Model can serve as an inspiration. One of the reviewers of this paper has rightly wondered whether it is possible to link the analogy in the context of the problem we are addressing. Before I~provide the link, let us briefly look at Hayek's analogy.



\subsubsection*{Map and Model }



Hayek 
%\label{ref:RNDXQBpA1CvRp}(1952),
\parencite*[][]{}, %
 in attempting a~conceptual account of the human mind, in my view, applies a~strategy of multi-order-ness: the interrelationships of the neuronal and sensory orders and many layers of mental orders are interpreted as a~new order (new order-whole), which we call the conscious mind. The principles of classification of information (the many-layers-ness of \textit{mental} order) that result from the interrelation of neuronal and sensory orders cause the mind to reflect in a~somewhat identical, but not fully identical way, the order of things in reality; the mind interprets reality and through this process differentiates itself from reality, causing the mind to perceive itself; basically, free will arises as some layer from previous lower-level \textit{mental} layers and develops itself into a~unique personality.



For the purposes of this paper, the analogy between the Map and the Model, which Hayek 
%\label{ref:RND4PVSQH2bFq}(1952)
\parencite*[][]{} %
 describes primarily in sections 5.17 to 5.91, is important to us.\footnote{The reader should, in my view, be warned to read Hayek contextually when describing this analogy. For sometimes he describes the Map and the Model in the context of any organism and sometimes in the context of the mental order of man. In doing so, he tries to link the mental level with the physically constituted neuronal and sensory order-ness that are driven by naturally defined laws. At the same time, he does not hesitate to remind us from time to time that the emergence of the Map and the Model is subject to the historical evolution of organisms. In other words, he suggests that the proto-origins of the Map and the Model, as well as of mentality as such, are to be sought in the evolutionary development of organisms. So, it is difficult to follow but once a~reader accept that all is deeply dynamic in the description, it should make sense.} The Map is described as a~mental, topological, not topographical, model of the mind (or any organism) of the surrounding reality (like a~subway map), where the various neuronal patterns that organism acquires through experience serve as the ``hardware'' that evokes the Map. The Map is not an accurate picture of reality. Rather, it serves for orientation of the organism in the environment and is created in the context of the environment in which the organism has lived. Although what it has experienced has had a~particularized form, as well as the associated patterns of stimuli must have been particular, the Map is universal in its character; it implies the possibility of combination and associativity of the various ``knowledge'' of a~given organism about its environment, which, according to Hayek's theory, is possible on the basis of the classification of stimuli in terms of an order/group/class. This allows to create some (mental) universals because any order/group/class implies that some phenomena are classified based on some similarities; and yes, it implies that any organisms create \textit{their own} universals to interpret reality. This in turn, meaning creating universals based on the classification, allows for the association of past experience to similar, but particularly novel, circumstances through which the organism grasps the reality. The Map is constructed on the basis of experience in a~gradual manner, where the later stimuli must be incorporated into the Map in the context of earlier ones; and also, many different maps of organisms evolve as species does. Hayek goes so far as to say that the analogy of the Map fades away gradually 
%\label{ref:RNDmKrYtgcrms}(Hayek, 1952, sec.5.30),
\parencite[][sec.5.30]{}, %
 and this is because what we are describing creates more and more new layers of classification.



The Model ``grows'' over the Map. It is predictive. In the context of less evolved organisms, Hayek describes a~Model as based on associativity of neuronal activity, which allows for the application of a~combination of patterns to new anticipated events, where the Model prepares the organism for various contingencies that may come based on cues from reality. Hayek writes 
%\label{ref:RND603axN6tkf}(Hayek, 1952, sec.5.87):
\parencite[][sec.5.87]{}: %
 ``Whenever the classifying mechanism treats as alike, or as alike in certain circumstances, any group of events, it will be able to transfer any experience with any one of them to all of them.'' The Model is therefore more robust and richer compared to Map. The reason for this is the universality of classification, where, on the basis of universals arising as (spontaneous\footnote{Interesting in the context of spontaneously arising thoughts is the research on the influence of the REM phase of sleep on people's mental states, which is manifested in the context of the brain seeming to integrate some cognition on its own without the deliberate action of consciousness (the so-called Eureka effect); see, e.g., Pstružina 
%\label{ref:RNDSvGAYSC7Th}(1994).
\parencite*[][]{}.%
}) products of classification, the Model allows the construction of broader circumstances that the Map doesn't and cannot contain. But at the same time, the Map and the Model influence each other. Changes in the Map affect changes in the Model and vice versa. These are thus semi-permanent dynamic structures 
%\label{ref:RND8TBCl8FwvK}(Hayek, 1952, sec.5.43).
\parencite[][sec.5.43]{}. %
 This means that if the Map did not contain some universals before, the influence of the Model causes the Map to develop in that direction. This results in the Model being able to combine and model new eventualities more robustly.



Although Hayek highlights the influence of the past on the construction of the Map-Model, it should be noted that the context of the Map-Model's continued operation is not tainted by the influence of the past. The past provides stimuli that are classified universally, which in turn allows for associativity, new forms of combinations in the classification of stimuli, or the predictability of the Model. Although the Map 
%\label{ref:RNDizY418Zkf5}(Hayek, 1952, sec.5.42)
\parencite[][sec.5.42]{} %
 represents a~picture of the past in which the organism lived, it does not in itself provide information about the current state. Thus, the past is relevant only in the sense that it provided the stimuli for the construction of the universal Map, but the whole mechanism of the Map-Model is oriented towards the future. And this, according to Hayek, is true for all kinds of organisms 
%\label{ref:RND3kDu4OqT06}(Hayek, 1952, sec.5.61),
\parencite[][sec.5.61]{}, %
 not just humans.



From the human point of view, the basic characteristics of the Map and the Model are already incorporated (evolutionarily) in the genetic equipment of man in the form of various kinds of automatisms (mechanical, instinctive behavior), which are subsequently developed into the abstract-mental form of the conceptual and conscious mind. However, many operations of the brain and mind remain at the level of automatisms, and only a~minor part of the phenomena remains in the attention of the conscious mind. The mind does not classify objective reality as it is (as a~thing-in-itself), but only in the context of a~pre-existing classification by classes of other objects of reality, when conceptual knowledge is formed as the formation of abstract concepts at higher levels of the informational and mental order. In the sense of the conscious mind, this permanent classification and reification according to Hayek 
%\label{ref:RNDLDK10F7Zn0}(1952, sec.6.47)
\parencite*[][sec.6.47]{} %
 comes crashing down on us, because this classification is still only an inaccurate representation of reality.



\subsubsection*{Map–Model and the Idea of economic orientation (portfolio) }



Having briefly introduced the analogy, we can, \textit{mutatis mutandis}, proceed to realize the link between the analogy and the above-mentioned \textit{Idea of economic orientation.}\footnote{As a~supportive argument to the here-presented attempt, we could use Horwitz 
%\label{ref:RNDT9NT6HZ3U6}(2010),
\parencite*[][]{}, %
 who followed a~similar approach to deal with an organizational learning problem using Map and Model as an explanation of Balance Sheets and Budgets. The Map and Model are essential phenomena to mental structuration of Balance Sheets and Budgets while Balance Sheets and Budgets reflect or represent a~real structuration of the historic activity (Balance Sheets/Map based) and planned activity (Budgets/Model based) of the firm within a~reality. } The Idea of economic orientation transpired into the portfolio concept is not the \textit{whole} Map–Model dynamics, but it is possible to consider it as some mental sub-order. The Idea, as a~sub-order, has all the characteristics of the Map–Model analogy: 1) The portfolio of actually owned (acquired-spent) goods is nothing else than the result of the person's past economic ideas and experiences. It thus reflects a~coherent \textit{universal} Idea of a~person's economic experience within his or her environment. 2) Some future–oriented state of affairs constructed as a~picture of desires and needs transpired into the desired portfolio has an anticipatory (Model–based) character. It expresses what we would like to have in order to be able to face an uncertain future.\footnote{As Pošvanc 
%\label{ref:RNDutiARrmACG}(2021a)
\parencite*[][]{} %
 shows, we do not prepare for future precisely, as if in all possible details. On the contrary. The predictiveness is based on relativizing the meaning of the particular state of future through the combination of means and diverse features of them with each other, reshaping and combining physical properties of things in a~way so that a~given different combinations of needs to be fulfilled are ``embedded'' in things/goods (e.g. steel is transformed into a~knife in such a~way that it can be used to cut meat, but also as a~weapon or a~tool for carving toys), as well as the evolutionary discovery of the universal characteristics of individual needs and the associated features of means (area of entrepreneurship), and last but not least, the construction of various economic tools through which we communicate this knowledge to each other, such as, e.g., money, or the price system providing information about bid-ask spreads, but also various social institutions guaranteeing universals for our orientation in reality in the form of, e.g., property rights or law enforcement we can rely on in the time continuum. } 3) There is also a~potential state of affairs which expresses what we can realistically achieve as kind of combination of the Map and Model universal knowledge.



The interpretation should start with our idea of how something ought to be (Model–based Idea). This normatively defined Idea is as if some ``picture'' about the desired state of well-being which is confronted with how something is and how we are able to (based on the knowledge) achieve the state in question (Map–based Idea). It is a~flexible–dynamic mental structure of desires and needs present as some future picture of state of affairs,\footnote{This image is made up of both real and unreal ideas.} to which flexible–dynamic mental structure of combination of means of their fulfillment is formed (the Idea of a~desired portfolio).



The Map part of the structure consists of an actual and a~potential idea of the portfolio. The reason to think about them in the area of Map is twofold: a) the Map structure has its correspondence/reflection into the real world as a~substance-based portfolio of real goods.\footnote{I~think that this is also the way how free will (as pure phenomenon) exerts causal influence on the noumenal world (thing-in-itself); by \textit{correspondence} or by \textit{reflection of this correspondence}.} It informs us about existing combination of means; what we really have as the existing portfolio in t\textsubscript{0}, b) the Idea of the potential portfolio always happens to be the actual one in t\textsubscript{1} and the change must happen in reality as well.



However, the knowledge of what we can attain as some potential portfolio in some defined t\textsubscript{1 }is as part of the Map (universal knowledge based on experiences) as well as of the Model (prediction–based knowledge). The mental decision to change existing combination of means and our ``problematic'' particular subsequent choice of a~real good is merely a~boundary state by which we change our overall portfolio in the reality, which subsequently causes actualization of the Idea of the portfolio within a~mental Map structure.



The interpretation has also led us to the wanted homogenous element which provides a~reference to indifference. It is the Idea how a~\textit{combination} of needs could be fulfilled by a~\textit{combination} of means (portfolio); or the Idea of economic orientation\footnote{It is not purely Kantian subjective kind of Idea. The Idea is also influenced by naturally given automatisms of men (e.g., instincts) and the socio-cultural intersubjective context and could be, therefore, interpreted better as part of Hegelian evolving Absolute Idea 
%\label{ref:RNDw94dhyAF3H}(see interpretation of Hegelian logic by Harris, 1983).
\parencite[see interpretation of][]{}. %
 }. Due to the Model–Map analogy the Idea is triadic: as the Idea of \textit{desired---actual---potential} portfolio; This mental structure persists, although it is still different, which is caused by different interconnections and combinations throughout time; if we use Hegelian expressions, what is achieved, as a~potential portfolio, sublates (German \textit{aufgehoben}) over what was desired and what was actual.\footnote{The explanation of the logic behind the Triad; see Maybee 
%\label{ref:RNDMoaGPUfvG8}(2020)
\parencite*[][]{} %
 or deeper analysis e.g., in Harris 
%\label{ref:RNDgGoIHSGetw}(1983).
\parencite*[][]{}.%
}



We can imagine it as the very simplified illustration No.2.: I~have illustrated separately the Map and the Model in order to illustrate the differences, but a~given Idea is the one coherent mental whole, where the desired and potential states are only imagined and compared to the actual state in reality based on the combination of knowledge of how to achieve the states of affairs in question; the illustration can also be imagined in such a~way that the given geometrical objects lie on top of each other as layers, which would equally illustrate the differences and at the same time they would be one dynamic geometric figure.



Illustration no.2.: Man's Idea of economic orientation



\includegraphics{Lawofdiminishingaslawoforderafter2ndroundofreviewsCopyeditedPP-img002.jpg}



I~illustrated mental structures also consisting of some partial states (Ps)---these states are like puzzle–parts of the whole picture of man's Idea, and they could represent personal ideas about anything, e.g., my idea about accommodation, food, work, leisure, holiday, socializing, etc. Basically, partial ideas of what different aspects of a~person's life should be like.



We can see illustrations of a~Model–based Idea (top left) and a~Map–based Idea (top right). Bellow the line are real goods (as circles) composed into the real existing and the potential portfolio; existing one would be changed into the potential one based on action; desired one is unreachable as a~kind of ideal state.



A~Model–based Idea as a~future–oriented mental state is a~dynamic---reference---state of affairs. It is composed of partial states. The Model models our ideas in the form of needs and puts them together with some ideal combinations of means (some ideal portfolio). It is illustrated by the correspondence of Desired state of affairs (as needs and desires) and Desired portfolio of means. This thinking could be about real or imagined connections between needs and means 
%\label{ref:RNDvKMUNx4jG1}(Menger, 2007),
\parencite[][]{}, %
 e.g., as part of the whole structural picture of all my needs, there could be some structural partial-need-state in the form of an idea of my ideal house connected to some idea of a~beach house with a~swimming pool and surfing possibilities. But the model can also produce totally imaginary partial needs, such as the desire to be able to do magic interconnected with the idea of a~magical wand in the context of the whole picture of man.



A~Map–based Idea of Portfolio (as a~mental state) is actually an idea of how something is at time t\textsubscript{0} (represented as what we already have as an existing portfolio of means, e.g., I~don't have the dream-house but I~have a~car) and what it is possible to achieve at time t\textsubscript{1}, e.g., I~can have some kind of knowledge of how to achieve my dream-house or just to make some kind of compromise, e.g., I'll just settle for a~holiday in a~house like that. A~Map-based Idea of Portfolio is changed based on the knowledge transformed into the plan about how to combine existing and new means implemented as action from time t\textsubscript{0} to t\textsubscript{1}.



The change is illustrated by the difference between the structure of the existing portfolio and the potential portfolio (a new partial state, P8, is illustrated and constructed, and the shape–line of P2 and P3 was changed to better represent the correspondence to the desired portfolio caused by a~new kind of combination of means). Basically, the point of the illustration is to show that the existing portfolio structure is more different than potential portfolio, and the non-correspondence is the motivation to act and improve the state of affairs.



The real portfolio is what I~have as real things, e.g., parts of my real accommodation, what I~eat, where I~work, what kind of leisure I~enjoy, where I~go on holiday, who my friends are, etc., and potential portfolio represents what I~am able to achieve. The potential portfolio is constructed by action and is always more similar to the desired portfolio we ideally want to have. So, to improve my state I~have to act, for example, by buying some accommodation to have my house to be more similar to my ideal idea. The potential portfolio represents our ability of what we can achieve under the constrains of individual knowledge, meaning that I~can desire Iron Man house-style on the beach, but my knowledge is inadequate for achieving this desire, so I~have what I~can afford.



The movement from an existing to a~potential portfolio is therefore both a~mental process as well as a~real act; in red there is indicated a~(action/choice) selection of real things---goods, which we combine and group/compose into a~new portfolio that will change its character within reality and thus in the mental area of the Map.



It follows that man senses the \textit{range of the uneasiness} by how much the desired portfolio doesn't correspond to the real portfolio, and he removes the uneasiness by action 
%\label{ref:RNDbpJoHRBdp9}(compare to Hayek, 1952, sec.5.69 and 5.70).
\parencite[compare to][sec.5.69 and 5.70]{}. %
 The value is derived from the range of the non/correspondence between the Model and the Map–based Idea of correspondence of mental states. If we choose a~real good and add it to the portfolio, it improves the map-model correspondence, and so we assign a~value to that good or a~bunch of goods in question.



The value then depends on what and how something improves the spread among the Idea of desired---actual---potential portfolio:



Illustration no.3 -- Subjective value



\includegraphics{Lawofdiminishingaslawoforderafter2ndroundofreviewsCopyeditedPP-img003.jpg}



So, when we implement a~plan, we try to implement it in the way that the Map Idea of Portfolio corresponds as closely as possible to the Model Idea of Portfolio, and it happens based on what kind of goods we add to the real portfolio; more suitable (valuable) the goods, the higher the correspondence. It has to be stated that these are \textit{also dynamic phenomena:}\footnotetext{ In fact, in the thus-presented context, we can also speculate about more enduring sub-states of the mental structure, which are created intersubjectively (beauty, love, traditions, …), but also of an individual nature (individual habits, stereotypes). }

just as the Model changes the Idea, so do the requirements for the changes/alterations to Idea within the Map (as an endogenous change), as well as the real combinations of goods in reality. The changes happening in reality outside of one's purview have the same consequence on the whole dynamism (which is an exogenous change, however, still mentally grasped).



\subsubsection*{Response to the Nozick's challenge}



The triadic Idea of ``desired---existing---potential portfolio'' is a~mental construct that is on the one hand a~homogenized whole, and on the other hand, it is (constantly) changing in its particular form with respect to the strictly directed action being performed. Here lies, therefore, also the proposed solution.



We act particularly while striving to achieve a~state of satisfaction (as some homogenized whole), where, in the case of a~coincidence between the combination of needs (as a~mental state) with the combinations of means to satisfy them (as a~mental state) and the real perception of the achievement of the combination of goods (as real things), the agent is indifferent to further action; his mental state of satisfaction is achieved precisely through the achieved combination of goods in reality. He would be in an economic rest.



The homogenized element necessary for the application of the law isn't, therefore, some value-homogenized class of goods. It is instead the perceived mental order between the Model and the Map–based Idea of portfolio and its correspondence with the state of affairs (goods composed into the portfolio) in reality. The correspondence doesn't have to be achieved because of the dynamism of both, but it is possible (at least in some moments of man's life, as we will see below). If the correspondence is reached, it is a~state where we would be in the maximal mode of indifference or a~personal equilibrium and so without having any interest in any new action.



As noted by one reviewer: Can we just rephrase Nozick in the way that ``the homogenized element is not some homogenized class of goods but instead the mental state of indifference against those goods''?. Possibly, but we must add a~necessary condition. The mental state of indifference against those goods is determined on the basis of \textit{internally} perceived \textit{relative} relationship of goods in question to other goods \textit{already perceived} within the actual portfolio.\footnote{Cf. with Wysocki's 
%\label{ref:RNDJzMMKCwqyR}(2021, pp.26–27)
\parencite*[][pp.26–27]{} %
 critique of Block where he made an important point about ``a particular state of affairs (as specified in extensional terms)'' and ``a content of the actor's intention''.} What is relevant here is not only the frequency and interrelations of goods, but also the significance they acquire in their relative positions to each other in the \textit{context} of Needs satisfaction. If we use the geometrical illustration No. 2 to show this point, what is \textit{also} important is the mereological structural arrangement, determined by subjective details the agent \textit{intentionally} thinks about or which the agent \textit{implicitly} follows based on some wider socio–cultural context.\footnote{This context used by the agent is embedded in linguistic structures, structures of social institutions, or is part of various automatisms created by the Nature in the form of instinctive human reactions or intentionally in the form of, e.g., product design or the provision of contractual services. This context is quite interesting because it is defined for a~subjective decision making in \textit{extensional terms}. Cf. also with the footnote No.3 about radical subjectivism and Wysocki's 
%\label{ref:RNDAWSS5Qr25y}(2021, pp.26–27)
\parencite*[][pp.26–27]{} %
 critique of Block.}



There could be, therefore, a~partial and a~maximal indifference. The partial one is concerned with some sub–partial–Idea of the state of portfolio. Maximal (theoretical) indifference is actually zero uneasiness, or a~state of rest, or total economic peace 
%\label{ref:RNDXGu6A77pKl}(as described by Mises, 1998)
\parencite[as described by][]{} %
 where we have no tendency to \textit{consciously} act; just automatically repeat reached success and enjoy satisfaction.\footnote{Mises describing these states within his concept of ERE was probably inspired here by Knight's notes about uncertainty or action under certainty and perfect knowledge; see Knight 
%\label{ref:RNDTkRGXaNCib}(1964; 1921, pp.201, 268, 294).
\parencites*[][]{}[][pp.201]{}.%
} Pošvanc 
%\label{ref:RND3MJFc6xxKI}(Pošvanc, 2021a, pp.203–211)
\parencite[][pp.203–211]{} %
 doesn't describe this theoretical maximal state as a~state of human–to–robot change, as Knight's or Mises's interpretations might imply, but as a~state of individual satisfaction when we seek to rely on pre–set–automatisms, e.g., contracts or automated provision of services and goods. We would feel full satisfaction or satisfaction at some (very theoretical) maximum.\footnote{The reader should be aware that as Mises / Rothbard did, so here-presented description of a~state of rest is used as some kind of theoretical mental construct.}



While we may never achieve it from a~praxeological point of view, we need to know it in order to get this information by differentiating between what is (existing), what we can achieve (potential) and what we desire (ideal) in some subjective state of ours. Acting causes a~lessening of the uneasiness that comes from the Model–based Idea of portfolio not corresponding to the Map–based Idea of portfolio. In fact, by not conforming to the Map Idea, the Model Idea is actually providing us with information that we need to change something in order to achieve the normatively determined state we want. The application of the law, based on the indifference within the decision–making process which leads into the particular action, can then be illustrated as follows:



Illustration no. 4 -- Law of diminishing marginal utility and indifference



\includegraphics{Lawofdiminishingaslawoforderafter2ndroundofreviewsCopyeditedPP-img004.jpg}



The illustration shows that we act in a~time continuum. We are trying to reach some \textit{desired state of affairs} where we would be indifferent to any new action. But we are able to achieve only \textit{possible state} \textit{of affairs}. Choice by choice we change the perception of what we can achieve. Each choice contributes to some marginal change in the \textit{possible} \textit{perceived state}, and each choice directs us to make the difference between the desired and potential states as small as possible (acting is intended as if to narrow the space between the desired and possible state of affairs in the illustration). Given that both states are dynamic phenomena and constantly changing, the given isn't attainable (but theoretically possible).



One of the reviewers expressed an interesting critical doubt about the proposed solution; he writes: ``the mental state is reached after the choice is made and needs are satisfied, while according to Nozick, it is the state before the choice which indicates the sameness of the goods.'' It could look like I~have shifted the meaning of the crucial term. It is a~proper point.



However, this is not a~changing of order of events–explanation but a~different level of description; I~modified the interpretation in a~two–level way. The homogenized whole (as if the utility denominator or the ever–present reference state) is the triadic Idea of a~portfolio (or mutatis mutandis a~sub–state of a~portfolio) and each choice within which subjective cost–gains are included just carries particular information about the utility change\footnote{Whether it is possible to define a~unit of utility (util) as some kind of nominator is a~topic that requires separate attention and I~reserve it for a~different paper. }.



A~definition of the indifference concept within this new context is, therefore, a~measure of order–ness of the order (system) in question; in the case of a~human being---range of economic order–ness or peace. Indifference is thus anchored in a~time continuum, always prior to action, but, at the same time, continuously perceived. It is like a~referential maximal state that we never reach, but perceive. We, therefore, act particularly, but always in the \textit{context} of the law; each action increases our marginal state of satisfaction, or ex post we find out that some action was erroneous. We know this based on the spread between desired and potential state of affairs we want to achieve. Once the spread is depressed, we marginally increase satisfaction and vice versa. It is also evident that choice is always a~limit state by which we change the structure of a~potentially attainable state, and the range of indifference is a~measure of our mental satisfaction.



That is why it seems to us that it is the addition of each additional unit of some good whose utility decreases while our total utility increases. This is derived from the fact that each unit of the good (each action/choice) causes our potential state of satisfaction (defined by the Map–based Idea of portfolio) to converge closer to the desired state of satisfaction (defined by the Model–based Idea of portfolio) and vice versa.



Let us apply this solution with the examples above.



\subsubsection*{Sequences of 1-2-3 … Beers }



Suppose the desired state of our current satisfaction is defined as, for example, sitting with friends over a~beer and music in a~pub. If we are not sitting in a~pub having a~beer with friends, but we are still at work, we feel uneasiness; the desired state doesn't correspond with the existing one. We change this by action; going to the pub and having our desired beers. The first beer shifts the state of satisfaction closer to the desired state, the second and third shift that state a~little less, given that we are already experiencing what we wanted (spread is depressing), and, at the same time, the desired state itself changes, because when we have, e.g., a~fourth, fifth or sixth beer, we know that there will be trouble at home, as our beloved wife is waiting for us. Of course, if we still have our fourth, fifth… and tenth beer, and arrive late and tipsy, the desired state of peace doesn't subsequently occur, and the next day we assess that we have made a~mistake, whether in terms of family politics (defined as some partial state of our whole satisfaction which causes an enlarging of spread between states) or the state of our body and mind.



\subsubsection*{72\textsuperscript{nd} unit of butter}



Why did we exchange 72\textsuperscript{nd} ounce of butter for money? It doesn't exactly matter whether it was 72\textsuperscript{nd} or 31\textsuperscript{st} ounce. We preferred an ounce of butter to money. What matters is whether a~given 72\textsuperscript{nd} piece of butter satisfies our needs in a~way that narrows the spread between the desired state of satisfaction and the potential state of satisfaction, defined in this case, for example, as using butter on the toast. However, suppose the vendor isn't very honest. And although he declares that he sold us an ounce of butter, we find out at home that it's only ¾ of an ounce. We know he deceived us. We know we made an economic error. Why? Because the spread between the desired state and the potential state isn't filled as we wanted; it is wider.\footnote{The problem of economic ignorance can also be noted here. Economically, we would probably ignore whether a~vendor sells us 99.999\% of an ounce of butter or 1.001\% of an ounce. The given difference wouldn't cause a~marked difference in the spread between the desired and potential state. And yes, some individuals may not be indifferent to even such a~difference; this is subjective.} What does this mean for our interpretation? By buying an ounce of butter (it doesn't matter which one) we expect a~physical kind of standardization of that product. So, it isn't important to us whether it was the 72\textsuperscript{nd} ounce, but whether it is an ounce of butter of some declared quality. At the same time, we exchanged it against money on a~particularistic basis as exactly that 72\textsuperscript{nd} because the vendor doesn't care either since his spread is depressed once he has money instead of some butter. However, as is seen, we act in the context of the law because the marginal state of satisfaction of the vendor, as well as ours, changes.



\subsubsection*{Peter and Paul }



Did the mother save Peter or her child? The mother saved Peter, her child. She lost Paul. Was she acting within the context of the law? Her potential state is certainly far from her desired state, but closer to her desired state if she had lost also Peter. Did she act on the basis of her instinctive or social-moral maxims? Probably yes, since she saved at least one child---Peter. However, if some subsequent investigation found that her potential state of satisfaction would suggest to others in a~society that she acted in some nefarious way (e.g., she profits from the death), our view of her situation could change.



Why did she choose Peter and not Paul? The reason could have been anything. Peter could have been a~worse swimmer, Paul could have been too far away given her strength, or her mental model could have (instinctively) assessed that the probability of rescuing Paul may have been lower. However, based on our interpretation, we know that the spread between the desired--- actual and potential state of affairs was narrower in the case of Peter than that of Paul, and wider, as if they had both drowned. However, the mother's mind evaluated the realization of Peter's rescue as the better alternative.



By this example we can set up also another point. Suppose she had saved Peter first and then Paul. It doesn't matter how she saved them both. What would happen assuming she loved them both equally? She would have achieved her desired state, a~state of utmost contentment, because nothing but her sons would matter to her \textit{at that moment}. A~given particularistic state (saving her sons or both Peter and Paul) would push all other partial states of her mental perception of the world into the counterfactual realm as not important. And that is why she would probably experience for the very moment only a~happy feeling of ultimate satisfaction without any action. However, as I~have shown above, the state in question is dynamic. As time would pass on (perhaps in just a~few seconds or minute) she would again care about other things as well, she would again act in a~particularistic way to improve her subjective state of satisfaction.



\subsection*{5. Conclusion}



I~claim that the work presented is a~solution to the Nozick's challenge. An interpretation conducted in this way gives us the necessary space to apply choice and the homogenized indifferent element within the evaluation process and in the action in which the evaluation results. The above apodictic assertions are preserved; action is always definitive and the decision-making process is based on the indifference and the law in time continuum.



In this way, the Law is linked to the real action. We have an ongoing element in time, as already suggested by Block 
%\label{ref:RNDUj72i7J8Kc}(1980; 2009).
\parencites*[][]{}[][]{}. %
 Choice is not an absolute breaking point, but only relative in the sense of particular changes in the state of affairs under consideration, which has been criticized by Wysocki 
%\label{ref:RNDMly8PkIU6b}(2021).
\parencite*[][]{}. %
 If we reached a~state of individual equilibrium, we would indeed be indifferent to everything else 
%\label{ref:RND91xpFYRsxV}(Hoppe, 2005),
\parencite[][]{}, %
 or we are indifferent to some pre–set automatisms (e.g., a~contract or a~robotic service/product) that benefit our well-being up to the point where the automatism in question does not fulfill pre–set goals or they need to be subordinated to the new ones. Thus, I~have shown the vitality of the approach of Block 
%\label{ref:RND66e3p7piDM}(1980)
\parencite*[][]{} %
 and eliminated problems of Hoppe–Wysocki attempts.



Given that this interpretation concerns one of the most fundamental economic laws, it has many implications. However, from the point of view of this paper, I~would like to draw attention to two directly related: a) the issues of a~utility unit (util) and b) the application of the Law to any order–system.



Concerning the unit, it has to be pointed that goods aren't a~unit of utility. Utility has an intrinsic character resulting from the correspondence of the Model and Map–based Idea of portfolio. This state is necessarily future–oriented\footnote{The past is lost and the present is already happening; see Mises 
%\label{ref:RNDYpeGqvCX0L}(2014 [1953]),
\parencite*[][]{}, %
 and Shackle 
%\label{ref:RND0nykkl92qY}(1992).
\parencite*[][]{}.%
}. This implies that the unit of the utility should be some \textit{unit of knowledge} put into the plan whose step-by-step realization has a~progressively decreasing utility rate in the context of achieving some state of the portfolio. However, this is very open for further investigation.



Concerning the application of the law in a~more general way is a~quite speculative philosophical question. I~apply the law only to some mental structured system, a~part of the mental order in Hayek's sense. But I~speculate that this interpretation can (arguably) be generalized to any order. Although I~enter a~very speculative ground, it has its logical basis in the argument that if we apply the law to the mental order, why not apply it to any structured order that exists in reality. The difference, of course, would be that we cannot use ``free will'' to define a~normative notion of a~desirable and possibly attainable state. But cannot ``free will'' be replaced by the nature-given regularities (laws) of the system in question to which those regularities pertain? Can't these very laws (as essences) determine the desired and potentially attainable state of the system (as substances) and of course in the context of the limitations implied by the absence of free will?



Of course, whether the above interpretation can be modified to any order must already be the subject of other philosophical speculations and arguments, but the interpretation is open to these and many associated investigations. I~consider this also as a~good argument that the here-presented solution is vital because it is interconnectible to different spheres of knowledge.



\subsection{References}



Bechtel, W. and Abrahamsen, A., 2005. Explanation: a~mechanist alternative. \textit{Studies in History and Philosophy of Science Part C: Studies in History and Philosophy of Biological and Biomedical Sciences}, [online] 36(2), pp.421–441. https://doi.org/10.1016/j.shpsc.2005.03.010.



Biľo, Š., 2004. \textit{The Theory of Imputation: A~context of value spreads between means and ends}. Available at: {\textless}http://www2.gcc.edu/dept/econ/ASSC/Papers2004/Imputation\_Bilo.pdf{\textgreater} [Accessed 9 January 2019].



Block, W.E., 1980. On Robert Nozick's ‘On Austrian methodology'. \textit{Inquiry}, [online] 23(4), pp.397–444. https://doi.org/10.1080/00201748008601918.



Block, W.E., 2009. Rejoinder to Machaj on Indi[FB00?]erence. 5(1), pp.65–71.



Block, W.E., 2012. Response to Ben O'Neill on indifference. \textit{Dialogue}, [online] (2), pp.76–93. Available at: {\textless}https://dlib.uni-svishtov.bg/bitstream/handle/10610/2380/DialogueBook2eng2012\_76\_93.pdf{\textgreater} [Accessed 15 September 2022].



Block, W.E. and Barnett II, W., 2010. Rejoinder to Hoppe on indifference, once again. \textit{Reason Papers}, [online] 32, pp.141–154. Available at: {\textless}http://reasonpapers.com/pdf/32/rp\_32\_9.pdf{\textgreater} [Accessed 15 September 2022].



Caldwell, B., 2014. Introduction. In: B. Caldwell, ed. \textit{The Market and Other Orders}, Collected works of F. A. Hayek. Chicago; London: The University of Chicago Press. pp.1–35.



Glennan, S.S., 1996. Mechanisms and the Nature of Causation. \textit{Erkenntnis (1975-)}, [online] 44(1), pp.49–71. https://doi.org/10.1007/BF00172853.



Grassl, W., 2017. Toward a~Unified Theory of Value: From Austrian Economics to Austrian Philosophy. \textit{Axiomathes}, [online] 27(5), pp.531–559. https://doi.org/10.1007/s10516-017-9348-0.



Harris, E.E., 1983. \textit{An Interpretation of the Logic of Hegel}. Lanham: University Press of America.



Hayek, F.A., 1952. \textit{The Sensory Order: An Inquiry into the Foundations of Theoretical Psychology}. London: Routledge \& Kegan Paul.



Hoppe, H.-H., 2005. Must Austrians embrace indifference? \textit{Quarterly Journal of Austrian Economics}, [online] 8(4), pp.87–91. Available at: {\textless}https://mises.org/library/must-austrians-embrace-indifference{\textgreater} [Accessed 15 September 2022].



Hoppe, H.-H., 2009. Further Notes on Preference and Indifference: Rejoinder to Block. \textit{The Quarterly Journal of Austrian Economics}, [online] 12(1), pp.60–64. Available at: {\textless}https://cdn.mises.org/qjae12\_1\_5.pdf{\textgreater} [Accessed 15 September 2022].



Horwitz, S., 2010. The Sensory Order and Organizational Learning. In: W.N. Butos, ed. \textit{The Social Science of Hayek's ‘the Sensory Order'}, Advances in Austrian economics, 1\textsuperscript{st} ed. [online] Bingley, UK: Emerald Group Pub. pp.263–284. https://doi.org/10.1108/S1529-2134(2010)0000013013.



Hudik, M., 2011. A~note on Nozick's problem. \textit{The Quarterly Journal of Austrian Economics}, 14(2), pp.256–261.



Ioannidis, S. and Psillos, S., 2018. Mechanisms, Counterfactuals and Laws. In: S. Glennan and P. Illari, eds. \textit{The Routledge Handbook of Mechanisms and Mechanical Philosophy}. Routledge. pp.144–156.



Knight, F.H., 1921. \textit{Risk, Uncertainty and Profit}. Boston; New York: Houghton Mifflin Company.



Knight, F.H., 1964. \textit{Risk, Uncertainty, and Profit}. Reprints of Economic Classics. [online] New York: Augustus M. Kelley. Available at: {\textless}https://mises.org/library/risk-uncertainty-and-profit{\textgreater} [Accessed 9 January 2023].



Lewis, P. and Lewin, P., 2015. Orders, Orders, Everywhere on Hayek's The Market and Other Orders. \textit{Cosmos Taxis}, [online] 2(2), pp.1–17. https://doi.org/10.2139/ssrn.2507643.



Machaj, M., 2007. A~praxeological case for homogeneity and indifference. \textit{New Perspectives on Political Economy}, 3(2), pp.231–238.



Machamer, P., Darden, L. and Craver, C.F., 2000. Thinking about Mechanisms. \textit{Philosophy of Science}, 67(1), pp.1–25. https://doi.org/10.1086/392759.



Maybee, J.E., 2020. Hegel's Dialectics. In: E.N. Zalta, ed. \textit{The Stanford Encyclopedia of Philosophy}, Winter 2020. [online] Stanford, Calif.: Metaphysics Research Lab, Stanford University. Available at: {\textless}https://plato.stanford.edu/archives/win2020/entries/hegel-dialectics/{\textgreater} [Accessed 9 January 2023].



Menger, C., 2007. \textit{Principles of Economics}. Auburn AL: Ludwig von Mises Institute.



Mises, L. von, 1998. \textit{Human Action: A~Treatise on Economics}. Scholar's ed. ed. Auburn AL: Ludwig von Mises Institute.



Mises, L. von, 2003. \textit{Epistemological Problems of Economics}. 3\textsuperscript{rd} ed. [online] Translated by G. Reisman Auburn, Alabama: Ludwig von Mises Institute. Available at: {\textless}https://mises.org/library/epistemological-problems-economics{\textgreater} [Accessed 9 January 2023].



Mises, L. von, 2014. \textit{The Theory of Money and Credit}. [online] Translated by J.E. Batson Auburn, Alabama: Ludwig von Mises Institute. Available at: {\textless}https://mises.org/library/theory-money-and-credit{\textgreater} [Accessed 9 January 2023].



Nozick, R., 1977. On Austrian Methodology. \textit{Synthese}, [online] 36(3), pp.353–392. https://doi.org/10.1007/bf00486025.



O'Driscoll, G.P. and Rizzo, M.J., 1996. \textit{Austrian Economics Re-Examined: The Economics of Time and Ignorance}. Routledge foundations of the market economy. New York: Routledge.



O'Neill, B., 2010. Choice and Indifference: A~Critique of the Strict Preference Approach. \textit{The Quarterly Journal of Austrian Economics}, 13(1), pp.71–98.



Pošvanc, M., 2019. Evolutionary possibilities of the emergence of economic calculation and money. \textit{Medium}. Available at: {\textless}https://matus-posvanc.medium.com/evolutionary-possibilities-of-the-emergence-of-economic-calculation-and-money-9458d667ae03{\textgreater} [Accessed 9 January 2023].



Pošvanc, M., 2021a. \textit{The Evolutionary Invisible Hand: The Problem of Rational Decision-Making and Social Ordering over Time}. Palgrave Studies in Classical Liberalism. [online] Cham: Palgrave Macmillan. https://doi.org/10.1007/978-3-030-71800-8.



Pošvanc, M., 2021b. The Problem of Indifference and Choice. Answer to Nozick's Challenge to Austrians. \textit{New Perspectives on Political Economy}, [online] 17(1), pp.20–52. Available at: {\textless}https://search.ebscohost.com/login.aspx?direct=true\&db=obo\&AN=149860775\&lang=pl\&site=ehost-live{\textgreater} [Accessed 9 January 2023].



Pstružina, K., 1994. \textit{Etudy o~mozku a~myšlení [Etudes about the brain and thinking]}. Praha: Vysoká škola ekonomická v~Praze.



Rothbard, M.N., 1997. Toward a~Reconstruction of Utility and Welfare Economics. In: \textit{The Logic of Action: I. Method, Money, and the Austrian School}, Economists of the twentieth century. [online] Cheltenham [etc.]: Edward Elgar Publishing Limited. pp.211–254. Available at: {\textless}https://mises.org/library/toward-reconstruction-utility-and-welfare-economics-1{\textgreater} [Accessed 9 January 2023].



Rothbard, M.N., 2009. \textit{Man, Economy, and State: A~Treatise on Economic Principles; with Power and Market: Government and the Economy}. Auburn AL: Ludwig von Mises Institute.



Shackle, G.L.S., 1992. \textit{Epistemics \& Economics: A~Critique of Economic Doctrines}. New Brunswick, N.J. (U.S.A.): Transaction Publishers.



Smith, B., 1994. \textit{Austrian Philosophy: The Legacy of Franz Brentano}. 1\textsuperscript{st} ed. Chicago, Ill.: Open Court.



Wysocki, I., 2016. Indifference -- in defense of orthodoxy. \textit{Societas et Ius}, [online] (5), pp.15–30. https://doi.org/10.12775/SEI.2016.002.



Wysocki, I., 2021. The problem of indifference and homogeneity in Austrian economics: Nozick's challenge revisited. \textit{Philosophical Problems in Science (Zagadnienia Filozoficzne w~Nauce)}, [online] (71), pp.9–44. Available at: {\textless}https://zfn.edu.pl/index.php/zfn/article/view/554{\textgreater} [Accessed 24 January 2022].



Wysocki, I. and Block, W., 2018. An analysis of the supply curve: does it depict homogeneity among its constituent elements? Another rejoinder to Nozick. \textit{MEST Journal}, [online] 6(1), pp.132–143. https://doi.org/10.12709/mest.06.06.01.14.



Wysocki, I. and Block, W., 2019. Homogeneity, Heterogeneity, the Supply Curve, and Consumer Theory. \textit{Quarterly Journal of Austrian Economics}, [online] 21(4), pp.398–416. https://doi.org/10.35297/qja.010004.

\end{document}


\begin{artengenv}{Norbert Slenzok}
	{Monarchy as private property government. A~chiefly methodological critique\edtfootnote{The article contains an extended and refined version of an analysis first presented in my recent book \parencite[][pp.214-231]{slenzok_filozofia_2024}. I wish to thank Paweł Nowakowski and Krzysztof Turowski for their helpful comments on an earlier draft of the article. I am also grateful to Paweł Słomka for historical consultation.}}
	{Monarchy as private property government\ldots}
	{Monarchy as private property government. A~chiefly methodological critique}
	{University of Zielona Góra\label{slenzok-first}}
	{Hans-Hermann Hoppe famously argued that monarchy is superior to democracy insofar as property rights protection is concerned. The present paper calls this claim into question, with much of the heavy lifting being done by methodological ponderings. More specifically, it is demonstrated that instead of \textit{a~priori}, praxeological truths, Hoppe's monarchy theory offers an ideal type of the politician bestowed with an inheritable title to the throne. Against this background, the ideal type in question is shown to be faulty in that it treats monarchs as capitalist landowners of sorts, thereby overlooking strictly political incentives they face, which can predictably push them in directions inimical to free markets.
	}
	{praxeology, sociology, Austrian methodology, time preference, ideal type, apriorism, Hans-Hermann Hoppe}








\section{Introduction}

\lettrine[loversize=0.13,lines=2,lraise=-0.03,nindent=0em,findent=0.2pt]%
{T}{}he topic of this paper is the well-known economic analysis of monarchy laid out by the anarcho-capitalist philosopher and economist Hans-Hermann Hoppe in his best-selling book \textit{Democracy---the God that Failed} 
%\label{ref:RNDSKHqmsW2lh}(2007a).
\parencite*[][]{hoppe_democracy_2007}. %
 In a~nutshell, Hoppe argued that of two opposite political systems, monarchy and democracy, the former is superior to the latter with respect to property rights protection. Hence, although any state is economically unviable (and ethically repugnant), ``\textit{if} one must have a~state, defined as an agency that exercises a~compulsory territorial monopoly of ultimate decision-making (jurisdiction) and of taxation, then it is economically and ethically advantageous to choose monarchy over democracy 
%\label{ref:RNDzBrpcHBu5r}(2007a, p.xx).
\parencite*[][p.xx]{hoppe_democracy_2007}.%
'' This is said to be the case by virtue of an elementary distinction: monarchy, so the argument goes, represents ``private government ownership,'' while democracy amounts to ``public government ownership.'' Now, given that private property doubtlessly fosters efficient management, and that honoring property rights is a~crucial element of managing a~country efficiently, it is monarchy that can be expected to fare better in protecting (or at least not violating) property rights. The key variable in this regard is time preference: private ownership lowers it, while public ownership results in its increase. It is precisely thanks to the longer planning horizon that monarchs are more likely to make good managers of their countries. All this, holds Hoppe 
%\label{ref:RNDnu3HmUhuSz}(2007a, p.xix),
\parencite*[][p.xix]{hoppe_democracy_2007}, %
 can be demonstrated ``in accordance with elementary theoretical insights regarding the nature of private property and ownership versus ‘public' property and administration.'' Put differently, our author seems to consider his contentions a~simple application of Austrian economics to the workings of political systems 
%\label{ref:RNDSjJTqy3qXv}(Hoppe, 2007a, p.xxii).
\parencite[][p.xxii]{hoppe_democracy_2007}.%




The task of the present inquiry is to furnish a~critical account of the above argument. Specifically, while taking no issue with Hoppe's staunchly critical position on democracy, this paper contends that his relative appreciation of monarchy is undue. This shall be proven largely in methodological terms. First, it will be shown that Hoppe's monarchy theory is best interpreted as a~sociological exercise in Weberian ideal-typological modeling rather than a~rendition of pure praxeology.\footnote{To avoid any misunderstandings, let us note that nowhere does Hoppe explicitly ascribe to his theses on democracy and monarchy the status of \textit{a~priori} propositions. Yet, as Gordon 
%\label{ref:RNDATU2AFZULJ}(2017, pp.98–99)
\parencite*[][pp.98–99]{gordon_austro-libertarian_2017} %
 points out, such an impression is created by a~lengthy defense of \textit{a~priori} knowledge in the introduction to \textit{Democracy…} 
%\label{ref:RNDrLQ6TgC1Kd}(Hoppe, 2007a, pp.xv–xix),
\parencite[][pp.xv–xix]{hoppe_democracy_2007}, %
 followed immediately by the exposition of the book's central claims 
%\label{ref:RNDZs20OF8ReO}(Hoppe, 2007a, pp.xix–xxi).
\parencite[][pp.xix–xxi]{hoppe_democracy_2007}. %
 All in all, if a~method is not brought to bear in a~study, then why defend it in the introduction? Nonetheless, elsewhere Hoppe 
%\label{ref:RNDp5HCUuNUyS}(2006, p.33; 2015, p.16)
\parencites*[][p.33]{hoppe_economics_2006}[][p.16]{hoppe_short_2015} %
 correctly asserts that, e.g., investigations into the nature and the growth of the state or the essence of class struggle---incontrovertibly akin to the investigations offered in \textit{Democracy…}, let us add---properly belong in the field of sociology, which lacks the apodictic validity that characterizes purely praxeological judgments (see section 3 of the present paper). More important than Hoppe's own intentions, however, is the inherent logic of his theory. And as will be shown, the entire case for monarchy that Hoppe is making \textit{presupposes} treating monarchy as an instantiation of a~catallactic, i.e., \textit{a~priori} category (more on this in section 4). } Second, it will be demonstrated how Hoppe's ideal type of monarchy conflates the catallactic (praxeological) categories of capitalist and landowner with the political (sociological) one of the ruler with a~permanent title to his office. This, in turn, will allow for identifying some particular limitations of Hoppe's monarchy theory insofar as its very substance is concerned. In effect, the model will be presented as unjustifiably one-sided, albeit not entirely faulty. In other words, it is contended here that while the Hoppean perspective succeeds in elucidating the commendable facets of the monarchical system, it simultaneously fails to capture the no less significant undesirable ones. Hence, the question of which form of government, monarchy or democracy, is preferable in terms of property rights protection remains undecided.



Surprisingly enough, little attention has thus far been paid to Hoppe's claims in the literature. Moreover, some critics 
%\label{ref:RNDOKEJo3x4cd}(Sierpiński, 2016)
\parencite[][]{sierpinski_critica_2016} %
 and sympathizers 
%\label{ref:RNDcJ7YM5uCSA}(DiLorenzo, 2009)
\parencite[][]{hulsmann_hoppean_2009} %
 alike focused exclusively on the substance of the theory at hand, not on its method.\footnote{However, DiLorenzo 
%\label{ref:RNDseVaR4NquV}(2009, p.274)
\parencite*[][p.274]{hulsmann_hoppean_2009} %
 tellingly asserts that Hoppe ``merely applies logic and economic reasoning to a~comparison of monarchy […].'' As will be demonstrated, this is not quite the case, since in order for Hoppe's monarchy theory to proceed as it in fact does, auxiliary---and highly contentious---motivational assumptions are needed. } Other commentators simply took Hoppe's declarations of apriorism at face value 
%\label{ref:RNDhopghVo5qJ}(Crovelli, 2007, p.116; Gabiś, 2005; Machaj, 2009, pp.113–114).
\parencites[][p.116]{crovelli_toward_2007}[][]{gabis_hans-hermann_2005}[][pp.113–114]{kalita_krytyka_2009}. %
 Doubts have been raised by David Gordon 
%\label{ref:RNDzMTNN1LIpO}(2017, p.99),
\parencite*[][p.99]{gordon_austro-libertarian_2017}, %
 who in his otherwise highly favorable review of Democracy… queried whether considerations presented in the book can really be regarded as \textit{a~priori} truths of praxeology. He nonetheless did not develop his doubts into a~full-fledged criticism. The proposition to view Hoppe's analyses as ideal type-based sociology has in turn been hinted at by Gerald Radnitzki 
%\label{ref:RNDqcslX7aAiQ}(2003, p.161),
\parencite*[][p.161]{hoppe_is_2003}, %
 yet he did not provide an in-depth treatment of the problem either. Another author paying attention to the methodological issues is Paweł Nowakowski 
%\label{ref:RNDisNCjegDEI}(2010, p.273),
\parencite*[][p.273]{nowakowski_dlaczego_2010}, %
 who astutely noticed that the comparison of monarchy and democracy transcends praxeology by attributing definite motives to the rulers. Still, he did not go on to show how this impinges on the validity mode \textit{(a priori} or \textit{a~posteriori}) of Hoppe's theory. Finally, Walter Block, William Barnett II, and Joseph Salerno 
%\label{ref:RNDqmoWMoxQLv}(2006)
\parencite*[][]{block_relationship_2006} %
 argued \textit{pace} Rothbard and Hoppe that time preference decreasing with the increment in wealth or income is not an apodictic law but rather an empirical generalization. Here, I~extend their criticism to Hoppe's thesis that time preference is also necessarily lowered by private ownership. In a~word, the present essay elaborates on the insights of all the mentioned authors and complements the existing literature by demonstrating how a~flawed application of the Austrian social science methodology derailed Hoppe's case for monarchy by imparting to it unnecessary lopsidedness.



The paper represents an immanent critique of Hoppe's theory. That is, the purpose is to beat it on its own (stated) methodological ground. Hence, all distinct features of Austrian school methodology in the Misesian tradition, particularly the account of praxeology as an \textit{a~priori} social science 
%\label{ref:RNDUX9sYdLXWN}(see Mises, 1962; 1998; 2007),
\parencites[see][]{mises_ultimate_1962}[][]{mises_human_1998}[][]{mises_theory_2007}, %
 are accepted here for the sake of argument. The method of the study is so-called rational reconstruction. That is to say, the burden of the article lies in exploring the internal logic and inconsistencies of Hoppe's views rather than in hermeneutic-interpretative work 
%\label{ref:RND88ZCD0XLwH}(Linsbichler, 2017).
\parencite[][]{linsbichler_was_2017}.%




The paper proceeds in the following order: section 2 succinctly recounts Hoppe's defense of monarchy. Section 3 supplies arguments for reading that theory as predicated upon a~sociological ideal type rather than on a~set of \textit{a~priori} propositions, with an eye on the question of time preference. Section 4, in turn, submits that the original error of Hoppe's analysis consists in blurring the distinction between ideal types and catallactic functions and the confusion of political means, characteristic of monarchy as a \textit{political} system, and \textit{economic} means, epitomized by undertakings of capitalists or landowners. In section 5, this fallacy is shown to result in further problems. To remedy these, several amendments to Hoppe's ideal type of monarch are proposed. They are intended to help explain certain widely known facts (e.g., the suppression of free speech in absolute monarchies or the incidence of war between them at least equaling that of democratic regimes) left on the cutting-room floor in Hoppe.\footnote{This essay is a~theoretical exercise, so the factual references it brings up are, because of space limitations, rather illustrative than exhaustive. Hopefully, future researchers will test the usefulness of my points in explaining more comprehensive sets of historical data. } The last section concludes.



\section{The Hoppean rehabilitation of monarchy: a~brief reconstruction}

Hoppe's typology of political systems is based on the criterion of ownership rights in the state. Thus, there are only two basic forms of government: monarchy and democracy.\footnote{Stated more precisely, as pure types, democracy and monarchy represent two extremes of a~continuum that comprises various intermediate forms. Medieval feudal monarchies do not fall within the theory's scope at all, since in the absence of sovereign powers on the monarch's part, they represent a~form of pre-state aristocracy, with the king acting as a \textit{primus inter pares}. Instead, pure monarchy is exemplified most fully by European absolute monarchies of the XVII and XVIII century. Classic constitutional monarchies such as those of 1791 and 1830 in France, in turn, whereby sovereignty was divided between the monarch and the people, are situated in the middle of the scale. Contemporary parliamentary monarchies, where the monarch's standing is largely ceremonial, are monarchies in name only, actually constituting democracies 
%\label{ref:RNDNEov7GE1w7}(Hoppe, 2007a, p.18, f.19; 2015, pp.108–112).
\parencites[][f.19]{hoppe_democracy_2007}[][pp.108–112]{hoppe_short_2015}. %
 Consequently, the totalitarian dictatorships of communism, Nazism, or fascism are classified as likewise democratic. For neither Hitler or Stalin, nor other such leaders (perhaps except the Kim family) are considered private owners of their governments. Rather, they are at the helm of mass democratic parties, and the ideologies that provide legitimation for their claims to power present them as mere agents of the \textit{Volk}, the revolutionary proletariat, or some other large group of people thought of as the sovereign 
%\label{ref:RNDiIEqquC4dw}(Hoppe, 1987, p.179).
\parencite[][p.179]{hoppe_eigentum_1987}. %
 } In the former system, the ruler (a king or a~prince) is conceived of as the private owner of a~state. In democracy, on the other hand, rulers are merely temporary caretakers of the government, which---in accordance with the popular sovereignty doctrine---represents public property. The kernel of the Hoppean argument for monarchy boils down to the superiority of private over public ownership. The most significant facet here is that private property fosters low time preference. Private proprietors are---\textit{ceteris paribus}---more willing to operate in a~far-sighted manner, for it is they who will reap future benefits. By contrast, public property is invariably characterized by the following defect: every user of a~common good is incentivized to exploit it shortsightedly lest others consume it before him. Thus, a~monarch, viewing himself and his successors as private owners of the government, will consider his realm a~capital good the productive output of which shall serve him until the rest of his days, later to be passed down to future generations of the dynasty. On the other hand, democratic politicians, given the temporal constraints of their term, will see the government they control solely as a~consumption good. Hence, they will be inclined to relentlessly exploit it regardless of long-term repercussions. Put differently, while the monarch possesses the title to the capital value of the country, democratic caretakers are entitled exclusively to the current use of it 
%\label{ref:RNDEabMfzH6Ns}(Hoppe, 2007a, pp.45–46).
\parencite[][pp.45–46]{hoppe_democracy_2007}. %
 As a~result, democracy ``promotes capital consumption'' 
%\label{ref:RNDAvIH1jGAlZ}(Hoppe, 2015, p.119).
\parencite[][p.119]{hoppe_short_2015}. %
 ``A private government owner will tend to have a~systematically longer planning horizon, i.e., his degree of time preference will be lower, and accordingly, his degree of economic exploitation will tend to be less than that of a~government caretaker''---concludes Hoppe 
%\label{ref:RNDIemBnzFu9U}(2007a, p.46).
\parencite*[][p.46]{hoppe_democracy_2007}.%




The ramifications of this systemic difference are far-reaching. Firstly, taking into account the capital value that his realm presents to him and his descendants, the monarch will refrain from pursuing policies that have detrimental impact on the economy: high taxation, overregulation, or indebtedness. Secondly, the class of tax-consumers will be comparatively small, limited to the reigning dynasty and a~narrow circle of state apparatus members. Thirdly, private ownership in government will induce the ruler to abide by private law. Although he will indeed embark on production and expansion of legislation, he will remain more of an arbitrator than a~law-maker. Fourthly, the impact of the monarchical system on the incidence and conduct of war will be moderating. Since armed conflicts will take place at the ruler's own expense and in order to appropriate new territories to his own benefit, he will be motivated to keep wars short and restrained. Moreover, the monarch can also try to acquire new lands in a~peaceful manner, by means of arranged marriages or trade, thereby avoiding unnecessary belligerence 
%\label{ref:RNDV1r379NWXi}(Hoppe, 2007a, pp.19–23, 46–50).
\parencite[][pp.19–23, 46–50]{hoppe_democracy_2007}. %
 All this is supposed to explain why the rapid aggrandizement of the welfare-warfare state in the XX century coincides with the downfall of Western monarchies and the subsequent shift to democracy 
%\label{ref:RNDNEXTR2WtMT}(Hoppe, 2007a, pp.50–74).
\parencite[][pp.50–74]{hoppe_democracy_2007}.%




\section{Aprioristic praxeology or a~sociological ideal type?}

As has been mentioned in the introduction, Hoppe seems to imply that his analysis belongs in the domain of praxeology---a system of aprioristic, i.e., apodictically, non-experientially valid claims. It is, however, doubtful whether judgments like ``monarchies conduct wars in a~less destructive manner than democracies'' or ``kings and princes exploit less than presidents and prime ministers'' could actually be considered on a~par with classic---and invoked by Hoppe 
%\label{ref:RNDDxCqYQWAiE}(2007a, p.xvi)
\parencite*[][p.xvi]{hoppe_democracy_2007} %
 himself---instances of the synthetic \textit{a~priori} such as ``No two objects can occupy the same space'' or ``Whatever object is colored is also extended,'' or even the Misesian ``Man acts.'' As regards this last proposition, it is also difficult to see how they could be deduced from it. Gordon 
%\label{ref:RNDAjnMnTSdk4}(2017, p.99)
\parencite*[][p.99]{gordon_austro-libertarian_2017} %
 suggests, then, that Hoppe's analysis should be construed as logically related to certain apodictic propositions (e.g., those describing the impact of taxation on the economy) rather than as constituting such judgments themselves. It is worth noting that hints to this effect can also be found in Hoppe, though scattered in texts other than \textit{Democracy}… For example, when explaining the method of his historical reconstructions from \textit{Short History of Man}, Hoppe 
%\label{ref:RNDfKC7Mqzmef}(2015, p.16)
\parencite*[][p.16]{hoppe_short_2015} %
 states: ``The events in human history that I~want to explain are not necessary and predetermined, but \textit{contingent empirical} events, and my studies then are not exercises in economic or libertarian theory.'' In the same vein, while exploring the nature of the state as an ``expropriating property protector'', whose members take advantage of the apparatus of power to satisfy their own power and money lust, Hoppe 
%\label{ref:RNDuEcXHJWHAz}(2006, p.33)
\parencite*[][p.33]{hoppe_economics_2006} %
 points out:



\begin{quote}
Why is there taxation; and why is there always more of it? Answering such questions is not the task of economic theory but of praxeologically informed and constrained sociological or historical interpretations and reconstructions, and from the very outset much more room for speculation in this field of intellectual inquiry exists.
\end{quote}



Note that there is one thing all those fields---historical reconstructions of momentous events such as the industrial revolution, theory of the state, and class analysis---have in common. To wit, they attribute to agents certain explicit assumptions regarding their preferences that cannot be traced back to \textit{a~priori} axioms. For instance, in order to argue that members of the state apparatus exhibit a~constant tendency to seek expansion of their power, it must be assumed---non-apodictically---that their minds harbor such a~preference to begin with. Hoppe 
%\label{ref:RNDMpge1w8aP4}(2007a, p.15)
\parencite*[][p.15]{hoppe_democracy_2007} %
 admits this implicitly as he writes: ``\textit{Under the assumption of self-interest} [italics added], every government will use this monopoly of expropriation to its own advantage---in order to maximize its wealth and income.'' The same applies to the comparative analysis of monarchy and democracy: in the former system, ``assuming no more than self-interest, the ruler tries to maximize his total wealth, i.e., the present value of his estate and his current income 
%\label{ref:RNDg1NB1vuX7U}(Hoppe, 2007a, p.18).
\parencite[][p.18]{hoppe_democracy_2007}.%
'' On the other hand, ``once again assuming no more than self-interest… democratic rulers tend to maximize current income 
%\label{ref:RNDtDxIjj9WYK}(Hoppe, 2007a, p.144).
\parencite[][p.144]{hoppe_democracy_2007}.%
''



But to posit this is to transcend the purview of pure praxeology. Let us briefly recount what, according to Mises, constitutes the difference between praxeology and history. Namely, praxeology, as a~purely formal discipline, explicates the form of action, i.e., studies deductively the logical consequences of the fact that persons act. History, on the other hand, deals with the substance of action, which comprises goals actually pursued by agents 
%\label{ref:RND7l1PPvyapr}(Mises, 2007, p.271).
\parencite[][p.271]{mises_theory_2007}. %
 Hoppe's harnessing the assumption of politicians' interestedness unambiguously positions his considerations in the realm of Misesian history. Furthermore, as should be clear from the foregoing summary of his claims, in his monarchy theory, Hoppe construes this assumption along reductionist lines. As Nowakowski 
%\label{ref:RNDbPulWo7Q3I}(2010, p.272)
\parencite*[][p.272]{nowakowski_dlaczego_2010} %
 aptly notes, Hoppe reduces the complex motivations behind the actions of people in power to the motive of pecuniary gain.\footnote{In all honesty, in the only explicit formulation I~have found in \textit{Democracy…,} Hoppe 
%\label{ref:RNDHUVliVgK2L}(2007a, p.144)
\parencite*[][p.144]{hoppe_democracy_2007} %
 defines self-interest more broadly as ``maximizing monetary and psychic income: money and power.'' As will be seen, however, his theory effectively throws the power motive overboard. Otherwise, the oversights pinpointed in the next two sections of this essay would have been avoided. The preponderance of the ``wealth and income'' talk is also visible in the quotes adduced above. } Concomitantly, he employs a~conception of rationality narrower than that characterizing Mises's praxeology and more akin to that of neoclassical economics 
%\label{ref:RNDlWD163Ede3}(see Long, 2006; Rizzo, 2015).
\parencites[see][]{long_realism_2006}[][]{coyne_problem_2015}. %
 For if it is assumed that members of the state---or, for that matter, monarchical or democratic heads thereof---are predisposed to do this-and-that on a~more or less permanent basis because their value scales are such-and-such, it must first be assumed (a) that those value scales \textit{are} such-and-such and that (b) they are at least fairly fixed. Were this not the case, Hoppe's monarchy theory could not claim any predictive validity, as it does in asserting that ``it is economically and ethically \textit{advantageous to choose} [italics added] monarchy over democracy'' 
%\label{ref:RNDESfbbbK6sy}(Hoppe, 2007a, p.xx).
\parencite[][p.xx]{hoppe_democracy_2007}.%
\footnote{One may add that in adopting the assumption of narrow self-interest, Hoppe not only goes beyond praxeology but also comes closer to political economy in the public choice tradition, which explicitly disposes of the notion of benevolence on the part of politicians. By contrast, Misesian economics sticks to praxeological ``formalism'' by declaring agnosticism with regard to motives and focusing on the absence of market process in political decision-making 
%\label{ref:RNDiUjyOMd3UO}(Boettke and López, 2002).
\parencite[][]{boettke_austrian_2002}. %
 However, what is presupposed by Hoppe's monarchy theory is in fact a~\textit{very} narrow self-interest, that is, one reducible to pecuniary profit. This resembles the classic 19\textsuperscript{th} century model of economic man much more than its contemporary incarnations, those used by public choice theorists included. }



As has already been discussed, the cornerstone of Hoppe's theory of political systems is the notion of time preference. Despite certain technical controversies, those followers of Mises who concur that ``the actor always prefers satisfaction sooner rather than later'' are unanimous in deeming this statement an apodictic theorem 
%\label{ref:RNDIwcUaYyPDh}(Herbener, 2011; Mises, 1998, pp.480–485; Rothbard, 2009, p.15).
\parencites[][]{herbener_introduction_2011}[][pp.480–485]{mises_human_1998}[][p.15]{rothbard_man_2009}. %
 Perhaps it is owing to the deployment of this conception that the monarchy theory may still claim an aprioristic status at least in part, as proposed by Gordon? Not really. For even if it can be known \textit{a~priori} that one's time preference must always be positive, it does not entail that the factors shaping the degree of time preference can also be discovered that way 
%\label{ref:RNDFb87Vi8rVF}(Block, Barnett and Salerno, 2006).
\parencite[][]{block_relationship_2006}. %
 Hoppe, recall, holds that one such factor is ownership, with private property fostering lower time preference, and public property inducing higher time preference. Realistic as this may sound, it is demonstrably short of an \textit{a~priori} proposition. To exemplify, imagine a~fellow, let us call him Paul, who seems to be a~man of contradictions. On the one hand, he is a~veritable spendthrift: in a~small accounting proprietorship that he owns, Paul always tries to work as little as possible, and all income he derives disappears within a~week, spent on whiskey, drugs, and women. Needless to say, no savings at all are made. On the other hand, Paul simultaneously happens to be a~card-carrying communist. As a~party member, he is anything but unreliable: he serves as the party's treasurer, and in this capacity he proves as pennywise as it gets. In actuality, Paul's attitude is not so contradictory: he simply resents capitalism so much that saving money earned as a~wicked petty bourgeois is the last thing he is interested in doing. When the longed-for revolution finally comes, Paul starts working himself to the bone for his recently collectivized company, and his years of drinking and womanizing are gone. Currently, Paul is saving half his pay so that the Party may one day inherit it.



What Paul's example evinces is, first, that it might indeed be the case that public ownership will lower one's time preference while private ownership will increase it, and second, that the direction in which one's time preference changes in response to a~change in ownership arrangements is contingent upon one's goals. Paul is a~communist altruist, so to him the time discount on the same good is higher when it is owned by himself and lower when it belongs to a~communist state. In contrast, Hoppe's politicians are rational egoists. Their time preference goes down when goods are theirs and up when they are owned publicly not because any praxeological law so dictates, but because they are self-interested. Once again, then, Hoppe's argument could not get off the ground if certain assumptions regarding human preferences were not made.



With that in mind, we are poised to demonstrate why Hoppe's conceptualizations of political systems should be viewed not as praxeological theories but as Weberian ideal types. Mises, who adopted this tool as well, insisted, however, that the use of the ideal type be restricted exclusively to the domain of history 
%\label{ref:RND4jPsFo05Qq}(Mises, 1998, pp.59–64; 2007, pp.315–322).
\parencites[][pp.59–64]{mises_human_1998}[][pp.315–322]{mises_theory_2007}. %
 Now there are various divergent positions in the literature on the nature and functions of ideal types 
%\label{ref:RNDTChHi1HU3v}(see Kuniński, 1980, pp.35–119).
\parencite[see][pp.35–119]{kuninski_myslenie_1980}. %
 What is nevertheless not up for the debate is that ideal types are built upon ``the one-sided accentuation of one or more points of view and by the synthesis of a~great many diffuse, discrete, more or less present and occasionally absent concrete individual phenomena'' 
%\label{ref:RNDQVMBGJZWQc}(Weber, 1949, p.90; see also Mises, 2007, pp.315–320).
\parencites[][p.90]{weber_methodology_1949}[see also][pp.315–320]{mises_theory_2007}. %
 The role of this procedure is, among other things, to help the researcher make sense of the infinitely complex reality of human action 
%\label{ref:RNDSULW9MspsQ}(Mises, 2007, p.320).
\parencite[][p.320]{mises_theory_2007}. %
 A~good example of how the ideal type works might be Weber's 
%\label{ref:RNDaMiziomIs1}(2001)
\parencite*[][]{weber_protestant_2001} %
 famous model of the Protestant, who, hoping that earthly success will prove evidential of his being predestined for salvation, is driven by the stringent precepts of labor ethics. Thus, of all urges that could possibly influence the Protestant's actions only one is picked and brought to the extreme. The same goes for Hoppe's model of monarch, which predicts that the ruler will act solely and consistently on the motive of personal enrichment. The way Hoppe makes use of the ideal type is nonetheless different from how their role was viewed by Mises. It is then worthwhile to take a~closer look at where those thinkers part company so as to better appreciate the specificity of Hoppe's position.



Crucially, Mises did not attribute to the ideal type nomothetic qualities, let alone the standing of an incontestable \textit{a~priori} truth. He chided Weber for his treatment of the laws of economics as ideal-typological simplifications, which shows that Mises regarded praxeological theories and ideal types as two distinct methodological categories 
%\label{ref:RNDdCyb6zFSFC}(Mises, 2003, pp.79–98).
\parencite[][pp.79–98]{mises_epistemological_2003}. %
 Moreover, based on his dichotomous division between theory and history, Mises unequivocally saw the ideal type as an instrument of the latter. Hence the requirement that ideal types be historically concrete, so as to capture the workings of a~given historical situation 
%\label{ref:RNDMpDV2goGtR}(Mises, 1998, p.62).
\parencite[][p.62]{mises_human_1998}.%




Unlike Mises, Hoppe is not consistent in dividing social science knowledge exhaustively into ``generalizing'' theory (praxeology) and ``individualizing'' history. In his early, German-language methodological treatise, between these two groups of disciplines, there emerges a~third field: sociology, which encompasses ``generalizing'' (although not \textit{a~priori}) explanations of historical facts in the form of articulated theories, both middle-range and grand. While informed and constrained by praxeology, sociological investigations contain the element of looser, non-apodictic speculations, a~part of which comes down to the explicit employment of fallible, substantive assumptions with respect to the goals or preferences of agents 
%\label{ref:RNDeB6wqcZHPX}(Hoppe, 1983, pp.33–38).
\parencite[][pp.33–38]{hoppe_kritik_1983}. %
 In brief, inquiries of this kind differ from Mises's history in that they operate at a~higher level of generality, and therefore do not pay heed to the postulate of historical concreteness. A~general, comparative analysis of monarchy and democracy could then be classified as belonging in this intermediary realm.\footnote{That is exactly how Hoppe designates his intellectual project in \textit{Democracy}… However, the meaning he attaches to the term is somewhat different this time. Hoppe writes: ``I wish to promote and contribute to the tradition of grand social theory, encompassing political economy, political philosophy and history and including normative as well as positive questions. An appropriate term for this sort of intellectual endeavor would seem to be sociology'' 
%\label{ref:RNDZUv3iG4Nks}(Hoppe, 2007a, p.xxiv).
\parencite[][p.xxiv]{hoppe_democracy_2007}. %
 The terminological confusion is compounded when in his other methodological piece, Hoppe 
%\label{ref:RND1n0LU5xuxQ}(Hoppe, 2007b, p.43)
\parencite[][p.43]{hoppe_economic_2007} %
 adopts Mises' bipartite division of social science into (praxeological) theory and history without mentioning sociology as a~distinct discipline. Be that as it may, ``sociology,'' as opposed to ``history,'' seems to be an accurate label for a \textit{theoretical} and generalizing inquiry that at the same time does not fall within the remit of praxeology 
%\label{ref:RNDYNhI2nhPpV}(on the standing of sociology in the Misesian tradition, see Robitaille, 2019).
\parencite[on the standing of sociology in the Misesian tradition, see][]{robitaille_ludwig_2019}.%
}



As will be seen, the limitations of Hoppe's theory of monarchy stem in no small part exactly from the excessive tendency toward simplification, or stated more precisely: from the mentioned overemphasis placed on monetary gain in explaining the actions of monarchs. For although, as Kenneth Waltz 
%\label{ref:RNDMepRvY35kh}(1979, chap. 1)
\parencite*[][chap. 1]{waltz_theory_1979} %
 keenly observes, to theorize is essentially to simplify, it does not follow from this premise that simplifying is always justified to the point of trimming a~theory to a~single explanatory variable. Praxeology is arguably capable of doing so without incurring any cognitive loss thanks to commencing with a~single yet unassailable axiom stating that humans act, coupled with a~few uncontroversial, auxiliary assumptions such as the disutility of labor 
%\label{ref:RNDTRLdMq09oN}(Rothbard, 2009, chap. 1).
\parencite[][chap. 1]{rothbard_man_2009}. %
 An ideal type, on the other hand, lacks this sort of ultimate grounding, so the choice of but one action motive as a~starting point always runs the risk of throwing out with the bathwater of futile minuteness the baby of adequacy.



Another upshot of classifying the Hoppean monarchy theory as based on a~sociological ideal type rather than a~praxeological \textit{a~priori} theorem is that as a~specimen of the former, it must not be confused with what Mises refers to as catallactic functions, i.e., ``distinct functions in the market operations'' such as entrepreneur, capitalist, landowner, or laborer 
%\label{ref:RNDcfhL2Vqygt}(Mises, 1998, p.252).
\parencite[][p.252]{mises_human_1998}. %
 As will be shown below, the understanding that underlies Hoppe's ideal-typological model of monarchy suffers precisely from this confusion.



\section{Politician, not capitalist landowner}

Having established the ideal-typological and sociological nature of Hoppe's conception, we are now in a~position to subject it to critical scrutiny. First and foremost, serious suspicions are raised by the very fact of underscoring the pursuit of pecuniary profit as the key driver of politics. In effect, Hoppe proceeds with his argument as if the concept of the monarch exemplifies at least in part the catallactic functions of the capitalist or the landowner. It is only by doing so that Hoppe can claim that ``elementary theoretical insights regarding the nature of private property and ownership versus ‘public' property and administration'' suffice to ground his theory of monarchy. Mises 
%\label{ref:RNDzuvucQ1cgH}(1998, p.255)
\parencite*[][p.255]{mises_human_1998} %
 defines the said functions in the following way:



\begin{quote}
Capitalist and landowner mean acting man in regard to the changes in value and price which, even with all the market data remaining equal, are brought about by the mere passing of time as a~consequence of the different valuation of present goods and of future goods.
\end{quote}



However, as Mises 
%\label{ref:RNDa05tHyz0nF}(1998, p.254)
\parencite*[][p.254]{mises_human_1998} %
 makes clear, all landowners and capitalists are at the same time entrepreneurs (i.e., actors facing uncertainty), which in the realities of the free-market economy necessitates the adjustment to the ever-changing preferences of consumers 
%\label{ref:RNDV8abRYhdft}(1998, pp.270–272).
\parencite*[][pp.270–272]{mises_human_1998}. %
 The monarch, on the other hand, is not a~capitalist or a~landowner in the sense explained by Mises. The latter two, as catallactic functions, belong in the free market economy 
%\label{ref:RND1hfAOLCBU6}(Mises, 1998, pp.252–256).
\parencite[][pp.252–256]{mises_human_1998}. %
 The free market or capitalism is, under Hoppe's 
%\label{ref:RNDCVaW7dEGZR}(2016, p.20; cf. Rothbard, 2009, p.92; 2011, p.320)
\parencites*[][p.20]{hoppe_theory_2016}[cf.][p.92]{rothbard_man_2009}[][p.320]{rothbard_wall_2011} %
 own definition, a~system based not on any old private property but precisely on titles derived from original appropriation, contracts, and subsequent production. By contrast, the state, whatever its form, exists only in contradiction to the acts of homesteading and contracting 
%\label{ref:RNDBuipu3On7L}(Hoppe, 2016, pp.49–52; Rothbard, 2009, p.877).
\parencites[][pp.49–52]{hoppe_theory_2016}[][p.877]{rothbard_man_2009}. %
 Thus, there can be no ``free market of states,'' wherein ``capitalists'' (kings and princes) could exchange and invest their wealth expecting positive returns that result from consumers' satisfaction. Such a~notion constitutes a~contradiction in terms on the grounds of Hoppe's own systematic commitments. Strictly speaking, in Franz Oppenheimer's 
%\label{ref:RNDuqWKSnEDtP}(1922, p.25)
\parencite*[][p.25]{oppenheimer_state_1922} %
 sense, monarchs are not economically active at all. Rather, all their operations are of a~political nature, i.e., derive income on a~coercive basis, which alters considerably the incentive structure kings are affected by.\footnote{Hoppe's definitions are evidently embedded in the political philosophy of Rothbardian libertarianism he advances. A~public choice economist would employ a~different terminology, perhaps one implying no such categorical distinction between voluntary market actions on the one hand and coercive government undertakings on the other. Nevertheless, despite this divergence, it is unambiguously clear for both Austro-libertarians and public-choicers that the free market and the state generate very different incentive structures. Which is precisely what Hoppe's argument obfuscates. }



Quite revealingly, elsewhere Hoppe 
%\label{ref:RNDHHpzR3BT6W}(2016, p.192)
\parencite*[][p.192]{hoppe_theory_2016} %
 names three reasons capitalism---i.e., the system founded upon respect for private property---proves more efficient than regimes of public ownership:



\begin{quote}
First, only capitalism can rationally, i.e., in terms of consumer evaluations, allocate means of production; second, only capitalism can ensure that, with the quality of the people and the allocation of resources being given, the quality of the output produced reaches its optimal level as judged again in terms of consumer evaluations; and third, assuming a~given allocation of production factors and quality of output, and judged again in terms of consumer evaluations, only a~market system can guarantee that the value of production factors is efficiently conserved over time.
\end{quote}



Observe now that these three advantages are not separate from one another but logically interconnected. Specifically, they all have to do with the supply side being dependent on the demand side. In sharp contradistinction, states, monarchies included, develop contrary to demand, owing to taxation and monopolization 
%\label{ref:RND4Leeihqb3f}(Hoppe, 2006, pp.49–52).
\parencite[][pp.49–52]{hoppe_economics_2006}. %
 What makes economic calculation, resulting in the means of production being allocated in the most effective fashion, both requisite and possible is the necessity of satisfying consumers' preferences. As brought home for us by Mises 
%\label{ref:RNDiPY1uvzhmV}(2012),
\parencite*[][]{mises_economic_2012}, %
 the problem is most vivid in the socialist economy, whereby central planners are in the dark when trying to decide what to produce and how. Nonetheless, even public enterprises operating in a~free-market environment still prove incapable of allocating resources efficiently, for being freed from the pressure of consumers' caprices, they simply do not need to do so 
%\label{ref:RNDYfZpIHXgvo}(Mises, 1944; Rothbard, 2009, pp.952–953).
\parencites[][]{mises_bureaucracy_1944}[][pp.952–953]{rothbard_man_2009}. %
 By the same token, were it not for that pressure, there would be no need for producers to seek the highest quality of output. In a~word, the source of capitalisms' efficiency is that the producer has to serve the consumer. And because that is not the monarch's occupation, neither of the above factors is at work in his case. Not surprisingly, Hoppe does not mention them either.



Things get somewhat more complicated when it comes to the notion of value conservation in time. Plainly, it is this element that undergirds Hoppe's time-preference-based monarchy theory. Indeed, one need not be a~demand-responsive entrepreneur to have a~vested interest in preserving the value of his estate, even if to say so is a~well-reasoned generalization made under the assumption of self-interest rather than an \textit{a~priori} proposition. Still, since all value is subjective, what counts as the long-term value of a~resource depends on who does the valuing. And it goes without saying that with the monarchical state, it is done not by the willing consumers but the ruler himself. It is true that, as Hoppe 
%\label{ref:RNDYtNZIT557P}(2007a, p.18)
\parencite*[][p.18]{hoppe_democracy_2007} %
 points out, rulers do trade their estates between one another every now and then. They nevertheless do so within a~political, not an economic (again in Oppenheimer's sense) structure. The factors that might add to the value of an area in the eyes of monarchs are therefore likewise political. What matters is not only the economic capacity but also strategic localization, significance for dynastic alignments, fortifications, manpower available for the military, and the like. The quality of the economy is certainly of paramount importance, yet it is only one among other relevant variables. This also explains why monarchs trading their estates is a~rather rare occurrence. After all, other monarchs---the potential buyers---are politicians as well, so they can use the purchase against the seller up to the point of wiping his kingdom or duchy off the face of the earth.



This, of course, does not invalidate Hoppe's analysis completely as long as one takes it to be what it really is, i.e., as an exploration of the consequences of the lower time preference of monarchs as compared to democratic leaders that comes with the hereditary claim to power. Needless to say, there is no need to remove the economy from the calculations of kings and princes. At the end of the day, the economic capacity of their country is one of the things all statesmen should care about, not least because it contributes predominantly--- as an element of so-called latent power\textit{---}to the military and diplomatic potential of the state 
%\label{ref:RNDzdMsx9Ri3P}(Mearsheimer, 2001, chap. 3).
\parencite[][chap. 3]{mearsheimer_tragedy_2001}. %
 The point is to see things in the right proportions, neither ignoring nor overemphasizing the role of the pecuniary profit factor in politicians' calculations.



\section{Applications: power and war}

With the epistemological status of Hoppe's monarchy theory explained, let me spend the remainder of this essay trying to improve on his approach in a~fashion avoiding the one-sidedness Hoppe himself fell prey to. What shall be done below is a~humble attempt at outlining an ideal type of monarch that, first, takes seriously the motive of power-seeking and, second, acknowledges those implications of the monarchical private ownership in government which, overlooked in Hoppe's original analysis, make monarchy not as benign as advertised.



To start with, whatever the importance of the economy, it is safe to assume that the lust for power for its own sake can rank as a~motive for political activity at least as strong as financial gain 
%\label{ref:RNDuh5SxS2AaK}(Nowakowski, 2010, p.272).
\parencite[][p.272]{nowakowski_dlaczego_2010}. %
 In fact, that is precisely what Austro-libertarians, Hoppe included, normally posit when investigating the nature of the state in the context of class analysis 
%\label{ref:RNDgkeOWQXhX1}(Hoppe, 2006, pp.117–138; Rothbard, 2000, pp.55–88).
\parencites[][pp.117–138]{hoppe_economics_2006}[][pp.55–88]{rothbard_egalitarianism_2000}. %
 As Sierpiński 
%\label{ref:RNDSPRzsL8Iiy}(2016, p.557)
\parencite*[][p.557]{sierpinski_critica_2016} %
 pointedly argues, whether supporting the prosperity of his people is actually desirable for the maintenance of the king's power is far from obvious. That the opposite will turn out to be the case is particularly likely in economically backward monarchies. For as Tocqueville 
%\label{ref:RNDuXhGtOj5Gg}(1955)
\parencite*[][]{tocqueville_old_1955} %
 famously noted, governments face the highest risk of revolution not when they are the most repressive, but when they begin to reform.



Furthermore, even a~monarch driven chiefly by financial aspirations can largely satisfy his craving regardless of the nation's economic condition. Examples of breathtakingly opulent dictators ruling more or less underdeveloped, and sometimes downright devastated countries abound, with the contemporary cases of Idi Amin, Mobutu Sese Seko, Vladimir Putin, or Kim Jong Un readily springing to mind. This is due to the already mentioned fact that the monarch, not being a~capitalist landowner but a~politician, can extract income from his property irrespective of the evaluation of his services on the part of his subjects.



Moreover, what matters are not only the motives of politicians but also institutions. Modern democracy, to invoke the well-known definition by Joseph Schumpeter 
%\label{ref:RNDd4LQPEZlOp}(2006, p.269),
\parencite*[][p.269]{schumpeter_capitalism_2006}, %
 is ``that institutional arrangement for arriving at political decisions in which individuals acquire the power to decide by means of a~competitive struggle for the people's vote.'' If ``competitive'' is to denote open entry, as it in fact does in contemporary democratic theory, it entails logically (even if not necessarily in practice) the guarantee of certain political freedoms (and related property rights) such as freedom of association or freedom of speech 
%\label{ref:RNDg3kZEXf81r}(Dahl, 1989; Tilly, 2007, pp.13–14).
\parencites[][]{dahl_democracy_1989}[][pp.13–14]{tilly_democracy_2007}.%
\footnote{Since the job of the present section is ideal-typological modeling, it is plain that such purely analytic truths about democracy and monarchy must be taken into account. It is, of course, another question whether those truths---and the ideal types built upon them---are successful in elucidating real-world facts and processes, that is, whether existing democracies live up to the ideals enshrined in their definition. As mentioned, to give a~full empirical account of the applicability of my suggestions would require a~separate study. Still, it is worth noting that unlike most absolute monarchies past and present, democracies (at least the Schumpeterian ones, consisting in free elections and universal suffrage) do refrain from instituting censorship. Although more subtle means of free speech suppression are deployed here and there, e.g., through the pressure exerted on Big Tech, and certain views considered extreme or totalitarian (particularly Nazism and fascism, though not necessarily communism) tend to be outlawed, the restrictions are always short of full-fledged, institutionalized censorship, which remains an anathema. Note that what I~am referring to is the type of democracy described by Schumpeter and Hoppe himself, at least when the latter author talks about the deleterious ramifications of popular voting. In Hoppe, one can also discern a~broader notion of democracy, which encompasses any system of public property in government whether free and fair elections take place or not 
%\label{ref:RNDJuKE4j3Y1H}(Nowakowski, 2010).
\parencite[][]{nowakowski_dlaczego_2010}. %
 Of course, there are regimes that satisfy this definition (most notably Nazism and communism) to which neither Schumpeter's definition nor my argument applies. } The opposite is true of monarchy. Indeed, maintaining the monopoly of power requires diminishing this exact freedom. It is then no coincidence that absolute monarchies, historical and contemporary alike, often impose some kind of censorship.\footnote{Admittedly, as noted by Henshall 
%\label{ref:RNDlEJ7sHspWs}(2013, pp.114–117),
\parencite*[][pp.114–117]{henshall_myth_2013}, %
 censorship in early-modern absolute monarchies was not as stringent as the commonplace narrative has it. Some of Rousseau's subversive books or Diderot's and d'Alembert's \textit{Encyclopédie} got through in France, while England was keeping in force libel and sedition laws that heavily diminished freedom of press long after the Glorious Revolution of 1688. On the other hand, in the wake of the post-Vienna Congress restauration, censorship took hold of most European nations 
%\label{ref:RND0vTHrp4jb0}(Henshall, 2013, p.208),
\parencite[][p.208]{henshall_myth_2013}, %
 which suggests that monarchs are inclined to impose it when they fear that their position might be threatened. Given the power of today's media and the level of social pluralism, that would presumably be their policy in contemporary Europe as long as the prince's authority were to be safeguarded. Not surprisingly, present-day monarchies outside the West also maintain pretty strict censorship. } The same applies to the question of checks and balances. However imperfect the mechanisms built in the constitutional systems of modern democracies might be, they at least exist. Whereas absolute monarchy, by assumption, seeks to do away with all such institutions altogether. The tacit conclusions of Hoppe's monarchy theory, on the other hand, turn the tenets of classical liberalism on their heads: if monarchy is better than democracy because it represents private property government, then---as Hoppe 
%\label{ref:RND4S2hNYN2r8}(2007a, p.18, f.9)
\parencite*[][f.9]{hoppe_democracy_2007} %
 himself asserts--- absolute monarchy is monarchy at its finest. Therefore, it proves preferable not only to democracy but also to any form of monarchy or mixed government that does implement checks and balances, since to do so is, by definition, to abridge the ownership rights of the monarch. That is to say, no constraints at all is allegedly better than weak constraints. Again, that Hoppe overlooks this objection may be explained by the absence of a~specifically political analysis in his theory. No one needs free speech guaranties, separation of powers, or checks and balances on a~ranch.\footnote{The complete neglect of the free speech question in Hoppe's 
%\label{ref:RNDUUNf3WZXon}(2007a, pp.50–62)
\parencite*[][pp.50–62]{hoppe_democracy_2007} %
 comparison of the historical achievements of monarchies and democracies is indeed quite startling. Are taxes and inflation really everything that matters for a~libertarian? }



This criticism seems even more formidable than the ones previously raised. For what those arguments testified to were only certain limitations of Hoppe's claim that monarchy means a~longer planning horizon, which in turn means a~lower level of property rights violations. They did not undermine this contention \textit{per se}, if only down to the importance of economic development for the relative strength of a~state in its relations with other states. In short, the reasoning so far has shown why a~monarch can rob and enslave \textit{despite} being a~monarch. What the argument from political freedoms and separation of powers demonstrates is, on the other hand, that in some respects a~monarch is indeed more dangerous than a~democratic caretaker precisely \textit{because} he is a~monarch (as Hoppe portrays him). It entails, moreover, that monarchs can be expected to infringe upon private property rights in departments such as the suppression of free speech to the \textit{greater} extent, the \textit{longer} their planning horizon is and the \textit{more} conscientious owners they are. A~king diverted from the pursuit of his dynasty's interests by sheer naivete or sincere devotion to libertarian principles can allow for free competition between political ideas, including those calling for his own overthrow; one preoccupied solely with the prosperity of his family business cannot.



The private character of monarchical ownership may incentivize rulers to engage in aggressive behavior in yet another way, to wit, by aggravating conflicts over power. True, in democracies, politicians do kill one another or, worse still, wage civil wars when incapable of seizing power in the wake of an election. However, such hostilities are usually fueled not by the narrow self-interest of politicians but rather by ethnic, religious, or ideological tensions 
%\label{ref:RNDh3KYICbAwV}(Megger, 2018).
\parencite[][]{megger_krytyczna_2018}. %
 For monarchs, violent games of thrones all too often become part and parcel of their profession. The reason is simple: the bigger the reward, the greater the lengths one is willing to go to in order to get it. He who recoils at the idea of slaughtering his rival (who may at the same time happen to be his friend or brother) for four or five years term in a~democratic office need not be that much appalled by the prospect of doing the same should that mean winning a~vast estate to be enjoyed by him and his posterity for centuries to come.



Relatedly, private ownership in government generates incentives for interstate bellicosity. True enough, the prospect of drawing income from the conquered economy should prevent the ruler from wrecking it in the course of military actions, and the lack of democratic-nationalist legitimation of power weakens the case for conscription. On the other hand, those very same factors increase the likelihood of waging war in the first place. First off, building empires or creating and protecting zones of influence are typically long-term projects that require relatively low time preference. Their time horizon usually exceeds the term of the democratic politician. Hence, the king is more likely to pursue such policies than the president. On top of that, newly captured lands offer an opportunity for additional profit for the monarch and his kin, which again makes him more likely to seek territorial acquisitions than his democratic counterpart, who has no business fighting for spoils that will come about when his term has long been over 
%\label{ref:RND4YSNR5dw99}(Mises, 1985, p.121).
\parencite[][p.121]{mises_liberalism_1985}. %
 It is for a~reason that Austro-libertarians standardly invoke the interests of the deep state when explaining imperialism of democratic nations such as the USA 
%\label{ref:RNDzWnTT9mxqH}(Rothbard, 2011; Hoppe, 2006, pp.77–116).
\parencites[][]{rothbard_wall_2011}[][pp.77–116]{hoppe_economics_2006}. %
 After all, bankers and industrial-military complex people are private owners with the right to bequeath, so their time preference can accordingly be expected to be comparatively low, just as that of kings. Furthermore, the ideology of the democratic nation-state mitigates imperialism in that it imposes non-negligible limits on the policies of conquest: the size of the state ought to be congruent with the borders of a~nation's settlement 
%\label{ref:RND9D7Qtery62}(Gellner, 1993, p.1; Mises, 1985, p.118).
\parencites[][p.1]{gellner_nations_1993}[][p.118]{mises_liberalism_1985}. %
 By contrast, monarchy, as a~regime whose cosmopolitanism stems from legitimation coming from the ruler's rightful claim rather than from popular (national) approval, faces no such constraints. The monarch, uninterested in who it is that he is going to reign over, can in principle seize whatever territory he finds attractive. Yet again, that Hoppe disregards this stems from his overemphasizing the economic value-conservation factor, while downplaying the political value-destruction factor.\footnote{In the literature, one may find far more arguments to the effect that democracy produces peaceability rather than belligerence, known under the umbrella name of democratic peace theory. Many of those claims are highly debatable, though (see a~plausible critique by Hoppe
%  \label{ref:RNDBdnWQgVZ1k}(2021, pp.232–237)
 \parencite*[][pp.232–237]{hoppe_great_2021} as well as a~recent realist discussion of democratic peace theories in
% \label{ref:RNDrT2VT5Ebsd}(Mearsheimer, 2018)).
 \parencite*{mearsheimer_great_2018}).
 The empirical record does not seem conclusive either. Although earlier research
% \label{ref:RNDhNbiNK0EMw}(Pinker, 2012; Rummel, 1983)
 \parencites{pinker_better_2012}{rummel_libertarianism_1983}
 seemed to support theories of democratic peace, Cirillo and Taleb recently
% \label{ref:RND3Z45Xo2zl8}(2015)
 \parencite*{cirillo_statistical_2015}
 argued that no such regularities might really be observed. At any rate, Hoppe's anti-democratic and pro-monarchical conclusions do not find confirmation.}



\section{Conclusion}

Hoppe's comparative analysis of monarchy and democracy undoubtedly furnishes a~number of original, intriguing, and thought-provoking vistas. Moreover, the very attempt to apply Austrian economics to problems typically penetrated by political scientists and public choice scholars is in and of itself worth the price of admission. It transpires, however, that the theory ultimately falls short of substantiating one of its central claims, namely, that monarchy, although beset by all sorts of evils inherent for all states, is after all a~preferable political setting insofar as property rights protection is concerned. The foregoing investigations have challenged this thesis chiefly from the methodological angle. More specifically, it has been demonstrated that rather than a~body of \textit{a~priori,} praxeological truths, Hoppe's monarchy theory contains an ideal type of the politician driven by self-interest and bestowed with an inheritable title to the throne. Yet, the analysis under criticism proceeds as though it did fall within the scope of praxeology, or even catallactics, which it does in mistakenly depicting monarchy as a~capitalist landowning enterprise of sorts. This article, on the other hand, argued that as a~sociological endeavor, explorations of political regimes have to take account of a~broader set of preferences ascribable to monarchs---i.e., a~broader concept of their self-interest---than bare monetary profit seeking. Furthermore, even when strictly profit-oriented, monarchs can still be inclined to pursue policies of aggression and parasitism. First, as statesmen and not capitalist landowners, they operate within a~legal framework that allows them to derive profit on a~coercive basis, which undermines the causal relation between the quality of the economy and the wealth of the king himself. More important still, the hereditary nature of their claim incentivizes monarchs to take greater pains---as compared to democratic politicians---in striving for power, which also involves more extensive use of political means. This does not mean that Hoppe's theory offers no benefits for our understanding of political systems and historical processes brought about by their succession. All in all, one takeaway from our inquiry is that it does matter whether a~politician can expect a~lifetime sitting on the throne for him and his descendants, or just a~few untransferable terms in office. However, an adequate analysis of political systems needs to grasp far more ramifications of this fact than Hoppe's own work does. Moreover, those ramifications, when judged from the liberal-libertarian vantage point, turn out to be ambiguous: some are conducive to property rights and free markets, some are not. Consequently, one is not warranted in concluding that monarchy surpasses democracy in preserving institutions cherished by free-marketeers as keys to freedom and prosperity. What we are left with is a~much more nuanced picture.
\enlargethispage{1.5\baselineskip}








\end{artengenv}

\label{slenzok-last}
\begin{artengenv}{Dawid Megger}
	{Demonstrated preference in the Austrian economic analysis}
	{Demonstrated preference in the Austrian economic analysis}
	{Demonstrated preference in the Austrian economic analysis}
	{Nicolaus Copernicus University in Toruń\label{megger-first}}
	{This paper is an attempt to clarify the concept of demonstrated preference in the economic analysis of the Austrian School of Economics. It considers several interpretations of this concept: (1) as a~thymological concept which matters in empirical interpretations of concrete human actions; (2) as a~preference expressed in voluntary actions; (3) as the only existing preference; (4) as the only preference which matters in economics; (5) as the only preference which matters in the economy. It is argued that despite the Austrian insistence on (4), the only interpretation resistant to criticism is (5). Unfortunately, it is not sufficient to draw or reinforce the conclusions that Murray N. Rothbard and his followers reach in their considerations on welfare economics, monopoly theory, public goods theory, and game theory. A~number of additional clarifications are also made (e.g., the concepts of ``voluntariness'' and ``psychologizing'' in economics).
	}
	{demonstrated preference; revealed preference; Murray Newton Rothbard; Austrian School of Economics.}








\section{Introduction}

\lettrine[loversize=0.13,lines=2,lraise=-0.03,nindent=0em,findent=0.2pt]%
{T}{}he concept of demonstrated preference (henceforth: DP), named and formulated in 1956 by Murray Newton Rothbard in his essay \textit{Toward a~Reconstruction of Utility and Welfare Economics} 
%\label{ref:RND4dKVflFkMW}(Rothbard, 2011b),
\parencite[][]{rothbard_present_2011}, %
 had a~significant impact on the later research practice of some representatives of the Austrian School of Economics (henceforth: ASE). It allowed Rothbard to draw radical conclusions in welfare economics, monopoly theory, or public goods theory 
%\label{ref:RNDqcMUwJjy2h}(cf. Rothbard, 2009a; 2009b; 2011b).
\parencites[cf.][]{rothbard_man_2009}[][]{rothbard_power_2009}[][]{rothbard_present_2011}. %
 Rothbard's arguments in these areas were adopted by other Austrians 
%\label{ref:RNDlcYtk5IiHh}(e.g., Block, 1983; Herbener, 2008; Hoppe, 1989; 2006; Wiśniewski, 2018).
\parencites[e.g.,][]{block_public_1983}[][]{herbener_defense_2008}[][]{hoppe_theory_1989}[][]{hoppe_economics_2006}[][]{wisniewski_economics_2018}. %
 Roughly speaking, it can be said that according to DP, in economics only those preferences that are demonstrated in the actions of individuals matter. Criticism of the concept 
%\label{ref:RNDn68sMyqofX}(e.g. Nozick, 1977; Caplan, 1999; Kvasnička, 2008),
\parencites[e.g.][]{nozick_austrian_1977}[][]{caplan_austrian_1999}[][]{kvasnicka_rothbards_2008}, %
 and the need for its subsequent defence 
%\label{ref:RNDjDeEz7QZpj}(e.g., Block, 1999; Wysocki, Block and Dominiak, 2019; Gordon, 2022),
\parencites[e.g.,][]{block_austrian_1999}[][]{wysocki_rothbards_2019}[][]{gordon_misunderstanding_2022}, %
 however, suggest that there are some ambiguities related to DP.



The research problem I~take up in this article is: what does it mean that in economics only those preferences that are being demonstrated in actions matter? My goal is to present a~systematic interpretation of DP. This will allow me to fill the research gap in the indicated area and solve an important problem regarding the research practice of some contemporary representatives of ASE. I~will try to solve the problem by using the principle of charity, which involves critiquing certain ideas in their strongest possible variants. In the course of my research, I~will primarily resort to logical analysis and hermeneutics. I~will try to account for the thesis that the most convincing interpretation of DP is not sufficient to draw the conclusions reached by Rothbard and his followers in the field of welfare economics, monopoly theory, public goods theory, or game theory.



In section 2, I~illuminate the Rothbardian understanding of DP, as laid out by Rothbard. In section 3, I~present DP as a~thymological concept and argue for rejecting this interpretation. In section 4, I~consider whether preferences can be demonstrated in coerced actions and show the consequences of an affirmative and a~negative answer to this question. In section 5, I~turn to an interpretation of DP that seems to best fit the intentions of the Austrians (that only demonstrated preferences matter in economics). I~consider this in two variants: strong, easily exposed to criticism (5.1), and weak, more resistant to criticism (5.2). However, I~submit that both interpretations have similar practical consequences and it is difficult to find a~convincing argument for their acceptance. In section 6, I~present what I~think is the most appealing understanding of DP (that only demonstrated preferences influence the social processes). Section 7 briefly concludes.



\section{Demonstrated preference---an outline of the concept}

Rothbard sees traces of DP in the writings of William Stanley Jevons, Irving Fisher, Frank Fetter, and Ludwig von Mises 
%\label{ref:RNDPwRQSLAwT5}(Rothbard, 2011b, p.290).
\parencite[][p.290]{rothbard_present_2011}. %
 Since my considerations are not of a~historical nature, I~will make Rothbard's formulation the starting point for further analysis. The reason for this is that it is his formulation that is at the core of his and his followers' research practice.



Rothbard's motivation for formulating DP is the belief that economics is a~science dealing with human action. Human action, in turn, is a~purposeful behaviour, or a~conscious pursuit of chosen goals by definite means. The main idea behind DP for Rothbard is the belief that action expresses the preferences of the person undertaking it. Or, in his words: ``The concept of demonstrated preference is simply this: that actual choice reveals, or demonstrates, a~man's preferences; that is, that his preferences are deducible from what he has chosen in action'' 
%\label{ref:RNDPa8Rkcv3b1}(Rothbard, 2011b, p.290).
\parencite[][p.290]{rothbard_present_2011}.%




Later in his article, Rothbard rejects Paul A. Samuelson's concept of revealed preferences because, he argues, it inevitably smuggles in the erroneous assumption of the constancy of preferences over time, while, in accordance with the ASE methodology, preferences of economic actors can be constantly changing 
%\label{ref:RNDoGwV4vz8pU}(Rothbard, 2011b, p.294).
\parencite[][p.294]{rothbard_present_2011}.%




Rothbard also rejects two extreme approaches to human action, which he calls psychologizing and behaviourism, respectively. Psychologizing is supposed to consist of (a) speculating about preferences not demonstrated in actions and (b) dealing with psychology's characteristic consideration of why people have certain preferences and how they are formed (as I~will try to show later, (a) and (b) are not the same, so it is difficult to understand why Rothbard writes about these ideas in the same breath). Behaviourism\footnote{More specifically, we could distinguish between methodological and ontological behaviourism. While the former underlies that referring to mental states (such as preferences) does not constitute scientific explanations, the latter says that mental states (such as beliefs and desires) do not exist at all. Although methodological behaviourists need not subscribe to ontological behaviourism, the latter must be recognized as a~sound basis for the former. Rothbard 
%\label{ref:RNDVj6Pz5Ct4f}(2011b, p.297)
\parencite*[][p.297]{rothbard_present_2011} %
 seems to be criticizing both of them. However, he does not make the distinction between them explicitly.}, on the other hand, consists in completely ignoring the mental dimension of human action (goals, preferences, beliefs) and focusing only on its physical dimension. Contrary to behaviourists, Rothbard points out that mental states (preferences) matter in scientific explanations of human actions. However, contrary to ``psychologists'', he argues that in economics, only those preferences that economic actors demonstrate in actions are relevant. Consequently, in practice, he recognizes that the concept of preference makes no sense apart from the actual action. According to DP: ``economics deals only with preference as demonstrated by real action'' 
%\label{ref:RNDYamv9Cdb2R}(Rothbard, 2011b, p.333).
\parencite[][p.333]{rothbard_present_2011}.%




Before proceeding, one further clarification might be regarded as important. More specifically, and as it happens to be unclear in the Austrian literature, it is worth asking what the concept of preference means? Most of all, we need to note that preference is not a~simple intention or desire. Rather, it expresses a~relation between at least two competing wants. In other words, \textit{preference} assumes the existence of at least two alternatives, one of which is valued more highly than the other. It looks consistent with the following quote from Rothbard: ``If a~man chooses to spend an hour at a~concert rather than a~movie, we deduce that the former was preferred, or ranked higher on his value scale. Similarly, if a~man spends five dollars on a~shirt we deduce that he preferred purchasing the shirt to any other uses he could have found for the money'' 
%\label{ref:RNDCLLGrnKjWP}(Rothbard, 2011b, p.290).
\parencite[][p.290]{rothbard_present_2011}.%




It seems that based on the last sentence, we can infer that according to Rothbard, it is not necessary to know what the second-best alternative is. He simply assumes that there was one. Given actual constraints, an agent chooses the course of action that appears to him best available. A~probable implicit premise in this reasoning is that human action is based on some kind of deliberation over different possible scenarios. Still, what is important in light of DP is not the content of the second-best alternative but that human action (conscious pursuit of ends) presupposes a~choice between alternatives, and that the first alternative is of particular interest to the sciences of human action. As David Gordon 
%\label{ref:RND1jxuB9YoSD}(2022)
\parencite*[][]{gordon_misunderstanding_2022} %
 put it, DP ``means that choice reveals the chooser's highest preference''.



\section{Demonstrated preference as a~thymological concept}

It is sometimes said that Samuelson's concept of revealed preference was intended to give economic theories empirical significance. By observing consumer behaviour, the assumptions and conclusions of economic theory could be confirmed or falsified. For example, if one of the assumptions of the rational choice theory is that the preferences of economic actors are transitive (someone who prefers A~over B~and B~over C~must prefer A~over C) and if empirical research has shown that sometimes their preferences are intransitive, then the assumption of transitivity could be considered falsified (this type of research is often conducted within the framework of so-called behavioural economics). The question is then: is DP also supposed to give economic theories empirical significance? At this point, I~will try to present the consequences of both possible answers to this question.



Let me begin by recalling the radical division between theory and history introduced by Mises and adapted by Rothbard. According to Mises, there are two branches of the sciences of human action: theoretical and historical. Economics, which is a~part of the theoretical sciences of human action (praxeology), is an axiomatic-deductive science, taking as its basis the axiom of human action as a~conscious pursuit of chosen ends. Economic theories and laws are not subject to empirical verification or falsification. The only way to undermine them is to demonstrate that the said theories and laws involve fallacious reasoning 
%\label{ref:RNDGu08IQ9tRb}(cf. Mises, 1998, pp.30–41).
\parencite[cf.][pp.30–41]{mises_human_1998}.%




Mises refers to the science that underlies the historical sciences of human action as thymology. The basic method of thymology is the so-called empathic understanding (or, to use a~term introduced by Max Weber, \textit{Verstehen}). The essence of this method is striving for interpretation of the meaning that other people give to their actions and to natural as well as social phenomena. This method requires the assumption that other people have a~mental structure similar to ours; that they are beings who pursue their own ends by definite means; that they are guided by certain beliefs and desires. Accordingly, if a~literary scholar wants to correctly interpret the work of a~poet, he should get to know his biography, get acquainted with the values he held dear, and deepen his knowledge of the literature he read. If a~historian wants to understand why Henry VIII renounced obedience to the Pope, he should understand his social situation, character traits, and the private and political goals he could have wanted to achieve by entering a~second marriage 
%\label{ref:RNDx3qB6c75PL}(Mises, 2007, pp.264–284).
\parencite[][pp.264–284]{mises_theory_2007}.%




The method of \textit{Verstehen} is also reflected in Austrian subjectivism 
%\label{ref:RNDmlzdy79tSR}(Lachmann, 1971).
\parencite[][]{lachmann_legacy_1971}. %
 The need to refer to this method in empirical research results from the observation that the ``data'' of the social sciences are inherently subjective 
%\label{ref:RNDc5LxReWD8q}(Hayek, 1952).
\parencite[][]{hayek_counter-revolution_1952}.%
Thymological knowledge is never certain. No one has a~direct insight into the minds of other people. Moreover, such data cannot be verified or falsified by any act of measuring external qualities. Thymological interpretations can be made more reliable or plausible by deepening historical, psychological, or sociological knowledge. The humanities and theoretical social sciences can also help here. Historical (thymological) research, however, cannot invalidate the findings of economics (praxeology). They play a~different role: they help to determine when a~given theory can be applied (they can also suggest the direction of development of the theory and be a~source of empirical auxiliary assumptions, cf. e.g., 
%\label{ref:RNDoI7vC6K9p5}(Wiśniewski, 2014)
\parencite[][]{wisniewski_methodology_2014}%
).



Although, as Mises points out, ``thymology has no special relation to praxeology and economics'' 
%\label{ref:RNDD9LBzIPous}(Mises, 2007, p.271),
\parencite[][p.271]{mises_theory_2007}, %
 there is no doubt that he treats these disciplines as complementary to one another. In order to properly interpret historical events, theoretical knowledge derived from economics and praxeology is necessary. In order to find out when a~particular theory is applicable, it is necessary to understand a~specific historical (i.e., taking place in history) situation. One discipline without the other is useless. As Roderick T. Long succinctly puts it: ``Praxeology without thymology is empty; thymology without praxeology is blind'' 
%\label{ref:RNDypJZx3D7pw}(Long, 2008, p.50).
\parencite[][p.50]{long_wittgenstein_2008}.%




At this point, we can restate the initial question: is DP a~praxeological or thymological concept? Is it possible to conclude that, thanks to DP, an observer gets a~certain empirical knowledge about the purposes and preferences of economic actors? By no means. As we have seen, according to Austrians, such knowledge can never be certain. From the mere observation of external manifestations of someone's actions, it is impossible to draw conclusions about their goals, beliefs, or preferences. As, Hans-Hermann Hoppe 
%\label{ref:RND3g2BJAuNwA}(2005),
\parencite*[][]{hoppe_note_2005}, %
 Rothbard's follower, observes, citing John Searle 
%\label{ref:RNDfXHFdrbbZa}(1984, pp.57–58),
\parencite*[][pp.57–58]{searle_minds_1984}, %
 human action has two aspects: external (physical, behavioural) and internal (mental, psychological). There is no doubt that only the physical movements of bodies can be observed, not human goals and preferences. Therefore, in order to fully understand someone's actions, it is also necessary to grasp the mental aspect, i.e., the intentions, beliefs, and desires of a~given person.



Consider the example of a~child being baptized in a~church. The goal of both the priest and the parents of this child is to blot out its original sin and include it in the community of the Church. To an outside observer, who has no knowledge of Christianity, however, it will only be an incomprehensible pouring of water onto the child's head and making the sign of the cross on its forehead. Observation of the mere physical movements of bodies, characteristic of behaviourism, does not allow for a~full description of this event. In order to grasp the essence of social situations, it is necessary to understand the goals and beliefs of the people involved in them. Both Mises and Rothbard reject the behaviourist approach 
%\label{ref:RNDglHD9rJKmv}(cf. Mises, 2007; Rothbard, 2011b).
\parencites[cf.][]{mises_theory_2007}[][]{rothbard_present_2011}.%




Nevertheless, there is no shortage of accusations in the literature that Rothbard's DP is essentially behaviourist 
%\label{ref:RNDFiLEXv5NGR}(e.g., Prychitko, 1993, p.574).
\parencite[e.g.,][p.574]{prychitko_formalism_1993}. %
 Bryan Caplan argues in a~similar vein, noting that external manifestations of actions do not allow us to draw conclusions about the mental states of the individuals undertaking them:
\begin{quote}
Indeed, Rothbard could have taken this principle further. When two people sign a~contract, do they actually demonstrate their preference for the terms of the contract? Perhaps they merely demonstrate their preference for writing their name on the piece of paper in front of them. There is no ironclad proof that putting one's name on a~piece of paper is not a~joke or an effort to improve one's penmanship. 
%\label{ref:RNDXqEz04mGRy}(Caplan, 1999, p.833)
\parencite[][p.833]{caplan_austrian_1999}%
\end{quote}
Therefore, it seems that DP cannot be regarded as a~method of interpreting human actions in empirical reality. Thymological knowledge consists of the proper interpretation of concrete actions. It is based on the method of \textit{Verstehen}. DP does not develop or change anything in this procedure. Accordingly, this concept seems unnecessary at best and misleading at worst in this context. Thus, it seems that the thymological interpretation of DP is also to be rejected.



Furthermore, Rothbard's objection to ``psychologizing'' with regard to ``speculating about preferences not being demonstrated in actions'' also does not seem well-grounded. Rothbard states that non-demonstrated preferences should not be of interest to economics because the assumption of their existence or their specific scale is based on uncertain conjectures 
%\label{ref:RND1N9UizdGaS}(Rothbard, 2011b, pp.296–298).
\parencite[][pp.296–298]{rothbard_present_2011}. %
 But if thymological knowledge can never be certain, does it not mean that the charge of psychologizing can be applied to both ``demonstrated'' and ``non-demonstrated'' preferences? Of course, to say that a~person has a~certain preference, based on the fact that he has undertaken a~definite action, is much better justified than to say that this person has a~preference for something he has never pursued in action (or at least no one has observed it). No external observer, however, can ever say with absolute certainty that the person has demonstrated a~preference for something particular because there is no direct insight into that person's mind (not even that person's self-declaration settles anything---he could simply lie, after all). Knowledge of mental states is not verifiable. Only an acting individual, through introspection, knows with certainty their own intentions and beliefs.



The considerations presented further in the article are based on the assumption that DP is an analytic tool of economic theory (praxeology), not thymology. Acting individuals demonstrate their goals to themselves or, in an imaginary world invented for didactic purposes, to an economic theorist. An economist who wants to explain an economic phenomenon or explain an economic theory can, after all, describe a~scenario in which he assumes that an economic actor has certain preferences (to the economist, both demonstrated and non-demonstrated preferences can be known with certainty). This is what Rothbard himself does, explaining, for example, the law of diminishing marginal utility 
%\label{ref:RNDN3ojodUZgD}(Rothbard, 2009a, pp.21–33).
\parencite[][pp.21–33]{rothbard_man_2009}.%




Summarizing the above considerations, it is worth emphasizing the following relationship; if DP were a~thymological concept, it would have no direct theoretical consequences. However, if it is a~praxeological concept, then it has no direct empirical consequences (it does not ``improve'' thymological interpretations of observed actions of individuals).



\section{Demonstrated preference as a~preference expressed in voluntary actions}

Another interpretation of DP worth considering is the suggestion that individuals demonstrate their preferences only in voluntary actions. According to this interpretation, if someone undertakes an action under coercion, then it cannot be said that he is demonstrating his preference. For example, if a~robber assaults someone and says: ``Give me your money or I~will kill you'', and as a~result, the victim gives up his money, it cannot be said that the victim has demonstrated a~preference for giving the money to the robber.



At first glance, this interpretation is quite appealing. There is no shortage of suggestions in the literature that it is appropriate. Mateusz Machaj 
%\label{ref:RNDs4RfDqCYhC}(2014),
\parencite*[][]{machaj_murray_2014}, %
 for example, agrees with it, stressing that, according to Rothbard, ``voluntary trade relations within the free market increase the welfare of both parties to the transaction'' 
%\label{ref:RNDvsXYCzcYCU}(Machaj, 2014, pp.10–11, own transl.),
\parencite[][own transl.]{machaj_murray_2014}, %
 and then in a~footnote he adds:



\begin{quote}
The shortcomings in Rothbard's theory stem from the fact that the concept of ``demonstrated preference'' implicitly implies that it is a~demonstrated preference with respect to personal property and what the individual owns. And if so, Rothbard's approach is also not ``value free'' because it presupposes some version of ``justly acquired'' property, which is already a~normative concept. For example, a~tax clerk and a~private building administrator, both of whom apply to someone for a~fee, are praxeologically no different from each other until we introduce additional assumptions about the nature of existing property titles.
%\label{ref:RND5c6DNWCTHU}(Machaj, 2014, pp.10–11, own transl.)
\parencite[][own transl.]{machaj_murray_2014}%
\end{quote}




As Machaj suggests above, and as other authors also note 
%\label{ref:RNDF7hEgYMRu3}(e.g., Cordato, 1992),
\parencite[e.g.,][]{cordato_welfare_1992}, %
 by voluntary actions Rothbard and his followers mean such actions in which the property rights of the individuals who undertake them are not violated, whereas the concept of ``justly acquired'' property would be inextricably tied with the libertarian theory of justice. This, in turn, would contradict the value freedom postulate (\textit{Wertfreitheit})\footnote{In the economic literature, the normative entanglement of the concept of voluntariness was noticed, e.g., in 
%\label{ref:RNDlY9ztBVk0O}(High, 1985; Hausman and McPherson, 2006).
\parencites[][]{high_is_1985}[][]{hausman_economic_2006}.%
} accepted by the Austrians 
%\label{ref:RND5qUSRuxvJR}(e.g., Mises, 1998; Rothbard, 2011a; Kirzner, 1994; Block, 2005).
\parencites[e.g.,][]{mises_human_1998}[][]{rothbard_praxeology_2011}[][]{kirzner_value-freedom_1994}[][]{block_value_2005}. %
 According to Machaj, however, this interpretation of DP is necessary to defend Rothbardian welfare economics. Without it, his theory would lose its grounds.



The fact that, in their research practice, Rothbard and his followers, before proceeding with economic analysis, assume the concept of ownership, was expressed by them explicitly.\footnote{The words of Rothbard himself are especially meaningful here: ``an economist cannot fully analyze the exchange structure of the free market without setting forth the theory of property rights, of justice in property, that would have to obtain in a~free-market society'' 
%\label{ref:RNDX3BoG3ccy3}(Block, 1995; 2000; Hülsmann, 2004; Rothbard, 2009b, p.1047; 2009a).
\parencites[][]{block_ethics_1995}[][]{block_private-property_2000}[][]{hulsmann_priori_2004}[][p.1047]{rothbard_power_2009}[][]{rothbard_man_2009}.%
} It is worth noting that, as a~consequence, they treat property rights as exogenous to economic theory and \textit{nolens volens} reject the area of research called economic analysis of law 
%\label{ref:RNDrtsNGkPSej}(see: Machaj, 2014).
\parencite[see:][]{machaj_murray_2014}.%




That the Rothbardians assume the libertarian theory of justice in the concept of voluntariness is especially visible in the considerations of Walter E. Block and David Gordon 
%\label{ref:RNDOQGcsX6Deo}(1985).
\parencite*[][]{block_blackmail_1985}. %
 According to these scholars, the proposition ``Give up your money or I~will destroy your reputation'' would not be a~coercive threat. As absurd as it may seem, it would be a~plain offer, categorically not different from the proposal ``If you give me \$2, I~will give you bread''.



Let us compare the types of offers and threats presented so far.

\medskip

\noindent (1) ``Give me your money or I~will kill you''.



\noindent (2) ``Give me your money or I~will destroy your reputation''.



\noindent (3) ``If you give me \$2, I~will give you bread''.

\medskip

In light of the concept of voluntariness adopted by Rothbard and his successors, it should be said that (according to the interpretation of DP considered in this point) an economic actor who, in the case of (1) gives money to the speaker, does not demonstrate a~preference because he acts under a~coercive threat. However, in both cases (2) and (3), he demonstrates his preferences. In both cases, he gives up his money voluntarily.



However counterintuitive case (2) may seem, there is no room for a~thorough analysis of the libertarian theory of justice and the related concept of voluntariness (a thorough analysis of this concept is presented by Igor Wysocki 
%\label{ref:RNDsg7z9jmQpP}(2021)
\parencite*[][]{wysocki_austro-libertarian_2021}%
). However, it is worth considering one more possibility. More specifically, it is possible that by property Austrians understand not so much the right to property as the physical control over a~resource. The distinction between these concepts was presented by Mises 
%\label{ref:RNDiBtTK66oYm}(1962, pp.37–39).
\parencite*[][pp.37–39]{mises_socialism_1962}. %
 He distinguishes between the legal concept of property (property right) and the physical control over a~resource (``natural ownership'' or ``possession''). Then, the concept of voluntariness adopted by them would not violate the postulate of \textit{Wertfreiheit}. This interpretation is suggested by Jeffrey Herbener 
%\label{ref:RNDzkb18suOad}(1997, p.99).
\parencite*[][p.99]{herbener_pareto_1997}.%




Unfortunately, it seems to be inconsistent with Austrian subjectivism. In accordance with this principle, economics does not deal with the physical world, but rather with the mental states of the acting individuals. This has been emphasized not only by Mises 
%\label{ref:RNDyPzg960f5J}(1998, p.92),
\parencite*[][p.92]{mises_human_1998}, %
 but also by Hayek 
%\label{ref:RND4cYlDCwg9u}(1952)
\parencite*[][]{hayek_counter-revolution_1952} %
 and Rothbard 
%\label{ref:RND37uP1tqqg8}(2011b, p.289).
\parencite*[][p.289]{rothbard-present}.%




Moreover, predictions resulting from the concept of voluntariness based on physical control over a~resource would differ significantly from those resulting from the concept of voluntariness based on the libertarian theory of justice. It is true that both of these concepts would consider giving money in cases (2) and (3) as voluntary actions (since you cannot have physical control over your reputation, and the seller of bread does not violate either the natural ownership or the property right of the buyer). However, let us consider two more cases:



\medskip

\noindent (4) A~lends B~a bike. When A~asks B~to give the bike back, B~says, ``Give me \$10 or I~will destroy the bike''.

\noindent (5) A~steals B's wallet. B~says to A: ``Give my wallet back or I~will hit you''.

\medskip


Would A~give B~the money or wallet voluntarily? In the light of the concept of voluntariness based on the libertarian theory of justice, in case (4) certainly not---after all, B~violates A's property right (because by lending the bike, A~did not renounce the property title to it, and did not agree to its destruction). In light of the concept of voluntariness based on physical control, however, it should be said that A~gives money to B~voluntarily. Though he did not relinquish his property right to the bike, he lost physical control over it. In accordance with this concept, a~``threat'' of destroying something over which one has no physical control cannot be a~coercive threat! Note, however, that in case (5), it is to the contrary. The heretofore provided examples can be summarized as follows:







\begin{table}[H]
    \centering
    \begin{adjustbox}{max width=\textwidth}
        \begin{tabularx}{\textwidth}{|L{4cm}|Y|Y|}
            \hline
            \textbf{Proposal} & \textbf{Is this coercion, based on the criterion of physical possession violation?} & \textbf{Is this coercion, based on the criterion of property rights violation?} \\ \hline
            (1) B comes to A and says: “Give your money or I will kill you”. & Yes & Yes \\ \hline
            (2) B comes to A and says: “Give your money or I will destroy your reputation”. & No & No \\ \hline
            (3) B comes to A and says: “If you give me \$2, I will give you bread”. & No & No \\ \hline
            (4) A lends B a bike. When A asks B to give the bike back, B says: “Give me \$10 or I will destroy your bike”. & No & Yes \\ \hline
            (5) A steals B’s wallet. B says to A: “Give my wallet back or I will hit you”. & Yes & No \\ \hline
        \end{tabularx}
    \end{adjustbox}
    \caption{Coercive threats and non-coercive propositions.}
\end{table}



As I~have tried to show, the concept of voluntariness, which is adopted by the cited representatives of the ASE, poses serious difficulties for their methodology. To deal with this problem, representatives of the ASE would have to present such a~concept that would be consistent with their methodology. Since it is usually believed that a~sufficient condition for the involuntary nature of an action is to undertake it under coercion, it would be worthwhile to begin such research with a~study of the rich philosophical literature on coercion (a review of the concepts formulated so far can be found in 
%\label{ref:RNDwGx7HSdTFx}(Anderson, 2021);
\parencite[][]{anderson_coercion_2021}; %
 an attempt at finding a~concept of coercion fitting the Austrian methodology can be found in 
%\label{ref:RNDpPzpML5WPU}(Megger and Wysocki, 2023)
\parencite[][]{megger_coercion_2023}%
).



However, even after finding appropriate concepts of coercion and voluntariness, one might still think that linking DP to voluntariness is unjustified. From the perspective of Austrian praxeology, in fact, every action is driven by some preference 
%\label{ref:RNDmb3Dz6GDTo}(Mises, 1998, pp.13–14, 92–98).
\parencite[][pp.13–14, 92–98]{mises_human_1998}. %
 It, therefore, seems quite convincing to say that if a~victim takes an action aimed at giving money to the robber, he demonstrates his preference for the preservation of life over the preservation of money. It is difficult to formulate a~serious objection against such an interpretation of DP. For this reason, it is worth taking a~closer look at the belief found among Rothbard and his followers that in economics only demonstrated preferences matter, and non-demonstrated preferences should not have any theoretical consequences.



\section{``Only demonstrated preferences matter in economics''}

At this point, I~will present two possible interpretations of the assumption that in economics, only demonstrated preferences matter. To the best of my knowledge, these interpretations were presented for the first time by Michal Kvasnička, who writes:



\begin{quote}
Rothbard rejects from analysis everything which is not demonstrated in an actual action, i.e. what goes beyond the scope of the demonstrated preference, as a~vain psychology. We can read this in two ways: 1) we can know nothing that was not demonstrated in an action, or 2) there is nothing more than what was demonstrated in an action. While Rothbard might have the first in mind, he spoke as if he believed the second. 
%\label{ref:RNDLs5MnzUUCg}(Kvasnička, 2008, p.44)
\parencite[][p.44]{kvasnicka_rothbards_2008}%
\end{quote}




As I~will try to prove, the strong version of DP (2) is contrary to common sense and has been effectively challenged by critics of Rothbardian methodology. Next, I~will try to show that a~weaker, more common-sense and criticism-resistant version of DP (1) avoids some accusations but does not have sufficient grounds. I~will also try to prove that both of these versions have similar theoretical consequences.



\subsection{Are there only demonstrated preferences?}



The strong version of DP could be defined as follows:

\medskip

\noindent \textbf{SVDP} (\textit{Strong Version of Demonstrated Preference}): There are only those preferences that are demonstrated in the actions of individuals (or: those that determine their actions).

\medskip

The rationale for such an interpretation of DP can be found in Rothbard's critique of so-called psychologizing 
%\label{ref:RNDdGAPN4T2uc}(Rothbard, 2011b, pp.296–298).
\parencite[][pp.296–298]{rothbard_present_2011}. %
 Additionally, according to Kvasnička 
%\label{ref:RNDiTnIGsPOG8}(2008),
\parencite*[][]{kvasnicka_rothbards_2008}, %
 it is indicated primarily by some of Rothbard's arguments.



First, let us deal with the problem of psychologizing. When Rothbard criticizes Samuelson's concept of revealed preference for its implicit assumption of the constancy of preferences over time, he writes:



\begin{quote}
The revealed-preference doctrine is one example of what we may call the fallacy of ``psychologizing,'' the treatment of preference scales as if they existed as separate entities apart from real action. Psychologizing is a~common error in utility analysis. It is based on the assumption that utility analysis is a~kind of ``psychology,'' and that, therefore, economics must enter into psychological analysis in laying the foundations of its theoretical structure. 
%\label{ref:RNDINM3s5bSyQ}(Rothbard, 2011b, p.296)
\parencite[][p.296]{rothbard_present_2011}%
\end{quote}




Here, Rothbard seems to be casting doubt upon the existence of preferences apart from real actions. However, it is difficult to understand why the assumption of the existence of given preferences apart from actions Rothbard calls psychologizing since according to him psychology deals with the reasons \textit{why} people have certain preferences and how the preferences are formed:



\begin{quote}
Psychology analyzes the \textit{how} and the \textit{why} of people forming values. It treats the concrete content of ends and values. Economics, on the other hand, rests simply on the assumption of the existence of ends, and then deduces its valid theory from such a~general assumption. 
%\label{ref:RNDLcz4ZvPxhf}(Rothbard, 2011b, pp.296–297)
\parencite[][pp.296–297]{rothbard_present_2011}%
\end{quote}




One who assumes that people have preferences not demonstrated in actions does not have to deal with the reasons \textit{why} people have these preferences and \textit{how} these preferences are formed. Therefore, it does not seem that ``psychologizing'' is a~sufficient reason to reject the assumption of the existence of preferences apart from actions.



To understand why Rothbard's research practice seems to be based on the assumption that there are no preferences other than those demonstrated in the actions of individuals, it is worth considering at least two examples of the economist's conduct: indifference and externalities. As can be seen, DP serves Rothbard to eliminate from economic theory the concept of indifference (and the indifference curves known in mainstream economics), supposedly based on ``psychologizing''. As Rothbard states, ``indifference'' may be an important concept in psychology, but not in economics (praxeology). This is because indifference cannot be demonstrated in action. Each action necessarily demonstrates a~strict preference for a~particular state of affairs, whereas: ``Indifference classes are assumed to exist somewhere underlying and apart from action'' 
%\label{ref:RND43VZaZlll7}(Rothbard, 2011b, pp.304–305).
\parencite[][pp.304–305]{rothbard_present_2011}.%




However, as Nozick 
%\label{ref:RNDYCeHNHdhH7}(1977)
\parencite*[][]{nozick_austrian_1977} %
 noted in his famous critique, the Austrians need the concept of indifference to define even such elementary concepts as the supply of goods or the law of diminishing marginal utility, because an economic actor must be indifferent to the units of the same good (each unit of a~particular good must be valued identically by him). Even if we cannot determine who perceives what things as units of the same good, in the economic analysis, we must assume that such a~phenomenon as indifference exists.\footnote{Nozick's paper launched a~long-lasting debate on indifference within the Austrian camp 
%\label{ref:RNDZeA2gwNcG7}(see e.g., Block, 1980; 2009; Block and Barnett II, 2010; Hoppe, 2005; 2009; Machaj, 2007; Wysocki, 2016; 2017).
\parencites[see e.g.,][]{block_robert_1980}[][]{block_rejoinder_2009}[][]{block_rejoinder_2010}[][]{hoppe_note_2005}[][]{hoppe_further_2009}[][]{machaj_praxeological_2007}[][]{wysocki_indifference_2016}[][]{wysocki_note_2017}. %
 For a~review of the debate on indifference and a~defence of this concept in ASE, see 
%\label{ref:RNDNIjQ4PVm5l}(Wysocki, 2021).
\parencite[][]{wysocki_austro-libertarian_2021}.%
}



Rothbard does the same for externalities.\footnote{The only negative externalities that Rothbard allows are those that violate someone's property rights.} In his essay, he cites the example of an envious man who could be worse off due to other people's voluntary actions, and unequivocally rejects this possibility because the envious man cannot demonstrate his preferences:



\begin{quote}
But what about Reder's bogey: the envious man who hates the benefits of others? To the extent that he himself has participated in the market, to that extent he reveals that he likes and benefits from the market. And we are not interested in his opinions about the exchanges made by others, since his preferences are not demonstrated through action and are therefore irrelevant. How do we know that this hypothetical envious one loses in utility because of the exchanges of others? Consulting his verbal opinions does not suffice, for his proclaimed envy might be a~joke or a~literary game or a~deliberate lie. 
%\label{ref:RND4E9xQ1IzLQ}(Rothbard, 2011b, p.320)
\parencite[][p.320]{rothbard_present_2011}%
\end{quote}




According to Kvasnička 
%\label{ref:RNDP8yOStvwYM}(2008),
\parencite*[][]{kvasnicka_rothbards_2008}, %
 the examples of indifference and externalities show the fact that Rothbard in practice treats non-demonstrated preferences as if they did not exist at all (even if he declares that they are simply unknown). However, as Kvasnička argues, it is one thing to say that indifference cannot be demonstrated, and another thing that indifference does not exist 
%\label{ref:RNDdn1motFX7y}(Kvasnička, 2008, p.44)
\parencite[][p.44]{kvasnicka_rothbards_2008}%
\footnote{As he writes: ``The inability to demonstrate indifference is no proof there is no indifference but only that an outside observer cannot observe it, which is quite a~different thing. […] The indifference curves just describe the agent's inner world, in which Rothbard takes no interest, or rather denies its existence altogether.'' 
%\label{ref:RNDR7R6ck0BxE}(Kvasnička, 2008, p.44)
\parencite[][p.44]{kvasnicka_rothbards_2008}%
}. Next, he notes that SVDP suffers from serious drawbacks such as, say, that an individual cannot demonstrate his preferences passively and, therefore, he cannot demonstrate his losses of welfare (frustration of preferences). If, therefore, one gets passively robbed, then (based on the SVDP) it cannot be said that his preferences have been thwarted 
%\label{ref:RNDOsVmC1SN16}(Kvasnička, 2008, pp.45–46).
\parencite[][pp.45–46]{kvasnicka_rothbards_2008}. %
 According to Kvasnička, such an understanding of DP is not only contrary to common sense but also to the principle of Pareto-efficiency accepted by Rothbard and the research practice of this economist.



Nozick comes to a~similar interpretation of DP when he argues that one of the Austrians' theses on preferences states that: ``The notion of preference makes no sense apart from an actual choice made'' 
%\label{ref:RNDaKw2RsSh8U}(Nozick, 1977, p.370).
\parencite[][p.370]{nozick_austrian_1977}. %
 Nozick observes, however, that Austrians must assume the existence of preferences that are not demonstrated in actions because otherwise they could not define even such an elementary concept as cost (or perhaps: opportunity cost):



\begin{quote}
If we are to speak of the cost of \textit{A}, when there is more than one other alternative rejected, it must \textit{make sense} to speak of preference apart from an actual choice or doing of the preferred alternative. If \textit{that} doesn't make sense, then neither does the notion of the \textit{cost} of the action which was actually chosen. 
%\label{ref:RNDFTI7LdQBmX}(Nozick, 1977, p.373)
\parencite[][p.373]{nozick_austrian_1977}%
\end{quote}




Caplan argues that Rothbard's refusal to acknowledge unobserved preferences is not only extreme behaviourism but also contrary to common sense. The introspective experience of the existence of preferences that are not demonstrated in actions is common. So, even if knowing someone else's mental states is more difficult, it is hard to deny that they exist 
%\label{ref:RNDqyPBWt8nr0}(Caplan, 1999, p.834).
\parencite[][p.834]{caplan_austrian_1999}.%




Finally, it is necessary to emphasize one thing related to preferences and preference scales. In this context, it is worth quoting Mises's remarks, reminiscent of SVDP:



\begin{quote}
one must not forget that the scale of values or wants manifests itself only in the reality of action. These scales have no independent existence apart from the actual behavior of individuals. The only source from which our knowledge concerning these scales is derived is the observation of a~man's actions. Every action is always in perfect agreement with the scale of values or wants because these scales are nothing but an instrument for the interpretation of a~man's acting. 
%\label{ref:RNDFbMuotTpv0}(Mises, 1998, p.95)
\parencite[][p.95]{mises_human_1998}%
\end{quote}




Of course, in the face of the above quote, it cannot be precluded that Mises would also subscribe to SVDP, which, I~argue, should be rejected. However, one more possibility may need to be considered. When Mises speaks of ``scales of values or needs'', he may mean fixed and perfectly ordered preferences of acting individuals (as in the rational choice theory). Undoubtedly, such a~phenomenon does not occur in reality, because human preferences can be constantly changing and, due to the actors' false beliefs, they can be contradictory (for example, one may simultaneously prefer socialism over capitalism and productive allocation of resources over waste, without recognizing that these preferences are mutually exclusive). So, even if speculation about ``fixed and ordered preference scales'' apart from the actions of individuals may be meaningless, preferences as such must exist.



Since common-sense realism can be considered an important element of economic theory in general 
%\label{ref:RNDJCCLQzL8a7}(Mäki, 2008),
\parencite[][]{hausman_realism_2008}, %
 and Austrians in their concepts and theories necessarily refer to preferences existing apart from actions, there is no doubt that SVDP should be rejected. There is no reason to maintain such a~strong ontological claim that leads to the conclusion that so-called dispositional mental states (that can exist apart from actual awareness, e.g., memories) do not exist at all.



\subsection{Only demonstrated preferences should be taken into account}



The weak version of DP could be defined as follows:

\medskip

\noindent \textbf{WVDP} (\textit{Weak Version of Demonstrated Preference}): Only those preferences that are demonstrated in actions (or: those that determine individuals' actions) can serve as the basis for economic theories.

\medskip

This interpretation of DP seems more common-sensical, more benevolent, and most likely corresponds to Rothbard's intentions. It seems to be exposed in Walter E. Block's reply to Caplan:



\begin{quote}
Rothbard's argument is that only demonstrated preferences are ``genuine'' for economic theory, i.e., related to action. Pie in the sky ``wishes'' that a~person has (e.g., to buy ice cream without money, or to purchase it ``later'') are not preferences at all in the technical sense. Preference is defined, in this technical sense, as the ranking of ends upon which an action is based. This is what makes Caplan's claims about acting on the basis of indifference incorrect. 
%\label{ref:RNDT8mTWYFX3s}(Block, 1999, p.23)
\parencite[][p.23]{block_austrian_1999}%
\end{quote}




Block recognizes the preferences demonstrated in actions as the only preferences ``in the technical sense'', as only these preferences can lead individuals to actions. Preferences that do not lead to actions are supposed to be irrelevant in economics. Only those preferences that lead to actions are to be of economic importance.



But what is the reason for making such an assumption? The argument used by Rothbard and Block is apparently as follows: ``Because only demonstrated preferences are known with certainty''. Rothbard rejects taking into account non-demonstrated preferences in economic theory because he considers it to be on a~par with ``psychologizing'', something illegitimate in economics. WVDP rejects such ``hypothetical imaginings'':



\begin{quote}
Demonstrated preference, as we remember, eliminates hypothetical imaginings about individual value scales. Welfare economics has until now always considered values as hypothetical valuations of hypothetical ``social states.'' But demonstrated preference only treats values as revealed through chosen action. 
%\label{ref:RNDpcEXa5dfX3}(Rothbard, 2011b, p.320)
\parencite[][p.320]{rothbard_present_2011}%
\end{quote}




Block writes in turn: ``as far as technical economics is concerned, we cannot take cognizance of those of Caplan's wishes which are not objectively revealed or demonstrated in action. How can we, as economists, even know they exist?'' 
%\label{ref:RNDNdMT80rPbQ}(Block, 1999, p.23).
\parencite[][p.23]{block_austrian_1999}.%




The question that arises in the face of the above remarks is: in what sense do we know demonstrated preferences ``objectively'' or ``with certainty''? Certainly not in such a~way that they are known to us in empirical reality since the knowledge of the mental states of others is based on thymology and can never be certain. In addition, thymology also seems to allow other people to have preferences apart from actions (since we recognize this phenomenon introspectively, we can---as part of \textit{Verstehen}---ascribe it to other people). Thymologically speaking, knowledge of both demonstrated and non-demonstrated preferences cannot be absolutely certain (see point 2). Since, as I~have tried to show above, the assumption that preferences exist apart from actions is well founded on the basis of common sense and is indispensable in the theories and concepts used in ASE, the presented argument does not seem sufficient to accept WVDP.



There is no doubt, however, that WVDP (in this context SVDP works as well as WVDP) is the foundation of the economic theory of Rothbard and his followers. For example, one of the reasons Rothbard rejects the standard monopoly theory, which allows monopoly prices to occur in the free market 
%\label{ref:RNDaIPjBgTEzd}(which Mises accepted, see: Mises, 1998),
\parencite[see:][]{mises_human_1998}, %
 is that: ``In praxeology we are interested only in preferences that result in, and are therefore demonstrated by, real choices, not in the preferences themselves'' 
%\label{ref:RND78AysLpPTB}(Rothbard, 2009a, p.701).
\parencite[][p.701]{rothbard_man_2009}. %
 The concept of the supply of an economic good, which is the necessary basis for the theory of monopoly prices, is problematic because: ``A good cannot be independently established as such apart from consumer preference on the market'' (ibid.)\footnote{Rothbard also raises other objections, based on the principle of subjectivism. These seem stronger and are not the subject of the presented critique.}.



The Austrian critique of public goods theory also seems implicitly based on WVDP (in this context SVDP works as well as WVDP). According to Rothbard, Block, and Hoppe, any voluntary, free-market (i.e., not resulting from coercion or violence) situation is optimal from the point of view of consumers. And since---according to the WVDP---economic actors demonstrate their preferences only in actions, preferences for something they do not pursue in actions are irrelevant in economic theory. For example, all members of a~community would rather breathe clean air and would be willing to heat their houses with slightly more expensive, but more ecological methods. However, they burn garbage only because they are convinced that other members of this community would not change their behaviour, and the expected cost of encouraging them to do so exceeds the expected benefits. According to the WVDP, there is no problem of market failure and, consequently, of public goods: apparently, clean air is not a~sufficiently valued good for members of this community to strive for its acquisition 
%\label{ref:RNDvBRoB07bwD}(Block, 1983; Hoppe, 1989; 2006; Wiśniewski, 2018).
\parencites[][]{block_public_1983}[][]{hoppe_theory_1989}[][]{hoppe_economics_2006}[][]{wisniewski_economics_2018}.%




Mark R. Crovelli proceeds in a~similar way, rejecting game theory, or more precisely, its paradigmatic concept of the prisoner's dilemma. As this scholar argues, even if we can imagine a~hypothetical scenario in which possible choices of given individuals create a~prisoner's dilemma, we can never say with certainty that any prisoner's dilemma ever existed. As Crovelli argues, citing Rothbard and DP, only real actions provide absolutely certain knowledge about human preferences 
%\label{ref:RND4s4oEPXCQK}(Crovelli, 2006).
\parencite[][]{crovelli_trouble_2006}.%




Although Crovelli's argument seems inaccurate in the face of the considerations presented here, it helps us to formulate an important argument against the supporters of WVDP. To paraphrase Kvasnička's argument, it is one thing to say that we can never know with certainty whether and when such situations (monopoly prices, public goods, prisoner's dilemmas) occur, it is another thing to say that such situations never occur and cannot occur. Undoubtedly, in reality, it can be difficult to determine when such situations occur. It should be noted, however, that this is a~thymological, not a~praxeological, issue, and is valid for all economic theories, not just welfare economics, monopoly theory, public goods theory, or prisoner's dilemmas.



As can be seen, although WVDP protects Austrian methodology from the objection of a~lack of common sense, it leads to similar practical consequences. In addition, as I~have tried to show, Austrians give a~dubious argument in favour of WVDP. Thus, rejecting from economic theory any considerations of preferences existing apart from actions seems unconvincing. In the next section, I~will try to present an interpretation of DP that is the only one that does not suffer from any serious objections.



\section{Only demonstrated preferences affect the social processes}

Even if it is difficult to assume that only demonstrated preferences matter in economics, perhaps it can be said that only demonstrated preferences matter in the economy. Undoubtedly, only those preferences that lead economic actors into actions have an impact on the market process. Since the analysis of the market process (as opposed to the analysis of economic equilibrium) is one of the fundamental distinctive features of ASE 
%\label{ref:RNDVcFAHmyH2I}(Rothbard, 2011b; Martin, 2015),
\parencites[][]{rothbard_present_2011}[][]{coyne_austrian_2015}, %
 it seems that the suggested interpretation fits well with the Austrian methodology.



The emphasis on the market process rather than on economic equilibrium dates back to the very beginnings of ASE. It can already be found in Carl Menger. Later, this is clearly visible in the works of Hayek 
%\label{ref:RNDsY5UoChRcm}(1945),
\parencite*[][]{hayek_use_1945}, %
 Mises 
%\label{ref:RNDk1GagtppBt}(1998),
\parencite*[][]{mises_human_1998}, %
 and Israel M. Kirzner 
%\label{ref:RND0jWufa1bsu}(1973).
\parencite*[][]{kirzner_competition_1973}. %
 Austrians agree that equilibrium analyses do not correspond to the rich and dynamic complexity of market processes. They also reject some assumptions found in neoclassical economic models such as, say, that market participants have complete information. Instead, they show how individuals, entangled in a~specific context of time and place, take actions with incomplete knowledge and subjective, constantly changing preferences, and as a~result, influence other people's actions.



There is no doubt that preferences that do not determine people's actions do not affect the empirical reality and, therefore, cannot affect social processes. The real demand for specific goods arises when a~certain quantity of goods is purchased. Market prices are shaped by concrete transactions. Spontaneous order is an unintended result of human actions, not unrevealed preferences. It seems impossible to deny these statements. Therefore, if DP is to find an important place in ASE, it should mean just that: \textit{only actions} (in contrast to dispositional mental states alone) \textit{are causally relevant for the social and economic processes}.



\section{Conclusion}

In this article, my goal was to present a~systematic interpretation of DP based on the available literature on the subject. As I~have argued, the thymological interpretation of DP should be rejected (because it does not develop the method of \textit{Verstehen}). The interpretation linking DP to the condition of voluntariness seems problematic (due to the very concept of voluntariness) and insufficiently well-justified (due to the fact that involuntary actions also express preferences). Then, I~proceeded to an analysis of two variants of the assumption that, in economics, only demonstrated preferences matter. As I~have tried to show, SVDP is contrary to common sense and the research practice of the Austrians, and WVDP does not seem justified well enough. As a~result, I~conclude that the only interpretation of DP that is not exposed to serious objections is as follows: \textit{only actions affect the social and economic processes} (that is to say, only actions are causally relevant). This interpretation, however, is not sufficient to draw some of the conclusions that Rothbard and his successors reach in the field of welfare economics, monopoly theory, public goods theory, or social dilemmas known from game theory. Other implications of these conclusions remain a~matter of further research.



\end{artengenv}

\label{megger-last}
\begin{artengenv}{Paweł Nowakowski}
	{A~praxeology of the value of life. A~critique of Rothbard's argument}
	{A~praxeology of the value of life\ldots}
	{A~praxeology of the value of life. A~critique of Rothbard's\\argument}
	{University of Wrocław\label{nowakowski-first}}
	{The present paper aims to study the issue of the value of life in Murray N. Rothbard's work, and to examine his argument for the contention that ``life \textit{should} be an objective ultimate value'' and that ``the preservation and furtherance of one's life takes on the stature of an incontestable axiom.'' Rothbard's assumptions and presuppositions are investigated and critically assessed. Using conceptual and logical analysis rooted mostly in the praxeological method of economics (as developed by Mises and Rothbard himself) and the theory of value (Scheffler, Raz, Elzenberg), it is demonstrated that Rothbard's account is fallacious both on its own as well as on broader theoretical grounds. It is argued that what Rothbard could---under his specific assumptions about valuing---correctly claim is only that an actor values life \textit{to some extent}, rather than that life has an objective ultimate value or preservation and furtherance of one's life has an axiomatic status. The theoretical argument is supported by empirical illustration from suicide terrorism. The paper submits that Rothbard's position on the value of life is unsound, and that using his argumentation as it stands cannot succeed.
	}
	{praxeology, time preference, Murray Newton Rothbard, valuing, value of life, libertarianism}







\section{Introduction}

\lettrine[loversize=0.13,lines=2,lraise=-0.03,nindent=0em,findent=0.2pt]%
{I}{}n this paper, I~investigate the issue of the value of life in the work of Murray N. Rothbard, an economist representative of the Austrian school of economics and one of the leading figures in modern libertarianism, understood as a~radical free-market current in the contemporary political philosophy, built upon two pillars: self-ownership and strong property rights 
%\label{ref:RND1th47sMADB}(see Rothbard, 2006; 2009)
\parencites[see][]{Rothbard2006For}[][]{Rothbard2009Man} %
 (and this is also how I~understand libertarianism herein). I~aim to assess the validity of Rothbard's 
%\label{ref:RNDGqeeRTt2Se}(1998, pp.32–33)
\parencite*[][pp.32–33]{Rothbard1998Ethics} %
 argument for the assertion that ``life \textit{should} be an objective ultimate value'' and that ``the preservation and furtherance of one's life takes on the stature of an incontestable axiom.''



It might appear the value of life is of rather secondary importance for libertarianism since it rests not so much on a~value-based discourse as on the rights-based one.\footnote{In this paper, I~make use of the distinction between axiology and ethics. I~refer to axiology as a~philosophical discipline centered around values which covers such problems as, for example, the concept of value, types of values and the way values exist. On the other hand, ethics (excluding metaethics) is primarily aimed at guiding our actions and is concentrated on norms, rights and duties. Thus, the discourse of rights is typically ethical (and legal), while the discourse of values is primarily axiological.} In other words, when libertarians claim it is impermissible to kill innocent people, they thereby argue that killing is impermissible not because it directly runs counter to the value of life but for this reason that it violates a~person's property right to his or her body, i.e., a~fundamental or natural right of a~person---self-ownership. Nonetheless, some scholars maintain that the assumption of the value of life is relevant to the derivation of libertarian rights, if only as far as some variants of libertarianism are concerned 
%\label{ref:RNDtBxg6NlKYc}(see Harris, 2002, p.115; Hoppe, 1998, pp.xxxiv–xxxv; Mack, 2022, p.14; Meng, 2002; Osterfeld, 1983; 1986, pp.60–61; Rasmussen, 1980; Share, 2012, p.134ff; Slenzok, 2018, p.29; Thrasher, 2018, p.217; Wissenburg, 2019, p.294).
\parencites[see][p.115]{Harris2002Rights}[][pp.xxxiv–xxxv]{Hoppe1998Introduction}[][p.14]{Mack2022Natural}[][]{Meng2002Hoppeing}[][]{Osterfeld1983Natural}[][pp.60–61]{Osterfeld1986Freedom}[][]{Rasmussen1980Groundwork}[][p.134ff]{Share2012Defense}[][p.29]{Slenzok2018Dwa}[][p.217]{Thrasher2018Social}[][p.294]{Wissenburg2019Concept}. %
 What is more, Rothbard's attempt to prove the axiomatic nature of the claim of the objective ultimate value of life is approvingly shared or even employed further by other libertarian authors 
%\label{ref:RNDdxGeyPBnIO}(see Hoppe, 1988; 1998, pp.xxxiv–xxxv; Meng, 2002; Rasmussen, 1980).
\parencites[see][]{Hoppe1988From}[][pp.xxxiv–xxxv]{Hoppe1998Introduction}[][]{Meng2002Hoppeing}[][]{Rasmussen1980Groundwork}. %
 Hoppe 
%\label{ref:RNDk7Vokc1JDt}(1988, p.66)
\parencite*[][p.66]{Hoppe1988From} %
 not only approves his argument but also maintains that its structure is the same as the one of Hoppe's argumentation ethics, which is his philosophical attempt to justify libertarianism. Also Meng 
%\label{ref:RNDnse7S0BCeo}(2002)
\parencite*[][]{Meng2002Hoppeing} %
 approves and broadly applies Rothbard's position on the value of life in the former's attempt to ground the principle of original appropriation. Therefore, the investigation of Rothbard's reasoning seems vindicated not only in terms of scrutinizing his argument itself, but also regarding its implication for deriving and justifying libertarian rights. It becomes even more evident when one considers that Rothbard is the only libertarian author who has delved into the value of life in some more detail.

\enlargethispage{1.5\baselineskip}

In the present article, I~make use of the praxeological methodology employed by Rothbard and characteristic of the Austrian school of economics in Ludwig von Mises' variation. Rothbard's ontological, epistemological, and methodological standpoints, which are substantial for his argument on the value of life, are the following: realism, foundationalism, cognitivism, apriorism, deduction, the law of non-contradiction, methodological individualism, the concept of human action as a~goal-oriented behavior using the means available thereto under the condition of the scarcity of resources, verbal logic as a~medium of reasoning, and the theory of time preference. I~adopt that research approach for the sake of analyzing Rothbard's considerations with respect to the validity of his premises, deductive moves, and reached conclusions. Additionally, I~make use of a~conceptual framework of the theory of value, particularly as regards the distinction between intrinsic value and instrumental value, and valuing.



As a~consequence, Rothbard's argumentation is subject to an analysis which leads to the following theses:

\begin{enumerate}[label=\arabic*)]
\item  Rothbard's argumentation does not provide a~proof of the objective ultimate value of life or the proof of the axiomatic status of preservation and furtherance of one's life.
\item Under his assumptions about \textit{valuing}, Rothbard could claim only that an actor values life \textit{to some extent}.
\end{enumerate}

The paper contributes to the ongoing debate by taking heed of the flawed nature of Rothbard's argumentation for the ultimate value of life, and by the indication that using his reasoning for advancing libertarianism cannot succeed.



In the following section, I~shall present Rothbard's position on value and value judgements through contrasting his conception with the theory of value, as conceived of by Mises. In the subsequent two sections, I~shall present Rothbard's argument for his proposition on the value of life and proffer my interpretation of his approach, including the issue of his presuppositions about valuing. Then, I~shall analytically refute Rothbard's argument. The critique of his reasoning shall be additionally strengthened in the next part of the article. Furthermore, the theoretical analysis shall be illustrated by a~case of a~suicide terrorist. Eventually, in the last section I~shall conclude.
\enlargethispage{2.5\baselineskip}


\section{Rothbard's position on value and value judgement}

As far as an economic theory of value goes, Rothbard favored a~definitely subjectivist approach, thus following Mises\footnote{On the differences in value theory between Mises and the founding father of the Austrian school of economics, Carl Menger, and another luminary thereof which was Eugen von Böhm-Bawerk, see Grassl 
%\label{ref:RNDzwdqMXMJwK}(2008, p.96; 2017, pp.531–559),
\parencite*[][pp.531–559]{Grassl2017Toward}, %
 and Hülsmann 
%\label{ref:RNDXw75SrxFFp}(2007, pp.388–391).
\parencite*[][pp.388–391]{Hulsmann2007Mises}.%
}, who stated that ``value is not intrinsic'' and instead defined it as the ``importance that acting man attaches to ultimate ends.'' A~secondary value he attributed also to the means employed for the sake of achieving an end. According to Mises 
%\label{ref:RNDf042XyLpW2}(2008, p.121),
\parencite*[][p.121]{Mises2008Human}, %
 what determines the value one ascribes to a~good is its utility, relative to the actor in question. Rothbard 
%\label{ref:RNDkUu9JNzcYv}(2009, pp.103, 21)
\parencite*[][pp.103]{Rothbard2009Man} %
 reasoned very much alike, positing that ``value exists in the valuing minds of individuals [...]'', and, more specifically, that ``the original source of value is the ranking of ends by human actors, who then impute value to consumers' goods, and so on to the orders of producers' goods, in accordance with their expected ability to contribute toward serving the various ends.''



However, Mises and Rothbard differed as to their respective views on the status of value judgements. Mises 
%\label{ref:RNDPeBoihkBZJ}(see, e.g. 2008, p.10)
\parencite*[see, e.g.][p.10]{Mises2008Human} %
 believed that in no science there is room for value judgements, that is for putting forward normative claims. Being a~proponent of a~free-market capitalism, he believed that the arguments in favor of it are provided by value-free economics and sociology, which were supposed to prove that it is only a~regime based on private property in the means of production that may function efficiently 
%\label{ref:RND6zmyfhCh9b}(Mises, 2010, p.86).
\parencite[][p.86]{Mises2010Liberalism}. %
 Moreover, he did not conceal his subjective, and thus unscientific, adherence to a~liberal political system, which was supposed to ensure the most congenial conditions to peace, prosperity, health, wealth, that is to the realization of the values which---as he stressed---he shared with the majority of society 
%\label{ref:RNDBNzj9B3ktd}(see Rothbard, 1997, pp.93–94).
\parencite[see][pp.93–94]{Rothbard1997Praxeology}.%




By contrast, Rothbard thought that values might be studied in their both positive and normative aspects. That is why, his position on value presented above should be rather done justice to by quoting the following excerpt from his \textit{Ethics of Liberty}: ``Value \textit{in the sense of valuation} or utility is purely subjective, and decided by each individual'' 
%\label{ref:RNDcltkKk07OX}(Rothbard, 1998, p.12, italics added).
\parencite[][italics added]{Rothbard1998Ethics}. %
 This \textit{sense of valuation} is pertinent precisely to economics since when it comes to ethics, Rothbard 
%\label{ref:RNDvhf9gFI7BT}(1997, p.78; 1998, p.12)
\parencites*[][p.78]{Rothbard1997Praxeology}[][p.12]{Rothbard1998Ethics} %
 claimed that there exist objective and rational criteria of an ethical assessment of value judgements as well as objective values derived from the natural law. Hence, in this respect he dissociated himself from Mises and subjected the latter's rational-utilitarian ethical approach to profound criticism based on a~consistent praxeological analysis 
%\label{ref:RNDu3O4hsetCR}(see Rothbard, 1997, pp.90–99; 1998, pp.201–214).
\parencites[see][pp.90–99]{Rothbard1997Praxeology}[][pp.201–214]{Rothbard1998Ethics}. %
 However, strictly axiological considerations, that is the ones concerning a~philosophical theory of value, are in Rothbard's output rather cursory. And when it comes to explicitly normative statements, these are confined to the issue of the value of human life, which is the subject matter scrutinized herein. This state of affairs was owing to the fact that Rothbard's normative reflection was mainly centered around natural rights that he deduced from the modified John Locke's statements on self-ownership and the principle of original appropriation.



\section{Rothbard's argument on the value of life}

In order to prove the axiomatic character of his claim, Rothbard 
%\label{ref:RNDkj6x5zHJYV}(1951, p.946; see also 2009, ch. 1-2)
\parencites*[][p.946]{Rothbard1951Praxeology}[see also][ch.~1-2]{Rothbard2009Man} %
 resorted to ``the theory of the isolated individual,'' which is representative of the Austrian school of economics, with the theory being also known as ``Crusoe economics'' 
%\label{ref:RND6cIvbGcfUi}(see Nozick, 1977, pp.353–392).
\parencite[see][pp.353–392]{Nozick1977On}. %
 To serve philosophical purposes, he adopted the theory in the form of the methodological tool labeled as ``a Crusoe social philosophy'' 
%\label{ref:RNDXyBnbf7szh}(see Rothbard, 1998, pp.29–34),
\parencite[see][pp.29–34]{Rothbard1998Ethics}, %
 in which he introduced a~second person in order to found his considerations upon an interpersonal relation.



In the hypothetical situation as depicted by Rothbard 
%\label{ref:RNDxVqlcztWqx}(1998, p.32),
\parencite*[][p.32]{Rothbard1998Ethics}, %
 the other inhabitant of the island warns Crusoe not to eat poisonous mushrooms growing there. Consequently, Robinson abandons the idea of consuming them, picking up berries instead. Rothbard claimed this situation evidences a~very strong conviction on the part of both of them that poison is detrimental to humans---the condition so strong that it is not even explicitly mentioned. So, it is in this manner that they both recognize that human life and health have value, unlike suffering and death. And that is the way in which the cognizing of ethical foundations proceeds, with the said ethical foundations reflecting the nature of things and the laws of nature pertaining to human beings.



Furthermore---argued Rothbard 
%\label{ref:RNDjpSvw6l2D4}(1998, pp.32–33)
\parencite*[][pp.32–33]{Rothbard1998Ethics}%
---if Crusoe had consumed---despite warnings---the poisonous mushrooms, this would constitute an act running counter to his life and health, which would be ``objectively \textit{immoral}.'' His actual motives would be irrelevant then: whether it was high time preference or the need to get intoxicated that would constitute underlying motives. Let us cite Rothbard's 
%\label{ref:RNDWiYOJ9ryS7}(1998, pp.32–33)
\parencite*[][pp.32–33]{Rothbard1998Ethics} %
 argument \textit{in extenso}:



\begin{quote}
It may well be asked why life \textit{should} be an objective ultimate value, why man should opt for life (in duration and quality). In reply, we may note that a~proposition rises to the status of an \textit{axiom} when he who denies it may be shown to be using it in the very course of the supposed refutation. Now, \textit{any} person participating in any sort of discussion, including one on values, is, by virtue of so participating, alive and affirming life. For if he were \textit{really} opposed to life, he would have no business in such a~discussion, indeed he would have no business continuing to be alive. Hence, the \textit{supposed} opponent of life is really affirming it in the very process of his discussion, and hence the preservation and furtherance of one's life takes on the stature of an incontestable axiom [footnotes deleted].
\end{quote}



Before I~embark on a~critique of Rothbard's argumentation, let us make some interpretative comments. Rothbard pointed out that recognizing life as an ultimate value implies that man should ``opt for life (in duration and quality).'' Based on this assertion as well as on the express statement that action running counter to one's life is, irrespective of the motives underlying it, objectively immoral, one may venture an interpretation of the concept of ``ultimate value,'' as employed by Rothbard, from an axiological point of view. Hence, it seems clear that ``ultimate value,'' as conceived of by him, is something over and above what Höffe 
%\label{ref:RNDC282eqJvZa}(1991, pp.26–27)
\parencite*[][pp.26–27]{Höffe1991Gerechtigkeit} %
 calls ``transcendental interests,'' i.e., ``logically higher-order interests,'' comprising the capability of action and thus requiring ``the integrity of one's health and life,'' as well as something beyond a~value considered as a~condition of attaching values to something. The latter approach would be incomplete, anyway, since life is a~precondition of both good and bad things 
%\label{ref:RNDhWC02Rl7ni}(Raz, 2001, p.8),
\parencite[][p.8]{Raz2001Value}, %
 and what is more, even when something is a~precondition of good, it does not make that precondition necessarily valuable. As Nozick 
%\label{ref:RNDfw3SDMYYOr}(1971, p.252)
\parencite*[][p.252]{Nozick1971On} %
 demonstrated, recovery from cancer is a~value the precondition of which is getting cancer, but it would be implausible to argue that getting cancer is a~value as being a~condition of the value of curing cancer. For Rothbard, the value of life must have meant a~value \textit{sensu proprio et stricto} because he formulated on its basis a~norm of requirement: ``man should opt for life (in duration and quality);'' and an implicit norm of prohibition by describing action running counter to one's life as objectively immoral. Owing to its purportedly objective character, it could not be a~value in the utilitarian or instrumental sense, that is a~relative value allowing for the realization of interests (including needs, desired, wants), the realization of which is valuable to one person but may be not valuable to another 
%\label{ref:RNDdVFbkrRRrT}(see Elzenberg, 1990, p.21; Raz, 2001, p.77).
\parencites[see][p.21]{Elzenberg1990Wartość}[][p.77]{Raz2001Value}. %
 The value that Rothbard labels as ``objective ultimate value'' bears the closest resemblance to an intrinsic or perfective value, that is---as opposed to a~utilitarian or instrumental value\footnote{Elzenberg 
%\label{ref:RNDGerilVSkOK}(1990, pp.28–29)
\parencite*[][pp.28–29]{Elzenberg1990Wartość} %
 distinguishes between an instrumental and utilitarian value. The former leads to a~perfective value, whereas the latter to the satisfaction of a~need.}---to a~value of unconditional character, which means that is not a~value \textit{for somebody} but precisely an \textit{objective} value 
%\label{ref:RNDe0A0t5A2Sx}(see Elzenberg, 1990, p.21ff; Raz, 2001, p.77; Schroeder, 2021; Zimmerman and Bradley, 2019).
\parencites[see][p.21ff]{Elzenberg1990Wartość}[][p.77]{Raz2001Value}[][]{Schroeder2021Value}[][]{Zimmerman2019Intrinsic}. %
 That said, philosophers distinguish two ways of understanding the term of intrinsic value. In the first sense, what is meant is an ultimate or non-instrumental value, whereas in the second, what is meant is an objective value in the ontological sense, which implies that the value in question exists independently of an actor 
%\label{ref:RNDc2TQSX7jjg}(see Frey and Morris, 1993, p.8).
\parencite[see][p.8]{Frey1993Value}. %
 Rothbard's argumentation is clearly related to the first sense because he attempts to prove the objectivity of the ultimate value of life based on an analysis of an individual action, which by necessity applies to each acting man. Thus, the conclusion on the objective value of life derives from an aggregate of subjective actions analyzed using the principle of performative non-contradiction.



For the sake of clarity, it is also worthwhile to elucidate the word ``affirm,'' which was used twice in the above-scrutinized fragment. The dictionary definitions are as follows: ``to state something as true'' (Cambridge Dictionary); ``to publicly show your support for an opinion or idea'' (Cambridge Dictionary); ``to assert (something, such as a~judgment or decree) as valid or confirmed'' (Merriam-Webster); ``to show or express a~strong belief in or dedication to (something, such as an important idea)'' (Merriam-Webster). Since in the above-quoted passage Rothbard ascribed affirming life to a~hypothetical denier of the value of life, affirming refers to \textit{showing} or \textit{expressing} (nonverbally) rather than \textit{stating} or \textit{asserting.} Further, the word ``affirm,'' as used by Rothbard, seems to have the same meaning as the word ``approval,'' used in the context of a~perfective value by Elzenberg\footnote{On Elzenberg's contribution to axiology see, e.g. 
%\label{ref:RNDfoYcrFCiQj}(Porębski, 2019, pp.73–86).
\parencite[][pp.73–86]{Porębski2019Elzenberg:}.%
} (however, ``affirmation'' is a~word of more unambiguously positive character). ``Each value judgement---states Elzenberg 
%\label{ref:RNDReYxOyaTDn}(1990, p.25)
\parencite*[][p.25]{Elzenberg1990Wartość}%
---is an approval. An approval is the very value judgement: an approval exhausts itself in a~value judgment and so does a~value judgement in an approval.''



On the basis of the above interpretations, it seems clear that when Rothbard used the phrase ``affirmation of life,'' what he thereby meant was an action demonstrating the recognition of life as being endowed with a~value in this sense which he tried to prove, which is the one of an intrinsic, perfective, ultimate value.



\section{Rothbard's presuppositions about valuing}

The foregoing findings are instrumental in identifying presuppositions which Rothbard held about valuing. Thus, from his words that ``the \textit{supposed} opponent of life is really affirming it in the very process of his discussion,'' we may contend that Rothbard took it for granted that it is possible to value something without being aware of valuing it, a~contention that comes from economics and is based on the concept of ``demonstrated preference'' 
%\label{ref:RNDurQzV0btZy}(cf. Osterfeld, 1986, p.61).
\parencite[cf.][p.61]{Osterfeld1986Freedom}. %
 In his seminal paper, Rothbard 
%\label{ref:RNDQ7Q2sJHgU6}(2011, p.289)
\parencite*[][p.289]{Rothbard2011Economic} %
 explained:



\begin{quote}
Action is the result of choice among alternatives, and choice reflects values, that is, individual preferences among these alternatives. [...] The concept of demonstrated preference is simply this: that actual choice reveals, or demonstrates, a~man's preferences; that is, that his preferences are deducible from what he has chosen in action. Thus, if a~man chooses to spend an hour at a~concert rather than a~movie, we deduce that the former was preferred, or ranked higher on his value scale.
\end{quote}



It has been pointed out above that Rothbard, unlike Mises, maintained that objective ethics is possible and that economic and ethical approaches to the study of value differ in that economics does not engage in value judgments, whereas ethics does. Having enabled Rothbard to engage in developing the political philosophy of modern libertarianism, that dissociation from Mises, however, was not pertinent to conceptual foundations. Accordingly, Rothbard's understanding of the concepts of value and valuing in the realm of ethics was rooted in economics or, more broadly, in praxeology, i.e., ``a~general theory of human action'' 
%\label{ref:RNDHLBJYqBRUC}(Mises, 2008, p.7).
\parencite[][p.7]{Mises2008Human}. %
 Hence, in both economics and ethics, he regarded value as inextricably linked to valuing, and valuing as, by definition, linked to a~concrete action which he believed to reveal a~value professed by an acting person. When considering the value of life, Rothbard also kept the assumption, taken from his economic writing, that value is just a~synonym for preference which is clearly presented in the above quotation, e.g., in the phrase ``choice reflects values, that is, individual preferences among these alternatives.'' This excerpt informs us of yet another of Rothbard's presuppositions, namely that each human action involves choosing between alternative values, so that an action cannot help but realize a~value professed by an acting being. Why didn't Rothbard abandon this conceptual background characteristic of the Austrian school of economics, when he was concentrating on libertarianism, e.g., in his \textit{Ethics of liberty}? A~probable answer is because, as a~student of Mises, he regarded praxeology as a~paradigmatic framework not merely for economics but for the whole edifice of the science of human action, ethics included. At the very beginning of \textit{Human Action}, Mises 
%\label{ref:RND60uY3O3SNX}(2008, p.3)
\parencite*[][p.3]{Mises2008Human} %
 set forth the boundaries of praxeology:

\begin{quote}
The general theory of choice and preference [praxeology] goes far beyond the horizon which encompassed the scope of economic problems [...]. It is much more than merely a~theory of the ‘economic side' of human endeavors and of man's striving for commodities and an improvement in his material well-being. It is the science of every kind of human action. Choosing determines all human decisions. In making his choice man chooses not only between various material things and services. All human values are offered for option. All ends and all means, both material and ideal issues, the sublime and the base, the noble and the ignoble, are ranged in a~single row and subjected to a~decision which picks out one thing and sets aside another.
\end{quote}



Mises conceived of praxeology very broadly indeed, and importantly, he made it clear that the praxeological analysis of values extends to ``all human values'' and ``ideal issues,'' thereby encompassing the problems typical for, among others, ethics and political philosophy. Rothbard himself emphasized that praxeology, although useful in analyzing ethical propositions, does not formulate ethical norms or public policies, and he stressed its positive (descriptive) character as opposed to normative (prescriptive) nature of ethics and political philosophy 
%\label{ref:RNDijKUL7bU5w}(see, e.g., Rothbard, 2009, p.1297ff).
\parencite[see, e.g.,][p.1297ff]{Rothbard2009Man}. %
 But on the other hand, when addressing value and valuing from the ethical point of view, particularly the value of human life, he presupposed the praxeological conceptual framework, which partly determined the way he attempted to justify the value of life as objective and ultimate.



As far as ethics or axiology is concerned, equating preference, as conceived of in the Austrian economics, with value is fallacious because while the former reduces to a~simple fact of choice between available options, the latter is a~more complex and comprehensive category. For instance, Scheffler 
%\label{ref:RNDzrKRSbAgcC}(2011, p.24)
\parencite*[][p.24]{Scheffler2011Valuing} %
 regards as peculiar using the term ``valuing'' when considering trivial desires, such as looking through a~newspaper at a~waiting room. In more detail, philosophers argue 
%\label{ref:RNDgE4mfkXes3}(see discussion in Scheffler, 2011)
\parencite[see discussion in][]{Scheffler2011Valuing} %
 that it is possible that a~person desires something (and, by extension, demonstrates her preference for it), but she does not value it because she considers it harmful or sinful---as it is possible in the case of desiring drugs by an addict or desiring to engage in disapproved sexual activity---and she would much prefer not to desire it (and, by extension, not to prefer it or not to demonstrate such a~preference). The reverse is also possible since valuing something may be not accompanied by a~desire for it or a~motivation to realize it, which might be the case when a~person suffers from mental disorders such as depression.



\section{Refuting Rothbard's argument directly}

In the following steps, I~shall reconstruct and refute the Rothbard's argument apparently bolstering the proposition that ``life \textit{should} be an objective ultimate value'' and that ``the preservation and furtherance of one's life takes on the stature of an incontestable axiom.''



There are two parts of the argument (which was quoted in full above). To avoid any possible misinterpretations, let us quote the particular parts once again:



\begin{enumerate}[label=\arabic*)]

\item ``[A]ny person participating in any sort of discussion, including one on values, is, by virtue of so participating, alive and affirming life.''



\item  ``[I]f he were \textit{really} opposed to life, he would have no business in such a~discussion, indeed he would have no business continuing to be alive.''



\item  ``Hence, the \textit{supposed} opponent of life is really affirming it in the very process of his discussion, and hence the preservation and furtherance of one's life takes on the stature of an incontestable axiom.''

\end{enumerate}

Based on the foregoing, I~suggest the following reconstruction of Rothbard's syllogism:



\begin{enumerate}[label=(\Roman*)]
\item  If A~is really against life, he does not have a~reason to argue about values or stay alive (based on point 2 above).
\item  If A~argues, he is alive (based on point 1).
\item  If A~stays alive, A~affirms life (based on points 1 and 3).
\end{enumerate}
Therefore,
\begin{enumerate}[label=(\Roman*), start=4]
\item A~is not really against life, and the objective and ultimate value of life is an axiom (based on point 3 and Rothbard's main thesis presented in the previous sections of the present paper).

\end{enumerate}

Now, although Rothbard's reasoning is not as precise as it might seem at first sight, his argument resembles a~logical principle \textit{modus tollens}---[(\textit{p}$\Rightarrow $\textit{q})${\wedge}$¬\textit{q}]$\Rightarrow $¬\textit{p}. When applied to his argument, the reasoning is as follows: If A~were really against life (\textit{p}), then A~would not have a~reason to stay alive and would terminate his life (\textit{q}), but A~is staying alive, arguing, and is not killing himself (¬\textit{q}), therefore A~is not really against life (¬\textit{p}). However, Rothbard's reasoning is erroneous since the assumed implication: if \textit{p}, then \textit{q}, is false and so is his finding (¬\textit{p}), which will be demonstrated below by scrutinizing parts (I) and (III) of the above syllogism. For the sake of clarity, I~shall start with (III), and proceed to (I), which is actually the order in which Rothbard presented his argument.



%\subsection{(III) If A~stays alive, A~affirms life }
%
%
%
%(``\textit{any} person participating in any sort of discussion, including one on values, is, by virtue of so participating, alive and affirming life'').

\subsection{\textbf{(III) If A stays alive, A \textit{affirms} life} (``\textit{any} person participating in any sort of discussion, including one on values, is, by virtue of so participating, alive and affirming life'')}



Suppose Crusoe decides to consume poisonous mushrooms in the hope of dying afterwards. After the ``meal,'' while he would be expecting the consequences of his act, he would receive a~visitor with whom he would engage in a~discussion over the value of life. In line with his previous act aimed against his life and health, he would try to convince his interlocutor that life is not (an ultimate) value, and it is bad and not good.



Would it be justified then to say that this man awaiting death affirms life? Certainly, he is alive and still benefits from living; yet he does not \textit{preserve} nor \textit{further} his life. It cannot be argued that if he were \textit{really} opposed to life, he would commit suicide instead of debating life with his visitor. After all, he already took an action aimed at terminating his life. And now, while awaiting expected death, he is trying to convince his interlocutor that life is bad. Thus, while keeping on living and discussing, he does not affirm life since he is using it merely as a~means of his argumentation against life (which means life can have only an instrumental value for him) rather than recognizes its objective ultimate value (perfective, intrinsic, unconditional value).



And so, (III) gets refuted (apart from the trivial implication from the quoted part of the argument that if a~person argues, he or she is alive).



\subsection{\textbf{(I) If A~is really against life, he does not have a~reason to argue about values or stay alive }
(``if he were \textit{really} opposed to life, he would have no business in such a~discussion, indeed he would have no business continuing to be alive'')}



It does not appear improbable---and what is even more important, it would be logically valid---if Crusoe were opposed not only to his life but also to life in general and that is why he would wish that his interlocutor also committed suicide. At this point, he might refer to, say, Kant's categorical imperative---deciding to terminate his life, he would consciously recognize that it is the act that he could wish that it become universal and that all the others also do so. In other words, Crusoe hated life and therefore he committed suicide. And since he hated life so much that his guiding principle became ``the greatest possible number of suicides,'' then while awaiting death he made use of an opportunity to convince somebody else of the justifiability of his view.



However, why didn't Crusoe, as a~hypothetical opponent of life, simply kill his interlocutor, and instead tried to persuade him to commit suicide? The answer seems fairly obvious: it is for the same reason for which people do not try (if they do not try) to force others to act according to the former's fundamental principles---because while the former recognize certain values as the most important, they also recognize individual rights, among which personal inviolability is one of the most fundamental ones.



As a~result, (I) is also fallacious.



\section{Further discussion}

To unfold the above critique of Rothbard's arguments, let us analyze the fact that there are numerous examples of people who sacrificed their lives for the sake of their values or other people. In case one person gives up her life so that another person or a~group of people could survive, one may claim that the thesis of the affirmation of life (as the ultimate value) remains unscathed. After all, by sacrificing one's life, one consciously sustains the life of another or of a~group of people. She thereby terminates her life \textit{in the name of} the value of life (maybe even the objective ultimate one). What is more 
%\label{ref:RNDVcaOriTF14}(see Raz, 2017, p.1; Weiss, 1949, p.76),
\parencites[see][p.1]{Raz2017On}[][p.76]{Weiss1949Sacrifice}, %
 the very notion of sacrifice presupposes that one can sacrifice something only provided that one ascribes to it some value.



It seems that Rothbard's position is easily reconcilable with the recognition that such a~sacrifice, even when materialized by committing suicide, is not at odds with the affirmation of life. As a~reminder, his thesis is that the ultimate moral value of life is objective and shared by all the living persons. However, in this context, it is again problematic to justify Rothbard's thesis. It is because, somewhat paradoxically, it transpires that:



\begin{enumerate}[label=(\arabic*)]

\item  One may keep on living without recognizing the ultimate value of life.



\item One may commit suicide, when recognizing the ultimate value of life.

\end{enumerate}

Although proposition (2) has already been stated, it might be supported by the indirect case from Scheffler. His definition of valuing is as follows: ``To value X~is normally both to believe that X~is valuable and to be emotionally vulnerable to X'' 
%\label{ref:RNDzOjWyphvFr}(Scheffler, 2011, p.31).
\parencite[][p.31]{Scheffler2011Valuing}. %
 Scheffler 
%\label{ref:RND1fruie6Vpy}(2011, pp.26–27)
\parencite*[][pp.26–27]{Scheffler2011Valuing} %
 presses the point that one can regard some activities or things as valuable without valuing them oneself. At first glance, it might sound curious, but Scheffler indeed has a~point. He submits that he indeed finds many activities \textit{valuable} without \textit{valuing} them himself, including folk dancing or studying Bulgarian history. He points out, though, that he usually does not engage in the activities he considers valuable without valuing them himself, which leads to the supposition that one values an activity when one finds it both valuable and engages in it. But Scheffler 
%\label{ref:RND3GsD9mZcKK}(2011, p.27)
\parencite*[][p.27]{Scheffler2011Valuing} %
 easily rejects this option by arguing in the following way: ``I may, for example, go to the opera from time to time, and I~may regard operagoing as a~valuable activity, and yet I~may still not value it myself. Even though I~participate in the activity and believe that it is a~valuable activity, operagoing may leave me cold.''



How does it translate into the value of life problem? It is possible to believe that life (\textit{resp.} living) is valuable, but not to value it oneself, even though we cannot help but be alive or, so to speak, participate in our own living when arguing. Hence, when we know that our family member or our best friend values life, whereas we do not, yet we do find it valuable, then terminating our life (even through suicide) to save them would be in accordance with the contention that we may commit suicide while recognizing the value of life, that is to regard life as valuable while not to value it ourselves. The question remains whether this argument also applies to the situation when the \textit{ultimate} value of life is the case, that is to (2). This question seems challenging as a~psychological empirical rather than the conceptual one. In the conceptual sense, it is not self-contradictory or self-defeating to assert that life is ultimately objectively valuable, while not to value one's own life, which runs counter to Rothbard's contention about the axiomatic character of the ultimate objective value of life. While it is true that this finding is based on a~specific, though compelling, definition of valuing proffered by Scheffler, it is also true that Rothbard's conceptualization of valuing as demonstrating one's preference is not universally binding, as was previously argued.



Let us now deal with the proposition (1). Let us pose the following question: How is it possible to sustain one's life when one does not affirm it so that one should not commit a~logical fallacy, which Rothbard warns us against?



To obtain a~correct answer to the above poser, it is enough to employ the praxeological reasoning of the Austrian school of economics; or, strictly speaking, its distinctive theory of time preference and the insights on ends and means, which Rothbard omitted when studying the problem of the value of life. By arguing that if a~particular living person were opposed to life, then---instead of merely declaring it---he would kill himself, Rothbard erred because in reality sustaining one's life while not affirming it is logically possible precisely due to recognizing the validity of knowledge on human action. He believed that a~person who is opposed to life should---not in a~normative sense, but rather as a~logical consequence---commit suicide instead of speaking of it at all. ``Instead of'' has two meanings here. The first is simply about choosing a~different conduct, more conducive to the realization of an adopted end. The second meaning is in turn related to time---it implies that the time dedicated to formulating one's position should (then again, only in a~logical sense and not in a~normative one) be used for suicide.



On the basis of the science of human action, Rothbard's reasoning may be called into question in two mutually related areas (incidentally, what also applies here is the above-mentioned argumentation based upon getting deliberately poisoned by mushrooms and engaging in a~conversation while awaiting death). The first of these rests on the theory of time preference.



The time preference principle is that an actor decides to give up a~present good in preference for a~future good only if she perceives in the latter a~possibility of greater satisfaction than the one she could have obtained if she had chosen the former. Individuals differ in their respective time preference rates (when it comes to the same individual, it may also vary over time), which stems from the estimation of satisfaction derivable from the present consumption as compared to the future one. Hence, one is warranted in speaking of both high and low time preference, which means, respectively, weaker or stronger tendency to prefer goods attainable later and providing more satisfaction over the ones attainable earlier but ensuring lesser satisfaction. For the individuals with high time preference, what counts is the present and whatever happens right afterwards. They expect immediate effects of their actions, allowing only for small delays. On the other hand, the individuals characterized by low time preference are oriented at future and that is why they appreciate the immediate consumption much less 
%\label{ref:RNDLqHCMVZHce}(see Hoppe, 2007, pp.1–6; Mises, 2008, pp.478, 481).
\parencites[see][pp.1–6]{Hoppe2007Democracy}[][pp.478, 481]{Mises2008Human}.%




In his analysis, Rothbard fails to consider this---crucial to the Austrian school---temporal aspect of action and its implications. In his reasoning with respect to the problem of the value of life he either presupposes very high time preference of an actor or ignores the principle of time preference altogether. In consequence, Rothbard does not heed the difference in the rate of time preference both across individuals and within the same individual but over time. Any lower than the highest possible rate of time preference (i.e., looking for an immediate satisfaction) is not even considered by Rothbard---as if no human could possibly instantiate it. To elucidate this mistake, let us assume that an individual who fails to see the ultimate value of life actually prefers death to life. In this sense, death in his view becomes a~good, quite as in the previously considered situations. Just for the sake of clarity, it does not imply the acceptance, at least for the sake of argument, of relativism, which would be incompatible with Rothbard's ethical absolutism 
%\label{ref:RNDxwtNXrYNAQ}(see Rothbard, 2008).
\parencite[see][]{Rothbard2008Toward}. %
 However, an actor might go astray, subjectively perceiving death as a~value 
%\label{ref:RNDFalGj7iknE}(cf. Nozick, 1971, p.252).
\parencite[cf.][p.252]{Nozick1971On}. %
 In this scenario, Rothbard's argumentation may be accepted only when immediate suicide would be the most preferred by the proponent of death analyzed herein. Then, there would be indeed no point in postponing one's suicidal act because, \textit{ceteris paribus}, committing it in the future would not bring any additional benefits as the optimal solution would be at hand---here and now.



However, interpreting life and death of an actor \textit{exclusively} in terms of highest, standing-alone and ultimate values is invalid. Again, based on Mises' 
%\label{ref:RNDe114bcTymm}(2008, pp.92–96, 216)
\parencite*[][pp.92–96, 216]{Mises2008Human} %
 praxeology, we can point out that each good, including life and health, may, depending on an actor and situation, serve either as an end or as a~means to realize one's end. Rothbard's reasoning applies to the first of these variations, whereas in reality denying the ultimate value of life does not necessarily nullify \textit{some} value of life. By the same token, denying the ultimate value of death does not necessarily nullify \textit{some} value of death\footnote{Raz 
%\label{ref:RND91eS6EtYsU}(2001, p.97)
\parencite*[][p.97]{Raz2001Value} %
 presented an interesting example of a~possible balance between the two. ``If, as I~suspect---he mentions---some people will take the option of dying younger, but not yet, it follows that some people value not dying soon even at a~cost to their longevity.''}. Hence, for an individual who lives and sustains his life (and thus does not kill himself), life \textit{may} (but does not have to) be only a~means to some end. In that case life constitutes a~good but only as a~means and thus it would not be endowed with the objective ultimate value. Its value in this case is instrumental or utilitarian---life is worth as much as it leads to a~given end. This would probably not be convincing for Raz 
%\label{ref:RNDfdXCznkk95}(Raz, 1999, p.191; 2001, pp.8, 77–78),
\parencites*[][p.191]{Raz1999Mixing}[][pp.8, 77–78]{Raz2001Value}, %
 who claimed life is a~precondition of both good and bad, thereby not recognizing life as such, as of value, however the foregoing findings bears some resemblance to the transcendental approach represented by Höffe 
%\label{ref:RNDLFX1W4e8n2}(1991; 1992),
\parencites*[][]{Höffe1991Gerechtigkeit}[][]{Höffe1992‘Even}, %
 except that he avoids the language of values, and only refers to interests. Hence, Höffe 
%\label{ref:RND41FuSo6U5V}(1992, p.131)
\parencite*[][p.131]{Höffe1992‘Even} %
 asserts that, by necessity, life is an elemental or transcendental interest of a~human being because it makes possible to desire something and to pursue it. Anyway, it is a~far cry from a~demonstration that life is a~perfective or intrinsic value.



That is why, Rothbard failed to prove the objective ultimate value of life. Rather, instead of proving the affirmation of life (as an ultimate value) on the part of a~given actor, he only proved---specifically under his praxeologically driven assumptions about \textit{valuing---}an actor's appreciating life to \textit{some} extent.\footnote{For a similar evaluation of Nathaniel Branden’s and Irfan Khawaja’s positions defending the claim on the existence of the ultimate value of live on the grounds of the objectivist philosophy
%(see Moen, 2012, pp.97–98).
\parencite[see][pp.97–98]{Moen2012Is}. %
However, because the dispute over Ayn Rand’s thesis is marked with other foundations and is rooted in the assumptions of objectivism, it is not compatible with the considerations herein and that is why it is of no interest to us here. See more in
%(Nozick, 1971; Rasmussen, 2002, pp.69–86; Hartford, 2017, pp.54–67). 
\parencites[][]{Nozick1971On}[][pp.69–86]{Rasmussen2002Rand}[][pp.54–67]{Hartford2017Ultimate}.%
%
%For similar evaluation of Nathaniel Branden's and Irfan Khawaja's positions who defend the claim on the existence of the ultimate value of live on the grounds of the objectivist philosophy 
%%\label{ref:RNDWrpaLReHab}(see Moen, 2012, pp.97–98).
%\parencite[see][pp.97–98]{Moen2012Is}. %
% However, the dispute over Rand's thesis is marked with other foundations and is rooted in the assumptions of objectivism; and hence, it is not compatible with the considerations herein and that is why it is of no interest to us here. See more in 
%%\label{ref:RND1rFeizSUw9}(Nozick, 1971; Rasmussen, 2002, pp.69–86; Hartford, 2017, pp.54–67).
%\parencites[][]{Nozick1971On}[][pp.69–86]{Rasmussen2002Rand}[][pp.54–67]{Hartford2017Ultimate}.%
} This in turn may imply an individual ascribing to life the highest as well as not the highest value. Moreover, this evaluation may vary over time. Therefore, Rothbard was right believing that a~person engaging in a~discussion recognizes the value of life; and yet he was wrong maintaining that based on the fact of participation in a~discussion, he managed to prove the axiomatic character of the proposition that ``life \textit{should} be an objective ultimate value.''



\section{Illustration from suicide terrorism}

In this section, I~will present an explicit illustration of the above considerations, that is an example of a~suicide terrorist, in order to support my claims by reference to empirical observations.



A~suicide terrorist does not believe that his life is divested of value but treats it as a~means to the end which is killing other people by dint of suicide terrorism. On the other hand, when it comes to jihadists, mere suicide is prohibited in Islam and that is why suicide terrorism is interpreted as a~heroic act of martyrdom, which is later rewarded with salvation 
%\label{ref:RNDtBLlnDgMaC}(see Bruce, 2013, pp.27–33; Roy, 2016, pp.15–24).
\parencites[see][pp.27–33]{Bruce2013Intrinsic}[][pp.15–24]{Roy2016Can}. %
 Moreover, according to jihadism, martyrdom is not the end of life; rather, it is the assurance of eternal life in paradise 
%\label{ref:RNDEaU40cdZIh}(Kruglanski et al., 2009, p.336).
\parencite[][p.336]{Kruglanski2009Fully}. %
 Kruglanski et al. 
%\label{ref:RNDAjcKXakDFB}(2009, p.336)
\parencite*[][p.336]{Kruglanski2009Fully} %
 note that ``paradoxically, \textit{the willingness to die in an act of suicidal terrorism may be motivated by the desire to live forever},'' however this excerpt deals with life after death, whereas the present paper concerns earthly lives, which are effectively terminated by suicide. That is why, at least from the vantage point of my analysis, the former position is not a~paradox. Amongst other motives for the terrorist's choice of an end in the form of suicidal terrorism, one normally enumerates such values as honor, dedication to the leader, social status, personal significance, feminism, restoration of the glory of Islam, moral obligation, money and support for one's family as well as other motives: pain and personal loss, group pressure, humiliation and injustice, vengeance, need to belong 
%\label{ref:RNDrS5AwO00G4}(cited in Kruglanski et al., 2009, p.332).
\parencite[cited in][p.332]{Kruglanski2009Fully}. %
 Furthermore, a~terrorist normally does not make his assault in an accidental place at accidental time because what matters to him is effectiveness. Alakoc 
%\label{ref:RNDFzDqqnCIUZ}(2017, p.1)
\parencite*[][p.1]{Alakoc2017When} %
 finds on the basis of statistical data that the popularity of suicide terrorism stems from its ``effective strategy for terrorizing by killing'' 
%\label{ref:RNDL7sUqX9gw1}(see also Hutchins, 2017, pp.7–11; Sheehan, 2014, pp.81–92).
\parencites[see also][pp.7–11]{Hutchins2017Islam}[][pp.81–92]{Sheehan2014Are}.%
\footnote{The statement that suicide terrorist attacks allow for killing a~greater number of people is undermined by Mroszczyk 
%\label{ref:RNDf1NguY0f0R}(2019).
\parencite*[][]{Mroszczyk2019To}. %
 However, he defends the crucial premise on the weight of the effectiveness of a~terrorist attack 
%\label{ref:RND7Vms12QaTo}(see Mroszczyk, 2019, pp.346–366).
\parencite[see][pp.346–366]{Mroszczyk2019To}.%
} It would be unjustifiable to generalize, but there are cases in which the suicide terrorist's utility is higher when more, rather than fewer, people die 
%\label{ref:RNDzKdCnO8RYg}(see Asthappan, 2010, p.25; BBC, 2019).
\parencites[see][p.25]{Asthappan2010Effectiveness}[][]{BBC2019German}. %
 Crabtree 
%\label{ref:RNDbEAT4wJIBg}(2006, p.577)
\parencite*[][p.577]{Crabtree2006Terrorist} %
 explains: ``Terrorist homicidal bombs are designed and detonated in a~manner that will maximize destructiveness against persons rather than against property. They are detonated in areas that are known to be occupied and often crowded and commonly are engineered to release metallic fragments for the purpose of increasing injury severity.'' In a~similar vein, states Alakoc 
%\label{ref:RNDW89FeIdkpA}(2017, p.6):
\parencite*[][p.6]{Alakoc2017When}: %
 ``Partial success occurs when a~suicide bomber detonates a~bomb earlier than originally planned but still causes civilian deaths and injuries.'' These as well as the other observations endorse the assumption that normally terrorist attacks are not spontaneous. Quite the contrary, terrorists prepare organizationally and logistically for an attack to maximize its expected effectiveness; although, when it comes to a~jihadist suicide, one must make a~caveat that in line with his ideology, even the minimum expected effectiveness is supposed to ensure him eternal life in heaven.



Based on the research conducted on lone-actor terrorists divided into two groups: the individuals closely connected with the radical groups---Autonomous (N=23), and the ones being unpredictable, impulsive and being more loosely connected with the radical groups---Volatiles (N=10), Lindekilde, O'Connor and Schuurman 
%\label{ref:RNDFtr5C7DJ4r}(2019, p.126)
\parencite*[][p.126]{Lindekilde2019Radicalization} %
 conclude that preparation periods for attacks in the case of Autonomous fluctuate around 48 months before a~scheduled attack on average; whereas in the case of Volatiles---on average four months before an attack. On the other hand, the time of planning an attack is, respectively, ten and four months. Furthermore, based on empirical research findings, Faria 
%\label{ref:RNDP8kgmPnIiW}(2003)
\parencite*[][]{Faria2003Terror} %
 argues that ``the number of terrorist activities decreases with terrorists' rate of time preference. That is, higher terrorist impatience leads to less successful terrorist activities.''



To conclude, this short illustration from suicide terrorism points to the consistency between the above theoretical analyses and the results of empirical studies on the phenomenon of contemporary terrorism.



\section{Conclusions}

In the present paper, I~have demonstrated that Rothbard's argumentation for the objective ultimate value of life is fallacious, which is mainly due to his failing to take into account the knowledge on individual value scales and the theory of time preference, as elaborated by the Austrian school of economics. This conclusion poses a~challenge for those libertarians who have adopted Rothbard's position, and particularly for those who recognize it as an argument for libertarian rights. However, although some scholars cited in the introduction to the present paper emphasize the relevance of the thesis of the value of life for the libertarian political philosophy, the general framework of libertarianism is not necessarily challenged by the conclusion of the present article because the libertarian rights-based discourse might be independent of the axiological one. It becomes clear when one bears it in mind that libertarian authors, including Rothbard, argue that the basic libertarian right, i.e., self-ownership, is an axiom and, as such, an adequate safeguarding measure ensuring a~conflict-avoiding and just social order 
%\label{ref:RNDR1Z1YPtWlV}(see, e.g., Child, 1994, p.736; Eabrasu, 2013; Kinsella, 2009, pp.184–186; Rothbard, 1998, p.60; 2006, pp.47–48; see also liberal account of Waldron, 1988, pp.399–400).
\parencites[see, e.g.,][p.736]{Child1994Can}[][]{Eabrasu2013Rothbard’s}[][pp.184–186]{Kinsella2009What}[][p.60]{Rothbard1998Ethics}[][pp.47–48]{Rothbard2006For}[see also liberal account of][pp.399–400]{Waldron1988Right}.%




One could possibly reach other conclusions if one were to regard as precise Hoppe's 
%\label{ref:RNDVRDvSUQlpL}(1998, p.xxxiv)
\parencite*[][p.xxxiv]{Hoppe1998Introduction} %
 following claim in which he comments upon the weight of Rothbard's argumentation for the axiomatic nature of the value of life:



\begin{quote}
Rothbard's distinct contribution to the natural-rights tradition is his reconstruction of the principles of self-ownership and original appropriation as the praxeological precondition---\textit{Bedingung} \textit{der Moeglichkeit---}of argumentation, and his recognition that whatever must be presupposed as valid in order to make argumentation possible in the first place cannot in turn be argumentatively disputed without thereby falling into a~practical self-contradiction.
\end{quote}



However, in reality, it was only Hoppe himself who made wider use of the principle of performative non-contradiction to justify the libertarian property rights 
%\label{ref:RNDmv1jBB9lnB}(see Hoppe, 1988; 1989, ch. 7),
\parencites[see][]{Hoppe1988From}[][ch. 7]{Hoppe1989Theory}, %
 whereas Rothbard's argumentation, scrutinized herein, does not transcend a~discussion over values; and that is why its rejection does not have to pose general challenges to the libertarian political philosophy, which is primarily based upon self-ownership and the principle of original appropriation. Having said that, any attempt, as the one by Meng 
%\label{ref:RNDv6s6nRoY4N}(2002),
\parencite*[][]{Meng2002Hoppeing}, %
 to develop the libertarian political philosophy based on Rothbard's argument rejected in the present study must be unsuccessful.
 
 \paragraph{Acknowledgments.} 
 I would like to thank Jakub Wozinski, Norbert Slenzok, Stanisław Wójtowicz, Arkadiusz Sieroń, Mateusz Zieliński, Przemysław Hankus, David Gordon, and Tate Fegley for their comments on earlier drafts.




\end{artengenv}

\label{nowakowski-last}



%\addtocontents{toc}{\protect\pagebreak}
%\sekcja{Essays}{Eseje}
%\sekcja{Proceedings of the PAU Commission\\\ on the Philosophy of Science}{Z prac Komisji Filozofii Nauk PAU}
%\sekcja{Reports}{Sprawozdania}
\sekcja{Discussions and polemics}{Dyskusje i polemiki}

\begin{artengenv}{Igor Wysocki}
	{Rejoinder to Block on indifference\edtfootnote{This research was funded in whole or in part by the National Science Centre, Poland, grant number 2020/39/B/HS5/00610. For the purpose of Open Access, the author has applied a CC-BY public copyright licence to any Author Accepted Manuscript (AAM) version arising from this submission.}}
	{Rejoinder to Block on indifference}
	{Rejoinder to Block on indifference}
	{Nicolaus Copernicus University in Toruń\label{wysocki-rejoinder-firstpage}}
	{This paper is a~rejoinder to Block's
%	(2022)
	\parencite*{block_response_2022}
	response to Wysocki's
%	(2021)
	\parencite{wysocki_problem_2021}
	essay on Nozick's challenge leveled at Austrian economics. Instead of merely reiterating Wysocki's
%	(2021)
		\parencite{wysocki_problem_2021}
	position, we try to highlight that the Blockean account of indifference and preference entails the views which are otherwise unwelcome, given his unyielding commitment to Austrian economics at large. To wit, we argue that Block's theory still fails to make sense of the law of diminishing marginal utility. Moreover, his extreme idea of choice, sadly, appears to jettison characteristically Austrian subjectivism and thus perilously verges on behaviourism. We conclude that, given all these predicaments the Blockean account is caught in, Block himself (\textit{qua} Austrian) has a~reason to embrace the Hoppean theory of preference and indifference.
	}
	{choice, indifference, preference, Hans-Hermann Hoppe, Walter Block.}








\section{Introduction: the points of agreement }

\lettrine[loversize=0.13,lines=2,lraise=-0.03,nindent=0em,findent=0.2pt]%
{B}{}efore we embark on criticizing Block's account of preference and indifference, it is vital to underline the points of agreement between us and our intellectual adversary. This is important as it will allow us to all the more sharply capture the real bone of contention. What we, most crucially, share with Block is the view that indifference cannot be demonstrated in action 
%\label{ref:RNDGDZvdK1pBN}(see e.g., Block, 2009; Rothbard, 2011).
\parencites[see e.g.,][]{block_rejoinder_2009}[][]{rothbard_toward_1997}. %
 Indeed, the very idea of action presupposes \textit{some} preference. That is, as Mises 
%\label{ref:RND5ywImgIg4I}([1949] 1998, p.97)
\parencite*[][p.97]{mises_human_1998} %
 put it:



\begin{quote}
Action is an attempt to substitute a~more satisfactory state of affairs for a~less satisfactory one… A~less desirable condition is bartered for a~more desirable. What gratifies less is abandoned in order to attain something that pleases more.
\end{quote}



Granted, it is due to the fact that individuals judge a~state of affairs that would obtain in the absence of their respective actions to be \textit{less preferable} to the one that they believe would be brought about by these actions that they engage in acting in the first place. Or in other words, if an economic actor believed that her action would render her no better off than if she were not to act at all, she would refrain from acting. It is in this sense that action at the very minimum presupposes \textit{some} preference. Sweeping indifference would result in no action whatsoever---no disagreement with Block just yet.



What we also concur on with Block is the relation between the concepts of choice, preference and indifference. We, quite much in the Blockean spirit, conceive of the relation between the impossibility of choice and indifference as that of logical equivalence. That is, formally, for all S's, S~an economic agent, S~is indifferent between \textit{x} and \textit{y}\footnote{The variables \textit{x} and \textit{y}~are best treated as mere place holders, for they may stand for such various entities as states of affairs, physical objects, actions. After all, an individual may well be indifferent between (or have a~preference for) particular states of affairs (e.g. whether it is raining or not), physical objects (e.g. tea of coffee) and between specific actions (e.g. whether to start playing tennis with the \textit{left} or \textit{right} hand---see Hausman's 
%\label{ref:RNDAbNk6QYMA5}(2011, p.27)
\parencite*[][p.27]{hausman_preference_2011} %
 ``final preferences'' defined as ``preferences among the immediate objects of choice''). } if and only if S~cannot choose between \textit{x}~and \textit{y}. On the other hand, it takes S's \textit{preferring x} to \textit{y}~for S~to \textit{choose} \textit{x} over \textit{y}. Technically, the fact that S~chooses \textit{x}~over \textit{y}~implies that S~(strictly) prefers \textit{x}~to \textit{y}. And Block 
%\label{ref:RNDhvYEDz2muG}(2022, p.47)
\parencite*[][p.47]{block_response_2022} %
 concurs, which is manifested in the passage wherein Block invites us to consider the case of a~grocer endowed with a~stock of one-pound packages of butter who ``must choose one of these one-pound packages, to give to the thief/customer.'' The grocer then, we are supposed to imagine, ``chooses the first one''. Block's conclusion is that ``he is no longer indifferent.''



Clearly then, we are on the same page with Block as far as the view of choice as preference-implying is concerned. Furthermore, we take no issue with the characteristically Austrian contention to the effect that it is \textit{some preference} rather than indifference that manifests itself in action. However, the devil is in the details. And so there are indeed subtle points of disagreement between our account and Block's, the points to which we are now turning.



\section{The real bone of contention}

Although, as mentioned above, we side with Block as far as the thesis that choice implies preference goes, our more nuanced position concerning \textit{individuating} alternatives subject to choice finally makes it the case that our account of indifference and preference diverges from Block's dramatically. Just to remind the reader, our view is that if one is indifferent between \textit{x}~and \textit{y}, then one cannot logically choose between them. Or still in other words, if one cannot choose between \textit{x} and \textit{y}, then \textit{x}~and \textit{y}~do not constitute economically distinct alternatives.\footnote{This sort of insight---with a~slight modification---is also present in the mainstream theory of action. Says Broome 
%\label{ref:RNDrCnHMKcd8E}(1991, p.103):
\parencite*[][p.103]{broome_weighing_1991}: %
 ``Outcomes should be distinguished as different if and only if they differ in a~way that makes it rational to have a~preference between them.'' Hoppe 
%\label{ref:RND1hOhsohe2q}(2005)
\parencite*[][]{hoppe_must_2005} %
 advances a~similar thesis. This author has it that alternatives subject to choice should be considered distinct if and only if they differ in a~way that an actor \textit{does actually} have a~preference over them. And hence, if any two ``alternatives'' do not differ in any economically relevant sense according to the economic actor, then the two alternatives are not really alternatives. There is no choice between them. } To illustrate our point, if an actor S~values watching football most and he values going for a~walk equally highly, whereas he values playing a~game of chess less, while valuing having a~nap just as much as a~game of chess, we can represent his \textit{choices} on the following value scale:







\begin{enumerate}[label=(\arabic*)]

\item[]\makebox[-1.7em][l]{}V\textsubscript{1}

\item Watching football \textit{or} going for a~walk

\item Playing a~game of chess \textit{or} having a~nap

\end{enumerate}

As can be seen, there are only \textit{two} economically distinct choices instead of four of them. And again, the reason is that since the stipulated actor S~is indifferent between watching football and going for a~walk as well as between playing a~game of chess and having a~nap, he cannot choose between watching football and going for a~walk. Neither can he choose between playing a~game of chess and having a~nap. In conclusion, he chooses \textit{only} between (1) and (2).



Equipped with this conceptual apparatus, we are now in a~position to spell out a~relevant difference between our account of choice and Block's. At this point, it is crucial to note that the individual's given behaviour \textit{underdetermines} a~value scale on which she has acted. Or, to put this point more technically, there is a~one-to-many relation between a~certain act-token and an underlying value (preference) scale. Still in other words, a~given behaviour might be \textit{evidential} of many value scales. That is, (infinitely) many value scales may manifest themselves in any particular act. For example, suppose we know nothing yet of how our stipulated actor S~actually ranks the four ``alternatives'' stated above. Further imagine that S~ends up watching football. We posit that from this fact alone we cannot infer a~specific value scale guiding S's action. For, S~might as well have been indifferent between watching football and going for a~walk. Alternatively, he might have (strictly) preferred watching football to anything else he saw as a~possibility. If so, then his value scale might be the following:



%V\textsubscript{2}



\begin{enumerate}[label=(\arabic*)]

\item[]\makebox[-1.7em][l]{}V\textsubscript{2}

\item Watching football

\item Going for a~walk

\item Having a~nap

\item Playing a~game of chess

\end{enumerate}
And this is \textit{precisely} where our account diverges from Block's. For, it seems that according to Block action is a~manifestation of preference \textit{all across the board}. At this point, we cannot do better than quote Block at length. Says our author about the Buridan's ass example:



\begin{quote}
Wysocki misconstrues Buridan's ass in the same manner. This beast, let us say, chooses the bale of hay to the right. The correct interpretation of this is two fold: one, this creature preferred life to death, and, two, he favored the hay on the right to the hay on the left. In Wysocki's correct interpretation of Hoppe, and his own, only the first is true. The second, amazingly, is not. But, but, but, the donkey moved to his right, not his left! If this is not evidence that he preferred the right to the left bale, there can be no such thing as evidence, at least not in cases like this. 
%\label{ref:RNDWkzc44Dug7}(Block, 2022, pp.51–52)
\parencite[][pp.51–52]{block_response_2022}%
\end{quote}




First thing to note here is that Block is clearly strawmanning against Hoppe 
%\label{ref:RNDHzVDcSlRXd}(2005)
\parencite*[][]{hoppe_must_2005} %
 and Wysocki 
%\label{ref:RND5w7BMDb2ic}(2021).
\parencite*[][]{wysocki_problem_2021}. %
 Neither of these authors claim that it is impossible for the Buridan's ass to \textit{prefer} the right bale to the left. Rather, Hoppe's and Wysocki's point is that the fact that the donkey moves to his right is in and of itself insufficient to establish whether the donkey does \textit{prefer} the right bale to the left one. For, the donkey might as well be indifferent between the two. In that case, the donkey \textit{would not be choosing} between the two bales but indeed between something else---most plausibly, between eating or starving. Certainly, it is possible for the donkey to \textit{choose} between the bales. But in that case, the donkey must have a~\textit{preference} for one over the other. All in all, how many choices the actor faces depends on the Hoppean 
%\label{ref:RND4KiQoajixd}(2005)
\parencite*[][]{hoppe_must_2005} %
 \textit{correct description of action} (or action under intentional description) and not on the actor's behaviour as extensionally described. Whereas the fact that the donkey moves to the right is, for Block, a~decisive reason to conclude that the donkey \textit{prefers} the right bale to the left one, we submit that this fact alone does not suffice to establish what the donkey prefers over what as it takes an \textit{intentional} description of his action to be able to determine his preferences. Remember, we agree on one thing. The donkey's action most definitely is a~manifestation of \textit{some} preference, for otherwise the donkey would not engage in action at all. However, the donkey's particular behaviour underdetermines the value scale guiding his action. To summarize, the donkey's behavior being fixed (i.e. the animal moves to the right bale of hay and eats it), we contend that it is evidential of (at least) the following two value scales.



%V\textsubscript{3}



\begin{enumerate}[label=(\arabic*)]

\item[]\makebox[-1.7em][l]{}V\textsubscript{3}

\item Eat from a~right bale of hay

\item Eat from a~left bale of hay

\item Starve

\end{enumerate}

\begin{enumerate}[label=(\arabic*)]
%\item[or~V\textsubscript{4}]
\item[]\makebox[-1.7em][l]{}V\textsubscript{4}
\item Eat from either a~right \textit{or} a left bale of hay
\item Starve
\end{enumerate}

By contrast, Block avers that the donkey's behaviour unambiguously points to V\textsubscript{3} as an underlying value scale, which we can allegedly infer from the very fact that the animal moved to the right rather than to the left.



Having, hopefully, spelled out the difference between the Hoppean (and Wysocki's) and Block's account of preference of indifference, let us move now to consider why the Blockean theory leads to unwelcome consequences.



\section{Block's \textit{ad hoc} after-action/before-action distinction }

It is precisely Block's distinction between the time \textit{before} an action and \textit{after} it that constitutes the crux of his response. Block's 
%\label{ref:RNDqTsTHLpyG5}(2022, p.52)
\parencite*[][p.52]{block_response_2022} %
 discussion of his famous thought experiment involving a~seller endowed with 100 units of butter shall serve as a~good illustration of our intellectual adversary's viewpoint. Block appears to be relegating indifference entirely outside the realm of action as he believes that the said butter seller is indifferent between the units of his stock \textit{only} \textit{before} some action involving those units is taken. Says our author: ``At time t\textsubscript{1}, before any choice was made, yes, all units of butter were ``equally serviceable.'' Their owner was indifferent between all of them. They were homogeneous as far as he was concerned''. However, when at t\textsubscript{2} the seller encounters a~customer who is willing to buy one unit of the commodity supplied by the former, and the seller gives up 72\textsuperscript{nd }unit, then this very fact, according to Block, establishes that he indeed disprefers \textit{this} (i.e. 72\textsuperscript{nd}) unit to any other. Or, in Block's words, ``[if] this does not establish that he valued this particular one, the 72\textsuperscript{nd} unit, less than the others, then there is no such thing as choice, utility, economic theory, common sense.''



We, by contrast, contend that the inference from the fact of giving up \textit{a~particular} unit to the conclusion that this very unit must have been dispreferred to any other is rather, if anything, a~travesty of common sense. After all, why \textit{should} it be the case that the seller indeed \textit{chooses} to give up the 72\textsuperscript{nd} unit? Why does Block draw this conclusion? Merely because the \textit{extensional description} of the seller's action is that he gives up this very unit? Fair enough. As far as the extensional description goes, it is a~rather accurate one. However, it is still a~far cry from establishing the seller's action \textit{under intentional description}, for we do not know from this action alone between \textit{what} the seller was choosing. Just to resort to value scales, the seller's action might have been guided by (at least) these two distinct value scales.



%V\textsubscript{5}



\begin{enumerate}[label=(\arabic*)]

\item[]\makebox[-1.7em][l]{}V\textsubscript{5}

\item To earn money by giving up the 72\textsuperscript{nd }unit of butter

\item To earn money by giving any other unit\footnote{This value scale and the following one---unlike others invoked in the present paper---apart from the actor's ends include also the means. However, this illustrates the point that the actor---as in Block's example---might clearly have a~preference for particular means, with his end being fixed. After all, Block's point is precisely that, the seller's end being equal, she prefers to give up the 72\textsuperscript{nd} unit of butter to giving up any other. }

\end{enumerate}

%or indeed by V\textsubscript{6}



\begin{enumerate}[label=(\arabic*)]

\item[]\makebox[-1.7em][l]{}or indeed by V\textsubscript{6}

\item To earn money by giving up \textit{any} unit of butter

\item To preserve all the units and earn no money

\end{enumerate}







Then again, our position is that the seller's action underdetermines a~value scale guiding it. That is, for example, it might be V\textsubscript{4} or V\textsubscript{5} that make sense of the seller's behaviour. By contrast, according to Block, the fact that the seller gave up (as extensionally described) the 72\textsuperscript{nd }unit \textit{exclusively} points to V\textsubscript{4} as the scale guiding his action. But why should that be a~\textit{correct description of the seller's action}? We claim that the actor in question might as well be indifferent between \textit{all the units} of butter involved. Granted, when it came to the seller's action, he must have been guided by \textit{some} preference but this fact by itself cannot establish that he was guided---among other things---by the dispreference for the 72\textsuperscript{nd} unit of butter. And, we submit, it is all the more natural to assume that the seller was guided by the preference for some money over \textit{any} particular unit of butter. And this preference will do for classifying the seller's behaviour as action. There is no need at all to claim that the actor \textit{also} dispreffered the actual unit given up to any other.



Now, it is crucial to note that it is precisely Block's contention that from the act of giving up a~particular unit we can infer a~dispreference for that very unit that leads him to the weird eponymous after-action/before-action distinction. Remember, Block believes that the seller starts with indifference among all the units of butter. However, since he believes that the actor's act of giving up a~particular unit implies a~dispreference for that unit, he must \textit{now} posit that the actor is no longer indifferent among all the units of his commodity. Sadly, Block never explains why there is this sudden change in the actor's mental state. By contrast, the Hoppean account does not need to resort to the before-action/after-action distinction at all to explain the seller's act. If, by assumption, the actor is indifferent among all the units of butter, then his act does not (and cannot) demonstrate dispreference for the actual unit given up. But this does not prevent us from making sense of the actor's act. If the actor is genuinely indifferent among all the units of butter, his action might be still conceived of in terms of---among other possible explanations---the preference of giving up \textit{a}~unit of butter rather over preserving all of them but earning no money (see: V\textsubscript{5}). That is, in the Hoppean account, it is, most naturally, the actor's preference guiding the actor's action: if the actor prefers \textit{x}~to \textit{y,} he chooses \textit{x}~over \textit{y}, whereas if he is indifferent between \textit{a}~and \textit{b,} he does not choose between \textit{a}~and \textit{b}. More concretely, if he is indifferent between particular units of butter, then he does not choose between them. If he prefers some money to \textit{any} unit of butter, then he chooses to give up \textit{a}~unit of butter for some money. There is no need to postulate \textit{any} arbitrary change in the actor's state of mind to understand his resultant behaviour. Block, by contrast, is powerless to explain the actor's \textit{choice}, for how can he \textit{choose} to give up the 72\textsuperscript{nd} unit if the actor was \textit{ex hypothesi} indifferent between all of them. For Block to conclude that the said economic agent \textit{chose} to give up that very unit, it must be assumed that he was not indifferent between \textit{that unit} and any other one; viz., that he dispreferred precisely the 72\textsuperscript{nd }unit. But if Block were to embrace this assumption, he could not in turn make sense of the supply of the same economic good. Thus, Block seems to be caught in an unenviable dilemma. On the one hand, if he wants to stick to his idea of action as demonstrating \textit{preference all across the board}, he has to compromise the notion of the supply of the same good. Alternatively, if wants to keep the robust notion of the supply of the same commodity, he would need to make a~major concession to Hoppe. To wit, he would have to concede that the seller does \textit{not} disprefer the 72\textsuperscript{nd} of butter when he gives it up.



To illustrate further the dilemma the Blockean framework faces, let us test how it fares when given the task of capturing the law of diminishing marginal utility. Suppose, Block starts out with a~stock of three apples (A\textsubscript{1}, A\textsubscript{2}, A\textsubscript{3}), which he finds equally serviceable. Further, Block envisages exactly three ends that he believes \textit{each} apple can satisfy. The ends are (in the descending order of importance):



\begin{enumerate}

\item Eating an apple

\item Giving it to a~friend

\item Throwing it for distance

\end{enumerate}

Now, in Block's preferred vernacular, here is Block ``before action'', equipped with three units of the same commodity. He finds them all ``equally serviceable'' and thus he is indifferent between all of them. Now it is time for Block to satisfy his consecutive ends by means of the apples. Naturally, Block eats his \textit{first} apple, which satisfies his most pressing end. Say, he eats A\textsubscript{2}. This, however, according to Block already implies that \textit{in fact} A\textsubscript{2} was not equally serviceable as the remaining two apples. Nay, A\textsubscript{2} was dispreferred to the two apples remaining. So, it magically turns out that Block's act of eating one apple demonstrates that he was dealing not with a~homogeneous set of apples but with \textit{two} distinct classes of economic goods: (1) with the dispreferred apple he actually ate and (2) a~homogeneous set of two remaining equally serviceable apples. Secondly, Block quite reasonably gives one apple to his friend. Say, he gets rid of A\textsubscript{3} for that purpose. Now, since Block indeed gave up A\textsubscript{3}, this means that he dispreferred it to the remaining apple (i.e. A\textsubscript{1}). So, in the end, contrary to the original assumption, Block's subsequent actions demonstrate that in fact the three apples were not economically homogeneous. More, Block's inference is that they were \textit{all} heterogeneous. However, remember, the three apples were, \textit{by assumption}, homogeneous. After all, we were after illustrating the law of diminishing marginal utility using Block's preferred framework. As can be seen, Block's account of preference and indifference completely fails. In the above scenario of employing three apples, Block's theory predicts that there is \textit{only one} preferred way to economize them over time; that is, the one that actually obtained; viz, \textit{first} A\textsubscript{2}, \textit{second} A\textsubscript{3}, and \textit{finally} A\textsubscript{1}. However, as demonstrated by Wysocki 
%\label{ref:RNDejJ6jqT9YM}(2021, p.41),
\parencite*[][p.41]{wysocki_problem_2021}, %
 we should expect 3! (which is six) ways to economize those three apples. After all, since they are assumed to be equally serviceable, then it would be---by assumption---equally good for Block to, say, first employ A\textsubscript{1}, then A\textsubscript{2} and finally A\textsubscript{3}. The same applies to \textit{any permutation} of the said three apples. How can it be otherwise when they are assumed to be equally serviceable? Finally, it is well-worth noting that the Hoppean account does not run into the same sort of problem, for, according to Hoppe, since the agent would be indifferent between three apples he would not choose \textit{among} them. Still, he would \textit{choose} between different ends each apple can satisfy. That is, as in the scenario above, the actor would first eat \textit{an(y)} apple, then give \textit{any} other of the two remaining apples to a~friend, and finally throw the remaining apple for distance. Hence, the actor would be throughout the process indifferent between the apples (means employed), while at the same time demonstrating \textit{some} preference (i.e. satisfying more pressing needs sooner later than later). Therefore, it is the Hoppean account and not Block's that does justice to both the fact that the agent was acting (i.e. there is \textit{some preference} getting demonstrated) and to the law of diminishing marginal utility (i.e. the apples are deemed equally serviceable through the whole sequence of actions). Concluding, given the fact that Block \textit{qua} Austrian fully subscribes to the law of diminishing marginal utility, he would do better to drop his before-action/after-action distinction as it seems to jeopardize the said law, clearly too high a~price to pay. Needless to say, the Hoppean account suffers from no such defects and so Block has all the reason to embrace it. Having said that, it is time to elucidate other problems the Blockean theory suffers from.



\section{Agency is not all about strict preference}

Another problem haunting Block's response is not taking heed of the distinction between \textit{agency} and what the actor does \textit{under an intentional description}.\footnote{The distinction being brilliantly illuminated by Davidson 
%\label{ref:RNDq7F3EvNAnI}(2001).
\parencite*[][]{davidson_agency_2001}. %
 } What motivates this distinction is that apparently an \textit{extensional} description of the agent's action does not necessarily coincide with its \textit{intentional description}. To wit, not every single aspect of the agent's external behaviour (at some level of description) is such that she intends it. To briefly illustrate the distinction yet again, let us analyse a~rather typical script of entering a~café to order coffee. So, as \textit{extensionally} described, the customer normally enters a~café with a~particular foot (either left or right one is the \textit{first} to enter the desired area). However, it certainly does not follow that \textit{once} the agent enters the café with her left foot, she thereby demonstrates her preference for entering with this particular foot to entering with the other one. For, the \textit{content} of the agent's intentional state (i.e. of \textit{what} the agent \textit{intends} to do) might be simply to enter the café with the ways of entering it being left unspecified. Similar remarks apply to the agent's ordering a~coffee. Suppose, the waiter approaches our economic actor and the latter says: ``I will have a~large cappuccino.'' It definitely does not follow that the actor had some preference for \textit{this particular} wording of her order over any other. That is to say, as long as \textit{any} wording constitutes a~speech act of ordering a~coffee, the actor might be perfectly indifferent between alternative ways of ordering the desired drink. Moreover, at still some finer-grained level of description, our actor's pronouncing her order necessarily has a~suprasegmental property of having a~definite pitch. For the actor might order a~coffee by pronouncing her order at, say, a~very high pitch. But then again, why should that follow that the agent did indeed intend to place an order at a~high pitch. She might as well \textit{simply} wanted to place an order (with the pitch remaining unspecified in her intentional state). But if so, then there is no reason to assume that the fact that the actor's order was delivered at a~high pitch demonstrates her preference for \textit{that} pitch over any other. By contrast, Block's position seems to predict that \textit{since} the agent does indeed enters with, say, the right foot, this \textit{ipso facto} is evidentiary of her preference for this particular way of entering the café. By the same token, the fact that the agent orders a~cappuccino at a~high pitch is, for Block, indicative of the agent's (strict) preference for \textit{that} pitch over any alternative one. Yet, Block's conclusion is implausible. Clearly, one cannot apodictically infer a~(strict) preference for such minute details of action-tokens as highly specific bodily movements or highly specific features of our linguistic behaviour. And the reason is that entering a~café \textit{with a~particular foot} would not typically figure in the content of our intentional states. Rather, the most natural description of the actor's practical syllogism\footnote{For an excellent elaboration on practical syllogism, see e.g. Moore 
%\label{ref:RNDB3OpSAI2IK}(1993; 2020).
\parencites*[][]{moore_act_1993}[][]{moore_mechanical_2020}. %
 } is the following. She \textit{desired to drink a~coffee} and because she \textit{believed} that by entering a~(particular) café she can satisfy her desire, she \textit{intended} to enter it. Under this description, the agent does not believe that it is \textit{only} by entering a~café with a~particular foot that she can ultimately satisfy her desire for coffee. Hence, neither does she \textit{intend} to enter the place with a~particular foot. She simply intends to walk in whether with her left or right foot. And because a~particular way of entering (i.e. either with the left or right foot) is outside the content of the agent's \textit{intentional states} (both her \textit{belief} and \textit{intention}), it would be far-fetched to infer the agent's preference for a~particular way of walking in merely from the fact that the agent \textit{in fact} does enter with a~particular foot.\footnote{Note that the Hoppean 
%\label{ref:RNDSIdEb6znlW}(2005)
\parencite*[][]{hoppe_must_2005} %
 account does not prevent us from saying that the agent described does indeed have a~preference for a~particular way of walking in. However, this preference does not, for Hoppe, follow automatically from the fact that the agent walks in with a~particular foot. According to Hoppe, the ultimate test for agent's preference is the correct description of her action, which coincides with the Davidsonian 
%\label{ref:RNDCQ94M0xhSP}(2001)
\parencite*[][]{davidson_agency_2001} %
 \textit{intentional description of an action}. } Such an inference would, to our mind, make a~mockery of preferences. If the economic agent strictly prefers \textit{A}~to \textit{B,} she values \textit{A}~higher than \textit{B}. Why should it be apodictically true then that if our actor enters a~café with her left foot rather than with right one, this demonstrates that she \textit{values} this particular entrance (i.e. with the left foot) higher than the alternative entrance with the right foot? It is most implausible to claim that \textit{this} particular valuation immediately follows. Surely, we are ready to concede that some differential valuation follows from the very fact that the agent is acting in the first place. As we insisted on above, action implies the demonstration of \textit{some} preference but that is everything that follows with apodictic certainty from the fact that the agent acts. Block's conclusion is therefore illegitimate and clearly too strong. And just as entering a~café with a~left foot is not normally preferred to entering it with a~right foot, so these two action-tokens do not normally---\textit{contra} Block---constitute two distinct choices. And again, insisting that the agent \textit{does choose} to enter a~café with her right foot because she \textit{actually} entered it with her right foot is to make the same mistake as the one involving the inference to the actor's preference mentioned above. After all, the agent does not have to \textit{conceive} of these two alternative ways of walking in as \textit{relevantly} different. Either, she may well believe, will serve her end equally well.



Finally, let us have a~look at Block's 
%\label{ref:RNDP2kuPsd6a0}(2022, pp.50–51)
\parencite*[][pp.50–51]{block_response_2022} %
 analysis of the Hoppean example involving a~poor mother who can rescue only one of her sons (i.e. either Peter or Paul) as the said analysis aptly illustrates the Blockean confusion between agency and intentional description of an action and allows us to raise our final objection to his theory. As expected, from the fact that the mother saves Peter Block draws an inference to the conclusion that she ``places a~higher value on Peter than Paul.'' But then again, just as---as we already saw---one cannot infer the preference for entering a~café with a~right foot from the fact the agent does actually enter with that very foot, so we cannot infer the mother's preference for Peter over Paul from the very fact that Peter was saved. As we reiterated throughout this essay, the fact that the mother saves Peter (extensional description) underdetermines the value scale guiding the mother's action, for the mother might equally well frame her end as saving \textit{a}~child rather than saving Peter. And if the former is true, then saving Peter serves this end equally well as saving Paul. That is why, she can remain (before and after action) indifferent between the two of her sons. And it is precisely for that reason that she does not (and cannot) choose between the two. No contradiction here.



However, Block 
%\label{ref:RNDiujECtkP9I}(2022, pp.50–51)
\parencite*[][pp.50–51]{block_response_2022} %
 protests: ``She did rescue the former, when she could have chosen differently, and selected the latter for retrieval, did she not?'' But this simply begs the question. We, following Hoppe, contend that the mother's action in and of itself is not determinative of the mother's value scale, for the mother might as well simply prefer rescuing \textit{a}~child to saving \textit{none}. And if the mother frames her ends in this way, then it logically follows that the mother does not choose between Peter and Paul. Rather, in this scenario, the mother is choosing between saving \textit{a}~child over saving \textit{none}. And that is why Block's assertion does no more than beg the question.



Eventually, to add insult to the injury, Block 
%\label{ref:RNDGFu0QQRTG5}(2022, p.51)
\parencite*[][p.51]{block_response_2022} %
 adds that even if the mother ``did this with her eyes closed, and just grabbed the nearest son'', this would still indicate that the mother chose to save Peter. Yet, how can grabbing a~certain son with one's eyes closed count as demonstration of preference for that son? If anything, it seems that under that scenario the mother prefers grabbing \textit{any one} son over saving none. It appears as though the most charitable take on the Blockean idea of choice is that the author---his protestations to the contrary notwithstanding---embraces methodological behaviourism.\footnote{Granted, Block may not be an ontological behaviorist. That is, he clearly does \textit{not} deny the existence of mental states. Neither does he reduce them to behaviours or mere dispositions to behaviour. However, he seems to \textit{model} (or define?) preferences in terms of the agent's external behaviour 
%\label{ref:RNDsQH9mf95KK}(Block, 2022, pp.54–55)
\parencite[][pp.54–55]{block_response_2022}. %
This, to our mind, looks very much like \textit{methodological behaviourism}, the view according to which positing mental states adds nothing to understanding the individual's external behaviour. As we are about to see to in the forthcoming part of the text, the Blockean construal of Peter-and-Paul scenario appears to abstract from the mother's preferences (as genuine mental states) completely and instead models the mother's apparent choice \textit{solely} around her external behaviour. For an exposition of different senses of behaviourism, see e.g. Moore 
%\label{ref:RNDPX5r3KaLmT}(2001).
\parencite*[][]{moore_distinguishing_2001}. %
 } For, if the mother were to indeed ``choose'' to save Peter with her eyes closed (i.e. being completely unaware of who she is in fact saving), in what sense is this ``choice'' even driven by preferences or any other mental states for that matter. We are afraid, in none. Rather, with her eyes closed, the mother simply \textit{happens} to save Peter. It is not the case, by stipulation, that she \textit{believed} that she is saving Peter. Worse, Block even goes to such great lengths to say that the mother does not even have to cherish an \textit{intention} to save \textit{either} of his sons for her act to count as an evidence that she \textit{chose} to save Peter. Says Block 
%\label{ref:RNDMG5ZBJhKaS}(2022, pp.54–55):
\parencite*[][pp.54–55]{block_response_2022}:%




\begin{quote}
we as praxeologists must note that you actually reached out and grabbed one of them, not the other. This is the essence of Hoppe's error, with support from Wysocki. What might well have been on her mind had nothing to do with Peter nor Paul. It might well have been as Hoppe opined, she was just preferring to save one of her sons, rather than none. Who knows, she might have been thinking about ice cream, as far as we praxeologists are concerned. This does not matter in the slightest for the praxeologist. We see her grabbing Peter, not Paul, to safety, and we are compelled by praxis logic, e.g., praxeology, to note that she was not indifferent between her sons, she could not have been indifferent between them, given that she chose the one, not the other.
\end{quote}



But this radical view comes perilously close to methodological behaviourism, for Block seems to dismiss the mother's mental states completely. Note, even if the mother were to think ``about ice cream'', she would still choose to save Peter in the event Peter would be ultimately saved. But this at a~stroke gives up characteristically Austrian methodological subjectivism\footnote{Let us not lose sight of Hayek's 
%\label{ref:RNDWqnr4QQKwE}(1952, p.31)
\parencite*[][p.31]{hayek_counter-revolution_1952} %
 famous dictum: ``It is probably no exaggeration to say that every important advance in economic theory during the last hundred years was a~further step in the consistent application of subjectivism.'' } and denies any role to the actor's mental states (preferences and beliefs) as determining choices. Again, Block's die-hard insistence on his radical idea of choice appears at the same time to compromise what he otherwise holds dearly, that is Austrian subjectivism with its insistence on \textit{purposeful} behaviour. Given this, we again submit that for Block to disown his account of choice is to pay a~relatively small price for saving what he \textit{qua} Austrian otherwise strongly believes. In other words, we claim that the most efficient way for Block to make his views coherent is to drop his problematic theory of choice, preference and indifference.



\section{Conclusion}

As we tried to show in this rejoinder, Block's account of choice, preference and indifference fails on three counts. First, Block's theory---despite his claims to the contrary notwithstanding---cannot make sense of the law of diminishing marginal utility. For it is precisely the Blockean radical idea of choice which predicts that allegedly homogeneous (i.e. equally serviceable) units ultimately prove to be heterogenous. Moreover, we demonstrated that Block's resorting to the before-action/after-action distinction is of no help. Not only is this distinction \textit{ad hoc} but also it fallaciously predicts that \textit{n} number of allegedly equally serviceable units can be economized in only \textit{one} optimal way, something immediately running counter to the original assumption of the economic homogeneity of the said units.



Later on, we illuminated two more unwelcome consequences on the Blockean theory under consideration. The first of them is that Block's 
%\label{ref:RNDmxlz1FgNA6}(2022)
\parencite*[][]{block_response_2022} %
 account fails to distinguish between what is attributable to the economic agent's \textit{agency} and what the agent does \textit{intentionally}. While trying to reduce Block's not observing this distinction to absurdity, we show that this author would have to conclude that \textit{literally} any single minute detail of the actor's act-token is preferred (to some other minute detail) and therefore chosen. This conclusion, in turn, is most clearly implausible, which serves to repudiate the Blockean theory of choice \textit{via modus tollens}.



Finally, we suggested that Block's theory dangerously verges on methodological behaviourism, the view that this author most definitely rejects \textit{qua} Austrian. Given all these unwelcome consequences stemming from Block's insistence on his account of choice, preference and indifference, we claim that this author has a~decisive reason to simply disown the said account. After all, as it seems, this particular theory of his is purchased at a~huge cost of jeopardizing other vital aspects of Austrian economics, especially the law of diminishing marginal utility and overall Austrian insistence on methodological subjectivism rather than methodological behaviourism. Needless to say, embracing the Hoppean 
%\label{ref:RNDBx29EHZukE}(2005)
\parencite*[][]{hoppe_must_2005} %
 account of preference and indifference would be a~right way for Block to go.





\end{artengenv}\label{wysocki-rejoinder-lastpage}


\begin{artengenv}{Walter Block}
	{Response to Wysocki's \textit{Rejoinder to Block on indifference}}
	{Response to Wysocki's \textit{Rejoinder to Block on indifference}}
	{Response to Wysocki's \textit{Rejoinder to Block on indifference}}
	{Loyola University New Orleans\label{block-rejoinder-firstpage}}
	{Wysocki 
	%\label{ref:RNDxkl9Os0nTO}(2024)
	\parencite*[][]{wysocki_rejoinder_2024} %
	 is a~critique of Block 
	%\label{ref:RNDrhLTsy344h}(2022).
	\parencite*[][]{block_response_2022}. %
	 The present paper is a~response to the former. We are in effect debating the best reaction to Nozick 
	%\label{ref:RNDzIdYuP1FWa}(1977)
	\parencite*[][]{nozick_austrian_1977} %
	 which criticized Austrian economics on the ground that it makes two claims that are incompatible with one another. On the one hand, the praxeological school is noted for its aversion to the concept of indifference. On the other hand, the Austrian school also accepts supply and demand curves, and diminishing marginal utility. These three concepts imply homogeneous elements that comprise them. But if they are truly homogeneous, people ought to be indifferent between the different elements of them. Hence, the tension, not to say logical contradiction, in this perspective. Block 
	%\label{ref:RND05at9zeX1m}(1980)
	\parencite*[][]{block_robert_1980} %
	 was an attempt to respond to Nozick 
	%\label{ref:RNDwfdndOLuML}(1977).
	\parencite*[][]{nozick_austrian_1977}. %
	 Hoppe 
	%\label{ref:RND2B8CvVtdhG}(2005a; 2005b; 2009)
	\parencites*[][]{hoppe_note_2005}[][]{hoppe_must_2005}[][]{hoppe_further_2009} %
	 and Wysocki 
	%\label{ref:RND74fJDSNrfQ}(2016; 2017; 2021; 2024)
	\parencites*[][]{wysocki_indifference_2016}[][]{wysocki_caplan_2017}[][]{wysocki_problem_2021}[][]{wysocki_rejoinder_2024} %
	 who supports Hoppe, maintain that Block's refutation of Nozick 
	%\label{ref:RND4Fu3WZTtSh}(1977)
	\parencite*[][]{nozick_austrian_1977} %
	 was not efficacious at all, at worst, or at best, certainly not fully successful.
	
	
	
	Specifically, Wysocki maintains that there is a~bifurcation between choosing and preferring; for example, no one is even aware of which foot goes first when entering a~restaurant, and, yet, one has to make a~choice about it. He avers that it is entirely possible to prefer to save either son, equally, while actually picking one, and not the other.
	}
	{indifference, supply and demand, diminishing marginal utility, subjectivism, behaviorism, psychologizing, preference.}




\section{Introduction}



\lettrine[loversize=0.13,lines=2,lraise=-0.03,nindent=0em,findent=0.2pt]%
{W}{}ysocki 
%\label{ref:RNDFqfRxaFZYj}(2024)
\parencite*[][]{wysocki_rejoinder_2024}%
\footnote{Unless otherwise indicated, all cites to Wysocki will be to this one article.} is a~critique of Block 
%\label{ref:RNDtFFctKpzqN}(2022).
\parencite*[][]{block_response_2022}.%
\footnote{And, also critique of 
%\label{ref:RNDTWMTNkwaq4}(Barnett, 2003; Block and Barnett II, 2010, and Block, 1980; 2009a; 2009b; 2012; 2019).
\parencites[][]{barnett_modern_2003}[][and Block]{block_robert_1980}[][]{block_rejoinder_2009}[][]{block_rejoinder_2009-1}[][]{block_response_2012}[][]{block_rejoinder_2019}.%
} The present paper will defend the latter against the former. My critic alleges that I~err with regard to indifference, subjectivism, Austrian economics in general, the law of diminishing marginal utility, and embraces the fallacy of behaviorism. I~shall be quoting sections of his 2024 paper, and responding to them, seriatim, in much the same order as he presents his critiques. This author urges me in my critique of Nozick 
%\label{ref:RNDVrxw8YTq0i}(Block, 1980)
\parencite[][]{block_robert_1980} %
 to embrace, instead, the defense of the praxeological school adumbrated by Hoppe 
%\label{ref:RNDASKOGpDFCK}(2005a; 2005b; 2009).
\parencites*[][]{hoppe_note_2005}[][]{hoppe_must_2005}[][]{hoppe_further_2009}.%
\footnote{For a~more complete list of Austrian concerns about indifference, see 
%\label{ref:RNDp8NFlKliI7}(Barnett, 2003; Block, 1980; 1999; 2003; 2007; 2009a; 2009b; 2012; 2019; Block and Barnett II, 2010; Block and Sotelo, 2012; Caplan, 2024; 1999; 2001; 2003; 2008; Hoppe, 2005b; 2005a; 2009; Hülsmann, 1999; Machaj, 2007; Nozick, 1977; O'Neill, 2010; Sotelo and Block, 2014; Wysocki, 2016; 2017; 2021; 2024).
\parencites[][]{barnett_modern_2003}[][]{block_robert_1980}[][]{block_austrian_1999}[][]{block_realism_2003}[][]{block_reply_2007}[][]{block_rejoinder_2009}[][]{block_rejoinder_2009-1}[][]{block_response_2012}[][]{block_rejoinder_2019}[][]{block_rejoinder_2010}[][]{block_response_2012}[][]{sotelo_indifference_2014}[][]{caplan_austrian_1999}[][]{caplan_probability_2001}[][]{caplan_probability_2003}[][]{caplan_trojan_2008}[][]{hoppe_must_2005}[][]{hoppe_note_2005}[][]{hoppe_further_2009}[][]{hulsmann_economic_1999}[][]{machaj_praxeological_2007}[][]{nozick_austrian_1977}[][]{oneill_choice_2010}[][]{sotelo_indifference_2014}[][]{wysocki_indifference_2016}[][]{wysocki_note_2017}[][]{wysocki_problem_2021}[][]{wysocki_rejoinder_2024}.%
}



In section II of this paper, we discuss my overlap with the views of Wysocki. Section III is given over to exploring the bones of contention between the two of us. The burden of section IV is to counter Wysocki's rejection of my time series claim: the after-action/before-action distinction. In section V~we discuss agency and strict preference; we conclude in section VI.



\section{Agreement}

Wysocki begins his paper by setting out the large areas of agreement. He states this fairly, even eloquently. I~have no criticism. However, I~would go further than he along these lines. He limits himself, not unreasonably to the narrow points at issue in this particular paper. I~would like to place it on the record that he and I~probably agree on 99\% of all issues in political economy. I~go even further: he and I~are co-authors on several occasions, and you don't get closer in this game than that 
%\label{ref:RNDXk5QPfj72e}(Block and Wysocki, 2018; Wysocki and Block, 2017; 2018; 2019; 2020; 2022; Wysocki, Block and Dominiak, 2019).
\parencites[][]{block_defense_2018}[][]{wysocki_note_2017}[][]{wysocki_analysis_2018}[][]{wysocki_homogeneity_2019}[][]{wysocki_crovelli_2020}[][]{wysocki_rejoinder_2022}[][]{wysocki_homogeneity_2019}.%




\section{Bones of contention}

Wysocki starts off this section of his paper with the



\begin{quote}
… view […] that if one is indifferent between x~and y, then one cannot logically choose between them. Or still in other words, if one cannot choose between x~and y, then x~and y~do not constitute economically distinct alternatives.
\end{quote}



The word, or the concept, ``indifference'' occurs in ordinary language all the time.\footnote{Physics, too, has a~technical language, which uses the same verbiage as ordinary language. For example, in that science, ``work'' = mass x~distance. But if someone is holding 20 pound barbells still, at arm's length, he will not be doing any ``work'' in the physics sense, since these weights do not travel through any distance. However, in ordinary language, this would constitute a~very heavy ``work'' out. } Before choosing, the grocer cares not one whit which pound of butter he gives to the customer. In Wysocki's case, chess versus taking a~nap, football vis a~vis going for a~walk, all is indifferent. Now consider technical language. We as economists, are never in a~position to say anything of the sort. All we see is the person choosing either football or a~walk, or choosing either chess or a~nap. We are not in a~position to aver, qua economist, any such thing as does Wysocki. Again, if this scholar is engaging in ordinary language, I~have no quarrel with his contention. But, if he thinks he is now speaking as an economist, which he now presumably also is doing, then I~cannot acquiesce in his statements.



Our author opines:



\begin{quote}
Equipped with this conceptual apparatus, we are now in a~position to spell out a~relevant difference between our account of choice and Block's. At this point, it is crucial to note that the individual's given behaviour underdetermines a~value scale on which she\footnote{I~cannot let pass my extreme annoyance at Wysocki's continual use of ``inclusive'' language. ``He'' includes ``he and she'' in the English language, whereas ``she'' includes only the fairer sex. However, I~forgive him. English is not his native language. He does not, then, perhaps, realize the importance of maintaining it as it was before the untoward influence of the feminists.} has acted. Or, to put this point more technically, there is a~one-to-many relation between a~certain act-token and an underlying value (preference) scale.
\end{quote}



My objection, is that no one can be indifferent between a~walk and football watching, and, yet, does the latter. This would appear to be a~violation of the areas of agreement that Wysocki acknowledges we share. To wit, he states: ``What we, most crucially, share with Block is the view that indifference cannot be demonstrated in action'' 
%\label{ref:RNDD35b9jV2NC}(see e.g., Block, 2009a; 2009b; Rothbard, 2011).
\parencites[see e.g.,][]{block_rejoinder_2009}[][]{block_rejoinder_2009-1}[][]{rothbard_toward_2011}. %
 Indeed, the very idea of action presupposes some preference. He now seems to be taking back this area in which we overlap.



Wysocki, nevertheless, continues down this path:



\begin{quote}
Alternatively, he might have (strictly) preferred watching football to anything else he saw as a~possibility. If so, then his value scale might be the following:
\vspace{-.7em}
\begin{enumerate}[label=(\arabic*)]

\item[]\makebox[-1.7em][l]{}V\textsubscript{2}

\item Watching football

\item Going for a~walk

\item Having a~nap

\item Playing a~game of chess

\end{enumerate}
\vspace{-.7em}
And this is precisely where our account diverges from Block's. For, it seems that according to Block action is a~manifestation of preference all across the board.
\end{quote}



The fact that I~choose bubblegum instead of readily available chewing gum is not \textit{insufficient} , as Wysocki avers, to establish that I~prefer the former to the latter. The fact that I~choose to propose marriage to woman A~instead of woman B~is not \textit{insufficient} to establish that I~prefer the former to the latter. The fact that I~choose biking instead of running is not \textit{insufficient} to establish that I~prefer the former to the latter. What, then, would be sufficient to demonstrate any of these claims? Let me repeat that just to make sure I~comprehend what he is saying. He is saying, and I~quote: ``the fact that the donkey moves to his right is in and of itself \textit{insufficient} to establish whether the donkey does prefer the right bale to the left one.'' I~find this highly problematic.



My debating partner explains:



\begin{quote}
For, the donkey might as well be indifferent between the two. In that case, the donkey would not be choosing between the two bales but indeed between something else -- most plausibly, between eating or starving.
\end{quote}



One possibility is to say that of course the animal is choosing life over death. But he is \textit{also} choosing right over left. This cannot be doubted, in the face of the fact the he\footnote{Not she.} actually moved to the right. I~cannot for the life of me see how a~movement to the right does not demonstrate preference for a~movement to the right, always assuming of course that there was no outside interference, such as being whipped on the left side, or anything like that. Does this also demonstrate, as Wysocki and Hoppe would have it, that this also reveals a~preference for life over death? That is a~bit of a~stretch. It is easy to avoid. \textit{Perhaps this is the burro's way of committing suicide, via overeating.} Wysocki and Hoppe are ignoring the basic element, a~move to the right indicates a~preference for a~move to the right. Instead, they are grasping at straws, maintaining that it necessarily discloses a~preference for life over death. It does no such thing, on the assumption that asses can die from overeating. My two Austrian colleagues focus on life and death, which is, in the best of cases for their side of the argument, uncertain. And they ignore what is directly in their faces: a~move to the right has to indicate something. And it could not possibly be anything else other than that the donkey preferred the hay to the right. It is as if they are asked ``why did the chicken cross the street?'' and they ignore the obvious answer: ``to get to the other side,'' and, instead, speculate on all sorts of irrelevancies: to save its life; to play chess with another chicken who lives across the street, etc.



But Wysocki is in no mood to concede anything to the arguments just made. Rather, he continues as follows:



\begin{quote}
Certainly, it is possible for the donkey to choose between the bales. But in that case, the donkey must have a~preference for one over the other. All in all, how many choices the actor faces depends on the Hoppean 
%\label{ref:RNDjuwpa2pyCF}(2005b)
\parencite*[][]{hoppe_must_2005} %
 correct description of action (or action under intentional description) and not on the actor's behaviour as extensionally described.
\end{quote}



I'm a~behaviorist because I~look to behavior to ferret out values, preferences. But human action constitutes behavior.



Nothing loath, Wysocki repeats this erroneous interpretation:



\begin{quote}
Whereas the fact that the donkey moves to the right is, for Block, a~decisive reason to conclude that the donkey prefers the right bale to the left one, we submit that this fact alone does not suffice to establish what the donkey prefers…
\end{quote}



My many times co-author puts his point more formally, by creating these two value scales:



\begin{quote}
\begin{enumerate}[label=(\arabic*)]
\item[]\makebox[-1.7em][l]{}V\textsubscript{3}
\item Eat from a~right bale of hay
\item Eat from a~left bale of hay
\item Starve
\end{enumerate}
\vspace{-.7em}
\begin{enumerate}[label=(\arabic*)]
%\item[or~V\textsubscript{4}]
\item[]\makebox[-1.7em][l]{}V\textsubscript{4}
\item Eat from either a~right \textit{or} a left bale of hay
\item Starve
\end{enumerate}
\end{quote}


He then states:



\begin{quote}
By contrast, Block avers that the donkey's behaviour unambiguously points to V3 as an underlying value scale, which we can allegedly infer from the very fact that the animal moved to the right rather than to the left.
\end{quote}



But maybe the donkey is a~right winger, and detests any move to the left. If so, V4 cannot be correct. More seriously, yes, V4 is correct, assuming away the suicide by overeating scenario. But just because V4 is correct does not logically imply that V3 is false. \textit{Both} could be true.



\section{The after-action/before-action distinction }

Wysocki now uses my time series analysis (indifference can exist before choices are made and preferences revealed, but not afterward), as a~vehicle to demonstrate my errors.



This Polish philosopher-economist again puts matters in a~formal manner:



\begin{quote}
\begin{enumerate}[label=(\arabic*)]
\item[]\makebox[-1.7em][l]{}V\textsubscript{5}
\item To earn money by giving up the 72\textsuperscript{nd }unit of butter
\item To earn money by giving any other unit
\end{enumerate}
\vspace{-1em}
\begin{enumerate}[label=(\arabic*)]
\item[]\makebox[-1.7em][l]{}or indeed by V\textsubscript{6}
\item To earn money by giving up \textit{any} unit of butter
\item To preserve all the units and earn no money
\end{enumerate}
\end{quote}



Wysocki clearly prefers V6 to V5. He states: ``Granted, when it came to the seller's action, he must have been guided by some preference but this fact by itself cannot establish that he was guided -- among other things -- by the dispreference for the 72\textsuperscript{nd} unit of butter.''



But this leaves open the question of why, then, did the grocer seize upon that precise unit of butter, if that was not the one he most wanted to get rid of, as demonstrated by his specific action of choosing that one to sell. To be sure, V6 will suffice as an accurate depiction \textit{before} the grocer's actual decision. But afterward, it is difficult to maintain V6, vis a~vis V5, given that only V5 is based on \textit{all} the facts in this case; that is, not only did he want to sell a~unit, any unit, of butter to the customer, but, also, in the event, he selected that 72\textsuperscript{nd} unit, and not any other. In contrast, V6 leaves out this fact. Again, as in V3 and V4, this author is wrongly concluding from the fact that V6 is correct that V5 is false. Both could be truthful.



Wysocki is not finished with his analysis, not by a~long shot:



\begin{quote}
Granted, when it came to the seller's action, he must have been guided by some preference but this fact by itself cannot establish that he was guided -- among other things -- by the dispreference for the 72\textsuperscript{nd} unit of butter. And, we submit, it is all the more natural to assume that the seller was guided by the preference for some money over any particular unit of butter.
\end{quote}



I~am trying to defend Austrian economics against Nozick's 
%\label{ref:RNDFRu8xur1Xy}(1977)
\parencite*[][]{nozick_austrian_1977} %
 critique of it. This eminent philosopher claims that if Austrians want to make use of supply and demand curves, diminishing marginal utility, we must, kicking and screaming if need be, acknowledge indifference as a~technical matter. I~go part way along in the direction Nozick lays out: yes, there is such a~thing as indifference, but it only applies \textit{before} human action occurs; before the grocer chooses a~pound of butter to sell. Afterwards, this can no longer be the case, for the grocer, must, of necessity, chose a~specific one pound of butter to rid himself of. He cannot, logically cannot, select a~non-specific unit of butter of which to rid himself. If that is to not disprefer it, then there simply is no such thing as preference and dispreference, which there certainly is, whenever we engage in human action. So, yes, I~accept V6; but I~also insist not only that V5, too, is correct, but that there is a~great ``need'' to maintain its truth.



According to Wysocki:



\begin{quote}
Now, it is crucial to note that it is precisely Block's contention that from the act of giving up a~particular unit we can infer a~dispreference for that very unit that leads him to the weird eponymous after-action/before-action distinction. Remember, Block believes that the seller starts with indifference among all the units of butter. However, since he believes that the actor's act of giving up a~particular unit implies a~dispreference for that unit, he most now posit that the actor is no longer indifferent among all the units of his commodity. Sadly, Block never explains why there is this sudden change in the actor's mental state.
\end{quote}



I~did exactly that in Block 
%\label{ref:RNDm6zpDVBqmm}(1980; 2009a; 2009b; 2012; 2019; 2022)
\parencites*[][]{block_robert_1980}[][]{block_rejoinder_2009}[][]{block_rejoinder_2009-1}[][]{block_response_2012}[][]{block_rejoinder_2019}[][]{block_response_2022} %
 but let me take up Wysocki's present invitation to do so once again. In my view, before the customer came into the store, the grocer was not thinking about which of his 100 packages of butter he liked most or least. If he would have asked himself at that time about his assessment of his butter stock, he would have told himself he was indifferent between them. So far no human action. Then, the customer arrives and asks for one package of butter. The grocer grabs the 72\textsuperscript{nd} one. We as Austrian economists have not one but two things to account for. First, there is the fact that he grabbed \textit{any} unit of butter. The answer is obvious: he preferred the money to this product. So far, Hoppe and Wysocki go along with me on this. But, second, we have to account for the fact that he selected \textit{this particular} element of his stock. Here, these economists steadfastly refuse to answer. They say it is not necessary to respond. They maintain that the first question is all that ``needs'' to be answered. I~cannot budge them from this position. But it seems clear to me that both questions are on the table, and that we are remiss if we refuse to answer both, decline even to contemplate each of them. Wysocki states, ``the Hoppean account does not need to resort to the before-action/after-action distinction at all to explain the seller's act.'' This account explains why the grocer prefers the money to any pound of butter, but not why this particular unit was chosen.



Continues Wysocki:



\begin{quote}
If, by assumption, the actor is indifferent among all the units of butter, then his act\footnote{Of selecting a~unit to sell.} does not (and cannot) demonstrate dispreference for the actual unit given up.
\end{quote}



My response is that the human actor \textit{is} not indifferent at the point of sale; rather, he \textit{was} indifferent beforehand, but, now, that he is called upon to select a~\textit{particular} unit of butter, it would be logically impossible for him to \textit{remain} indifferent.



According to Professor Wysocki, I~am logically hoist by my own petard unless I~embrace:



\begin{quote}
the Hoppean account, (where) it is, most naturally, the actor's preference guiding the actor's action: if the actor prefers x~to y, he chooses x~over y, whereas if he is indifferent between a~and b, he does not choose between a~and b. More concretely, if he is indifferent between particular units if butter, then he does not choose between them. If he prefers some money to any unit of butter, then he chooses to give up a~unit of butter for some money. There is no need to postulate any arbitrary change in the actor's state of mind to understand his resultant behaviour. Block, by contrast, is powerless to explain the actor's choice, for how can he choose to give up the 72\textsuperscript{nd} unit if the actor was \textit{ex hypothesi} indifferent between all of them.
\end{quote}



However, Wysocki overlooks the fact that there are two time periods, and that the grocer does not engage in human action in the first of these, hence indifference may prevail, but in the second, he most certainly does engage in human action, he selects one specific unit of butter to sell, not any other unit. Here, indifference must be banished. Thus, on this account, ``the robust notion of the supply of the same commodity'' may remain adhered to, and so may I~be allowed to ``capture… the law of diminishing marginal utility'' where, again, we are dealing with homogeneous elements of a~given stock. I~shall have no more to say about Wysocki's intriguing apple example except for the fact that he goes astray there, again, in the same manner.



\section{Agency and strict preference}

Here is Wysocki's next critique:



\begin{quote}
…it certainly does not follow that once the agent enters the café with her left foot, she thereby demonstrates her preference for entering with this particular foot to entering with the other one. For, the content of the agent's intentional state (i.e. of what the agent intends to do) might be simply to enter the café with the ways of entering it being left unspecified.
\end{quote}



This is a~brilliant attempt to undermine my thesis. I~full well recognize its power, and I~salute Wysocki for coming up with it. It is a~really good try, but no cigar shall be awarded. Here is my response: human action is purposeful behavior. If you were to ask the person which foot he (sic!) entered into the restaurant with, he would undoubtedly be unable to answer correctly. He would have been totally unaware of this choice. He could guess, but would have no more than a~50\% chance of coming up with the correct answer. In sharp contrast, if you asked the grocer which package of butter he was offering his customer, it would appear reasonable for me to ascribe to him the statement: ``Why that one, over there, the one in the customer's basket.'' So I~reject this attempted refutation of Wysocki's while acknowledging its creativity.



Here is another response to this brilliant riposte of Wysocki's. I~am logically obligated to give only one instance of reconciling supply, demand and ordinal utility with indifference. I~do not have to explain all instances of human action on the basis of this analysis of mine. Nozick's criticism of the Austrians is that as long as there is any indifference, we cannot maintain supply and demand of homogeneous objects, nor marginal utility. So, all we Austrians have to demonstrate is that in at least one case, there is no indifference, and, yet, homogeneity. This I~have done with the case of butter, with my before and after scenario. Wysocki is trying to paint me into a~corner, and in effect demanding that this analysis need apply to all sorts of other examples: when you walk into a~restaurant, which foot steps into its premises first; the exact verbiage with which you order coffee; the pitch of your voice when you offer coffee. I~need do no such thing. Indeed, I~am unable to do any such thing. Further, the examples he chooses are not cases of purposeful human action. Wysocki calls the coffee drinker ``an economic agent.'' He certainly qualifies for that rubric regarding the coffee, but not at all concerning which foot is forward when entering the restaurant.



Here is yet another of our author's criticisms:



\begin{quote}
…from the fact that the mother saves Peter Block draws an inference to the conclusion that she ‘places a~higher value on Peter than Paul.' But then again, just as -- as we already saw -- one cannot infer the preference for entering a~café with a~right foot from the fact the agent does actually enter with that very foot, so we cannot infer the mother's preference for Peter over Paul from the very fact that Peter was saved.
\end{quote}



Let us posit that the poor mother grabs Peter to save him. From this, Wysocki and Hoppe posit that she preferred to save one son, rather than none. But they offer no evidence for this claim. It is entirely possible that she really didn't care which son to save, wanted to save both of course but for some reason could only save one. But, in the event, she grabbed onto Peter and pulled him to safety. I~find it difficult to go along with Hoppe and Wysocki and say she was indifferent to which son she saved, in the face of the fact that she grabbed ahold of Peter, and did not let him go even though she could have done so at no risk, let us assume, and saved Paul instead. But she did no such thing.



Let me try again. Person A~does act X. I~say that this demonstrates that person A~preferred to do act X~to any other act he could have undertaken. Wysocki and Hoppe maintain, instead, without a~scintilla of evidence for this claim, that person A~was indifferent between X~and act Y, and preferred either X~or Y~to doing nothing. They are making this up out of the whole cloth. It simply does not logically follow from the fact that person A~does act X~that he does not prefer X~to anything else. It is simply fallacious to deduce from the fact that person A~does act X~that he really preferred X~or Y~to all alternatives. Yet, this is precisely the logic of their argument in all of these cases: the butter, the mother and sons, left foot right foot, Buridan's Ass, etc.



States Wysocki:



\begin{quote}
…the fact that the mother saves Peter (extensional description) underdetermines the value scale guiding the mother's action, for the mother might equally well frame her end as saving a~child rather than saving Peter. And if the former is true, then saving Peter serves this end equally well as saving Paul. That is why, she can remain (before and after action) indifferent between the two of her sons…
\end{quote}



But Wysocki cannot justify his claim that the mother is indifferent between saving either of her son's lives, or the other.? He certainly cannot deduce this from the fact that she saved Peter. There is no specific human action, moreover, that could unequivocally \textit{demonstrate} that she was indifferent between saving the lives of her two sons. There is no act could she perform that would unambiguously reveal this. I~have asked my two friends to demonstrate this many times in our previous debates, and have never seen any answer to it, let alone a~satisfactory one.



In Wysocki's view:



\begin{quote}
However, Block 
%\label{ref:RNDpRmGHk185r}(2022, pp.50–51)
\parencite*[][pp.50–51]{block_response_2022} %
 protests: Wysocki quotes me [
%\label{ref:RNDaYIxuQrx3o}(Block, 2022, pp.50–51)
\parencite[][pp.50–51]{block_response_2022}%
] as stating: ``She did rescue the former, when she could have chosen differently, and selected the latter for retrieval, did she not?'' Our author's response: ``But this simply begs the question. We, following Hoppe, contend that the mother's action in and of itself is not determinative of the mother's value scale, for the mother might as well simply prefer rescuing a~child to saving none.
\end{quote}


``Might'' have had this preference will not suffice. If I~say you might have eaten an apple when you are not now eating an apple, it is incumbent upon me to at least be able to draw a~picture of you eating an apple. If I~say you might have been sitting, similarly, I~should be able to draw a~picture of you doing just that.\footnote{I~don't like to brag, but my stick figure artistry is capable of so doing. Move over, Picasso.} If I~say that the fact that you purchased a~shirt for \$30 indicates that ex ante you valued something about that purchase more than the money you spent on it, I~cannot draw a~picture of that, but I~can appeal to people's understanding of the English language to not only know what I~am talking about, but to enthusiastically acquiesce in agreement with my contention.



Wysocki and Hoppe do not fare very well in this test. They certainly cannot draw any picture of a~human action which clearly and unmistakably depicts indifference. Nor can they even verbally describe what such a~situation could be. All they can do is assert that it has occurred in the cases under discussion. Talk about begging the question.



Let us now consider a~very powerful criticism that Wysocki launches against my analysis of indifference:



\begin{quote}
Eventually, to add insult to the injury, Block 
%\label{ref:RNDsQdYGokNYL}(2022, p.51)
\parencite*[][p.51]{block_response_2022} %
 adds that even if the mother ‘did this with her eyes closed, and just grabbed the nearest son' this would still indicate that the mother chose to save Peter. Yet, how can grabbing a~certain son with one's eyes closed count as demonstration of preference for that son? If anything, it seems that under that scenario the mother prefers grabbing any one son over saving none. It appears as though the most charitable take on the Blockean idea of choice is that the author -- his protestations to the contrary notwithstanding -- embraces methodological behaviourism.
\end{quote}



Let us make a~few stipulations about this ``eyes closed'' scenario. There are only two people who are drowning, or otherwise in danger: her sons, Peter and Paul. Without know whom she is grabbing, she latches onto Peter. She knows, moreover, that it is ``only one to a~customer'': she cannot possibly save both children, by stipulation. Now, to be sure, Wysocki is correct in asserting that she prefers to save either of her sons, rather than none.\footnote{This is an ordinary language statement, not one of technical economics. As far as the latter is concerned, we are not entitled to deduce any such thing from the fact that she saved Peter.} But the fact of the matter is that she is now clutching Peter's hand, let us say, not Paul's. From this I~deduce that she prefers to save Peter, rather than Paul. It cannot be denied that, at this point, she does not know who she is in the midst of saving. But, still, she does not let go of this son's hand, and grab onto the hand of the other son. Let us stipulate, also, that she could do that if she wished without any danger of losing both sons.\footnote{Hey, I~need all the help I~can get here. Wysocki is on my trail, and he is a~worthy opponent.} I~thus conclude that based on this behavior of hers, she prefers to save the son whose hand she is now gripping. True, she will not know his name until and unless she opens her eyes, but, still, it is a~true statement to say, contrary to Wysocki, that she prefers to save Peter, vis a~vis Paul. Just because she is unaware of the identity of the person whose hand she is now grasping cannot gainsay this primordial fact.



Wysocki's next criticism is that my view



\begin{quote}
…comes perilously close to methodological behaviourism, for Block seems to dismiss the mother's mental states completely. Note, even if the mother were to think ‘about ice cream', she would still choose to save Peter in the event Peter would be ultimately saved. But this at a~stroke gives up characteristically Austrian methodological subjectivism and denies any role to the actor's mental states (preferences and beliefs) as determining choices.
\end{quote}



Wysocki and I, both strong advocates of subjectivism, and adamant opponents of behaviorism, have sharply different views as to what this concept signifies. In my view, we as praxeologists simply have no insight as to what this eyes-closed mother was thinking about when she grabbed Peter's hand. Presumably, she was thinking along the lines of ``I love both my sons, I~wish I~could save them both but I~can't, so I'll at least save this one here, whoever he is.''\footnote{This is a~guess on my part. It does not logically follow, inexorably, from her actions. This is not praxeological truth, as is the case of inferring when one purchases a~shirt for \$30, that he at time valued something about that article of apparel more than that amount of money.} Yet, for all we know, qua praxeologists, she could have been thinking about anything else under the sun, yes, certainly including ice cream. We are not psychics; we are not ESPers; we are not psychologists; we are not mind readers; we are not magicians. We are none of these things. We have to be modest about our abilities. We are merely praxeologists. Our scope is limited to deducing from \textit{behavior} not from thoughts that are inevitably and necessarily hidden from us. All we know, all we \textit{can} know from her behavior, is that she unknowingly grabbed Peter's hand. We have no option, while still remaining true to praxeology, to deduce anything other than that she preferred to save Peter to Paul.\footnote{In my view, Wysocki is guilty of the fallacy of 'psychologizing,' the treatment of preference scales as if they existed as separate entities apart from real action. Psychologizing is a~common error in utility analysis. It is based on the assumption that utility analysis is a~kind of ‘psychology,' and that, therefore, economics must enter into psychological analysis in laying the foundations of its theoretical structure. ``Praxeology, the basis of economic theory, differs from psychology, however. Psychology analyzes the how and the why of people forming values. It treats the concrete content of ends and values. Economics, on the other hand, rests simply on the assumption of the existence of ends, and then deduces its valid theory from such a~general assumption. It therefore has nothing to do with the content of ends \textit{or with the internal operations of the} \textit{mind of the acting man}. 
%\label{ref:RNDne7up6s1aL}(Rothbard, 2011; emphasis added by present author)
\parencites[][]{rothbard_toward_2011}[emphasis added by present author,][]{}%
}



Behaviorism will not pass muster. Here is what Rothbard 
%\label{ref:RNDF4cblaYW8A}(2011)
\parencite*[][]{rothbard_toward_2011} %
 had to say about that concept:



\begin{quote}
The behaviorist wishes to expunge ‘subjectivism, that is, motivated action, completely from economics, since he believes that any trace of subjectivism is unscientific. His ideal is the method of physics in treating observed movements of unmotivated, inorganic matter. In adopting this method, he throws away the subjective knowledge of action upon which economic science is founded; indeed, he is making any scientific investigation of human beings impossible.
\end{quote}



I~can see why Wysocki launches this charge against me. I~do indeed base my interpretation on the behavior of the choosing individual, or the human actor, be he a~grocer, a~person seeking coffee in a~restaurant or a~mother trying to save life. But I~plead innocence. I~am not, as is Prof. Little, rejecting ``demonstrated preference theory.'' Rather, I~am adhering to it, through thick and thin, despite the criticisms of Wysocki and Hoppe to the effect, in my interpretation of them, that I~am sticking too closely to it. I~am exulting in it. I~am insisting that there is no way that indifference can be logically deduced from any human action, any behavior. I~am insisting that demonstrated preference, the foundation here of Rothbardianism, is human behavior, and that this is not the behaviorism against which Rothbard warns.



\section{Conclusion}

I~am very grateful to Wysocki for this critical essay of his. He has forced me, in my response, to dig far deeper into these issues than ever I~would have otherwise contemplated; than ever I~would have been able to do on my own, without his splendid challenges. We all learn from each other, and I~am greatly in the debt of this author for in effect compelling me to learn from him. I~hope and trust this is at least partially reciprocal.




\end{artengenv}\label{block-rejoinder-lastpage}



%
\sekcja{Review essays}{Artykuły recenzyjne}
\begin{newrevengenv}{Mateusz Czyżniewski}
	{Are there really any errors in the Austrian theory of welfare?}
	{Are there really any errors in the Austrian theory of welfare?}
	{Are there really any errors in the Austrian theory of welfare?}
	{Gdansk University of Technology}
	{Dawid Megger, \textit{Sprawiedliwość 
	w~Ekonomii Dobrobytu, Liberatarianizm i~Szkoła Austriacka}, Wydawnictwo Naukowe UMK, Toruń, 2021.}
	
	















%Dawid Megger, \textit{Justice in Welfare Economics. Libertarianism and the Austrian School} (in Polish: \textit{Sprawiedliwość 
%w~Ekonomii Dobrobytu, Liberatarianizm i~Szkoła Austriacka}), Wydawnictwo Naukowe UMK, Toruń, 2021. \url{https://doi.org/10.12775/978-83-231-4689-6}









\section{Introduction}

\lettrine[loversize=0.13,lines=2,lraise=-0.03,nindent=0em,findent=0.2pt]%
{O}{}ne of the most controversial domains in economics is welfare (growth) economics, which encompasses both applied and theoretical aspects. Given the normative character of the deliberations conducted within this branch, it is unsurprising that a~number of issues that cut across positive economics, sociology, ethics, philosophy, or political science intersect in a~vast array of possible conclusions 
%\label{ref:RNDT15YZAIlgL}(Blaug, 1998).
\parencite[][]{davis_positive-normative_1998}.%




Supporters of a~free-market economy typically point out that both sides of transactions always benefit from voluntary market exchanges. One can venture even stronger assertions to the effect that voluntary exchanges enhance overall social well-being. Murray Rothbard, one of the most recognised representatives of the Austrian School of Economics (ASE), was the strongest advocate of this view 
%\label{ref:RND281nnabjyi}(Rothbard, 1998; [1956] 2008; [1962] 2009).
\parencites[][]{rothbard_ethics_1998}[][]{rothbard_toward_2008}[][]{rothbard_man_2009}. %
 To put it succinctly, Rothbard argues that it is unfeasible to establish a~universally applicable measure that is based on rigorous scientific principles and can gauge the satisfaction of individuals. Utility rankings \textit{ordinally} reflect the corresponding importance of subjectively framed ends rather than assign \textit{cardinal} numbers representing levels of satisfaction arising from the fulfilment of the said ends. These statements stand in contrast to mainstream economics, which involves optimisation, measurement, comparison, and utility calculation 
%\label{ref:RNDHLQhnxQmSp}(Samuelson, 1971, pp.173–183, 203–256 [orig. 1947]).
\parencite[][pp.~173--183, 203--256 \mbox{[orig. 1947]}]{samuelson_foundations_1971}. %
 As a~result, Rothbard's contribution cannot be overestimated, as ASE related scholars like Gordon 
%\label{ref:RND1AEneN6XM7}(1993),
\parencite*[][]{gordon_toward_1993}, %
 Herbener 
%\label{ref:RND4FYRLEkimm}(1997),
\parencite*[][]{herbener_pareto_1997}, %
 and Hoppe 
%\label{ref:RND1k2IOJNaLZ}([1993] 2006)
\parencite*[][]{hoppe_economics_2006} %
 readily acknowledge. Furthermore, Rothbard developed his own welfare theory, which was heavily based on the Pareto efficiency rule, also referred to as ``unanimity rule'', and the doctrine of demonstrated preference 
%\label{ref:RNDcL1wx8i06I}(Rothbard, 2008 [1956])
\parencite[][]{rothbard_toward_2008}%
\footnote{Henceforth, his theorems are called Rothbard's Austrian Welfare Economics (RAWE).}. The main idea is that, without drawing any moral (normative) conclusions, we can determine, using the Unanimity Rule and the concept of demonstrated preference, that:



\begin{enumerate}

\item 
Free-market (voluntary) transactions always improve society's welfare.

\item 
Government interference can never raise social welfare.

\end{enumerate}

Even if the Rothbardian theorems were regarded by ASE representatives as untroubled for many years, some voices inside the Austro-Libertarian community present critical views on that matter, e.g. 
%\label{ref:RNDZjWkgC15SU}(Prychitko, 1993; Gunning, 2005; Kvasnička, 2008; Wysocki, 2023; Wysocki and Dominiak, 2023).
\parencites[][]{prychitko_formalism_1993}[][]{gunning_did_2005}[][]{kvasnicka_rothbards_2008}[][]{wysocki_how_2023}[][]{wysocki_how_2023}. %
 Dawid Megger joined a~group of opponents of Rothbard's classical approach a~few years ago in his book \textit{Justice in Welfare Economics. Libertarianism and the Austrian School} (in Polish: \textit{Sprawiedliwość w~Ekonomii Dobrobytu, Liberatarianizm i~Szkoła Austriacka})\textit{,} which delves deeply into welfare theory, offers constructive criticism of RAWE, and introduces some original concepts 
%\label{ref:RNDzxWQJS3FB0}(Megger, 2021)
\parencite[][]{megger_sprawiedliwosc_2021}%
\footnote{The original edition of the book does not contain an English translation of the title, so the author of the review allowed himself to translate the Polish title \textit{Sprawiedliwość w~Ekonomii Dobrobytu, Liberatarianizm i~Szkoła Austriacka} into English literally. Subsequent sections of the review make reference to the book by the abbreviation \textit{Justice} or Megger's book.}. This review will concentrate on the previously mentioned work.



The review is divided into the following sections. Section 2 involves a~slight introduction of Megger's scientific achievements and career details. Section 3 presents and gradually discusses the review's most significant observations. Section 4 concludes the discussion. Section 5, which contains the bibliography, concludes.



\section{A~few words about the author and his works}

In 2023, Megger earned his doctoral degree with honours for his dissertation, \textit{The Austrian School of Economics as a~causal-realist research program}. \textit{Methodological investigations}. In 2021, he published a~book entitled \textit{Justice in Welfare Economics. Libertarianism and the Austrian School} 
%\label{ref:RNDPgg0eHMrwD}(Megger, 2021),
\parencite[][]{megger_sprawiedliwosc_2021}, %
 which built on his master's thesis \textit{Austro-Libertarian Welfare Economics and its Aporias} 
%\label{ref:RNDmVBfsGgMWC}(Megger, 2023),
\parencite[][]{megger_austriacka_2023}, %
 defended in 2019. The master's thesis itself was awarded in a~contest conducted within the Faculty of Economic Sciences and Management of Nicolaus Copernicus University in Toruń for the best master's thesis in 2019.



Despite the passage of several years since the publication of this book and the appearance of multiple articles on Megger's account concerning the aforementioned issues of the welfare economy, such as Wysocki and Megger 
%\label{ref:RNDszwEZPmlgM}(2019),
\parencite*[][]{wysocki_austrian_2019}, %
 Wysocki and Megger 
%\label{ref:RNDeCuVOUY7yG}(2020),
\parencite*[][]{wysocki_rejoinder_2020}, %
 and Megger and Wysocki 
%\label{ref:RNDO2bjbyBfx9}(2023),
\parencite*[][]{megger_austriacka_2023}, %
 this review pertains exclusively to the aforementioned book. The author's intention is not to make a~critical cross-section of Megger's achievement or the relationship between the book and the articles, but rather to refer to the theses presented in \textit{Justice}. In fact, during those few years, two critics presented their arguments against him and his collaborator's statements. Both Wiśniewski 
%\label{ref:RNDAtgAhltIH0}(2019)
\parencite*[][]{wisniewski_austrian_2019} %
 and Juszczak 
%\label{ref:RNDq6tcgSI02j}(2021)
\parencite*[][]{juszczak_o_2021} %
 claimed that Megger's doubts put on the classical RAWE misfire and are not properly methodologically justified. However, as the author of this review, my goal is not to confront their statements with Megger's reasoning but rather to present my original assessment of the latter's work.



\section{Substantive assessment of the book content}

This particular part of the text contains the most important remarks that the reviewer wants to emphasise in the context of the book's substantial content.



\subsection{Chapter One: \textit{Introduction}}



Let's start with the first chapter of the book. Section 1.1, called \textit{The Problematic} (Pol: \textit{Problematyka}), aims to outline the ideas and objectives of the work, briefly introducing the reader to the framework and general considerations related to welfare economics, especially from the perspective of ASE (pp. 13–17).



Next, the author decides to describe the research objectives, which he introduces in Section 1.2 called Research \textit{Aims} (Pol: \textit{Cele}), pages 17–19. Megger gives the following statement: ``[…] Our aim is not to completely dismiss the Austro-Libertarian theory of wealth, but simply to demonstrate its inaccuracies and to make selected claims more justified. […]''. More specifically, the author aims to demonstrate that: 1) Rothbard's welfare theory often relies on hidden assumptions and circular reasoning; 2) economic effectiveness does not necessarily hinge on voluntariness; and 3) the praxeological research can incorporate game-theoretical considerations and the expected utility methodology.



The last part of the first chapter, i.e., 1.3, called \textit{Methodology} (Pol: \textit{Metoda}), provides a~description of the methodology that the author applied to the intended research objectives 
%\label{ref:RNDuMoAc83e5t}(Megger, 2021, pp.19–24).
\parencite[][pp.19–24]{megger_sprawiedliwosc_2021}. %
 In a~way that fits with the ASE tradition,. Megger stresses how necessary and important it is to use the praxeological method, which is familiar from von Mises's writings 
%\label{ref:RNDwLb7Cmjn3f}([1957] 1997; [1949] 1998).
\parencites*[][]{mises_theory_1997}[][]{mises_human_1998}. %
 In addition, the author emphasises the need to carry out research using methods characteristic of rationalist philosophical traditions, rejecting positivist-styled empiricism or instrumentalism based on measurements, experiments, or statistical analysis. It is regrettable that at this stage of the book, even if it is only an introduction, Megger did not outline some general methodological aspects, referring, for example, to such works as Robbins 
%\label{ref:RNDi5E1UGDXKm}(1932),
\parencite*[][]{robbins_essay_1932}, %
 Mises 
%\label{ref:RND4TQJANmTK9}(1962),
\parencite*[][]{mises_ultimate_1962}, %
 Lachmann 
%\label{ref:RNDMmVs8CwjzR}(1971),
\parencite*[][]{lachmann_legacy_1971}, %
 Machlup 
%\label{ref:RNDAb6gp52P4b}(1978),
\parencite*[][]{machlup_methodology_1978}, %
 Hausman 
%\label{ref:RNDSdgrz9MvIA}(1995),
\parencite*[][]{hausman_impossibility_1995}, %
 Mises 
%\label{ref:RNDISfzJcCgQt}([1957] 1997),
\parencite*[][]{mises_theory_1997}, %
 Mises 
%\label{ref:RNDvAP10EktzZ}([1949] Mises, 1998),
\parencite*[][]{mises_human_1998}, %
 Hoppe 
%\label{ref:RND4bfC8QYSsF}([1993] 2006),
\parencite*[][]{hoppe_economics_2006}, %
 and O'Driscoll and Rizzo 
%\label{ref:RNDI215IELdbK}(2014).
\parencite*[][]{odriscoll_austrian_2014}.%




\subsection{Chapter Two: \textit{Economic efficiency and the issue of rational social order}}



The second chapter makes some state-of-the-art claims and presents some introductory issues. Section 2.1, called \textit{Wealth and Prosperity in the History of Economic Thought} (Pol: \textit{Bogactwo i~Dobrobyt w~Historii Myśli Ekonomicznej})\textit{,} conducts a~selective review of the literature focused on the evolution of welfare economics 
%\label{ref:RND5Iq7Mkx47C}(Megger, 2021, pp.25–34).
\parencite[][pp.25–34]{megger_sprawiedliwosc_2021}. %
 In fact, this review should be broader and more extensive in terms of methodology. Aspects such as the transformation of the neoclassical school into a~mathematical one, the addition or comparison of preferences, and the lack of neutrality in redistributive acts should be highlighted.



Next, the author of the book, in Section 2.2 called \textit{The Austrian Approach: Praxeology} (Pol: \textit{Stanowisko austriackie: Prakseologia}), presents some arguments in favour of the methodology characterizing ASE 
%\label{ref:RNDsKFzlLDSZh}(Megger, 2021, pp.34–43).
\parencite[][pp.34–43]{megger_sprawiedliwosc_2021}. %
 The review itself is appropriate from a~substantive standpoint, and Megger's original comments prove valuable. This section incorporates many of the ASE's key points, and it serves as a~concise summary of the Austrian framework.



Section 2.3, called \textit{The Problem of Socialism} (Pol: \textit{Problem Socjalizmu})\textit{,} describes issues related to the problem of economic calculation. A~special emphasis is put on extreme economic system, i.e., \textit{pure free-market capitalism} and \textit{fully controlled socialism} 
%\label{ref:RNDCUH6Dh1EOu}(Megger, 2021, pp.43–49).
\parencite[][pp.43–49]{megger_sprawiedliwosc_2021}. %
 Even though this section has been described flawlessly and in some way complements the considerations on the welfare economy, I~think this book would do equally well without it, and valuable insights can be transferred to sections 2.2 and 2.4.



The last section of the second chapter, i.e., 2.4, called \textit{Libertarian Solution} (Pol: \textit{Rozwiązanie Libertariańskie}), addresses the libertarian solution to the problem of establishing an efficient economic system 
%\label{ref:RNDjmlXfO3QCk}(Megger, 2021, pp.48–49).
\parencite[][pp.48–49]{megger_sprawiedliwosc_2021}. %
 Megger, after introducing Rothbard's most important works on the libertarian order and its justification 
%\label{ref:RND1bN9BCE02x}(Rothbard, 1998; [1956] 2008),
\parencites[][]{rothbard_ethics_1998}[][]{rothbard_toward_2008}, %
 elucidated the most essential doubts associated with this theory. The author of the book refers to the issues of space, body, and the scarcity of resources (finite amounts), which spells potential conflicts between people 
%\label{ref:RNDL2KFqEO1Qt}(Gordon, 1993; Herbener, 1997; Hoppe, 2006, pp.311–330, 341–345; Wiśniewski, 2019).
\parencites[][]{gordon_toward_1993}[][]{herbener_pareto_1997}[][pp.311–330]{hoppe_economics_2006}[][]{wisniewski_austrian_2019}. %
 To resolve the fundamental issues in terms of property rights, RAWE supporters claim that a~full-free market society is the ultimate solution, allowing efficient economic calculation and then fair and productive goods allocation. Nevertheless, the author of the book, at the very end of the chapter, formulates interesting statements regarding potential errors in RAWE, which he plans to develop in the third chapter.



\section{Chapter three: \textit{The Issues of Austro-Libertarian Welfare Economics}}

This whole chapter 
%\label{ref:RNDSp7tSwKvdi}(Megger, 2021, pp.59–102)
\parencite[][pp.59–102]{megger_sprawiedliwosc_2021} %
 presents the most original content of the book, where Megger's critical claims against RAWE are presented.



\subsection{\itshape Ex ante and ex post analysis}



Section 3.1 is entitled \textit{Ex ante and ex post analysis. Voluntary exchange and mutually beneficial exchange} (Pol: \textit{Analiza ex ante i~ex post. Wymiana dobrowolna a~wymiana wzajemnie korzystna}) refers to an attempt to criticise the statement that any voluntary exchange is \textit{ex ante} always mutually beneficial and involuntary exchanges must be considered unfavourable 
%\label{ref:RND7jENv9hWik}(Megger, 2021, pp.59–73).
\parencite[][pp.59–73]{megger_sprawiedliwosc_2021}. %
 Firstly, the text outlined the issues associated with actual physical possession vis-à-vis property rights. Megger believes that in a~voluntary exchange, so that all parties should benefit and no one should lose, it is necessary not so much to transfer property rights as not to violate them 
%\label{ref:RNDE10qKQp035}(Megger, 2021, p.59).
\parencite[][p.59]{megger_sprawiedliwosc_2021}. %
 As long as the ``physical'' transfer of goods can naturally infringe on ownership rights, is this conclusion trivial from the point of view of the RAWE doctrine? The thieves-case serves as an essential example to deliver an appropriate justification. If an exchange among thieves enhances their psychical ``utility'', what about the exchange-related utility of those deprived of property? In their view, \textit{ex ante} transferring goods to others was not a~preferred act, and subsequently, someone stole their property.



If, according to the author of the book, ``[…] this is not irrelevant to the utility function of the people […]'' 
%\label{ref:RNDzMy938dfwt}(Megger, 2021, p.60)
\parencite[][p.60]{megger_sprawiedliwosc_2021}%
\footnote{What does the utility function mean? Is there something in the mathematical framework? If yes, is it a~static (time-invariant) or dynamic (time-related) construct? }, then how do we define the type of ``usefulness'' related to goods acquired voluntarily or stolen? Isn't this sometimes an obvious violation of Pareto's orpimality rule, defined both as in RAWE 
%\label{ref:RNDPTNvsfEhMf}(Rothbard, 2008 [1956])
\parencite[][]{rothbard_toward_2008} %
 and using the mathematical neoclassical framework 
%\label{ref:RNDP7DOtIoRq0}(Samuelson, 1971, pp.173–183, 203–256 [orig. 1947])?
\parencite[][pp.~173--183, 203--256 \mbox{[orig. 1947]}]{samuelson_foundations_1971}? %
 Doesn't this create a~problem of psychologizing and comparing utility if some divergent types of ``exchange`` are considered 
%\label{ref:RNDTahCVDXhGF}(Hausman, 1995)?
\parencite[][]{hausman_impossibility_1995}? %
 What about the temporal aspect? Specifically, does considering an extensive situation with numerous potential outcomes violate the foundations of a~``thought experiment''? In terms of psychological investigations, there is a~potential for specific research related to pragmatically done analysis and thymological issues 
%\label{ref:RNDeUtevmnR0k}(Mises, 1997, pp.264–284, 303–320 [1\textsuperscript{st} ed. 1957]).
\parencite[][pp.~264--284, 303--320 \mbox{[1\textsuperscript{st} ed. 1957]}]{mises_theory_1997}. %
 However, any significant violation of RAWE principles requires a~more detailed discussion. Perhaps we should elaborate on and justify a~clear definition of coercion, contrasting it with direct property right violations.



Certainly, thieves consciously and \textit{voluntarily} carry out their actions with the intention of stealing property from a~person who desires to exchange it and who does not prefer the ``loss`` of this property. The very fact that a~person owns and possesses a~given good and does not give up on it shows that his holding demand (reservation demand and transactional demand) exists in this context, and he perceives the potential uses of that good, no matter whether it is production, consumption, or sale 
%\label{ref:RND3RUv74NHBp}(Rothbard, 2009, pp.137–142).
\parencite[][pp.137–142]{rothbard_man_2009}. %
 Therefore, if there is no mutual exchange (as defined by Böhm-Bawerk)\footnote{To be found in, e.g., Mises 
%\label{ref:RNDg3p4cRirXY}([1949] 1998, pp.213–232, 268–316)
\parencite[pp.213–232, 268–316][]{mises_human_1998} %
 and Rothbard 
%\label{ref:RNDrHAa56Y8xF}([1962] 2009, pp.95–169).
\parencite[pp.95–169,][]{rothbard_man_2009}. %
 }, it cannot be possible to discuss and quantitatively compare the ``increase'' in utility after theft among thieves and the apparent decrease in utility among the robbed from an \textit{ex ante} perspective.



Furthermore, certain aspects of exchanges between residents and workers require specificity. All these individuals function within a~specific property rights environment, which necessitates their establishment prior to any exchanges. It should be remembered that both workers and residents agree to exchange money for the completion of a~certain current or futures transaction. In this sense, the successive transfer of equipment or belongings to the owner does not interfere in any way with the specified provisions within the context of property rights. Rather, ``exchanges'' occur within the context of certain property rights, allowing them to swap ownership between parties. Regardless of whether such exchanges resolve, deny, or modify the conditions of previous agreements, the utility of these individuals increases. The exchange on the so-called time market 
%\label{ref:RNDOANOtObST5}(Rothbard, 2009, pp.390–410),
\parencite[][pp.390–410]{rothbard_man_2009}, %
 which refers to the temporal structure of exchange between the suppliers of future goods and the present goods, is particularly significant.



Next, the author of the book analyses the problem of blackmail and the productivity of exchange 
%\label{ref:RND47Vnauybpx}(Megger, 2021, pp.62–67).
\parencite[][pp.62–67]{megger_sprawiedliwosc_2021}. %
 In Megger's opinion, blackmail, as a~form of threat, cannot be perceived as potentially mutually beneficial, even if it is \textit{voluntarily} carried out in order to get rid of the blackmailer. The problem is highlighted here: such a~``voluntary exchange'' can lead to negative consequences and have similar effects to actions undertaken under severe threat, i.e., actions of a~non-voluntary and essentially harmful, property-violating nature. Megger gives two examples with a~similar analytical structure to illustrate some issues with RAWE applied to the threat problem 
%\label{ref:RNDoPrUaOluOp}(Megger, 2021, p.63).
\parencite[][p.63]{megger_sprawiedliwosc_2021}. %
 In both cases, you have to pay \$10,000 for successively removing the unwanted influence of the blackmailer and fulfilling the ``obligation'' to the tax collector. However, in the first situation, blackmail does not have a~direct and unavoidable effect linked to violence and further state sanctions, as in the second case of an action undertaken by a~tax collector. In a~sense, the influence of the blackmailer has a~chance to affect the feelings of his victim, but does the attempt to solve this problem not involve psychologizing or a~violation of the principle of \textit{ceteris paribus}? Is this in any way linked to the catallactic properties defined by Mises 
%\label{ref:RNDOba4ohrYMo}(1998, pp.233–257)?
\parencite*[][pp.233–257]{mises_human_1998}? %
 Why is it necessary to immediately associate ``negative feelings`` with the utility changes that occur during the voluntary exchange, as understood in the context of RAWE? Furthermore, what counterfactual scenarios could we regard as possible and not subject to this type of reasoning? While these scenarios are essentially different, the book's author believes that their effects on welfare are similar, necessitating a~correction of Rothbard's considerations. Furthermore, he criticises the attempt to explain this problem on the basis of alternative costs 
%\label{ref:RNDM24cRSJcyW}(Megger, 2021, p.64),
\parencite[][p.64]{megger_sprawiedliwosc_2021}, %
 showing that comparing these scenarios to each other must involve a~utility comparison rather than a~qualitative structural analysis. It also highlights some issues with the concept of demonstrated preference. Here, I~have to agree with Megger that Rothbard's theory leaves some gaps, even significant ones. I~think in this case, referring to alternative costs and not comparing utility in the context of invoking the two scenarios may be a~bit of a~double-edged sword in terms of maintaining the coherence of the argumentation. However, there is a~sort of example where, if there is the possibility of experiencing ``worse'' consequences, people would prefer to pay taxes, perceiving this action as unfavourable but not as bad as government sanctions. Still, there is some preference demonstration, but not as Rothbard would claim, and there is still a~gap to be filled.



Megger proposes to solve this problem by applying the Nozickian 
%\label{ref:RNDc1c2eJc1z9}(Megger, 2021, pp.66–67)
\parencite[][pp.66–67]{megger_sprawiedliwosc_2021} %
 tripartite division of exchanges 
%\label{ref:RND6H1V1cbxGi}(Nozick, 1999, pp.84–87 [orig. 1974]).
\parencite[][pp.84–87 \mbox{[orig. 1974]}]{nozick_anarchy_1999}. %
 I~think that while this methodology presents a~number of differences compared to RAWE, its application to the theory of welfare may have interesting implications, but it requires some additional assumptions and considerations. I~agree with Juszczak 
%\label{ref:RNDTxzlTpfyvL}(2021)
\parencite*[][]{juszczak_o_2021} %
 that those frameworks have some issues when it comes to comparing utility in a~psychological sense. Moreover, other doubts include the inability to clearly define the difference between ``sharp boundary conditions associated with a~given situation'' that can be involved, the development of causal links over time, and, more importantly, deciding what productivity means, what kind of one-sided benefits can really be claimed as advantageous, and what psychological and social factors need to be considered. Nozick's theory offers numerous benefits in specific cases, but it doesn't solve the problem of debunking Rothbard's theory in general.



I~agree with Megger on the statements relating to the continuity or simply conversion of preferences 
%\label{ref:RNDRUa5JAjHdn}(Megger, 2021, pp.68–69).
\parencite[][pp.68–69]{megger_sprawiedliwosc_2021}. %
 That being said, Austrian economists use the value scale or the preference list as a~``model'' to explain what actions are entirely about and identify what people really desire. Thus, it would be wrong to say that there is no mental connection between value scales and the temporal evolution of actions, which involves other preference scales. However, an explanation is necessary to understand the attachment of specific preferences to the ``behavioural'' aspect of action, as well as their appearance and disappearance. But still, preferences cannot exist separately from the actions, because even the acts of thinking, reminding, deliberating, or planning are actions \textit{par excellence} 
%\label{ref:RNDCNhvuedUtU}(Mises, 1998, pp.11–72).
\parencite[][pp.11–72]{mises_human_1998}. %
 I~believe that individuals subconsciously recall past preferences, or entire preference scales, and then apply them during specific action evaluations when a~subjective link between them is present. However, it does not allow for separating some \textit{preferences} from \textit{actions}.



Then, the author of the book starts dealing with the \textit{ex post} analysis 
%\label{ref:RNDvh2Hca1Wtf}(Megger, 2021, pp.70–72).
\parencite[][pp.70–72]{megger_sprawiedliwosc_2021}. %
 On the one hand, Megger agrees with Rothbard's supporters, but he also introduces some comparative aspects between the \textit{ex post} and \textit{ex ante} domains. In my opinion, the author's considerations are correct on the one hand, and, on the other hand, they do not dismiss RAWE completely. While the question of whether it is possible to find genuinely positive aspects of certain ``interactions'' with the violent side is a~valid one, some of the conclusions drawn are overly broad. Furthermore, the general scheme is entangled in certain problems. First, the economic actor's choice of means is always subjective. Hence, the evaluation of the selection's correctness is always rational \textit{ex ante} (even if it is meaningless in strictly physical or conceptual terms)\footnote{See Mises 
%\label{ref:RNDbwPKmBlvy1}([1949] 1998, pp.13–23)
\parencite[pp.13–23][]{mises_human_1998} %
 and 
%\label{ref:RNDoUAXjbqNek}([1957] 1997, pp.264–271).
\parencite[pp.264–271][]{mises_theory_1997}.%
}. Second, if person A~is doomed to make a~mistake but does not know it, then why does the ``mechanism'' of the thought experiment preclude the implementation of plan modification? Third, doesn't the author himself fall into Rothbard's trap of comparing utility in a~quasi-qualitative manner? Fourth, why are the situation's dynamics so limited? And fifth, why isn't the actor able to act intentionally, meaning that they are fully aware of the potential consequences of committing a~murder but still choose to proceed? In my view, this specific reasoning holds significance in such areas as action support systems known from computer science and control engineering, ergonomics, or game theory concepts, yet it is weakly related to the Rothbardian framework.



\subsection{\itshape Preference and Risk }



Section 3.2, entitled \textit{Preference and Risk} (Pol: \textit{Preferencja a~ryzyko),} deals with eponymous preference and risk 
%\label{ref:RND7mRfs39JEr}(Megger, 2021, pp.73–82).
\parencite[][pp.73–82]{megger_sprawiedliwosc_2021}. %
 First, Megger attempts to deal with the issue of the coherence between demonstrated preference and pure choice, which constitute the essence of human action. He claims that ``[…] the more desirable the praxeological statement is to admit that the acting person prefers not the highest valued goal but the highest valued action'' 
%\label{ref:RNDnEVN4enFF2}(Megger, 2021, p.74),
\parencite[][p.74]{megger_sprawiedliwosc_2021}, %
 and then turns into the theory of expected utility. In his opinion, it is possible to make this connection by combining the theory of action with risk assessment methodologies.



The book's author introduces some examples 
%\label{ref:RNDTh1LrdevNn}(Megger, 2021, pp.79–81)
\parencite[][pp.79–81]{megger_sprawiedliwosc_2021} %
 to illustrate his concept, linking risk assessment with time preference 
%\label{ref:RNDsc3DkpPVmA}(Mises, 1998, pp.476–486; Rothbard, 2009, pp.13–17, 49–56).
\parencites[][pp.476–486]{mises_human_1998}[][pp.13–17, 49–56]{rothbard_man_2009}. %
 The two initial examples are almost canonical, and thus no controversy arises as far as the Austrian methodology goes. The following cases, however, show the ``Austrian theory of expected utility'', which substitutes risk assessment for time preference. People with particular risk evaluations will always choose less risky activities when the financial cost and revenue are the same in both cases (\textit{ceteris paribus} clause), so risk assessment ``takes the place of time.'' This is also noncontroversial in terms of the expected utility theorems. The next situation seems to be convincing because, \textit{ceteris paribus}, \$1,000 is associated with a~lower risk, whereas \$10,000 is associated with a~higher level of risk, and thus the risk preference is the key factor in selecting a~particular action. Despite the subjective perception of risk, it's crucial to keep in mind that these assessments, which are merely human projections or models of reality, must have a~strong connection to actual, empirical phenomena.



In fact, even with a~future assessment, the conditions resulting from a~certain risk level could change, and these conditions are not the same as those associated with changes in time preference. It's possible that certain factors could make the second investment less risky. Furthermore, why is it impossible to integrate risk assessments into the straightforward and conventional theory of choice, also known as the Misesian theory of action? People use different methods to attempt to accomplish different goals in different circumstances, so why do we need to distinguish the ``risk category'' as another relevant factor affecting human action? When discussing subjective assessments, we often refer to risk. However, why not consider ``pure uncertainty'' or single-case probability 
%\label{ref:RNDAo19aEu1GW}(Mises, 1998, pp.105–118)?
\parencite[][pp.105–118]{mises_human_1998}?%




Also, what about the cases where the time preference factor would be complementary to the risk assessment perceived in this way? How do we quantitatively evaluate the action's profitability? Is it necessary to make an analytical distinction between those two factors? How can we integrate the deductively described pure time preference into the risk assessment framework, given the numerous tools available for probability calculation?



I~think those attempts at connecting time preference with risk preference sound very interesting and would gain some added value, but this description is not comprehensive and would apply only to some constrained aspects of the theory of action. However, the presented reasoning, which takes into account certain specific assumptions, is coherent and clear.



\subsection{Maximising of utility and social dilemmas}



Section 3.3, entitled \textit{Maximising of utility and social dilemmas} (Pol: \textit{Maksymalizacja użyteczności a~dylematy społeczne}), deals with the issue of maximising utility in specific social contexts 
%\label{ref:RND0oHv0dEgAo}(Megger, 2021, pp.82–94).
\parencite[][pp.82–94]{megger_sprawiedliwosc_2021}. %
 The author of the book, at the very beginning of the chapter 
%\label{ref:RNDnAHaxVnTSo}(Megger, 2021, pp.83–86),
\parencite[][pp.83–86]{megger_sprawiedliwosc_2021}, %
 gives an introduction to the concept of combining praxeology with some elements of game theory. However, as I~properly understood this concept, there is a~chance of making some particular type of game theory framework possible to be interpreted in terms of praxeological reasoning. Even though some ASE representatives had doubts about how this method could be used, Megger says, ``What distinguishes game theory from praxeology will be the fact that it is a~subcategory of the general decision-making theory […], whereas praxeology emphasises »action as such« […]'' 
%\label{ref:RNDtoMuQNkbKu}(Megger, 2021, p.85).
\parencite[][p.85]{megger_sprawiedliwosc_2021}. %
 We have to agree with this because the basic ideas of game theory allow for a~formal comparison or measurement of utility in certain specified, ``constructed'' and controlled scenarios and situations. The proposed ideas suggested replacing cardinal numbers, which describe quantifiable utility values, with order numbers connected to scheduled actions. The author presents two typical types of games, the prisoner's dilemma (pp. 87-88) and the confidence dilemma 
%\label{ref:RNDgNmUnQMtZi}(Megger, 2021, pp.89–91).
\parencite[][pp.89–91]{megger_sprawiedliwosc_2021}.%




The presented praxeological analysis, in my opinion, construes the game theory in a~proper manner, enabling it to transcend certain rigid schemes. However, this interpretation still requires that the phenomena analysed be within a~fixed and persistent context, which shows some limitations in the application of this methodology. Although the explanations based on real scales of values are valuable, it is necessary to treat the game theory as ``lower in hierarchy'' than the general claims of praxeology. Furthermore, Megger controversially says that ``[…] we will use the subjective degree of conviction that certain events can occur to explain why people who »take part in games« will undertake certain actions'' 
%\label{ref:RNDPEsAb0NuHP}(Megger, 2021, p.92).
\parencite[][p.92]{megger_sprawiedliwosc_2021}. %
 How do we understand subjective probability here? Game theory cannot work unless there is a~certain ``top-up coordination'' or a~set of conditions that maintain a~given situation in the game model frames. On the other hand, I~believe this is due to the possibility of players having ``outside'' feelings about the game scheme, additional costs associated with participating in the game, and different valuations in a~strictly axiological sense. Then there's no controversy.



Nevertheless, the inclusion of certain aspects of game theory in the general praxeological framework should be considered successful and merit further development, despite some trouble. In particular, the author of the book emphasises on page 90 that ``compliance'' in the course of the game, which can naturally occur in the real world, ``disrupts'' the game in an ontological way, i.e., its principles do not fit the temporary model of the real situation.



Further on, Megger 
%\label{ref:RNDio21YOWIAt}(2021, pp.91–93)
\parencite*[][pp.91–93]{megger_sprawiedliwosc_2021} %
 refers to recursive problems, where he analyses the prisoner's dilemma for a~finite number of rounds in which players know the amount of iteration and in which they don't. Here too, we must agree with the author's analysis. However, the problem remains that the analyses presented are not as general and important as praxeological theorems.



\subsection{\itshape Intellectual Property and Welfare}



The last subsection 3.4., called \textit{Intellectual Property and Welfare} (Pol: \textit{Własność Intelektualna a~Dobrobyt}), involves aspects of intellectual property rights and welfare 
%\label{ref:RNDuh7Fxc4TQZ}(Megger, 2021, pp.94–101).
\parencite[][pp.94–101]{megger_sprawiedliwosc_2021}. %
 After providing an appropriate introduction to various positions in this field, particularly the libertarian perspective based on RAWE principles\footnote{The libertarian philosophy generally opposes intellectual property rights, viewing them as significant restrictions placed on the economy by the government. Indeed, those regulations have a~vital effect on economic development.}, the discussion shifts to the rate of development, the spread of innovation, and technical implementations. Megger convinces us that he supports Rothbard's claim that patents, copyright, and other similar legislation can greatly divert the monetary streams spent on diverse research, which greatly affects the process of ``real'' innovation emergence 
%\label{ref:RNDX7eyMyCvE0}(Megger, 2021, p.97).
\parencite[][p.97]{megger_sprawiedliwosc_2021}. %
 However, he later expresses some doubts about this claim. On the one hand, it is true that potential intellectual property protection would greatly benefit the particular companies. These institutions, thanks to the greater profits from monopoly rent, were able to accumulate a~greater amount of funds, which they could then dedicate to the development or expansion of their production base. However, comprehending innovation and determining which inventions or concepts to ``test in practice`` in a~developing economy pose challenges.



In fact, the game theory scheme presented 
%\label{ref:RNDmMdDTckMDa}(Megger, 2021, p.98)
\parencite[][p.98]{megger_sprawiedliwosc_2021} %
 is interesting, and the lessons drawn from it are informative, but such methods or algorithms would rather support potential decisions if their mathematical structure were more extensive. Of course, this ``dilemma of innovation implantation'' is correct in certain stable and predictable conditions, but in its current form, it is too simplistic. Perhaps some dynamical game theory tools, as well as some estimation tools associated with risk assessment, would be more appropriate. Furthermore, the implementation of this scheme does not contradict any assertions about a~free-market or regulated economy, meaning that the conclusions drawn from the presented reasoning can be applied to various types of economic environments. Only when certain conditions are satisfied in a~specific manner does a~framework for decision support emerge.



However, from the ASE perspective, the introduction of technical, organisational, and other innovations always requires the appropriate adaptation to the current production structure, both in real and intellectual terms. In order for the ``innovation'' understood in any way to become a~true development, it must be matched to an existing combination of complementary goods closed and coordinated under the given institutional circumstances---for example, in a~production company 
%\label{ref:RNDrdDkngyQbV}(Bylund, 2015).
\parencite[][]{bylund_explaining_2015}. %
 Moreover, the external aspect necessitates adjustment, where the production of new or improved goods consistently faces the pressure of profit-driven economic calculations. In a~market economy, we always verify innovation for goods that are closer to consumption within a~specific production structure, never \textit{in vacuo} 
%\label{ref:RNDVc5l6VCw64}(Rothbard, 2009, pp.509–556 [1962])
\parencite[][pp.509–556 \mbox{[1962]}]{rothbard_man_2009} %
 Only the market mechanism is able to verify real innovations in practice, not just those \textit{on paper}. Applying only some equilibrium-like scheme will not enable a~proper perception of this phenomenon, as will introducing some static comparison between different, completely counterfactual scenarios of dynamical nature.



Moreover, the assessment of differently estimated risk measures is not a~very objective thing because it also depends on various types of information that belong to both social and technical dimensions. Megger undoubtedly highlights crucial aspects of running a~company in a~dynamic economic environment where the need for innovation necessitates the recognition of risk. But is it not a~fundamental aspect of human action, whether or not we operate in a~market economy? Neverlethess, the book's author, correctly invokes Huerta de Soto's, Kirzner's, Mises's, and Schumpeter's claims about entrepreneurship and economic development 
%\label{ref:RNDeSc5hWo4IV}(Megger, 2021, pp.99–100).
\parencite[][pp.99–100]{megger_sprawiedliwosc_2021}. %
 Thus, by aiming for the leading role of an entrepreneur as an innovator, manager, and creative coordinator, it becomes clear how to properly perceive these issues. However, when it comes to safeguarding innovation through specific regulatory frameworks, it's important to acknowledge that in a~free market, businesses have the freedom to handle their intellectual inventions in any manner, such as by securing them as a~technological or organisational secret. Companies can create solutions to make it impossible to ``decode'' their potentially copyable products, sign appropriate contracts with workers, or bind recipients and contractors by appropriate clauses. The ball is still on the free market side.



Even if some companies copy some solutions from other ``innovative and advanced'' firms, consumer tastes can still be associated with different characteristics of goods or companies' activities. It is not a~given that economic agents would view theoretically similar goods with different purchase prices as homogeneous or even substitutable. Keep in mind that the possibility of raising goods' prices or restricting competition, \textit{ceteris paribus}, significantly impacts the ability to allocate a~given amount of funds to alternative purposes. Remember, expenditures for both consumption and production heavily depend on the level of capital accumulation and its forms of \textit{release}, which vary based on time preferences and monetary demand 
%\label{ref:RNDDUJvXbXj2z}(Rothbard, 2009, pp.348–362, 367–420 [1962]).
\parencite[][pp.348–362, 367–420]{rothbard_man_2009}.%




As a~result, protection policies in one sector of the economy can positively influence one company's real productivity but significantly slow down the productiveness (as well as the real value) of other companies, which would not benefit from greater capital accumulation. This illustrates that the argument does not target ``market'' innovations, but rather the dissemination of ``scientifically'' perceived innovations.



\section{Summary}

While the reviewed book represents an improved master thesis of Megger's authorship, his subsequent works present refined and extended aspects of its substantial analysis, establishing and elucidating the content that forms the basis for qualitative research. In fact, even though many of his claims are controversial, not exhaustively described, and reveal some understatements, Megger presented mature arguments against the fundamental theories of Rothbard and other Austro-libertarians. Moreover, we must appreciate Megger's extensive literature review.



In fact, as a~reviewer, I~strongly advocate many of Rothbard's claims on welfare economics and, on the other hand, support various novel arguments by Polish writers in this issue. However, my goal was not to present my substantial position or conduct a~comprehensive survey in this field, but rather to highlight key aspects of the book under review. I~hope that the discussion in the substantial part of the reviewed book delivers resolution to this very intense debate on RAWE.



In conclusion, I~believe that the book \textit{Justice in Welfare Economics. Libertarianism and the Austrian School} was and still is an interesting position in welfare economics, and I~think that the author would significantly improve his statements and prepare very valuable content on the substantial field of welfare economics.



\vspace{15mm}%
{\subsubsectit{\hfill Abstract}}\\
{This text reviews David Megger's 2021 book entitled \textit{Justice in Welfare Economics. Libertarianism and the Austrian School} (in Polish: \textit{Sprawiedliwość w~Ekonomii Dobrobytu, Liberatarianizm i~Szkoła Austriacka}). The review takes a~critical approach, highlighting the most significant aspects of the presented considerations and emphasising their uniqueness and complexity. I~intend to extensively discuss the author's theses concerning the modification of the fundamental claims of Austrian school representatives about justice and welfare, highlighting both their strengths and weaknesses.}\par%
\vspace{2mm}%
{\subsubsectit{\hfill Keywords}}\\
{Austrian school of economics, libertarianism, welfare economics.
}%



\end{newrevengenv}




%\sekcja{Review articles}{Artykuły recenzyjne}
%\input{REV_Mscislawski/Mscislawski.tex}









%\clearpage
%\thispagestyle{plain}
%\input{Autorzy.tex}
%\thispagestyle{plain}
%\input{Stopka.tex}
%\thispagestyle{plain}

\end{document}
