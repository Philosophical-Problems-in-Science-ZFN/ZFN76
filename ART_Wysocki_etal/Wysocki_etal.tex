\begin{artengenv}{Igor Wysocki, Łukasz Dominiak}
	{Social welfare, interventionism, and indeterminacy: In defense of Rothbard\edtfootnote{This research was funded in whole or in part by the National Science Centre, Poland, grant number 2020/39/B/HS5/00610.}}
	{Social welfare, interventionism, and indeterminacy\ldots}
	{Social welfare, interventionism, and indeterminacy: In defense of Rothbard}
	{Nicolaus Copernicus University in Toruń}
	{The present paper argues that Rothbard's economic case against the state is more robust than suggested by his critics. The charge that it might be anemic is based on the suggestion that we can say literally nothing about the way governmental acts bear on social utility. Contra this supposition we submit that Rothbard's critics missed the fact that the effects of governmental interventions might be actually indeterminate in two ways: weakly or strongly. If the indeterminacy involved in his welfare theory is weak, then his economic criticism of the state is more robust than envisaged by these authors. To the effect that this indeterminacy is indeed weak we advance the following reasons: Rothbard's understanding of the Unanimity Rule; the avoidance of the contradiction allegedly committed by Rothbard over one and the same page of his famous essay; his economic criticism of interventionism being better aligned with his overall ethical anti-governmental stance; the principle of charitable reading, which cuts across all of the previously stated reasons. If our arguments count for something, then we are warranted in claiming that Rothbard is indeed able to say something about social utility under interventionism. And if so, then his criticism of interventionism should be viewed as robust rather than anemic.
	}
	{indeterminacy, interventionism, social welfare, Murray~N.~Rothbard, welfare.}



\section{Introduction}

\lettrine[loversize=0.13,lines=2,lraise=-0.03,nindent=0em,findent=0.2pt]%
{B}{}ryan Caplan 
%\label{ref:RNDKcMbzp7U9L}(1999, p.834)
\parencite*[][p.834]{caplan_austrian_1999} %
 claims that Murray Rothbard's welfare theory provides only a~weak basis for the criticism of governmental interventions. Specifically, Caplan argues that what Rothbard can at most establish is that these interventions have indeterminate effects on social utility. It is true, as demonstrated by Joseph Salerno 
%\label{ref:RNDwyjoeKjIF6}(1993, p.131),
\parencite*[][p.131]{salerno_mises_1993}, %
 that Rothbard does not show that governmental interventions decrease social welfare and so this fact might have prompted Caplan to make the said charge. However, there are still two possible sorts of indeterminacies left to consider, given Rothbard's anti-governmental stance. For the impact of governmental interventions on social welfare might be indeterminate in a~strong or a~weak sense. In the strong sense, we cannot say whether these interventions increase, decrease or leave social utility unaffected. By contrast, in the weak sense, we cannot say only whether they decrease or leave social utility as it was although what we can say is that they never increase it. Now if Rothbard's criticism of governmental intervention were to involve the strong indeterminacy, then it would indeed be anemic. If, on the other hand, the indeterminacy appealed to in his welfare theory were to be weak, then his criticism of the government would be much more radical than suggested by Caplan.



In the present paper we argue that the Rothbardian welfare economics\footnote{An anonymous reviewer rightly noted that it is not very clear whether in this paper we defend Rothbard himself or his welfare theory. What we can offer as a~reply is that this paper is meant to be primarily theoretical (even if interpretive at times). Therefore, its main focus is to defend Rothbard's welfare \textit{theory} rather than its author. However, by defending the theory, we, \textit{nolens volens}, defend its author. Given this, irrespective of whether we speak of ``the Rothbardian welfare theory'' or indeed of ``Rothbard's welfare theory'', it is \textit{always} the theory itself that we intend to defend. } should be interpreted as claiming that the effects governmental interventions have on social welfare are indeterminate\footnote{A~compelling case can be made that according to Rothbard's welfare economics it does not make sense to call effects of state interventions ‘indeterminate' to start with. Besides the fact that Rothbard himself does not call them ‘indeterminate', the idea that they could be indeterminate presupposes that interpersonal comparisons of utility can be made, although the result of such comparisons is, sometimes, indeterminate. However, Rothbard rejected the very possibility of making such comparisons. It is therefore better to say that Rothbard's conclusion that state interventions cannot increase social utility simply and trivially follows from his premise---afforded by his doctrine of demonstrated preferences---that interpersonal comparisons of utility are impossible than to say that the effects of state interventions are indeterminate. Nevertheless, Rothbard's critics base their argument on the concept of indeterminacy. Thus, our ambition in this paper is to meet them on their own grounds and show that even if one accepts their problematic conceptual framework, Rothbard still comes out victorious.} only in the weak sense, that is, that they can never increase it and that the only indeterminacy they involve reduces to whether they decrease or leave social utility unaffected. Hence, we believe that Rothbard's critique of governmental interventions should be viewed as much more radical than Caplan contends. We posit that unless we construed the concept of indeterminacy in the weak way, we would have to conclude that Rothbard contradicts himself over one and the same page of his paper \textit{Toward a~Reconstruction of Utility and Welfare Economics}, which is a~highly unlikely diagnosis and an extremely uncharitable thing to say. On the other hand, once we interpret the indeterminacy involved as the weak one, no contradiction ensues and the Rothbardian welfare theory is then unproblematically coherent. Moreover, this interpretation tallies better with both what we argue is the proper Rothbardian understanding of Pareto-Superiority and with his overall anti-governmental anarcho-capitalist stance\footnote{Rothbard's commitment to anarcho-capitalism is probably most plainly laid down in Rothbard 
%\label{ref:RNDIHdWgif7zQ}(2006 [1973]; 2009 [1970]; 2002 [1982]).
\parencites*[][]{rothbard_for_2006}[][]{rothbard_man_2009}[][]{rothbard_ethics_2002}. %
 For an excellent exposition of the Rothbardian \textit{moral} argument for the free market, see also Juruś 
%\label{ref:RNDP4gfdk63tg}(2012).
\parencite*[][]{jurus_w_2012}. %
 } and therefore with the broader Austro-libertarian framework adopted by this author.\footnote{One of the anonymous referees wondered why it is all important to revisit the debate over the Rothardian welfare economics. First of all, we believe that we (at least to some extent) contribute to showing that the free market---as opposed to governmental interventions---bears positively on social utility not only \textit{ex ante} but also \textit{ex post}. Granted, for libertarians, the defense of the free market is primarily of \textit{moral} nature. However, as acknowledged by Hausman and McPherson 
%\label{ref:RNDwO9MjrW0yK}(2006, p.172),
\parencite*[][p.172]{hausman_economic_2006}, %
 ``libertarians would like it to be the case that protecting freedom also makes people better off.'' After all, it is precisely the task of providing a~purely \textit{economic} argument in favor of the free market regime that Rothbard set himself in his paper \textit{Toward a~Reconstruction of Utility and Welfare Economics}. And we believe that our paper to some degree fills in the lacuna between the free market (understood as a~totality of \textit{rights}{-respecting exchanges) and its beneficial }\textit{economic} consequences. Second, we submit that the present paper also sheds more light on the Paretian Unanimity Rule, not only a~central tenet of the Austrian welfare economics in its Rothbardian version but also an important device adopted in mainstream economics. }



The present paper proceeds in the following fashion. Section 2 introduces the distinction between weak and strong indeterminacy, in terms of which the Rothbardian conception of the impact of governmental interventions on social utility can be analyzed. Section 3 illuminates the relation between the kind of indeterminacy and the strength of his economic criticism of the state. Section 4 argues that the weakly indeterminate character of state interventions into economy follows as a~corollary from Rothbard's commitment to the Paretian Unanimity Rule. Section 5 addresses the challenge levelled at the Rothbardian welfare theory to the effect that he contradicts himself in his assessment of the effects of governmental interventions upon social utility. Section 6 undertakes the problem of coherence of Rothbard's overall theoretical system under alternative interpretations of indeterminacy. Section 7 concludes.



\section{Strong \textit{vs} Weak Indeterminacy}

It is incontrovertible that Rothbard does not say that governmental interventions necessarily decrease social utility. As he himself points out 
%\label{ref:RNDxs4sSCJa1z}(Rothbard, 1976, p.100),
\parencite[][p.100]{rothbard_praxeology_1976}, %
 ``we cannot say that any action of the State \textit{decreases} social utility.'' This fact is further confirmed by Salerno 
%\label{ref:RND777DlroeQv}(1993, p.131),
\parencite*[][p.131]{salerno_mises_1993}, %
 who says that, contrary to his own ``more radical conclusion'' which is indeed ``able to completely discount any gains, in terms of direct utility or exchangeable goods, that accrue to the interveners and their beneficiaries,'' what Rothbard ``has ably demonstrated on purely scientific grounds'' was only that governmental interventions never ``increase social welfare.'' 
%\label{ref:RNDPAKoLkrqJz}(Salerno, 1993, p.131)
\parencite[][p.131]{salerno_mises_1993} %
 This is also acknowledged by Caplan 
%\label{ref:RNDS1VrnMwyxj}(1999, p.833)
\parencite*[][p.833]{caplan_austrian_1999} %
 saying that Salerno's argument to the effect that the government does reduce social welfare is ``stronger than Rothbard's.'' Likewise, Kvasnička's 
%\label{ref:RND2kuyp0Q2YA}(2008, p.49)
\parencite*[][p.49]{kvasnicka_rothbards_2008} %
 criticism of Herbener 
%\label{ref:RND8zCYNKLPxm}(1997, pp.103–104)
\parencite*[][pp.103–104]{herbener_pareto_1997} %
 allegedly getting it wrong that ``involuntary interaction [is] ‘Pareto-Inferior''' implies that ``Rothbard says it correctly'' when he submits that ``it is only indeterminate.'' 
%\label{ref:RND4BySAMMWx8}(Kvasnička, 2008, p.49)
\parencite[][p.49]{kvasnicka_rothbards_2008}%




It seems that the fact that Rothbard does not claim that state interventions necessarily decrease social utility prompted some of the above authors to make a~charge against Rothbard that his economic criticism of state interventions is anemic. Most notably, Caplan 
%\label{ref:RNDgEXNTiPuEC}(1999, p.834)
\parencite*[][p.834]{caplan_austrian_1999} %
 argued that:



\begin{quote}
Rothbard could only claim the welfare effect of government intervention upon social utility is indeterminate. This is an important point because it shows that Rothbard's welfare economics provides a~much weaker defense of laissez-faire than usually assumed. In particular, Rothbard's own theory strips him of the ability to call any act of government inefficient. By denying others the ability to endorse state action in the name of efficiency, Rothbard also implicitly denies his own ability to reject state action in the name of efficiency. His welfare criterion justifies agnosticism about---not denial of---the benefits of state.
\end{quote}



There are other authors making a~similar point. For instance, Kvasnička 
%\label{ref:RNDhdryjHewnm}(2008, p.49)
\parencite*[][p.49]{kvasnicka_rothbards_2008} %
 concurs with Caplan to the effect that ``even if Rothbard's welfare theory was correct (which it is not), it would be a~very weak basis for a~critique of governmental meddling with the economy'' because governmental interventions, as any involuntary interactions, instead of being Pareto-Inferior are ``only indeterminate.'' Moreover, Prychitko 
%\label{ref:RNDaaIZjicZr8}(1993, p.576)
\parencite*[][p.576]{prychitko_formalism_1993} %
 maintains that, according to Rothbard, ``we must remain agnostic: we simply don't know'' what the effects of state interventions are. All these charges find some additional support in Rothbard 
%\label{ref:RNDmLfvHOPtXI}(2008, p.252)
\parencite*[][p.252]{rothbard_toward_2008} %
 himself saying that ``[a]s economists, we can therefore say nothing about social utility in this case, since some individuals have demonstrably gained and some demonstrably lost in utility from the governmental action.''



There are, however, two ways in which the effects of governmental interventions on social utility can be indeterminate. The first way in which they might be indeterminate is that we cannot say whether social utility decreases, increases or is left unaffected by governmental interventions. This sort of indeterminacy we label \textit{strong indeterminacy}. Note that if the impact of governmental interventions on social utility were strongly indeterminate, Rothbard would be right saying that ``we cannot say that any action of the State \textit{decreases} social utility.'' Indeed, we would not be able to say that because we would not be able to say anything, that is, whether these interventions increase, decrease or leave social welfare unaffected.



Now the second way in which governmental interventions might have indeterminate influence on social utility is that we cannot say whether social utility decreases or is left unaffected, even though what we can say for sure is that it never increases as a~result of such interventions. This kind of indeterminacy we label \textit{weak indeterminacy}. Note again that if the influence of governmental interventions on social utility were to be weakly indeterminate, Rothbard could neither say ``that any action of the State \textit{decreases} social utility'' because he would not be able to say whether state interventions decrease or leave social welfare unaffected. Therefore, more specifically, even though he would be justified in saying that state interventions never increase social utility, he would not be able to determine whether they decrease or leave it unaffected and so, he would not be prepared to state with certainty that they decrease it.



\section{Indeterminacy and Economic Criticism of the Government}

As we mentioned above, Caplan and other authors criticize Rothbard for making a~very anemic economic case against the state. The reason they cite for this criticism is that, according to Rothbard, the effects of governmental interventions on social welfare are indeterminate. However, they are not specific enough about the kind of indeterminacy involved in Rothbard's welfare theory. After all, as we saw above, there are two possible types of such indeterminacy and we submit that the Rothbardian criticism of the state would indeed be anemic, as the above-mentioned authors claim, only if the indeterminacy involved in his theory were \textit{strong indeterminacy}. By contrast, his criticism would by no means be anemic if the indeterminacy he talks about were \textit{weak indeterminacy}. For, if the indeterminacy in question were weak, Rothbard would indeed be able to say that governmental interventions can never increase social utility. And that does not seem to be a~weak criticism of the state at all.



What is yet due at this point is a~word of more precise explanation of why the criticism of governmental interventions following from the adoption of weak indeterminacy would be robust indeed. Note that if the state were an institution which is inherently powerless to increase social utility, there would be no welfare-related point of having it in the first place. Additionally, it would be possible for the state to decrease social welfare although it must be granted that one cannot say with apodictic certainty whether the state would do so in any particular case of its intervention. Given the fact that under this interpretation the state could not increase social welfare and might indeed even decrease it, the Rothbardian criticism appears to be almost as robust as it can get. After all, if showing that a~given institution is structurally unable to ever improve social utility does not amount to a~robust criticism of it, then almost nothing does.



Now note that Caplan and those other authors do not provide a~single reason to prefer strong indeterminacy as the proper way of interpreting the Rothbardian welfare theory. This should come as no surprise because they do not even draw the very distinction between strong and weak indeterminacy. Thus, even if their criticism of Rothbard's economic case against the government happened to be true, it would nonetheless be unjustified as far as their argument goes. For, as we already made clear, the anemic character of the economic criticism of the government does not follow from the indeterminate nature of its impact on social utility. It would only follow if the indeterminacy in question were to be weak---but this, however, was not established. Moreover, we submit that there are actually four reasons to believe that the indeterminacy in question should be construed as \textit{weak indeterminacy}. First of all, it follows from the way Rothbard understands the Unanimity Rule, a~crucial element of his welfare economics. Second of all, it is only weak indeterminacy that would save Rothbard from contradicting himself within the confines of one and the same page of his seminal essay \textit{Toward a~Reconstruction of Utility and Welfare Economics}. On the other hand, assuming strong indeterminacy would enmesh him in the contradiction. Certainly, it would be uncharitable to maintain that this author makes two mutually exclusive claims over one and the same page, especially when there is an interpretation available that can easily block making such an improbable charge. Third, weak indeterminacy translates into more robust economic criticism of the state and therefore it best aligns with his anti-governmental ethical stance, thus rendering Rothbard's overall position more coherent. Finally, as already suggested while presenting the second reason, interpreting Rothbard's welfare economics in terms of weak indeterminacy would abide by the principle of charity.



\section{Rothbardian Understating of the Unanimity Rule }

We submit that the fact that Rothbard adopts the Unanimity Rule as a~criterion of welfare-enhancing exchanges provides a~reason to believe that the indeterminacy involved in his theory about the impact of governmental interventions on social utility is weak (and hence, that his criticism of the state is robust rather than anemic). How Rothbard conceives of the said rule is evinced by the following lengthy quote:



\begin{quote}
This Rule runs as follows: We can only say that ``social welfare'' (or better, ``social utility'') has increased due to a~change, if no individual is worse off because of the change (and at least one is better off). If one individual is worse off, the fact that interpersonal utilities cannot be added or subtracted prevents economics from saying anything about social utility. Any statement about social utility would, in the absence of unanimity, imply an ethical interpersonal comparison between the gainers and the losers from a~change. If X~number of individuals gain, and Y~number lose, from a~change, any weighting to sum up in a~``social'' conclusion would necessarily imply an ethical judgment on the relative importance of the two groups. 
%\label{ref:RNDI7LWQSSyob}(Rothbard, 2008, pp.244–245)
\parencite[][pp.244–245]{rothbard_toward_2008}%
\end{quote}




Note that, according to Rothbard, there is only one sort of change after the occurrence of which an increase in social utility can be justifiably predicated and that is the situation wherein at least one party benefits and nobody loses. By contrast, in case in which one party gains while the other loses, that is, ``in the absence of unanimity,'' we must be left with an indeterminate verdict as to the impact of such changes on social utility. Now the question arises: is the verdict under consideration strongly or weakly indeterminate?



We claim that the corollary of Rothbard's contention to the effect that ``we can only say that ‘social welfare' [...] has \textit{increased} [...], if no individual is worse off because of the change (and at least one is better off)'' is the weak indeterminacy interpretation of the way governmental interventions influence social utility. After all, if ``we can only say'' that social welfare increases if nobody loses utility and at least one person gains it, then in the situation wherein there are both utility gainers and losers it must be the case---by way of contraposition---that what we cannot say is precisely one thing only: that social welfare was enhanced. And since we cannot say that it was enhanced, we are justified in saying that it was not enhanced. This in turn leaves us with indeterminacy only about two things, that is, whether (a) social utility diminished or (b) remained at the same level. But this is exactly the weakly indeterminate reading of the way Rothbard conceives of governmental acts vis-à-vis social utility. For indeed, it is the weak indeterminacy interpretation that has it that we are warranted in being agnostic only about whether governmental interventions decrease social utility or leave it unaffected.



To make our point even clearer, note that what Rothbard claims is that ‘We can only say that social welfare increases if no one loses in utility' (and at least one person gains). We contend that what it means is that only then it is true that social welfare increased. Now by contraposition it must be the case that ‘If someone loses in utility, then we cannot say that social welfare increases.' Again, we submit that what it means is that it is false that social utility increases in such a~case.\footnote{But why do we claim so? Does not Rothbard say that ``[i]f one individual is worse off, the fact that interpersonal utilities cannot be added or subtracted prevents economics from saying anything about social utility'' rather than it prevents economics from saying that social welfare increases? He does, but then he adds that we should ``conclude therefore that \textit{no government interference with exchanges can ever increase social utility}.'' Thus, the point is that it is up for debate how to understand Rothbard's stance on what is going on when someone loses in utility. Our claim is that it is better to understand him as saying that it is false that social utility increases in such a~case than that we cannot say absolutely anything about it. Why? For one thing, because it avoids what Prychitko called ``a careless self-contradiction'' in Rothbard (see section 5 below). Second, because opting for the agnostic reading renders Rothbard's second welfare theorem---that ``no act of government can ever increase social utility''---disappointingly uninformative. Of course, ``no act of government can ever increase social utility'' if no act of government can ever decrease it, increase it or leave it as it is (due to impossibility of interpersonal comparisons of utility). To be sure, then Rothbard's second welfare theorem follows as a~matter of logic, but it follows vacuously, due to the antecedent being false. Finally, the agnostic reading gives rise to the question of why, if we cannot say absolutely anything about social utility in the case of governmental intervention, Rothbard is so keen on saying that therefore ``no act of government can ever increase social utility'' rather than that no act of government can ever \textit{decrease} social utility? We are equally in the dark about both of these effects. Would then honesty not require that an economist use less prejudicial language in expressing his agnosticism about the effects of state interventions? Our reading of Rothbard avoids these and other problems. Or so it seems to us.} However, if it is false that social utility increases, then it must be true that it does not increase. But does it mean that, therefore, social utility decreases? This does not follow. For even though social utility does not increase, it is still not clear whether it decreases or stays at the same level. This, of course, means that social utility is indeterminate but only in the weak sense, that is, only between two possibilities of decreasing or remaining constant. As to the third possibility, it is determined: ``no act of government can ever increase social utility.'' 
%\label{ref:RNDsMoEH91qUU}(Rothbard, 2008, p.253)
\parencite[][p.253]{rothbard_toward_2008} %
 Therefore, it seems that the weakly indeterminate character of the governmental bearing on social utility also follows from the Rothbardian understanding and commitment to the Unanimity Rule.



\section{The Contradiction Problem}

But why assuming \textit{strong indeterminacy} would portray Rothbard as committing simple contradiction? For on the very same page he says that: ``[a]s economists, we can therefore say nothing about social utility in this case, since some individuals have demonstrably gained and some demonstrably lost in utility from the governmental action.'' 
%\label{ref:RNDK8mEHJNkz8}(Rothbard, 2008, p.252)
\parencite[][p.252]{rothbard_toward_2008} %
 And right after it, he states that: ``[w]e conclude therefore that \textit{no government interference with exchanges can ever increase social utility}… Given the fact that coercion is used for taxes, therefore, and since all government actions rest on its taxing power, we deduce that: \textit{no act of government whatever can increase social utility}.'' 
%\label{ref:RNDkRnf4XTlLi}(Rothbard, 2008, p.252)
\parencite[][p.252]{rothbard_toward_2008} %
 Now if the indeterminacy were to be strong, the latter passage would be inconsistent with the former because the former would exclude the possibility of knowing that governmental interventions never increase social utility. After all, strong indeterminacy implies not knowing whether social welfare diminished, stayed unchanged or increased.



Indeed, this was perspicuously noted by Prychitko 
%\label{ref:RND2IJpLBXfkC}(1993, p.575),
\parencite*[][p.575]{prychitko_formalism_1993}, %
 who contends that ``the additional claim Rothbard makes about social welfare under interventionism---specifically, that no state intervention can ever increase social utility---is a~careless self-contradiction.'' This author goes on to indicate that ``Rothbard argues, ‘economics can say nothing about social utility in this case. Again. We must remain agnostic: we simply don't know.'' In the very next paragraph, Prychitko 
%\label{ref:RNDiuxAubW4Qw}(1993, p.576)
\parencite*[][p.576]{prychitko_formalism_1993} %
 additionally notes that:



\begin{quote}
Yet his next sentence reads: ``We conclude therefore that \textit{no government interference with exchanges can ever increase social utility}.'' In fact, he goes so far as to proclaim that ``since some lose by the existence of taxes, therefore, and since all government actions rest on its taxing power, we deduce that: \textit{no act of government whatever can increase social utility}.'' Somehow Rothbard has leapt from agnosticism to certainty: the state definitely cannot increase social utility. His italics suggest we take his claim seriously, as an apodictic truth. But it's more apoplectic than apodictic.
\end{quote}



Granted, as we pointed out above, at least \textit{prima facie} there seems to be a~tension between Rothbard's prior assertion to the effect that ``economics can say nothing about social utility'' in case of state interventions and his apparently bolder statement which has it that ``no act of government whatever can increase social utility.'' Clearly, if it is literally \textit{nothing} that economics can say about the impact of governmental interventions upon social welfare, then this statement warrants greater skepticism than his more informative assertion to the effect that it is only increases in social utility that the state cannot bring about. In other words, Rothbard's first assertion does not seem to rule out \textit{any} effect of governmental acts on social welfare, whereas his subsequent statement explicitly rules out the possibility of governmental interventions ever increasing social utility.



And yet, there is a~neat way out of this seeming contradiction. A~solution appears to hinge on the way we interpret the Rothbardian contention as to the alleged inability of economics to issue \textit{any} welfare-related verdicts concerning the impact of governmental acts on social utility. We posit that if only we construe the first skeptical assertion by Rothbard along the lines of weak indeterminacy, then the contradiction between his two statements disappears. After all, if \textit{nothing} that economics can say about social utility in case of governmental interventions is only weakly indeterminate \textit{nothing}, then the proposition expressed by Rothbard's first pronouncement is identical with the one expressed by his next sentence. But, most certainly, if the relation between two statements is that of propositional identity, then they cannot contradict one another by any means. Still in other words, if the indeterminacy is interpreted as weak, then it only means that we cannot say whether social utility decreased or stayed unchanged, something perfectly consistent with saying that it necessarily did not increase. By contrast, if we were to conceive of the first Rothbardian assertion in terms of \textit{strong indeterminacy}, then the contradiction would inevitably ensue, for Rothbard would be effectively saying two inconsistent things at the same time, that is, (a) that we cannot say literally anything about the way governmental acts impact social utility and (b) that whatever the effect of state's intervention upon social welfare is, one thing we know for certain is that the state is powerless to increase social utility.



Now given that it would be most uncharitable to attribute to Rothbard self-contradiction within the space of one and the same page of his essay; taking into consideration the fact that the hypothesis according to which Rothbard contradicted himself over one and the same page is highly unlikely; and, most importantly, having at one's disposal an alternative hypothesis that easily explains away the alleged contradiction and coheres better with the rest of Rothbard's theory, we claim that the most plausible interpretation of \textit{nothing} that economics can say about the influence of state's intervention on social welfare is only weakly indeterminate \textit{nothing}, that is, such that is indeed informative, for it rules out the possibility of governmental acts ever increasing social utility.



\section{Coherence of Rothbard's Economic and Ethical Criticisms of the State}

Now Caplan and other authors suggest that there is something wrong with a~putative fact that Rothbard's economic criticism of the government is anemic. If they had not thought so, they would not have made a~charge of it in the first place. Allegedly, it has something to do with his overall anti-governmental stance, for, on the one hand, he is an adamant enemy of the state as far as ethics is concerned while he is presumably only a~weak critic of the government on economic grounds on the other. Besides this fact suggesting that the Rothbardian system might be incoherent across these two branches, it also does not tally well with what Rothbard says about ``a fortunate utilitarian result of the free market'', which is ``by far the most productive form of economy known to man''.\footnote{It is well-worth stressing that, according to Rothbard, it is not only \textit{ex ante} but also \textit{ex post} that the free market is economically more efficient than interventionism. Says Rothbard 
%\label{ref:RNDAli5XD0Vd0}(2009, p.891):
\parencite*[][p.891]{rothbard_man_2009}: %
 ``[T]he free market has a~smooth, efficient mechanism to bring anticipated, \textit{ex ante} utility into the realization and fruition of \textit{ex post}. The free market always maximizes \textit{ex ante} social utility; it always tends to maximize \textit{ex post} social utility as well.'' More, he goes on saying that ``the divergence in \textit{ex post} results between free market and intervention is even greater than in \textit{ex ante}, anticipated utility.'' Upon saying it, Rothbard brilliantly illustrates how the state's interventions prove to be counter-productive. For example, the imposition of a~\textit{maximum} price set \textit{below} a~market-clearing price (i.e. one of the two types of effective price control) inevitably leads to the creation of an artificial shortage. Hence, however benevolently motivated and however beneficial \textit{in expectation}, price control policies fail spectacularly \textit{ex post}. By contrast, as demonstrated by Rothbard, free market is a~self-correcting system. It is losses that allow for weeding out those entrepreneurs that do not serve their customers well and it is continual profits that constitute a~signal that given entrepreneurs do increase the consumers' utility \textit{ex post}. Granted, there is no guarantee that \textit{each} market exchange is going to be mutually beneficial \textit{ex post}. However, as perspicuously observed by Rothbard 
%\label{ref:RNDsYLfxrjoa4}(2009, p.885),
\parencite*[][p.885]{rothbard_man_2009}, %
 ``[p]rofits and losses spur rapid adjustment to consumer demands''. All in all, as far as the \textit{ex post} welfare goes, the market still performs better than interventionism. } 
%\label{ref:RND6GODaJ8Ghj}(Rothbard, 2006, p.48)
\parencite[][p.48]{rothbard_for_2006}%




However, we contend that the apparent incoherence cited above would be attenuated or would disappear completely if the indeterminacy of state's interventions were to be interpreted as weak. The reason is that then Rothbard's economic criticism would be more robust, proving that state's interventions cannot ever increase social utility and thus calling into question the very economic \textit{raison d'être} of the state. After all, then the state would transpire to be at least redundant since it would be economically indifferent at best and harmful at worst. This, of course, would tally much better with Rothbard's otherwise well-known vehement criticism of the state and with his overall anarcho-capitalist stance.



It should be clear that coherence is a~virtue of any theoretical system. So, whenever possible, we should strive for it either \textit{via} theoretical revisions or reinterpretations that allow us to achieve it. Because our distinction between weak and strong indeterminacy, and especially the appeal to the former, bolsters coherence within the Rothbardian system whereas its critics' indiscriminate idea of indeterminacy threatens it, this very fact speaks in favor of supporting our reading of Rothbard's welfare economics. Besides, interpreting his utility theory in a~way that suggests incoherence in his overall system would run against the principle of charity, particularly when there is an alternative interpretation easily avoiding it. Finally, because Caplan and other critics believe, as we pointed out above, that the alleged weakness of Rothbard's economic case against the state and the incoherence it engenders constitute a~vice in his general theoretical system, these authors too should conceive of our distinction as preferable to their own indiscriminate idea of indeterminacy, for it would enable them to get rid of what they themselves consider a~vice.



\section{Conclusion }

The aim of this paper was to argue that---contrary to what some critics maintain---the Rothbardian theory of social utility under interventionism is by no means anemic. That is, the verdicts it reaches are more informative, and therefore less indeterminate, than its critics believe them to be. Specifically, we posit that Rothbard's welfare theory should be indeed construed as saying that there is one thing that we can say for certain; namely, that governmental acts are powerless to increase social utility.



The reasons we provided for the above contention are four-fold. First of all, in his welfare economics, Rothbard explicitly adopts the Paretian Unanimity Rule as the determinant of welfare-enhancing exchanges. What clearly follows as a~corollary from the way Rothbard interprets the said rule is only weakly rather than strongly indeterminate character of state interventions into economy. This in turn means that the best the government can do is to leave social utility unaffected, which calls into question this very institution at least as far social welfare is concerned. It should be noted that such a~conclusion reached by the Rothbardian welfare theory does not even remotely resemble supposedly agnostic conclusions attributed to it by its critics. Second of all, we argued that unless we construed the concept of indeterminacy in the weak way, we would have to conclude that Rothbard contradicts himself over one and the same page of his famous paper \textit{Toward a~Reconstruction of Utility and Welfare Economics}. On the other hand, if we interpret the indeterminacy involved as the weak one, no contradiction ensues and the Rothbardian welfare theory is then rendered consistent. Third, it is only under weak indeterminacy interpretation that Rothbard's overall theoretical system achieves coherence. And fourth, we pointed to the principle of charity, which would be obeyed only if we stick to our fine-grained distinction between weak and strong indeterminacy. All these reasons operating \textit{via} the discrimination between weak and strong indeterminacy support the final conclusion that Rothbard's economic criticism of the state is much more radical than his critics believe it to be.





\end{artengenv}

