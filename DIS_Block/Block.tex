\begin{artengenv}{Walter Block}
	{Response to Wysocki's \textit{Rejoinder to Block on indifference}}
	{Response to Wysocki's \textit{Rejoinder to Block on indifference}}
	{Response to Wysocki's \textit{Rejoinder to Block on indifference}}
	{Loyola University New Orleans\label{block-rejoinder-firstpage}}
	{Wysocki 
	%\label{ref:RNDxkl9Os0nTO}(2024)
	\parencite*[][]{wysocki_rejoinder_2024} %
	 is a~critique of Block 
	%\label{ref:RNDrhLTsy344h}(2022).
	\parencite*[][]{block_response_2022}. %
	 The present paper is a~response to the former. We are in effect debating the best reaction to Nozick 
	%\label{ref:RNDzIdYuP1FWa}(1977)
	\parencite*[][]{nozick_austrian_1977} %
	 which criticized Austrian economics on the ground that it makes two claims that are incompatible with one another. On the one hand, the praxeological school is noted for its aversion to the concept of indifference. On the other hand, the Austrian school also accepts supply and demand curves, and diminishing marginal utility. These three concepts imply homogeneous elements that comprise them. But if they are truly homogeneous, people ought to be indifferent between the different elements of them. Hence, the tension, not to say logical contradiction, in this perspective. Block 
	%\label{ref:RND05at9zeX1m}(1980)
	\parencite*[][]{block_robert_1980} %
	 was an attempt to respond to Nozick 
	%\label{ref:RNDwfdndOLuML}(1977).
	\parencite*[][]{nozick_austrian_1977}. %
	 Hoppe 
	%\label{ref:RND2B8CvVtdhG}(2005a; 2005b; 2009)
	\parencites*[][]{hoppe_note_2005}[][]{hoppe_must_2005}[][]{hoppe_further_2009} %
	 and Wysocki 
	%\label{ref:RND74fJDSNrfQ}(2016; 2017; 2021; 2024)
	\parencites*[][]{wysocki_indifference_2016}[][]{wysocki_caplan_2017}[][]{wysocki_problem_2021}[][]{wysocki_rejoinder_2024} %
	 who supports Hoppe, maintain that Block's refutation of Nozick 
	%\label{ref:RND4Fu3WZTtSh}(1977)
	\parencite*[][]{nozick_austrian_1977} %
	 was not efficacious at all, at worst, or at best, certainly not fully successful.
	
	
	
	Specifically, Wysocki maintains that there is a~bifurcation between choosing and preferring; for example, no one is even aware of which foot goes first when entering a~restaurant, and, yet, one has to make a~choice about it. He avers that it is entirely possible to prefer to save either son, equally, while actually picking one, and not the other.
	}
	{indifference, supply and demand, diminishing marginal utility, subjectivism, behaviorism, psychologizing, preference.}




\section{Introduction}



\lettrine[loversize=0.13,lines=2,lraise=-0.03,nindent=0em,findent=0.2pt]%
{W}{}ysocki 
%\label{ref:RNDFqfRxaFZYj}(2024)
\parencite*[][]{wysocki_rejoinder_2024}%
\footnote{Unless otherwise indicated, all cites to Wysocki will be to this one article.} is a~critique of Block 
%\label{ref:RNDtFFctKpzqN}(2022).
\parencite*[][]{block_response_2022}.%
\footnote{And, also critique of 
%\label{ref:RNDTWMTNkwaq4}(Barnett, 2003; Block and Barnett II, 2010, and Block, 1980; 2009a; 2009b; 2012; 2019).
\parencites[][]{barnett_modern_2003}[][and Block]{block_robert_1980}[][]{block_rejoinder_2009}[][]{block_rejoinder_2009-1}[][]{block_response_2012}[][]{block_rejoinder_2019}.%
} The present paper will defend the latter against the former. My critic alleges that I~err with regard to indifference, subjectivism, Austrian economics in general, the law of diminishing marginal utility, and embraces the fallacy of behaviorism. I~shall be quoting sections of his 2024 paper, and responding to them, seriatim, in much the same order as he presents his critiques. This author urges me in my critique of Nozick 
%\label{ref:RNDVrxw8YTq0i}(Block, 1980)
\parencite[][]{block_robert_1980} %
 to embrace, instead, the defense of the praxeological school adumbrated by Hoppe 
%\label{ref:RNDASKOGpDFCK}(2005a; 2005b; 2009).
\parencites*[][]{hoppe_note_2005}[][]{hoppe_must_2005}[][]{hoppe_further_2009}.%
\footnote{For a~more complete list of Austrian concerns about indifference, see 
%\label{ref:RNDp8NFlKliI7}(Barnett, 2003; Block, 1980; 1999; 2003; 2007; 2009a; 2009b; 2012; 2019; Block and Barnett II, 2010; Block and Sotelo, 2012; Caplan, 2024; 1999; 2001; 2003; 2008; Hoppe, 2005b; 2005a; 2009; Hülsmann, 1999; Machaj, 2007; Nozick, 1977; O'Neill, 2010; Sotelo and Block, 2014; Wysocki, 2016; 2017; 2021; 2024).
\parencites[][]{barnett_modern_2003}[][]{block_robert_1980}[][]{block_austrian_1999}[][]{block_realism_2003}[][]{block_reply_2007}[][]{block_rejoinder_2009}[][]{block_rejoinder_2009-1}[][]{block_response_2012}[][]{block_rejoinder_2019}[][]{block_rejoinder_2010}[][]{block_response_2012}[][]{sotelo_indifference_2014}[][]{caplan_austrian_1999}[][]{caplan_probability_2001}[][]{caplan_probability_2003}[][]{caplan_trojan_2008}[][]{hoppe_must_2005}[][]{hoppe_note_2005}[][]{hoppe_further_2009}[][]{hulsmann_economic_1999}[][]{machaj_praxeological_2007}[][]{nozick_austrian_1977}[][]{oneill_choice_2010}[][]{sotelo_indifference_2014}[][]{wysocki_indifference_2016}[][]{wysocki_note_2017}[][]{wysocki_problem_2021}[][]{wysocki_rejoinder_2024}.%
}



In section II of this paper, we discuss my overlap with the views of Wysocki. Section III is given over to exploring the bones of contention between the two of us. The burden of section IV is to counter Wysocki's rejection of my time series claim: the after-action/before-action distinction. In section V~we discuss agency and strict preference; we conclude in section VI.



\section{Agreement}

Wysocki begins his paper by setting out the large areas of agreement. He states this fairly, even eloquently. I~have no criticism. However, I~would go further than he along these lines. He limits himself, not unreasonably to the narrow points at issue in this particular paper. I~would like to place it on the record that he and I~probably agree on 99\% of all issues in political economy. I~go even further: he and I~are co-authors on several occasions, and you don't get closer in this game than that 
%\label{ref:RNDXk5QPfj72e}(Block and Wysocki, 2018; Wysocki and Block, 2017; 2018; 2019; 2020; 2022; Wysocki, Block and Dominiak, 2019).
\parencites[][]{block_defense_2018}[][]{wysocki_note_2017}[][]{wysocki_analysis_2018}[][]{wysocki_homogeneity_2019}[][]{wysocki_crovelli_2020}[][]{wysocki_rejoinder_2022}[][]{wysocki_homogeneity_2019}.%




\section{Bones of contention}

Wysocki starts off this section of his paper with the



\begin{quote}
… view […] that if one is indifferent between x~and y, then one cannot logically choose between them. Or still in other words, if one cannot choose between x~and y, then x~and y~do not constitute economically distinct alternatives.
\end{quote}



The word, or the concept, ``indifference'' occurs in ordinary language all the time.\footnote{Physics, too, has a~technical language, which uses the same verbiage as ordinary language. For example, in that science, ``work'' = mass x~distance. But if someone is holding 20 pound barbells still, at arm's length, he will not be doing any ``work'' in the physics sense, since these weights do not travel through any distance. However, in ordinary language, this would constitute a~very heavy ``work'' out. } Before choosing, the grocer cares not one whit which pound of butter he gives to the customer. In Wysocki's case, chess versus taking a~nap, football vis a~vis going for a~walk, all is indifferent. Now consider technical language. We as economists, are never in a~position to say anything of the sort. All we see is the person choosing either football or a~walk, or choosing either chess or a~nap. We are not in a~position to aver, qua economist, any such thing as does Wysocki. Again, if this scholar is engaging in ordinary language, I~have no quarrel with his contention. But, if he thinks he is now speaking as an economist, which he now presumably also is doing, then I~cannot acquiesce in his statements.



Our author opines:



\begin{quote}
Equipped with this conceptual apparatus, we are now in a~position to spell out a~relevant difference between our account of choice and Block's. At this point, it is crucial to note that the individual's given behaviour underdetermines a~value scale on which she\footnote{I~cannot let pass my extreme annoyance at Wysocki's continual use of ``inclusive'' language. ``He'' includes ``he and she'' in the English language, whereas ``she'' includes only the fairer sex. However, I~forgive him. English is not his native language. He does not, then, perhaps, realize the importance of maintaining it as it was before the untoward influence of the feminists.} has acted. Or, to put this point more technically, there is a~one-to-many relation between a~certain act-token and an underlying value (preference) scale.
\end{quote}



My objection, is that no one can be indifferent between a~walk and football watching, and, yet, does the latter. This would appear to be a~violation of the areas of agreement that Wysocki acknowledges we share. To wit, he states: ``What we, most crucially, share with Block is the view that indifference cannot be demonstrated in action'' 
%\label{ref:RNDD35b9jV2NC}(see e.g., Block, 2009a; 2009b; Rothbard, 2011).
\parencites[see e.g.,][]{block_rejoinder_2009}[][]{block_rejoinder_2009-1}[][]{rothbard_toward_2011}. %
 Indeed, the very idea of action presupposes some preference. He now seems to be taking back this area in which we overlap.



Wysocki, nevertheless, continues down this path:



\begin{quote}
Alternatively, he might have (strictly) preferred watching football to anything else he saw as a~possibility. If so, then his value scale might be the following:
\vspace{-.7em}
\begin{enumerate}[label=(\arabic*)]

\item[]\makebox[-1.7em][l]{}V\textsubscript{2}

\item Watching football

\item Going for a~walk

\item Having a~nap

\item Playing a~game of chess

\end{enumerate}
\vspace{-.7em}
And this is precisely where our account diverges from Block's. For, it seems that according to Block action is a~manifestation of preference all across the board.
\end{quote}



The fact that I~choose bubblegum instead of readily available chewing gum is not \textit{insufficient} , as Wysocki avers, to establish that I~prefer the former to the latter. The fact that I~choose to propose marriage to woman A~instead of woman B~is not \textit{insufficient} to establish that I~prefer the former to the latter. The fact that I~choose biking instead of running is not \textit{insufficient} to establish that I~prefer the former to the latter. What, then, would be sufficient to demonstrate any of these claims? Let me repeat that just to make sure I~comprehend what he is saying. He is saying, and I~quote: ``the fact that the donkey moves to his right is in and of itself \textit{insufficient} to establish whether the donkey does prefer the right bale to the left one.'' I~find this highly problematic.



My debating partner explains:



\begin{quote}
For, the donkey might as well be indifferent between the two. In that case, the donkey would not be choosing between the two bales but indeed between something else -- most plausibly, between eating or starving.
\end{quote}



One possibility is to say that of course the animal is choosing life over death. But he is \textit{also} choosing right over left. This cannot be doubted, in the face of the fact the he\footnote{Not she.} actually moved to the right. I~cannot for the life of me see how a~movement to the right does not demonstrate preference for a~movement to the right, always assuming of course that there was no outside interference, such as being whipped on the left side, or anything like that. Does this also demonstrate, as Wysocki and Hoppe would have it, that this also reveals a~preference for life over death? That is a~bit of a~stretch. It is easy to avoid. \textit{Perhaps this is the burro's way of committing suicide, via overeating.} Wysocki and Hoppe are ignoring the basic element, a~move to the right indicates a~preference for a~move to the right. Instead, they are grasping at straws, maintaining that it necessarily discloses a~preference for life over death. It does no such thing, on the assumption that asses can die from overeating. My two Austrian colleagues focus on life and death, which is, in the best of cases for their side of the argument, uncertain. And they ignore what is directly in their faces: a~move to the right has to indicate something. And it could not possibly be anything else other than that the donkey preferred the hay to the right. It is as if they are asked ``why did the chicken cross the street?'' and they ignore the obvious answer: ``to get to the other side,'' and, instead, speculate on all sorts of irrelevancies: to save its life; to play chess with another chicken who lives across the street, etc.



But Wysocki is in no mood to concede anything to the arguments just made. Rather, he continues as follows:



\begin{quote}
Certainly, it is possible for the donkey to choose between the bales. But in that case, the donkey must have a~preference for one over the other. All in all, how many choices the actor faces depends on the Hoppean 
%\label{ref:RNDjuwpa2pyCF}(2005b)
\parencite*[][]{hoppe_must_2005} %
 correct description of action (or action under intentional description) and not on the actor's behaviour as extensionally described.
\end{quote}



I'm a~behaviorist because I~look to behavior to ferret out values, preferences. But human action constitutes behavior.



Nothing loath, Wysocki repeats this erroneous interpretation:



\begin{quote}
Whereas the fact that the donkey moves to the right is, for Block, a~decisive reason to conclude that the donkey prefers the right bale to the left one, we submit that this fact alone does not suffice to establish what the donkey prefers…
\end{quote}



My many times co-author puts his point more formally, by creating these two value scales:



\begin{quote}
\begin{enumerate}[label=(\arabic*)]
\item[]\makebox[-1.7em][l]{}V\textsubscript{3}
\item Eat from a~right bale of hay
\item Eat from a~left bale of hay
\item Starve
\end{enumerate}
\vspace{-.7em}
\begin{enumerate}[label=(\arabic*)]
%\item[or~V\textsubscript{4}]
\item[]\makebox[-1.7em][l]{}V\textsubscript{4}
\item Eat from either a~right \textit{or} a left bale of hay
\item Starve
\end{enumerate}
\end{quote}


He then states:



\begin{quote}
By contrast, Block avers that the donkey's behaviour unambiguously points to V3 as an underlying value scale, which we can allegedly infer from the very fact that the animal moved to the right rather than to the left.
\end{quote}



But maybe the donkey is a~right winger, and detests any move to the left. If so, V4 cannot be correct. More seriously, yes, V4 is correct, assuming away the suicide by overeating scenario. But just because V4 is correct does not logically imply that V3 is false. \textit{Both} could be true.



\section{The after-action/before-action distinction }

Wysocki now uses my time series analysis (indifference can exist before choices are made and preferences revealed, but not afterward), as a~vehicle to demonstrate my errors.



This Polish philosopher-economist again puts matters in a~formal manner:



\begin{quote}
\begin{enumerate}[label=(\arabic*)]
\item[]\makebox[-1.7em][l]{}V\textsubscript{5}
\item To earn money by giving up the 72\textsuperscript{nd }unit of butter
\item To earn money by giving any other unit
\end{enumerate}
\vspace{-1em}
\begin{enumerate}[label=(\arabic*)]
\item[]\makebox[-1.7em][l]{}or indeed by V\textsubscript{6}
\item To earn money by giving up \textit{any} unit of butter
\item To preserve all the units and earn no money
\end{enumerate}
\end{quote}



Wysocki clearly prefers V6 to V5. He states: ``Granted, when it came to the seller's action, he must have been guided by some preference but this fact by itself cannot establish that he was guided -- among other things -- by the dispreference for the 72\textsuperscript{nd} unit of butter.''



But this leaves open the question of why, then, did the grocer seize upon that precise unit of butter, if that was not the one he most wanted to get rid of, as demonstrated by his specific action of choosing that one to sell. To be sure, V6 will suffice as an accurate depiction \textit{before} the grocer's actual decision. But afterward, it is difficult to maintain V6, vis a~vis V5, given that only V5 is based on \textit{all} the facts in this case; that is, not only did he want to sell a~unit, any unit, of butter to the customer, but, also, in the event, he selected that 72\textsuperscript{nd} unit, and not any other. In contrast, V6 leaves out this fact. Again, as in V3 and V4, this author is wrongly concluding from the fact that V6 is correct that V5 is false. Both could be truthful.



Wysocki is not finished with his analysis, not by a~long shot:



\begin{quote}
Granted, when it came to the seller's action, he must have been guided by some preference but this fact by itself cannot establish that he was guided -- among other things -- by the dispreference for the 72\textsuperscript{nd} unit of butter. And, we submit, it is all the more natural to assume that the seller was guided by the preference for some money over any particular unit of butter.
\end{quote}



I~am trying to defend Austrian economics against Nozick's 
%\label{ref:RNDFRu8xur1Xy}(1977)
\parencite*[][]{nozick_austrian_1977} %
 critique of it. This eminent philosopher claims that if Austrians want to make use of supply and demand curves, diminishing marginal utility, we must, kicking and screaming if need be, acknowledge indifference as a~technical matter. I~go part way along in the direction Nozick lays out: yes, there is such a~thing as indifference, but it only applies \textit{before} human action occurs; before the grocer chooses a~pound of butter to sell. Afterwards, this can no longer be the case, for the grocer, must, of necessity, chose a~specific one pound of butter to rid himself of. He cannot, logically cannot, select a~non-specific unit of butter of which to rid himself. If that is to not disprefer it, then there simply is no such thing as preference and dispreference, which there certainly is, whenever we engage in human action. So, yes, I~accept V6; but I~also insist not only that V5, too, is correct, but that there is a~great ``need'' to maintain its truth.



According to Wysocki:



\begin{quote}
Now, it is crucial to note that it is precisely Block's contention that from the act of giving up a~particular unit we can infer a~dispreference for that very unit that leads him to the weird eponymous after-action/before-action distinction. Remember, Block believes that the seller starts with indifference among all the units of butter. However, since he believes that the actor's act of giving up a~particular unit implies a~dispreference for that unit, he most now posit that the actor is no longer indifferent among all the units of his commodity. Sadly, Block never explains why there is this sudden change in the actor's mental state.
\end{quote}



I~did exactly that in Block 
%\label{ref:RNDm6zpDVBqmm}(1980; 2009a; 2009b; 2012; 2019; 2022)
\parencites*[][]{block_robert_1980}[][]{block_rejoinder_2009}[][]{block_rejoinder_2009-1}[][]{block_response_2012}[][]{block_rejoinder_2019}[][]{block_response_2022} %
 but let me take up Wysocki's present invitation to do so once again. In my view, before the customer came into the store, the grocer was not thinking about which of his 100 packages of butter he liked most or least. If he would have asked himself at that time about his assessment of his butter stock, he would have told himself he was indifferent between them. So far no human action. Then, the customer arrives and asks for one package of butter. The grocer grabs the 72\textsuperscript{nd} one. We as Austrian economists have not one but two things to account for. First, there is the fact that he grabbed \textit{any} unit of butter. The answer is obvious: he preferred the money to this product. So far, Hoppe and Wysocki go along with me on this. But, second, we have to account for the fact that he selected \textit{this particular} element of his stock. Here, these economists steadfastly refuse to answer. They say it is not necessary to respond. They maintain that the first question is all that ``needs'' to be answered. I~cannot budge them from this position. But it seems clear to me that both questions are on the table, and that we are remiss if we refuse to answer both, decline even to contemplate each of them. Wysocki states, ``the Hoppean account does not need to resort to the before-action/after-action distinction at all to explain the seller's act.'' This account explains why the grocer prefers the money to any pound of butter, but not why this particular unit was chosen.



Continues Wysocki:



\begin{quote}
If, by assumption, the actor is indifferent among all the units of butter, then his act\footnote{Of selecting a~unit to sell.} does not (and cannot) demonstrate dispreference for the actual unit given up.
\end{quote}



My response is that the human actor \textit{is} not indifferent at the point of sale; rather, he \textit{was} indifferent beforehand, but, now, that he is called upon to select a~\textit{particular} unit of butter, it would be logically impossible for him to \textit{remain} indifferent.



According to Professor Wysocki, I~am logically hoist by my own petard unless I~embrace:



\begin{quote}
the Hoppean account, (where) it is, most naturally, the actor's preference guiding the actor's action: if the actor prefers x~to y, he chooses x~over y, whereas if he is indifferent between a~and b, he does not choose between a~and b. More concretely, if he is indifferent between particular units if butter, then he does not choose between them. If he prefers some money to any unit of butter, then he chooses to give up a~unit of butter for some money. There is no need to postulate any arbitrary change in the actor's state of mind to understand his resultant behaviour. Block, by contrast, is powerless to explain the actor's choice, for how can he choose to give up the 72\textsuperscript{nd} unit if the actor was \textit{ex hypothesi} indifferent between all of them.
\end{quote}



However, Wysocki overlooks the fact that there are two time periods, and that the grocer does not engage in human action in the first of these, hence indifference may prevail, but in the second, he most certainly does engage in human action, he selects one specific unit of butter to sell, not any other unit. Here, indifference must be banished. Thus, on this account, ``the robust notion of the supply of the same commodity'' may remain adhered to, and so may I~be allowed to ``capture… the law of diminishing marginal utility'' where, again, we are dealing with homogeneous elements of a~given stock. I~shall have no more to say about Wysocki's intriguing apple example except for the fact that he goes astray there, again, in the same manner.



\section{Agency and strict preference}

Here is Wysocki's next critique:



\begin{quote}
…it certainly does not follow that once the agent enters the café with her left foot, she thereby demonstrates her preference for entering with this particular foot to entering with the other one. For, the content of the agent's intentional state (i.e. of what the agent intends to do) might be simply to enter the café with the ways of entering it being left unspecified.
\end{quote}



This is a~brilliant attempt to undermine my thesis. I~full well recognize its power, and I~salute Wysocki for coming up with it. It is a~really good try, but no cigar shall be awarded. Here is my response: human action is purposeful behavior. If you were to ask the person which foot he (sic!) entered into the restaurant with, he would undoubtedly be unable to answer correctly. He would have been totally unaware of this choice. He could guess, but would have no more than a~50\% chance of coming up with the correct answer. In sharp contrast, if you asked the grocer which package of butter he was offering his customer, it would appear reasonable for me to ascribe to him the statement: ``Why that one, over there, the one in the customer's basket.'' So I~reject this attempted refutation of Wysocki's while acknowledging its creativity.



Here is another response to this brilliant riposte of Wysocki's. I~am logically obligated to give only one instance of reconciling supply, demand and ordinal utility with indifference. I~do not have to explain all instances of human action on the basis of this analysis of mine. Nozick's criticism of the Austrians is that unless there is any indifference, we cannot maintain supply and demand of homogeneous objects, nor marginal utility. So, all we Austrians have to demonstrate is that in at least one case, there is no indifference, and, yet, homogeneity. This I~have done with the case of butter, with my before and after scenario. Wysocki is trying to paint me into a~corner, and in effect demanding that this analysis need apply to all sorts of other examples: when you walk into a~restaurant, which foot steps into its premises first; the exact verbiage with which you order coffee; the pitch of your voice when you order coffee. I~need do no such thing. Indeed, I~am unable to do any such thing. Further, the examples he chooses are not cases of purposeful human action. Wysocki calls the coffee drinker ``an economic agent.'' He certainly qualifies for that rubric regarding the coffee, but not at all concerning which foot is forward when entering the restaurant.



Here is yet another of our author's criticisms:



\begin{quote}
…from the fact that the mother saves Peter Block draws an inference to the conclusion that she ‘places a~higher value on Peter than Paul.' But then again, just as -- as we already saw -- one cannot infer the preference for entering a~café with a~right foot from the fact the agent does actually enter with that very foot, so we cannot infer the mother's preference for Peter over Paul from the very fact that Peter was saved.
\end{quote}



Let us posit that the poor mother grabs Peter to save him. From this, Wysocki and Hoppe posit that she preferred to save one son, rather than none. But they offer no evidence for this claim. It is entirely possible that she really didn't care which son to save, wanted to save both of course but for some reason could only save one. But, in the event, she grabbed onto Peter and pulled him to safety. I~find it difficult to go along with Hoppe and Wysocki and say she was indifferent to which son she saved, in the face of the fact that she grabbed ahold of Peter, and did not let him go even though she could have done so at no risk, let us assume, and saved Paul instead. But she did no such thing.



Let me try again. Person A~does act X. I~say that this demonstrates that person A~preferred to do act X~to any other act he could have undertaken. Wysocki and Hoppe maintain, instead, without a~scintilla of evidence for this claim, that person A~was indifferent between X~and act Y, and preferred either X~or Y~to doing nothing. They are making this up out of the whole cloth. It simply does not logically follow from the fact that person A~does act X~that he does not prefer X~to anything else. It is simply fallacious to deduce from the fact that person A~does act X~that he really preferred X~or Y~to all alternatives. Yet, this is precisely the logic of their argument in all of these cases: the butter, the mother and sons, left foot right foot, Buridan's Ass, etc.



States Wysocki:



\begin{quote}
…the fact that the mother saves Peter (extensional description) underdetermines the value scale guiding the mother's action, for the mother might equally well frame her end as saving a~child rather than saving Peter. And if the former is true, then saving Peter serves this end equally well as saving Paul. That is why, she can remain (before and after action) indifferent between the two of her sons…
\end{quote}



But Wysocki cannot justify his claim that the mother is indifferent between saving either of her son's lives, or the other. He certainly cannot deduce this from the fact that she saved Peter. There is no specific human action, moreover, that could unequivocally \textit{demonstrate} that she was indifferent between saving the lives of her two sons. There is no act could she perform that would unambiguously reveal this. I~have asked my two friends to demonstrate this many times in our previous debates, and have never seen any answer to it, let alone a~satisfactory one.



In Wysocki's view:



\begin{quote}
However, Block 
%\label{ref:RNDpRmGHk185r}(2022, pp.50–51)
\parencite*[][pp.50–51]{block_response_2022} %
 protests: Wysocki quotes me [%
%\label{ref:RNDaYIxuQrx3o}(Block, 2022, pp.50–51)
\parencite[][pp.50–51]{block_response_2022}%
] as stating: ``She did rescue the former, when she could have chosen differently, and selected the latter for retrieval, did she not?'' Our author's response: ``But this simply begs the question. We, following Hoppe, contend that the mother's action in and of itself is not determinative of the mother's value scale, for the mother might as well simply prefer rescuing a~child to saving none.
\end{quote}


``Might'' have had this preference will not suffice. If I~say you might have eaten an apple when you are not now eating an apple, it is incumbent upon me to at least be able to draw a~picture of you eating an apple. If I~say you might have been sitting, similarly, I~should be able to draw a~picture of you doing just that.\footnote{I~don't like to brag, but my stick figure artistry is capable of so doing. Move over, Picasso.} If I~say that the fact that you purchased a~shirt for \$30 indicates that ex ante you valued something about that purchase more than the money you spent on it, I~cannot draw a~picture of that, but I~can appeal to people's understanding of the English language to not only know what I~am talking about, but to enthusiastically acquiesce in agreement with my contention.



Wysocki and Hoppe do not fare very well in this test. They certainly cannot draw any picture of a~human action which clearly and unmistakably depicts indifference. Nor can they even verbally describe what such a~situation could be. All they can do is assert that it has occurred in the cases under discussion. Talk about begging the question.



Let us now consider a~very powerful criticism that Wysocki launches against my analysis of indifference:



\begin{quote}
Eventually, to add insult to the injury, Block 
%\label{ref:RNDsQdYGokNYL}(2022, p.51)
\parencite*[][p.51]{block_response_2022} %
 adds that even if the mother ‘did this with her eyes closed, and just grabbed the nearest son' this would still indicate that the mother chose to save Peter. Yet, how can grabbing a~certain son with one's eyes closed count as demonstration of preference for that son? If anything, it seems that under that scenario the mother prefers grabbing any one son over saving none. It appears as though the most charitable take on the Blockean idea of choice is that the author -- his protestations to the contrary notwithstanding -- embraces methodological behaviourism.
\end{quote}



Let us make a~few stipulations about this ``eyes closed'' scenario. There are only two people who are drowning, or otherwise in danger: her sons, Peter and Paul. Without knowing whom she is grabbing, she latches onto Peter. She knows, moreover, that it is ``only one to a~customer'': she cannot possibly save both children, by stipulation. Now, to be sure, Wysocki is correct in asserting that she prefers to save either of her sons, rather than none.\footnote{This is an ordinary language statement, not one of technical economics. As far as the latter is concerned, we are not entitled to deduce any such thing from the fact that she saved Peter.} But the fact of the matter is that she is now clutching Peter's hand, let us say, not Paul's. From this I~deduce that she prefers to save Peter, rather than Paul. It cannot be denied that, at this point, she does not know who she is in the midst of saving. But, still, she does not let go of this son's hand, and grab onto the hand of the other son. Let us stipulate, also, that she could do that if she wished without any danger of losing both sons.\footnote{Hey, I~need all the help I~can get here. Wysocki is on my trail, and he is a~worthy opponent.} I~thus conclude that based on this behavior of hers, she prefers to save the son whose hand she is now gripping. True, she will not know his name until and unless she opens her eyes, but, still, it is a~true statement to say, contrary to Wysocki, that she prefers to save Peter, vis a~vis Paul. Just because she is unaware of the identity of the person whose hand she is now grasping cannot gainsay this primordial fact.



Wysocki's next criticism is that my view



\begin{quote}
…comes perilously close to methodological behaviourism, for Block seems to dismiss the mother's mental states completely. Note, even if the mother were to think ‘about ice cream', she would still choose to save Peter in the event Peter would be ultimately saved. But this at a~stroke gives up characteristically Austrian methodological subjectivism and denies any role to the actor's mental states (preferences and beliefs) as determining choices.
\end{quote}



Wysocki and I, both strong advocates of subjectivism, and adamant opponents of behaviorism, have sharply different views as to what this concept signifies. In my view, we as praxeologists simply have no insight as to what this eyes-closed mother was thinking about when she grabbed Peter's hand. Presumably, she was thinking along the lines of ``I love both my sons, I~wish I~could save them both but I~can't, so I'll at least save this one here, whoever he is.''\footnote{This is a~guess on my part. It does not logically follow, inexorably, from her actions. This is not praxeological truth, as is the case of inferring when one purchases a~shirt for \$30, that he at time valued something about that article of apparel more than that amount of money.} Yet, for all we know, qua praxeologists, she could have been thinking about anything else under the sun, yes, certainly including ice cream. We are not psychics; we are not ESPers; we are not psychologists; we are not mind readers; we are not magicians. We are none of these things. We have to be modest about our abilities. We are merely praxeologists. Our scope is limited to deducing from \textit{behavior} not from thoughts that are inevitably and necessarily hidden from us. All we know, all we \textit{can} know from her behavior, is that she unknowingly grabbed Peter's hand. We have no option, while still remaining true to praxeology, to deduce anything other than that she preferred to save Peter to Paul.\footnote{In my view, Wysocki is guilty of the fallacy of 'psychologizing,' the treatment of preference scales as if they existed as separate entities apart from real action. Psychologizing is a~common error in utility analysis. It is based on the assumption that utility analysis is a~kind of ‘psychology,' and that, therefore, economics must enter into psychological analysis in laying the foundations of its theoretical structure. ``Praxeology, the basis of economic theory, differs from psychology, however. Psychology analyzes the how and the why of people forming values. It treats the concrete content of ends and values. Economics, on the other hand, rests simply on the assumption of the existence of ends, and then deduces its valid theory from such a~general assumption. It therefore has nothing to do with the content of ends \textit{or with the internal operations of the} \textit{mind of the acting man}. 
%\label{ref:RNDne7up6s1aL}(Rothbard, 2011; emphasis added by present author)
\parencites[][]{rothbard_toward_2011}[emphasis added by present author,][]{}%
}



Behaviorism will not pass muster. Here is what Rothbard 
%\label{ref:RNDF4cblaYW8A}(2011)
\parencite*[][]{rothbard_toward_2011} %
 had to say about that concept:



\begin{quote}
The behaviorist wishes to expunge ‘subjectivism, that is, motivated action, completely from economics, since he believes that any trace of subjectivism is unscientific. His ideal is the method of physics in treating observed movements of unmotivated, inorganic matter. In adopting this method, he throws away the subjective knowledge of action upon which economic science is founded; indeed, he is making any scientific investigation of human beings impossible.
\end{quote}



I~can see why Wysocki launches this charge against me. I~do indeed base my interpretation on the behavior of the choosing individual, or the human actor, be he a~grocer, a~person seeking coffee in a~restaurant or a~mother trying to save life. But I~plead innocence. I~am not, as is Prof. Little, rejecting ``demonstrated preference theory.'' Rather, I~am adhering to it, through thick and thin, despite the criticisms of Wysocki and Hoppe to the effect, in my interpretation of them, that I~am sticking too closely to it. I~am exulting in it. I~am insisting that there is no way that indifference can be logically deduced from any human action, any behavior. I~am insisting that demonstrated preference, the foundation here of Rothbardianism, is human behavior, and that this is not the behaviorism against which Rothbard warns.



\section{Conclusion}

I~am very grateful to Wysocki for this critical essay of his. He has forced me, in my response, to dig far deeper into these issues than ever I~would have otherwise contemplated; than ever I~would have been able to do on my own, without his splendid challenges. We all learn from each other, and I~am greatly in the debt of this author for in effect compelling me to learn from him. I~hope and trust this is at least partially reciprocal.
%\enlargethispage{1.5\baselineskip}



\end{artengenv}\label{block-rejoinder-lastpage}

