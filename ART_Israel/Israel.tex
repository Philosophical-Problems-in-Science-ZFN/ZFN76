\begin{artengenv2auth}{Tate Fegley, Karl-Firedrich Israel}
	{A defense of Austrian welfare economics}
		{A defense of Austrian welfare economics}
		{A defense of Austrian welfare economics}
	{University of Sydney}
	{Murray N. Rothbard's \textit{Toward a~Reconstruction of Utility and Welfare Economics} is the defining contribution outlining the Austrian school's approach to welfare theory. A~recent attack on this approach is by Wysocki and Dominiak 
	%\label{ref:RNDFpq36Uzdfc}(2023),
	\parencite*[][]{wysocki_how_2023}, %
	 who argue, contra Rothbard, that whether an exchange is welfare-enhancing is not necessarily related to whether that exchange is just, and therefore the Rothbardian framework is wrong. This paper shows that their argument misconceives how Austrians treat the concept of welfare. They also misunderstand the crucial role of the principle of demonstrated preference. Properly conceived, Rothbard's reconstruction remains intact.
		}
		{welfare economics, Austrian economics, Murray N. Rothbard.}
	{%
		{\flushright\subbold{Tate Fegley}\\\subsubsectit\small{Montreat College}\label{israel-first}\par}%
		{\flushright\subbold{Karl-Firedrich Israel}\\\subsubsectit\small{Catholic University of the West}\par}%
	}




\section{Introduction}

\lettrine[loversize=0.13,lines=2,lraise=-0.03,nindent=0em,findent=0.2pt]%
{A}{seminal} %A~seminal
contribution in Austrian welfare economics is Rothbard's 1956 essay \textit{Toward a~Reconstruction of Utility and Welfare Economics} 
%\label{ref:RNDClORnqgdmS}(Rothbard, 2011).
\parencite[][]{rothbard_toward_2011}. %
 As the title suggests, Rothbard aimed to reconstruct welfare economics on solid scientific grounding, avoiding the pitfalls of previous attempts. Although it was a~scientific breakthrough, his argument was not without controversy, but has produced decades of criticism and replies 
%\label{ref:RND9hy2ArGPps}(Block, 1999; Caplan, 1999; Cordato, 1992; Gordon, 1993; Herbener, 1997; 2008; Hülsmann, 1999; Kvasnička, 2008; Prychitko, 1993).
\parencites[][]{block_austrian_1999}[][]{caplan_austrian_1999}[][]{cordato_welfare_1992}[][]{gordon_toward_1993}[][]{herbener_pareto_1997}[][]{herbener_defense_2008}[][]{hulsmann_economic_1999}[][]{kvasnicka_rothbards_2008}[][]{prychitko_formalism_1993}.%




A~recent argument against Rothbard's reconstruction is provided by  Wysocki and Dominiak
%\label{ref:RNDvZAcMn6xM2} (2023)}.
\parencite*[][]{wysocki_how_2023}.
The authors argue that the welfare theorems he derives---that free market exchanges always increase social utility and that government intervention can never increase social utility---are false. Our goal in this paper is to defend Rothbard from Wysocki and Dominiak 
%\label{ref:RND0tl7w99Nev}(2023)
\parencite*[][]{wysocki_how_2023} %
 by demonstrating that their criticisms are misplaced and that Rothbard's contribution stands unscathed.



As many critics before them, Wysocki and Dominiak implicitly rely on the assumption of welfare or utility being a~magnitude that can be assessed independently from demonstrated preferences under specific circumstances. This, however, is not the case. We can of course construct all kinds of imaginary scenarios, where all the relevant knowledge about the underlying preferences is assumed into existence, but this does not help us in applications to the real world, where that kind of knowledge remains hidden from us, unless it is demonstrated in voluntary and just interaction.



The paper is structured as follows: the next section summarizes Rothbard's reconstruction of welfare economics. The third section summarizes and replies to the criticism of Wysocki and Dominiak 
%\label{ref:RNDKncprYNjab}(2023),
\parencite*[][]{wysocki_how_2023}, %
 and the fourth section provides a~conclusion and some further reflections on the importance of welfare economics and its relation to moral philosophy.



\section{Rothbard's reconstruction of welfare economics}

Rothbard's reconstruction of welfare economics is firmly based on the theory of subjective value espoused by the Austrian School. As Rothbard explains, welfare theory is utility theory applied to the context of society with the goal of drawing scientific conclusions about the desirability of alternative arrangements:



\begin{quote}
Utility theory analyzes the laws of the values and choices of an individual; welfare theory discusses the relationship between the values of many individuals, and the consequent possibilities of a~scientific conclusion on the ``social'' desirability of various alternatives. 
%\label{ref:RND6HNsiOxKXI}(Rothbard, 2011, p.289)
\parencite[][p.289]{rothbard_toward_2011}%
\end{quote}




To achieve this goal Rothbard invokes two principles: 1) the unanimity rule, and 2) the principle of demonstrated preference. The unanimity rule is better known as the Pareto criterion, which states that social welfare has improved if at least one person is made better off, and nobody is made worse off. Rothbard argues that this rule provides the only way in which we can scientifically speak of an improvement in social welfare. Since value and utility are subjective and we lack an objective unit of measurement, there is no way of comparing the loss in utility for some person with the gain in utility of another person. There is \textit{a~fortiori} no way of determining whether a~loss in utility for some person is outweighed by the gain in utility for another person. But subjectivity is by no means the only problem here. Even if there was an objective unit of measurement, it would still be questionable whether a~benefit for some person can ever outweigh the harm of another.\footnote{Utility can be understood as multi-dimensional, especially if we think of the utility of a~group of people. Social utility, in particular, is not one-dimensional, that is, harm and benefit are not necessarily received by the same people and do thus not occur along the same dimension. They cannot necessarily be lumped together even if they could be quantitatively compared. }



The crucial question is how can we know if somebody gains or loses utility? Here the principle of demonstrated preference comes in. Rothbard argues that we can only know about what a~person prefers, that is, what makes that person better off, from observing their choices and actions. If a~person chooses option A~over an alternative option B~that is also available, we can infer that the person attaches a~higher subjective value to option A~than to option B~and is made better off by choosing option A~(in the \textit{ex ante} sense). The person has demonstrated their preference in action.



We can hypothetically imagine all kinds of preferences of one or more persons and reason through how they would interact in various situations and what outcome would be socially optimal. But the crucial point that Rothbard makes is that we can only know about actual preferences to the extent that they are demonstrated in real action at a~specific point in time under specific circumstances. As Rothbard 
%\label{ref:RND4PCtrODkte}(2011, p.320)
\parencite*[][p.320]{rothbard_toward_2011} %
 describes it:



\begin{quote}
Demonstrated preference […] eliminates hypothetical imaginings about individual value scales. Welfare economics has until now always considered values as hypothetical valuations of hypothetical ``social states.'' But demonstrated preference only treats values as revealed through chosen action.
\end{quote}



Importantly, Rothbard emphasizes that there is no reason to believe that preferences are constant over time. For all we know they can and do change. Preferences as revealed at one point in time by an individual are not necessarily relevant for another point in time.



The assumption of constant preferences is indeed an important feature of Paul Samuelson's theory of revealed preference 
%\label{ref:RNDrzqg9QWabl}(Samuelson, 1938).
\parencite[][]{samuelson_empirical_1938}. %
 Rothbard explicitly distinguishes his own theory from Samuelson's by choosing the term ``demonstrated preference'', admitting that ``revealed preference'' would have been a~very fitting term too. According to Rothbard's principle of demonstrated preference, our limited knowledge of preferences as demonstrated under the specific circumstances of a~given historic situation cannot be extrapolated to other situations. There is no scientific basis for assuming preferences to remain what they have been before. We can know about them only for that specific situation in which they are demonstrated in action, and even then our knowledge about them is never complete.



To make sense of a~given historic situation, interpretive understanding is required and the observer can of course err. If Murray, for example, offers Paul the choice between an apple and a~pear, and Paul picks the pear, we know that Paul did what he preferred to do. But we do not know whether he expected to like the taste of the pear more than the taste of the apple. Maybe Paul just wanted Murray to falsely believe that he likes pears more than apples, although he really prefers apples in general. We only know for certain that Paul attached a~higher expected marginal utility to the option he chose than to the alternatives forgone.



It is important to understand that the principle of demonstrated preference does not allow the economist to make any inference on whether the level of utility of a~person---from a~point in time before the action takes place to a~point in time thereafter---has increased or not. Take the above example. Maybe Paul's utility increased from taking the pear compared to what it was before Murray made his offer. But maybe Murray made his offer to Paul in a~way that made him feel uncomfortable. The penetrating look in his eyes and the sarcastic smile made him tremble with fear, so that Paul really had a~higher level of utility before Murray showed up and made the offer. All of that is possible. So economists can infer nothing about the absolute changes in the level of utility between different points in time---neither for one person nor for a~group of people or society as a~whole. We are not the first ones to make this clarification in response to a~criticism of Rothbard's reconstruction. The same point is explained very well by Herbener 
%\label{ref:RNDM4wCd4Jzed}(Herbener, 2008, p.63)
\parencite[][p.63]{herbener_defense_2008} %
 in his reply to \href{https://www.zotero.org/google-docs/?MN1lOj}{Kvasnička }\label{ref:RNDkFYd3iLdgE}\href{https://www.zotero.org/google-docs/?MN1lOj}{(2008)}. It is worth quoting him at some length:



\begin{quote}
Deducing the effects on social utility from voluntary and involuntary exchanges requires considering each action in turn given the conditions as they are at that point. Nothing can be deduced about the level of utility a~person has at the beginning of a~series of actions compared to the level of utility he has at the end of the series of actions. For example, a~person having dinner with his friends orders steak from the menu. The economist observing him can objectively conclude that, given his options, he selected what he preferred. He is enjoying the conversation when it turns to a~subject he dislikes, but he stays and endures it. The economist observing him, lacking access to what he is experiencing in his mind, can objectively conclude that he prefers to continue dining with his friends. At some point, one of his companions makes a~remark so objectionable to him that he says, ``Anymore such talk and I~shall leave.'' The economist observing him can objectively conclude that he preferred to make this remark. The economist cannot objectively conclude that this line of conversation has lowered the level of his utility. To conclude that would require the economist to make a~judgment about his utility. The economist would have to interpret the meaning of the remark as it relates to his utility. The economist would have to decide whether it was a~serious remark or a~joke and if it was serious did making the remark push his utility up or down. Bullies, after all, like to intimidate others with such remarks. No such judgments are necessary for the economist to conclude that he preferred making the remark. It follows from the objective evidence of his action and the conceptual meaning of action. And so it goes for the rest of the evening with the level of his utility sometimes rising and sometimes falling, but he continues dining with his friends and leaves only after the party breaks up. Is he enjoying a~higher level of utility after the evening is over compared to before it began? Who can objectively say but the person himself? He is the only person with experiential knowledge of his own utility. What another person can objectively deduce is that he preferred what he did each step of the way. 
%\label{ref:RND3vYHZXOYPE}(Herbener, 2008, p.63)
\parencite[][p.63]{herbener_defense_2008}%
\end{quote}
Hence, to say that somebody is made better off as the result of a~voluntary choice involves a~counterfactual comparison between the option chosen (the factual) and the alternative foregone (the counterfactual) at the very same point in time.\footnote{On the counterfactual nature of economic theory in general, see 
%\label{ref:RNDpxxCZtI0tY}(Hülsmann, 2003).
\parencite[][]{hulsmann_facts_2003}. %
 For an interesting critique of Hülsmann, see Machaj 
%\label{ref:RNDKWuHXQB4dq}(2012).
\parencite*[][]{machaj_counterfactuals_2012}. %
 } It does not involve a~comparison between the absolute level of utility before and after the choice. We only know that the expected marginal utility of the option chosen is higher than that of the alternative options not chosen. The actor gains utility relative to the alternative options forgone.



Another important contribution of Rothbard's reconstruction of welfare economics is the clarification of the notion of marginal utility. He explains that the term does not refer to some marginal increment in utility, but rather to the utility of the marginal unit of some good, which is subjective and ordinal. Otherwise, the notion of marginal utility would indeed suggest that utility is something that can be measured and computed mathematically, and that marginal utilities can be added to and subtracted from one another, and that total utility is nothing other than a~sum of marginal utilities. But that is not so. Rothbard 
%\label{ref:RNDcOzUTPrIOp}(2011, p.301)
\parencite*[][p.301]{rothbard_toward_2011} %
 argues that ``there is no such thing as total utility; all utilities are marginal''. And most importantly we can only draw scientific conclusions about welfare and utility on the margin based on demonstrated preferences. People are of course passively affected by all kinds of changes in the environment, including the actions of others. These changes cannot, however, be dealt with scientifically in the realm of welfare economics, because we lack the means of assessing their welfare implications.



All of this imposes radical constraints on what welfare economics can accomplish. But Rothbard argues that despite the fundamental subjectivity of utility, we can at least draw some scientific conclusions. We cannot calculate total utility, but following the unanimity rule, we can in some situations, conclude that overall or social utility has improved, that is, when demonstrated preferences are satisfied. For example, ``welfare economics can make the statement that the free market increases social utility, while still keeping to the framework of the Unanimity Rule'' 
%\label{ref:RNDqptw5GYZaS}(Rothbard, 2011, p.320).
\parencite[][p.320]{rothbard_toward_2011}. %
 The important word here is ``increase'' instead of ``maximize''.\footnote{Rothbard 
%\label{ref:RNDmKyN7PowXv}(2011, p.323)
\parencite*[][p.323]{rothbard_conceived_2011} %
 uses the word ``maximize'' in quotation marks and he makes the following clarification: ``[…] we may conclude that the maintenance of a~free and voluntary market ``maximizes'' social utility (provided we do not interpret ``maximize'' in a~cardinal sense).'' That is, since the free market is the absence of government intervention, it implies that no voluntary and mutually beneficial exchanges are prevented, thus social utility is ``maximized''.} There is nothing to be maximized, but there are mutually beneficial exchange opportunities which are discovered and exploited within the framework of the free market, leading to improvements in social utility as individuals voluntarily interact without rights violations.



When it comes to government intervention or any rights-violating action by individuals, we can draw no such conclusion. As Rothbard 
%\label{ref:RNDMvZw83RBoQ}(2011, p.322)
\parencite*[][p.322]{rothbard_toward_2011} %
 explains:



\begin{quote}
Suppose that the government prohibits A~and B~from making an exchange they are willing to make. It is clear that the utilities of both A~and B~have been lowered, for they are prevented by threat of violence from making an exchange that they otherwise would have made. On the other hand, there has been a~gain in utility (or at least an anticipated gain) for the government officials imposing this restriction, otherwise they would not have done so. As economists, we can therefore say nothing about social utility in this case, since some individuals have demonstrably gained and some demonstrably lost in utility from the governmental action.
\end{quote}



An analogous explanation can be given in cases where governments do not prevent but enforce a~transaction. In such cases, too, there is a~violation of the unanimity rule and no conclusion can be drawn about whether social utility has improved or not.\footnote{See, in this special issue, Wysocki and Dominiak 
%\label{ref:RNDoZwDIp2f29}(2024)
\parencite*[][]{wysocki_social_2024} %
 on clarifying the dispute over what precisely Rothbard meant by saying ``we can therefore say nothing about social utility in this case…'' In this regard, Rothbard was making a~statement about the epistemological limitations of scientific economics, though elsewhere he allowed for the possibility of knowledge under other disciplines. Regarding the possibility of third parties to a~voluntary exchange being envious, he writes, ``[W]e may know as historians, from interpretive understanding of the hearts and minds of envious neighbors, that they do lose in utility. But we are trying to determine in this paper precisely what scientific economists can say about social utility or can advocate for public policy, and since they must confine themselves to demonstrated preference, they must affirm that social utility has increased'' 
%\label{ref:RNDHfGWvPK7Dv}(Rothbard, 1997, p.89).
\parencite[][p.89]{rothbard_praxeology_1997}.%
} There is no scientific basis for supporting such a~claim if one sticks firmly to the unanimity rule and the subjectivity of utility and value.



Rothbard 
%\label{ref:RNDYs15ehFx8Q}(2024, p.323)
\parencite*[][p.323]{rothbard_toward_2011} %
 then draws two main conclusions that have aroused much criticism among his readers:



\begin{quote}
Economics, therefore, without engaging in any ethical judgment whatever, and following the scientific principles of the Unanimity Rule and Demonstrated Preference, concludes: (1) that the free market always increases social utility; and (2) that no act of government can ever increase social utility. These two propositions are the pillars of the reconstructed welfare economics.
\end{quote}
Some aspects underpinning these claims are not spelled out in detail in Rothbard's reconstruction. But these elements can be provided from the rest of Rothbard's works and the works of his intellectual followers to make his two conclusions whole and defend his analysis from many criticisms.



\section{A~defense against recent critics}

In their recent criticism of Rothbard's reconstruction, Wysocki and Dominiak 
%\label{ref:RNDvqtsRs3Qv8}(2023)
\parencite*[][]{wysocki_how_2023} %
 claim to demonstrate that his two pillars---that the free market always increases social utility and that no act of government can ever increase social utility---are false, and that whether a~particular exchange is welfare-enhancing or welfare-diminishing is a~separate question from whether the exchange is just or unjust.



To show this, Wysocki and Dominiak 
%\label{ref:RNDlGtAmWKEZA}(2023)
\parencite*[][]{wysocki_how_2023} %
 provide counter-examples of exchanges that are alleged exceptions to Rothbard's pillars---one being an example of a~just, that is, property rights respecting, exchange that is not welfare enhancing and the other an example of an unjust, that is, property rights violating, exchange that is welfare enhancing.



\subsection{Just but ``welfare-decreasing'' exchanges}



It is worth noting from the outset that ``welfare-increasing'' and ``welfare-decreasing'' are meant in the ex ante sense of the word. There are of course just exchanges that people regret. They are welfare-decreasing in the \textit{ex post} sense. Nobody would deny their existence. The point of contention is whether there are just and welfare-decreasing exchanges in the \textit{ex ante} sense. Wysocki and Dominiak 
%\label{ref:RNDOFxRjx42jj}(2023)
\parencite*[][]{wysocki_how_2023} %
 think there are.



The supposed exception to the idea that free and voluntary exchange always leads to improved welfare from the \textit{ex ante} perspective of both trading partners is a~blackmail offer. Wysocki and Dominiak 
%\label{ref:RNDuHmJ0QqpIc}(2023, p.22)
\parencite*[][p.22]{wysocki_how_2023} %
 have the reader



\begin{quote}
[S]uppose that a~blackmailer makes the following proposal to the blackmailee:\\
(1) If you pay me \$1.000.000 (demand), I~will let your reputation remain untarnished (relative benefit).\\
(2) If you don't pay me (refusal), I~will gossip about your secrets (threat).
\end{quote}
They argue that the blackmailee, if he accepts the blackmailer's proposal and pays him, demonstrates his preference to have an untarnished reputation and paying \$1 million over the alternative of having a~tarnished reputation but keeping \$1 million, and therefore benefits relative to not paying. However, since he would be better off if the blackmailer had had nothing to do with him at all (since he would then have both his \$1 million and an untarnished reputation), he is not better off in an absolute sense.



But this conception of being better or worse off in an absolute sense is irrelevant to Rothbard's welfare theory as we outlined above, quoting from Herbener's 
%\label{ref:RNDGVme3JlOr4}(2008, p.63)
\parencite*[][p.63]{herbener_defense_2008} %
 excellent exposition. Welfare economics can say nothing about the absolute level of utility. Wysocki and Dominiak 
%\label{ref:RNDlGCmB8Jh2v}(2023, pp.61–62, fn. 12)
\parencite*[][fn. 12]{wysocki_how_2023} %
 appear to fully appreciate this point in a~rather extensive footnote of their article. Given this, it is strange that they pursue this line of argument based on a~different conception of welfare, as if it could provide exceptions to Rothbard's propositions. Rather, the question that is relevant to Rothbard is whether property rights are respected and a~voluntary exchange is made: if so, social welfare increased. Imaginary counterfactuals involving the non-existence or existence of other individuals are irrelevant. Imaginary counterfactuals are very different from the relevant counterfactuals of alternative choices in a~given situation. Only the latter matter. The former do not.



Imagine a~person who voluntarily buys an apple for \$1, but the person would have much rather bought a~banana for \$1. There was no one willing to sell a~banana for \$1. Is it in any way relevant that the apple buyer is made better off, because she prefers an apple over \$1, but would have been still better off if she could have bought a~banana instead? No, given the constraints of the situation in terms of money, time, knowledge, and the rights-respecting actions of others, social welfare has increased because of the exchange made. This is true for the blackmail transaction as for any other free-market transaction.



There is another perspective on the blackmail case. When we consider all of the parties involved in the blackmail transaction, we can more easily see that social welfare increases from the voluntary exchange. That is, unaddressed by Wysocki and Dominiak are the potential beneficiaries of the gossip.\footnote{With blackmail, there is necessarily a~third party. If Friday learns embarrassing information about Robinson Crusoe but they are alone on an island, Friday will not be able to blackmail Crusoe.} What is being traded by the blackmailer is a~property right to decide whether embarrassing information is published or kept secret. The end of the blackmailee to have his reputation untarnished conflicts with the ends of buyers of gossip magazines to read about his secrets. If the blackmailer allows both the blackmailee and publishers of gossip magazines to bid over this property right (that is, the free market is allowed to operate), resources will be allocated to their most highly valued uses and all Pareto-improving transactions that people perceive will be made. In this example, government intervention cannot be demonstrated to lead to a~more preferable allocation of property rights.



We have seen that the distinction between absolute and relative improvements in welfare for one of the two exchange partners is irrelevant. What is relevant from the vantage point of Austrian welfare economics is whether the benchmark for comparison involves a~rights violation or not. The blackmailer threatens to gossip about the blackmailee's secrets, but gossiping is not a~rights violation. He has the right to gossip, although some people might not like it. So the blackmailee who pays and prevents his secrets from being published is made better off relative to a~scenario that involves no rights violation and in which his secrets are made public. Contrast this with a~highwayman who threatens to kill his victim unless she pays money. In that scenario, as Wysocki and Dominiak 
%\label{ref:RNDb1tp3NOZKs}(2023, pp.54–55)
\parencite*[][pp.54–55]{wysocki_how_2023} %
 emphasize, the victim who pays and lives is made better off relative to the alternative of being killed. But that alternative involves a~rights violation and is unjust. The victim is forced into an unjust exchange to protect herself against a~violation of her rights. She has to pay for something that is already rightfully hers---her life. In other words, she has to pay and receives nothing in exchange that is not already hers. And in this sense she is made worse off.



There is indeed a~philosophical discussion to be had as to what constitutes mere gossip and what crosses the demarcation line to libel and should be considered a~rights violation. More generally, a~theory of justice, or in Rothbard's view, a~theory of property rights 
%\label{ref:RNDcpsfNizy6i}(Rothbard, 1998),
\parencite[][]{rothbard_ethics_1998}, %
 is the very foundation that sets the rules according to which people are allowed to demonstrate their preferences and according to which people's choices and actions are allowed to change the environment in which others act. Choices and actions of people do change the conditions under which we act all the time, but as long as their choices and actions do not violate our rights, they are, like the weather, elements of the uncertain environment in which we act according to our own preferences. They sometimes increase and sometimes decrease our level of utility, but we cannot deal with these changes scientifically.



The theory of justice and property rights is independent of welfare economics in the sense that it is its logical prerequisite. It sets the stage for us to engage in welfare economics scientifically. When Rothbard wrote in 1956 that he can draw his welfare economic conclusions without any ethical judgment, he really took the ethics underpinning a~system of free-market exchange for granted. Rothbard realized that, which is why he later worked towards a~broader social philosophy integrating economics and ethics, sometimes referred to as Austro-libertarianism 
%\label{ref:RNDyCjck3pAaa}(Hoppe, 1999).
\parencite[][]{holcombe_murray_1999}.%




\subsection{A~voluntary and welfare-enhancing rights violation}



The second claim of Wysocki and Dominiak 
%\label{ref:RNDZBbOIl2xsP}(2023)
\parencite*[][]{wysocki_how_2023} %
 is that there are rights violations that are welfare-enhancing. Again, this is meant to be the case in the \textit{ex ante} sense. We can all think of scenarios in which a~\textit{prima facie} rights violation turns out to be a~good thing from the perspective of the person whose rights were violated. Take a~drug addict who is forced to have a~cold turkey by a~close relative who locks him in a~room for the time he needs to detox. The addict might later on be grateful for it, although the close relative had no right to lock him up. Wysocki and Dominiak have something else in mind.



To show that unjust exchanges are not necessarily welfare diminishing, Wysocki and Dominiak 
%\label{ref:RNDZi44rrcc2b}(2023)
\parencite*[][]{wysocki_how_2023} %
 offer the example of an individual with a~broken refrigerator in his backyard that he would like to be rid of, but the costs of selling it or hauling it off to the junkyard are deemed too high. However, one day a~thief absconds with the fridge and the owner decides not to interfere, given that his unwanted fridge is being removed for free.



Wysocki and Dominiak argue that this ``exchange'' is unjust because the owner of the fridge never relinquished his ownership rights and he never consented to the fridge being taken, either explicitly or tacitly. They also argue that the owner demonstrated his preference for the fridge being stolen over it remaining in his yard because of his choice not to interfere with the thief. As such, they conclude that he benefited from the theft. Further, the fridge owner benefitted not only in relative terms, but also in absolute terms because if there were no thief, he would still be stuck with the fridge in his backyard.



Does this example show that Rothbard's second pillar---that government intervention can never increase social welfare---is false? No. The primary issue with their argument, from the vantage point of the principle of demonstrated preference, is the limited inference we can make about the fridge owner's preferences based on his action. We can rightfully infer that the owner preferred not to interfere, but we cannot from his act of non-interference infer that he preferred the fridge to be stolen rather than remain in his yard. We could also suspect that he feared that the thief may attack him if he had tried to stop him, or that he would rather enjoy his leisure than have to get up and stop the thief (he was, after all, presumably too lazy to do so little as put a~sign that reads ``FREE'' on the fridge). Therefore, Wysocki and Dominiak do not successfully side-step the ``fallacy of psychologizing'' as they claim since a~real-world equivalent to their thought experiment would require that we are able to analyze the internal thoughts of the fridge owner in order to be able to determine the reason for non-interference, without which we cannot say that he prefers his fridge taken away over remaining in his yard. The fact that we can simply assume all of that in a~thought experiment is completely irrelevant. We emphasize again, as Rothbard 
%\label{ref:RNDYwcNyo7ffn}(2011, p.320)
\parencite*[][p.320]{rothbard_toward_2011} %
 put it, that the principle of demonstrated preference ``eliminates hypothetical imaginings about individual value scales.''



Wysocki and Dominiak 
%\label{ref:RNDEgZGC1fA0G}(2023, pp.63–64)
\parencite*[][pp.63–64]{wysocki_how_2023} %
 further criticize Rothbard's position for assuming that only rights-respecting exchanges can be voluntary. They challenge Rothard's rights-based understanding of voluntariness. They argue that the thief of the fridge is violating the property rights of the fridge owner, but that the fridge owner is agreeing to that rights violation voluntarily. For them, the scenario gives an example of a~voluntary rights-violating exchange and hence of a~welfare-enhancing rights violation. But this is an unsubstantial play with words.\footnote{These quibblings are equally sterile as the debates on the concept of voluntary slavery 
%\label{ref:RNDmVh6cmNpPk}(Block, 2003; Casey, 2011; Dominiak, 2017).
\parencites[][]{block_toward_2003}[][]{casey_can_2011}[][]{dominiak_problem_2017}. %
 Of course we can define our terms in such a~way that ``voluntary slavery'' can exist, but we can do the same for ``married bachelors'' or ``huge midgets.'' It does not help. For more on the concept of voluntariness and rights under Austro-libertarianism according to Dominiak and Wysocki see 
%\label{ref:RNDwGgyIbRXsO}(Dominiak, 2018; 2022; 2023; Dominiak and Fegley, 2022; Megger and Wysocki, 2023; Wysocki, 2020; 2021; Wysocki, Block and Dominiak, 2019; Wysocki and Megger, 2019; 2020).
\parencites[][]{dominiak_libertarianism_2018}[][]{dominiak_contract_2022}[][]{dominiak_proceeds_2023}[][]{dominiak_contract_2022}[][]{megger_coercion_2023}[][]{wysocki_problems_2020}[][]{wysocki_austro-libertarian_2021}[][]{wysocki_austrian_2019}[][]{wysocki_austrian_2019}[][]{wysocki_problems_2020}.%
} Nothing in the thought experiment suggests that the fridge owner's property rights are actually violated. Quite to the contrary, the fridge owner decides to execute his property rights in just the way that allows the thief to freely take the fridge. Economically speaking, the fridge in his backyard is not a~good but a~bad---not an asset, but a~liability. The thief renders a~free service to the fridge owner by removing it, albeit unknowingly.\footnote{For a~general theory of gratuitous goods, see Hülsmann 
%\label{ref:RNDDYslf4v9T4}(2023).
\parencite*[][]{hulsmann_wirtschaft_2023}.%
}



Let us give another example to show that this semantic play is unhelpful. If a~man advances to kiss a~woman, he does not know whether she likes it or not. He has no right to use the woman's lips for his pleasure. She can refuse or reciprocate. If she refuses, but the man forces her, it is an involuntary rights violation. If she instead reciprocates, it must constitute a~voluntary rights violation according to Wysocki and Dominiak 
%\label{ref:RNDk42sOhYRQj}(2023).
\parencite*[][]{wysocki_how_2023}. %
 But the kiss then is always a~rights violation. We can of course define terms in this way, but it does not facilitate or clarify the analysis. And where is the love, if every kiss is a~rights violation?



The difference between the kisser and the thief is that the thief (presumably) assumes that his action is unwelcome and the kisser (presumably) hopes that his advance is welcome. The action of the thief seems like a~rights violation from his own point of view. He does not intend to benefit the fridge owner and is willing to violate his rights, but that seems irrelevant. Sometimes we do not intend to violate anyone's rights, but do, and sometimes we do not violate anyone's rights, although we willingly take the risk of doing so. The intent is not what matters for the welfare economic analysis of the situation.



Interestingly, given that the thief rendered a~welcome service to the fridge owner, he could have charged a~price for it. If he were an honest chap and had asked the owner whether the fridge should be removed, he could have fetched a~better deal for himself. He could have been even better off than from just taking the fridge. From a~welfare economic perspective, it would have been better for the thief himself, if he had intended to respect the fridge owners property rights. He would have benefited absolutely, not just relatively, so to speak.



Wysocki and Dominiak additionally argue that Rothbard is incorrect when he argues that there are two distinct cases that can be made in favor of the free market: the moral and the economic. According to them, it really boils down to only one argument. For if it is the rights-respecting character of an exchange that guarantees mutual benefits and the free market increases welfare by virtue of it being the set of all rights-respecting exchanges that people engage in, then there are no separate moral and economic cases. But this misunderstands Rothbard's argument, for he writes in the passage that Wysocki and Dominiak themselves quote,



\begin{quote}
[i]t so happens that the free-market economy, and the specialization and division of labor it implies, is by far the most productive form of economy known to man, and has been responsible for industrialization and for the modern economy on which civilization has been built […] Even if a~society of despotism and systematic invasion of rights could be shown to be more productive than what Adam Smith called ``the system of natural liberty,'' the libertarian would support this system. 
%\label{ref:RNDcyh3b4SIqg}(Rothbard, 2006, p.48)
\parencite[][p.48]{rothbard_for_2006}%
\end{quote}

We see clearly that for Rothbard, the ``economic case'' for the free market is not synonymous with welfare ``maximization'' based on free exchange. Rather, it is about the production of wealth or material goods and services which widen the possibilities of mutually beneficial exchanges. Material wealth and welfare are distinct, and therefore there really are two separate cases being made, not just one. A~free-market economy does not only respect private property rights and is thus preferable on moral grounds, it also brings about a~greater material abundance and is thus preferable on economic grounds. The potential counterargument that some people might not like material abundance can be discarded, since every person is free to live a~life in poverty amidst an otherwise wealthy society.



\section{Conclusion and some further reflections}

Wysocki and Dominiak 
%\label{ref:RNDFzpr3sgmFD}(2023, pp.58–59)
\parencite*[][pp.58–59]{wysocki_how_2023} %
 anticipate a~counterargument to their fridge example that some readers might believe is similar to ours. They expect that critics might rely on some notion of tacit consent to claim that the thief did not actually violate the fridge owners rights. But this line of argument they say is not available to Austro-libertarians, because they ``repudiate the juridical significance of tacit or implicit consent'' 
%\label{ref:RNDeVR3M4WaAa}(Wysocki and Dominiak, 2023, p.58).
\parencite[][p.58]{wysocki_how_2023}. %
 While it is true that Austro-libertarians reject and sometimes even mock the idea of tacit consent to justify specific state interventions or the institution of the state as such 
%\label{ref:RNDY6LLoBCM6M}(Hoppe, 2006),
\parencite[][]{hoppe_economics_2006}, %
 it is not the case that one has to rely on tacit consent to recognize that the fridge owners rights were not violated. Wysocki and Dominiak give us a~thought experiment after all, and they make it perfectly clear that the owner welcomes the fridge being taken from his yard. In the thought experiment there is nothing implicit about the fridge owner's consent. Wysocki and Dominiak 
%\label{ref:RND6dCZfxt952}(2023, p.58)
\parencite*[][p.58]{wysocki_how_2023} %
 explicitly tell us that ``[o]ne day [the owner] sees, to his delight, a~thief absconding with the fridge. Having realized his fridge is thus being removed for free, he decides not to interfere.''



In a~real-world scenario we could never know. This is why rights violations should not be allowed, neither from a~moral nor welfare-economic point of view. There is no way of demonstrating a~preference for one's own rights to be violated. If you agree to getting smacked in the face, and you get smacked in the face, your rights are not violated. If on the other hand you get smacked in the face without consent, it is still possible that you enjoyed it. You just got lucky. The important point is that if you happen to enjoy such things, the free market allows you to demonstrate your preference for it, for example, by joining a~fight club or a~group of hooligans.



From the example given by Wysocki and Dominiak 
%\label{ref:RNDlgFErvIpA7}(2023)
\parencite*[][]{wysocki_how_2023} %
 it is not clear how government inflicted rights violations could be shown to increase social welfare. One could give an endless number of similar examples:
\begin{itemize}
\item A~student assistant sneaking into the professor's office to correct all of the 250 macroeconomics exams of last semester
\item A~girlfriend taking money out of her boyfriend's wallet to buy groceries to cook his favorite dish
\item A~stranger going into an apartment to clean it up, leaving all of the owner's belongings in their rightful place
\item …
\end{itemize}
In all of these scenarios we can imagine the person whose ``rights were violated'' being perfectly fine with it. A~system of free and voluntary interaction, in which property rights are respected, would allow the persons involved to express these preferences explicitly. The boyfriend could tell his girlfriend that he would appreciate it. The professor could hire the student assistant under the condition that he corrects the exams. And of course anyone could look for free cleaning services. None of these examples is sufficient to disprove Rothbard's second pillar of welfare economics---that ``no act of government can ever increase social utility'' 
%\label{ref:RNDpi9y7PXvjj}(Rothbard, 2011, p.323).
\parencite[][p.323]{rothbard_toward_2011}.%




Rothbard's formulation would have been more on point if he had used the word ``state'' instead of ``government.'' We can imagine forms of government that do not involve rights violations, that is, governments to which everyone affected consents, but that is decidedly not the case for the modern state. By virtue of it being financed through coercive taxation it violates by its very nature the unanimity rule. Its actions therefore cannot increase social utility if one accepts that rule.



Now, one could imagine a~fictitious world in which every single citizen pays ``taxes'' voluntarily, believing that what their respective ``state'' does is necessary and welfare-enhancing. This would be a~world of implicit consent. Rothbard would probably have loved to live in such a~world. But in the real world, institutions would have to radically change for us to know whether we are in such an admirable state. Institutions would have to change in such a~way that implicit consent can be made explicit. This would mean among many other things the end of coercive taxation.


\end{artengenv2auth}

\label{israel-last}