\begin{artengenv}{Igor Wysocki}
	{Rejoinder to Block on indifference}
	{Rejoinder to Block on indifference}
	{Rejoinder to Block on indifference}
	{Nicolaus Copernicus University in Toruń\label{wysocki-rejoinder-firstpage}}
	{This paper is a~rejoinder to Block's
%	(2022)
	\parencite*{block_response_2022}
	response to Wysocki's
%	(2021)
	\parencite{Wysocki2021problem}
	essay on Nozick's challenge leveled at Austrian economics. Instead of merely reiterating Wysocki's
%	(2021)
		\parencite{Wysocki2021problem}
	position, we try to highlight that the Blockean account of indifference and preference entails the views which are otherwise unwelcome, given his unyielding commitment to Austrian economics at large. To wit, we argue that Block's theory still fails to make sense of the law of diminishing marginal utility. Moreover, his extreme idea of choice, sadly, appears to jettison characteristically Austrian subjectivism and thus perilously verges on behaviourism. We conclude that, given all these predicaments the Blockean account is caught in, Block himself (\textit{qua} Austrian) has a~reason to embrace the Hoppean theory of preference and indifference.
	}
	{choice, indifference, preference, Hans-Hermann Hoppe, Walter Block.}








\section{Introduction: the points of agreement }

\lettrine[loversize=0.13,lines=2,lraise=-0.03,nindent=0em,findent=0.2pt]%
{B}{}efore we embark on criticizing Block's account of preference and indifference, it is vital to underline the points of agreement between us and our intellectual adversary. This is important as it will allow us to all the more sharply capture the real bone of contention. What we, most crucially, share with Block is the view that indifference cannot be demonstrated in action 
%\label{ref:RNDGDZvdK1pBN}(see e.g., Block, 2009; Rothbard, 2011).
\parencites[see e.g.,][]{block_rejoinder_2009}[][]{rothbard_toward_1997}. %
 Indeed, the very idea of action presupposes \textit{some} preference. That is, as Mises 
%\label{ref:RND5ywImgIg4I}([1949] 1998, p.97)
\parencite*[p.97,][]{mises_human_1998} %
 put it:



\begin{quote}
Action is an attempt to substitute a~more satisfactory state of affairs for a~less satisfactory one… A~less desirable condition is bartered for a~more desirable. What gratifies less is abandoned in order to attain something that pleases more.
\end{quote}



Granted, it is due to the fact that individuals judge a~state of affairs that would obtain in the absence of their respective actions to be \textit{less preferable} to the one that they believe would be brought about by these actions that they engage in acting in the first place. Or in other words, if an economic actor believed that her action would render her no better off than if she were not to act at all, she would refrain from acting. It is in this sense that action at the very minimum presupposes \textit{some} preference. Sweeping indifference would result in no action whatsoever---no disagreement with Block just yet.



What we also concur on with Block is the relation between the concepts of choice, preference and indifference. We, quite much in the Blockean spirit, conceive of the relation between the impossibility of choice and indifference as that of logical equivalence. That is, formally, for all S's, S~an economic agent, S~is indifferent between \textit{x} and \textit{y}\footnote{The variables \textit{x} and \textit{y}~are best treated as mere place holders, for they may stand for such various entities as states of affairs, physical objects, actions. After all, an individual may well be indifferent between (or have a~preference for) particular states of affairs (e.g. whether it is raining or not), physical objects (e.g. tea of coffee) and between specific actions (e.g. whether to start playing tennis with the \textit{left} or \textit{right} hand---see Hausman's 
%\label{ref:RNDAbNk6QYMA5}(2011, p.27)
\parencite*[][p.27]{hausman_preference_2011} %
 ``final preferences'' defined as ``preferences among the immediate objects of choice''). } if and only if S~cannot choose between \textit{x}~and \textit{y}. On the other hand, it takes S's \textit{preferring x} to \textit{y}~for S~to \textit{choose} \textit{x} over \textit{y}. Technically, the fact that S~chooses \textit{x}~over \textit{y}~implies that S~(strictly) prefers \textit{x}~to \textit{y}. And Block 
%\label{ref:RNDhvYEDz2muG}(2022, p.47)
\parencite*[][p.47]{block_response_2022} %
 concurs, which is manifested in the passage wherein Block invites us to consider the case of a~grocer endowed with a~stock of one-pound packages of butter who ``must choose one of these one-pound packages, to give to the thief/customer.'' The grocer then, we are supposed to imagine, ``chooses the first one''. Block's conclusion is that ``he is no longer indifferent.''



Clearly then, we are on the same page with Block as far as the view of choice as preference-implying is concerned. Furthermore, we take no issue with the characteristically Austrian contention to the effect that it is \textit{some preference} rather than indifference that manifests itself in action. However, the devil is in the details. And so there are indeed subtle points of disagreement between our account and Block's, the points to which we are now turning.



\section{The real bone of contention}

Although, as mentioned above, we side with Block as far the thesis that choice implies preference goes, our more nuanced position concerning \textit{individuating} alternatives subject to choice finally makes it the case that our account of indifference and preference diverges from Block's dramatically. Just to remind the reader, our view is that if one is indifferent between \textit{x}~and \textit{y}, then one cannot logically choose between them. Or still in other words, if one cannot choose between \textit{x} and \textit{y}, then \textit{x}~and \textit{y}~do not constitute economically distinct alternatives.\footnote{This sort of insight---with a~slight modification---is also present in the mainstream theory of action. Says Broome 
%\label{ref:RNDrCnHMKcd8E}(1991, p.103):
\parencite*[][p.103]{broome_weighing_1991}: %
 ``Outcomes should be distinguished as different if and only if they differ in a~way that makes it rational to have a~preference between them.'' Hoppe 
%\label{ref:RND1hOhsohe2q}(2005)
\parencite*[][]{hoppe_must_2005} %
 advances a~similar thesis. This author has it that alternatives subject to choice should be considered distinct if and only if they differ in a~way that an actor \textit{does actually} have a~preference over them. And hence, if any two ``alternatives'' do not differ in any economically relevant sense according to the economic actor, then the two alternatives are not really alternatives. There is no choice between them. } To illustrate our point, if an actor S~values watching football most and he values going for a~walk equally highly, whereas he values playing a~game of chess less, while valuing having a~nap just as much as a~game of chess, we can represent his \textit{choices} on the following value scale:







\begin{enumerate}[label=(\arabic*)]

\item[]\makebox[-1.7em][l]{}V\textsubscript{1}

\item Watching football \textit{or} going for a~walk

\item Playing a~game of chess \textit{or} having a~nap

\end{enumerate}

As can be seen, there are only \textit{two} economically distinct choices instead of four of them. And again, the reason is that since the stipulated actor S~is indifferent between watching football and going for a~walk as well as between playing a~game of chess and having a~nap, he cannot choose between watching football and going for a~walk. Neither can he choose between playing a~game of chess and having a~nap. In conclusion, he chooses \textit{only} between (1) and (2).



Equipped with this conceptual apparatus, we are now in a~position to spell out a~relevant difference between our account of choice and Block's. At this point, it is crucial to note that the individual's given behaviour \textit{underdetermines} a~value scale on which she has acted. Or, to put this point more technically, there is a~one-to-many relation between a~certain act-token and an underlying value (preference) scale. Still in other words, a~given behaviour might be \textit{evidential} of many value scales. That is, (infinitely) many value scales may manifest themselves in any particular act. For example, suppose we know nothing yet of how our stipulated actor S~actually ranks the four ``alternatives'' stated above. Further imagine that S~ends up watching football. We posit that from this fact alone we cannot infer a~specific value scale guiding S's action. For, S~might as well have been indifferent between watching football and going for a~walk. Alternatively, he might have (strictly) preferred watching football to anything else he saw as a~possibility. If so, then his value scale might be the following:



%V\textsubscript{2}



\begin{enumerate}[label=(\arabic*)]

\item[]\makebox[-1.7em][l]{}V\textsubscript{2}

\item Watching football

\item Going for a~walk

\item Having a~nap

\item Playing a~game of chess

\end{enumerate}
And this is \textit{precisely} where our account diverges from Block's. For, it seems that according to Block action is a~manifestation of preference \textit{all across the board}. At this point, we cannot do better than quote Block at length. Says our author about the Buridan's ass example:



\begin{quote}
Wysocki misconstrues Buridan's ass in the same manner. This beast, let us say, chooses the bale of hay to the right. The correct interpretation of this is two fold: one, this creature preferred life to death, and, two, he favored the hay on the right to the hay on the left. In Wysocki's correct interpretation of Hoppe, and his own, only the first is true. The second, amazingly, is not. But, but, but, the donkey moved to his right, not his left! If this is not evidence that he preferred the right to the left bale, there can be no such thing as evidence, at least not in cases like this. 
%\label{ref:RNDWkzc44Dug7}(Block, 2022, pp.51–52)
\parencite[][pp.51–52]{block_response_2022}%
\end{quote}




First thing to note here is that Block is clearly strawmanning against Hoppe 
%\label{ref:RNDHzVDcSlRXd}(2005)
\parencite*[][]{hoppe_must_2005} %
 and Wysocki 
%\label{ref:RND5w7BMDb2ic}(2021).
\parencite*[][]{wysocki_problem_2021}. %
 Neither of these authors claim that it is impossible for the Buridan's ass to \textit{prefer} the right bale to the left. Rather, Hoppe's and Wysocki's point is that the fact that the donkey moves to his right is in and of itself insufficient to establish whether the donkey does \textit{prefer} the right bale to the left one. For, the donkey might as well be indifferent between the two. In that case, the donkey \textit{would not be choosing} between the two bales but indeed between something else---most plausibly, between eating or starving. Certainly, it is possible for the donkey to \textit{choose} between the bales. But in that case, the donkey must have a~\textit{preference} for one over the other. All in all, how many choices the actor faces depends on the Hoppean 
%\label{ref:RND4KiQoajixd}(2005)
\parencite*[][]{hoppe_must_2005} %
 \textit{correct description of action} (or action under intentional description) and not on the actor's behaviour as extensionally described. Whereas the fact that the donkey moves to the right is, for Block, a~decisive reason to conclude that the donkey \textit{prefers} the right bale to the left one, we submit that this fact alone does not suffice to establish what the donkey prefers over what as it takes an \textit{intentional} description of his action to be able to determine his preferences. Remember, we agree on one thing. The donkey's action most definitely is a~manifestation of \textit{some} preference, for otherwise the donkey would not engage in action at all. However, the donkey's particular behaviour underdetermines the value scale guiding his action. To summarize, the donkey's behavior being fixed (i.e. the animal moves to the right bale of hay and eats it), we contend that it is evidential of (at least) the following two value scales.



%V\textsubscript{3}



\begin{enumerate}[label=(\arabic*)]

\item[]\makebox[-1.7em][l]{}V\textsubscript{3}

\item Eat from a~right bale of hay

\item Eat from a~left bale of hay

\item Starve

\end{enumerate}

\pagebreak[2]



\begin{enumerate}[label=(\arabic*)]
%\item[or~V\textsubscript{4}]
\item[]\makebox[-1.7em][l]{}V\textsubscript{4}
\item Eat from either a~right \textit{or} a left bale of hay
\item Starve
\end{enumerate}

By contrast, Block avers that the donkey's behaviour unambiguously points to V\textsubscript{3} as an underlying value scale, which we can allegedly infer from the very fact that the animal moved to the right rather than to the left.



Having, hopefully, spelled out the difference between the Hoppean (and Wysocki's) and Block's account of preference of indifference, let us move now to consider why the Blockean theory leads to unwelcome consequences.



\section{Block's \textit{ad hoc} after-action/before-action distinction }

It is precisely Block's distinction between the time \textit{before} an action and \textit{after} it that constitutes the crux of his response. Block's 
%\label{ref:RNDqTsTHLpyG5}(2022, p.52)
\parencite*[][p.52]{block_response_2022} %
 discussion of his famous thought experiment involving a~seller endowed with 100 units of butter shall serve as a~good illustration of our intellectual adversary's viewpoint. Block appears to be relegating indifference entirely outside the realm of action as he believes that the said butter seller is indifferent between the units of his stock \textit{only} \textit{before} some action involving those units is taken. Says our author: ``At time t\textsubscript{1}, before any choice was made, yes, all units of butter were ``equally serviceable.'' Their owner was indifferent between all of them. They were homogeneous as far as he was concerned''. However, when at t\textsubscript{2} the seller encounters a~customer who is willing to buy one unit of the commodity supplied by the former, and the seller gives up 72\textsuperscript{nd }unit, then this very fact, according to Block, establishes that he indeed disprefers \textit{this} (i.e. 72\textsuperscript{nd}) unit to any other. Or, in Block's words, ``[if] this does not establish that he valued this particular one, the 72\textsuperscript{nd} unit, less than the others, then there is no such thing as choice, utility, economic theory, common sense.''



We, by contrast, contend that the inference from the fact of giving up \textit{a~particular} unit to the conclusion that this very unit must have been dispreferred to any other is rather, if anything, a~travesty of common sense. After all, why \textit{should} it be the case that the seller indeed \textit{chooses} to give up the 72\textsuperscript{nd} unit? Why does Block draw this conclusion? Merely because the \textit{extensional description} of the seller's action is that he gives up this very unit? Fair enough. As far as the extensional description goes, it is a~rather accurate one. However, it is still a~far cry from establishing the seller's action \textit{under intentional description}, for we do not know from this action alone between \textit{what} the seller was choosing. Just to resort to value scales, the seller's action might have been guided by (at least) these two distinct value scales.



%V\textsubscript{5}



\begin{enumerate}[label=(\arabic*)]

\item[]\makebox[-1.7em][l]{}V\textsubscript{5}

\item To earn money by giving up the 72\textsuperscript{nd }unit of butter

\item To earn money by giving any other unit\footnote{This value scale and the following one---unlike others invoked in the present paper---apart from the actor's ends include also the means. However, this illustrates the point that the actor---as in Block's example---might clearly have a~preference for particular means, with his end being fixed. After all, Block's point is precisely that, the seller's end being equal, she prefers to give the 72\textsuperscript{nd }unit of butterto giving up any other. }

\end{enumerate}

%or indeed by V\textsubscript{6}



\begin{enumerate}[label=(\arabic*)]

\item[]\makebox[-1.7em][l]{}or indeed by V\textsubscript{6}

\item To earn money by giving up \textit{any} unit of butter

\item To preserve all the units and earn no money

\end{enumerate}







Then again, our position is that the seller's action underdetermines a~value scale guiding it. That is, for example, it might be V\textsubscript{4} or V\textsubscript{5} that make sense of the seller's behaviour. By contrast, according to Block, the fact that the seller gave up (as extensionally described) the 72\textsuperscript{nd }unit \textit{exclusively} points to V\textsubscript{4} as the scale guiding his action. But why should that be a~\textit{correct description of the seller's action}? We claim that the actor in question might as well be indifferent between \textit{all the units} of butter involved. Granted, when it came to the seller's action, he must have been guided by \textit{some} preference but this fact by itself cannot establish that he was guided---among other things---by the dispreference for the 72\textsuperscript{nd} unit of butter. And, we submit, it is all the more natural to assume that the seller was guided by the preference for some money over \textit{any} particular unit of butter. And this preference will do for classifying the seller's behaviour as action. There is no need at all to claim that the actor \textit{also} dispreffered the actual unit given up to any other.



Now, it is crucial to note that it is precisely Block's contention that from the act of giving up a~particular unit we can infer a~dispreference for that very unit that leads him to the weird eponymous after-action/before-action distinction. Remember, Block believes that the seller starts with indifference among all the units of butter. However, since he believes that the actor's act of giving up a~particular unit implies a~dispreference for that unit, he must \textit{now} posit that the actor is no longer indifferent among all the units of his commodity. Sadly, Block never explains why there is this sudden change in the actor's mental state. By contrast, the Hoppean account does not need to resort to the before-action/after-action distinction at all to explain the seller's act. If, by assumption, the actor is indifferent among all the units of butter, then his act does not (and cannot) demonstrate dispreference for the actual unit given up. But this does not prevent us from making sense of the actor's act. If the actor is genuinely indifferent among all the units of butter, his action might be still conceived of in terms of---among other possible explanations---the preference of giving up \textit{a}~unit of butter rather over preserving all of them but earning no money (see: V\textsubscript{5}). That is, in the Hoppean account, it is, most naturally, the actor's preference guiding the actor's action: if the actor prefers \textit{x}~to \textit{y,} he chooses \textit{x}~over \textit{y}, whereas if he is indifferent between \textit{a}~and \textit{b,} he does not choose between \textit{a}~and \textit{b}. More concretely, if he is indifferent between particular units of butter, then he does not choose between them. If he prefers some money to \textit{any} unit of butter, then he chooses to give up \textit{a}~unit of butter for some money. There is no need to postulate \textit{any} arbitrary change in the actor's state of mind to understand his resultant behaviour. Block, by contrast, is powerless to explain the actor's \textit{choice}, for how can he \textit{choose} to give up the 72\textsuperscript{nd} unit if the actor was \textit{ex hypothesi} indifferent between all of them. For Block to conclude that the said economic agent \textit{chose} to give up that very unit, it must be assumed that he was not indifferent between \textit{that unit} and any other one; viz., that he dispreferred precisely the 72\textsuperscript{nd }unit. But if Block were to embrace this assumption, he could not in turn make sense of the supply of the same economic good. Thus, Block seems to be caught in an unenviable dilemma. On the one hand, if he wants to stick to his idea of action as demonstrating \textit{preference all across the board}, he has to compromise the notion of the supply of the same good. Alternatively, if wants to keep the robust notion of the supply of the same commodity, he would need to make a~major concession to Hoppe. To wit, he would have to concede that the seller does \textit{not} disprefer the 72\textsuperscript{nd} of butter when he gives it up.



To illustrate further the dilemma the Blockean framework faces, let us test how it fares when given the task of capturing the law of diminishing marginal utility. Suppose, Block starts out with a~stock of three apples (A\textsubscript{1}, A\textsubscript{2}, A\textsubscript{3}), which he finds equally serviceable. Further, Block envisages exactly three ends that he believes \textit{each} apple can satisfy. The ends are (in the descending order of importance):



\begin{enumerate}

\item Eating an apple

\item Giving it to a~friend

\item Throwing it for distance

\end{enumerate}

Now, in Block's preferred vernacular, here is Block ``before action'', equipped with three units of the same commodity. He finds them all ``equally serviceable'' and thus he is indifferent between all of them. Now it is time for Block to satisfy his consecutive ends by means of the apples. Naturally, Block eats his \textit{first} apple, which satisfies his most pressing end. Say, he eats A\textsubscript{2}. This, however, according to Block already implies that \textit{in fact} A\textsubscript{2} was not equally serviceable as the remaining two apples. Nay, A\textsubscript{2} was dispreferred to the two apples remaining. So, it magically turns out that Block's act of eating one apple demonstrates that he was dealing not with a~homogeneous set of apples but with \textit{two} distinct classes of economic goods: (1) with the dispreferred apple he actually ate and (2) a~homogeneous set of two remaining equally serviceable apples. Secondly, Block quite reasonably gives one apple to his friend. Say, he gets rid of A\textsubscript{3} for that purpose. Now, since Block indeed gave up A\textsubscript{3}, this means that he dispreferred it to the remaining apple (i.e. A\textsubscript{1}). So, in the end, contrary to the original assumption, Block's subsequent actions demonstrate that in fact the three apples were not economically homogeneous. More, Block's inference is that they were \textit{all} heterogeneous. However, remember, the three apples were, \textit{by assumption}, homogeneous. After all, we were after illustrating the law of diminishing marginal utility using Block's preferred framework. As can be seen, Block's account of preference and indifference completely fails. In the above scenario of employing three apples, Block's theory predicts that there is \textit{only one} preferred way to economize them over time; that is, the one that actually obtained; viz, \textit{first} A\textsubscript{2}, \textit{second} A\textsubscript{3}, and \textit{finally} A\textsubscript{1}. However, as demonstrated by Wysocki 
%\label{ref:RNDejJ6jqT9YM}(2021, p.41),
\parencite*[][p.41]{wysocki_problem_2021}, %
 we should expect 3! (which is six) ways to economize those three apples. After all, since they are assumed to be equally serviceable, then it would be---by assumption---equally good for Block to, say, first employ A\textsubscript{1}, then A\textsubscript{2} and finally A\textsubscript{3}. The same applies to \textit{any permutation} of the said three apples. How can it be otherwise when they are assumed to be equally serviceable? Finally, it is well-worth noting that the Hoppean account does not run into the same sort of problem, for, according to Hoppe, since the agent would be indifferent between three apples he would not choose \textit{among} them. Still, he would \textit{choose} between different ends each apple can satisfy. That is, as in the scenario above, the actor would first eat \textit{an(y)} apple, then give \textit{any} other of the two remaining apples to a~friend, and finally throw the remaining apple for distance. Hence, the actor would be throughout the process indifferent between the apples (means employed), while at the same time demonstrating \textit{some} preference (i.e. satisfying more pressing needs sooner later than later). Therefore, it is the Hoppean account and not Block's that does justice to both the fact that the agent was acting (i.e. there is \textit{some preference} getting demonstrated) and to the law of diminishing marginal utility (i.e. the apples are deemed equally serviceable through the whole sequence of actions). Concluding, given the fact that Block \textit{qua} Austrian fully subscribes to the law of diminishing marginal utility, he would do better to drop his before-action/after-action distinction as it seems to jeopardize the said law, clearly too high a~price to pay. Needless to say, the Hoppean account suffers from no such defects and so Block has all the reason to embrace it. Having said that, it is time to elucidate other problems the Blockean theory suffers from.



\section{Agency is not all about strict preference}

Another problem haunting Block's response is not taking heed of the distinction between \textit{agency} and what the actor does \textit{under an intentional description}.\footnote{The distinction being brilliantly illuminated by Davidson 
%\label{ref:RNDq7F3EvNAnI}(2001).
\parencite*[][]{davidson_agency_2001}. %
 } What motivates this distinction is that apparently an \textit{extensional} description of the agent's action does not necessarily coincide with its \textit{intentional description}. To wit, not every single aspect of the agent's external behaviour (at some level of description) is such that she intends it. To briefly illustrate the distinction yet again, let us analyse a~rather typical script of entering a~café to order coffee. So, as \textit{extensionally} described, the customer normally enters a~café with a~particular foot (either left or right one is the \textit{first} to enter the desired area). However, it certainly does not follow that \textit{once} the agent enters the café with her left foot, she thereby demonstrates her preference for entering with this particular foot to entering with the other one. For, the \textit{content} of the agent's intentional state (i.e. of \textit{what} the agent \textit{intends} to do) might be simply to enter the café with the ways of entering it being left unspecified. Similar remarks apply to the agent's ordering a~coffee. Suppose, the waiter approaches our economic actor and the latter says: ``I will have a~large cappuccino.'' It definitely does not follow that the actor had some preference for \textit{this particular} wording of her order over any other. That is to say, as long as \textit{any} wording constitutes a~speech act of ordering a~coffee, the actor might be perfectly indifferent between alternative ways of ordering the desired drink. Moreover, at still some finer-grained level of description, our actor's pronouncing her order necessarily has a~suprasegmental property of having a~definite pitch. For the actor might order a~coffee by pronouncing her order at, say, a~very high pitch. But then again, why should that follow that the agent did indeed intend to place an order at a~high pitch. She might as well \textit{simply} wanted to place an order (with the pitch remaining unspecified in her intentional state). But if so, then there is no reason to assume that the fact that the actor's order was delivered at a~high pitch demonstrates her preference for \textit{that} pitch over any other. By contrast, Block's position seems to predict that \textit{since} the agent does indeed enters with, say, the right foot, this \textit{ipso facto} is evidentiary of her preference for this particular way of entering the café. By the same token, the fact that the agent orders a~cappuccino at a~high pitch is, for Block, indicative of the agent's (strict) preference for \textit{that} pitch over any alternative one. Yet, Block's conclusion is implausible. Clearly, one cannot apodictically infer a~(strict) preference for such minute details of action-tokens as highly specific bodily movements or highly specific features of our linguistic behaviour. And the reason is that entering a~café \textit{with a~particular foot} would not typically figure in the content of our intentional states. Rather, the most natural description of the actor's practical syllogism\footnote{For an excellent elaboration on practical syllogism, see e.g. Moore 
%\label{ref:RNDB3OpSAI2IK}(1993; 2020).
\parencites*[][]{moore_act_1993}[][]{moore_mechanical_2020}. %
 } is the following. She \textit{desired to drink a~coffee} and because she \textit{believed} that by entering a~(particular) café she can satisfy her desire, she \textit{intended} to enter it. Under this description, the agent does not believe that it is \textit{only} by entering a~café with a~particular foot that she can ultimately satisfy her desire for coffee. Hence, neither does she \textit{intend} to enter the place with a~particular foot. She simply intends to walk in whether with her left or right foot. And because a~particular way of entering (i.e. either with the left or right foot) is outside the content of the agent's \textit{intentional states} (both her \textit{belief} and \textit{intention}), it would be far-fetched to infer the agent's preference for a~particular way of walking in merely from the fact that the agent \textit{in fact} does enter with a~particular foot.\footnote{Note that the Hoppean 
%\label{ref:RNDSIdEb6znlW}(2005)
\parencite*[][]{hoppe_must_2005} %
 account does not prevent us from saying that the agent described does indeed have a~preference for a~particular way of walking in. However, this preference does not, for Hoppe, follow automatically from the fact that the agent walks in with a~particular foot. According to Hoppe, the ultimate test for agent's preference is the correct description of her action, which coincides with the Davidsonian 
%\label{ref:RNDCQ94M0xhSP}(2001)
\parencite*[][]{davidson_agency_2001} %
 \textit{intentional description of an action}. } Such an inference would, to our mind, make a~mockery of preferences. If the economic agent strictly prefers \textit{A}~to \textit{B,} she values \textit{A}~higher than \textit{B}. Why should it be apodictically true then that if our actor enters a~café with her left foot rather than with right one, this demonstrates that she \textit{values} this particular entrance (i.e. with the left foot) higher than the alternative entrance with the right foot? It is most implausible to claim that \textit{this} particular valuation immediately follows. Surely, we are ready to concede that some differential valuation follows from the very fact that the agent is acting in the first place. As we insisted on above, action implies the demonstration of \textit{some} preference but that is everything that follows with apodictic certainty from the fact that the agent acts. Block's conclusion is therefore illegitimate and clearly too strong. And just as entering a~café with a~left foot is not normally preferred to entering it with a~right foot, so these two action-tokens do not normally---\textit{contra} Block---constitute two distinct choices. And again, insisting that the agent \textit{does choose} to enter a~café with her right foot because she \textit{actually} entered it with her right foot is to make the same mistake as the one involving the inference to the actor's preference mentioned above. After all, the agent does not have to \textit{conceive} of these two alternative ways of walking in as \textit{relevantly} different. Either, she may well believe, will serve her end equally well.



Finally, let us have a~look at Block's 
%\label{ref:RNDP2kuPsd6a0}(2022, pp.50–51)
\parencite*[][pp.50–51]{block_response_2022} %
 analysis of the Hoppean example involving a~poor mother who can rescue only one of her sons (i.e. either Peter or Paul) as the said analysis aptly illustrates the Blockean confusion between agency and intentional description of an action and allows us to raise our final objection to his theory. As expected, from the fact that the mother saves Peter Block draws an inference to the conclusion that she ``places a~higher value on Peter than Paul.'' But then again, just as---as we already saw---one cannot infer the preference for entering a~café with a~right foot from the fact the agent does actually enter with that very foot, so we cannot infer the mother's preference for Peter over Paul from the very fact that Peter was saved. As we reiterated throughout this essay, the fact that the mother saves Peter (extensional description) underdetermines the value scale guiding the mother's action, for the mother might equally well frame her end as saving \textit{a}~child rather than saving Peter. And if the former is true, then saving Peter serves this end equally well as saving Paul. That is why, she can remain (before and after action) indifferent between the two of her sons. And it is precisely for that reason that she does not (and cannot) choose between the two. No contradiction here.



However, Block 
%\label{ref:RNDiujECtkP9I}(2022, pp.50–51)
\parencite*[][pp.50–51]{block_response_2022} %
 protests: ``She did rescue the former, when she could have chosen differently, and selected the latter for retrieval, did she not?'' But this simply begs the question. We, following Hoppe, contend that the mother's action in and of itself is not determinative of the mother's value scale, for the mother might as well simply prefer rescuing \textit{a}~child to saving \textit{none}. And if the mother frames her ends in this way, then it logically follows that the mother does not choose between Peter and Paul. Rather, in this scenario, the mother is choosing between saving \textit{a}~child over saving \textit{none}. And that is why Block's assertion does no more than beg the question.



Eventually, to add insult to the injury, Block 
%\label{ref:RNDGFu0QQRTG5}(2022, p.51)
\parencite*[][p.51]{block_response_2022} %
 adds that even if the mother ``did this with her eyes closed, and just grabbed the nearest son'', this would still indicate that the mother chose to save Peter. Yet, how can grabbing a~certain son with one's eyes closed count as demonstration of preference for that son? If anything, it seems that under that scenario the mother prefers grabbing \textit{any one} son over saving none. It appears as though the most charitable take on the Blockean idea of choice is that the author---his protestations to the contrary notwithstanding---embraces methodological behaviourism.\footnote{Granted, Block may not be an ontological behaviourism. That is, he clearly does \textit{not} deny the existence of mental states. Neither does he reduce them to behaviours or mere dispositions to behaviour. However, he seems to \textit{model} (or define?) preferences in terms of the agent's external behaviour. 
%\label{ref:RNDsQH9mf95KK}(Block, 2022, pp.54–55)
\parencite[][pp.54–55]{block_response_2022}%
This, to our mind, looks very much like \textit{methodological behaviourism}, the view according to which positing mental states adds nothing to understanding the individual's external behaviour. As we are about to see to in the forthcoming part of the text, the Blockean construal of Peter-and-Paul scenario appears to abstract from the mother's preferences (as genuine mental states) completely and instead models the mother's apparent choice \textit{solely} around her external behaviour. For an exposition of different senses of behaviourism, see e.g. Moore 
%\label{ref:RNDPX5r3KaLmT}(2001).
\parencite*[][]{moore_distinguishing_2001}. %
 } For, if the mother were to indeed ``choose'' to save Peter with her eyes closed (i.e. being completely unaware of who she is in fact saving), in what sense is this ``choice'' even driven by preferences or any other mental states for that matter. We are afraid, in none. Rather, with her eyes closed, the mother simply \textit{happens} to save Peter. It is not the case, by stipulation, that she \textit{believed} that she is saving Peter. Worse, Block even goes to such great lengths to say that the mother does not even have to cherish an \textit{intention} to save \textit{either} of his sons for her act to count as an evidence that she \textit{chose} to save Peter. Says Block 
%\label{ref:RNDMG5ZBJhKaS}(2022, pp.54–55):
\parencite*[][pp.54–55]{block_response_2022}:%




\begin{quote}
we as praxeologists must note that you actually reached out and grabbed one of them, not the other. This is the essence of Hoppe's error, with support from Wysocki. What might well have been on her mind had nothing to do with Peter nor Paul. It might well have been as Hoppe opined, she was just preferring to save one of her sons, rather than none. Who knows, she might have been thinking about ice cream, as far as we praxeologists are concerned. This does not matter in the slightest for the praxeologist. We see her grabbing Peter, not Paul, to safety, and we are compelled by praxis logic, e.g., praxeology, to note that she was not indifferent between her sons, she could not have been indifferent between them, given that she chose the one, not the other.
\end{quote}



But this radical view comes perilously close to methodological behaviourism, for Block seems to dismiss the mother's mental states completely. Note, even if the mother were to think ``about ice cream'', she would still choose to save Peter in the event Peter would be ultimately saved. But this at a~stroke gives up characteristically Austrian methodological subjectivism\footnote{Let us not lose sight of Hayek's 
%\label{ref:RNDWqnr4QQKwE}(1952, p.31)
\parencite*[][p.31]{hayek_counter-revolution_1952} %
 famous dictum: ``It is probably no exaggeration to say that every important advance in economic theory during the last hundred years was a~further step in the consistent application of subjectivism.'' } and denies any role to the actor's mental states (preferences and beliefs) as determining choices. Again, Block's die-hard insistence on his radical idea of choice appears at the same time to compromise what he otherwise holds dearly, that is Austrian subjectivism with its insistence on \textit{purposeful} behaviour. Given this, we again submit that for Block to disown his account of choice is to pay a~relatively small price for saving what he \textit{qua} Austrian otherwise strongly believes. In other words, we claim that the most efficient way for Block to make his views coherent is to drop his problematic theory of choice, preference and indifference.



\section{Conclusion}

As we tried to show in this rejoinder, Block's account of choice, preference and indifference fails on three counts. First, Block's theory---despite his claims to the contrary notwithstanding---cannot make sense of the law of diminishing marginal utility. For it is precisely the Blockean radical idea of choice which predicts that allegedly homogeneous (i.e. equally serviceable) units ultimately prove to be heterogenous. Moreover, we demonstrated that Block's resorting to the before-action/after-action distinction is of no help. Not only is this distinction \textit{ad hoc} but also it fallaciously predicts that \textit{n} number of allegedly equally serviceable units can be economized in only \textit{one} optimal way, something immediately running counter to the original assumption of the economic homogeneity of the said units.



Later on, we illuminated two more unwelcome consequences on the Blockean theory under consideration. The first of them is that Block's 
%\label{ref:RNDmxlz1FgNA6}(2022)
\parencite*[][]{block_response_2022} %
 account fails to distinguish between what is attributable to the economic agent's \textit{agency} and what the agent does \textit{intentionally}. While trying to reduce Block's not observing this distinction to absurdity, we show that this author would have to conclude that \textit{literally} any single minute detail of the actor's act-token is preferred (to some other minute detail) and therefore chosen. This conclusion, in turn, is most clearly implausible, which serves to repudiate the Blockean theory of choice \textit{via modus tollens}.



Finally, we suggested that Block's theory dangerously verges on methodological behaviourism, the view that this author most definitely rejects \textit{qua} Austrian. Given all these unwelcome consequences stemming from Block's insistence on his account of choice, preference and indifference, we claim that this author has a~decisive reason to simply disown the said account. After all, as it seems, this particular theory of his is purchased at a~huge cost of jeopardizing other vital aspects of Austrian economics, especially the law of diminishing marginal utility and overall Austrian insistence on methodological subjectivism rather than methodological behaviourism. Needless to say, embracing the Hoppean 
%\label{ref:RNDBx29EHZukE}(2005)
\parencite*[][]{hoppe_must_2005} %
 account of preference and difference would be a~right way for Block to go.





\end{artengenv}\label{wysocki-rejoinder-lastpage}

