\begin{artengenv}{Łukasz Dominiak}
	{Free market, blackmail, and Austro-libertarianism\edtfootnote{This research was funded in whole or in part by the National Science Centre, Poland, grant number 2020/39/B/HS5/00610.}}
	{Free market, blackmail, and Austro-libertarianism}
	{Free market, blackmail, and Austro-libertarianism}
	{Nicolaus Copernicus University in Toruń}
	{In the present paper we examine the standard Austro-liberrtarian account of blackmail according to which blackmail should be legal as it does not coerce the blackmailee to part with his property and so cannot be subsumed under extortion. Against this account we put forth a~preliminary argument or a~hypothesis, if you will, that even if blackmail cannot be subsumed under extortion, it still does not follow that it should be legal, for it might be subsumed under fraud. Indeed, the hypothesis we would like to offer for consideration is that blackmail is fraud, at least under some circumstances. To wit, we claim that even if the blackmailer does not coerce the blackmailee, in cases in which the blackmailer does not have an intention to execute his otherwise legal threats, he nonetheless deceives the blackmailee, thereby inducing him to part with his property. This is fraud and it renders the blackmailee's property transfer involuntary and invalid. As fraud should be illegal under Austro-libertarianism, so should blackmail.
	}
	{blackmail, fraud, coercion, Austro-libertarianism, Walter Block.}



\footnotetext{This research was funded in whole or in part by the National Science Centre, Poland, grant number 2020/39/B/HS5/00610.}












\section{Introduction}

\lettrine[loversize=0.13,lines=2,lraise=-0.03,nindent=0em,findent=0.2pt]%
{T}{}he standard Austro-libertarian view on blackmail is that blackmail would be legal on the free market due to the fact that exchanges effectuated under blackmail are not coerced and thus free or voluntary.\footnote{Writes Rothbard 
%\label{ref:RNDjUbFoVONYE}(2009, p.183):
\parencite*[][p.183]{rothbard_man_2009}: %
 ``[\textit{B}]\textit{lackmail} would not be illegal in the free society. For blackmail is the receipt of money in exchange for the service of not publicizing certain information about the other person. No violence or threat of violence to person or property is involved.'' } Since the free market is nothing else than the entirety of voluntary exchanges,\footnote{Again, writes Rothbard 
%\label{ref:RNDp1HOXIHIjd}(2011, p.320):
\parencite*[][p.320]{rothbard_toward_2011}: %
 ``The free market is the name for the array of all the voluntary exchanges that take place in the world.'' } blackmail would be part and parcel thereof. Certainly, blackmail is immoral, but there are many immoral things taking place on the free market.\footnote{On immoral although legitimate free market practices see 
%\label{ref:RNDLcJ4HJkm6P}(Block, 2018).
\parencite[][]{block_defending_2018}.%
} What is crucial, is not whether it is moral or immoral, but whether it is free or unfree, voluntary or involuntary. Similarly, it can be argued that blackmail is a~threat rather than an offer, but then again, it is inconsequential if it is a~threat or an offer.\footnote{Walter Block sometimes calls blackmail a~threat, sometimes an offer. See, for example, Block and Anderson 
%\label{ref:RNDP2L8VyRXtU}(2000, p.546),
\parencite*[][p.546]{block_blackmail_2000}, %
 Block 
%\label{ref:RNDRCqHeE2EGG}(1998a, p.218).
\parencite*[][p.218]{block_libertarian_1998}.%
} On the free market there are many offers that cannot be refused and many threats that can be withstood. What matters is whether it coerces another to part with his goods or services. Since it does not, it is voluntary and would therefore be legal on the free market. Or so argue Austro-libertarians.



In the present paper we supplement this argument with an observation concerning another dimension of blackmail. More specifically, we put forth a~preliminary argument or a~hypothesis, if you will, that even if one accepts the Austro-liberrtarian premise that blackmail proposals do not coerce, it does not follow that exchanges induced by such proposals are free or voluntary\footnote{It might be viewed as a~bit clumsy to use ‘free' and ‘voluntary' synonymously. We are aware of that. Regardless, we take our liberty to do so because Austro-liberrtarians themselves do so. } and so would find their place on the free market as it is understood by Austro-liberrtarians. The reason for which the non-coercive character of blackmail proposals does not entail voluntariness of the resultant exchanges is that there are two ways in which voluntariness of human actions can be vitiated and coercion is only one of them. Another is ignorance. As we try to argue and explain, at least some blackmail proposals deceive the blackmailee and thus render the resultant exchange involuntary due to the blackmailee's induced ignorance. In other words, the blackmailer, at least in some cases, defrauds the blackmailee. Thus, the hypothesis we would like to offer for consideration is that blackmail is oftentimes fraud. Now since Austro-liberrtarians strongly believe that fraud ought to be illegal, they should reject their current view on blackmail as inconsistent with this strong belief and instead embrace the view that \textit{qua} fraud, blackmail should also be illegal.



The present paper is organized in the following way. Section 2 offers an in-depth analysis of the standard Austro-liberrtarian account of blackmail. Section 3 argues against this account and puts forth what can be called a~preliminary revisionist Austro-liberrtarian account of blackmail. The crucial aspect of this account is that it tries to subsume some types of blackmail under fraud. Section 4 elucidates reasons for which fraud, and thus blackmail, should be illegal under Austro-libertarianism. Section 5 concludes.



\section{The Standard Austro-liberrtarian Account of Blackmail}

The standard Austro-liberrtarian account of blackmail begins with the distinction between blackmail and extortion. According to this account, extortion---regardless of how this term is actually used in existing legal systems\footnote{For example, Glanville Williams 
%\label{ref:RND0ml2CeCOEc}(1983, p.838)
\parencite*[][p.838]{williams_textbook_1983} %
 reminds us that as far as English law is concerned, ``[i]t is the offence of extortion at common law for a~public officer to take, by colour of his office, any money or thing that is not due to him.'' In turn, \textit{California Penal Code (2006)}, as reported by Sanford H. Kadish, Stephen J. Schulhofer and Carol S. Steiker 
%\label{ref:RNDjkPItdvcdA}(2007, p.941),
\parencite*[][p.941]{kadish_criminal_2007}, %
 defines extortion under Section 518 in the following way: ``Extortion is the obtaining of property from another, with his consent, or the obtaining of an official act of a~public officer, induced by a~wrongful use of force or fear, or under color of official right.''}---consists in obtaining another's goods or services by coercion or effective threats of property rights violation. A~typical case of extortion would be the proverbial highwayman threatening a~traveler with his ‘Your money or your life' proposal or John Locke's 
%\label{ref:RNDwB8yHTPBMU}(2003, p.385 [1698, II, Chap. XVI, §176])
\parencite[p.~385\ \mbox{[1698, II, Chap.~XVI, §176]}]{locke_two_2003}\nocite{locke_two_1689} %
 robber who ``break[s] into my House, and with a~Dagger at my Throat make[s] me seal Deeds to convey my Estate to him.'' What renders extortion illegal according to the standard Austro-liberrtarian account is that the transfer of money to the highwayman is involuntary. As a~result, although the money physically travels to the highwayman, title thereto stays with the victim. In other words, victim's waiver is void or what comes to the same thing, his consent is invalid. Now the fact that the highwayman gets hold of the victim's property without having title thereto results in a~broadly construed theft, appropriation of another's property\footnote{On the crime of appropriation see George P. Fletcher 
%\label{ref:RNDjFzcnT8z5P}(2000, pp.7–22).
\parencite*[][pp.7–22]{fletcher_rethinking_2000}.%
} or what Murray Rothbard 
%\label{ref:RNDaFslS9qDov}(1998, p.77)
\parencite*[][p.77]{rothbard_ethics_1998} %
 calls an implicit theft. Since theft should be illegal on the free market, so should extortion.



The crucial step in the above argument concerns the reason for which extortive exchanges are involuntary and thus result in invalid title transfers. Generally speaking, this reason can be identified as acting under duress or coercion. It is the fact that the victim acts under duress or coercion that renders his actions involuntary and in consequence invalidates his title transfers. However, and here a~peculiarity of the Austro-liberrtarian account comes to the fore, it is not (only?) due to the fact that the extortion victim's will is overborne, fettered or somehow influenced by threats that the victim's actions are involuntary. Rather, it is (also?) a~function of the content of the extortive proposal that renders these actions involuntary. To wit, it is because the extortive proposal threatens the victim with \textit{rights violation} that makes the victim's actions involuntary. To use Richard Epstein's 
%\label{ref:RND5kqTzR0p6J}(1975, p.296)
\parencite*[][p.296]{epstein_unconscionability_1975} %
 pertinent words, the victim's actions are involuntary and his title transfer invalid because ``in the case of duress by the threat of force, B~has required A~to abandon one of his rights to protect another.'' Thus, as explained by Robert Nozick\footnote{Note, however, that Nozick cannot be classified as a~representative of the standard Austro-liberrtarian account of blackmail and this is so not only for the reason that he was not Austrian in his economic thinking. More importantly, in his discussion of blackmail, Nozick focuses mainly on the question of productivity of blackmail exchanges rather than on the question of rights. Moreover, exactly due to its unproductivity, he is quite critical of blackmail legalization.} 
%\label{ref:RNDjkXGhJnNF4}(1974, p.262),
\parencite*[][p.262]{nozick_anarchy_1974}, %
 under libertarianism:



\begin{quote}
Whether a~person's actions are voluntary depends on what it is that limits his alternatives. If facts of nature do so, the actions are voluntary. (I may voluntarily walk to someplace I~would prefer to fly to unaided.) Other peoples' actions place limits on one's available opportunities. Whether this makes one's resulting actions non-voluntary depends upon whether these others had the right to act as they did.
\end{quote}



And this view is further confirmed by Rothbard who also believes that whether an action or an exchange is free depends on the question of property rights. After all, for Rothbard freedom as such is defined in terms of property rights. As he 
%\label{ref:RNDglNd2aabKy}(2006, p.50)
\parencite*[][p.50]{rothbard_for_2006} %
 puts it, ``[f]reedom is a~condition in which a~person's ownership rights in his own body and his legitimate material property are \textit{not} invaded.'' Thus, for example, Rothbard 
%\label{ref:RNDuo4SagrLXZ}(2009, pp.182–183)
\parencite*[][pp.182–183]{rothbard_man_2009} %
 ``completely overthrows the basis for a~law of defamation'' because ``a man has no such objective property'' in his reputation. Rather, ``[h]is reputation is simply what others think of him, i.e., it is purely a~function of the subjective thoughts of others. But a~man cannot own the minds or thoughts of others. Therefore, I~cannot invade a~man's property by criticizing him publicly.''



Indeed, analyzing extortion, Walter Block and Gary M. Anderson 
%\label{ref:RND2yxPgVFm9F}(2000, p.546)
\parencite*[][p.546]{block_blackmail_2000} %
 point out that in the case of extortion ``there is no voluntary exchange'' since ``the victim's rights are violated, in that he must give up something to which he was legally entitled.'' And further they 
%\label{ref:RNDu2jLgJ1uqk}(2000, p.546)
\parencite*[][p.546]{block_blackmail_2000} %
 elaborate that ``[w]hen someone extorts money from you with the statement ‘your money or your life!' and you give up the former, you are wronged since you own both.'' Thus, for Block and Anderson 
%\label{ref:RNDAPVS2dq6nW}(2000, p.545)
\parencite*[][p.545]{block_blackmail_2000} %
 the highwayman's proposal ``would not constitute a~voluntary contract'' because regarding the threatened consequence, his ``right does not exist, since we have no right to murder other people.'' Clearly, one does not have a~right to kill, rape, maim or rob another and since it is and should be ``illegal to murder or rape, it should also be a~criminal act to threaten such acts.'' 
%\label{ref:RNDXx6n6esyNb}(Block and Anderson, 2000, p.543)
\parencite[][p.543]{block_blackmail_2000}%




Now Block and Anderson 
%\label{ref:RNDXdT6GoRXhI}(2000, p.544)
\parencite*[][p.544]{block_blackmail_2000} %
 draw a~very sharp distinction ``between blackmail and extortion, and argues that the former does, under all circumstances, represent an entirely voluntary transaction.'' Or as they 
%\label{ref:RNDTW9FyHRo1s}(2000, p.560)
\parencite*[][p.560]{block_blackmail_2000} %
 put it in slightly different terms, ``\textit{blackmail} per se, the exchange of silence for cash, is an uncomplicated voluntary act between consenting adults.'' And when they 
%\label{ref:RNDi1ti0nn07X}(2000, p.546)
\parencite*[][p.546]{block_blackmail_2000} %
 identify the fact that ``the victim's rights are violated'' as the reason for which ``there is no voluntary exchange'' in extortion, they 
%\label{ref:RND11m9wqpmlV}(2000, p.546)
\parencite*[][p.546]{block_blackmail_2000} %
 in turn point out that in the case of blackmail, ``[i]n sharp contrast, when someone threatens ‘Give me money or I~reveal your secret,' you are not wronged since you do not have title to both.'' More specifically, you do not have title to the blackmailer's forbearance to exercise his freedom of speech. After all, as pointed out by Block and Anderson 
%\label{ref:RNDEbtQt944Ua}(2000, p.546),
\parencite*[][p.546]{block_blackmail_2000}, %
 while ``extortion is the threat to do something which should be illegal (murder, rape, pillage)…, in blackmail the offer is to commit the paradigm lawful act (i.e. engage in free speech or gossip about secrets which embarrass or humiliate other people).'' Thus, ultimately, for Block 
%\label{ref:RNDjP7dzJegg1}(1998a, p.281)
\parencite*[][p.281]{block_libertarian_1998} %
 the difference between voluntary blackmail and involuntary extortion stems from the fact that although ``[i]n both cases, a~threat is made, coupled with a~demand (usually for money, but it might include sexual or other services, etc.) But in the former case, as we have seen, the threat is to do something licit; e.g., indulge in free speech. In the latter, the threat is anything but legal.''\footnote{Note, for example, that Rothbard concurs with this analysis. As he 
%\label{ref:RNDR9pph6KYMZ}(1998, p.124)
\parencite*[][p.124]{rothbard_ethics_1998} %
 points out, ``Smith has the right to ‘blackmail' Jones. As in all voluntary exchanges, both parties benefit from such an exchange. Smith receives money, and Jones obtains the service of Smith's not disseminating information about him which Jones does not wish to see others possess. The right to blackmail is deducible from the general property right in one's person and knowledge and the right to disseminate or not disseminate that knowledge.''}



Hence, the standard Austro-liberrtarian account of blackmail can be summarized in the following way. Even though it can be viewed as a~threat, a~blackmail proposal does not coerce the blackmailee to part with his goods or services. It does not coerce the blackmailee because the threat it involves is legitimate, that is, it threatens the blackmailee with something he does not have a~right against. Or from a~different angle, it threatens the blackmailee with something that the blackmailer has a~right to. Now since the blackmail proposal does not coerce the blackmailee, his parting with his goods or services is voluntary. In consequence, the blackmailee's consent or waiver or title transfer, if you will, is valid and so the blackmailer acquires not only the blackmailee's goods or services, but also the rights thereto. Accordingly, the blackmailer cannot be considered liable for theft, be it implicit, explicit or attempted. In sharp contrast, an extortive proposal involves an illegitimate threat (of something that the victim has a~right against and the offender does not have a~right to) and so coerces the victim, rendering his actions involuntary and thus invalidating his consent, waivers or title transfers. In consequence, the perpetrator of extortion acquires the victim's goods or services without having the rights thereto and so becomes liable for an implicit theft (or attempted one if his actions are not carried out to completion).



The standard Austro-liberrtarian account of blackmail can therefore be reduced to the following reasoning:



\begin{enumerate}

\item Since blackmail proposals are legitimate (they do not threaten with rights violations), they do not coerce.

\item Since blackmail proposals do not coerce, the blackmailee's actions are voluntary.

\item Since the blackmailee's actions are voluntary, the blackmailee's waivers are valid.

\item Since the blackmailee's waivers are valid, the blackmailer acquires rights to blackmailee's goods and services.

\item Since the blackmailer acquires rights to blackmailee's goods and services, blackmail is not an implicit, explicit or attempted theft.

\item Since blackmail is not a~theft (implicit, explicit or attempted), it is legitimate itself.

\end{enumerate}

We assume the truth of the first premise for the sake of discussion. We also believe that if the second premise were true, all the steps from 3 to 6 would be true as well. However, we submit that the second premise is false. Hence, it is to the second premise that we now turn.



\section{The Revisionist Austro-liberrtarian Account of Blackmail}

The standard Austro-liberrtarian account of blackmail boils down to the claim that since blackmail proposals are not extortive, that is, they do not coerce the blackmailee to part with his property or, what comes to the same thing, they do not threaten the blackmailee with rights violation so that he has to give up one of his rights, they are legitimate. This claim can be debunked in two different ways which yet in the end come to the same thing. The first approach is to argue that even if blackmail is not extortive, it is still illegitimate under a~different heading. The second approach is to submit that even if blackmail does not coerce, it still renders the blackmailee's actions involuntary via a~different route. In this section we take the first approach. In the next one, the second.



Consider Block and Anderson's 
%\label{ref:RNDPjQtihxvLc}(2000, p.546)
\parencite*[][p.546]{block_blackmail_2000} %
 aforementioned typical blackmail formula: ``Give me money or I~reveal your secret.'' Clearly, this typical formula covers almost infinite number of blackmail instances (for example, ‘Give me your money or I~reveal your affair to your wife,' ‘Give me your money or I~enter this year's music competition' etc.), so if we show that it can be illegitimate, we will show---\textit{pace} Block and Rothbard---that indeed innumerable cases of blackmail can be illegitimate as well. Now Block and Anderson 
%\label{ref:RNDxGonCBUCEN}(2000, p.546)
\parencite*[][p.546]{block_blackmail_2000} %
 believe that it is legitimate to make a~proposal of this type because ``you are not wronged since you do not have title to both'' and it is only an ``offer to commit the paradigm lawful act (i.e. engage in free speech or gossip about secrets which embarrass or humiliate other people).'' In other words, it is legitimate because it is not extortive, where ``extortion is the threat to do something which should be illegal (murder, rape, pillage).''



However, assume that the blackmailer does not want to ``reveal your secret.'' The only thing he wants, quite typically as it seems, is your money. So, he leverages the fact that he knows your secret which you do not want to be revealed to induce you to pay---similarly to Nozick's 
%\label{ref:RNDJFK7EyBV16}(1974, pp.84–85)
\parencite*[][pp.84–85]{nozick_anarchy_1974} %
 architectonic monstrosity case in which the blackmailer ``has no desire to erect the structure on the land; he formulates his plan and informs you of it solely in order to sell you his abstention from it.'' In such cases, nothing changes as far as the extortion/blackmail distinction is concerned, for the blackmailer still proposes to commit what Block and Anderson 
%\label{ref:RNDZRNh01V6a4}(2000, p.546)
\parencite*[][p.546]{block_blackmail_2000} %
 call ``the paradigm lawful act'' of engaging in free speech (and free speech clearly comprises speaking as well as abstaining from speaking) or building on his own land (and private property rights to land equally clearly comprise rights to build as well as to abstain from building on the land). And indeed, Block and David Gordon 
%\label{ref:RNDqXAF8bXe26}(1985, p.49)
\parencite*[][p.49]{block_blackmail_1985} %
 admit that ``[i]t is difficult to see… why ‘unproductive' exchanges, in this sense, ought to be prohibited or singled out for special regulations.'' Alas, there is a~pretty straightforward Austro-liberrtarian reason why they ought to.



For note that in such cases the blackmailer deceives the blackmailee about his intentions. Even though he proposes to reveal a~secret or to build a~monstrosity, he does not intend to do so. He intentionally misrepresents crucial facts about his plans, purposes or, if you will, mental states (desires and intentions to reveal a~secret etc.) in order to deprive the blackmailee of his money and to acquire it himself. This is already an attempted fraud. And if the blackmailer successfully induces by such an intentional misrepresentation the blackmailee to part with his property, it is a~completed crime of fraud, period. There is (1) the \textit{actus reus} of fraud in the shape of the blackmailer making a~false representation of the blackmailer's mental state, thereby deceiving the blackmailee about the said mental state and inducing or causing him to part with his property, accompanied by (2) the \textit{mens rea} of fraud in the shape of making the false representation \textit{knowingly} (re its falsehood) while \textit{intending} to deceive the blackmailee and to deprive him of his property.\footnote{Compare, for example, 
%\label{ref:RNDw0ylt9eLIE}(Regina v. Théroux, [1993])
\parencite[][]{noauthor_regina_1993} %
 2 S.C.R. 5, where the court identifies the \textit{actus reus} of fraud as an act of ``deceit, falsehood, or some other dishonest act'' which ``consequence is depriving another of what is or should be his'' while the accompanying ``\textit{mens rea} would then consist in the subjective awareness that one was undertaking a~prohibited act (the deceit, falsehood or other dishonest act) which could cause deprivation in the sense of depriving another of property or putting that property at risk.~If this is shown, the crime is complete.'' Even more pertinently, compare the 
%\label{ref:RNDDkVV8diKx7}(Fraud Act 2006)
\parencite[][]{noauthor_fraud_nodate} %
 of the Parliament of the United Kingdom, section 2 (Fraud by false representation):
s\begin{enumerate}
\item A~person is in breach of this section if he---
\begin{enumerate}
\item dishonestly makes a~false representation, and
\item intends, by making the representation---
\begin{enumerate}
\item to make a~gain for himself or another, or
\item to cause loss to another or to expose another to a~risk of loss.
\end{enumerate}
\end{enumerate}
\item A~representation is false if---
\begin{enumerate}
\item it is untrue or misleading, and
\item the person making it knows that it is, or might be, untrue or misleading.
\end{enumerate}
\item ``Representation'' means any representation as to fact or law, including a~representation as to the state of mind of---
\begin{enumerate}
\item the person making the representation, or
\item any other person.
\end{enumerate}
\end{enumerate}
} Hence, we can conclude, in a~nutshell, that any blackmailer (such as, for example, Nozick's monstrosity builder) who says anything that falls under Block and Anderson's 
%\label{ref:RNDwL0ap4dqCo}(2000, p.546)
\parencite*[][p.546]{block_blackmail_2000} %
 generic formula ``Give me money or I~reveal your secret'' while not being keen on executing his threats, commits a~crime of fraud.\footnote{An anonymous referee of this journal put some strain on our present argument by inviting us to consider a~scenario in which a~car dealer makes the following proposal to his potential customer: ``Pay me \$20,000 and I~will give you a~car. Don't pay me \$20,000 and I~will not give you the car.'' The customer decides to not pay the \$20,000, but the car dealer gives him the car anyway (and planned to do so regardless if the customer paid him or not). Is this an attempted fraud since the car dealer gave the car when he said he would not do so? And further: Should the customer be free to accept the car? Should the car dealer not be free to give the car for free since he said he would not? By the same token, should the blackmailer who never intended to reveal secrets be punished? Should the blackmailee be free to accept the silence for free? Should the blackmailer not be free to not reveal secrets for free since he said he would not? Now this ingenious thought experiment of the referee is supposed to provide a~\textit{reductio ad absurdum} of our argument, for we have a~strong intuition that the car dealer does not do anything wrong. However, if he commits no wrong, neither does the blackmailer and our argument is debunked. One response to this challenge is to point out that in normal circumstances the customer is not caused to pay \$20,000 for the car simply by the car dealer saying that otherwise he will not give it to the customer and thus there is no \textit{actus reus} of fraud. But since the customer is not normally caused to part with his \$20,000 by the car dealer simply saying this, then saying this does not seem to constitute coming to a~dangerous proximity of causing such a~deprivation and so is not sufficient for the \textit{actus reus} of attempted fraud either. Once this is established, answers to other questions follow quite straightforwardly.}



Now Austro-liberrtarians are as much against fraud as they are against extortion, for they both are kinds of implicit (attempted or completed) theft. Thus, for example, Rothbard 
%\label{ref:RNDpog49veGns}(1998, p.77)
\parencite*[][p.77]{rothbard_ethics_1998} %
 argues that invasion of private property ``may include two corollaries to actual physical aggression: \textit{intimidation}, or a~direct threat of physical violence; and \textit{fraud}, which involves the appropriation of someone else's property without his consent, and is therefore ‘implicit theft'.'' And further he asks: ``Under our proposed theory, would fraud be actionable at law? Yes, because fraud is failure to fulfill a~voluntarily agreed upon transfer of property, and is therefore implicit theft.'' 
%\label{ref:RNDdbbgVxxDkt}(Rothbard, 1998, p.143)
\parencite[][p.143]{rothbard_ethics_1998} %
 Also in other places Rothbard expresses a~similar view about fraud, for he 
%\label{ref:RNDAYaE8wgbTP}(2009, p.803)
\parencite*[][p.803]{rothbard_man_2009} %
 believes that ``[t]he purely free market is, by definition, one where theft and fraud (implicit theft) are illegal and do not exist.'' After all, as he 
%\label{ref:RNDz1YiDhftrP}(Rothbard, 2011, p.216)
\parencite[][p.216]{rothbard_toward_2011} %
 explains, ``a ‘free market' necessarily implies total respect for and protection of private property… This implies not only a~cracking down on assault and murder, but also on all forms of theft and fraud.'' Thus, Rothbard 
%\label{ref:RNDjDEEgcxCzp}(2009, p.184)
\parencite*[][p.184]{rothbard_man_2009} %
 contends that we should ``exclude both explicit violence and the implicit violence of fraud from our definition of the free market---the pattern of voluntary interpersonal exchanges.''



By the same token, Block 
%\label{ref:RNDlLcBMgbD1O}(1998a, p.294)
\parencite*[][p.294]{block_libertarian_1998} %
 also claims that ``fraud is equivalent to theft'' and to this effect references the above quoted passages from Rothbard 
%\label{ref:RNDV2NpI9vXyu}(1998, pp.77–78).
\parencite*[][pp.77–78]{rothbard_ethics_1998}. %
 According to Block, this is so both in special cases of fraud such as, for example, counterfeiting or false witness, and in fraud \textit{per se}, regardless of its specific subject-matter. Thus, when he 
%\label{ref:RNDBWAuFWZnWU}(2015, p.38)
\parencite*[][p.38]{block_natural_2015} %
 probes the relation between the Ten Commandments and the libertarian Non-Aggression Principle (NAP), Block intimates that the biblical prohibition of bearing false witness could also find its place in the libertarian penal code \textit{qua} fraud prohibition. As he 
%\label{ref:RNDkVl6MVnksO}(2015, p.38)
\parencite*[][p.38]{block_natural_2015} %
 puts it, ``[m]urder, stealing, and false witness (fraud) are explicitly prohibited by libertarian law.'' In turn writing about counterfeiting, he 
%\label{ref:RND10eCuCDrbl}(2018, p.99)
\parencite*[][p.99]{block_defending_2018} %
 submits that ``counterfeiting is a~special case of fraud… This special case of fraud constitutes theft, just as fraud in general does.'' Now of course this anti-fraud stance stems from Block's 
%\label{ref:RNDMNzFGU0omm}(2004, p.275)
\parencite*[][p.275]{block_libertarianism_2004} %
 belief that ``libertarianism is a~deontological theory of law… [where] [p]roper legal enactments are these that support this basic premise (e.g. prohibitions of murder, rape, theft, fraud, etc.)''. Or as he 
%\label{ref:RNDSf4xIsahP8}(1998b, p.1889)
\parencite*[][p.1889]{block_environmentalism_1998} %
 explains it in a~different place, in ``libertarianism… the only improper human activity is the initiation of threat or force against another or his property'' while ``[t]o prevent murder, theft, rape, trespass, fraud, arson, etc., and all other such invasions is the only proper function of legal enactments.''



It is therefore clear that Austro-liberrtarians believe that fraud should be illegal. However, when juxtaposed with the above analysis of blackmail, this belief puts them in the following predicament. The blackmailer who does not intend to execute his otherwise legal threats and is only after the blackmailee's money does not commit extortion, but he does commit fraud. Since fraud is illegal under libertarian law, so should any blackmail that is perpetrated without intention to execute its otherwise legal threats. To put it as transparently as possible:


\medskip

\noindent P\textsubscript{1}: Fraud is illegal under libertarian law.



\noindent P\textsubscript{2}: Blackmail (without intention to execute its threats) is fraud.



\noindent C: Blackmail (without intention to execute its threats) is illegal under libertarian law.

\medskip

Yet, Austro-liberrtarians, including the most prominent ones, that is, Rothbard and Block, want to ``legalize blackmail.'' 
%\label{ref:RNDkv37M7UvWp}(see Block, 2013)
\parencite[see][]{block_legalize_2013} %
 This position, as far as it pertains to blackmail without intention to execute its threats, clearly fails to account for the possibility of blackmail being fraud and so to cohere with their own stance on fraud. That they do not see it can only be explained by what Judith Jarvis Thomson 
%\label{ref:RNDdPAoLHVhWW}(1990, pp.25–33)
\parencite*[][pp.25–33]{thomson_realm_1990} %
 called ``failing to connect.''\footnote{Thomson quotes here Edward Morgan Forster 
%\label{ref:RNDflGmxLKR1K}(1941)
\parencite*[][]{forster_howards_1941} %
 as the author of the term.} Thus, once presented with the proper connection between blackmail and fraud, they should withdraw their support for the legalization of (this sort of) blackmail, for the legalization of fraud as such would have much more profound and far-reaching consequences for the libertarian theory of justice than opposing legalization of blackmail and since there is no third way, they should oppose legalization of blackmail. Hence, they should embrace what we called a~revisionist Austro-liberrtarian account of blackmail. It is revisionist because it proposes that (a) blackmail without intention to execute its threats should be illegal and that (b) this sort of blackmail is better subsumed under fraud than extortion. It is nonetheless Austro-liberrtarian because it acknowledges that (c) blackmail proposals do not coerce and that (d) fraud should be illegal. It basically connects these dots, as the standard Austro-liberrtarian account fails to do.



\section{The Logic of the Austro-liberrtarian Ban on Fraud}

Now let us turn to the second way of debunking the standard Austro-liberrtarian account of blackmail, that is, to the claim that even if blackmail does not coerce, it still renders the blackmailee's actions involuntary via a~different route. In other words, let us try to demonstrate that from the fact (assumed for the sake of discussion) that the blackmailer does not (due to the legitimate nature of his threats) coerce the blackmailee, it does not follow that the blackmailee's actions are voluntary.



As pointed out by Michael S. Moore 
%\label{ref:RNDSpzIWMKdwv}(1984, p.85),
\parencite*[][p.85]{moore_law_1984}, %
 beginning with Aristotle's \textit{Nicomachean Ethics}, human actions have always been deemed ``involuntary when they are performed (\textit{a}) under compulsion, (\textit{b}) as the result of ignorance.'' 
%\label{ref:RND187ZTKVeln}(Aristotle, 1955, p.77 [Book III, Chap. I, 1110a]).
\parencite[p.~77\ \mbox{[Book III, Chap.~I, 1110a]}]{aristotle_nicomachean_1955}. %
 Compulsion assumes either a~form of necessity when a~natural threat of, say, a~sudden tempest compels a~captain to jettison cargo in order to save the ship or a~form of duress or coercion, if you will, when a~human threat of, for example, death compels a~man to hand his money to a~robber. One peculiarity of Austro-libertarianism is that it rejects the claim that necessity compels in a~way that can justify or excuse property rights violations or invalidate title transfers. Another peculiarity of Austro-libertarianism is that it believes that only illegal threats, that is, proposals of rights violations, compel in a~way that can invalidate consent (although it is not clear whether such illegal threats can also excuse or justify violations of the third party's property rights). As we saw above, it is ultimately for this reason that Austro-libertarianism contends that only extortion should be prohibited whereas blackmail should be legal.



However, as we also saw above, Austro-liberrtarians believe that fraud should be illegal despite the fact that there is no illegal threat involved in it. For instance, Rothbard 
%\label{ref:RNDVNkJ4EFOvN}(1998, p.77)
\parencite*[][p.77]{rothbard_ethics_1998} %
 explicitly distinguishes a~``threat of physical violence; and \textit{fraud}, which involves the appropriation of someone else's property without his consent'' and Block 
%\label{ref:RND2qyymZt9jb}(2015, p.38)
\parencite*[][p.38]{block_natural_2015} %
 links the biblical prohibition of bearing false witness with the libertarian prohibition of fraud. Thus, it stands to reason to say that fraud must affect consent in some other way than via threat or coercion and that this way has something to do with the falsehood of the representation made by the offender. Indeed, as pointed out by Hillel Steiner 
%\label{ref:RNDmcS2x9KCmI}(2019, p.100),
\parencite*[][p.100]{steiner_asymmetric_2019}, %
 it is most natural for libertarians to try to oppose fraud by taking the second Aristotelian route, that is, the route of ignorance or mistake. As Steiner 
%\label{ref:RND5zIYzBqMnb}(2019, p.100)
\parencite*[][p.100]{steiner_asymmetric_2019} %
 puts it, for an exchange to be valid, there must be a~title transfer between the parties and ``[f]or that waiver-generated transfer to be normatively valid---for the waiver to effect the transfer of the right in question---it is necessary that it be done \textit{voluntarily}.'' Since coercion is here beside the point, it is therefore sufficient for the preservation of this voluntariness condition that the transferee, to put it in Steiner's 
%\label{ref:RNDq0NEK35t7s}(2019, p.100)
\parencite*[][p.100]{steiner_asymmetric_2019} %
 own words, ``is not falsely informed, or what I'll simply call \textit{ignorant}…. The buyer's waiver, to be normally valid, must also be performed non-ignorantly. And the duplicity of the fraudulent seller is held to defeat that condition.'' Hence, it is the ignorance of the defrauded party that accounts for the fact that his consent is invalid or as Rothbard 
%\label{ref:RND3wj8xukl2f}(1998, p.77)
\parencite*[][p.77]{rothbard_ethics_1998} %
 puts it, that ``\textit{fraud}… involves the appropriation of someone else's property without his consent.''



Certainly, if the defrauded party knew that, for example, a~car he was buying was a~lemon, he would not have bought it. It is only because he thought that the car is in good condition that he decided to purchase it. Unfortunately, he was deceived and thus ignorant about the crucial fact, that is, the car's poor condition. Accordingly, he did not know what he was really buying. He thought he was purchasing a~good car while what he was getting was a~lemon. Therefore, if he consented to anything at all, it was to exchange his money for a~different car than the one he actually got. For the latter, he did not consent to pay. Hence, now the other party has his money without his consent. This is an implicit theft, for although the offender did not take the money himself, it was handed to him without the title travelling therewith and so he now has the money without any rights thereto.



By the same token, if the blackmailee knew that the blackmailer had no intention to reveal his secrets, he could have decided not to pay him. It is only because he thought that the blackmailer would reveal his secrets that he chose to pay him. In a~sense, the blackmailee paid for what he already had. If he knew that he was paying for what he already had, he most likely would not have paid for it. Or still in other words, it was a~crucial fact for the blackmailee that the blackmailer was willing to reveal his secrets. As it turned out, he was deceived and so mistaken about this crucial fact. Thus, he did not know what he was paying for. He thought he was paying for \textit{x} while what he was getting was \textit{y}. Hence, if he voluntarily and validly consented to anything at all, it was to paying for \textit{x}, not for \textit{y}. He decidedly did not consent to exchange his money for \textit{y}. For this purpose, his waiver-generated title transfer was invalid. Accordingly, the blackmailer got the blackmailee's money without his consent. This is an implicit theft, for even though the offender did not take the money himself, it was transferred to him without proper waiver-generated title transfer and so he now enjoys the money without having any title thereto.



We can therefore see that it is not true---\textit{contra} what the standard Austro-liberrtarian account of blackmail claims in the second step of its case for the legalization of blackmail---that:
\begin{enumerate}[label=(\arabic*), start=2]
\item Since blackmail proposals do not coerce, the blackmailee's actions are voluntary.
\end{enumerate}
It is not true because even though the blackmailee is indeed (by assumption) not coerced, his actions are nonetheless involuntary due to his---induced by the blackmailer---mistake or ignorance. In consequence, nothing that follows from (2) can be true either. Thus, it is likewise not the case that:
\begin{enumerate}[label=(\arabic*), start=3]
\item Since the blackmailee's actions are voluntary, the blackmailee's waivers are valid.
\item Since the blackmailee's waivers are valid, the blackmailer acquires rights to blackmailee's goods and services.
\item Since the blackmailer acquires rights to blackmailee's goods and services, blackmail is not an implicit, explicit or attempted theft.
\item Since blackmail is not a~theft (implicit, explicit or attempted), it is legitimate itself.
\end{enumerate}
No, blackmail (without intention to execute its threats) is fraud and as such it is neither legitimate nor should it be legalized.



\section{Conclusions}

In the present paper we examined the standard Austro-liberrtarian account of blackmail. According to this account, blackmail should be legal because the blackmailer's threat---in contradistinction to the extortionist's threat---is in itself legal and so does not coerce the blackmailee. In consequence, the blackmailee's property transfer is voluntary and valid and the blackmailer does not commit any theft by acquiring it. Against this account we argued that even if the blackmailer's threat does not coerce, it does not follow that the blackmailee's property transfer is voluntary and valid. Or in other words, even if blackmail cannot be subsumed under extortion, it does not follow that it cannot be subsumed under some other crime. Indeed, as we demonstrated, in the case of blackmail which is not accompanied by the blackmailer's intention to execute his threats, the blackmailee is deceived by the blackmailer about the latter's mental state and thus ignorant about the crucial fact regarding the service he is buying. Accordingly, even though the blackmailee is not coerced to pay, his title transfer is involuntary and invalid due to ignorance. Likewise and for the same reason, blackmail which is not accompanied by the blackmailer's intention to execute his threats can be viewed as the one in which the blackmailer intentionally deceives the blackmailee about his mental state and induces him by this intentional misrepresentation to part with his property. This is fraud. Thus, even if blackmail cannot be subsumed under extortion, it can nonetheless be subsumed under fraud. As such, it is illegitimate even by Austro-liberrtarians' own lights and so should not be legalized.



\end{artengenv}

