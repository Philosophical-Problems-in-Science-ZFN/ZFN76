\begin{artengenv}{Norbert Slenzok}
	{Monarchy as private property government. A~chiefly methodological critique}
	{Monarchy as private property government\ldots}
	{Monarchy as private property government. A~chiefly methodological critique}
	{University of Zielona Góra}
	{Hans-Hermann Hoppe famously argued that monarchy is superior to democracy insofar as property rights protection is concerned. The present paper calls this claim into question, with much of the heavy lifting being done by methodological ponderings. More specifically, it is demonstrated that instead of \textit{a~priori}, praxeological truths, Hoppe's monarchy theory offers an ideal type of the politician bestowed with an inheritable title to the throne. Against this background, the ideal type in question is shown to be faulty in that it treats monarchs as capitalist landowners of sorts, thereby overlooking strictly political incentives they face, which can predictably push them in directions inimical to free markets.
	}
	{praxeology, sociology, Austrian methodology, time preference, ideal type, apriorism, Hans-Hermann Hoppe}








\section{Introduction}

\lettrine[loversize=0.13,lines=2,lraise=-0.03,nindent=0em,findent=0.2pt]%
{T}{}he topic of this paper is the well-known economic analysis of monarchy laid out by the anarcho-capitalist philosopher and economist Hans-Hermann Hoppe in his best-selling book \textit{Democracy---the God that Failed} 
%\label{ref:RNDSKHqmsW2lh}(2007a).
\parencite*[][]{hoppe_democracy_2007}. %
 In a~nutshell, Hoppe argued that of two opposite political systems, monarchy and democracy, the former is superior to the latter with respect to property rights protection. Hence, although any state is economically unviable (and ethically repugnant), ``\textit{if} one must have a~state, defined as an agency that exercises a~compulsory territorial monopoly of ultimate decision-making (jurisdiction) and of taxation, then it is economically and ethically advantageous to choose monarchy over democracy 
%\label{ref:RNDzBrpcHBu5r}(2007a, p.xx).
\parencite*[][p.xx]{hoppe_democracy_2007}.%
'' This is said to be the case by virtue of an elementary distinction: monarchy, so the argument goes, represents ``private government ownership,'' while democracy amounts to ``public government ownership.'' Now, given that private property doubtlessly fosters efficient management, and that honoring property rights is a~crucial element of managing a~country efficiently, it is monarchy that can be expected to fare better in protecting (or at least not violating) property rights. The key variable in this regard is time preference: private ownership lowers it, while public ownership results in its increase. It is precisely thanks to the longer planning horizon that monarchs are more likely to make good managers of their countries. All this, holds Hoppe 
%\label{ref:RNDnu3HmUhuSz}(2007a, p.xix),
\parencite*[][p.xix]{hoppe_democracy_2007}, %
 can be demonstrated ``in accordance with elementary theoretical insights regarding the nature of private property and ownership versus ‘public' property and administration.'' Put differently, our author seems to consider his contentions a~simple application of Austrian economics to the workings of political systems 
%\label{ref:RNDSjJTqy3qXv}(Hoppe, 2007a, p.xxii).
\parencite[][p.xxii]{hoppe_democracy_2007}.%




The task of the present inquiry is to furnish a~critical account of the above argument. Specifically, while taking no issue with Hoppe's staunchly critical position on democracy, this paper contends that his relative appreciation of monarchy is undue. This shall be proven largely on the methodological plain. First, it will be shown that Hoppe's monarchy theory is best interpreted as a~sociological exercise in Weberian ideal-typological modeling rather than a~rendition of pure praxeology.\footnote{To avoid any misunderstandings, let us note that nowhere does Hoppe explicitly ascribe to his theses on democracy and monarchy the status of \textit{a~priori} propositions. Yet, as Gordon 
%\label{ref:RNDATU2AFZULJ}(2017, pp.98–99)
\parencite*[][pp.98–99]{gordon_austro-libertarian_2017} %
 points out, such an impression is created by a~lengthy defense of \textit{a~priori} knowledge in the introduction to \textit{Democracy…} 
%\label{ref:RNDrLQ6TgC1Kd}(Hoppe, 2007a, pp.xv–xix),
\parencite[][pp.xv–xix]{hoppe_democracy_2007}, %
 followed immediately by the exposition of the book's central claims 
%\label{ref:RNDZs20OF8ReO}(Hoppe, 2007a, pp.xix–xxi).
\parencite[][pp.xix–xxi]{hoppe_democracy_2007}. %
 All in all, if a~method is not brought to bear in a~study, then why defend it in the introduction? Nonetheless, elsewhere Hoppe 
%\label{ref:RNDp5HCUuNUyS}(2006, p.33; 2015, p.16)
\parencites*[][p.33]{hoppe_economics_2006}[][p.16]{hoppe_short_2015} %
 correctly asserts that, e.g., investigations into the nature and the growth of the state or the essence of class struggle---incontrovertibly akin to the investigations offered in \textit{Democracy…}, let us add---properly belong in the field of sociology, which lacks the apodictic validity that characterizes purely praxeological judgments (see section 3 of the present paper). More important than Hoppe's own intentions, however, is the inherent logic of his theory. And as will be shown, the entire case for monarchy that Hoppe is making \textit{presupposes} treating monarchy as an instantiation of a~catallactic, i.e., \textit{a~priori} category (more on this in section 4). } Second, it will be demonstrated how Hoppe's ideal type of monarchy conflates the catallactic (praxeological) categories of capitalist and landowner with the political (sociological) one of the ruler with a~permanent title to his office. This, in turn, will allow for identifying some particular limitations of Hoppe's monarchy theory insofar as its very substance is concerned. In effect, the model will be presented as unjustifiably one-sided, albeit not entirely faulty. In other words, it is contended here that while the Hoppean perspective succeeds in elucidating the commendable facets of the monarchical system, it simultaneously fails to capture the no less significant undesirable ones. Hence, the question of which form of government, monarchy or democracy, is preferable in terms of property rights protection remains undecided.



Surprisingly enough, little attention has thus far been paid to Hoppe's claims in the literature. Moreover, some critics 
%\label{ref:RNDOKEJo3x4cd}(Sierpiński, 2016)
\parencite[][]{sierpinski_critica_2016} %
 and sympathizers 
%\label{ref:RNDcJ7YM5uCSA}(DiLorenzo, 2009)
\parencite[][]{hulsmann_hoppean_2009} %
 alike focused exclusively on the substance of the theory at hand, not on its method.\footnote{However, DiLorenzo 
%\label{ref:RNDseVaR4NquV}(2009, p.274)
\parencite*[][p.274]{hulsmann_hoppean_2009} %
 tellingly asserts that Hoppe ``merely applies logic and economic reasoning to a~comparison of monarchy […].'' As will be demonstrated, this is not quite the case, since in order for Hoppe's monarchy theory to proceed as it in fact does, auxiliary---and highly contentious---motivational assumptions are needed. } Other commentators simply took Hoppe's declarations of apriorism at face value 
%\label{ref:RNDhopghVo5qJ}(Crovelli, 2007, p.116; Gabiś, 2005; Machaj, 2009, pp.113–114).
\parencites[][p.116]{crovelli_toward_2007}[][]{gabis_hans-hermann_2005}[][pp.113–114]{kalita_krytyka_2009}. %
 Doubts have been raised by David Gordon 
%\label{ref:RNDzMTNN1LIpO}(2017, p.99),
\parencite*[][p.99]{gordon_austro-libertarian_2017}, %
 who in his otherwise highly favorable review of Democracy… queried whether considerations presented in the book can really be regarded as \textit{a~priori} truths of praxeology. He nonetheless did not develop his doubts into a~full-fledged criticism. The proposition to view Hoppe's analyses as ideal type-based sociology has in turn been hinted at by Gerald Radnitzki 
%\label{ref:RNDqcslX7aAiQ}(2003, p.161),
\parencite*[][p.161]{hoppe_is_2003}, %
 yet he did not provide an in-depth treatment of the problem either. Another author paying attention to the methodological issues is Paweł Nowakowski 
%\label{ref:RNDisNCjegDEI}(2010, p.273),
\parencite*[][p.273]{nowakowski_dlaczego_2010}, %
 who astutely noticed that the comparison of monarchy and democracy transcends praxeology by attributing definite motives to the rulers. Still, he did not go on to show how this impinges on the validity mode \textit{(a priori} or \textit{a~posteriori}) of Hoppe's theory. Finally, Walter Block, William Barnett II, and Joseph Salerno 
%\label{ref:RNDqmoWMoxQLv}(2006)
\parencite*[][]{block_relationship_2006} %
 argued \textit{pace} Rothbard and Hoppe that time preference decreasing with the increment in wealth or income is not an apodictic law but rather an empirical generalization. Here, I~extend their criticism to Hoppe's thesis that time preference is also necessarily lowered by private ownership. In a~word, the present essay elaborates on the insights of all the mentioned authors and complements the existing literature by demonstrating how a~flawed application of the Austrian social science methodology derailed Hoppe's case for monarchy by imparting to it unnecessary lopsidedness.



The paper represents an immanent critique of Hoppe's theory. That is, the purpose is to beat it on its own (stated) methodological ground. Hence, all distinct features of Austrian school methodology in the Misesian tradition, particularly the account of praxeology as an \textit{a~priori} social science 
%\label{ref:RNDUX9sYdLXWN}(see Mises, 1962; 1998; 2007),
\parencites[see][]{mises_ultimate_1962}[][]{mises_human_1998}[][]{mises_theory_2007}, %
 are accepted here for the sake of argument. The method of the study is so-called rational reconstruction. That is to say, the burden of the article lies in exploring the internal logic and inconsistencies of Hoppe's views rather than in hermeneutic-interpretative work 
%\label{ref:RND88ZCD0XLwH}(Linsbichler, 2017).
\parencite[][]{linsbichler_was_2017}.%




The paper proceeds in the following order: section 2 succinctly recounts Hoppe's defense of monarchy. Section 3 supplies arguments for reading that theory as predicated upon a~sociological ideal type rather than on a~set of \textit{a~priori} propositions, with an eye on the question of time preference. Section 4, in turn, submits that the original error of Hoppe's analysis consists in blurring the distinction between ideal types and catallactic functions and the confusion of political means, characteristic of monarchy as a \textit{political} system, and \textit{economic} means, epitomized by undertakings of capitalists or landowners. In section 5, this fallacy is shown to result in further problems. To remedy these, several amendments to Hoppe's ideal type of monarch are proposed. They are intended to help explain certain widely known facts (e.g., the suppression of free speech in absolute monarchies or the incidence of war between them at least equaling that of democratic regimes) left on the cutting-room floor in Hoppe.\footnote{This essay is a~theoretical exercise, so the factual references it brings up are, due to space limitations, rather illustrative than exhaustive. Hopefully, future researchers will test the usefulness of my points in explaining more comprehensive sets of historical data. } The last section concludes.



\section{The Hoppean rehabilitation of monarchy: a~brief reconstruction}

Hoppe's typology of political systems is based on the criterion of ownership rights in the state. Thus, there are only two basic forms of government: monarchy and democracy.\footnote{Stated more precisely, as pure types, democracy and monarchy represent two extremes of a~continuum that comprises various intermediate forms. Medieval feudal monarchies do not fall within the theory's scope at all, since in the absence of sovereign powers on the monarch's part, they represent a~form of pre-state aristocracy, with the king acting as a \textit{primus inter pares}. Instead, pure monarchy is exemplified most fully by European absolute monarchies of the XVII and XVIII century. Classic constitutional monarchies such as those of 1791 and 1830 in France, in turn, whereby sovereignty was divided between the monarch and the people, are situated in the middle of the scale. Contemporary parliamentary monarchies, where the monarch's standing is largely ceremonial, are monarchies in name only, actually constituting democracies 
%\label{ref:RNDNEov7GE1w7}(Hoppe, 2007a, p.18, f.19; 2015, pp.108–112).
\parencites[][f.19]{hoppe_democracy_2007}[][pp.108–112]{hoppe_short_2015}. %
 Consequently, the totalitarian dictatorships of communism, Nazism, or fascism are classified as likewise democratic. For neither Hitler or Stalin, nor other such leaders (perhaps except the Kim family) are considered private owners of their governments. Rather, they are at the helm of mass democratic parties, and the ideologies that provide legitimation for their claims to power present them as mere agents of the \textit{Volk}, the revolutionary proletariat, or some other large group of people thought of as the sovereign 
%\label{ref:RNDiIEqquC4dw}(Hoppe, 1987, p.179).
\parencite[][p.179]{hoppe_eigentum_1987}. %
 } In the former system, the ruler (a king or a~prince) is conceived of as the private owner of a~state. In democracy, on the other hand, rulers are merely temporary caretakers of the government, which---in accordance with the popular sovereignty doctrine---represents public property. The kernel of the Hoppean argument for monarchy boils down to the superiority of private over public ownership. The most significant facet here is that private property fosters low time-preference. Private proprietors are---\textit{ceteris paribus}---more willing to operate in a~far-sighted manner, for it is they who will reap future benefits. By contrast, public property is invariably characterized by the following defect: every user of a~common good is incentivized to exploit it shortsightedly lest others consume it before him. Thus, a~monarch, viewing himself and his successors as private owners of the government, will consider his realm a~capital good the productive output of which shall serve him until the rest of his days, later to be passed down to future generations of the dynasty. On the other hand, democratic politicians, given the temporal constraints of their term, will see the government they control solely as a~consumption good. Hence, they will be inclined to relentlessly exploit it regardless of long-term repercussions. Put differently, while the monarch possesses the title to the capital value of the country, democratic caretakers are entitled exclusively to the current use of it 
%\label{ref:RNDEabMfzH6Ns}(Hoppe, 2007a, pp.45–46).
\parencite[][pp.45–46]{hoppe_democracy_2007}. %
 As a~result, democracy ``promotes capital consumption'' 
%\label{ref:RNDAvIH1jGAlZ}(Hoppe, 2015, p.119).
\parencite[][p.119]{hoppe_short_2015}. %
 ``A private government owner will tend to have a~systematically longer planning horizon, i.e., his degree of time preference will be lower, and accordingly, his degree of economic exploitation will tend to be less than that of a~government caretaker''---concludes Hoppe 
%\label{ref:RNDIemBnzFu9U}(2007a, p.46).
\parencite*[][p.46]{hoppe_democracy_2007}.%




The ramifications of this systemic difference are far-reaching. Firstly, taking into account the capital value that his realm presents to him and his descendants, the monarch will refrain from pursuing policies that have detrimental impact on the economy: high taxation, overregulation, or indebtedness. Secondly, the class of tax-consumers will be comparatively small, limited to the reigning dynasty and a~narrow circle of state apparatus members. Thirdly, private ownership in government will induce the ruler to abide by private law. Although he will indeed embark on production and expansion of legislation, he will remain more of an arbitrator than a~law-maker. Fourthly, the impact of the monarchical system on the incidence and conduct of war will be moderating. Since armed conflicts will take place at the ruler's own expense and in order to appropriate new territories to his own benefit, he will be motivated to keep wars short and restrained. Moreover, the monarch can also try to acquire new lands in a~peaceful manner, by means of arranged marriages or trade, thereby avoiding unnecessary belligerence 
%\label{ref:RNDV1r379NWXi}(Hoppe, 2007a, pp.19–23, 46–50).
\parencite[][pp.19–23, 46–50]{hoppe_democracy_2007}. %
 All this is supposed to explain why the rapid aggrandizement of the welfare-warfare state in the XX century coincides with the downfall of western monarchies and the subsequent shift to democracy 
%\label{ref:RNDNEXTR2WtMT}(Hoppe, 2007a, pp.50–74).
\parencite[][pp.50–74]{hoppe_democracy_2007}.%




\section{Aprioristic praxeology or a~sociological ideal type?}

As has been mentioned in the introduction, Hoppe seems to imply that his analysis belongs in the domain of praxeology---a system of aprioristic, i.e. apodictically, non-experientially valid claims. It is, however, doubtful whether judgments like ``monarchies conduct wars in a~less destructive manner than democracies'' or ``kings and princes exploit less than presidents and prime ministers'' could actually be considered on a~par with classic---and invoked by Hoppe 
%\label{ref:RNDDxCqYQWAiE}(2007a, p.xvi)
\parencite*[][p.xvi]{hoppe_democracy_2007} %
 himself---instances of the synthetic \textit{a~priori} such as ``No two objects can occupy the same space'' or ``Whatever object is colored is also extended,'' or even the Misesian ``Man acts.'' As regards the latter proposition, it is also difficult to see how they could be deduced from it. Gordon 
%\label{ref:RNDAjnMnTSdk4}(2017, p.99)
\parencite*[][p.99]{gordon_austro-libertarian_2017} %
 suggests, then, that Hoppe's analysis should be construed as logically related to certain apodictic propositions (e.g., those describing the impact of taxation on the economy) rather than as constituting such judgments themselves. It is worth noting that hints to this effect can also be found in Hoppe, though scattered in texts other than \textit{Democracy}… For example, when explaining the method of his historical reconstructions from \textit{Short History of Man}, Hoppe 
%\label{ref:RNDfKC7Mqzmef}(2015, p.16)
\parencite*[][p.16]{hoppe_short_2015} %
 states: ``The events in human history that I~want to explain are not necessary and predetermined, but \textit{contingent empirical} events, and my studies then are not exercises in economic or libertarian theory.'' In the same vein, while exploring the nature of the state as an ``expropriating property protector'', whose members take advantage of the apparatus of power to satisfy their own power and money lust, Hoppe 
%\label{ref:RNDuEcXHJWHAz}(2006, p.33)
\parencite*[][p.33]{hoppe_economics_2006} %
 points out:



Why is there taxation; and why is there always more of it? Answering such questions is not the task of economic theory but of praxeologically informed and constrained sociological or historical interpretations and reconstructions, and from the very outset much more room for speculation in this field of intellectual inquiry exists.



Note that there is one thing all those fields---historical reconstructions of momentous events such as the industrial revolution, theory of the state, and class analysis---have in common. To wit, they attribute to agents certain explicit assumptions regarding their preferences that cannot be traced back to \textit{a~priori} axioms. For instance, in order to argue that members of the state apparatus exhibit a~constant tendency to seek expansion of their power, it must be assumed---non-apodictically---that their minds harbor such a~preference to begin with. Hoppe 
%\label{ref:RNDMpge1w8aP4}(2007a, p.15)
\parencite*[][p.15]{hoppe_democracy_2007} %
 admits this implicitly as he writes: ``\textit{Under the assumption of self-interest} [italics added], every government will use this monopoly of expropriation to its own advantage---in order to maximize its wealth and income.'' The same applies to the comparative analysis of monarchy and democracy: in the former system, ``assuming no more than self-interest, the ruler tries to maximize his total wealth, i.e., the present value of his estate and his current income 
%\label{ref:RNDg1NB1vuX7U}(Hoppe, 2007a, p.18).
\parencite[][p.18]{hoppe_democracy_2007}.%
'' On the other hand, ``once again assuming no more than self-interest… democratic rulers tend to maximize current income 
%\label{ref:RNDtDxIjj9WYK}(Hoppe, 2007a, p.144).
\parencite[][p.144]{hoppe_democracy_2007}.%
''



But to posit this is to transcend the purview of pure praxeology. Let us briefly recount what, according to Mises, constitutes the difference between praxeology and history. Namely, praxeology, as a~purely formal discipline, explicates the form of action, i.e., studies deductively the logical consequences of the fact that persons act. History, on the other hand, deals with the substance of action, which comprises goals actually pursued by agents 
%\label{ref:RND7l1PPvyapr}(Mises, 2007, p.271).
\parencite[][p.271]{mises_theory_2007}. %
 Hoppe's harnessing the assumption of politicians' self-interest unambiguously positions his considerations in the realm of Misesian history. Furthermore, as should be clear from the foregoing summary of his claims, in his monarchy theory, Hoppe construes this assumption along reductionist lines. As Nowakowski 
%\label{ref:RNDbPulWo7Q3I}(2010, p.272)
\parencite*[][p.272]{nowakowski_dlaczego_2010} %
 aptly notes, Hoppe reduces the complex motivations behind the actions of people in power to the motive of pecuniary gain.\footnote{In all honesty, in the only explicit formulation I~have found in \textit{Democracy…,} Hoppe 
%\label{ref:RNDHUVliVgK2L}(2007a, p.144)
\parencite*[][p.144]{hoppe_democracy_2007} %
 defines self-interest more broadly as ``maximizing monetary and psychic income: money and power.'' As will be seen, however, his theory effectively throws the power motive overboard. Otherwise, the oversights pinpointed in the next two sections of this essay would have been avoided. The preponderance of the ``wealth and income'' talk is also visible in the quotes adduced above. } Concomitantly, he employs a~conception of rationality narrower than that characterizing Mises's praxeology and more akin to that of neoclassical economics 
%\label{ref:RNDlWD163Ede3}(see Long, 2006; Rizzo, 2015).
\parencites[see][]{long_realism_2006}[][]{coyne_problem_2015}. %
 For if it is assumed that members of the state---or, for that matter, monarchical or democratic heads thereof---are predisposed to do this-and-that on a~more or less permanent basis because their value scales are such-and-such, it must first be assumed (a) that those value scales \textit{are} such-and-such and that (b) they are at least fairly fixed. Were this not the case, Hoppe's monarchy theory could not claim any predictive validity, as it does in asserting that ``it is economically and ethically \textit{advantageous to choose} [italics added] monarchy over democracy'' 
%\label{ref:RNDESfbbbK6sy}(Hoppe, 2007a, p.xx).
\parencite[][p.xx]{hoppe_democracy_2007}.%
\footnote{One may add that in adopting the assumption of narrow self-interest, Hoppe not only goes beyond praxeology but also comes closer to political economy in the public choice tradition, which explicitly disposes of the notion of benevolence on the part of politicians. By contrast, Misesian economics sticks to praxeological ``formalism'' by declaring agnosticism with regard to motives and focusing on the absence of market process in political decision-making 
%\label{ref:RNDiUjyOMd3UO}(Boettke and López, 2002).
\parencite[][]{boettke_austrian_2002}. %
 However, what is presupposed by Hoppe's monarchy theory is in fact a~\textit{very} narrow self-interest, that is, one reducible to pecuniary profit. This resembles the classic 19\textsuperscript{th} century model of economic man much more than its contemporary incarnations, those used by public choice theorists included. }



As has already been discussed, the cornerstone of Hoppe's theory of political systems is the notion of time preference. Despite certain technical controversies, those followers of Mises who concur that ``the actor always prefers satisfaction sooner rather than later'' are unanimous in deeming this statement an apodictic theorem 
%\label{ref:RNDIwcUaYyPDh}(Herbener, 2011; Mises, 1998, pp.480–485; Rothbard, 2009, p.15).
\parencites[][]{herbener_introduction_2011}[][pp.480–485]{mises_human_1998}[][p.15]{rothbard_man_2009}. %
 Perhaps it is owing to the deployment of this conception that the monarchy theory may still claim an aprioristic status at least in part, as proposed by Gordon? Not really. For even if it can be known \textit{a~priori} that one's time preference must always be positive, it does not entail that the factors shaping the degree of time preference can also be discovered that way 
%\label{ref:RNDFb87Vi8rVF}(Block, Barnett and Salerno, 2006).
\parencite[][]{block_relationship_2006}. %
 Hoppe, recall, holds that one such factor is ownership, with private property fostering lower time preference, and public property inducing higher time preference. Realistic as this may sound, it is demonstrably short of an \textit{a~priori} proposition. To exemplify, imagine a~fellow, let us call him Paul, who seems to be a~man of contradictions. On the one hand, he is a~veritable spendthrift: in a~small accounting proprietorship that he owns, Paul always tries to work as little as possible, and all income he derives disappears within a~week, spent on whiskey, drugs, and women. Needless to say, no savings at all are made. On the other hand, Paul simultaneously happens to be a~card-carrying communist. As a~party member, he is anything but unreliable: he serves as the party's treasurer, and in this capacity he proves as pennywise as it gets. In actuality, Paul's attitude is not so contradictory: he simply resents capitalism so much that saving money earned as a~wicked petty bourgeois is the last thing he is interested in doing. When the longed-for revolution finally comes, Paul starts working himself to the bone for his recently collectivized company, and his years of drinking and womanizing are gone. Currently, Paul is saving half his pay so that the Party may one day inherit it.



What Paul's example evinces is, first, that it might indeed be the case that public ownership will lower one's time preference while private ownership will increase it, and second, that the direction in which one's time preference changes in response to a~change in ownership arrangements is contingent upon one's goals. Paul is a~communist altruist, so to him the time discount on the same good is higher when it is owned by himself and lower when it belongs to a~communist state. In contrast, Hoppe's politicians are rational egoists. Their time preference goes down when goods are theirs and up when they are owned publicly not because any praxeological law so dictates, but because they are self-interested. Once again, then, Hoppe's argument could not get off the ground if certain assumptions regarding human preferences were not made.



With that in mind, we are poised to demonstrate why Hoppe's conceptualizations of political systems should be viewed not as praxeological theories but as Weberian ideal types. Mises, who adopted this tool as well, insisted, however, that the use of the ideal type be restricted exclusively to the domain of history 
%\label{ref:RND4jPsFo05Qq}(Mises, 1998, pp.59–64; 2007, pp.315–322).
\parencites[][pp.59–64]{mises_human_1998}[][pp.315–322]{mises_theory_2007}. %
 Now there are various divergent positions in the literature on the nature and functions of ideal types 
%\label{ref:RNDTChHi1HU3v}(see Kuniński, 1980, pp.35–119).
\parencite[see][pp.35–119]{kuninski_myslenie_1980}. %
 What is nevertheless not up for the debate is that ideal types are built upon a~``the one-sided accentuation of one or more points of view and by the synthesis of a~great many diffuse, discrete, more or less present and occasionally absent concrete individual phenomena'' 
%\label{ref:RNDQVMBGJZWQc}(Weber, 1949, p.90; see also Mises, 2007, pp.315–320).
\parencites[][p.90]{weber_methodology_1949}[see also][pp.315–320]{mises_theory_2007}. %
 The role of this procedure is, among other things, to help the researcher make sense of the infinitely complex reality of human action 
%\label{ref:RNDSULW9MspsQ}(Mises, 2007, p.320).
\parencite[][p.320]{mises_theory_2007}. %
 A~good example of how the ideal type works might be Weber's 
%\label{ref:RNDaMiziomIs1}(2001)
\parencite*[][]{weber_protestant_2001} %
 famous model of the Protestant, who, hoping that earthly success will prove evidential of his being predestined for salvation, is driven by the stringent precepts of labor ethics. Thus, of all urges that could possibly influence the Protestant's actions only one is picked and brought to the extreme. The same goes for Hoppe's model of monarch, which predicts that the ruler will act solely and consistently on the motive of personal enrichment. The way Hoppe makes use of the ideal type is nonetheless different from how their role was viewed by Mises. It is then worthwhile to take a~closer look at where those thinkers part company so as to better appreciate the specificity of Hoppe's position.



Crucially, Mises did not attribute to the ideal type nomothetic qualities, let alone the standing of an incontestable \textit{a~priori} truth. He chided Weber for his treatment of the laws of economics as ideal-typological simplifications, which shows that Mises regarded praxeological theories and ideal types as two distinct methodological categories 
%\label{ref:RNDdCyb6zFSFC}(Mises, 2003, pp.79–98).
\parencite[][pp.79–98]{mises_epistemological_2003}. %
 Moreover, based on his dichotomous division between theory and history, Mises unequivocally saw the ideal type as an instrument of the latter. Hence the requirement that ideal types be historically concrete, so as to capture the workings of a~given historical situation 
%\label{ref:RNDMpDV2goGtR}(Mises, 1998, p.62).
\parencite[][p.62]{mises_human_1998}.%




Unlike Mises, Hoppe is not consistent in dividing social science knowledge exhaustively into ``generalizing'' theory (praxeology) and ``individualizing'' history. In his early, German-language methodological treatise, between these two groups of disciplines, there emerges a~third field: sociology, which encompasses ``generalizing'' (although not \textit{a~priori}) explanations of historical facts in the form of articulated theories, both middle-range and grand. While informed and constrained by praxeology, sociological investigations contain the element of looser, non-apodictic speculations, a~part of which comes down to the explicit employment of fallible, substantive assumptions with respect to the goals or preferences of agents 
%\label{ref:RNDeB6wqcZHPX}(Hoppe, 1983, pp.33–38).
\parencite[][pp.33–38]{hoppe_kritik_1983}. %
 In brief, inquiries of this kind differ from Mises's history in that they operate at a~higher level of generality, and therefore do not pay heed to the postulate of historical concreteness. A~general, comparative analysis of monarchy and democracy could then be classified as belonging in this intermediary realm.\footnote{That is exactly how Hoppe designates his intellectual project in \textit{Democracy}… However, the meaning he attaches to the term is somewhat different this time. Hoppe writes: ``I wish to promote and contribute to the tradition of grand social theory, encompassing political economy, political philosophy and history and including normative as well as positive questions. An appropriate term for this sort of intellectual endeavor would seem to be sociology'' 
%\label{ref:RNDZUv3iG4Nks}(Hoppe, 2007a, p.xxiv).
\parencite[][p.xxiv]{hoppe_democracy_2007}. %
 The terminological confusion is compounded when in his other methodological piece, Hoppe 
%\label{ref:RND1n0LU5xuxQ}(Hoppe, 2007b, p.43)
\parencite[][p.43]{hoppe_economic_2007} %
 adopts Mises' bipartite division of social science into (praxeological) theory and history without mentioning sociology as a~distinct discipline. Be that as it may, ``sociology,'' as opposed to ``history,'' seems to be an accurate label for a \textit{theoretical} and generalizing inquiry that at the same time does not fall within the remit of praxeology 
%\label{ref:RNDYNhI2nhPpV}(on the standing of sociology in the Misesian tradition, see Robitaille, 2019).
\parencite[on the standing of sociology in thesee][]{robitaille_ludwig_2019}.%
}



As will be seen, the limitations of Hoppe's theory of monarchy stem in no small part exactly from the excessive tendency toward simplification, or stated more precisely: from the mentioned overemphasis placed on monetary gain in explaining the actions of monarchs. For although, as Kenneth Waltz 
%\label{ref:RNDMepRvY35kh}(1979, chap. 1)
\parencite*[][chap. 1]{waltz_theory_1979} %
 keenly observes, to theorize is essentially to simplify, it does not follow from this premise that simplifying is always justified to the point of trimming a~theory to a~single explanatory variable. Praxeology is arguably capable of doing so without incurring any cognitive loss thanks to commencing with a~single yet unassailable axiom stating that humans act, coupled with a~few uncontroversial, auxiliary assumptions such as the disutility of labor 
%\label{ref:RNDTRLdMq09oN}(Rothbard, 2009, chap. 1).
\parencite[][chap. 1]{rothbard_man_2009}. %
 An ideal type, on the other hand, lacks this sort of ultimate grounding, so the choice of but one action motive as a~starting point always runs the risk of throwing out with the bathwater of futile minuteness the baby of adequacy.



Another upshot of classifying the Hoppean monarchy theory as based on a~sociological ideal type rather than a~praxeological \textit{a~priori} theorem is that as a~specimen of the former, it must not be confused with what Mises refers to as catallactic functions, i.e., ``distinct functions in the market operations'' such as entrepreneur, capitalist, landowner, or laborer 
%\label{ref:RNDcfhL2Vqygt}(Mises, 1998, p.252).
\parencite[][p.252]{mises_human_1998}. %
 As will be shown below, the understanding that underlies Hoppe's ideal-typological model of monarchy suffers precisely from this confusion.



\section{Politician, not capitalist landowner}

Having established the ideal-typological and sociological nature of Hoppe's conception, we are now in a~position to subject it to critical scrutiny. First and foremost, serious suspicions are raised by the very fact of underscoring the pursuit of pecuniary profit as the key driver of politics. In effect, Hoppe proceeds with his argument as if the concept of the monarch exemplifies at least in part the catallactic functions of the\textcolor[rgb]{1.0,0.2,0.2}{ }capitalist or the landowner. It is only by doing so that Hoppe can claim that ``elementary theoretical insights regarding the nature of private property and ownership versus ‘public' property and administration'' suffice to ground his theory of monarchy. Mises 
%\label{ref:RNDzuvucQ1cgH}(1998, p.255)
\parencite*[][p.255]{mises_human_1998} %
 defines said functions in the following way:



\begin{quote}
Capitalist and landowner mean acting man in regard to the changes in value and price which, even with all the market data remaining equal, are brought about by the mere passing of time as a~consequence of the different valuation of present goods and of future goods.
\end{quote}



However, as Mises 
%\label{ref:RNDa05tHyz0nF}(1998, p.254)
\parencite*[][p.254]{mises_human_1998} %
 makes clear, all landowners and capitalists are at the same time entrepreneurs (i.e., actors facing uncertainty), which in the realities of the free-market economy necessitates the adjustment to the ever-changing preferences of consumers 
%\label{ref:RNDV8abRYhdft}(1998, pp.270–272).
\parencite*[][pp.270–272]{mises_human_1998}. %
 The monarch, on the other hand, is not a~capitalist or a~landowner in the sense explained by Mises. The latter two, as catallactic functions, belong in the free market economy 
%\label{ref:RND1hfAOLCBU6}(Mises, 1998, pp.252–256).
\parencite[][pp.252–256]{mises_human_1998}. %
 The free market or capitalism is, under Hoppe's 
%\label{ref:RNDCVaW7dEGZR}(2016, p.20; cf. Rothbard, 2009, p.92; 2011, p.320)
\parencites*[][p.20]{hoppe_theory_2016}[cf.][p.92]{rothbard_man_2009}[][p.320]{rothbard_wall_2011} %
 own definition, a~system based not on any old private property but precisely on titles derived from original appropriation, contracts, and subsequent production. By contrast, the state, whatever its form, exists only in contradiction to the acts of homesteading and contracting 
%\label{ref:RNDBuipu3On7L}(Hoppe, 2016, pp.49–52; Rothbard, 2009, p.877).
\parencites[][pp.49–52]{hoppe_theory_2016}[][p.877]{rothbard_man_2009}. %
 Thus, there can be no ``free market of states,'' wherein ``capitalists'' (kings and princes) could exchange and invest their wealth expecting positive returns that result from consumers' satisfaction. Such a~notion constitutes a~contradiction in terms on the grounds of Hoppe's own systematic commitments. Strictly speaking, in Franz Oppenheimer's 
%\label{ref:RNDuqWKSnEDtP}(1922, p.25)
\parencite*[][p.25]{oppenheimer_state_1922} %
 sense, monarchs are not economically active at all. Rather, all their operations are of a~political nature, i.e., derive income on a~coercive basis, which alters considerably the incentive structure kings are affected by.\footnote{Hoppe's definitions are evidently embedded in the political philosophy of Rothbardian libertarianism he advances. A~public choice economist would employ a~different terminology, perhaps one implying no such categorical distinction between voluntary market actions on the one hand and coercive government undertakings on the other. Nevertheless, despite this divergence, it is unambiguously clear for both Austro-libertarians and public-choicers that the free market and the state generate very different incentive structures. Which is precisely what Hoppe's argument obfuscates. }



Quite revealingly, elsewhere Hoppe 
%\label{ref:RNDHHpzR3BT6W}(2016, p.192)
\parencite*[][p.192]{hoppe_theory_2016} %
 names three reasons capitalism---i.e., the system founded upon respect for private property---proves more efficient than regimes of public ownership:



\begin{quote}
First, only capitalism can rationally, i.e., in terms of consumer evaluations, allocate means of production; second, only capitalism can ensure that, with the quality of the people and the allocation of resources being given, the quality of the output produced reaches its optimal level as judged again in terms of consumer evaluations; and third, assuming a~given allocation of production factors and quality of output, and judged again in terms of consumer evaluations, only a~market system can guarantee that the value of production factors is efficiently conserved over time.
\end{quote}



Observe now that these three advantages are not separate from one another but logically interconnected. Specifically, they all have to do with the supply side being dependent on the demand side. In sharp contradistinction, states, monarchies included, develop contrary to demand, owing to taxation and monopolization 
%\label{ref:RND4Leeihqb3f}(Hoppe, 2006, pp.49–52).
\parencite[][pp.49–52]{hoppe_economics_2006}. %
 What makes economic calculation, resulting in the means of production being allocated in the most effective fashion, both requisite and possible is the necessity of satisfying consumers' preferences. As brought home for us by Mises 
%\label{ref:RNDiPY1uvzhmV}(2012),
\parencite*[][]{mises_economic_2012}, %
 the problem is most vivid in the socialist economy, whereby central planners are in the dark when trying to decide what to produce and how. Nonetheless, even public enterprises operating in a~free-market environment still prove incapable of allocating resources efficiently, for being freed from the pressure of consumers' caprices, they simply do not need to do so 
%\label{ref:RNDYfZpIHXgvo}(Mises, 1944; Rothbard, 2009, pp.952–953).
\parencites[][]{mises_bureaucracy_1944}[][pp.952–953]{rothbard_man_2009}. %
 By the same token, were it not for that pressure, there would be no need for producers to seek the highest quality of output. In a~word, the source of capitalisms' efficiency is that the producer has to serve the consumer. And because that is not the monarch's occupation, neither of the above factors is at work in his case. Not surprisingly, Hoppe does not mention them either.



Things get somewhat more complicated when it comes to the notion of value conservation in time. Plainly, it is this element that undergirds Hoppe's time-preference-based monarchy theory. Granted, one need not be a~demand-responsive entrepreneur to have a~vested interest in preserving the value of his estate, even if to say so is a~well-reasoned generalization made under the assumption of self-interest rather than an \textit{a~priori} proposition. Still, since all value is subjective, what counts as the long-term value of a~resource depends on who does the valuing. And it goes without saying that with the monarchical state, it is done not by the willing consumers but the ruler himself. It is true that, as Hoppe 
%\label{ref:RNDYtNZIT557P}(2007a, p.18)
\parencite*[][p.18]{hoppe_democracy_2007} %
 points out, rulers do trade their estates between one another every now and then. They nevertheless do so within a~political, not an economic (again in Oppenheimer's sense) structure. The factors that might add to the value of an area in the eyes of monarchs are therefore likewise political. What matters is not only the economic capacity but also strategic localization, significance for dynastic alignments, fortifications, manpower available for the military, and the like. The quality of the economy is certainly of paramount importance, yet it is only one among other relevant variables. This also explains why monarchs trading their estates is a~rather rare occurrence. After all, other monarchs---the potential buyers---are politicians as well, so they can use the purchase against the seller up to the point of wiping his kingdom or duchy off the face of the earth.



This, of course, does not invalidate Hoppe's analysis completely as long as one takes it to be what it really is, i.e., as an exploration of the consequences of the lower time preference of monarchs as compared to democratic leaders that comes with the hereditary claim to power. Needless to say, there is no need to remove the economy from the calculations of kings and princes. At the end of the day, the economic capacity of their country is one of the things all statesmen should care about, not least because it contributes predominantly--- as an element of so-called latent power\textit{---}to the military and diplomatic potential of the state 
%\label{ref:RNDzdMsx9Ri3P}(Mearsheimer, 2001, chap. 3).
\parencite[][chap. 3]{mearsheimer_tragedy_2001}. %
 The point is to see things in the right proportions, neither ignoring nor overemphasizing the role of the pecuniary profit factor in politicians' calculations.



\section{Applications: power and war}

With the epistemological status of Hoppe's monarchy theory explained, let me spend the remainder of this essay trying to improve on his approach in a~fashion avoiding the one-sidedness Hoppe himself fell prey to. What shall be done below is a~humble attempt at outlining an ideal type of monarch that, first, takes seriously the motive of power-seeking and, second, acknowledges those implications of the monarchical private ownership in government which, overlooked in Hoppe's original analysis, make monarchy not as benign as advertised.



To start with, whatever the importance of the economy, it is safe to assume that the lust for power for its own sake can rank as a~motive for political activity at least as strong as financial gain 
%\label{ref:RNDuh5SxS2AaK}(Nowakowski, 2010, p.272).
\parencite[][p.272]{nowakowski_dlaczego_2010}. %
 In fact, that is precisely what Austro-libertarians, Hoppe included, normally posit when investigating the nature of the state in the context of class analysis 
%\label{ref:RNDgkeOWQXhX1}(Hoppe, 2006, pp.117–138; Rothbard, 2000, pp.55–88).
\parencites[][pp.117–138]{hoppe_economics_2006}[][pp.55–88]{rothbard_egalitarianism_2000}. %
 As Sierpiński 
%\label{ref:RNDSPRzsL8Iiy}(2016, p.557)
\parencite*[][p.557]{sierpinski_critica_2016} %
 pointedly argues, whether supporting the prosperity of his people is actually desirable for the maintenance of the king's power is far from obvious. That the opposite will turn out to be the case is particularly likely in economically backward monarchies. For as Tocqueville 
%\label{ref:RNDuXhGtOj5Gg}(1955)
\parencite*[][]{tocqueville_old_1955} %
 famously noted, governments face the highest risk of revolution not when they are the most repressive, but when they begin to reform.



Furthermore, even a~monarch driven chiefly by financial aspirations can largely satisfy his craving regardless of the nation's economic condition. Examples of breathtakingly opulent dictators ruling more or less underdeveloped, and sometimes downright devastated countries abound, with the contemporary cases of Idi Amin, Mobutu Sese Seko, Vladimir Putin, or Kim Jong Un readily springing to mind. This is due to the already mentioned fact that the monarch, not being a~capitalist landowner but a~politician, can extract income from his property irrespective of the evaluation of his services on the part of his subjects.



Moreover, what matters are not only the motives of politicians but also institutions. Modern democracy, to invoke the well-known definition by Joseph Schumpeter 
%\label{ref:RNDd4LQPEZlOp}(2006, p.269),
\parencite*[][p.269]{schumpeter_capitalism_2006}, %
 is ``that institutional arrangement for arriving at political decisions in which individuals acquire the power to decide by means of a~competitive struggle for the people's vote.'' If ``competitive'' is to denote open entry, as it in fact does in contemporary democratic theory, it entails logically (even if not necessarily in practice) the guarantee of certain political freedoms (and related property rights) such as freedom of association or freedom of speech 
%\label{ref:RNDg3kZEXf81r}(Dahl, 1989; Tilly, 2007, pp.13–14).
\parencites[][]{dahl_democracy_1989}[][pp.13–14]{tilly_democracy_2007}.%
\footnote{Since the job of the present section is ideal-typological modelling, it is plain that such purely analytic truths about democracy and monarchy must be taken into account. It is, of course, another question whether those truths---and the ideal types built upon them---are successful in elucidating real-world facts and processes, that is, whether existing democracies live up to the ideals enshrined in their definition. As mentioned, to give a~full empirical account of the applicability of my suggestions would require a~separate study. Still, it is worth noting that unlike most absolute monarchies past and present, democracies (at least the Schumpeterian ones, consisting in free elections and universal suffrage) do refrain from instituting censorship. Although more subtle means of free speech suppression are deployed here and there, e.g., through the pressure exerted on Big Tech, and certain views considered extreme or totalitarian (particularly Nazism and fascism, though not necessarily communism) tend to be outlawed, the restrictions are always short of full-fledged, institutionalized censorship, which remains an anathema. Note that what I~am referring to is the type of democracy described by Schumpeter and Hoppe himself, at least when the latter author talks about the deleterious ramifications of popular voting. In Hoppe, one can also discern a~broader notion of democracy, which encompasses any system of public property in government whether free and fair elections take place or not 
%\label{ref:RNDJuKE4j3Y1H}(Nowakowski, 2010).
\parencite[][]{nowakowski_dlaczego_2010}. %
 Of course, there are regimes that fulfill this definition (most notably Nazism and communism) to which neither Schumpeter's definition nor my argument applies. } The opposite is true of monarchy. Indeed, maintaining the monopoly of power requires diminishing this exact freedom. It is then no coincidence that absolute monarchies, historical and contemporary alike, often impose some kind of censorship.\footnote{Admittedly, as noted by Henshall 
%\label{ref:RNDlEJ7sHspWs}(2013, pp.114–117),
\parencite*[][pp.114–117]{henshall_myth_2013}, %
 censorship in early-modern absolute monarchies was not as stringent as the commonplace narrative has it. Some of Rousseau's subversive books or Diderot's and d'Lambert's \textit{Encyclopédie} got through in France, while England was keeping in force libel and sedition laws that heavily diminished freedom of press long after the Glorious Revolution of 1688. On the other hand, in the wake of the post-Vienna Congress restauration, censorship took hold of most European nations 
%\label{ref:RND0vTHrp4jb0}(Henshall, 2013, p.208),
\parencite[][p.208]{henshall_myth_2013}, %
 which suggests that monarchs are inclined to impose it when they fear that their position might be threatened. Given the power of today's media and the level of social pluralism, that would presumably be their policy in contemporary Europe as long as the prince's authority were to be safeguarded. Not surprisingly, present-day monarchies outside the West also maintain pretty strict censorship. } The same applies to the question of checks and balances. However imperfect the mechanisms built-in the constitutional systems of modern democracies might be, they at least exist. Whereas absolute monarchy, by assumption, seeks to do away with all such institutions altogether. The tacit conclusions of Hoppe's monarchy theory, on the other hand, turn the tenets of classical liberalism on their heads: if monarchy is better than democracy because it represents private property government, then---as Hoppe 
%\label{ref:RND4S2hNYN2r8}(2007a, p.18, f.9)
\parencite*[][f.9]{hoppe_democracy_2007} %
 himself asserts--- absolute monarchy is monarchy at its finest. Therefore, it proves preferable not only to democracy but also to any form of monarchy or mixed government that does implement checks and balances, since to do so is, by definition, to abridge the ownership rights of the monarch. That is to say, no constraints at all is allegedly better than weak constraints. Again, that Hoppe overlooks this objection may be explained by the absence of a~specifically political analysis in his theory. No one needs free speech guaranties, separation of powers, or checks and balances on a~ranch.\footnote{The complete neglect of the free speech question in Hoppe's 
%\label{ref:RNDUUNf3WZXon}(2007a, pp.50–62)
\parencite*[][pp.50–62]{hoppe_democracy_2007} %
 comparison of the historical achievements of monarchies and democracies is indeed quite startling. Are taxes and inflation really everything that matters for a~libertarian? }



This criticism seems even more formidable than the ones previously raised. For what those arguments testified to were only certain limitations of Hoppe's claim that monarchy means a~longer planning horizon, which in turn means a~lower level of property rights violations. They did not undermine this contention \textit{per se}, if only down to the importance of economic development for the relative strength of a~state in its relations with other states. In short, the reasoning so far has shown why a~monarch can rob and enslave \textit{despite} being a~monarch. What the argument from political freedoms and separation of powers demonstrates is, on the other hand, that in some respects a~monarch is indeed more dangerous than a~democratic caretaker precisely \textit{because} he is a~monarch (as Hoppe portrays him). It entails, moreover, that monarchs can be expected to infringe upon private property rights in departments such as the suppression of free speech to the \textit{greater} extent, the \textit{longer} their planning horizon is and the \textit{more} conscientious owners they are. A~king diverted from the pursuit of his dynasty's interests by sheer naivete or sincere devotion to libertarian principles can allow for free competition between political ideas, including those calling for his own overthrow; one preoccupied solely with the prosperity of his family business cannot.



The private character of monarchical ownership may incentivize rulers to engage in aggressive behavior in yet another way, to wit, by aggravating conflicts over power. True, in democracies, politicians do kill one another or, worse still, wage civil wars when incapable of seizing power in the wake of an election. However, such hostilities are usually fueled not by the narrow self-interest of politicians but rather by ethnic, religious, or ideological tensions 
%\label{ref:RNDh3KYICbAwV}(Megger, 2018).
\parencite[][]{megger_krytyczna_2018}. %
 For monarchs, violent games of thrones all too often become part and parcel of their profession. The reason is simple: the bigger the reward, the greater the lengths one is willing to go to in order to get it. He who recoils at the idea of slaughtering his rival (who may at the same time happen to be his friend or brother) for four or five years term in a~democratic office need not be that much appalled by the prospect of doing the same should that mean winning a~vast estate to be enjoyed by him and his posterity for centuries to come.



Relatedly, private ownership in government generates incentives for interstate bellicosity. True enough, the prospect of drawing income from the conquered economy should prevent the ruler from wrecking it in the course of military actions, and the lack of democratic-nationalist legitimation of power weakens the case for conscription. On the other hand, those very same factors increase the likelihood of waging war in the first place. First off, building empires or creating and protecting zones of influence are typically long-term projects that require relatively low time-preference. Their time horizon usually exceeds the term of the democratic politician. Hence, the king is more likely to pursue such policies than the president. On top of that, newly captured lands offer an opportunity for additional profit for the monarch and his kin, which again makes him more likely to seek territorial acquisitions than his democratic counterpart, who has no business fighting for spoils that will come about when his term has long been over 
%\label{ref:RND4YSNR5dw99}(Mises, 1985, p.121).
\parencite[][p.121]{mises_liberalism_1985}. %
 It is for a~reason that Austro-libertarians standardly invoke the interests of the deep state when explaining imperialism of democratic nations such as the USA 
%\label{ref:RNDzWnTT9mxqH}(Rothbard, 2011; Hoppe, 2006, pp.77–116).
\parencites[][]{rothbard_wall_2011}[][pp.77–116]{hoppe_economics_2006}. %
 After all, bankers and industrial-military complex people are private owners with the right to bequeath, so their time preference can accordingly be expected to be comparatively low, just as that of kings. Furthermore, the ideology of the democratic nation-state mitigates imperialism in that it imposes non-negligible limits on the policies of conquest: the size of the state ought to be congruent with the borders of a~nation's settlement 
%\label{ref:RND9D7Qtery62}(Gellner, 1993, p.1; Mises, 1985, p.118).
\parencites[][p.1]{gellner_nations_1993}[][p.118]{mises_liberalism_1985}. %
 By contrast, monarchy, as a~regime whose cosmopolitanism stems from legitimation coming from the ruler's rightful claim rather than from popular (national) approval, faces no such constraints. The monarch, uninterested in who it is that he is going to reign over, can in principle seize whatever territory he finds attractive. Yet again, that Hoppe disregards this stems from his overemphasizing the economic value-conservation factor, while downplaying the political value-destruction factor.\footnote{In the literature, one may find far more arguments to the effect that democracy produces peaceability rather than belligerence, known under the umbrella name of democratic peace theory. Many of those claims are highly debatable, though (see a~plausible critique by Hoppe
%  \label{ref:RNDBdnWQgVZ1k}(2021, pp.232–237)
 \parencite*[][pp.232–237]{hoppe_great_2021} as well as a~recent realist discussion of democratic peace theories in
% \label{ref:RNDrT2VT5Ebsd}(Mearsheimer, 2018)).
 \parencite{mearsheimer_great_2018}.
 The empirical record does not seem conclusive either. Although earlier research
% \label{ref:RNDhNbiNK0EMw}(Pinker, 2012; Rummel, 1983)
 \parencites{pinker_better_2012}{rummel_libertarianism_1983}
 seemed to support theories of democratic peace, Cirillo and Taleb recently
% \label{ref:RND3Z45Xo2zl8}(2015)
 \parencite*{cirillo_statistical_2015}
 argued that no such regularities might really be observed. At any rate, Hoppe's anti-democratic and pro-monarchical conclusions do not find confirmation.}



\section{Conclusion}

Hoppe's comparative analysis of monarchy and democracy undoubtedly furnishes a~number of original, intriguing, and thought-provoking vistas. Moreover, the very attempt to apply Austrian economics to problems typically penetrated by political scientists and public choice scholars is in and of itself worth the price of admission. It transpires, however, that the theory ultimately falls short of substantiating one of its central claims, namely, that monarchy, although beset by all sorts of evils inherent for all states, is after all a~preferable political setting insofar as property rights protection is concerned. The foregoing investigations have challenged this thesis chiefly from the methodological angle. More specifically, it has been demonstrated that rather than a~body of \textit{a~priori,} praxeological truths, Hoppe's monarchy theory contains an ideal type of the politician driven by self-interest and bestowed with an inheritable title to the throne. Yet, the analysis under criticism proceeds as though it did fall within the scope of praxeology, or even catallactics, which it does in mistakenly depicting monarchy as a~capitalist landowning enterprise of sorts. This article, on the other hand, argued that as a~sociological endeavor, explorations of political regimes have to take account of a~broader set of preferences ascribable to monarchs---i.e., a~broader concept of their self-interest---than bare monetary profit seeking. Furthermore, even when strictly profit-oriented, monarchs can still be inclined to pursue policies of aggression and parasitism. First, as statesmen and not capitalist landowners, they operate within a~legal framework that allows them to derive profit on a~coercive basis, which undermines the causal relation between the quality of the economy and the wealth of the king himself. More important still, the hereditary nature of their claim incentivizes monarchs to take greater pains---as compared to democratic politicians---in striving for power, which also involves more extensive use of political means. This does not mean that Hoppe's theory offers no benefits for our understanding of political systems and historical processes brought about by their succession. All in all, one takeaway from our inquiry is that it does matter whether a~politician can expect a~lifetime sitting on the throne for him and his descendants, or just a~few untransferable terms in office. However, an adequate analysis of political systems needs to grasp far more ramifications of this fact than Hoppe's own work does. Moreover, those ramifications, when judged from the liberal-libertarian vantage point, turn out to be ambiguous: some are conducive to property rights and free markets, some are not. Consequently, one is not warranted to conclude that monarchy surpasses democracy in preserving institutions cherished by free-marketeers as keys to freedom and prosperity. What we are left with is a~much more nuanced picture.









\end{artengenv}

