\begin{artengenv}{Alexander Linsbichler}
	{What Rothbard could have done but did not do: The merits of Austrian economics without extreme apriorism}
	{What Rothbard could have done but did not do\ldots}
	{What Rothbard could have done but did not do: The merits of Austrian economics without extreme apriorism}
	{Johannes Kepler University Linz\label{linsbichler-first}}
	{Austrian economics emphasizes a~priori components of social scientific theory. Most emphatically, Ludwig Mises and Murray Rothbard champion praxeology, a~methodology often criticized as extremely aprioristic. Among the numerous justifications and interpretations of praxeology to be found in the primary and secondary literature, conventionalism avoids the charge of extreme apriorism by construing the fundamental axiom of praxeology as analytic instead of synthetic. This paper (1) explicates the tentative structure of the fundamental axiom, (2) clarifies some aspects of a~conventionalist defense of praxeology, and (3) appraises conventionalist praxeology according to Rothbardian criteria. While Rothbard provides an essentialist justification of praxeology and embraces extreme apriorism, a~mildly aprioristic conventionalist defense of praxeology fares better on Rothbard's own criteria and is much more compatible with other contemporary methodological positions and economic theories.
	}
	{Austrian economics, praxeology, conventionalism, apriorism, analyticity, Ludwig Mises, Murray N. Rothbard.}







\section{Apriorism and praxeology}

\lettrine[loversize=0.13,lines=2,lraise=-0.03,nindent=0em,findent=0.2pt]%
{P}{}roponents of Austrian Economics have emphasized a~priori aspects of economic theorizing ever since the publication of the Austrian School's ``founding document'', Carl Menger's \textit{Principles of Economics} 
%\label{ref:RNDnSZZ3nuwtm}(1871).
\parencite*[][]{menger_grundsatze_1871}. %
 Yet, the extent to which members of the Austrian School theoretically endorsed and practically applied apriorism varies considerably between different scholars and perhaps even between different writings of a~single author. One of the most famous Austrian economists, F. A. Hayek, radically changed his stance towards apriorism once or twice---at least according to some but not all of his discordant interpreters.\footnote{See Caldwell 
%\label{ref:RNDCrv4IytRnq}(2009)
\parencite*[][]{caldwell_skirmish_2009} %
 and Scheall 
%\label{ref:RND69HaOrw7iT}(2015).
\parencite*[][]{scheall_hayek_2015}.%
} Sympathizers and critics alike identify the praxeological branch of Austrian economics as the most extremely apriorist.



Praxeology as a~methodology for the social sciences was introduced by Ludwig Mises and most famously continued by Murray Rothbard. Although these two most prominent champions of praxeology justify their position with different arguments, the basic idea is the same. We will merely sketch it in two steps here, referring readers to more extensive expositions, reconstructions, and discussions in the literature.\footnote{See e.g. Linsbichler 
%\label{ref:RNDG22bqR6a2k}(2017; 2021a),
\parencites*[][]{linsbichler_was_2017}[][]{linsbichler_austrian_2021}, %
 Long 
%\label{ref:RNDfShiK5kH0W}(2008),
\parencite*[][]{long_wittgenstein_2008}, %
 Mises 
%\label{ref:RNDQXgDZPa7KF}(1940; 2003; 2007; 1962; 2012).
\parencites*[][]{mises_nationalokonomie_1940}[][]{mises_epistemological_2003}[][]{mises_theory_2007}[][]{mises_ultimate_2012}.%
}



As Step One, the praxeologist has to prove that the fundamental axiom of praxeology, ``man acts'' 
%\label{ref:RNDuxK6gL3UCe}(see, e.g. Mises, 2012, p.4),
\parencite[see, e.g.][p.4]{mises_ultimate_2012}, %
 is an a~priori true starting point. Explications of the overly short ``man acts'' identify its content along the following lines: human individuals and only human individuals (as opposed to viruses, planets, or social classes) at least sometimes behave purposefully, i.e. they choose goals and apply means they subjectively consider expedient to attain these goals on the basis of their beliefs. Strictly speaking, the way Mises and other Austrian economists apply the fundamental axiom only suggests that human individuals act and none of the other known types of objects act. In case we encounter intelligent aliens, praxeologists might reconsider the ``and only human individuals'' clause.\footnote{Compare Rothbard's related take on aliens or animals having rights 
%\label{ref:RNDYRhTmAkccB}(Rothbard, 1998, pp.155–157)
\parencite[][pp.155–157]{rothbard_ethics_1998} %
 and children having rights 
%\label{ref:RNDiRjeHyq9o0}(Rothbard, 1998, pp.97–112).
\parencite[][pp.97–112]{rothbard_ethics_1998}.%
}



Note that our explication of the fundamental axiom presupposes an independent characterization of ``human beings'' in advance but does not state that human beings exist. When a~social scientist ascribes goals and beliefs to each human being (and only to human beings), she has finished her job instantaneously if there are no human beings in the universe of discourse. \textit{If (and only if) x~is a~human being, then x~acts.} This is vacuously true, if there are no human beings.\footnote{One reviewer invited us to consider that there might not have been men if a~meteorite wiped out not just dinosaurs but Mother Earth too. Can we nevertheless uphold the analytic truth of ``man acts''? We hope that our further explication of the fundamental axiom addresses this worry---without engaging in discussions on the relationship between necessity and analyticity.} Strictly speaking, the fundamental axiom is of no help in ascertaining whether a~certain human behavior is merely behavior or an action either. According to Mises, this assessment is neither praxeological nor natural scientific but a~thymological matter, i.e. obtained by the ``method'' of \textit{Verstehen} (understanding) and a~posteriori. We hope that the following tentative structure of the fundamental axiom---the first explication of its kind---will facilitate further clarifications of its content:



\begin{enumerate}[label=\alph*)]
\item \textit{For all x: If and only if x~is a~human individual, then certain theoretical entities t}\textit{\textsubscript{1x}}\textit{, t}\textit{\textsubscript{2x}}\textit{, t}\textit{\textsubscript{3x}}\textit{, … (goals, preferences, beliefs, interpretations, …) exist, such that $\varphi $(x, t}\textit{\textsubscript{1x}}\textit{, t}\textit{\textsubscript{2x}}\textit{, t}\textit{\textsubscript{3x}}\textit{, …).}
\item \textit{For all x, y: If y~is behavior of a~human individual x, then: (If and only if y~is an action then $\psi $ (x, y, t}\textit{\textsubscript{1x}}\textit{, t}\textit{\textsubscript{2x}}\textit{, t}\textit{\textsubscript{3x}}\textit{, …)).}
\end{enumerate}

In Step Two, together with auxiliary hypotheses and empirical claims including claims about the content of the actors' preferences and beliefs, economic theorizing proceeds in a~purely deductive manner. Hence, the praxeological ``economist need not displace himself; he can, in spite of all sneers, like the logician and the mathematician, accomplish his job in an armchair'' 
%\label{ref:RND3t5KuzxSEk}(Mises, 2012, p.78).
\parencite[][p.78]{mises_ultimate_2012}. %
 Since deduction preserves truth and aprioricity, all logical consequences of the fundamental axiom are a~priori true---provided Step One was successful.\footnote{Strictly speaking, given an a~priori true axiom f, a~posteriori auxiliary hypotheses h1 and h2, and the a~posteriori thymological statement t, the a~priori praxeological theorems could at best have the form ‘((h1 \& h2 \& t) $\to$ x)'. Typically, the statement ‘x' will not be a~priori true. For a~similar analysis of mathematics and potential ensuing problems, see Carnap 
%\label{ref:RND2bBReCHa6a}(1955; 2000)
\parencites*[][]{carnap_foundations_1955}[][]{carnap_untersuchungen_2000} %
 and Jeffreys 
%\label{ref:RNDzLzRvRGf9m}(1938)
\parencite*[][]{jeffreys_nature_1938}%
respectively.} The wonderful result would be an a~priori true economic theory, immune to empirical criticisms.



Advancing the intricate methodological and epistemological discussions regarding praxeology and economic principles in general, this paper reviews two types of problems with praxeology raised in the literature (section 2), sketches how conventionalist praxeology aims to circumvent and solve these problems (section 3), clarifies misunderstood aspects of conventionalist praxeology (section 4), contrasts conventionalism with Rothbard's essentialist defense of praxeology (section 5), presents Rothbard's criteria for the acceptability of fundamental axioms (section 6), and appraises whether his own arguments (section 7) or conventionalist proposals (section 8) succeed in meeting the criteria, and finally indicates one of many open problems for future research (section 9).



Before concluding this introduction, a~clarification regarding ‘apriorism' will be expedient for the remainder. The concept of ‘apriorism' as discussed in the Austrian School literature and beyond comprises three distinct, yet not always sufficiently distinguished, constituents: First, and foremost for our purposes, apriorism is an epistemological notion referring to those elements of economic theory that are not subject to falsification, verification, confirmation, corroboration, or challenge on an empirical basis. It would be a~category mistake to employ experience as a~critical standard for an a~priori statement.\footnote{Some Austrian economists including Rothbard reject sensory experience as a~critical standard for economic theory but highlight the justificatory role of inner experience (intuition, introspection). We will return to this in sections 5-8.} Second, experience as well as interpretative understanding are enabled and directed by a~theory and interpretational standpoints. Thirdly, experience is not the source or origin of ideas for theories. This final component is contested even within the Austrian School, especially if ‘experience' is meant to encompass inner, intuitive experience. Yet, for the purposes of this paper, we are mainly concerned with justifications and criticisms of praxeology, not with its psychological origins.



\section{Two types of criticisms of praxeology}

Arguably, almost every scientific research program contains implicit a~priori elements and perhaps Austrian economists merely tend to be explicit and reflective about their presuppositions. That being said, extreme apriorism which immunizes large parts of theory from empirical criticisms, has become highly suspect in the development of philosophy of science and, with some time lag, also among economists 
%\label{ref:RND8Gl3x8WMa4}(Scheall and Linsbichler, 2023).
\parencite[][]{scheall_rise_2024}. %
 Accordingly, and since the standard view maintains that praxeology depends on extreme apriorism, philosophers and economists have condemned praxeological methodology as well as economic claims based on praxeological research.\footnote{See e.g. the quotes in Linsbichler 
%\label{ref:RNDUeu9GiRQmL}(2021a, p.3360).
\parencite*[][p.3360]{linsbichler_austrian_2021}.%
}



While some Austrian economists, most prominently Rothbard, embrace extreme apriorism, others challenged the standard interpretation of Mises's justification of praxeology.\footnote{See e.g. 
%\label{ref:RNDy2Gi0QQiqH}(Zanotti, Borella and Cachanosky, 2023).
\parencite[][]{zanotti_hermeneutics_2023}.%
} They tried to argue that, rightly understood, Mises's position is not extremely aprioristic after all. Scheall 
%\label{ref:RNDj9VW96RfwL}(Scheall, 2017)
\parencite*[][]{Scheall2017a} %
 clarified these debates by explicating the notion of ``extreme apriorism'' as involving three dimensions. Unlike some overblown statements, especially in popularized portrayals of Austrian economics and in uncharitable criticisms, the \textit{extent} of apriorism in Mises's (and Rothbard's) account of praxeology is not extreme after all. Only the fundamental axiom is a~priori and very little is implied by the fundamental axiom without additional premises. Yet, the \textit{kind of justification} given for the fundamental axiom and its purported \textit{certainty} are indeed extreme on almost all accounts of Misesean epistemology because they invoke introspection, intuition, or some other form of inner experience as, possibly infallible, justification.



Partly motivated by the attempt to gauge the extremeness of Mises's apriorism, partly for its own sake, a~considerable bulk of secondary literature has emerged that engages in exegetical discussions concerning Mises's justification of praxeology. A~radical but convincingly argued assessment of the state of research by Scheall 
%\label{ref:RNDPKolnKTXsG}(2023)
\parencite*[][]{Scheall2024} %
 maintains that Mises's own writings are so incoherent that a~wide range of epistemological positions can be ascribed to him.\footnote{Zilian 
%\label{ref:RNDSF4T7VBSae}(1990)
\parencite*[][]{zilian_klarheit_1990} %
 also identifies indications of inconsistencies in Mises's epistemological and methodological writings.}



Praxeology in Mises's tradition faces two problems: (i) if it is extremely aprioristic (as most interpreters hold), then it is considered untenable in light of contemporary philosophy of science; (ii) Mises's writings seem to allow for radically different interpretations as to how he attempts to justify praxeology and consequently how extreme his apriorism is.



\section{The conventionalist turn: a~few clarifications}

A~recent proposal by Linsbichler 
%\label{ref:RNDlW1QIwTJo3}(Linsbichler, 2017; 2021a)
\parencites*[][]{linsbichler_was_2017}[][]{linsbichler_austrian_2021} %
 addresses the problem of extreme apriorism and circumvents the problem of the exegesis of Mises's works. Instead of engaging in the exegetical debates, Linsbichler proposes a~defense of praxeology that is supported by some passages in Mises's writings but, more importantly, aims at ``dispelling charges according to which praxeology is untenable because it relies on extreme apriorism'' 
%\label{ref:RNDXZ0MfOu4Mw}(Linsbichler, 2021b, p.204)
\parencite[][p.204]{linsbichler_otto_2021}%
---independently of whether Mises defended praxeology in this manner or not.\footnote{This reformist and constructive agenda was already present in the first presentation 
%\label{ref:RNDn49BghLwDD}(Linsbichler, 2017, see e~.g. p.124)
\parencite[][see e~.g. p.124]{linsbichler_was_2017} %
 but is more accentuated in 
%\label{ref:RNDx6k5IXKjmn}(Linsbichler, 2021a; 2021b).
\parencites[][]{linsbichler_austrian_2021}[][]{linsbichler_otto_2021}[][]{Linsbichler2024b}.%
}



Other justifications of praxeology which avoid extreme apriorism are not precluded, but Linsbichler 
%\label{ref:RND55SBTeNGJR}(2017)
\parencite*[][]{linsbichler_was_2017} %
 proposes a~conventionalist defense of analytic praxeology, first embedded in a~broader reconstruction of Mises's methodological views and later more focused and detailed on conventionalist praxeology 
%\label{ref:RNDe7HDimJLVQ}(Linsbichler, 2021a).
\parencite[][]{linsbichler_austrian_2021}.%




The vital step is to construe the fundamental axiom as analytic instead of synthetic a~priori. This shift is prompted by the insight that, contrary to claims by many praxeologists, it is perfectly conceivable to explain human behavior employing alternatives to the fundamental axiom. Neither direct observation nor intuition nor introspection can rule out these alternatives conclusively. This is a~challenge to the interpretation of the fundamental axiom as a~Kantian synthetic a~priori, which would preclude the existence of any alternatives.



An analytic sentence is true in virtue of the definitions and semantic rules of the language in which it is formulated (and logical rules of inference). Hence, an analytic fundamental axiom would also be a~priori.\footnote{See Kripke 
%\label{ref:RNDuEnwiZcTQR}(Kripke, 1972; 1980, pp.122–123)
\parencite*[][pp.122–123]{kripke_naming_1980} %
 for potential complications that can, however, be avoided by choosing a~suitable semantic theory and adequate definitions. } This is in line with Oliva Córdoba's presentation of ``Analytic Praxeology'' according to which ``it is conceded on all sides that being analytic is sufficient for being a~priori'' 
%\label{ref:RNDONbynCZeey}(Oliva Córdoba, 2017, p.528; see also 523).
\parencites[][p.528; see also p.523]{oliva_cordoba_uneasiness_2017}.%




That being said, there seems to be a~subtle difference between Linsbichler's and Oliva Córdoba's versions of analytic praxeology.\footnote{Linsbichler also separates his approach from Oliva Córdoba's while commending the ``logical and explicatory aspect'' of the latter 
%\label{ref:RNDYoACIP99Qz}(Linsbichler, 2021a, p.3374).
\parencite[][p.3374]{linsbichler_austrian_2021}.%
} On the one hand, Linsbichler stresses that definitions of terms and rules of a~language can in principle be set at will, as long as they are consistent with each other. Which definitions and rules of language to adopt is a~matter of choice. Definitions are true in virtue of being definitions and, more generally, analytic truths are true by convention.\footnote{One reviewer objects that analytic truths being true by convention ``is a~controversial and ultimately untenable position in the philosophy of language and logic''. While an encompassing defense would go far beyond the scope of this paper, we submit, first, that our conceptions of analyticity and conventionalism as well as the internal/external distinction sketched below are particularly amenable to deriving the conclusion that analytic truths are truths by convention. Second, although there is indeed controversy about the open problems of this account, it is by no means outlandish. For a~contemporary defense of conventionalism in logic and mathematics in natural and formalized languages, see Warren 
%\label{ref:RNDMjzcmUgHCQ}(2020).
\parencite*[][]{warren_shadows_2020}. %
 } Introspection or intuition play a~minor role at best, and the resulting approach is only mildly aprioristic with respect to the kind of justification. On the other hand, Oliva Córdoba 
%\label{ref:RNDIeI4jD99FV}(2017, p.527)
\parencite*[][p.527]{oliva_cordoba_uneasiness_2017} %
 states that ``what accounts for the truth of the conceptual explications […] is nothing over and above a~proper grasp of the concepts involved''. This approach seems to assume the existence of concepts, independent of and prior to language. Apparently, these concepts can be ``grasped'' and ``explicated'' more or less properly. Oliva Córdoba 
%\label{ref:RNDLJ4qokZIWz}(2017)
\parencite*[][]{oliva_cordoba_uneasiness_2017} %
 gives partial ``conceptual explications'' or ``conceptual clarifications'' of the concepts of uneasiness, action, and scarcity, which he then skillfully employs as premises in proofs. While Linsbichler would likely construe these starting points of the proofs as partial definitions and thus analytically true by convention (and probably scientifically fruitful and in broad agreement with everyday language to boot), Oliva Córdoba's notion of analyticity seems to require more. These starting points have to reflect a~``proper grasp of the concepts involved'' 
%\label{ref:RND91xBTCkiZW}(Oliva Córdoba, 2017, p.527)
\parencite[][p.527]{oliva_cordoba_uneasiness_2017} %
 to yield analytic truths. Arguably, the judgment of whether such a~grasp is indeed proper or not involves some sort of intuition. Hence the resulting research program is more extremely aprioristic than Linsbichler's.\footnote{See also footnote 24 on essentialist conceptual analysis.} \textit{If} there is an ultimate standard to assess the correctness of logical rules, semantic rules, and definitions of terms, then analytic praxeology is not conventional.


\begin{table}[h!]
    \centering
    \begin{adjustbox}{max width=\textwidth}
        \begin{tabular}{|>{\centering\arraybackslash}m{4cm}|>{\centering\arraybackslash}m{4cm}|>{\centering\arraybackslash}m{4cm}|}
            \hline
            & \textbf{fundamental axiom claimed to be analytic} & \textbf{fundamental axiom claimed to be synthetic} \\ \hline
            \textbf{conventionalist justification} & Linsbichler & \footnotemark \\ \hline
            \textbf{non-conventionalist justification} & Oliva Córdoba & standard interpretation of Mises and Rothbard \\ \hline
        \end{tabular}
    \end{adjustbox}
    %\caption{Comparison of Justifications for Fundamental Axioms}
\end{table}
\footnotetext{In principle, there could also be a~conventionalist defense of a~synthetic fundamental axiom. It would have to ignore certain unpleasant empirical findings though. See also footnote 32.}

Once the fundamental axiom is understood analytically and the existence of different linguistic systems is acknowledged, conventionalism stands to reason. Unless one invokes a~strongly essentialist philosophy of language aiming for ``the one correct notion of action,'' the choice of linguistic systems is guided by pragmatic arguments. Arguably, linguistic systems in which ``man acts'' is analytic are quite close to natural language and also fruitful for social scientific investigations.



Given an analytic fundamental axiom and provided the deductions are valid, there is a~limited theoretical core of analytic statements that potentially facilitates more fruitful theorizing about economic phenomena.\footnote{Cf. Linsbichler's 
%\label{ref:RNDVSDfIfhjYT}(2023a)
\parencite*[][]{linsbichler_ultra-refined_2023} %
 reconstruction of Aumann's position in the philosophy of game theory, according to which game theory is an ultra-refined, analytic grammar for talking and thinking about interactions.} Some clarifications of the notion of conventionalism might be helpful.



Quite different epistemological positions have been labeled ``conventionalism'' in the history of philosophy of science and beyond. Linsbichler discusses some of the problems of many variants and immunizing strategies of conventionalism. While he deployed a~specific Popperian notion of conventionalism first 
%\label{ref:RND7DERgtGqG2}(Linsbichler, 2017; Popper, 2009, pp.367–511),
\parencites[][]{linsbichler_was_2017}[][pp.367–511]{popper_two_2009}, %
 he extended and generalized the approach later 
%\label{ref:RNDBcx9r4Apec}(Linsbichler, 2021a)
\parencite[][]{linsbichler_austrian_2021} %
 by highlighting two necessary conditions for a~methodology to be conventionalist. Arguably, these conditions---which we will also adopt in this paper---encompass almost all positions labelled as ``conventionalist'':



\begin{enumerate}[label=(\Alph*)]
\item The conventions could in principle have been chosen differently, i.e. alternative theories or research programs are possible.
\item The conventions are not justified by observation or intuition, but by pragmatic arguments for the superior expediency of the resulting theory or research program. 
%\label{ref:RNDXB8Ph8FEJ2}(Linsbichler, 2021a, p.3371)
\parencite[][p.3371]{linsbichler_austrian_2021}%
\end{enumerate}

While Linsbichler 
%\label{ref:RNDwYmkRVug3B}(2021a)
\parencite*[][]{linsbichler_austrian_2021} %
 indirectly hints at the Carnapian inspiration throughout the paper, it should perhaps have been stated more clearly that the formulations of (A) and (B) draw on a~distinction between internal questions to be solved within a~linguistic framework on the one hand and external questions about such frameworks to be discussed in a~meta-language on the other 
%\label{ref:RNDYtqYstHR21}(cf. Carnap, 1950).
\parencite[cf.][]{carnap_empiricism_1950}. %
 Once definitions and the rules of language are postulated, such that a~version of the fundamental axiom is analytic, there are no alternatives to the fundamental axiom \textit{in this framework.} The axiom is analytically true and it cannot be otherwise \textit{internally}. The alternatives mentioned in (A) exist in different frameworks which are visible from the meta-perspective outside the framework only. By contrast, a~non-conventionalist Kantian reading of Mises holds that the fundamental axiom is a~synthetic statement a~priori and as humans, equipped with a~particular structure of mind, we are not capable of imagining or experiencing the world in a~manner that would contradict the fundamental axiom, so there are no alternatives to it, full stop.



The second condition, (B), can be elucidated by the framework approach as well. Internal questions have to be distinguished from external questions again. Within a~linguistic framework, the justification of an analytic fundamental axiom rests solely on the definitions and rules of language of that framework. Note that being true in virtue of definitions does not imply triviality. The respective proofs can be highly intricate. The formulation of (B), stating that conventions are justified by pragmatic arguments, refers to the \textit{external} question how to set up a~framework or which framework to choose. Empirical evidence and intuitions play a~role in such pragmatic arguments and in the decisions they inform but so do evaluative elements. We expand on this point at some length to emphasize that the choice of frameworks and the choice of conventions is arbitrary only in the limited sense that it is usually not determined but requires a~decision. The choice of conventions is eminently \textit{not} arbitrary in the sense that it is a~purely subjective matter of taste 
%\label{ref:RNDstA1J070U7}(see also Linsbichler, 2021a, p.3379; 2024).
\parencites[see also][p.3379]{linsbichler_austrian_2021}[][]{dambock_factvalue_2024}.%




\section{Dissipating further worries about praxeology without extreme apriorism}

An explication of praxeology with an analytic fundamental axiom and with limited a~priori scope is only mildly aprioristic. Accordingly, it is much more amenable to empiricist and other contemporary positions in philosophy of science and methodology of economics. Results of praxeological investigations thus cannot be dismissed off-hand but should and can be discussed constructively instead of dogmatically between Austrian and non-Austrian economists.



Yet, perhaps conventionalist praxeology is only praxeology by name, whereas in substance it is completely detached from Mises's and Rothbard's original approach. Surely not any theory of human action should be subsumed under praxeology in the sense of Austrian economics. We offer a~fivefold response to this worry: First, the originator of praxeology, Mises, proclaims remarks and arguments that contain at least traces of the idea of an analytic fundamental axiom and arguably even of conventionalism 
%\label{ref:RNDeJPPfGeeyZ}(Linsbichler, 2021a, pp.3376–3378).
\parencite[][pp.3376–3378]{linsbichler_austrian_2021}.%




Second, aprioricity is the crucial property of the fundamental axiom and of praxeology that Mises, Rothbard, virtually all praxeologists, and critics of praxeology stress again and again as quintessential to the approach. Since analyticity implies aprioricity, this requirement is unequivocally fulfilled. By contrast, Lipski 
%\label{ref:RNDz8nrHmtDro}(2021)
\parencite*[][]{lipski_austrian_2021} %
 proposes a~more radical reform of praxeology by explicitly adding empirical hypotheses as axioms to obtain directly testable predictions. Thereby, he drops aprioricity. Lipski's diligently argued venture might well be advisable to promote the explanatory power of theories of human action. However, for better or worse, without aprioricity it ceases to qualify as praxeology in the Misesean tradition in our assessment.



Third, analogously to aprioricity, other epistemological or methodological traits might be considered indispensable from a~praxeological or Austrian School perspective. Although there is no clear consensus on the details, some forms of realism and of anti-instrumentalism are often regarded as a~trademark of the philosophy of Austrian economics 
%\label{ref:RNDSNfPPAFDPk}(Linsbichler, 2021c; 2021d).
\parencites[][]{heilmann_philosophy_2021}[][]{linsbichler_philosophy_2021}. %
 If conventionalism contradicted realism or anti-instrumentalism, it might be unpalatable to many Austrian School methodologists. Yet, Linsbichler 
%\label{ref:RNDWepo4LZD4u}(2021a, pp.3380–3383)
\parencite*[][pp.3380–3383]{linsbichler_austrian_2021} %
 substantiates at length why his variant of conventionalism is anti-instrumentalist as well as compatible with many versions of both realism and anti-realism.



Fourth, prima facie the conventionalist proposal concerns the \textit{justification} of praxeology, not its \textit{content}.\footnote{Compare the related assessment that Mises's praxeological research program can be reconstructed in Lakatosian terms with a~hard core that is de facto barred from empirical tests---but that this does not imply that Mises was a~conventionalist 
%\label{ref:RND9a1JWIwMHm}(see Zanotti, Borella and Cachanosky, 2022).
\parencite[see][]{zanotti_can_2022}. %
 Different justifications are in principle available as to why certain statements should be barred from empirical testing and these immunized statements need not even be analytic.} If successful, working Austrian economist can ignore the methodological disputes and by and large continue to use the fundamental axiom as before, i.e. draw logical conclusions from it and not submit it to direct empirical tests.\footnote{Linsbichler's 
%\label{ref:RNDxidXuxOg8p}(2017; 2021a)
\parencites*[][]{linsbichler_was_2017}[][]{linsbichler_austrian_2021} %
 conventionalist defense of praxeology hinges on a~construal of the fundamental axiom as analytic, though. To be sure, it seems plausible to assume that such an explication of the axiom's exact content is in line with the meaning of the fundamental axiom described by Mises and Rothbard and employed by praxeologists, as well as quite coherent with natural language usage. Yet, future more detailed logical inquiries might reveal that in order to be kept analytic, the exact specification of the fundamental axiom would have to be altered to an extent unacceptable to working economists. Only in this unlikely case, a~conventionalist defense of praxeology would trigger a~substantial change in the content of praxeological theory.}



Fifth, in contrast to Mises's justification of praxeology, Rothbard's does not contain any traces of conventionalism. At one point, he explicitly dismisses the idea that the law expressed in the fundamental axiom is a~disguised definition 
%\label{ref:RNDWu4c6k8o1q}(Rothbard, 1957, p.318).
\parencite[][p.318]{rothbard_defense_1957}. %
 Consequently, an analytic praxeology with a~conventionalist justification seems to be out of the question for him.



Having said that, his methodological writings allow for the reconstruction of four requirements which a~praxeological fundamental axiom must fulfill. Linsbichler 
%\label{ref:RNDJokiZDcpL2}(2021a, p.3376)
\parencite*[][p.3376]{linsbichler_austrian_2021} %
 bluntly states that ``these requirements pose more severe difficulties to Rothbard's own arguments […] than to a~conventionalist justification of praxeology'' but does not exhibit an argument. Sections 6 to 8 in the paper at hand address that lacuna. If an analytic fundamental axiom meets the Rothbardian criteria, we corroborate that conventionalist praxeology is perfectly compatible with the philosophy of the Austrian School of economics.



One final type of worry casts doubt on almost the entire debate on praxeological epistemology. Aspects of those doubts can be traced back to Rothbard who wonders whether ``epistemological pigeon-holing'' into a~priori and a~posteriori, analytic and synthetic, or empirical and theoretical, might not be a~waste of time. After all, is not the only relevant point that the fundamental axiom is self-evidently true 
%\label{ref:RNDkh8WQl609w}(see e.g. Rothbard, 1957, p.318)?
\parencite[see e.g.][p.318]{rothbard_defense_1957}? %
 Working economists might indeed be well advised not to spend too much time on methodology. However, for an analysis of the logical structure of the arguments those economists make, of the consequences of the fundamental axiom, and of the questionable claim regarding self-evidence---for these and other questions, economic methodologists employ specialized technical terms. They are picked out of the conceptual toolbox of philosophy. It is inconsequential whether Mises, Rothbard, or other scholars themselves used these concepts. They are analytical tools which---if well-defined---might facilitate the analysis of the ideas of these authors.



A~priori, analytic, and the other concepts mentioned above are applicable to sentences. So, if praxeology does not consist of sentences and the fundamental axiom is not a~sentence, this would indeed constitute a~deep problem for the entire enterprise. In this spirit, Bylund 
%\label{ref:RNDfETRhfiEw1}(2023)
\parencite*[][]{bylund_alexander_2023} %
 demurs that the term ``fundamental axiom'' was only coined by Rothbard and the very idea of axiomatizing a~set of sentences was foreign to Mises. The latter's conception of praxeology rather targets at delimiting a~realm of study, according to Bylund. Given the vital role that Mises and virtually all praxeologists attribute to truth and to deduction, we strongly suggest sticking with the standard interpretation of praxeological theory as a~set of sentences. The predicate ‘truth' is only applicable to sentences; and it is sentences that are governed by deductive rules of inference as well 
%\label{ref:RNDV0PuVjA7oy}(Linsbichler, 2023a).
\parencite[][]{linsbichler_ultra-refined_2023}.%




\section{Rothbard's essentialist defense of praxeology}

Mises's significance as the originator and principal proponent of praxeology notwithstanding, the secondary literature has perhaps overly focused on his contributions---at the expense of more profound examination of his main successor. An analysis of the defenses of praxeology ought to take into account the view of Rothbard. After all, his first monograph on human action 
%\label{ref:RNDt9O9iJznjf}(Rothbard, 1962a; 2009)
\parencites[][]{rothbard_man_1962}[][]{rothbard_man_2009_lin} %
 is considered the most important theoretical work of the praxeological branch of the Austrian School, which formed as a~self-conscious group from the 1970s onwards, not least through Rothbard's initiatives, articles, and personal conversations 
%\label{ref:RND4QBFH93Gtt}(Gordon, 2007, pp.122–124; Holcombe, 2014; Rockwell Jr., 2010; White, 1977; 2003, pp.26–27).
\parencites[][pp.122–124]{gordon_essential_2007}[][]{holcombe_advanced_2014}[][]{rockwell__jr_murray_2010}[][pp.26–27]{white_methodology_2003}. %
 In reconstructing his defense of praxeology below, we will place particular emphasis on deviations from Mises. This should not obscure the fact that Rothbard rightly sees his work as a~continuation of Mises's work and always writes most admiringly of his mentor. Conversely, Mises expressly praised the contents of the sections of the monograph presented first: ``I would subscribe to every word Rothbard has written in his study.'' 
%\label{ref:RNDTvH7rmxIeU}(see Mises, 1976, p.158; cf. also Gordon, 2008, p.2).
\parencites[see][p.158]{mises_my_1976}[cf. also][p.2]{gordon_who_2008}.%




Rothbard, like Mises, considers the deduction of economic theorems from the fundamental axiom to be the task of economics. He refines and expands on the elaboration of praxeological theory formation in relation to economics. In doing so, he identifies the a~posteriori auxiliary axioms and discusses their role in derivations more clearly than his predecessors. As with Mises, the extension of praxeology to all human activity remains largely programmatic. The economic sphere may be somewhat broader than in the mainstream of economics but the theorems of the ``general, formal theory of human action'' 
%\label{ref:RNDMCM5jCH3iW}(Rothbard, 1951b, p.945)
\parencite[][p.945]{rothbard_praxeology_1951} %
 rarely stray far from the sphere of catallactics. Rothbard does indeed mention ``largely unexplored areas'' of praxeology: the theory of war,\footnote{Compare Taghizadegan \& Otto 
%\label{ref:RNDYdcziWWUMQ}(2015).
\parencite*[][]{taghizadegan_praxeology_2015}.%
} game theory,\footnote{Mises 
%\label{ref:RND0MPTgvQsyY}(1998, pp.116–117)
\parencite*[][pp.116–117]{mises_human_1998} %
 does not concede the slightest connection between game theory and praxeology. A~similar remark can also be found later 
%\label{ref:RNDwU45DyMn50}(Mises, 2012, p.135).
\parencite[][p.135]{mises_ultimate_2012}. %
 However, the more detailed passage on the relationship between Morgenstern's and Neumann's work and praxeology could also be read as indicating a~shift of opinion 
%\label{ref:RNDl5uTDlT5gW}(Mises, 2012, pp.89–90).
\parencite[][pp.89–90]{mises_ultimate_2012}. %
 It is difficult to assess the extent to which substantive reasons were responsible for this, such as the advances within game theory, which was now able to deliver results beyond zero-sum games. A~certain distancing of parts of the Austrian school from game theory, which still exists today, could be explained by the general aversion to formalization. For a portrayal of game theory as a manifestation of the Austrian School's research program, see Streissler \parencite{Streissler2000}.} and ``unknown'' 
%\label{ref:RND2YaaLT3CDn}(Rothbard, 1951b, p.946).
\parencite[][p.946]{rothbard_praxeology_1951}.%




Rothbard goes into much more detail than Mises regarding the structure of social scientific explanations and predictions. The reconstruction of praxeological explanations by Linsbichler 
%\label{ref:RND8itbPl9s4E}(2017, pp.52–55; see also Gordon, 1999)
(\cite*[][pp.52–55]{linsbichler_was_2017}, \cite[see also][]{gordon_economics_1999}) %
 draws on Mises's and Rothbard's expositions and are coherent with a~conventionalist methodology: ``If his prediction proves erroneous, it is not praxeology that has failed, but his judgement of the future behavior of the elements in the praxeological theorem. Praxeology is indispensable, but it does not provide omniscience'' 
%\label{ref:RNDC5iIntOb8F}(Rothbard, 1951b, p.945).
\parencite[][p.945]{rothbard_praxeology_1951}. %
 In the interplay of praxeology and thymology, Rothbard transfers the entire empirical content to the boundary conditions. However, Rothbard 
%\label{ref:RNDjUUbQjJzFr}(1989)
\parencite*[][]{rothbard_hermeneutical_1989} %
 not only describes the application of praxeology and distinguishes thymology from hermeneutics, but applied these conceptual insights practically. While Mises largely confines himself to economics, philosophy of science and, in his later work, some social philosophy, Rothbard, a~student of mathematics and economics, tries to pursue and combine economics, philosophy, political theory, ethics, natural law, history, and the history of ideas. In addition to this already versatile oeuvre, he published a~(not entirely serious) drama. The interplay between praxeology and thymology is consciously applied in many of Rothbard's historical works 
%\label{ref:RND08J9vCIZE1}(Rothbard, 1962b; 1963; 2000; 1975; 2011a; 1996; 2012; 2020).
\parencites[][]{rothbard_panic_1962}[][]{rothbard_americas_2000_lin}[][]{rothbard_conceived_2011}[][]{rothbard_origins_1996}[][]{rothbard_war_2012}[][]{rothbard_history_2020}. %
 His methodological approach is in line with Mises' and Hayek's ideas of the interplay between theory and history 
%\label{ref:RNDwoQdRh3wsS}(White, 1977; 2003, p.26).
\parencite[][p.26]{white_methodology_2003}. %
 In the context of this paper, however, we are primarily interested in the epistemological status of the fundamental axiom in Rothbard's approach. How does he try to justify the truth of praxeology and how successful is he? We will explore these questions in sections 5-7.



Rothbard phrases the fundamental axiom in some places, as Mises does, with ``man acts'', but more often with ``human action exists''. If one tries to explicate the content of these vague phrases from the explanations as well as from the use in the deductions, one arrives at very similar results as in Mises and as sketched in section 1.\footnote{We deem the apparent form as an existential proposition, which Rothbard's short form of the fundamental axiom assumes, of no particular importance.}



Let us next turn to attempts to establish the truth or the necessary truth of the fundamental axiom, and thus to Rothbard's entire perspective on praxeology. He prefaces his arguments with an extremely revealing remark. Rothbard reminds the reader, almost apologetically, that the undertaking is difficult and, in a~sense, useless. He quotes Toohey and the choice of words is indicative of Rothbard's view of social scientific knowledge:



\begin{quote}
Proving means making evident something which is not evident. If a~truth or proposition is self-evident, it is useless to attempt to prove it, to attempt to prove it would be to attempt to make evident which is already evident 
%\label{ref:RNDKK0YiIOE3V}(Rothbard, 1976, p.28).
\parencite[][p.28]{rothbard_praxeology:_1976}.%
\end{quote}




Both Rothbard's and Mises's defense of praxeology are dealing with ``proofs'', i.e. the establishment of certain knowledge. Likewise, Rothbard, in his anticipation of arguments of Hoppe's discourse ethics, writes of attempts at a~``refutation'' 
%\label{ref:RNDjJ746CI5yD}(Rothbard, 1976, pp.28–29).
\parencite[][pp.28–29]{rothbard_praxeology:_1976}. %
 Of course, in Rothbard's work, too, the economist is a~fallible human being, and critical debate is the key to scientific progress. For Rothbard, however, current social science knowledge does not contain bold, fallible hypotheses that provide good explanations at the moment and have corroborated their worth, but established truths.\footnote{Similarly, Mises 
%\label{ref:RNDSQ8XGlNzEJ}(1998, p.68)
\parencite*[][p.68]{mises_human_1998} %
 draws a~sharp line between tentative laws in the natural sciences and praxeology: ``Praxeology---and consequently economics too---is a~deductive system. It draws its strength from the starting point of its deductions, from the category of action. No economic theorem can be considered sound that is not solidly fastened upon its foundation by an irrefutable chain of reasoning. A~statement proclaimed without such a~connection is arbitrary and floats in midair.''} They might only be overturned if some researcher made a~mistake. This conception of science as a~search for certainty may be partly responsible, at least psychologically, for the vehemence with which Rothbard and many other Austrians advocate their economic and oftentimes also their political positions.



Rothbard's epistemological and methodological writings hardly suffer from the tensions discerned in Mises's. He offers a~straightforward, strongly essentialist justification of the fundamental axiom.



In the Aristotelian and Thomistic tradition, Rothbard does not want to deal primarily with isolated sensory impressions, atomistic units, or superficial economic quantities. The goal is rather to holistically uncover the essences of phenomena by means of a~cognitive synthesis: ``The empiricism is broad and qualitative, stemming from the essence of human experience'' 
%\label{ref:RNDM4nDuoFAW8}(Rothbard, 2007, p.xvi).
\parencite[][p.xvi]{mises_preface_2007}. %
 The grasp of potentialities and essences is purportedly possible through targeted attention. Rothbard controversially attributes a~similar position to Alfred Schütz 
%\label{ref:RNDrFuJHBOSbB}(Rothbard, 2011b [1973]; 1976),
\parencites[][]{rothbard_praxeology_2011}[][]{rothbard_praxeology:_1976}, %
 arguably these essentialist tenets rather have Austrian precursors in Spann, Wieser, and Mayer 
%\label{ref:RNDT68jV0GE7t}(Milford and Rosner, 1997; Linsbichler, 2022).
\parencites[][]{milford_abkopplung_1997}[][]{linsbichler_viel_2022}. %
 How the synthetic, holistic grasp of certain, intersubjectively verifiable truths is to proceed is not even hinted at, let alone precise methodological regulations given. Anyways, Rothbard insists that introspection, without any inductive steps, warrants the necessary and universal truth of ``man acts'' with certainty.



\section{Rothbard's requirements for a~fundamental axiom}

From Rothbard's methodological and epistemological writings, four criteria for the acceptability of a~fundamental axiom for the social sciences can be reconstructed. To be sure, Rothbard's deliberations on ``man acts'' are much more exhaustive. Since our goal is to evaluate \textit{justifications} of a~fundamental axiom, we only include pertinent statements, though. Among other things, statements about the context of discovery are not included in the criteria. For instance, Rothbard obtains the fundamental axiom via introspection and uses alleged attributes of introspection to argue for the a~priori truth of the axiom. We deem a~priori truth to be the desired goal, the criterion which a~fundamental axiom must meet. Introspection is merely a~means to this end and thus not a~necessary requirement.



The four Rothbardian criteria, on the basis of which we will assess Rothbard's essentialist and Linsbichler's conventionalist defense of ``man acts,'' are that a~fundamental axiom of praxeology must have the following four properties: (I) Its falsification is inconceivable. (II) It is empirically meaningful. (III) It is a~priori with respect to complex historical events. (IV) It is absolutely true. We will now explicate and briefly discuss these four claims in turn 
%\label{ref:RNDU8yJRItL0L}(Rothbard, 1957, pp.314, 317–319; 1976, p.25).
\parencites[][pp.314, 317–319]{rothbard_defense_1957}[][p.25]{rothbard_praxeology:_1976}.%
\footnote{Note, however, that this paper commits neither to an endorsement nor to a~criticism of these criteria for social scientific research in general. They serve an instrumental purpose only. Given Rothbard's eminent status within the praxeological branch of Austrian economics, it is plausible that his criteria are or should be important for many praxeologists. Hence the Rothbardian requirements are a~prime candidate for the intended ``undogmatic methodological critique'' 
%\label{ref:RNDslLhUTEKvP}(Caldwell, 1984, p.129),
\parencite[][p.129]{caldwell_praxeology_1984}, %
 i.e. an appraisal of (praxeological) Austrian School claims and arguments from the perspective of (praxeological) Austrian economics.}

\medskip

\noindent (I) A~falsification of the fundamental axiom is inconceivable 
%\label{ref:RNDvvIS51VpZd}(Rothbard, 1957, p.318).
\parencite[][p.318]{rothbard_defense_1957}. %
 While Rothbard classifies the auxiliary axioms of praxeology as obviously true in our world, a~counterfactual scenario in which they are false can be thought of without contradiction, so they are not necessarily true. As for the fundamental axiom however, the fact that human individuals have some goals and pursue them by any means must apply in every possible world in which there are human individuals 
%\label{ref:RND4io5TrPEIB}(Rothbard, 1957, pp.314–315):
\parencite[][pp.314–315]{rothbard_defense_1957}:%




\begin{quote}
In short, we can imagine a~world where resources are not diverse, but not one where people exist but don't act. We have seen that the other postulates, although ``empirical'', are so obvious and acceptable that they can hardly be called ``falsifiable'' in the usual empiricist sense. How much more is this true of the axiom, which is not even conceivably falsifiable! 
%\label{ref:RNDaZ2L4VueKN}(Rothbard, 1957, p.317)
\parencite[][p.317]{rothbard_defense_1957}%
\end{quote}




Quoting Toohey, Rothbard 
%\label{ref:RNDUmKxgZZmiV}(1976, p.28)
\parencite*[][p.28]{rothbard_praxeology:_1976} %
 provides another illustrative example of a~proposition, the falsification of which is inconceivable. He asserts that one cannot think that one has seen a~round square. Although Rothbard phrases this and similar remarks in terms of impossible thought processes, he can plausibly be interpreted in line with the approach in this paper. Since Rothbard disclaims impositionist views, he arguably holds that the justification of a~priori statements is not concerned with the limitations of the human cognitive apparatus but with conceptual analysis. That being said, conceptual analysis is not a~purely analytic method for Rothbard but involves intuitive access to essences.\footnote{For analytic conceptual analysis, see Linsbichler 
%\label{ref:RNDisDzAtvzGk}(2017, pp.81–83).
\parencite*[][pp.81–83]{linsbichler_was_2017}. %
 For another variant of essentialist conceptual analysis, see Wieser 
%\label{ref:RNDF6FogNfTlc}(1884),
\parencite*[][]{wieser_uber_1884}, %
 Linsbichler 
%\label{ref:RNDlY3HvkkXUB}(2021e; 2023b),
\parencites*[][]{linsbichler_sprachgeist_2021}[][]{linsbichler_case_2023}, %
 Schweinzer 
%\label{ref:RNDmFVAxCGos6}(2000),
\parencite*[][]{schweinzer_two_2000}, %
 Tokumaru 
%\label{ref:RNDSwFdzfo5P1}(2015).
\parencite*[][]{tokumaru_wiesers_2015}.%
}

\medskip

\noindent (II) The fundamental axiom is ``empirically meaningful'' 
%\label{ref:RNDQnYdJDqgyr}(Rothbard, 1957, p.318).
\parencite[][p.318]{rothbard_defense_1957}. %
 Without clearly distinguishing the two, Rothbard situates Mises's epistemology sometimes in a~Kantian framework 
%\label{ref:RND7CBMgHMBtO}(Rothbard, 2011b, p.33),
\parencite[][p.33]{rothbard_praxeology_2011}, %
 and sometimes in neo-Kantian one 
%\label{ref:RNDsph6QajynC}(Rothbard, 1957, pp.317–318).
\parencite[][pp.317–318]{rothbard_defense_1957}. %
 The salient point is that, according to Rothbard, Mises considers the fundamental axiom to be a~``law of thought'' 
%\label{ref:RND7dSPtxX13P}(Rothbard, 1957, p.318),
\parencite[][p.318]{rothbard_defense_1957}, %
 a~categorical truth a~priori to all experience, and apodictically true.



Rothbard 
%\label{ref:RNDpQYrIs96gq}(2011b, pp.33–34)
\parencite*[][pp.33–34]{rothbard_praxeology_2011} %
 asserts that most praxeologists, like himself and in contrast to Mises, interpret the fundamental axiom \textit{empirically}, albeit apodictically true nonetheless. However, such references to experience or to ‘the real world' are a~far cry from modern conceptions of empiricism, as Rothbard 
%\label{ref:RNDqtVsM41HHy}(1957, p.318)
\parencite*[][p.318]{rothbard_defense_1957} %
 himself acknowledges. For Rothbard, in order to be ``empirically meaningful'' some indirect, possibly vague relationship between the terms of praxeological theory and phenomena in the physical world suffices. In particular, note that in more contemporary terminology, his idiosyncratic use of ``empirically meaningful'' neither implies falsifiability 
%\label{ref:RNDY9Mukvlib6}(Rothbard, 1976, p.25),
\parencite[][p.25]{rothbard_praxeology:_1976}, %
 nor testability, nor does it establish intersubjective experience as a~critical standard for the truth of the statement. With a~criterion of meaning or of empirical significance, such as that discussed in the Vienna Circle and in today's philosophy of science, Rothbard's demand has little more in common than the name.



An upshot of Rothbard's view why the fundamental axiom counts as empirically meaningful is that it is neither a~law of thought nor a~psychological theory about the capacity of the human sensory and cognitive apparatus. For Rothbard, ``human action exists.'' purports to make an assertion about the world outside the human cognitive apparatus, not merely about the human limits of the possibility of perceiving this world. Thus, an interpretation of the fundamental axiom as a~genetic or psychological a~priori can be ruled out for Rothbard's defense of praxeology.\footnote{Mises's late work 
%\label{ref:RNDDNpFD88wmg}(Mises, 1962; 2012)
\parencite[][]{mises_ultimate_2012} %
 is not entirely clear on this point.} When Rothbard calls the fundamental axiom ``empirically meaningful'', he excludes not only the genetic a~priori but also other interpretations: it cannot be a~methodological principle because such a~principle would be a~normative rule and not a~descriptive assertion. Moreover, we can conclude from Rothbard's demand for empirical meaning that an explication of the fundamental axiom should not be understood as an uninterpreted axiom system or as a~group of pseudo-propositions. Instead, the meanings of the terms contained, for example ``human'', are at least partially fixed independently of the fundamental axiom. Some of Rothbard's objections to the mathematization of economics underpin this reading of ``empirically meaningful'' as well.\footnote{See 
%\label{ref:RND9l9D44emh0}(Rothbard, 1976, pp.21–24)
\parencite[][pp.21–24]{rothbard_praxeology:_1976} %
 and also Linsbichler 
%\label{ref:RNDC4XyuHZvSQ}(2021e; 2023b).
\parencites*[][]{linsbichler_sprachgeist_2021}[][]{linsbichler_case_2023}. %
 }

\medskip

\noindent (III) The fundamental axiom is a~priori with respect to complex historical events 
%\label{ref:RNDjIaa2dS4Sn}(Rothbard, 1957, p.318; 1976, p.25).
\parencites[][p.318]{rothbard_defense_1957}[][p.25]{rothbard_praxeology:_1976}. %
 Rothbard's engagement with the complexity of social scientific situations (in alleged contrast to less complex natural scientific situations) is typical, if relatively extreme, for Austrian economics. He describes the fundamental axiom not only as ``empirically meaningful'' but even as ``radically empirical'' 
%\label{ref:RNDQK4p3xnGRM}(Rothbard, 1976, p.24).
\parencite[][p.24]{rothbard_praxeology:_1976}. %
 We explained above that these statements are not to be misunderstood in a~post-Humean sense of empiricism. According to Rothbard, complex historical events can only illustrate conclusions from the fundamental axiom. They are not suitable as proof or test 
%\label{ref:RND03nZtGseIi}(Rothbard, 1951a, p.181; 1951b, pp.944–945).
\parencites[][p.181]{rothbard_mises_1951}[][pp.944–945]{rothbard_praxeology_1951}. %
 If, like Rothbard, one understands forms of introspection or reflection as a~form of experience, the fundamental axiom regarding this specific inner experience is a~posteriori. In any case, it is a~priori with regard to complex historical events. External experience is not a~critical standard by which praxeological sentences are measured.

\medskip

\noindent (IV) The fundamental axiom is ``absolutely true'' 
%\label{ref:RNDUQtTEj9JT8}(Rothbard, 1957, pp.314, 317).
\parencite[][pp.314]{rothbard_defense_1957}. %
 What distinguishes truth from absolute truth in Rothbard's nomenclature is not entirely clear. The formulation can be read as an expression of the lack of differentiation between truth and certainty. In other passages, Rothbard seems to have in mind truth without exception in our world or the much stronger truth in all possible worlds, i.e. necessary truth.
 
 \enlargethispage{1.5\baselineskip}

			
In any case, it is crucial for praxeology that the fundamental axiom be true. Only if the starting point of the deductive chains is true, this desired truth value is transferred to all conclusions. This fourth and final requirement that Rothbard makes of the fundamental axiom is therefore the most important for the project to justify general social scientific laws, i.e. to solve the problem of induction in the theoretical social sciences by means of praxeology.\footnote{For praxeology as a~solution to the problem of induction, see Linsbichler 
%\label{ref:RND8GgGs1SToS}(2017)
\parencite*[][]{linsbichler_was_2017} %
 and Tokumaru 
%\label{ref:RNDGfDZvmYmFZ}(2009).
\parencite*[][]{tokumaru_poppers_2009}. %
 Note that the essentialist Rothbard explicitly contends, quoting John Elliott Cairnes, that no process of induction is necessary for the discovery of praxeological knowledge because strictly general knowledge can be obtained directly by turning attention to our consciousness 
%\label{ref:RND834S1QkBrv}(Rothbard, 2011b, pp.65–68).
\parencite[][pp.65–68]{rothbard_praxeology_2011}.%
} As we shall see, however, Rothbard's argument is quite problematic.



First of all, it is striking that he tries to establish the truth of the fundamental axiom with considerations that can be assigned to the context of discovery instead of the context of justification.\footnote{See Reichenbach 
%\label{ref:RNDWWO1OcjzKc}(1938)
\parencite*[][]{reichenbach_experience_1938} %
 and also Hoyningen-Huene 
%\label{ref:RNDT7VlkQJ1Bb}(1987)
\parencite*[][]{hoyningen-huene_context_1987} %
 for complications with the distinction.} According to Rothbard, the fundamental axiom as well as its truth are grounded in ``universal inner experience, and not simply on external experience, that is, its evidence is reflective rather than physical'' 
%\label{ref:RND8OlJCT90lz}(Rothbard, 1957, p.318).
\parencite[][p.318]{rothbard_defense_1957}.%
\footnote{Elsewhere, Rothbard 
%\label{ref:RNDojeceSLUM1}(2011b, pp.33–34)
\parencite*[][pp.33–34]{rothbard_praxeology_2011} %
 also emphasizes introspection. Yet, the basis of knowledge about human action is not always solely universal inner or reflective experience but external physical experience is additionally invoked.} The special character of this holistic introspection as a~source of knowledge is supposed to prove the universality of the fundamental axiom:



\begin{quote}
However, although the axioms are \textit{a~priori} to history, they are a~posteriori to the universal observations of the logical structure of the human mind and human action. The axioms are therefore open to the test of observation in the sense that, once postulated, they are universally recognized as true. Such recognition may be accused of being ‘introspective‘, but it is nonetheless scientific, since it is an introspection that can command the agreement of all. 
%\label{ref:RND12Z6Ef9JUu}(Rothbard, 1951a, p.181)
\parencite[][p.181]{rothbard_mises_1951}%
\footnote{See also Rothbard 
%\label{ref:RNDir9I42DiTe}(1957, pp.317–318).
\parencite*[][pp.317–318]{rothbard_defense_1957}.%
}
\end{quote}



Rothbard, like Mises, considers intersubjectivity to be a~hallmark of scientificity. With a~wide variety of formulations, he tries to suggest that the specific view of the nature of human action provides and guarantees intersubjectivity in addition to truth and certainty: The fundamental axiom is allegedly evident to anyone who contemplates it---just as evident as sense experience 
%\label{ref:RNDI8K1rKWhUU}(Rothbard, 2011b, p.35).
\parencite[][p.35]{rothbard_praxeology_2011}.%
\footnote{Note that standard empiricist epistemologies which Rothbard apparently aims to emulate here do not accept reports of sensory data as infallible guarantees of certainty.} Every individual, in the face of a~reflection on the axiom of action, must agree to its truth and to its importance for the social sciences 
%\label{ref:RNDPqCeqb8cZy}(Rothbard, 1951b, p.943).
\parencite[][p.943]{rothbard_praxeology_1951}. %
 A~person could, of course, claim to deny the existence of these self-evident principles. You can say whatever you want; but there are limits to thinking and doing 
%\label{ref:RNDXB4yXj5XdJ}(Rothbard, 1976, p.28).
\parencite[][p.28]{rothbard_praxeology:_1976}. %
 For logical reasons, for example, no one can imagine a~round square.\footnote{So far, so good, but Rothbard does not merely reject claims involving inconsistent concepts. Going a~decisive step further, he maintains that certain reports of alleged logical or empirical findings have ``no epistemological validity'' either. If data contradict ``established truths of the real world'', they can and should be ignored in Rothbard's methodology 
%\label{ref:RNDC3iv5tDkqp}(1976, p.28).
\parencite*[][p.28]{rothbard_praxeology:_1976}. %
 Such ad hoc immunizing strategies are also characteristic of some variants of conventionalism but decidedly not what Linsbichler 
%\label{ref:RNDQvwBT7B5xP}(2021a, p.3370)
\parencite*[][p.3370]{linsbichler_austrian_2021} %
 suggests, not least because they ``facilitate [...] dogmatic tendencies''.}



\section{Appraisal of Rothbard's account according to his own criteria}


\begin{flushright}
\textit{``You can't always get what you want.''} (Keith Richards, Mick Jagger)
\end{flushright}

\medskip




Given the discussion of Rothbard's position in the previous two sections, we can put on record that an essentialist account is able to render the fundamental axiom ``man acts'' empirically meaningful in Rothbard's weak sense (II) and a~priori to complex historical events (III).



Requirement (IV), absolute truth, turns out more questionable. Rothbard invokes a~special form of introspection as a~source of knowledge and approves of it as a~criterion of truth. The postulation of truth criteria is extremely problematic in the context of a~fallibilistic conception of science, even if only sentences about one's own consciousness would be affected. For someone with only the slightest empiricist inclination (in the modern sense), the description of an empirical fact, such as a~personal psychic experience, can only ever be a~hypothesis, not a~certified truth.\footnote{The only exception to the epistemological impossibility of truth criteria may be some formal systems with no reference to experience or an external world. Ironically, the early Mises 
%\label{ref:RND0tUEGbPQAB}(1940, p.18)
\parencite*[][p.18]{mises_nationalokonomie_1940} %
 characterizes experience, including inner experience, as yielding findings that always could have been expected differently and infers explicitly that neither outer nor inner experience can justify the universal propositions of praxeology.} The compelling conviction that Rothbard obviously feels, and which perhaps many humans feel regarding some specific inner or outer experience, does not guarantee that the sentence describing the content of the compelling experience is true. A~mental conviction of truth, no matter how intense, is not proof of the validity of the content of a~sentence or of a~chain of deductive steps.



Yet, for the sake of argument let us concede to Rothbard that he has intuited, with necessary truth, that he himself has goals and uses means to achieve them. The main difficulties for establishing the fundamental axiom in a~Rothbardian manner arise when one tries to infer statements about the minds of other people from inner experience. How is it possible to draw necessary conclusions about other people from the exploration of one's own consciousness?



Since Rothbard requires and considers the fundamental axiom to be empirically meaningful, the term 'human' is at least partly interpreted, i.e. at least for many paradigmatic cases it is determined which physical objects are in the extension of ‘human' and which are not. Suppose m~is one such human individual and suppose it turns out that m~does not act. Then, for Rothbard, the potential immunization strategy of simply not calling everything that does not act
%as
 a~human being is blocked.



Rothbard seems to be aware of the problem and the respective rejection of inner experience as a~reliable source of knowledge in Mises's earlier writings 
%\label{ref:RNDk6cICUoXyQ}(e.g. Mises, 1940, pp.17–19).
\parencite[e.g.][pp.17–19]{mises_nationalokonomie_1940}.%
\footnote{Cf also the following criticism of Spann's essentialist intuitive universalism by Mises, which would incidentally be applicable to Rothbard as well: ``However, what Spann has in mind when he declares the a~priori method to be the only one appropriate for sociology as he conceives it is not at all a~priori reasoning, but intuitive insight into a~whole'' 
%\label{ref:RNDslCDspFDYH}(Mises, 2003, p.46).
\parencite[][p.46]{mises_epistemological_2003}.%
} Thus, Rothbard struggles to demonstrate why this particular form of intuition is not tantamount to ``the arbitrariness of intuitive flights of fancy'' 
%\label{ref:RND3EVK1VqwC5}(Mises, 2003, p.52)
\parencite[][p.52]{mises_epistemological_2003} %
 but would necessarily command universal intersubjective agreement. Without providing new arguments, he repeats and reformulates the claim that it is so, sometimes quoting supposed authorities like Aristotle, Thomas Aquinas, Say, Cairnes, Toohey, Schütz, and Knight.\footnote{See section 6, (IV) above.}



Let us suppose that every person states that introspection made her realize that she is acting. This is insufficient for intersubjectivity, though. Intersubjectivity would require several people to be able to focus their attention on the consciousness of the same person M. Then, for the time being, consensus could possibly be reached on the truth value of the statement 'M acts.' Inner experience, however, does not allow us to explore the consciousness of other people---at least not without analogical conclusions. Such an analogy inference involves induction. According to Rothbard, however, inductive methods are not possible or necessary in the sphere of human action. By Rothbard's own standards, not even the proposition 'M acts' is intersubjectively verifiable. How much more problematic is the demand that the fundamental axiom 'All people act' can be established as true.



Furthermore, it is dubious how Rothbard's account can show that a~falsification of the fundamental axiom is inconceivable. This criterion (I) does not concern falsifiability in a~Popperian sense but negates the existence of a~consistent alternative. Yet, behaviorism and the ontology of what Mises calls ``primitive man'' contradict the fundamental axiom ``man and only man acts'' by attributing purpose, goals, desires, and beliefs to no objects at all (behaviorism) or to more objects than human individuals respectively (``primitive man'' speaks of angry rivers and sad clouds and their intentions).\footnote{Cf. the discussions of these alternatives and their acknowledgement by Mises in Linsbichler 
%\label{ref:RNDoaDrm74RGo}(2017; 2021a).
\parencites*[][]{linsbichler_was_2017}[][]{linsbichler_austrian_2021}.%
} Behaviorist monism may be rejected for pragmatic reasons, as conventionalist praxeologists and arguably Mises do. But theories in which other people merely behave instead of acting purposefully can be conceived and formulated without special problems. Indeed, some radical post-Humean empiricists call for people to be treated in social scientific theories in the same way as animals, plants, crystals, buildings, swamps, rain, rivers, cities, X-rays, and the Milky Way 
%\label{ref:RNDiEouYpi0iB}(see Neurath, 1944; 1970).
\parencite[][]{Foundations1970}.%




Rothbard's attempt to establish certainty, intersubjectivity, and truth for the fundamental axiom is on shaky ground.\footnote{Since Tarski's work, the conceptual distinction between certainty and truth can be made without epistemological concerns. ``Once this is noted, it is obvious that truth is distinct from certainty and that the supposed unattainability of the latter does not undermine the legitimacy or utility of the former'' 
%\label{ref:RND9FFZlBGahH}(Soames, 1999, p.32).
\parencite[][p.32]{soames_understanding_1999}.%
} Vague references to a~specific source of knowledge cannot close the gaps in the arguments needed. A~more precise specification of procedures and methods of application of introspection would most likely reveal its inductive character. Moreover, the object of cognition---the category of one's own actions---is not accessible to others.



\section{Appraisal of Linsbichler's conventionalist praxeology according to Rothbard's criteria}


\begin{flushright}
\textit{``You can get it if you really want.``} (Jimmy Cliff)
\end{flushright}






In sections 1,3, and 4, we outlined the conventionalist research program with an analytic fundamental axiom as a~starting point. Despite Rothbard's deprecating stance towards such a~project, we now investigate whether it meets the four criteria he staunchly upholds.

\medskip

\noindent (I) A~falsification of the fundamental axiom is inconceivable. This is the most challenging hurdle to overcome. While conventionalism fares better than essentialism, neither approach fully meets the requirement.



Although Rothbard does not refer to the standard notion of falsifiability, note that such a~Popperian falsification is hard to achieve for common versions of the fundamental axiom. Which observational statements would contradict that human individuals and only human individuals behave purposefully, i.e. act? The goals, preferences, and knowledge which according to praxeology play a~crucial role in acting are not directly observable. Maybe humans do not have goals, maybe door handles do, but how could we experience this? Following Mises 
%\label{ref:RNDmwS0e6eH6O}(1940, p.85),
\parencite*[][p.85]{mises_nationalokonomie_1940}, %
 one might consider future improved neurophysiological aids for falsifying the fundamental axiom. Such prospective methods would identify observable physical and chemical processes in the brain with the very content of specific thoughts. Such a~decision between behaviorism and praxeology would, among other things, depend on non-trivial theories of translation, though.



The very fact that---except for debatable future neurophysiology---no potentially observable states of affairs are excluded by the fundamental axiom motivated the very idea to construe it as analytic and renders the variant of conventionalism regarding the fundamental axiom ``one of the least controversial versions'' 
%\label{ref:RNDc1cn3lAhzP}(Linsbichler, 2021a, p.3371).
\parencite[][p.3371]{linsbichler_austrian_2021}.%




Reverting back to Rothbard's demand, alternatives to the fundamental axiom remain conceivable, no matter what essentialist or conventionalist arguments are brought forward. Even when granting the validity of introspection for one's own mental states, it remains possible and conceivable---in principle---that all other human individuals do not act but merely behave. Metaphysical speculation cannot definitively decide the prima facie logical tie between praxeology and behaviorism either, as Mises actually acknowledges at one point 
%\label{ref:RNDbJIGryZ1C1}(Mises, 1940, pp.84–86).
\parencite[][pp.84–86]{mises_nationalokonomie_1940}.%




What conventionalism---in contrast to essentialism---can provide is an approximation to meeting criterion (I). Once a~specific version of the fundamental axiom is construed as analytic, praxeological reasoning proceeds in a~framework in which it is true by definition. Thus, within the conceptual scheme of this framework, a~negation of the fundamental axiom is analytically false after all. In this limited sense, criterion (I) is almost fulfilled, as long as the economist stays within her linguistic framework. Of course, she can step out of her linguistic framework, abandon her research program, and conceive of behaviorism in a~meta-language. These remaining objections to Rothbard's criterion (I) are ultimately unavoidable.

\medskip

\noindent (II) The fundamental axiom is empirically meaningful. In the conventionalist research program, the term `human individuals' is intended to designate objects in the physical world,\footnote{Strictly speaking, theories are not interpreted in ``the physical world'' or ``reality'' but in the model(s) which serve as a~proxy for the ``real world'' 
%\label{ref:RNDa93nusHe6z}(cf. Linsbichler, 2023a; Przelecki, 1969).
\parencites[cf.][]{linsbichler_ultra-refined_2023}[][]{przelecki_logic_1969}.%
} even though some borderline cases might be left undecided. On top of that, in all likelihood, the fundamental axiom can be ascribed the truth value 'true' in all situations without changing the observable extension of ‘human' and by merely tinkering with the structure of theoretical terms if need be. The fundamental axiom is thus clearly empirically meaningful in the weak sense demanded by Rothbard.

\medskip

\noindent (III) The fundamental axiom is a~priori with regard to complex historical events. The fundamental axiom is analytic, and therefore a~priori to any experience, both in the sense of epistemology and in the sense of primacy. Capturing complex historical events is not possible without praxeology or a~comparable other theory.



\begin{quote}
It is only with the aid of a~theory that we can determine what the facts are. Even a~complete stranger to scientific thinking, who naively believes in being nothing if not ``practical,'' has a~definite theoretical conception of what he is doing. Without a~``theory'' he could not speak about his action at all, he could not think about it 
%\label{ref:RNDz23fAOhBEi}(Mises, 2003, p.29).
\parencite[][p.29]{mises_epistemological_2003}.%
\end{quote}


\medskip

\noindent (IV) The fundamental axiom is absolutely true. In the conventionalist approach, the fundamental axiom is always true, namely true per conventionem. This result is already an improvement over the arguments available to the essentialist. Yet, Rothbard's criterion requires ``absolute'' truth (whatever that exactly amounts to). If we interpret this as being true in all linguistic frameworks, no matter how the terms are defined in them, any justification must obviously fail. The sentence ‘Murray is a~libertarian' is true if the terms have their usual meaning, but we can easily render the sentence false by changing the meaning of ‘Murray' or of ‘libertarian'. And to ask for the ‘truth' of a~sentence, \textit{independently} of a~framework which assigns meanings to the sentence, is unintelligibly with standard notions of truth. No sentence, considered as a~purely syntactic string of signs, is true independently of the meaning attached to it.



It is neither clear whether Rothbard demands necessary truth for the fundamental axiom, nor which notion of necessity such a~demand would draw on. We therefore suspend judgement on whether a~conventionalist justification substantiates the necessary truth of the fundamental axiom. However, Linsbichler's conventionalist praxeology does offer more than mere plain truth. The fundamental axiom is \textit{analytically} true, and thus can be plausibly interpreted as fulfilling (IV). Our analysis results in the following summarizing chart:\footnote{The quotation marks serve as a~reminder that Rothbard uses these terms with idiosyncratic meaning.}






\begin{table}[H]
    \centering
    \begin{adjustbox}{max width=\textwidth}
        \begin{tabularx}{\textwidth}{|L{4.5cm}|Y|Y|}
            \hline
            & \textbf{Rothbardian Essentialism} & \textbf{Conven\-tion\-alism} \\ \hline
            \textbf{not conceivably ``falsifiable''} & \ding{55} & (\ding{51}) \\ \hline
            \textbf{``empirically meaningful''} & \ding{51} & \ding{51} \\ \hline
            \textbf{\textit{a priori} to complex historical events} & \ding{51} & \ding{51} \\ \hline
            \textbf{absolutely true} & \ding{55} & \ding{51} \\ \hline
        \end{tabularx}
    \end{adjustbox}
%    \caption{Comparison between Rothbardian Essentialism and Conventionalism}
\end{table}

Rothbard does not provide a~conventionalist defense of praxeology at all. His methodological and epistemological writings do not even contain the traces of this idea which we find in Mises. Given the problems Rothbard's own justification of praxeology faces in light of his self-imposed criteria, perhaps he should have considered conventionalism after all.



\section{Outlook: What is to be done?}

This paper, hopefully, clarified some details of Linsbichler's conventionalist defense of praxeology and compared its merits with Rothbard's essentialist arguments. Whereas previous work 
%\label{ref:RNDhnBqp7f94K}(Linsbichler, 2017; 2021a; Lipski, 2021; Scheall, 2017; Tokumaru, 2018)
\parencites[][]{linsbichler_was_2017}[][]{linsbichler_austrian_2021}[][]{lipski_austrian_2021}[][]{scheall_review_2017}{Scheall2017a}[][]{tokumaru_review_2018} %
 mainly emphasized the superiority of a~conventionalist defense from the external stance of empirically minded contemporary philosophy of science, the paper at hand takes a~different approach. Using Rothbard's methodological and epistemological writings as the source, we reconstructed four desired properties of a~fundamental axiom which its justification should be able to establish. These four are by no means the only internal Austrian criteria by which conventionalism can be evaluated but at least it passed this first test more successfully than intuitive universalism.



One---but certainly not the only--- major open problem that remains to be addressed is the exact formulation of the fundamental axiom, ideally both in natural language and in a~formal language.\footnote{The structure of the fundamental axiom explicated in section 1 of this paper as well as the partial definitions and proofs in Oliva Córdoba 
%\label{ref:RNDHxH5dUvjj8}(2017)
\parencite*[][]{oliva_cordoba_uneasiness_2017} %
 are first steps in that direction. See also Linsbichler 
%\label{ref:RNDATKCNmFn9u}(2023b).
\parencite*[][]{linsbichler_case_2023}.%
} This task is vital for all praxeologists because it enables an assessment whether certain intended consequences are actually derivable deductively. If the fundamental axiom is construed as analytic, the issue is even more pressing. It would be desirable to establish that such a~construal is possible.



\paragraph{Acknowledgments.}
I~am grateful to two anonymous reviewers, Agustina Borella, Daniel Eckert, David Mayer, Karl Milford, Berthold Molden, William Peden, Scott Scheall, Lukas Starchl, Richard Sturn, Igor Wysocki, and Gabriel Zanotti for helpful discussions, questions, and comments.



\paragraph{Funding information.} This research was funded in whole or in part by the Austrian Science Fund (FWF) [grant DOI {10.55776/ESP206}]. For open access purposes, the author has applied a~CC BY public copyright license to any author-accepted manuscript version arising from this submission.

\end{artengenv}
\label{linsbichler-last}
