\begin{artengenv2auth}{Anna Ceglarska, Katarzyna Cymbranowicz}
	{The role of \textit{phronesis} in Knowledge-Based Economy}
		{The role of \textit{phronesis} in Knowledge-Based Economy}
		{The role of \textit{phronesis} in Knowledge-Based Economy}
	{University of Sydney}
	{The aim of this paper is to reflect on contemporary understanding of ``knowledge'' within the Knowledge-Based Economy. Since the pursuit of knowledge has been a~longstanding focus of European culture since Greek philosophy, we employ the original ancient terminology. Applying the hermeneutics of ancient texts along with critical and comparative analysis can aid in differentiating between ``knowledge'' and ``wisdom'', often linked in modern theories, while also connecting this issue to the Aristotelian concept of \textit{phronesis}. The authors argue that since human relations impact social (and so---economic) spheres, the issue of phronesis, a~relational type of knowledge, should not go unexamined. The idea that application of knowledge (rather than its mere acquisition), crucial for the Knowledge-Based Economy, was embedded in the Greek term \textit{oikonomiké}, which provides a~basis for considering oneself a~\textit{phronimos}. Our aim is to demonstrate the value of phronesis particularly in the fields of management and the philosophical foundations of economics, as the skills encompassed within it have the potential to aid in educating not only a~``sage'' but also an active member of the community, capable of acting in a~manner that benefits both themselves and the society.
		}
		{wisdom, knowledge, knowledge-based economy, phronesis, Aristotle, prudence, entrepreneurship.}
	{%
		{\flushright\subbold{Anna Ceglarska}\\\subsubsectit\small{Jagiellonian University}\label{ceglarska-firstpage}\par}%
		{\flushright\subbold{Katarzyna Cymbranowicz}\\\subsubsectit\small{Krakow University of Economics}\par}%
	}




\section{Introduction}

\lettrine[loversize=0.13,lines=2,lraise=-0.03,nindent=0em,findent=0.2pt]%
{T}{}he inspiration for this text was the question posed by David Rooney and Bernard McKenna in their article: ``Should the Knowledge-based Economy be a~Savant or a~Sage?'' 
%\label{ref:RNDHcH6MjXY2B}(2005).
\parencite*[][]{rooney_should_2005}. %
 The answer we propose is---neither. The problem that Rooney and McKenna pose is: should we not be demanding an economy based on wisdom, i.e. that decision-makers should not only acquire knowledge but also the wisdom that results from it? On the basis of ancient considerations, we put up for discussion the fact that replacing knowledge with wisdom (or making the Knowledge-Based Economy a~Sage) may not be as desirable as it seems. Yet the alternative seems to be the economy based on the knowledge of technocratic experts, who could measure its development with indicators, treating it like another material that can be measured and formed by some higher authority. This opening question firmly establishes a~distinction between a~fully ethical and almost altruistic economy (that of the sages), or a~technocratic one, focused on the goal, expressed in sets of indicators (that of the savants). And given the importance of the Knowledge-Based Economy nowadays, the first alternative is quite tempting. The idea that wisdom should govern our lives in all aspects: political, economic, social, etc. is not new. As Alfred North Whitehead famously said: ``The safest general characterization of the European philosophical tradition is that it consists of a~series of footnotes to Plato'' 
%\label{ref:RNDh1GxWPqkx7}(Whitehead, 2010, p.39).
\parencite[][p.39]{whitehead_process_2010}. %
 And yet, according to Karl Popper, among others, Plato's ideal state, ruled by wise men (we deliberately avoid the term ``philosophers'' here), is seen as a~pre-totalitarian one, denying its citizens most of the freedoms we cherish today.



The question that we see posed before us is not the one that obliges us to choose between savant, sage, or a~fool or an \textit{astute ignoramus}, as these terms are also used by Rooney and McKenna 
%\label{ref:RNDqeDmHPcgRi}(2005, p.315),
\parencite*[][p.315]{rooney_should_2005}, %
 but rather between a~sage, a~savant, and a~prudent man, who, as we argue, contrary to the aforementioned authors and following the footsteps of Aristotle and Greek philosophy, is not the same as the wise-man. Since ancient times, knowledge has had many different names, referring to its different aspects and qualities: \textit{episteme} (scientific knowledge), \textit{techne} (technical knowledge) and \textit{phronesis} (practical knowledge, prudence), while \textit{sophia} has been associated with wisdom 
%\label{ref:RNDulINJNZRMa}(Aristotle, 1934, VI.6.3).
\parencite[][]{rackham_nicomachean_1934}. %
 The latter has also been perceived for centuries as the ultimate goal of man and has been the object of interest and analysis since ancient times. The wise-man knows everything and can therefore make the best decisions and give the best advice. In Plato's ideal state the philosophers---those who love wisdom---should be the decision-makers. Instead, we would propose to understand the ``knowledge'' in the Knowledge-Based Economy as the Aristotelian concept of \textit{phronesis}, usually translated as ``practical knowledge'' or ``prudence''. Therefore, our first goal is to establish the difference between \textit{sophia} and \textit{phronesis} in the present-day world. We wish to reflect on today's understanding of ``knowledge'' within the Knowledge-Based Economy and, by referring to ancient terminology, to determine if there has been an unjustifiable association of ``knowledge'' with the notion of ``wisdom''. Furthermore, we would like to consider whether this ``knowledge'' is not treated similarly to all other resources, such as labour and capital. Finally, we shall emphasise the importance of the relational nature of knowledge, which involves not only acquiring it but also the ability to apply it in practical situations. That would lead to our second goal: to see whether, and if so, how, the concept of \textit{phronesis} can be found useful in current interpretations and what can still be learnt from it. Given the specificity of the \textit{phronesis} itself, as it relates to individuals and their relations with the community, we aim to demonstrate the value of \textit{phronesis} for leaders and managers\footnote{Since we are basing our discussion on Aristotle's concepts, we use the terms ``manager'', ``leader'', and ``phronimos'' interchangeably, as Aristotle viewed these spheres of life (economic, social, and political as well) as deeply interconnected. Further elaboration on Aristotle's thoughts regarding the relationship between the economic and political spheres is provided in the subsequent chapter.}, as it pertains to the relational aspects of human society. Our argument is that the ability to function in relationships cannot be left unexamined, as the ``human factor'' significantly affects both economic and social spheres.



In order to achieve these goals, we have mostly employed a~desk research method combined with hermeneutics of ancient texts, enriched with the elements of the critical analysis, developed by CLS. The comparative analysis was undertaken to confront the meaning of different types of knowledge in our source material (Aristotle) and its modern interpretations. As we were primarily interested in exploring the philosophical foundations of economics, we deliberately avoided researching specific economic trends that would require the research methods and techniques characteristic of the discipline of economics. First, we will discuss some basic ancient concepts, starting with the \textit{oikonomiké} (\textit{oeconomica}) itself, as it shall, along with its differentiation from \textit{chrematistiké}, will lead us to an important distinction between ``action'' and ``accumulation''. The ability to ``act'' we see as a~fundamental value that leads to the attainment of knowledge, a~good life and general progress. The next part shall be devoted to establishing some general meaning of the Knowledge-Based Economy and within this we will consider whether, despite the lexical distinction, we have not made an unauthorised identification of knowledge with wisdom in contemporary discourse.



These two concepts have been distinguished since ancient times. Today, however, most theories focus on what we possess, be it wealth, capital or knowledge, rather than on what we do with this good (or how we do it). We therefore remain mostly in the realm of \textit{chrematistiké}, and---transferring it to the philosophical realm---\textit{vita contemplativa}. A~true sage who lives such a~life becomes more and more immersed in himself through contemplation, striving to see the Good and the Truth---universal and unchanging ideas\footnote{At this point, it should be noted that the terms Good and Truth, written with a~capital letter in the text, denote these values understood as the highest ones, ideas in the Platonic sense. ``Ordinary'' good and truth (in lower case), which appear e.g. in the context of the Aristotelian ``good life'', are already relativised. Hence, they are not Platonic Ideas, but only exist in the realm of opinion (\textit{doxa}).}---and thus alienates himself from other people. On the other hand, \textit{vita activa}, like \textit{oeconomica}, is concerned with human affairs, relationships and actions that arise from ``the fact that men, not Man, live on the earth'' 
%\label{ref:RNDe9sDCL8Id7}(Arendt, 1998, p.7).
\parencite[][p.7]{arendt_human_1998}. %
 It belongs to the realm of \textit{praxis}, which includes all kinds of active engagement with the things of this world. Thus, politics, like economics, is about people and is only realised in relationships with others and in action.



\section{From \textit{oikos} to economy}

The term ``economy'' itself comes from the Greek \textit{oikos} (\textgreek{οἶκος}),
%($o\text{\textgreek{>~i}}\kappa o\varsigma $),
 which is often translated simply as ``household'', but it is worth noting that its use in Greek is often broader and also changes depending on the context 
%\label{ref:RNDgh8e6xnQbO}(Roy, 1999, p.2).
\parencite[][p.2]{roy_polis_1999}. %
 For \textit{oikos} is not only a~mere ``house'' in the sense of building, but also ``home'', understood as a~community living in the same house (in the narrowest sense---family; however, it could also include more distant relatives, servants or slaves), but it could also be used to mean the ``king's house'', as the whole dynasty 
%\label{ref:RND9lTeKUR8P5}(Herodotus, 1920, V.31, VI.9)
\parencite[][V.31, VI.9]{herodotus_histories_1920} %
 and in material terms was akin to the \textit{patrimonium}, the inheritance running through the whole lineage, and extending beyond a~single generation. Thus, while \textit{oikos} was a~basic social unit, it was not limited to what Cheryl Anne Cox 
%\label{ref:RND4DhHfzL5Yj}(1998)
\parencite*[][]{cox_household_1998} %
 calls the ``nuclear family'', since the activities associated with \textit{oikos} could be, in today's sense, strictly private ones (concerning only the family unit), but also extended to public affairs and people outside the particular \textit{oikos} 
%\label{ref:RNDMrQ2gLXGjK}(Martin, 2016).
\parencite[][]{martin_urban_2016}.%




The inherent relationship between these two spheres of life---the private and the public---is a~point of emphasis for Werner Jaeger: man leads ``besides his private life a~sort of second life, his \textit{bios politikos}. Now, every citizen belongs to two orders of existence'' 
%\label{ref:RNDGM9BYZCb6p}(Jaeger, 1946, p.111).
\parencite[][p.111]{jaeger_paideia_1946}. %
 Indeed, there were important and unmistakable links between the \textit{oikos} and the \textit{polis} (state). For Aristotle 
%\label{ref:RNDKq7obwJ3DY}(Aristotle, 1934, 1253b.1),
\parencite[][1253b.1]{rackham_nicomachean_1934}, %
 the \textit{oikos} is the basis for the functioning of society and the state, the smallest unit of the human community. Because of the primacy of the community prevailing in ancient Greece, actions within the \textit{oikos}, while remaining private, had political significance for the \textit{polis} \label{ref:RNDgDEuVBJb3a}\textit{(Roy, 1999, p.4)}. Therefore, even if treaties devoted to ``economy''\footnote{From which the most famous ones were the dialogue of Xenophon (\textit{Oeconomicus}) and a~treatise attributed to Aristotle (\textit{Oeconomica}).} dealt mainly with the household management, some scholars emphasize that they could also be seen as a~guide to the successful management of state. Aristotle 
%\label{ref:RND6ONKIbu2Pz}(1920, p.1345b)
\parencite*[][p.1345b]{aristotle_oeconomica_1920} %
 for example distinguishes the type of ``political economy'', which seems to be the closest to our current understanding\footnote{Especially since, due to changes in the political arena, other types distinguished by the philosopher (namely royal and satrapic) have been absorbed into it, as the coinage, exports, imports or taxes are now also the areas of state regulation and action.}.



Naturally, the direct application of the Aristotelian framework of concepts and definitions raises a~number of questions today. It has been argued more than once in the literature that the Stagirite used a~different concept of the economy, focusing primarily on domestic issues. However, as Ricardo Crespo 
%\label{ref:RNDn6ZuWTwuj2}(2010)
\parencite*[][]{crespo_aristotle_2010} %
 rightly points out, this does not imply that Aristotelian thought is completely irrelevant, nor that it lacks links or foundations for today's thinking about the economy, its goals and the rules that govern it. However, our intention is not to analyse the Aristotelian concept of the economy, as this has already been done quite well by many scholars 
%\label{ref:RNDbG68dQ89fZ}(e.g., Crespo, 2006; 2010; Meikle, 1995; Soudek, 1952; Pack, 2008; Finley, 1970).
\parencites[e.g.,][]{crespo_ontology_2006}[][]{crespo_aristotle_2010}[][]{meikle_aristotles_1995}[][]{soudek_aristotles_1952}[][]{pack_aristotles_2008}[][]{finley_aristotle_1970}.%




In the context of our analysis, it is worth noting Aristotle's distinction between \textit{oikonomiké} and \textit{chrematistiké}. Moreover, we would be well-advised to reflect on the implications of this division. In \textit{Politics} Aristotle poses the question: is the art of the acquisition of wealth
(\textgreek{χρηματιστική})
%($\chi \rho \eta \mu \alpha \tau \iota \sigma \tau \iota \kappa \text{\textgreek{'h}}$)
the same as the art of the management of the household
(\textgreek{οἰκονομικῇ}).
%($o\text{\textgreek{>i}}\kappa o\nu o\mu \iota \kappa \text{\textgreek{~h|}}$).
The answer he gives is negative 
%\label{ref:RNDlYCRRM81fB}(Aristotle, 1944, 1.1256a).
\parencite[][1.1256a]{aristotle_politics_1944}. %
 Whereas \textit{chrematistiké} is concerned with the acquisition of wealth, \textit{oikonomiké} focuses on its use. Moreover, if the accumulation of wealth (\textit{chrematistiké}) consists solely in the pursuit of capital accumulation and the expansion of one's wealth, rather than in the acquisition of things necessary to live, or to love well, it becomes, for Aristotle, something contrary to nature:



Consequently some people suppose that it is the function of household management to increase property, and they keep believing that it is their duty to be either safeguarding their substance in money or increasing it \textit{ad infinitum}. The cause of this state of mind is that their interests are set upon mere life but not upon the good life 
%\label{ref:RND4UP3Blb0Hi}(Aristotle, 1944, 1.1257b).
\parencite[][1.1257b]{aristotle_politics_1944}.%




Given the fact that the ``good life'' was the main objective of the \textit{polis} and that moderation was one of the most important virtues not only for Aristotle but for other philosophers as well, we can clearly see that the mere ``accumulation'' of wealth is not a~suitable way to live and manage the household, i.e., to participate in the broadly understood economy. The natural, proper acquisition of wealth occurs when the goods are being used not to obtain more goods, but for a~good life. Therefore, the basic factor that distinguishes between the natural, useful way of accumulating wealth from the unnatural, and therefore requiring the introduction of certain restrictions, is the way in which the accumulated wealth is used, or more precisely, how it is acted upon. Economics presupposes precisely the use and therefore the action on and with the goods. As stated by R. Crespo 
%\label{ref:RNDPkfW5EhU4e}(2006, p.772),
\parencite*[][p.772]{crespo_ontology_2006}, %
 ``\textit{Oikonomiké} is an action of using, in Greek, \textit{chresasthai}''.



This focus on action and human activity, particularly within a~community like the \textit{oikos} or \textit{polis}, appears to be fundamental in Aristotle's philosophy. It also applies to other areas of life and, in our opinion, can and should be used when interpreting the fundamental goals and objectives of the Knowledge-Based Economy.



\section{What is Knowledge-Based Economy?}

However, as the initial definitions of economics placed a~stronger emphasis on wealth, the notion of the Knowledge-Based Economy presents a~substantial breakthrough. This concept deviates significantly from these early endeavours in defining the economy and its corresponding regulations and customs, given that knowledge---rather than material goods, their value and distribution---now plays a~crucial role. In the 21\textsuperscript{st} century, knowledge became the crucial element at the heart of management theory and the pursuit of economic achievement. And so, ``Knowledge-Based Economy'' can be included into the rich catalogue of modern economic schools that focus on sustainable development 
%\label{ref:RNDqB6xoBbNAW}(e.g., Rogall, 2010; Shmelev, 2012; Raworth, 2017; Govender, 2021).
\parencites[e.g.,][]{rogall_ekonomia_2010}[][]{shmelev_ecological_2012}[][]{raworth_doughnut_2017}[][]{govender_rise_2021}. %
 It strives for development through the rational use of human resources (information, knowledge), not natural resources (land, raw materials) or financial resources (capital). Therefore, according to its assumptions, if we want to develop, we should invest in human capital. This leads to the conclusion that knowledge as a~resource plays an important role in shaping the socio-economic reality and effective harnessing of knowledge potential, including human intellectual potential, science, and research and development sphere, are the strategic factors that determine the pace and extent of socio-economic development today 
%\label{ref:RND8qWhMOHbcZ}(Skrzypek, 2012, p.193).
\parencite[][p.193]{skrzypek_gow_2012}. %
 In light of the above, it can be concluded that the continuous creation and use of knowledge is a~source of innovation and provides innovative solutions that are the basis for the creation of Knowledge-Based Economy 
%\label{ref:RNDgzfW31E5S6}(Zienkowski, 2003, p.15).
\parencite[][p.15]{zienkowski_gospodarka_2003}. %
 While traditional factors such as land, natural resources, labour or capital continue to impact socio-economic development opportunities, knowledge plays a~crucial role as it not only acts as a~new factor of production, but also coordinates others. The significance of knowledge is continuously rising, rendering it the primary factor of production and the key source of wealth. Elżbieta Skrzypek's statement that knowledge is the ``raw material of the future'' and the currency of the ``new economy'' is fitting in this context 
%\label{ref:RNDXIB6aQVPXn}(Skrzypek, 2018, p.20).
\parencite[][p.20]{skrzypek_gow_2018}.%




The phrase ``Knowledge-Based Economy'' is a~fairly recent addition to the world's literature, but it is receiving increasing attention. The first academic study to define the ``knowledge-based economies'' as ``economies which are directly based on the production, distribution and use of knowledge and information'' was a~research report produced by the OECD, entitled \textit{The Knowledge-Based Economy} 
%\label{ref:RNDsk8yzt3tgB}(OECD, 1996, p.7).
\parencite[][p.7]{oecd_knowledge-based_1996}. %
 Since then, however, the term ``Knowledge-Based Economy'' has not received a~single universally accepted definition although it should be noted that the vast majority of proposed definitions are based on an attempt to list its distinctive features
%\label{ref:RNDTTTY3NLs4T}(Chojnacki, 2001, p.80).
\parencite[][p.80]{kuklinski_wiedza_2001}. %
 However, this poses a~significant problem which was observed \textit{nota bene} already in ancient times. In the Platonic dialogue \textit{Meno}, Socrates points out that providing numerous characteristics or instances of a~defined concept does not significantly aid our comprehension but rather poses additional issues. Using the case of colours, Socrates argues to Meno that attempting to define colour by giving an example or even listing all possible colours does not bring us any closer to a~general definition of what colour itself is. Furthermore, encountering a~new phenomenon poses a~significant problem in attributing it. A~better approach is to seek identifying features that they share in common 
%\label{ref:RNDtTciZTgOzk}(Plato, 1967a, 74c-77a).
\parencite[][74c-77a]{plato_plato_1967}. %
 Studies and analyses concerning the Knowledge-Based Economy encounter a~similar challenge. While emphasising the importance of information, knowledge, and intellectual capital in modern society and economy, they tend to expand this list to include other elements such as data, experience and wisdom. This catalogue of attributes can be almost endless, but paradoxically, it can divert attention from the fundamental concept. This can be clearly observed in the literature on the subject, where the general term ``Knowledge-Based Economy'' itself has several significant ``competitors'' to claim the title of the most precise definition of contemporary socio-economic reality. These include, among others: ``new economy'', ``digital economy'', ``knowledge-driven economy'', ``post-industrial economy'' or ``post-industrial society'', ``post-capitalist economy'', ``network economy'', ``third wave economy'' (or, again, society), ``service economy'', ``intangible resource economy'', ``information age'', ``knowledge society'' and several others. All of them describe the same reality, but it is impossible to definitively determine which set of characteristics accurately represents the current state of affairs.



Hence, without attempting a~universal definition, yet emulating Socrates in search of a~shared feature among them, enabling us to affirm that they ``all have one common character'' 
%\label{ref:RND3zSfnyjVk9}(Plato, 1967a, 72c),
\parencite[][72c]{plato_plato_1967}, %
 our attention centres on the realm of knowledge, with the aim of highlighting some fundamental problems.



\section{Between knowledge and wisdom}

Various definitions of knowledge can be found in the literature. This presents a~terminological challenge which persists in a~Knowledge-Based Economy 
%\label{ref:RNDtkuJnEpVTX}(Winter, 1987; OECD, 2000).
\parencites[][]{teece_knowledge_1987}[][]{oecd_knowledge_2000}. %
 This is due to the fact that knowledge is an elusive resource that is complex to define, measure and apply in practice, given the limited conceptual resources, methods and techniques that are available in the current, post-industrial era 
%\label{ref:RNDFkXjpIAIPC}(Strojny, 2000, p.20).
\parencite[][p.20]{strojny_zarzadzanie_2000}.%




Peter F. Drucker 
%\label{ref:RNDuYUbdmITLw}(2013, p.7)
\parencite*[][p.7]{drucker_post-capitalist_2013} %
 emphasises that:
\begin{quote}
 […] the basic economic resource---‘the means of production' to use the economist's term---is no longer capital, nor natural resources (the economist's ‘land') nor ‘labour' […] Value is now created by `productivity' and `innovation', both applications of knowledge to work. The leading social groups of the knowledge society will be ‘knowledge workers'---knowledge executives who know how to allocate knowledge to productive use.
\end{quote}
 Marcin Kłak 
%\label{ref:RNDR8f84tvwst}(2010, p.42)
\parencite*[][p.42]{klak_zarzadzanie_2010} %
 highlights that this unique situation is a~result of knowledge's indeterminate nature and the need for constant renewal, updating and modification. Only knowledge that is applied has any value as it serves progress, development and change, in other words, it is useful.



The pursuit of systematisation and the effort to create reasonably uniform yet comprehensive definitions of scientific concepts have generated several definitions of knowledge. According to Thomas H. Davenport and Laurence Prusak 
%\label{ref:RND6Vr58dCvG9}(1998),
\parencite*[][]{davenport_working_1998}, %
 knowledge, in contrast to data and information, is produced, developed and consolidated in the human mind as a~result of accumulated experience and learning---it is, so to speak, a~``product'' of the human mind, therefore it can be classified as either conscious (acquired systematically and intentionally through education) or unconscious (acquired unsystematically and unintentionally). In light of the above, it can be argued, in line with Michael Polanyi's thinking, that individuals are not always conscious of the knowledge they possess, and therefore also unaware of its worth 
%\label{ref:RND98uFE4rc5d}(Polanyi, 1966, p.37).
\parencite[][p.37]{polanyi_tacit_1966}. %
 Thus, data and information form the foundation of knowledge, which only becomes knowledge after it has been analysed. It is noteworthy to mention in this context the definition of knowledge proposed by Wiesław M. Grudzewski and Irena K. Hajduk, who differentiate between the concept of knowledge, understood as the application of information in practice, and wisdom, which is a~combination of knowledge, intuition and experience 
%\label{ref:RNDh20Yn52Jk1}(Grudzewski and Hejduk, 2004, p.73).
\parencite[][p.73]{grudzewski_zarzadzanie_2004}. %
 The distinction between these two elements, knowledge and wisdom, also dates back to antiquity. The philosopher, according to Plato, is defined as the one who ``loved wisdom'' (\textit{sophia}), and the acquisition of it constitutes his ultimate goal and desire. The famous metaphor of the cave depicted in the book VII of the \textit{Republic}, portrays \textit{sophia} as the sun, the source of pure, primal light that is, however, unattainable for most individuals 
%\label{ref:RNDGXaj6zfN6D}(Plato, 1969, VII, 514-516).
\parencite[][VII, 514-516]{plato_plato_1969}. %
 They sit in the cave observing only shadows, which are imperfect representations of the true object. The philosopher, however, can liberate themselves from constraints and, upon exiting the cave, step out into the sun and see the Truth.



The association of wisdom with the Truth holds significance in this context. An average individual typically possesses mere opinions (\textit{doxa}). Such opinions can have varying degrees of accuracy (or inaccuracy), lack the quality of certainty and completeness. In Plato's view, opinion is starkly contrasting to truth 
%\label{ref:RNDu3fhsJne72}(Arendt, 2005, pp.7–8).
\parencite[][pp.7–8]{arendt_promise_2005}. %
 Thus, while the multiplicity of opinions allows for discourse and persuasion, which are, after all, the foundation of Athenian politics, Truth is not subject to doubt or criticism. Furthermore, someone who has attained knowledge of the Truth through this intuition often chooses to retreat into \textit{vita contemplativa}, instead of taking action in a~social field, for they are incapable of describing the ``light'' to people mired in darkness. Such a~person does not receive understanding or attention from society, and he himself above all wishes to see more, to know more. The ancient \textit{sophia}, the knowledge of the sage, was the knowledge of the observer who merely watches the truth without interacting with it\footnote{This matter is also connected to the notion of ``theory''. The \textit{theoroi} were special envoys who observed customs and rituals in other \textit{poleis} (without engaging in them) and then reported their observations, enriching the knowledge of their homeland. For a~more comprehensive analysis of the role of observation in Greek culture, see 
%\label{ref:RND2JhLJamaGJ}(Ceglarska, 2022).
\parencite[][]{ceglarska_od_2022}.%
}. Such wisdom is absolute, but only few are able to possess it. Socrates' renowned statement, ``I know that I~know nothing'', arose from the fact that, unlike others, he was aware of his own limitations.



The paradox of modern understanding is that we expect the ``wise'' to possess full knowledge while being capable of challenging it. The wise person understands the workings of various social domains, including political, economic, and cultural aspects. Based on this understanding, they can accurately predict behaviours, consequences, and changes. However, their opinions may face criticism. As wise-men, they should be capable of defending their viewpoint and persuading others of its validity. Thus, they are, firstly, closer to Socrates, who walked among people, questioned and taught them, than to Plato, who preferred to observe. Secondly, they should possess the ability to accomplish what the archetypal philosopher, Socrates, failed to do, namely to persuade others to adopt their viewpoint.



It was actually Plato's disciple, Aristotle, who adopted an approach that aligned more closely with Socrates' beliefs. For Aristotle, relationships play a~fundamental role in the human world, where practical knowledge, rather than wisdom, reigns supreme. For him prudence involves above all the ability to act---and after all, proclaiming one's position, teaching and persuading is an action. It is called \textit{phronesis}.



\section{Practical knowledge in Knowledge-Based Economy}

Based on Aristotle's \textit{Nicomachean Ethics,} it is indicated that the Greeks distinguished between three types of knowledge: \textit{episteme} (scientific knowledge), \textit{techne} (technical, manufacturing knowledge) and \textit{phronesis} (practical knowledge, although perhaps it should rather be called knowledge of action and its consequences, also known as ``prudence'' thanks to Cicero's translation). Notably, the concept of ``wisdom'' is absent from this framework. This is because wisdom is not simply knowledge, but rather something that can only be attained through it, in addition to some other essential elements, as defined earlier by W.M. Grudzewski and I.K. Hajduk. Aristotle believed that one of the most crucial elements is \textit{nous}, which translates to intuitive thinking or intuition. Thanks to \textit{nous}, individuals can discover the initial premises that form the foundation of knowledge even if they are often indescribable. Although a~child may not be able to articulate the laws of physics, they intuitively comprehend the concept of gravitation to a~certain degree; their intuition informs them that objects fall. The possession of this intuition enables further exploration and acquisition of knowledge; however, not everyone possessing it, nor even those who specialise in a~particular field, count as a~``sage''. Socrates raises this matter somewhat mischievously in Plato's \textit{Republic}: ``Is it then owing to the science of her carpenters that a~city is to be called wise and well advised?'', to which his interlocutor replies: ``By no means for that, but rather mistress of the arts of building.'' 
%\label{ref:RNDTn2EF3lYv9}(Plato, 1969, IV, 428b-c).
\parencite[][IV, 428b-c]{plato_plato_1969}.%




``Master'' (of some craft) does not equate to being a~``sage''. This does not disregard the importance of craftsmen and their role in the state, which was considered the optimal community by Greeks. They are essential to meeting the needs of citizens, as are farmers, merchants or warriors (although Plato had reservations with poets). Nevertheless, they lack ``true'' wisdom and only possess the knowledge of a~producer, focused on the goal. It is worth noting that the philosophers included the sophists in this group of specialists in \textit{techne}. According to them, the sophists did not strive to attain \textit{sophia---}wisdom, contrary to their name. Instead, they used their skilful manipulation of language as a~tool to influence, shape, and convince their listeners of their own reasoning, just as a~craftsman skilfully shapes wood to obtain the desired piece of furniture. This is also the foundation of the sophists' teachings: refining the ability to use eloquence in a~competent manner, craftsman-like, rather than seeking the Truth and wisdom. Sophists were not truly ``sages'' but rather ``experts in craft''.



Nowadays, experts are widely respected. Dating back to the era of Saint-Simon, they have been viewed as the individuals who set goals for and direct global development. As it was already stated by Friedrich von Hayek 
%\label{ref:RNDE1fOlW6GAK}(1945, p.521),
\parencite*[][p.521]{hayek_use_1945}, %
 the kind of knowledge we ``expect to find in the possession of an authority made up of suitably chosen experts […] occupies now so prominent a~place in public imagination that we tend to forget that it is not the only kind that is relevant.'' This knowledge of the experts---``scientific knowledge''---is seen as an organized system that encompasses all knowledge and can help define development objectives. It should be noted, however, that this kind of knowledge is not one of the ``sages''. Rather it belongs to the ``savants'' who utilise their accumulated knowledge as a~tool to mould their surrounding reality, similar to how ancient sophists used words. The emergence of a~new \textit{techne} required new craftsmen and tools. As a~result, this kind of knowledge was enclosed within sets of parameters or indicators. The mainstream economy, with a~focus on promoting economic and social development, has embraced GDP per capita as the key indicator of progress. Later on, various alternative measures, including the Human Development Index (HDI) and the Genuine Progress Indicator (GPI), have emerged. However, while indicators can measure progress, they fail to address the fundamental questions: how to achieve balanced development and well-being. How to act? The social (and so---economic) sphere is treated as a~material to work on, ``design'', and the quality of this design is evaluated solely through established indicators. Those appear to be practical, but only in the sense of \textit{techne}, which focuses on a~goal, expressed through the said indicators. Yet, while they aid in influencing societies and governments to reach established goals, yet they do not provide any information regarding actual progress, values, consequences, nor regarding the human actions. The ``savant economy'' can be called a~``technocratic one'', which was defined by Howard Scott 
%\label{ref:RNDAiGwiOTBZ5}(1965, p.10)
\parencite*[][p.10]{scott_history_1965} %
 in the following words: ``Technocracy has proposed the design of almost every component of a~large scale social system.'' It is also knowledge of the experts but intended not to uncover the truth, but instead to manage the unpredictable reality within precise bounds of indicators that give the impression of command over the rapidly changing environment.



Therefore, we are still consequently stuck in the dichotomy between savants and sages. Aristotle, however, leaves a~caveat. While wisdom is the highest value, those who wish to engage with worldly matters, to immerse themselves in \textit{vita activa}, ought to pursue \textit{phronesis}---practical knowledge. Although this pursuit does not result in becoming a~philosopher, it can help one be a~good leader, ruler, or, in modern times, manager, without succumbing to mere ``technical'' or rather ``technocratic'' approach. This \textit{phronetic} knowledge pertains to interpersonal relationships and facilitates a~community's functioning, with the goal of ensuring a~``good life'' for the general public.



In this regard, the goals of ancient philosophers and the Knowledge-Based Economy share a~commonality. They both assume the establishment of a~community, whether political or economic, founded on knowledge. However, this knowledge is not to be understood in abstract, as wisdom or truth (finally, even Plato deemed this impossible to achieve in ``real'' world). It is also not merely a~``technique'' used to control the reality or to reach a~certain goal. Rather, it refers to knowledge that has practical applications and therefore enables peaceful coexistence and development. This particular type of knowledge is called \textit{phronesis} by Aristotle, and the individual who possesses it is called \textit{phronimos}.



As mentioned, Knowledge-Based Economy has mostly integrated the concept of \textit{phronesis} through the work of Ikujiro Nonaka, Ryoko Toyama and Toru Hirata, entitled \textit{Managing Flow. A~Process Theory of the Knowledge-Based Firm}. The Japanese researchers define \textit{phronesis} as a~type of tacit knowledge, ``the ability to grasp the essence of a~situation in process and take the action necessary to create change.'' 
%\label{ref:RNDYTLHA5e0YQ}(Nonaka et al., 2008, p.4).
\parencite[][p.4]{nonaka_managing_2008}. %
 It is a~unique attribute of leaders who strive to benefit the collective interests of the enterprise they manage. According to them, ``\textit{phronesis} synthesizes ``knowing why'' as in scientific theory, and ``knowing how'' as in practical skill, with ``knowing what'' as a~goal to be realized.'' 
%\label{ref:RNDsvtANIHNiA}(Nonaka et al., 2008, pp.14–15).
\parencite[][pp.14–15]{nonaka_managing_2008}. %
 This concept aligns with the economic definition of knowledge put forward by the OECD. The organization has introduced a~classification system that divides knowledge into four distinct categories:
\begin{enumerate}
\item know-what, descriptive-informational knowledge---this is normative knowledge based on experience, context and common sense; it refers to fundamental knowledge used in everyday functioning; its meaning is very close to information and it is easily communicated and passed on;

\item know-how, practical-technological knowledge---it refers to people's skills and capabilities and means the ability to do something; it is instrumental, contextual and related to experience;

\item know-why, exploratory-prognostic knowledge---this is universal and theoretical knowledge; it explains the principles and laws of nature and is closest to what we would call ``scientific knowledge'';

\item know-who, descriptive-informational knowledge---this knowledge mostly refers to information about social relationships, such as who knows whom and what they know. This type of knowledge is becoming increasingly important due to the growing level of specialisation and constant changes 
%\label{ref:RNDqJ8XSdtaFB}(OECD, 1996, p.12; Clarke, 2001, p.190).
\parencites[][p.12]{oecd_knowledge-based_1996}[][p.190]{clarke_knowledge_2001}.%
\end{enumerate}

Considering both the OECD classification of knowledge and Nonaka, Toyama and Hirata's definition of practical knowledge, we can observe that \textit{phronesis} appears to be a~kind of ``super-knowledge'' that combines elements of various knowledge types listed by the OECD. It encompasses both ``knowing why/how/what'' and so is not limited to the Aristotelian concept of the ability to ``calculate well'', but is akin to the all-encompassing ``full knowledge'' of the world that only a~``good manager'' can possess. Meanwhile, as indicated above, in Greek thought there already is an appropriate term for ``certain'' and ``full'' knowledge, namely wisdom (\textit{sofia}). \textit{Phronesis} is not so much knowledge \textit{per se}, but the ability to act. According to Nonaka, Toyama and Hirata 
%\label{ref:RNDRUclcghpN5}(2008, p.53),
\parencite*[][p.53]{nonaka_managing_2008}, %
 it involves ``the ability to determine and undertake the best action in a~specific situation to serve the common good''. Aristotle provides a~seemingly similar definition. In the \textit{Nicomachean Ethics}, he defines \textit{phronesis} as ability to ``deliberate well about what is good and advantageous for himself, not in some one department […] but what is advantageous as a~means to the good life in general'' and also ``rational quality, concerned with action in relation to things that are good and bad for human beings.'' Furthermore, he completes his definition with an example: the one deemed prudent was Pericles, since he was one of the men able to judge ``what things are good for themselves and for mankind'' 
%\label{ref:RNDkorQPLdq55}(Aristotle, 1934, VI.5).
\parencite[][]{rackham_nicomachean_1934}.%




There is a~fundamental difference between these definitions. Whereas Aristotle's definition focuses on the action itself (``to deliberate well'', ``action in relation to things''), later definitions refer to the effects of that action (``action […] to serve common good''). Moreover, in modern definitions \textit{phronimos} is the one who HAS \textit{phronesis}---possesses this ability or skill. In Aristotle, one is CONSIDERED to be a~\textit{phronimos}. Thus, for the Stagirite, the emphasis was on the relationship between the \textit{phronimos} and the community. It was the community, which, judging somebody's actions, could recognise him as the possessor of practical knowledge, and therefore---deem him a~\textit{phronimos}. It was not a~given quality, but one that depended on the judgement of others. This element of judgement firstly established the relationship between the leader and his followers as a~mutual one, and secondly, while allowing the leader to act for his own benefit, it also ensured concern for the benefit of others. However, in later times, \textit{phronesis} came to be identified with one of the many qualities that a~leader is entitled to, that he should acquire and possess as an attribute---another sceptre that he can show to his subjects (or subordinates) to gain their obedience. The Aristotelian \textit{phronesis} was shifted to either \textit{episteme} or \textit{techne}.



This first aspect, the identification of prudence (\textit{phronesis}) with knowledge (\textit{episteme}), is a~particular merit of Christian doctrine. As St Thomas Aquinas notes, Augustine ascribes to prudence ``the avoidance of ambushes'', thus associating it not only with knowledge but also with the most common colloquial understanding of it: the ability to avoid unnecessary risk. St Thomas himself emphasises the ``commanding'' aspect of prudence, since it establishes order and applies the previous judgement, thus restraining the will and ensuring one's proper conduct. It does not allow any action, but only the ``proper'' one---those who sin voluntarily do not possess prudence, since they lack the right reason. Prudence that is ``both true and perfect, […] commands aright in respect of the good end of man's whole life'' 
%\label{ref:RNDDyfEVRPnTI}(Thomas Aquinas, 1947, II-II, q.47 a.8,13).
\parencite[][q.47 a.8,13]{thomas_aquinas_summa_1947}. %
 An important implication follows from this---in Christian thought, the one who has prudence has knowledge of right conduct. Therefore, he does not need recognition from his subordinates; on the contrary, as in Plato's ideal state, they should give him unconditional obedience. St Thomas makes this argument directly in relation to political power---the best system would be a~monarchy, because one person is better able to govern, without having to consult with others and listen to their opinions. For the whole may not be as reasonable and wise as the chosen individual, especially since Aquinas's doctrine assumes that the earthly monarch is a~reflection of the one God, so that the community will be best if it comes as close as possible to the ideal of a~community subject to a~single, eminently wise ruler who most resembles God 
%\label{ref:RNDtHUyoNpZRj}(Thomas Aquinas, 1949, I.2-3).
\parencite[][]{thomas_aquinas_kingship_1949}. %
 The possibilities of opposing the will of this monarch, on the other hand, are relatively limited and concern the situation in which he goes against the word of God---de facto manifesting a~lack of \textit{episteme}, knowledge of higher matters and first premises.



On the other side of the spectrum \textit{phronesis} is placed by Niccolo Machiavelli. For the Florentine philosopher, it becomes identical to \textit{techne}. A~prudent leader is one who knows the secrets of the art of governing and is able to use them to achieve specific goals. In Machiavelli's political theory, the ultimate goal is to raise the state from decline, and so a~good leader needs to ``differentiate between the lion and the fox'' 
%\label{ref:RND7aL8bA6ZR9}(Machiavelli, 2003, p.96),
\parencite[][p.96]{machiavelli_prince_2003}, %
 and so possess a~certain knowledge---not of the highest premises, but a~knowledge of the craft. For a~change, he will not resemble a~Platonic philosopher, but an ancient sophist who, thanks to his knowledge of the art of eloquence, argumentation and rhetorical techniques, will be able to shape the audience to agree with his position and concede the point\footnote{It is worth noting that the Machiavellian prince first and foremost acts for the good of the state, to develop it or save it from decline, not just to pursue his own ends, no matter the consequences.}.



In both cases, the understanding of prudence differs from that proposed by Aristotle. First of all, it is directed towards an end---be it salvation or the survival of the state---rather than being an activity in itself. Moreover, it is treated as a~kind of virtue that only a~few chosen possess. The general public should submit to them and listen to them, accepting their wisdom, and if they do not do so---this only shows the stupidity of the general public, and does not undermine the virtue of the \textit{phronimos}. For Aristotle, however, it was precisely in the eyes of the public that the \textit{phronimos} had to prove himself. Notice the wording: Pericles ``is deemed'' \textit{phronimos}, about Thales people ``say'' he is not. Thus, the recognition of a~leader's prudence is something that depends on the community in which he functions---Pericles ``is'' not, but ``is recognised as'', by a~particular group, in specific situations. Moreover, his prudence is not a~fixed and unquestionable virtue. Thucydides, describing the activities of Pericles in the \textit{Peloponnesian War}, in addition to emphasising his merits, also repeatedly refers to the criticism or opposition of the citizens of Athens, who constantly comment on, praise or blame the actions of their leader 
%\label{ref:RNDYsmWDTByye}(Thucydides, 2009, II.21).
\parencite[][]{hammond_peloponnesian_2009}. %
 Their opinion is not always correct, but it is what positions Pericles in relation to the community. He is aware that his actions are being watched and evaluated. His leadership role also depends on this assessment---he can be re-elected or removed. Pericles does not act from the height of infallible authority, nor is he a~simple manipulator. He strives to ensure that his actions benefit Athens as well as himself because the interests of the community are not separable from the interests of the individual. A~good leader is one who cares about the group he leads, be it a~state, an organisation or a~company, but at the same time expects (and has the right to expect) certain benefits for himself---respect and recognition, another term of office, remuneration.



Contemporary conceptions, on the other hand, emphasise mostly the aspect of looking only after the good of the community. Machiavelli's image of the leader has taken on a~negative connotation, in which the leader is concerned only with himself and pursues only his own interests, using the community for this purpose\footnote{This is, as we have said, a~fundamental distortion of Machiavelli's concept, but because of the different leitmotif of this text, we do not attempt to rehabilitate the Florentine's theory here.}. A~good leader should therefore become someone close to the image presented by Plato or Aquinas. In both Nonaka, Toyama and Hirata's theory and in the quoted text by Rooney and MacKenna, \textit{phronesis} is something that should lead to the common good, while the interests of the individual are overlooked or seen as merely a~side effect of concern for the whole. Moreover, \textit{phronesis} actually becomes a~tool for transforming knowledge into wisdom 
%\label{ref:RNDLUEbgHqXo4}(Nonaka et al., 2008, p.67).
\parencite[][p.67]{nonaka_managing_2008}. %
 As stated by Germán Scalzo and Guillermo Fariñas 
%\label{ref:RNDAPHuyaU7iA}(2018, p.30):
\parencite*[][p.30]{scalzo_aristotelian_2018}: %
 ``an original interest in knowledge, with the idea of \textit{phronesis}, clearly evolved into a~more ambitious purpose: wisdom''.



It may appear that the concept of ``dispersed knowledge'' proposed by F. von Hayek 
%\label{ref:RNDIyXOLbhV7v}(1945)
\parencite*[][]{hayek_use_1945} %
 is closest to the original meaning of phronesis. He strongly emphasized that no one has complete and perfect knowledge---there are no Platonic sages. Furthermore, knowledge itself never exists in a~concentrated form but rather is scattered, with multiple individuals possessing bits and pieces of it. In contrast to the aforementioned ``experts' knowledge'', Hayek acknowledged that individuals' ``dispersed knowledge'' is frequently marginalised. Meanwhile, this type of knowledge relates to specific temporal and spatial circumstances and therefore requires (and promotes) quick adaptation to new circumstances. As knowledge is not evenly spread, those who hold the presently relevant portion of ``dispersed knowledge'' are best equipped to make informed decisions, based on the possessed premises.



It would seem that this is the knowledge of \textit{phronimos}, who is able to consider the context of a~situation and its various possible developments, adapting and modifying the undertaken actions accordingly to effectively achieve their goals in given circumstances. This individual does not need to be aware of all circumstances or their consequences, but should be capable of adjusting their behaviour as necessary in response to the situation. However, a~crucial difference exists that prevents us from classifying Hayek's possessor of ``dispersed knowledge'' as a~\textit{phronimos}, and that is the postulated lack of deliberation. Hayek, to affirm his point, quotes A. Whitehead: ``Civilization advances by extending the number of important operations which we can perform without thinking about them'' 
%\label{ref:RNDChwhy8Zp7N}(Hayek, 1945, p.528).
\parencite[][p.528]{hayek_use_1945}. %
 Hayek's man operates intuitively, activating the knowledge he possesses subconsciously. However, this is not true in the case of \textit{phronimos,} who not only thinks, but thinks well and thoroughly. Aristotle defined humans as beings that are not only the \textit{zoon politikon}, but also possess the ``rational principle'' 
%\label{ref:RNDFqbpXPH2vU}(Aristotle, 1934, I.13)
\parencite[][]{rackham_nicomachean_1934} %
 which distinguishes them from other animals. We have the ability not just to think, but to think rationally. Relying solely on intuition when interpreting ``formulas, symbols, and rules whose meaning we do not understand'', as Hayek 
%\label{ref:RND8KqGZz1V43}(1945, p.528)
\parencite*[][p.528]{hayek_use_1945} %
 stated, limits our knowledge solely to \textit{nous}---Aristotelian ``intuitive thinking''. And \textit{nous} is only the initial phase of acquiring knowledge. \textit{Phronimos}, besides \textit{nous}, must also possess the knowledge of time and place, which cannot be merely gained through intuition. These are indeed the elements of ``dispersed knowledge'', but to undertake successful action, these circumstances must be acknowledged and analysed. This notion is present in Hayek's considerations, despite his subsequent affirmation of intuitive thinking. In order to plan and act, one must use, exchange, and put one's knowledge into action. Hayek provides an example: ``All that the users of tin need to know is that some of the tin they used to consume is now more profitably employed elsewhere and that, in consequence, they must economize tin.'' 
%\label{ref:RNDgIdHHe1DHi}(Hayek, 1945, p.526).
\parencite[][p.526]{hayek_use_1945}. %
 It is not essential for them to possess a~complete understanding of all the circumstances that have contributed to this situation, nor is it necessary for them to gain more knowledge. However, they must draw on their existing, ``dispersed'' knowledge to take action that would be most beneficial to them, and so act consciously rather than intuitively.



The action itself is the core of Aristotelian concept\footnote{Not gaining the full knowledge (which would mean gaining \textit{episteme} and so becoming a~sage) nor reaching some goal or level of indicator---that falls into the realm of \textit{techne}.}. To better illustrate this aspect, let us return to Aristotle's distinction between \textit{chrematistiké} and \textit{oikonomiké}. The former is the pursuit of the accumulation of goods, the latter the use of goods. An important aspect of ``use'' is a~certain possibility of its evaluation---one can use something well or badly, more or less carefully, achieving the intended goal or not. However, our predictions and expectations may be wrong or not accurate enough, we may lack certain information for various reasons, or we may succumb to bad advice. Following Hayek's example: we economised tin, only for the sudden demand for it to stop abruptly. As a~result, we were left with loads of now useless and worthless tin. According to Aristotle, those who have practical knowledge are able to analyse all the conditions they know of in order to take what they consider to be the best action. Nevertheless, its effect remains uncertain. Moreover, this action is judged \textit{post factum}, and in the case of the leader or manager, not only by him, but also by the whole community, which then will be able to give him (or take away) the title of \textit{phronimos}. Mere \textit{chrematistiké} (economization or accumulation of goods, e.g. tin) is not enough to obtain it. \textit{Phronesis} is the ability to adapt to changing situations, based on the dispersed knowledge possessed, but not absolute and infallible because it is about what is contingent.



It seems somewhat ironic that in theoretical approach, Knowledge-Based Economy is more about \textit{chrematistiké} than \textit{oikonomiké}, since the emphasis is on acquiring, deepening and developing knowledge; of course, knowledge that can be used practically, but the latter aspect arouses much less interest. There seems to be an implicit assumption that someone who acquires this knowledge (and we mean the various types of knowledge mentioned above) will also know how to use it correctly. This knowledge should come from experience, but since our leader, after accumulating \textit{chrematistiké}, always acted properly---for the common good, it is virtually impossible to point out when and where they could have gained such experience. Meanwhile, the ancient concept leads to a~disturbing implication---even with theoretical knowledge and experience, the outcomes of our actions are always uncertain and subject to evaluation. Conditions, people, premises change, purely subjective or emotional factors come to the fore. \textit{Phronesis}, then, is not so much the ability to act effectively towards a~specific goal, as it is the ability to take the risk of action---action that the \textit{phronimos}, on the basis of their knowledge, believes to be effective, but the effect of which is not a~foregone conclusion.



\textit{Phronesis} is thus the most social of all the dispositions that Aristotle writes about, and it can only be realised within a~community, be it a~state, a~company, or any other kind of society. It is shaped not so much by experience as by relationships and constant evaluation, as the case of Pericles shows: the Athenian made a~series of decisions that were evaluated both positively and negatively by the citizens, and he was able to adapt his behaviour to the situation not only because of the influence of external factors (e.g. by changing his strategy), but precisely because of opinion. He was able to negotiate and persuade to such an extent that he ``was recognized'' as a~\textit{phronimos}. An experienced \textit{phronimos} evaluates and draws conclusions from it not only from the perspective of the results achieved, but also taking into account the way his behaviour is evaluated by others, while this evaluation (feedback) should not so much set new or different goals for him, but show the possibility of a~change at the level of behaviour. Moreover, the Aristotelian \textit{phronimos} is not obliged to act altruistically only for the benefit of the community. Nowadays, it is the sphere of the ``common good'' that is most emphasized. Nonaka, Toyama and Hirata see profit as a~side effect, resulting from pursuing the standards of excellence, rather than ultimate goal. The aim is to produce an almost infallible leader who will always make the right decisions, of course, ``right'' in the sense of ``virtuous''. This is the result of the Thomistic transformation of \textit{phronesis}, which became another virtue. Therefore, the prudence expected of a~leader is prudence understood as caution in activities, impartiality, virtue, and action for the benefit of society. Aristotle, on the other hand, allows \textit{phronimos} to concentrate also on what is good for himself.



\textit{Phronesis} does not require the sacrifice of one's own interests on the altar of the common good, but precisely the kind of reflection that makes it possible to achieve both the particular interests of the individual (whether they be benefits or, for example, recognition and respect) and the interests of the whole community (the benefits achieved will affect the whole company, a~good manager will lead to greater trust on the part of contractors, etc.). It can therefore be understood as making the ``right'' decisions and actions, but these are not just virtuous ones, they are also beneficial to the person who makes them, to the community, and to whom it is dedicated. It is by no means strictly utilitarian, though it is not strictly virtuous either. Moreover, it is emphasized that the leader should act for the common good, without trying to think about how to do it (since he somehow already has the knowledge of how to do it), focusing on the fact that they should strive for development, success, improvement in quality and what is good for everyone. Here, too, the focus is on the objectives to be achieved, rather than on the value of the action itself.



Returning to Nonaka, Toyama and Hirata, the authors illustrate their concept with a~vivid comparison to constructing a~car. ``If \textit{techne} is the knowledge of how to make a~car well, \textit{phronesis} is the knowledge of what a~‘good' car is (value judgment) and how to build such a~car (realize the value judgment).'' 
%\label{ref:RNDrL5MveplTy}(Nonaka et al., 2008, p.54).
\parencite[][p.54]{nonaka_managing_2008}. %
 However, with reference to Aristotle's definitions, we would consider it more reasonable to combine \textit{techne} with the knowledge of ``how to make a~car'', and \textit{phronesis} is not so much the knowledge what a~``good'' car is, since this aspect fits more with \textit{episteme}, the scientific knowledge of things. The authors suggest that the \textit{episteme} cannot answer the question of what is a~``good'' car, because the question is subjective. No doubt, but if our understanding of a~``good car'' is completely subjective, what about the concept of ``common good'', that a~prudent manager should pursue? And since the ``common good'' is presented as a~kind of superior one, in order to maintain consistency, a~truly ``good'' car should also refer to some superior values, and thus try to match as many elements of the ``ideal'' car as possible. Therefore, either the ``common good'' (and, consequently, a~``good car'') is subjective and thus depends on the will of the manager, or it has to satisfy additional requirements. From the previous arguments, we can conclude that it is the latter, since the ``common good'' of the company means pursuing the interests of employees, shareholders or customers, and so many different and sometimes conflicting ones.



The difference here lies both in what can be judged as a~``good life'' or ``common good'' and in the actions of a~leader. For Aristotle, the \textit{phronimos} acts to achieve the ``good life'', which is defined rather vaguely as self-sufficiency. \textit{Phronimos} is not expected to achieve the ``perfect life'', because that is impossible---it would be achievable in Plato's world of ideas. \textit{Phronimos} has to act to make the normal earthly life as good as possible. To use the car analogy, is a~``good car'' a~safe car in the sense that it guarantees survival in the event of an accident, or should it prevent injury or even be automated enough to avoid accidents? For Aristotle, each of these goals is important, but what matters is what the designer or builder actually does. If he wants only and at all costs a~car that will never allow an accident---which is the realisation of the ``highest'' common good, namely safety---and for this reason does not take any measures to i.e. increase the chances of survival during an accident, he will not be deemed a~\textit{phronimos}, although he will strive to achieve a~good cause. In this case he will become like the philosopher Thales, who, dealing with the affairs of the universe, did not notice the well on his way.



This example is cited by both Plato and Aristotle. Plato gives the following anecdote: ``While he [Thales] was studying the stars and looking upwards, he fell into a~pit, and a~neat, witty Thracian servant girl jeered at him, they say, because he was so eager to know the things in the sky that he could not see what was there before him at his very feet.'' 
%\label{ref:RNDqdQRWiqqrJ}(Plato, 1921, 174a).
\parencite[][ {174a}]{plato_plato_1921}. %
 Aristotle, referring to this anecdote, claims that people like Thales can be attributed theoretical wisdom (\textit{episteme} and \textit{sophia})\footnote{For Plato, this particular anecdote is also the story of all the philosophers who study fundamental and universal things. They are like a~wise-man who, blinded by the light, returns to the cave to share his knowledge with the rest of the people there, but since his eyes are no longer accustomed to the darkness, he is unable to move smoothly in it and thus exposes himself to ridicule.} but not practical knowledge (\textit{phronesis}), because ``these sages do not seek to know the things that are good for human beings.'' 
%\label{ref:RNDFkMVLEhROp}(Aristotle, 1934, VI.7.5).
\parencite[][]{rackham_nicomachean_1934}. %
 However the same Thales in \textit{Politics} displays some practical knowledge. When his fellow citizens reproached him for the uselessness of philosophy, Thales, on the basis of his knowledge of astrology, predicted an extraordinarily rich olive harvest for the coming year. Then he rented out all the olive presses in advance for a~pittance. When his theory was confirmed and the harvest was indeed bountiful, everyone had to turn to him for the use of the presses, and then Thales---as the current monopolist---could set any rental price. In this way he made a~fortune, but the purpose of his activities was not to get rich, but to show that ``it is easy for philosophers to be rich if they choose, but this is not what they care about'' 
%\label{ref:RNDDW1TNkf2r9}(Aristotle, 1944, 1.1259a).
\parencite[][1.1259a]{aristotle_politics_1944}.%




An attempt to reconcile these two images of Thales, the sage and the \textit{phronimos}, leads to a~simple conclusion---a true philosopher has both theoretical and practical knowledge, and is able to forge one into the other. At the same time, \textit{sophia} is more important to him, and therefore he often does not do what is useful to him (for example, live in poverty or endure the ridicule of his fellow citizens), because above all he wants to finally achieve wisdom. This does not mean, however, that he could not act and use his knowledge if he wanted to. The problem, from a~practical point of view, is that he does not want to. Paradoxically, full knowledge encourages neither action nor risk. Thales, who bought olive presses, did not turn out to be a~good manager and \textit{phronimos}---his behaviour did not bring much benefit to the community, unless we consider as such a~greater respect for philosophy.



Similarly, our creator of the car may be a~brilliant inventor, a~sage, but has no prudence, because the knowledge he accumulates is not applicable. Not only does it not benefit society (e.g. by slightly increasing safety), but it also does not benefit the owner himself, who, locked in his studio, leads a~kind of \textit{vita} \textit{contemplativa}, searching for the final, ideal solution. Meanwhile, the Aristotle \textit{phronetic} leader uses what he has gathered (\textit{chrematistiké}) to act on the accumulated goods---knowledge, experience, knowledge of the craft. He acts with the awareness that his action is subject to the risk of lack of success, but at the same time, basing on his knowledge, he considers it good and beneficial. This is because only action allows him to verify this knowledge. The postulate of achieving a~more ambitious goal---wisdom---and making Knowledge-Based Economy a~Sage, as in Rooney and McKenna's text, paradoxically leads to the inhibition of its development. For a~true sage, having perceived the whole truth, does not feel the need to interact, to engage in human, less important matters, because ``human affairs do not deserve to be given great importance'', as Plato 
%\label{ref:RNDPFfnbqlyWg}(Plato, 1967b, 803b)
\parencite[][ {803b}]{plato_plato_1967-1} %
 wrote. And as Hannah Arendt 
%\label{ref:RNDBcZPvH1J3X}(2005, p.32)
\parencite*[][p.32]{arendt_promise_2005} %
 notes, the philosopher devotes himself entirely to the \textit{vita contemplativa}. He participates in community life solely because that community may be an obstacle to his complete engagement in philosophy.



The true \textit{phronimos} is the one who predicts---but does not know. Nonetheless, he is willing to take some risks. He assumes some possibilities of development, but he takes the risk that his decision is flawed. He introduces a~fresh invention on the assumption that it will be successful---but people accustomed to the old methods may decline to employ it. However, development and the accumulation of new knowledge are only possible due to uncertain activities that bear the risk of error.



\section{Conclusions}

It should be noted that modern research often dismisses the significance of ancient ideas or interprets aspects of the ancient world using contemporary terminology. Scott Meikle stresses that ``The ‘modernist' view is that the ancient economy is to be understood as an early restricted version of what we are familiar with today'' 
%\label{ref:RNDfgHrtKZO66}(Meikle, 1995, p.2).
\parencite[][p.2]{meikle_aristotles_1995}. %
 Hence, there is emerging criticism regarding the relevance of Aristotle's theories in contemporary research, as the Stagirite addressed a~significantly different economy. On the other hand, there is an attempt to adapt past phenomena and events to modern schemes. That unfortunately results in the loss of historical context. Ancient situations or myths are described without reference to their contemporary background, which included different values and concepts (like the role of fate, concept of justice, and punishment)\footnote{A~good example can be also found in Mielke, where there is an attempt to describe Prometheus' trick at Mecone as ``an example of a~pure isolated distribution where two parties meet on an equal footing and negotiate the division of a~joint asset'' 
%\label{ref:RND5o3ixQ3XwU}(Meikle, 1995, p.178).
\parencite[][p.178]{meikle_aristotles_1995}. %
 While this may be adequate in economic terms, it fails to present the complexity behind the myth and, more importantly, does not address its main purpose. The myth was meant to explain sacrificial customs as well as the reason why mankind is plagued by troubles, illnesses and sorrow. What is worth noting in this context is that Zeus and Prometheus were certainly not ``on an equal footing'' and ``negotiating'', as it is clear that one party (Prometheus), aware of the other's (Zeus) superiority, attempted to cheat in order to reach the desired outcome. Additionally, in one of the earliest descriptions of the myth, Hesiod suggests that Zeus was aware of the deception, but gave in to it, since Fate demanded so.}.



Therefore, our aim was not to reinterpret ancient theories in contemporary terminology, nor to shoehorn modern theories into the ancient conceptual framework. Rather, by drawing on the wealth of philosophical ideas, our objective was to highlight the potential relevance of the ancient Greek notion of ``knowledge'' and its associated elements in present-day analyses. Knowledge played a~crucial role in ancient thinking, regarded both as an intrinsic value and a~means to attain virtue. It served as the foundation for many aspects of life, including political, cultural, and economic spheres. And Greeks understood quite well the different types of this knowledge, including not only \textit{episteme}, (pure knowledge) and \textit{techne} (knowledge of the craft) but also knowledge of human relations that influences the community in which we live and work - the \textit{phronetic} one.



Considering the volatile nature of the modern world, including the rapidly changing social and economic relations, we believe that the concept of \textit{phronesis} remains relevant in updating the prevailing perception of the Knowledge-Based Economy and contemporary management theories.



Above all, we advocate for the prioritisation of the acquisition and application of knowledge (\textit{oikonomiké}) over its mere accumulation and possession (\textit{chrematistiké}) as the fundamental principle of the Knowledge-Based Economy. In numerous instances, attempts to characterise Knowledge-Based Economy focus on the stage where knowledge is already possessed or assume that its acquisition occurs during the learning process, therefore seeking to streamline this process by minimizing errors, introducing indicators and forming recommendations to enable the largest number of people to acquire knowledge. Unfortunately, in this manner, we only elevate the level of \textit{chrematistiké} and delve deeper into the ``savant economy'', quantifying our attained knowledge through grades, diplomas, or certificates, without due consideration of how to apply it. This aspect, the significant role of education in the Knowledge-Based Economy was highlighted by the International Commission on Education for the 21\textsuperscript{st} Century chaired by Jacques Delors and by Benjamin R. Barber, referring to the infantilisation of knowledge and education\footnote{Barber explains this phenomenon by referencing three dichotomous pairs of concepts: the dominance of ``easy over hard'', ``simple over complex'', and ``fast over slow'' 
%\label{ref:RNDwOmI94reKx}(Barber, 2008, pp.85–107).
\parencite[][pp.85–107]{barber_consumed_2008}.%
}. In the midst of these complex issues, it may be worthwhile to follow Hayek's advice and perceive the idea of knowledge as ``dispersed'', while preserving the Aristotelian elements of risk and action, which we deem particularly valuable. Progress can be achieved not by attaining higher levels of indicators, but by equipping future leaders\footnote{Managers, business leaders, political ones, etc.} in various social fields with competencies that empower them to apply their knowledge while being mindful of potential risks. This requires acting with due consideration and not only as a~leader, but also as a~team member because \textit{phronesis} can only be achieved through communal relationships. As Aristotle previously explained, it is necessary to possess a~certain level of adaptability in a~constantly changing reality. Rather than having complete control through certainty and expertise (\textit{episteme}), both the ability to think and act are required, accepting the possibility of failure and receiving criticism from others involved in the interaction. \textit{Phronetic} knowledge is not an unequivocal or definitive knowledge, given once and for all. It evolves, adapts and moulds itself to suit the various types and requirements of human societies. Therefore, the endeavour to assign the ``knowledge'' only to ``sages'', as in the question posed in the introduction, automatically reduces its complexity. As Nonaka, Toyama and Hirata 
%\label{ref:RNDzCMnFBtgWi}(2008, p.242)
\parencite*[][p.242]{nonaka_managing_2008} %
 accurately note, ``knowledge is created by human beings in relationships, knowledge-based theory of the firm has to broaden its perspective from the static, atomistic, substance-based worldview typical of conventional economic theory, to a~view of the firm as a~dynamic entity in flow.''



The competences included within \textit{phronesis} might lead to very (\textit{nomen omen}) practical recommendations. The concept in its original, Aristotelian meaning appears worthwhile for implementation, particularly in management theories, since \textit{phronesis} pertains to an individual's knowledge expressed through action. Hence, it can only be realized in situations that necessitate interactions among diverse actors and not merely in the theoretical sphere. Thus, our aim should not be to create a~know-it-all Platonic philosopher, but rather an Aristotelian \textit{phronetic} leader who is willing to take action, make errors, and receive feedback from others and who is not afraid to act or make difficult but deliberate decisions that influence the whole society, with their (and his own) best interests in mind. This requires focusing not only on the desired qualities of the manager in terms of their character and skills but also providing them with tools from both \textit{techne} and \textit{episteme}---abilities related to managing stress, decision-making, holding challenging conversations or negotiations, thinking creatively or out-of-the-box, and mentoring. Those could assist in educating a~conscious, mindful individual, able to use their particular, individual knowledge to operate and interact within the dynamic domain of social relationships in a~manner that would benefit both themselves and the surrounding community.



\end{artengenv2auth}

\label{ceglarska-lastpage}
