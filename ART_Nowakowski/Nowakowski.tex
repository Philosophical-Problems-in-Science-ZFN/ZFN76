\begin{artengenv}{Paweł Nowakowski}
	{A~praxeology of the value of life. A~critique of Rothbard's argument}
	{A~praxeology of the value of life\ldots}
	{A~praxeology of the value of life. A~critique of Rothbard's\\argument}
	{University of Wrocław\label{nowakowski-first}}
	{The present paper aims to study the issue of the value of life in Murray N. Rothbard's work, and to examine his argument for the contention that ``life \textit{should} be an objective ultimate value'' and that ``the preservation and furtherance of one's life takes on the stature of an incontestable axiom.'' Rothbard's assumptions and presuppositions are investigated and critically assessed. Using conceptual and logical analysis rooted mostly in the praxeological method of economics (as developed by Mises and Rothbard themselves) and the theory of value (Scheffler, Raz, Elzenberg), it is demonstrated that Rothbard's account is fallacious both on its own as well as on broader theoretical grounds. It is argued that what Rothbard could---under his specific assumptions about valuing---correctly claim is only that an actor values life \textit{to some extent}, rather than that life has an objective ultimate value or preservation and furtherance of one's life has an axiomatic status. The theoretical argument is supported by empirical illustration from suicide terrorism. The paper submits that Rothbard's position on the value of life is unsound, and that using his argumentation as it stands cannot succeed.
	}
	{praxeology, time preference, Murray Newton Rothbard, valuing, value of life, libertarianism}







\section{Introduction}

\lettrine[loversize=0.13,lines=2,lraise=-0.03,nindent=0em,findent=0.2pt]%
{I}{}n this paper, I~investigate the issue of the value of life in the work of Murray N. Rothbard, an economist representative of the Austrian school of economics and one of the leading figures in modern libertarianism, understood as a~radical free-market current in the contemporary political philosophy, built upon two pillars: self-ownership and strong property rights 
%\label{ref:RND1th47sMADB}(see Rothbard, 2006; 2009)
\parencites[see][]{Rothbard2006For}[][]{Rothbard2009Man} %
 (and this is also how I~understand libertarianism herein). I~aim to assess the validity of Rothbard's 
%\label{ref:RNDGqeeRTt2Se}(1998, pp.32–33)
\parencite*[][pp.32–33]{Rothbard1998Ethics} %
 argument for the assertion that ``life \textit{should} be an objective ultimate value'' and that ``the preservation and furtherance of one's life takes on the stature of an incontestable axiom.''



It might appear the value of life is of rather secondary importance for libertarianism since it rests not so much on a~value-based discourse as on the rights-based one.\footnote{In this paper, I~make use of the distinction between axiology and ethics. I~refer to axiology as a~philosophical discipline centered around values which covers such problems as, for example, the concept of value, types of values and the way values exist. On the other hand, ethics (excluding metaethics) is primarily aimed at guiding our actions and is concentrated on norms, rights and duties. Thus, the discourse of rights is typically ethical (and legal), while the discourse of values is primarily axiological.} In other words, when libertarians claim it is impermissible to kill innocent people, they thereby argue that killing is impermissible not because it directly runs counter to the value of life but for this reason that it violates a~person's property right to his or her body, i.e., a~fundamental or natural right of a~person---self-ownership. Nonetheless, some scholars maintain that the assumption of the value of life is relevant to the derivation of libertarian rights, if only as far as some variants of libertarianism are concerned 
%\label{ref:RNDtBxg6NlKYc}(see Harris, 2002, p.115; Hoppe, 1998, pp.xxxiv–xxxv; Mack, 2022, p.14; Meng, 2002; Osterfeld, 1983; 1986, pp.60–61; Rasmussen, 1980; Share, 2012, p.134ff; Slenzok, 2018, p.29; Thrasher, 2018, p.217; Wissenburg, 2019, p.294).
\parencites[see][p.115]{Harris2002Rights}[][pp.xxxiv–xxxv]{Hoppe1998Introduction}[][p.14]{Mack2022Natural}[][]{Meng2002Hoppeing}[][]{Osterfeld1983Natural}[][pp.60–61]{Osterfeld1986Freedom}[][]{Rasmussen1980Groundwork}[][p.134ff]{Share2012Defense}[][p.29]{Slenzok2018Dwa}[][p.217]{Thrasher2018Social}[][p.294]{Wissenburg2019Concept}. %
 What is more, Rothbard's attempt to prove the axiomatic nature of the claim of the objective ultimate value of life is approvingly shared or even employed further by other libertarian authors 
%\label{ref:RNDdxGeyPBnIO}(see Hoppe, 1988; 1998, pp.xxxiv–xxxv; Meng, 2002; Rasmussen, 1980).
\parencites[see][]{Hoppe1988From}[][pp.xxxiv–xxxv]{Hoppe1998Introduction}[][]{Meng2002Hoppeing}[][]{Rasmussen1980Groundwork}. %
 Hoppe 
%\label{ref:RNDk7Vokc1JDt}(1988, p.66)
\parencite*[][p.66]{Hoppe1988From} %
 not only approves his argument but also maintains that its structure is the same as the one of Hoppe's argumentation ethics, which is his philosophical attempt to justify libertarianism. Also Meng 
%\label{ref:RNDnse7S0BCeo}(2002)
\parencite*[][]{Meng2002Hoppeing} %
 approves and broadly applies Rothbard's position on the value of life in the former's attempt to ground the principle of original appropriation. Therefore, the investigation of Rothbard's reasoning seems vindicated not only in terms of scrutinizing his argument itself, but also regarding its implication for deriving and justifying libertarian rights. It becomes even more evident when one considers that Rothbard is the only libertarian author who has delved into the value of life in some more detail.

\enlargethispage{1.5\baselineskip}

In the present article, I~make use of the praxeological methodology employed by Rothbard and characteristic of the Austrian school of economics in Ludwig von Mises' variation. Rothbard's ontological, epistemological, and methodological standpoints, which are substantial for his argument on the value of life, are the following: realism, foundationalism, cognitivism, apriorism, deduction, the law of non-contradiction, methodological individualism, the concept of human action as a~goal-oriented behavior using the means available thereto under the condition of the scarcity of resources, verbal logic as a~medium of reasoning, and the theory of time preference. I~adopt that research approach for the sake of analyzing Rothbard's considerations with respect to the validity of his premises, deductive moves, and reached conclusions. Additionally, I~make use of a~conceptual framework of the theory of value, particularly as regards the distinction between intrinsic value and instrumental value, and valuing.



As a~consequence, Rothbard's argumentation is subject to an analysis which leads to the following theses:

\begin{enumerate}[label=\arabic*)]
\item  Rothbard's argumentation does not provide a~proof of the objective ultimate value of life or the proof of the axiomatic status of preservation and furtherance of one's life.
\item Under his assumptions about \textit{valuing}, Rothbard could claim only that an actor values life \textit{to some extent}.
\end{enumerate}

The paper contributes to the ongoing debate by taking heed of the flawed nature of Rothbard's argumentation for the ultimate value of life, and by the indication that using his reasoning for advancing libertarianism cannot succeed.



In the following section, I~shall present Rothbard's position on value and value judgements through contrasting his conception with the theory of value, as conceived of by Mises. In the subsequent two sections, I~shall present Rothbard's argument for his proposition on the value of life and proffer my interpretation of his approach, including the issue of his presuppositions about valuing. Then, I~shall analytically refute Rothbard's argument. The critique of his reasoning shall be additionally strengthened in the next part of the article. Furthermore, the theoretical analysis shall be illustrated by a~case of a~suicide terrorist. Eventually, in the last section I~shall conclude.
\enlargethispage{2.5\baselineskip}


\section{Rothbard's position on value and value judgement}

As far as an economic theory of value goes, Rothbard favored a~definitely subjectivist approach, thus following Mises\footnote{On the differences in value theory between Mises and the founding father of the Austrian school of economics, Carl Menger, and another luminary thereof which was Eugen von Böhm-Bawerk, see Grassl 
%\label{ref:RNDzwdqMXMJwK}(2008, p.96; 2017, pp.531–559),
\parencite*[][pp.531–559]{Grassl2017Toward}, %
 and Hülsmann 
%\label{ref:RNDXw75SrxFFp}(2007, pp.388–391).
\parencite*[][pp.388–391]{Hulsmann2007Mises}.%
}, who stated that ``value is not intrinsic'' and instead defined it as the ``importance that acting man attaches to ultimate ends.'' A~secondary value he attributed also to the means employed for the sake of achieving an end. According to Mises 
%\label{ref:RNDf042XyLpW2}(2008, p.121),
\parencite*[][p.121]{Mises2008Human}, %
 what determines the value one ascribes to a~good is its utility, relative to the actor in question. Rothbard 
%\label{ref:RNDkUu9JNzcYv}(2009, pp.103, 21)
\parencite*[][pp.103]{Rothbard2009Man} %
 reasoned very much alike, positing that ``value exists in the valuing minds of individuals [...]'', and, more specifically, that ``the original source of value is the ranking of ends by human actors, who then impute value to consumers' goods, and so on to the orders of producers' goods, in accordance with their expected ability to contribute toward serving the various ends.''



However, Mises and Rothbard differed as to their respective views on the status of value judgements. Mises 
%\label{ref:RNDPeBoihkBZJ}(see, e.g. 2008, p.10)
\parencite*[see, e.g.][p.10]{Mises2008Human} %
 believed that in no science there is room for value judgements, that is for putting forward normative claims. Being a~proponent of a~free-market capitalism, he believed that the arguments in favor of it are provided by value-free economics and sociology, which were supposed to prove that it is only a~regime based on private property in the means of production that may function efficiently 
%\label{ref:RND6zmyfhCh9b}(Mises, 2010, p.86).
\parencite[][p.86]{Mises2010Liberalism}. %
 Moreover, he did not conceal his subjective, and thus unscientific, adherence to a~liberal political system, which was supposed to ensure the most congenial conditions to peace, prosperity, health, wealth, that is to the realization of the values which---as he stressed---he shared with the majority of society 
%\label{ref:RNDBNzj9B3ktd}(see Rothbard, 1997, pp.93–94).
\parencite[see][pp.93–94]{Rothbard1997Praxeology}.%




By contrast, Rothbard thought that values might be studied in their both positive and normative aspects. That is why, his position on value presented above should be rather done justice to by quoting the following excerpt from his \textit{Ethics of Liberty}: ``Value \textit{in the sense of valuation} or utility is purely subjective, and decided by each individual'' 
%\label{ref:RNDcltkKk07OX}(Rothbard, 1998, p.12, italics added).
\parencite[][italics added]{Rothbard1998Ethics}. %
 This \textit{sense of valuation} is pertinent precisely to economics since when it comes to ethics, Rothbard 
%\label{ref:RNDvhf9gFI7BT}(1997, p.78; 1998, p.12)
\parencites*[][p.78]{Rothbard1997Praxeology}[][p.12]{Rothbard1998Ethics} %
 claimed that there exist objective and rational criteria of an ethical assessment of value judgements as well as objective values derived from the natural law. Hence, in this respect he dissociated himself from Mises and subjected the latter's rational-utilitarian ethical approach to profound criticism based on a~consistent praxeological analysis 
%\label{ref:RNDu3O4hsetCR}(see Rothbard, 1997, pp.90–99; 1998, pp.201–214).
\parencites[see][pp.90–99]{Rothbard1997Praxeology}[][pp.201–214]{Rothbard1998Ethics}. %
 However, strictly axiological considerations, that is the ones concerning a~philosophical theory of value, are in Rothbard's output rather cursory. And when it comes to explicitly normative statements, these are confined to the issue of the value of human life, which is the subject matter scrutinized herein. This state of affairs was owing to the fact that Rothbard's normative reflection was mainly centered around natural rights that he deduced from the modified John Locke's statements on self-ownership and the principle of original appropriation.



\section{Rothbard's argument on the value of life}

In order to prove the axiomatic character of his claim, Rothbard 
%\label{ref:RNDkj6x5zHJYV}(1951, p.946; see also 2009, ch. 1-2)
\parencites*[][p.946]{Rothbard1951Praxeology}[see also][ch.~1-2]{Rothbard2009Man} %
 resorted to ``the theory of the isolated individual,'' which is representative of the Austrian school of economics, with the theory being also known as ``Crusoe economics'' 
%\label{ref:RND6cIvbGcfUi}(see Nozick, 1977, pp.353–392).
\parencite[see][pp.353–392]{Nozick1977On}. %
 To serve philosophical purposes, he adopted the theory in the form of the methodological tool labeled as ``a Crusoe social philosophy'' 
%\label{ref:RNDXyBnbf7szh}(see Rothbard, 1998, pp.29–34),
\parencite[see][pp.29–34]{Rothbard1998Ethics}, %
 in which he introduced a~second person in order to found his considerations upon an interpersonal relation.



In the hypothetical situation as depicted by Rothbard 
%\label{ref:RNDxVqlcztWqx}(1998, p.32),
\parencite*[][p.32]{Rothbard1998Ethics}, %
 the other inhabitant of the island warns Crusoe not to eat poisonous mushrooms growing there. Consequently, Robinson abandons the idea of consuming them, picking up berries instead. Rothbard claimed this situation evidences a~very strong conviction on the part of both of them that poison is detrimental to humans---the condition so strong that it is not even explicitly mentioned. So, it is in this manner that they both recognize that human life and health have value, unlike suffering and death. And that is the way in which the cognizing of ethical foundations proceeds, with the said ethical foundations reflecting the nature of things and the laws of nature pertaining to human beings.



Furthermore---argued Rothbard 
%\label{ref:RNDjpSvw6l2D4}(1998, pp.32–33)
\parencite*[][pp.32–33]{Rothbard1998Ethics}%
---if Crusoe had consumed---despite warnings---the poisonous mushrooms, this would constitute an act running counter to his life and health, which would be ``objectively \textit{immoral}.'' His actual motives would be irrelevant then: whether it was high time preference or the need to get intoxicated that would constitute underlying motives. Let us cite Rothbard's 
%\label{ref:RNDWiYOJ9ryS7}(1998, pp.32–33)
\parencite*[][pp.32–33]{Rothbard1998Ethics} %
 argument \textit{in extenso}:



\begin{quote}
It may well be asked why life \textit{should} be an objective ultimate value, why man should opt for life (in duration and quality). In reply, we may note that a~proposition rises to the status of an \textit{axiom} when he who denies it may be shown to be using it in the very course of the supposed refutation. Now, \textit{any} person participating in any sort of discussion, including one on values, is, by virtue of so participating, alive and affirming life. For if he were \textit{really} opposed to life, he would have no business in such a~discussion, indeed he would have no business continuing to be alive. Hence, the \textit{supposed} opponent of life is really affirming it in the very process of his discussion, and hence the preservation and furtherance of one's life takes on the stature of an incontestable axiom [footnotes deleted].
\end{quote}



Before I~embark on a~critique of Rothbard's argumentation, let us make some interpretative comments. Rothbard pointed out that recognizing life as an ultimate value implies that man should ``opt for life (in duration and quality).'' Based on this assertion as well as on the express statement that action running counter to one's life is, irrespective of the motives underlying it, objectively immoral, one may venture an interpretation of the concept of ``ultimate value,'' as employed by Rothbard, from an axiological point of view. Hence, it seems clear that ``ultimate value,'' as conceived of by him, is something over and above what Höffe 
%\label{ref:RNDC282eqJvZa}(1991, pp.26–27)
\parencite*[][pp.26–27]{Höffe1991Gerechtigkeit} %
 calls ``transcendental interests,'' i.e., ``logically higher-order interests,'' comprising the capability of action and thus requiring ``the integrity of one's health and life,'' as well as something beyond a~value considered as a~condition of attaching values to something. The latter approach would be incomplete, anyway, since life is a~precondition of both good and bad things 
%\label{ref:RNDhWC02Rl7ni}(Raz, 2001, p.8),
\parencite[][p.8]{Raz2001Value}, %
 and what is more, even when something is a~precondition of good, it does not make that precondition necessarily valuable. As Nozick 
%\label{ref:RNDfw3SDMYYOr}(1971, p.252)
\parencite*[][p.252]{Nozick1971On} %
 demonstrated, recovery from cancer is a~value the precondition of which is getting cancer, but it would be implausible to argue that getting cancer is a~value as being a~condition of the value of curing cancer. For Rothbard, the value of life must have meant a~value \textit{sensu proprio et stricto} because he formulated on its basis a~norm of requirement: ``man should opt for life (in duration and quality);'' and an implicit norm of prohibition by describing action running counter to one's life as objectively immoral. Owing to its purportedly objective character, it could not be a~value in the utilitarian or instrumental sense, that is a~relative value allowing for the realization of interests (including needs, desired, wants), the realization of which is valuable to one person but may be not valuable to another 
%\label{ref:RNDdVFbkrRRrT}(see Elzenberg, 1990, p.21; Raz, 2001, p.77).
\parencites[see][p.21]{Elzenberg1990Wartość}[][p.77]{Raz2001Value}. %
 The value that Rothbard labels as ``objective ultimate value'' bears the closest resemblance to an intrinsic or perfective value, that is---as opposed to a~utilitarian or instrumental value\footnote{Elzenberg 
%\label{ref:RNDGerilVSkOK}(1990, pp.28–29)
\parencite*[][pp.28–29]{Elzenberg1990Wartość} %
 distinguishes between an instrumental and utilitarian value. The former leads to a~perfective value, whereas the latter to the satisfaction of a~need.}---to a~value of unconditional character, which means that is not a~value \textit{for somebody} but precisely an \textit{objective} value 
%\label{ref:RNDe0A0t5A2Sx}(see Elzenberg, 1990, p.21ff; Raz, 2001, p.77; Schroeder, 2021; Zimmerman and Bradley, 2019).
\parencites[see][p.21ff]{Elzenberg1990Wartość}[][p.77]{Raz2001Value}[][]{Schroeder2021Value}[][]{Zimmerman2019Intrinsic}. %
 That said, philosophers distinguish two ways of understanding the term of intrinsic value. In the first sense, what is meant is an ultimate or non-instrumental value, whereas in the second, what is meant is an objective value in the ontological sense, which implies that the value in question exists independently of an actor 
%\label{ref:RNDc2TQSX7jjg}(see Frey and Morris, 1993, p.8).
\parencite[see][p.8]{Frey1993Value}. %
 Rothbard's argumentation is clearly related to the first sense because he attempts to prove the objectivity of the ultimate value of life based on an analysis of an individual action, which by necessity applies to each acting man. Thus, the conclusion on the objective value of life derives from an aggregate of subjective actions analyzed using the principle of performative non-contradiction.



For the sake of clarity, it is also worthwhile to elucidate the word ``affirm,'' which was used twice in the above-scrutinized fragment. The dictionary definitions are as follows: ``to state something as true'' (Cambridge Dictionary); ``to publicly show your support for an opinion or idea'' (Cambridge Dictionary); ``to assert (something, such as a~judgment or decree) as valid or confirmed'' (Merriam-Webster); ``to show or express a~strong belief in or dedication to (something, such as an important idea)'' (Merriam-Webster). Since in the above-quoted passage Rothbard ascribed affirming life to a~hypothetical denier of the value of life, affirming refers to \textit{showing} or \textit{expressing} (nonverbally) rather than \textit{stating} or \textit{asserting.} Further, the word ``affirm,'' as used by Rothbard, seems to have the same meaning as the word ``approval,'' used in the context of a~perfective value by Elzenberg\footnote{On Elzenberg's contribution to axiology see, e.g. 
%\label{ref:RNDfoYcrFCiQj}(Porębski, 2019, pp.73–86).
\parencite[][pp.73–86]{Porębski2019Elzenberg:}.%
} (however, ``affirmation'' is a~word of more unambiguously positive character). ``Each value judgement---states Elzenberg 
%\label{ref:RNDReYxOyaTDn}(1990, p.25)
\parencite*[][p.25]{Elzenberg1990Wartość}%
---is an approval. An approval is the very value judgement: an approval exhausts itself in a~value judgment and so does a~value judgement in an approval.''



On the basis of the above interpretations, it seems clear that when Rothbard used the phrase ``affirmation of life,'' what he thereby meant was an action demonstrating the recognition of life as being endowed with a~value in this sense which he tried to prove, which is the one of an intrinsic, perfective, ultimate value.



\section{Rothbard's presuppositions about valuing}

The foregoing findings are instrumental in identifying presuppositions which Rothbard held about valuing. Thus, from his words that ``the \textit{supposed} opponent of life is really affirming it in the very process of his discussion,'' we may contend that Rothbard took it for granted that it is possible to value something without being aware of valuing it, a~contention that comes from economics and is based on the concept of ``demonstrated preference'' 
%\label{ref:RNDurQzV0btZy}(cf. Osterfeld, 1986, p.61).
\parencite[cf.][p.61]{Osterfeld1986Freedom}. %
 In his seminal paper, Rothbard 
%\label{ref:RNDQ7Q2sJHgU6}(2011, p.289)
\parencite*[][p.289]{Rothbard2011Economic} %
 explained:



\begin{quote}
Action is the result of choice among alternatives, and choice reflects values, that is, individual preferences among these alternatives. [...] The concept of demonstrated preference is simply this: that actual choice reveals, or demonstrates, a~man's preferences; that is, that his preferences are deducible from what he has chosen in action. Thus, if a~man chooses to spend an hour at a~concert rather than a~movie, we deduce that the former was preferred, or ranked higher on his value scale.
\end{quote}



It has been pointed out above that Rothbard, unlike Mises, maintained that objective ethics is possible and that economic and ethical approaches to the study of value differ in that economics does not engage in value judgments, whereas ethics does. Having enabled Rothbard to engage in developing the political philosophy of modern libertarianism, that dissociation from Mises, however, was not pertinent to conceptual foundations. Accordingly, Rothbard's understanding of the concepts of value and valuing in the realm of ethics was rooted in economics or, more broadly, in praxeology, i.e., ``a general theory of human action'' 
%\label{ref:RNDHLBJYqBRUC}(Mises, 2008, p.7).
\parencite[][p.7]{Mises2008Human}. %
 Hence, in both economics and ethics, he regarded value as inextricably linked to valuing, and valuing as, by definition, linked to a~concrete action which he believed to reveal a~value professed by an acting person. When considering the value of life, Rothbard also kept the assumption, taken from his economic writing, that value is just a~synonym for preference which is clearly presented in the above quotation, e.g., in the phrase ``choice reflects values, that is, individual preferences among these alternatives.'' This excerpt informs us of yet another of Rothbard's presuppositions, namely that each human action involves choosing between alternative values, so that an action cannot help but realize a~value professed by an acting being. Why didn't Rothbard abandon this conceptual background characteristic of the Austrian school of economics, when he was concentrating on libertarianism, e.g., in his \textit{Ethics of liberty}? A~probable answer is because, as a~student of Mises, he regarded praxeology as a~paradigmatic framework not merely for economics but for the whole edifice of the science of human action, ethics included. At the very beginning of \textit{Human Action}, Mises 
%\label{ref:RND60uY3O3SNX}(2008, p.3)
\parencite*[][p.3]{Mises2008Human} %
 set forth the boundaries of praxeology:

\begin{quote}
The general theory of choice and preference [praxeology] goes far beyond the horizon which encompassed the scope of economic problems [...]. It is much more than merely a~theory of the ‘economic side' of human endeavors and of man's striving for commodities and an improvement in his material well-being. It is the science of every kind of human action. Choosing determines all human decisions. In making his choice man chooses not only between various material things and services. All human values are offered for option. All ends and all means, both material and ideal issues, the sublime and the base, the noble and the ignoble, are ranged in a~single row and subjected to a~decision which picks out one thing and sets aside another.
\end{quote}



Mises conceived of praxeology very broadly indeed, and importantly, he made it clear that the praxeological analysis of values extends to ``all human values'' and ``ideal issues,'' thereby encompassing the problems typical for, among others, ethics and political philosophy. Rothbard himself emphasized that praxeology, although useful in analyzing ethical propositions, does not formulate ethical norms or public policies, and he stressed its positive (descriptive) character as opposed to normative (prescriptive) nature of ethics and political philosophy 
%\label{ref:RNDijKUL7bU5w}(see, e.g., Rothbard, 2009, p.1297ff).
\parencite[see, e.g.,][p.1297ff]{Rothbard2009Man}. %
 But on the other hand, when addressing value and valuing from the ethical point of view, particularly the value of human life, he presupposed the praxeological conceptual framework, which partly determined the way he attempted to justify the value of life as objective and ultimate.



As far as ethics or axiology is concerned, equating preference, as conceived of in the Austrian economics, with value is fallacious because while the former reduces to a~simple fact of choice between available options, the latter is a~more complex and comprehensive category. For instance, Scheffler 
%\label{ref:RNDzrKRSbAgcC}(2011, p.24)
\parencite*[][p.24]{Scheffler2011Valuing} %
 regards as peculiar using the term ``valuing'' when considering trivial desires, such as looking through a~newspaper at a~waiting room. In more detail, philosophers argue 
%\label{ref:RNDgE4mfkXes3}(see discussion in Scheffler, 2011)
\parencite[see discussion in][]{Scheffler2011Valuing} %
 that it is possible that a~person desires something (and, by extension, demonstrates her preference for it), but she does not value it because she considers it harmful or sinful---as it is possible in the case of desiring drugs by an addict or desiring to engage in disapproved sexual activity---and she would much prefer not to desire it (and, by extension, not to prefer it or not to demonstrate such a~preference). The reverse is also possible since valuing something may be not accompanied by a~desire for it or a~motivation to realize it, which might be the case when a~person suffers from mental disorders such as depression.



\section{Refuting Rothbard's argument directly}

In the following steps, I~shall reconstruct and refute the Rothbard's argument apparently bolstering the proposition that ``life \textit{should} be an objective ultimate value'' and that ``the preservation and furtherance of one's life takes on the stature of an incontestable axiom.''



There are two parts of the argument (which was quoted in full above). To avoid any possible misinterpretations, let us quote the particular parts once again:



\begin{enumerate}[label=\arabic*)]

\item ``[A]ny person participating in any sort of discussion, including one on values, is, by virtue of so participating, alive and affirming life.''



\item  ``[I]f he were \textit{really} opposed to life, he would have no business in such a~discussion, indeed he would have no business continuing to be alive.''



\item  ``Hence, the \textit{supposed} opponent of life is really affirming it in the very process of his discussion, and hence the preservation and furtherance of one's life takes on the stature of an incontestable axiom.''

\end{enumerate}

Based on the foregoing, I~suggest the following reconstruction of Rothbard's syllogism:



\begin{enumerate}[label=(\Roman*)]
\item  If A~is really against life, he does not have a~reason to argue about values or stay alive (based on point 2 above).
\item  If A~argues, he is alive (based on point 1).
\item  If A~stays alive, A~affirms life (based on points 1 and 3).
\end{enumerate}
Therefore,
\begin{enumerate}[label=(\Roman*), start=4]
\item A~is not really against life, and the objective and ultimate value of life is an axiom (based on point 3 and Rothbard's main thesis presented in the previous sections of the present paper).

\end{enumerate}

Now, although Rothbard's reasoning is not as precise as it might seem at first sight, his argument resembles a~logical principle \textit{modus tollens}---[(\textit{p}$\Rightarrow $\textit{q})${\wedge}$¬\textit{q}]$\Rightarrow $¬\textit{p}. When applied to his argument, the reasoning is as follows: If A~were really against life (\textit{p}), then A~would not have a~reason to stay alive and would terminate his life (\textit{q}), but A~is staying alive, arguing, and is not killing himself (¬\textit{q}), therefore A~is not really against life (¬\textit{p}). However, Rothbard's reasoning is erroneous since the assumed implication: if \textit{p}, then \textit{q}, is false and so is his finding (¬\textit{p}), which will be demonstrated below by scrutinizing parts (I) and (III) of the above syllogism. For the sake of clarity, I~shall start with (III), and proceed to (I), which is actually the order in which Rothbard presented his argument.



%\subsection{(III) If A~stays alive, A~affirms life }
%
%
%
%(``\textit{any} person participating in any sort of discussion, including one on values, is, by virtue of so participating, alive and affirming life'').

\subsection{\textbf{(III) If A stays alive, A \textit{affirms} life} (``\textit{any} person participating in any sort of discussion, including one on values, is, by virtue of so participating, alive and affirming life'')}



Suppose Crusoe decides to consume poisonous mushrooms in the hope of dying afterwards. After the ``meal,'' while he would be expecting the consequences of his act, he would receive a~visitor with whom he would engage in a~discussion over the value of life. In line with his previous act aimed against his life and health, he would try to convince his interlocutor that life is not (an ultimate) value, and it is bad and not good.



Would it be justified then to say that this man awaiting death affirms life? Certainly, he is alive and still benefits from living; yet he does not \textit{preserve} nor \textit{further} his life. It cannot be argued that if he were \textit{really} opposed to life, he would commit suicide instead of debating life with his visitor. After all, he already took an action aimed at terminating his life. And now, while awaiting expected death, he is trying to convince his interlocutor that life is bad. Thus, while keeping on living and discussing, he does not affirm life since he is using it merely as a~means of his argumentation against life (which means life can have only an instrumental value for him) rather than recognizes its objective ultimate value (perfective, intrinsic, unconditional value).



And so, (III) gets refuted (apart from the trivial implication from the quoted part of the argument that if a~person argues, he or she is alive).



\subsection{\textbf{(I) If A~is really against life, he does not have a~reason to argue about values or stay alive }
(``if he were \textit{really} opposed to life, he would have no business in such a~discussion, indeed he would have no business continuing to be alive'')}



It does not appear improbable---and what is even more important, it would be logically valid---if Crusoe were opposed not only to his life but also to life in general and that is why he would wish that his interlocutor also committed suicide. At this point, he might refer to, say, Kant's categorical imperative---deciding to terminate his life, he would consciously recognize that it is the act that he could wish that it become universal and that all the others also do so. In other words, Crusoe hated life and therefore he committed suicide. And since he hated life so much that his guiding principle became ``the greatest possible number of suicides,'' then while awaiting death he made use of an opportunity to convince somebody else of the justifiability of his view.



However, why didn't Crusoe, as a~hypothetical opponent of life, simply kill his interlocutor, and instead tried to persuade him to commit suicide? The answer seems fairly obvious: it is for the same reason for which people do not try (if they do not try) to force others to act according to the former's fundamental principles---because while the former recognize certain values as the most important, they also recognize individual rights, among which personal inviolability is one of the most fundamental ones.



As a~result, (III) is also fallacious.



\section{Further discussion}

To unfold the above critique of Rothbard's arguments, let us analyze the fact that there are numerous examples of people who sacrificed their lives for the sake of their values or other people. In case one person gives up her life so that another person or a~group of people could survive, one may claim that the thesis of the affirmation of life (as the ultimate value) remains unscathed. After all, by sacrificing one's life, one consciously sustains the life of another or of a~group of people. She thereby terminates her life \textit{in the name of} the value of life (maybe even the objective ultimate one). What is more 
%\label{ref:RNDVcaOriTF14}(see Raz, 2017, p.1; Weiss, 1949, p.76),
\parencites[see][p.1]{Raz2017On}[][p.76]{Weiss1949Sacrifice}, %
 the very notion of sacrifice presupposes that one can sacrifice something only provided that one ascribes to it some value.



It seems that Rothbard's position is easily reconcilable with the recognition that such a~sacrifice, even when materialized by committing suicide, is not at odds with the affirmation of life. As a~reminder, his thesis is that the ultimate moral value of life is objective and shared by all the living persons. However, in this context, it is again problematic to justify Rothbard's thesis. It is because, somewhat paradoxically, it transpires that:



\begin{enumerate}[label=(\arabic*)]

\item  One may keep on living without recognizing the ultimate value of life.



\item One may commit suicide, when recognizing the ultimate value of life.

\end{enumerate}

Although proposition (2) has already been stated, it might be supported by the indirect case from Scheffler. His definition of valuing is as follows: ``To value X~is normally both to believe that X~is valuable and to be emotionally vulnerable to X'' 
%\label{ref:RNDzOjWyphvFr}(Scheffler, 2011, p.31).
\parencite[][p.31]{Scheffler2011Valuing}. %
 Scheffler 
%\label{ref:RND1fruie6Vpy}(2011, pp.26–27)
\parencite*[][pp.26–27]{Scheffler2011Valuing} %
 presses the point that one can regard some activities or things as valuable without valuing them oneself. At first glance, it might sound curious, but Scheffler indeed has a~point. He submits that he indeed finds many activities \textit{valuable} without \textit{valuing} them himself, including folk dancing or studying Bulgarian history. He points out, though, that he usually does not engage in the activities he considers valuable without valuing them himself, which leads to the supposition that one values an activity when one finds it both valuable and engages in it. But Scheffler 
%\label{ref:RND3GsD9mZcKK}(2011, p.27)
\parencite*[][p.27]{Scheffler2011Valuing} %
 easily rejects this option by arguing in the following way: ``I may, for example, go to the opera from time to time, and I~may regard operagoing as a~valuable activity, and yet I~may still not value it myself. Even though I~participate in the activity and believe that it is a~valuable activity, operagoing may leave me cold.''



How does it translate into the value of life problem? It is possible to believe that life (\textit{resp.} living) is valuable, but not to value it oneself, even though we cannot help but be alive or, so to speak, participate in our own living when arguing. Hence, when we know that our family member or our best friend values life, whereas we do not, yet we do find it valuable, then terminating our life (even through suicide) to save them would be in accordance with the contention that we may commit suicide while recognizing the value of life, that is to regard life as valuable while not to value it ourselves. The question remains whether this argument also applies to the situation when the \textit{ultimate} value of life is the case, that is to (2). This question seems challenging as a~psychological empirical rather than the conceptual one. In the conceptual sense, it is not self-contradictory or self-defeating to assert that life is ultimately objectively valuable, while not to value one's own life, which runs counter to Rothbard's contention about the axiomatic character of the ultimate objective value of life. While it is true that this finding is based on a~specific, though compelling, definition of valuing proffered by Scheffler, it is also true that Rothbard's conceptualization of valuing as demonstrating one's preference is not universally binding, as was previously argued.



Let us now deal with the proposition (1). Let us pose the following question: How is it possible to sustain one's life when one does not affirm it so that one should not commit a~logical fallacy, which Rothbard warns us against?



To obtain a~correct answer to the above poser, it is enough to employ the praxeological reasoning of the Austrian school of economics; or, strictly speaking, its distinctive theory of time preference and the insights on ends and means, which Rothbard omitted when studying the problem of the value of life. By arguing that if a~particular living person were opposed to life, then---instead of merely declaring it---he would kill himself, Rothbard erred because in reality sustaining one's life while not affirming it is logically possible precisely due to recognizing the validity of knowledge on human action. He believed that a~person who is opposed to life should---not in a~normative sense, but rather as a~logical consequence---commit suicide instead of speaking of it at all. ``Instead of'' has two meanings here. The first is simply about choosing a~different conduct, more conducive to the realization of an adopted end. The second meaning is in turn related to time---it implies that the time dedicated to formulating one's position should (then again, only in a~logical sense and not in a~normative one) be used for suicide.



On the basis of the science of human action, Rothbard's reasoning may be called into question in two mutually related areas (incidentally, what also applies here is the above-mentioned argumentation based upon getting deliberately poisoned by mushrooms and engaging in a~conversation while awaiting death). The first of these rests on the theory of time preference.



The time preference principle is that an actor decides to give up a~present good in preference for a~future good only if she perceives in the latter a~possibility of greater satisfaction than the one she could have obtained if she had chosen the former. Individuals differ in their respective time preference rates (when it comes to the same individual, it may also vary over time), which stems from the estimation of satisfaction derivable from the present consumption as compared to the future one. Hence, one is warranted in speaking of both high and low time preference, which means, respectively, weaker or stronger tendency to prefer goods attainable later and providing more satisfaction over the ones attainable earlier but ensuring lesser satisfaction. For the individuals with high time preference, what counts is the present and whatever happens right afterwards. They expect immediate effects of their actions, allowing only for small delays. On the other hand, the individuals characterized by low time preference are oriented at future and that is why they appreciate the immediate consumption much less 
%\label{ref:RNDLqHCMVZHce}(see Hoppe, 2007, pp.1–6; Mises, 2008, pp.478, 481).
\parencites[see][pp.1–6]{Hoppe2007Democracy}[][pp.478, 481]{Mises2008Human}.%




In his analysis, Rothbard fails to consider this---crucial to the Austrian school---temporal aspect of action and its implications. In his reasoning with respect to the problem of the value of life he either presupposes very high time preference of an actor or ignores the principle of time preference altogether. In consequence, Rothbard does not heed the difference in the rate of time preference both across individuals and within the same individual but over time. Any lower than the highest possible rate of time preference (i.e., looking for an immediate satisfaction) is not even considered by Rothbard---as if no human could possibly instantiate it. To elucidate this mistake, let us assume that an individual who fails to see the ultimate value of life actually prefers death to life. In this sense, death in his view becomes a~good, quite as in the previously considered situations. Just for the sake of clarity, it does not imply the acceptance, at least for the sake of argument, of relativism, which would be incompatible with Rothbard's ethical absolutism 
%\label{ref:RNDxwtNXrYNAQ}(see Rothbard, 2008).
\parencite[see][]{Rothbard2008Toward}. %
 However, an actor might go astray, subjectively perceiving death as a~value 
%\label{ref:RNDFalGj7iknE}(cf. Nozick, 1971, p.252).
\parencite[cf.][p.252]{Nozick1971On}. %
 In this scenario, Rothbard's argumentation may be accepted only when immediate suicide would be the most preferred by the proponent of death analyzed herein. Then, there would be indeed no point in postponing one's suicidal act because, \textit{ceteris paribus}, committing it in the future would not bring any additional benefits as the optimal solution would be at hand---here and now.



However, interpreting life and death of an actor \textit{exclusively} in terms of highest, standing-alone and ultimate values is invalid. Again, based on Mises' 
%\label{ref:RNDe114bcTymm}(2008, pp.92–96, 216)
\parencite*[][pp.92–96, 216]{Mises2008Human} %
 praxeology, we can point out that each good, including life and health, may, depending on an actor and situation, serve either as an end or as a~means to realize one's end. Rothbard's reasoning applies to the first of these variations, whereas in reality denying the ultimate value of life does not necessarily nullify \textit{some} value of life. By the same token, denying the ultimate value of death does not necessarily nullify \textit{some} value of death\footnote{Raz 
%\label{ref:RND91eS6EtYsU}(2001, p.97)
\parencite*[][p.97]{Raz2001Value} %
 presented an interesting example of a~possible balance between the two. ``If, as I~suspect---he mentions---some people will take the option of dying younger, but not yet, it follows that some people value not dying soon even at a~cost to their longevity.''}. Hence, for an individual who lives and sustains his life (and thus does not kill himself), life \textit{may} (but does not have to) be only a~means to some end. In that case life constitutes a~good but only as a~means and thus it would not be endowed with the objective ultimate value. Its value in this case is instrumental or utilitarian---life is worth as much as it leads to a~given end. This would probably not be convincing for Raz 
%\label{ref:RNDfdXCznkk95}(Raz, 1999, p.191; 2001, pp.8, 77–78),
\parencites*[][p.191]{Raz1999Mixing}[][pp.8, 77–78]{Raz2001Value}, %
 who claimed life is a~precondition of both good and bad, thereby not recognizing life as such, as of value, however the foregoing findings bears some resemblance to the transcendental approach represented by Höffe 
%\label{ref:RNDLFX1W4e8n2}(1991; 1992),
\parencites*[][]{Höffe1991Gerechtigkeit}[][]{Höffe1992‘Even}, %
 except that he avoids the language of values, and only refers to interests. Hence, Höffe 
%\label{ref:RND41FuSo6U5V}(1992, p.131)
\parencite*[][p.131]{Höffe1992‘Even} %
 asserts that, by necessity, life is an elemental or transcendental interest of a~human being because it makes possible to desire something and to pursue it. Anyway, it is a~far cry from a~demonstration that life is a~perfective or intrinsic value.



That is why, Rothbard failed to prove the objective ultimate value of life. Rather, instead of proving the affirmation of life (as an ultimate value) on the part of a~given actor, he only proved---specifically under his praxeologically driven assumptions about \textit{valuing---}an actor's appreciating life to \textit{some} extent.\footnote{For similar evaluation of Nathaniel Branden's and Irfan Khawaja's positions who defend the claim on the existence of the ultimate value of live on the grounds of the objectivist philosophy 
%\label{ref:RNDWrpaLReHab}(see Moen, 2012, pp.97–98).
\parencite[see][pp.97–98]{Moen2012Is}. %
 However, the dispute over Rand's thesis is marked with other foundations and is rooted in the assumptions of objectivism; and hence, it is not compatible with the considerations herein and that is why it is of no interest to us here. See more in 
%\label{ref:RND1rFeizSUw9}(Nozick, 1971; Rasmussen, 2002, pp.69–86; Hartford, 2017, pp.54–67).
\parencites[][]{Nozick1971On}[][pp.69–86]{Rasmussen2002Rand}[][pp.54–67]{Hartford2017Ultimate}.%
} This in turn may imply an individual ascribing to life the highest as well as not the highest value. Moreover, this evaluation may vary over time. Therefore, Rothbard was right believing that a~person engaging in a~discussion recognizes the value of life; and yet he was wrong maintaining that based on the fact of participation in a~discussion, he managed to prove the axiomatic character of the proposition that ``life \textit{should} be an objective ultimate value.''



\section{Illustration from suicide terrorism}

In this section, I~will present an explicit illustration of the above considerations, that is an example of a~suicide terrorist, in order to support my claims by reference to empirical observations.



A~suicide terrorist does not believe that his life is divested of value but treats it as a~means to the end which is killing other people by dint of suicide terrorism. On the other hand, when it comes to jihadists, mere suicide is prohibited in Islam and that is why suicide terrorism is interpreted as a~heroic act of martyrdom, which is later rewarded with salvation 
%\label{ref:RNDtBLlnDgMaC}(see Bruce, 2013, pp.27–33; Roy, 2016, pp.15–24).
\parencites[see][pp.27–33]{Bruce2013Intrinsic}[][pp.15–24]{Roy2016Can}. %
 Moreover, according to jihadism, martyrdom is not the end of life; rather, it is the assurance of eternal life in paradise 
%\label{ref:RNDEaU40cdZIh}(Kruglanski et al., 2009, p.336).
\parencite[][p.336]{Kruglanski2009Fully}. %
 Kruglanski et al. 
%\label{ref:RNDAjcKXakDFB}(2009, p.336)
\parencite*[][p.336]{Kruglanski2009Fully} %
 note that ``paradoxically, \textit{the willingness to die in an act of suicidal terrorism may be motivated by the desire to live forever},'' however this excerpt deals with life after death, whereas the present paper concerns earthly lives, which are effectively terminated by suicide. That is why, at least from the vantage point of my analysis, the former position is not a~paradox. Amongst other motives for the terrorist's choice of an end in the form of suicidal terrorism, one normally enumerates such values as honor, dedication to the leader, social status, personal significance, feminism, restoration of the glory of Islam, moral obligation, money and support for one's family as well as other motives: pain and personal loss, group pressure, humiliation and injustice, vengeance, need to belong 
%\label{ref:RNDrS5AwO00G4}(cited in Kruglanski et al., 2009, p.332).
\parencite[cited in][p.332]{Kruglanski2009Fully}. %
 Furthermore, a~terrorist normally does not make his assault in an accidental place at accidental time because what matters to him is effectiveness. Alakoc 
%\label{ref:RNDFzDqqnCIUZ}(2017, p.1)
\parencite*[][p.1]{Alakoc2017When} %
 finds on the basis of statistical data that the popularity of suicide terrorism stems from its ``effective strategy for terrorizing by killing'' 
%\label{ref:RNDL7sUqX9gw1}(see also Hutchins, 2017, pp.7–11; Sheehan, 2014, pp.81–92).
\parencites[see also][pp.7–11]{Hutchins2017Islam}[][pp.81–92]{Sheehan2014Are}.%
\footnote{The statement that suicide terrorist attacks allow for killing a~greater number of people is undermined by Mroszczyk 
%\label{ref:RNDf1NguY0f0R}(2019).
\parencite*[][]{Mroszczyk2019To}. %
 However, he defends the crucial premise on the weight of the effectiveness of a~terrorist attack 
%\label{ref:RND7Vms12QaTo}(see Mroszczyk, 2019, pp.346–366).
\parencite[see][pp.346–366]{Mroszczyk2019To}.%
} It would be unjustifiable to generalize, but there are cases in which the suicide terrorist's utility is higher when more, rather than fewer, people die 
%\label{ref:RNDzKdCnO8RYg}(see Asthappan, 2010, p.25; BBC, 2019).
\parencites[see][p.25]{Asthappan2010Effectiveness}[][]{BBC2019German}. %
 Crabtree 
%\label{ref:RNDbEAT4wJIBg}(2006, p.577)
\parencite*[][p.577]{Crabtree2006Terrorist} %
 explains: ``Terrorist homicidal bombs are designed and detonated in a~manner that will maximize destructiveness against persons rather than against property. They are detonated in areas that are known to be occupied and often crowded and commonly are engineered to release metallic fragments for the purpose of increasing injury severity.'' In a~similar vein, states Alakoc 
%\label{ref:RNDW89FeIdkpA}(2017, p.6):
\parencite*[][p.6]{Alakoc2017When}: %
 ``Partial success occurs when a~suicide bomber detonates a~bomb earlier than originally planned but still causes civilian deaths and injuries.'' These as well as the other observations endorse the assumption that normally terrorist attacks are not spontaneous. Quite the contrary, terrorists prepare organizationally and logistically for an attack to maximize its expected effectiveness; although, when it comes to a~jihadist suicide, one must make a~caveat that in line with his ideology, even the minimum expected effectiveness is supposed to ensure him eternal life in heaven.



Based on the research conducted on lone-actor terrorists divided into two groups: the individuals closely connected with the radical groups---Autonomous (N=23), and the ones being unpredictable, impulsive and being more loosely connected with the radical groups---Volatiles (N=10), Lindekilde, O'Connor and Schuurman 
%\label{ref:RNDFtr5C7DJ4r}(2019, p.126)
\parencite*[][p.126]{Lindekilde2019Radicalization} %
 conclude that preparation periods for attacks in the case of Autonomous fluctuate around 48 months before a~scheduled attack on average; whereas in the case of Volatiles---on average four months before an attack. On the other hand, the time of planning an attack is, respectively, ten and four months. Furthermore, based on empirical research findings, Faria 
%\label{ref:RNDP8kgmPnIiW}(2003)
\parencite*[][]{Faria2003Terror} %
 argues that ``the number of terrorist activities decreases with terrorists' rate of time preference. That is, higher terrorist impatience leads to less successful terrorist activities.''



To conclude, this short illustration from suicide terrorism points to the consistency between the above theoretical analyses and the results of empirical studies on the phenomenon of contemporary terrorism.



\section{Conclusions}

In the present paper, I~have demonstrated that Rothbard's argumentation for the objective ultimate value of life is fallacious, which is mainly due to his failing to take into account the knowledge on individual value scales and the theory of time preference, as elaborated by the Austrian school of economics. This conclusion poses a~challenge for those libertarians who have adopted Rothbard's position, and particularly for those who recognize it as an argument for libertarian rights. However, although some scholars cited in the introduction to the present paper emphasize the relevance of the thesis of the value of life for the libertarian political philosophy, the general framework of libertarianism is not necessarily challenged by the conclusion of the present article because the libertarian rights-based discourse might be independent of the axiological one. It becomes clear when one bears it in mind that libertarian authors, including Rothbard, argue that the basic libertarian right, i.e., self-ownership, is an axiom and, as such, an adequate safeguarding measure ensuring a~conflicting-avoiding and just social order 
%\label{ref:RNDR1Z1YPtWlV}(see, e.g., Child, 1994, p.736; Eabrasu, 2013; Kinsella, 2009, pp.184–186; Rothbard, 1998, p.60; 2006, pp.47–48; see also liberal account of Waldron, 1988, pp.399–400).
\parencites[see, e.g.,][p.736]{Child1994Can}[][]{Eabrasu2013Rothbard’s}[][pp.184–186]{Kinsella2009What}[][p.60]{Rothbard1998Ethics}[][pp.47–48]{Rothbard2006For}[see also liberal account of][pp.399–400]{Waldron1988Right}.%




One could possibly reach other conclusions if one were to regard as precise Hoppe's 
%\label{ref:RNDVRDvSUQlpL}(1998, p.xxxiv)
\parencite*[][p.xxxiv]{Hoppe1998Introduction} %
 following claim in which he comments upon the weight of Rothbard's argumentation for the axiomatic nature of the value of life:



\begin{quote}
Rothbard's distinct contribution to the natural-rights tradition is his reconstruction of the principles of self-ownership and original appropriation as the praxeological precondition---\textit{Bedingung} \textit{der Moeglichkeit---}of argumentation, and his recognition that whatever must be presupposed as valid in order to make argumentation possible in the first place cannot in turn be argumentatively disputed without thereby falling into a~practical self-contradiction.
\end{quote}



However, in reality, it was only Hoppe himself who made wider use of the principle of performative non-contradiction to justify the libertarian property rights 
%\label{ref:RNDmv1jBB9lnB}(see Hoppe, 1988; 1989, ch. 7),
\parencites[see][]{Hoppe1988From}[][ch. 7]{Hoppe1989Theory}, %
 whereas Rothbard's argumentation, scrutinized herein, does not transcend a~discussion over values; and that is why its rejection does not have to pose general challenges to the libertarian political philosophy, which is primarily based upon self-ownership and the principle of original appropriation. Having said that, any attempt, as the one by Meng 
%\label{ref:RNDv6s6nRoY4N}(2002),
\parencite*[][]{Meng2002Hoppeing}, %
 to develop the libertarian political philosophy based on Rothbard's argument rejected in the present study must be unsuccessful.




\end{artengenv}

\label{nowakowski-last}