\setcounter{secnumdepth}{1}





\title{Szablon-EN}

\begin{document}

On the philosophy and logic of human action: A~Neo-Austrian Contribution to the Methodology of the Social Sciences





Dr. Michael Oliva Córdoba



University of Hamburg



Department of Philosophy



Von-Melle-Park 6



20146 Hamburg



Germany



\href{mailto:michael.oliva-cordoba@uni-hamburg.de}{\textstyleInternetLink{\textit{michael.oliva-cordoba@uni-hamburg.de}}}



\url{https://orcid.org/0000-0001-7299-3070}



Abstract



Philosophical action theory seems to be in pretty good shape. The same may not be true for the study of human action in economics. Famous is the rant that the study of human action in economics gives reason to tremble for the reputation of the subject. But how does this come about? Since economic action is about action, the broader study must surely have a~strong impact on the more specific field. The paper sets out, from the ground up, how an essential concept in economic theory–the concept of competition–can fundamentally benefit from insights derived exclusively from analytical action theory broadly conceived. In doing so, the paper delivers on an old Austrian promise: it is sometimes claimed that Austrian economists understand competition better than most economists. This may be a~bold claim, since Austrian economists have neither traced the understanding of subjectivity to its very origin (the theory of intentionality), nor have they traced their sympathy for methodological individualism in relation to market processes to its very ground (the theory of (human) action). This paper aims to fill this gap. Moreover, by grounding an Austrian view of competition in analytic action theory, it succeeds in avoiding the serious problems of the dominant equilibrium approach. By explaining competition as rivalry, the paper draws on the philosophy and logic of human action to bring the (economic) agent back into play. In this way, a~case is made for an integrated view of Austrian theory as an amalgam of Austrian economics and analytic action theory.



Key words:



Competition • Rivalry • Equilibrium Theory • Action Theory • Subjectivism • Ludwig von Mises



\section{Introduction}

In the last century, much attention has been paid to the philosophy and logic of human action. Milestones in its development were Anscombe's \textit{Intention} 
%\label{ref:RNDwpmhr1p7kL}(1957),
\parencite*[][]{}, %
 Davidson's ``Actions, reasons and causes'' 
%\label{ref:RNDrARiCC0I40}(1963)
\parencite*[][]{} %
 and von Wright's \textit{Explanation and understanding} 
%\label{ref:RNDgx1JdbCQ0v}(1971).
\parencite*[][]{}. %
 Anscombe sought to highlight the knowledge basis that must be invoked when attributing an action to someone. Davidson defended the claim that action explanations are a~kind of causal explanations. Von Wright pointed out that explanations in history and the social sciences take very different forms. These studies arguably shaped the form of the philosophical discipline now known as action theory. They triggered a~multitude of philosophical contributions that eventually broadened the perspective on the philosophy and logic of human action to encompass approaches as diverse as critical reviews of ancient problems 
%\label{ref:RNDqWaAMGRskc}(such as the problem of weakness of will, cf., e.g., Mele, 2010; Walker, 1989; Davidson, 2001 [1970])
\parencites[such as the problem of weakness of will, cf., e.g.,][]{mele_weakness_2010}[][]{walker_problem_1989}[][]{davidson_actions_2001} %
 and contemporary concerns about normative aspects of reason-based approaches 
%\label{ref:RNDTM6NYuMMlD}(such as patient autonomy in medical ethics and related problems, cf., e.g., Zambrano, 2017; Flanigan, 2016; Jennings, 2009).
\parencites[such as patient autonomy in medical ethics and related problems, cf., e.g.,][]{zambrano_patient_2017}[][]{flanigan_obstetric_2016}[][]{jennings_agency_2009}. %
 Thus, the stream became a~river, and the river became an ocean. Today, there is no denying that action theory is in pretty good shape. Of course, there are controversies and difficulties in action theory, as in all other scientific disciplines. But there is a~solid consensus on the phenomena to be explained, there are paradigmatic theories that are referred to again and again, and there are classic contributions that offer points of contact for old insights and new debates. Although there are specialists in the field, philosophical action theory is by no means marginalised. Even theorists who do not specialise in action theory acknowledge its relevance for practical disciplines without hesitation. Philosophers of any provenance also usually have more than a~hunch that the relevance of action theory must somehow spill over into the social sciences themselves. And last but not least: Being a~philosopher of action is neither leftist, centrist or rightist. It has no hidden or obvious implications for your ideology, political and moral views or creed. So, it is safe to say that as a~scientific discipline, theory of action is a~decent, well-established and worthwhile subject to study.



In the social sciences, and especially in economics, this seems to be different. Apart from occasional lip service, the study of human action does not seem to have a~high priority in economics. This is especially true for \textit{praxeology}, the most comprehensive and complete economic approach towards the study of human action, which emerged from the Austrian school of economics. Praxeology has antedated philosophical action theory by about a~quarter of a~century. Unlike philosophical action theory, however, praxeology was not particularly well-received. One gets the impression that the study of praxeology is seen as a~trivial, partisan, dogmatic or shadowy endeavour. Some economists openly toy with the idea that praxeology is not a~scientific enterprise at all. The picture is emerging that the study of human agency in economics is considered to be a~serious threat to the respectability of economic theory. But how can the study of human action in economics, as \textit{Paul Samuelson} once put it 
%\label{ref:RNDPF4y5zZI10}(1964, p.736),
\parencite*[][p.736]{}, %
 give ``reason to tremble for the reputation'' of the subject, when economics is, as \textit{Alfred Marshall} 
%\label{ref:RNDkSaOWa77Q4}(1890, p.1)
\parencite*[][p.1]{marshall_principles_1890} %
 famously observed, ``a study of mankind in the ordinary business of life'' and a~study of ``individual and social action''?



The present paper is intended to help to resolve this tension and to make a~new attempt at justifying the importance that the study of human action can have for the social sciences and for economics in particular. This will be done by tracing economic problems, especially the problem of competition, back to their action-theoretical foundations. A~welcome side effect will be a~belated rehabilitation of the research programme that has unjustly brought Ludwig von Mises and the Austrian School of Economics into disrepute in the social sciences: If Mises' praxeology is ultimately interpreted as merely an early variant of what analytical action theory does in philosophy, then there is no reason to worry about the foundations of economics, quite the opposite.



\section{The need for a~better understanding}

The contrasting views of Marshall and Samuelson make it clear that something is fundamentally wrong with economists' understanding of the basics of their science. What has gone wrong? As always, the explanation is complex. I~can only hint at a~few elements. Certainly, the rise of socialism to scientific respectability in the early 20\textsuperscript{th} century played a~role. It raised hopes of the feasibility of a~supposedly superior system of objective central planning, freed from the arbitrariness of consideration for the individual. The same applies to the missionary impetus of the Vienna Circle. Even if some of its members liked to think they could keep their scientific work separate from their political goals, \textit{Otto Neurath} being a~prominent exception 
%\label{ref:RND3PGecJj6L9}(cf. Richardson, 2009, p.23; Carnap, 1963, p.23),
\parencites[cf.][p.23]{richardson_left_2009}[][p.23]{schilpp_intellectual_1963}, %
 the strong socialist undercurrent ensured a~remarkable anti-individualist tendency. Thus, the positivist view of science triggered by the Vienna Circle and the astonishing advances in the natural sciences led to a~view that was incompatible with a~subjectivist and individualist understanding of society and its sciences. No wonder many thinkers were tempted to align the social sciences with the natural sciences and mathematics. They still are today. The last element, but not the least, was the triumphant emergence of equilibrium theory. It gradually led to a~transformation of economic theory as a~whole. Contrary to its original intention, positive economics ultimately developed into a~normative enterprise. And as positivism, naturalism and normativism gained more and more influence, a~tendency towards objectivism seemed more and more inevitable.



All these issues have been discussed elsewhere. They have contributed significantly to the diminished importance of the study of human action in economics. Consequently, they led to the marginalisation of Austrian economics to the point where it was declared dead and mentioned only in historical retrospect. On this occasion, however, I~do not want to go into this research. The reason is that it is not entirely clear whether the study of action in economics is really best placed in the mainstream of the Austrian school of economics, at least in its present state. To be sure, there is no doubt that the Austrian school of economics openly professes \textit{subjectivism}, the central element in explaining human action. In the words of one of its most important representatives, \textit{Israel Kirzner}, ,,the Austrian school is usually and quite correctly identified with subjectivism. Subjectivism in economics means that Austrian economists are convinced that the regularities in economic life […] can be understood only by focusing analytical attention on individual actions`` of the human agent 
%\label{ref:RNDvEwaYbAVxW}(Kirzner, 2016, p.2:12).
\parencite[][p.2:12]{kirzner_history_2016}. %
 But this concession seems half-hearted in more ways than one.



First, what Kirzner calls the ``modern version of subjectivism'' aims to find a~middle ground between the ``flawed subjectivism of Menger'' and the ``nihilistic conclusions'' of the Shackle-Lachmann view 
%\label{ref:RNDUwquUHZxP1}(Kirzner, 1995, pp.14, 19; cf. Lachmann, 1982).
\parencites[][pp.14]{}[cf.][]{kirzner_ludwig_1982}. %
 This modern Austrian view thus rejects both Menger's ``heritage'' of perfect knowledge\footnote{
%\label{ref:RND18zwZr9hbq}(Kirzner, 1995, pp.14 \& 16)
\parencite[][pp.14 \& 16]{meijer_subjectivism_1995} %
 seems to assume such a~``legacy of perfect knowledge'' and I~will not dispute that: ``We have seen the central subjectivist thrust of Menger's vision. And we have seen the incompleteness of that vision (in its assumption of the normalcy of perfect knowledge.). […] We shall [steer] clear of […] the incompleteness in Menger's view (which led to the death of subjectivism in mainstream microeconomics).''} \textit{and} the idea of the radical spontaneity of choice. But while the first rejection is fully justified, the second is not. Denying the ``radical spontaneity of choice'' comes dangerously close to denying the essential autonomy of the agent. From the point of view of action theory, then, it remains a~mystery how the individual actions of the acting individual can be given the full weight they deserve without accepting much of what Kirzner calls ``nihilistic conclusions''. Therefore, one would really hope that ``Lachmann's influence on modern Austrian economics'' would be ``underappreciated'' and that his positions ``especially [on] subjectivism'' would be ``the dominant positions within the school'' 
%\label{ref:RNDVrtpYazpkd}(Storr, 2019, p.63).
\parencite[][p.63]{storr_ludwig_2019}. %
 Unfortunately, however, this may be an overly optimistic assessment.



Second, and more importantly, Kirzner's Austrian commitment to subjectivism underlines the importance of subjectivism in economics and the economisation of human action without really analysing subjectivism and human action in sufficient detail. Kirzner's ``modern Austrian subjectivism'' proceeds as if the \textit{differentia specifica} can be understood without understanding the \textit{genus proximum}. It treats subjectivism in economics and the economisation of human action as simple concepts whose meanings do not need to be broken down into their conceptual components. It seeks to redeem Ludwig von Mises' claim that economics is grounded in the theory of (human) action, but shies away from going beyond the boundaries of economic theory. And since it deals only with subjectivism in economics, it is silent on the nature of subjectivism itself.



Mises' assertion that economics is grounded in action theory was naturally quite disturbing to his fellow economists, even to some Austrians. One can understand why: reductive claims of this magnitude are rarely met with enthusiasm, especially by those whose discipline is subsumed under another. Consider the resistance that the positivist credo of the unity of the sciences met with in some natural sciences. Chemists and biologists usually pay lip service at best to the assumption that they are really concerned with physics. However, Mises' rallying cry found at least some support. The sociologist Alfred Schütz, a~long-time member of Mises' private seminar in Vienna 
%\label{ref:RNDaa9uzQdiX0}(Prendergast, 1986, p.5ff),
\parencite[][p.5ff]{prendergast_alfred_1986}, %
 echoed it: ‘\textit{All} social phenomena can be traced back to actions of agents in the social world, which in turn can be observed by social scientists' 
%\label{ref:RND58rPTjK0YJ}(Schütz, 1996, p.96; cf. 1953, p.26; Kurrild-Klitgaard, 2001, p.122).
\parencites[][p.96]{schutz_political_1996}[cf. 1953, p.26,][]{}[][p.122]{kurrild-klitgaard_rationality_2001}. %
 In order to give more substance to the claim that the social sciences, especially economics, are based on action theory, this paper will focus on the two aspects that have not yet received all the attention they deserve. We will focus on a~more general and thorough understanding of subjectivism and human agency. Subjectivism in economics and the economic aspects of human action will then emerge only as special cases.



It is clear that these investigations must be carried out independently of what they are later applied to. The impatient reader may therefore get the impression of a~somewhat lengthy diversions. However, since this is a~paper on proper foundations, there is no alternative to starting from scratch. Our reward will be a~picture of what the study of human action can contribute to the study of the social sciences. A~systematic and integrated approach will be outlined, showing what the philosophy and logic of human action can contribute to the social sciences at large and economics in particular. It will also show that it can contribute in this way without compromising the rigour, richness, and seriousness it deserves as the decent, well-established and worthwhile field of study that it is.



\section{The subjective and the objective: A~fundamental distinction}

\footnotetext{ For a~more detailed discussion of the following c\textit{f.} my forthcoming paper ``Subjectivity and objectivity. Intentional inexistence and the independence of the mind''.}

We can only understand subjectivism if we understand the subjective. Starting from scratch means going beyond economics and social sciences. Therefore, a~more fundamental science, \textit{i.e.}, philosophy, will be our guide. There, the distinction between the subjective and the objective has a~very long tradition. The terms go back to \textit{Aristotle's} Categories. In his translation, \textit{Boethius} 
%\label{ref:RNDL2rsHFWOjq}(cf. Minio-Paluello, 1961, 5:22; Aristotle, 1938, \textit{Cat.} 1a20)
\parencites[cf.][]{}[][]{aristotle_categories_1938} %
 uses the Latin word \textit{subiectum} as a~counterpart of the original Greek \textit{$\text{\textgreek{<u}}\pi o\kappa \varepsilon \text{\textgreek{'i}}\mu \varepsilon \nu o\nu $} (\textit{hypokeímenon}, the ``underlying thing''). However, our modern understanding of these terms dates back only to the early modern period. The distinction they mark as a~pair of opposites is usually described as a~kind of \textit{mind-(in)dependence}. As mathematician and logician \textit{Gottlob Frege} put it:



If we say ``The North Sea is 10,000 square miles in extent'' then neither by ``North Sea'' nor by ``10,000'' do we refer to any state of or process in our minds: on the contrary, we assert something quite objective, which is independent of our ideas and everything of the sort. 
%\label{ref:RNDXp0nrd47LI}(Frege, 1953, p.34)
\parencite[][p.34]{frege_foundations_1953}%




This understanding it is echoed time and again:



An element in some subject-matter conceptions of objectivity is \textit{mind independence}: an objective subject matter is a~subject matter that is constitutively mind-independent. […] By contrast, minds, beliefs, feelings, […] are not constitutively mind-independent, and hence not objective, in this sense 
%\label{ref:RNDf5sKcDXSwR}(Burge, 2010, p.46).
\parencite[][p.46]{burge_origins_2010}.%




So, according to the common view, the objective is objective insofar as it is independent of the mind, and the subjective is subjective insofar as it is not. But what exactly are the elements that make the subjective and the objective independent or dependent?



There are two paths open to us, the \textit{cognitive} and the \textit{attitudinal}. The cognitive way describes the element as a~\textit{perspective} or a~\textit{point of view}. To take a~subjective attitude towards something would be to look at it from a~particular perspective: the individual perspective of the subject. To take an objective attitude towards something would be not to look at it from a~particular perspective. In this way, it has become popular to distinguish the \textit{view from somewhere} against the \textit{view from nowhere} 
%\label{ref:RNDGi21Uuu7kQ}(cf. Nagel, 1979).
\parencite[cf.][]{nagel_subjective_1979}. %
 The most important metaphor of the cognitive path is the metaphor of the eye and what and how it sees. A~powerful metaphor indeed, but ultimately not a~very helpful metaphor: surely there can be no looking from nowhere. Therefore, we had better explore the other path, \textit{i.e.}, the path of attitude. In doing so, we implicitly acknowledge the importance of the \textit{intentional}. This is what the Austro-German philosopher \textit{Franz Brentano} considered to be the very characteristic of the mental 
%\label{ref:RND8DCAKIYKPd}(see, e.g., Crane, 1998; 2001; 2013).
\parencites[see, e.g.,][]{ohear_intentionality_1998}[][]{crane_elements_2001}[][]{crane_objects_2013}. %
 Brentano's much quoted illustration reads:



Every mental phenomenon is characterized by what the Scholastics of the Middle Ages called the \textit{intentional} (or mental) \textit{inexistence} of an object, and what we might call, though not wholly unambiguously, reference to a~content, direction toward an object (which is not to be understood here as meaning a~[real] thing), or immanent objectivity. Every mental phenomenon includes something as object within itself, although they do not all do so in the same way. In presentation something is presented, in judgement something is affirmed or denied, in love loved, in hate hated, in desire desired and so on. […] \textit{This intentional inexistence is characteristic exclusively of mental phenomena.} No physical phenomenon exhibits anything like it. We can, therefore, define mental phenomena by saying that they are those phenomena which contain an object intentionally within themselves. 
%\label{ref:RNDI7VRH284tM}(Brentano, 2009, p.68; orig. 1874; emphasis added)
\parencites[][p.68]{brentano_psychology_2009}[orig. 1874,][]{}[emphasis added,][]{}%




It is this passage where Brentano rediscovers the intentional. Eventually, this discovery led to the development of the \textit{theory of propositional attitudes}. This is because in natural language we are familiar with a~common feature that pretty much shows what Brentano took to be the defining feature of the mental: We recall that in natural language we very often attribute propositional attitudes to persons: We say, for example, that Tom \textit{believes} that the earth is flat, or that Dick \textit{wants} the man in the doorway to stop staring at him, or that little Harry \textit{hopes} that Father Christmas will come to visit next Christmas. Believing, wanting and hoping (and others) are \textit{propositional attitudes}; they are \textit{mental states} or \textit{events} attributed by reference to a~person experiencing the mental state or event and described by (the \textit{nominalisation} of) a~sentence within the \textit{scope} of an appropriate \textit{attitude verb}. That's a~lot of new vocabulary to learn, of course, but despite the new jargon it is not sophistry. It is a~natural feature of humans to have propositional attitudes, and it is a~natural feature of language that they can be expressed in natural language. Propositional attitudes are not sophisticated theoretical gimmicks, but part of the cognitive toolbox with which humans encounter the world. And, very importantly in this context, propositional attitudes have that very important feature of intentionality. This is the link to Brentano. For as the examples illustrate, someone can be in such a~state of mind that it can be correct to attribute a~certain propositional attitude to him, even if the object given in the attitude does not exist or is not as the subject imagines it. The earth is not flat, there is no Father Christmas, and sometimes we mistake a~reflection of ourselves for something or someone else. Nevertheless, Tom can believe that the earth is flat, Harry can hope that Father Christmas will come to visit next Christmas, and Dick can want the man at the door to stop staring at him. So, attitudes can have a~``real'' object, but they don't have to. You could say they provide an ``internal'' or ``intentional'' object. Or, as philosophers choose to express it, intentional objects are \textit{inexistent}, (propositional) attitudes display \textit{intentionality.}\footnotetext{ For the sake of simplicity, I~will refrain from adding ``propositional'' in the following where no misunderstandings are to be expected. In general, however, I~have no other attitudes in mind in this work than propositional ones. Moreover, for what could theoretically be called \textit{subpropositional} attitudes (like, \textit{e.g.}, \textit{making reference} to an object) I~would argue that these are only partial aspects in which one can regard ``fully-fledged'' or ``complete'' propositional attitudes and not a~separate category of attitudes in their own right.}





The intentional inexistence of objects and, by extension, that of attitudes is what best illustrates the attitude's intentionality 
%\label{ref:RND5TUDCgB8CW}(cf. Simons, 2009, p.xvi).
\parencite[cf.][p.xvi]{brentano_introduction_2009}. %
 It is also what constitutes the subjective. By providing an intentional object, attitudes bring out the subjective view of the individual who holds the attitude. Put differently: By describing attitudes, we describe \textit{the peculiar view} that Tom, Dick and Harry have of the earth, the man in the door and next Christmas. \textit{We are describing their subjective perspective.} Thus, we have an explanation of subjectivity that both makes the metaphor of the eye superfluous and is able to incorporate it: The cognitively subjective is subjective if and insofar as it is grounded in the attitudinally subjective. The mind-dependence that explains the subjective turns out to be a~dependence on the attitudes of the individual. The objective is thus objective because it is independent of the attitudes of the individual, and the subjective is subjective because it is not. So, all's well that ends well: The cognitive path leads to the attitudinal path, and the attitudinal path leads to the correct understanding of the matter.



In closing, let us illustrate the specificity of both subjectivity and individuality in a~more formal way. To do this, we use the basic language of modern attitudinal logic along the lines proposed in 
%\label{ref:RNDLzegm7vLXI}(Hintikka, 1962)
\parencite[][]{hintikka_knowledge_1962} %
 and explained, for example, in 
%\label{ref:RNDT3NCNO62Ji}(Ditmarsch et al., 2015, p.7).
\parencite[][p.7]{ditmarsch_handbook_2015}. %
 Let us extend it to apply to attitudes in general, using ``$\Delta $\textit{\textsubscript{x}}'' as a~proxy for any adequate form of an attitude operator, \textit{e.g}., ``B\textit{\textsubscript{x}}'' for ``\textit{x} believes that'', ``F\textit{\textsubscript{x}}'' for ``\textit{x} fears that'', and so on. Note that what ``$\Delta $\textit{\textsubscript{x}}'' is representative of involves the expression of an attitude subject and takes an indicative sentence as an argument (\textit{p}). We can now express that \textit{subjectivity} lies in the following fact of mutual non-entailment:



(Subjectivity)



(i) \textit{p} ${\nvdash}$ $\Delta $\textit{\textsubscript{x}} \textit{p}



(ii) $\Delta $\textit{\textsubscript{x}} \textit{p} ${\nvdash}$ \textit{p}



Thus, from the fact that Columbus discovered America (\textit{p}), it does not follow (${\nvdash}$) that he believed he discovered America (B\textit{\textsubscript{x}} \textit{p}). Nor does it follow (${\nvdash}$) from the fact that George VI did not want to follow his brother to the throne (W\textit{\textsubscript{x}} \textit{p}) that he did not follow his brother to the throne (\textit{p}). No special knowledge of early modern or modern history is needed to see this. It is already analytically contained in our understanding of behavioural verbs. By extension, we can characterise \textit{individuality} by the following fact of intrapersonal non-entailment (for \textit{x} ${\neq}$ \textit{y}, of course):



(Individuality)



(i) $\Delta $\textit{\textsubscript{x}} \textit{p} ${\nvdash}$ $\Delta $\textit{\textsubscript{y}} \textit{p}



(ii) $\Delta $\textit{\textsubscript{y}} \textit{p} ${\nvdash}$ $\Delta $\textit{\textsubscript{x}} \textit{p}



From the fact that Cleopatra (\textit{x}) feared being brought to Rome and paraded in the streets as part of Octavian's triumphal procession (F\textit{\textsubscript{x}} \textit{p}), it does not follow (${\nvdash}$) that Octavian (\textit{y}) feared bringing Cleopatra to Rome and parading her in the streets as part of his triumphal procession (F\textit{\textsubscript{y}} \textit{p}). Nor does it follow from the fact that Odysseus hoped that the Trojans would drag the wooden horse to their city (H\textit{\textsubscript{y}} \textit{p}) that Laocoon hoped this (H\textit{\textsubscript{x}} \textit{p}). Again, all that is required is a~proper understanding of the corresponding verbs. So, in the end, mind-independence amounts to mutual non-entailment.\footnote{In my ``Subjectivity and objectivity'' I~argue it is even stronger and comprises \textit{causal independence} as well.}



Let us summarise: One's attitudes are independent of both the world at stake and the attitudes of others. We happen to have stumbled upon the fact that the subjective-objective gap is, from a~certain point of view, simply the gap between mind and world. What is subjective is subjective because it depends on someone's attitudes. What is objective is objective because it does not depend on anyone's attitudes. Certainly, more could be said about the subjective, the objective and their distinction. But none of what has been said could be a~sound insight into the matter if it were not ultimately based on this. So basically, we have just based the subjective-objective distinction on the unique mental feature of intentionality, \textit{i.e}., intentional inexistence. We must leave it at that, however, because we need to move on quickly to the next topic, the topic of (human) action. To this I~turn now.



\section{Foundations of action theory}

\footnotetext{ For more detailed discussions, please refer to my book \textit{Analytical Action Theory, Fundamentals and Applications} [in German], forthcoming from Academia-Verlag, Baden-Baden.}

We have understood what the subjective is: it is what we understand to be dependent on a~person's attitude. Now we need to understand what action is. Our everyday talk about our actions will serve as a~guide. Using the long-established \textit{method of variation}, we can identify the underlying basic categories of action in what the average person would regard as accounts of action. This sort of corpus analysis is basically best practice among logicians, semanticists and linguists. They all use this method when defining basic categories via \textit{distribution}, even if they apply it to different domains 
%\label{ref:RNDYd59Ns4Wnw}(see, e.g., Burton-Roberts, 2016, p.46; Tallerman, 2015, p.34; Lewis, 1970, p.20ff; Lyons, 1968, p.147; Ajdukiewicz, 1935, p.3; Husserl, 1913, p.242; all anticipated by Frege, 1891; Engl. transl. 1960, p.189; and Plato, 1921 [Sophist 261d-262e]).
\parencites[see, e.g.,][p.46]{burton-roberts_analysing_2016}[][p.34]{tallerman_understanding_2015}[][p.20ff]{lewis_general_1970}[][p.147]{lyons_introduction_1968}[][p.3]{ajdukiewicz_syntaktische_1935}[][p.242]{husserl_prolegomena_1913}[all anticipated by][]{frege_function_1891}[p.189,][]{}[and][]{}.%




Our starting point is that accounts of action, when properly ordered, are \textit{substitution instances} of each other. This is true across contexts, styles and registers. So



(1) Peter eases the jib because he thinks that will stop the main from backing (and he wants it to)\footnote{Natural language is quite economical, \textit{cf.} 
%\label{ref:RNDfl51DVSlzk}(Davidson, 1963, 6f.):
\parencite[][]{}: %
 ``[I]t is generally otiose to mention both, If you tell me you are easing the jib because you think that will stop the main from backing, I~don't need to be told that you \textit{want} to stop the main from backing; and if you say you are biting your thumb at me because you want to insult me, there is no point in adding that you \textit{think} that by biting your thumb at me you will insult me'' (emphasis added).}



and



(2) Oedipus married Jocasta because he wanted to ascend the throne of Thebes (and thought he would if he did)



can be understood as resulting from each other by substitution \textit{salva congruitate}. That means that the substitution of an appropriate non-logical part of speech with a~categorically equivalent one does not transform an account of action into something that would not count as such. Of course, substituting ``Oedipus'' in (2) with ``Peter'' from (1) or ``wanted to ascend the throne of Thebes'' in (2) with ``wants to stop the main from backing'' in (1), \textit{etc.}, may turn a~correct action report into one that is most likely false. However, since we are not concerned with truth, but only with logical form, conceptual structure, and, ultimately, understanding, this difference does not matter. On the contrary, it gives us the \textit{canonical form of action reports} (A):



(A) \textit{x} \textit{$\varphi $-s} because \textit{x} wants that \textit{p} \& \textit{x} believes that \textit{x~$\varphi $-s} \ding{213} \textit{p}



This rendering now brings our logico-linguistic approach to fruition. For (A) manifests, understood distributively, the basic categories of action. We can thus distinguish the formal categories of \textit{agent}, \textit{doing}, \textit{wanting} and \textit{believing} in the following way: We take an \textit{agent} to be whatever is made reference to by an appropriate substitution instance \textit{salva congruitate} in the argument place indicated by [231C?]\textit{x}[231D?]; we take a~\textit{doing} to be whatever is made reference to by an appropriate substitution instance \textit{salva congruitate} in the argument place indicated [231C?]\textit{$\varphi $-s}[231D?]; and we proceed in the same way with regard to the remaining categories. When done correctly, we arrive at something closely resembling the classic Davidson \textit{belief desire model} of human action, where acting would be doing something for a~reason. This is the general model favoured by Anscombe 
%\label{ref:RNDw4miDnJrB8}(1957),
\parencite*[][]{anscombe_intention_1957}, %
 Davidson 
%\label{ref:RNDartTj7xlUB}(1963)
\parencite*[][]{} %
 and Wright 
%\label{ref:RND49dalMUE3q}(1971)
\parencite*[][]{} %
 in their respective versions, and it is probably fair to say that it is generally accepted nowadays. However, we arrive at our version of this model in a~purely formal, purely descriptive way, with the fewest possible theoretical presuppositions and without unwanted ballast. This spares us a~whole series of substantial and often controversial theoretical assumptions that are common in action theory today.\footnote{Is acting a~kind of doing and doing a~kind of bodily movement? But then how about \textit{mental actions}? 
%\label{ref:RNDi0JtkevpFW}(cf., e.g., O'Brien and Soteriou, 2009).
\parencite[cf., e.g.,][]{obrien_mental_2009}. %
 And are all doings extended in time? 
%\label{ref:RNDjrdhQNlYvw}(Frankfurt, 1978, p.158)
\parencite[][p.158]{frankfurt_problem_1978} %
 But then how about \textit{point actions} like, \textit{e.g.}, finishing a~paper or taking Mary to be your lawfully wedded wife? Other questions in this context would be whether there is a~causal sense of ``because'' that ensures that action explanations are causal explanations 
%\label{ref:RNDgpo0HfVbSy}(cf. Davidson, 1963)
\parencite[cf.][]{} %
 and, frankly, even whether the agents must necessarily be human beings. We need not go into all these thorny issues here: They only arise if one adds substantial assumptions to our minimalist explanation of action, which is not at all necessary at this point. }



The formal understanding we have arrived at also rewards us with a~formal understanding of what reasons for action (also known as ``motivating reasons'') are. Recall that it is common to call anything that starts with the connective ``because'' in response to a~``why?'' question a~\textit{reason}. Why is four even? \textit{Because it is divisible by two}. Why did the dinosaurs become extinct? \textit{Because the Chicxulub asteroid hit the Gulf of Mexico some 65 million years ago}. In relation to (1) and (2), the reason for Peter's manœuvre and Oedipus' marriage to Jocasta is what is given in response to a~corresponding question in (1) and (2) respectively. Reasons for action are thus hybrid. They are given by the combination of two particular attitudes: Peter's (or Oedipus') \textit{wanting} that \textit{p} in conjunction with his (or Oedipus') \textit{believing} that if he \textit{$\varphi $-}s, then \textit{p}.



This fits in well with our findings from the previous section. Given that reasons for action are described by complex attitudes, it is clear that motivating reasons are (i) \textit{subjective} and (ii) \textit{individual} in the following ways: (i) someone's reason is neither implied nor otherwise determined by how things are, nor does it imply or otherwise determine how things are; (ii) a~reason for one need not be a~reason for the other. Moreover, because of the intentionality of attitudes, the reason of the agent can, but need not, collide with reality. It can lead to failure. But that is just grist to our mill because, surely, an unsuccessful action is still an action. On closer examination, this raises an even more interesting question: If the reasons for action must necessarily be seen as subjective and individual, what about the talk of objective reasons that is so prominent today? Indeed, the essential subjectivity and individuality of motivation bears a~striking resemblance to the \textit{eye of the needle} in Matthew 19:24: ``And again I~say unto you, It is easier for a~camel to go through the eye of a~needle, than for a~rich man to enter into the kingdom of God.'' Since objective reasons are neither necessary nor sufficient for an agent's actions, but his subjective reasons are, it seems that objective reasons are like the rich man in the Gospel. Like him, who would have to divest himself of his wealth in order to enter the kingdom of God, objective reasons would have to divest themselves of their objectivity and instead become subjective in order to truly motivate. Thus, in order to truly explain human action, one cannot ultimately abstract from the individual agent and his subjective reasons.



If we take stock now, we see that to act is to do something for a~reason. For most people this is just a~platitude. But the way we have derived it has unlocked the foundations of action theory. And since we started from scratch, we now know exactly the theoretical presuppositions we encountered. In particular, we see that in our approach they are minimal and purely descriptive. Interestingly, human action is also seen as necessarily subjective and individual in the Austrian School of Economics. This is what Austrian subjectivism boils down to, or at least it should be based on. Our brief examination of the philosophy and logic of human action, however, was conducted independently of any economic and social science presuppositions. Frankly, it was independent of all questions of practical disciplines, including moral philosophy, political theory, law and economics and so on. Our subjectivism is thus based on nothing other than a~fundamental understanding of intentionality and a~distributional analysis of action reports. As a~result, it is much more comprehensive than the surrogate discussed in economic methodology or the social sciences at large. Subjectivism in economics or the social sciences now appears only as a~special case.



It should be noted in passing that those branches of philosophy that usually come into play when economists discuss the foundations of their subject\textit{, i.e.}, Kantianism, positivism, sometimes even phenomenology or hermeneutics, were neither necessary nor helpful. To dispel a~common misunderstanding about the ultimate foundation of economic science, it must also be pointed out that our enquiry was by no means \textit{epistemological} either 
%\label{ref:RNDtyxHyOgZWT}(\textit{pace} Mises, 1962).
\parencite*[][]{}. %
 Thus, since the ultimate foundation of economic science is the philosophy and logic of human action---just as (young) Mises rightly said, Austrians should like to assume, and as was demonstrated in the previous reasoning---what (old) Mises arrived at at the \textit{end} of his intellectual development, namely that these ultimate foundations were epistemological, cannot also be true. It is not: Action theory is \textit{not} epistemology; it has nothing essential in common with it. To assume otherwise is simply to commit an error in judgement.\footnote{Unfortunately, it will hardly help to make the Kantian point that ``in some sense epistemology is the basis of all the sciences''. At the end of the day, that is quite a~strong statement. It presupposes its own truth and, lamentably, proves nothing. Following 
%\label{ref:RNDaNem1NNmgA}(Fumerton, 2017, p.3)
\parencite[][p.3]{fumerton_epistemology_2017} %
 one could complain that proponents of such a~view are ``simply trying to legislate a~meaning for the term ‘[science]' a~meaning that has little bearing on how the term is actually used.'' \textit{Kant}, of course, thought otherwise. But the history of philosophy has not been kind to this kind of epistemological imperialism. Kantian idealism is in part excused, however, since Kant planted his flag well before the advent of modern-day logic and formal semantics. But it is fair to say that advances in philosophical reasoning, particularly in logic and semantics, have ensured that the idealist stance in philosophy has not aged well. It may well be that the present foundational stance in philosophy, adopted by the Vienna Circle and acknowledged by Fumerton 
%\label{ref:RNDznyFyvfiQv}(2017, p.14),
\parencite*[][p.14]{fumerton_epistemology_2017}, %
 is an exaggeration too: ``All philosophy is a~‘critique of language''' 
%\label{ref:RNDpQECsk5Coz}(Wittgenstein, 1922; 2013, 4.0031).
\parencites[][]{wittgenstein_tractatus_1922}[][]{}. %
 But surely, \textit{that} is a~different kind of exaggeration. One that places logic and semantics at the heart of science. Not epistemology.} This should be a~serious warning to all those Austrians who are in the habit of saying that there is an epistemological problem at the core of economics 
%\label{ref:RNDa61qO57Q7R}(cf. Condic and Morefield, 2021; Rajagopalan and Rizzo, 2019, p.94; Knudsen, 2004; Yeager, 1994; Ebeling, 1993, p.63f.; Boettke, 1990, p.23ff.; Lavoie, 2015, p.50; Hayek, 1945; 1948, p.33; Schütz, 1996, 98f).
\parencites[cf.][]{condic_hayek_2021}[][p.94]{rajagopalan_austrian_2019}[][]{knudsen_alfred_2004}[][]{yeager_mises_1994}[][p.63f.]{herbener_economic_1993}[][p.23ff.]{boettke_political_1990}[][p.50]{lavoie_rivalry_2015}[][]{hayek_use_1945}[][p.33]{hayek_economics_1948}[][]{}. %
 More importantly, however, we have seen the sketch of a~sound and solid philosophical basis for the study of human action. As the study of human action in economics is quite often accused of resting ``on a~weak philosophical foundation'' 
%\label{ref:RNDMbwknyPHli}(cf. Barrotta, 1996, p.65),
\parencite[cf.][p.65]{barrotta_neo-kantian_1996}, %
 it is almost vital to be able to show that we are not like the foolish man in Matthew 7:27 who built his house on sand, and it rained, and the flood came, and the winds blew and beat against the house, and it fell. And that is precisely what we have shown.



\section{The study of human action in economics}

We have now acquired a~sufficiently thorough and solid understanding of the concept of human agency and the phenomenon of subjectivity. If economics is really a~part of the ``study of mankind in the ordinary business of life'' and an examination ``of individual and social action'' (Marshall 1890, 1), we should expect these insights to bear fruit in relation to essential economic questions. In fact, first steps in this direction have already been taken when, with the help of analytic action theory, it was shown that two cornerstones of praxeology, the \textit{Uneasiness Theorem} and the \textit{Scarcity Theorem}, are analytic, hence not synthetic, but nevertheless a~priori 
%\label{ref:RNDM3l39q66MF}(Oliva Córdoba, 2017).
\parencite[][]{oliva_cordoba_uneasiness_2017}. %
 The Uneasiness Theorem, which states that the incentive to act is always uneasiness 
%\label{ref:RNDFSRgIQxEo3}(Mises, 1998, p.13),
\parencite[][p.13]{mises_human_1998}, %
 and the Scarcity Theorem, which states that action is the manifestation of scarcity 
%\label{ref:RNDWK5Y8nAnaZ}(Mises, 1998, p.70),
\parencite[][p.70]{mises_human_1998}, %
 are at the centre of Mises' programme to ground economic theory in action theory. Given the controversial nature of this programme even among Austrian economists, it seems that this justification of the proper study of human action in economics was far too subtle to leave a~more lasting impression on economists. But, as the saying goes, a~house is built by wisdom and erected by understanding; fools tear it down with impatience. Having demonstrated the purity and soundness of its foundations, we can now take the study of human action in economics a~step further and address a~subject that must certainly be classified as essential in both theory and practice: the problem of competition.



Competition is both an ancient phenomenon and a~central concept in economics. With the increasing importance of welfare economics for policy advice, competition has acquired an increasingly important role as the main criterion for assessing the so-called efficiency of actual markets 
%\label{ref:RNDvbNc3Twdn2}(e.g., Motta, 2004; Armentano, 1972, p.31ff.).
\parencites[e.g.,][]{motta_competition_2004}[][p.31ff.]{armentano_myths_1972}. %
 This importance stands in stark contrast to the still inadequate understanding of the phenomenon and the insufficient understanding of the concept. It is true that the development of the theory of perfect competition, a~centrepiece of general equilibrium theory,\footnote{Thanks to an anonymous reviewer for pointing out that this is true only relative to the nature of your approach to equilibrium theory. Thus, while the starting point of Walras 1874–1877/1896---entirely in line with his conviction that ``economic theory is essentially the theory of the determination of prices in a~hypothetical regime of perfectly free competition'' 
%\label{ref:RNDWn1v8aNxVK}(Walras, 2019, VIII)
\parencite[][]{walras_lewalras_2019}%
---, and also the classic works 
%\label{ref:RNDd1ZqS9HHR1}(Edgeworth, 1881; Marshall, 1890; Arrow and Debreu, 1954; McKenzie, 1954)
\parencites[][]{edgeworth_mathematical_1881}[][]{marshall_principles_1890}[][]{arrow_existence_1954}[][]{mckenzie_equilibrium_1954} %
 do seem to make this essential connection, this is less obvious in the case of, say, 
%\label{ref:RNDAgpWJk7EqI}(Wald, 1935; Samuelson, 1947; Mas-Colell, 1974).
\parencites[][]{wald_uber_1935}[][]{samuelson_foundations_1947}[][]{mas-colell_equilibrium_1974}. %
 \textit{Cf.} also 
%\label{ref:RNDaO6Djas59W}(McKenzie, 1981; Weintraub, 2011).
\parencites[][]{mckenzie_classical_1981}[][]{weintraub_retrospectives_2011}. %
 Nevertheless, some importance must be attached to the fact that nowadays there still seems to be a~widespread belief ``that GE theory describes with sufficient approximation the result of the unfettered working of competitive markets'' 
%\label{ref:RNDflMWzgksHc}(Petri and Hahn, 2003, p.8).
\parencite[][p.8]{petri_general_2003}.%
} was hoped to improve understanding; and today's mainstream economic theory seems more or less satisfied on this issue. However, as we shall see in a~moment, there are even more serious difficulties with the equilibrium approach to competition, precisely because it aims to explain competition in terms of that perfectly realised market structure it describes.



This market structure is criticised even within mainstream economic theory 
%\label{ref:RND39XlTt3fQn}(cf., e.g., Ackerman and Nadal, 2004; Petri and Hahn, 2003).
\parencites[cf., e.g.,][]{ackerman_flawed_2004}[][]{petri_general_2003}. %
 Completely unimpressed, however, economics textbooks reiterate \textit{ad nauseam}, that it exists when (i) the number of suppliers is very large and (ii) the goods traded are homogeneous 
%\label{ref:RNDVRCibbzDF0}(see, e.g., Mankiw, 2020, p.62).
\parencite[see, e.g.,][p.62]{mankiw_principles_2020}. %
 As a~rule, the requirements are also added, at least implicitly, that in a~perfectly competitive market (iii) transaction costs or other obstacles to free and direct exchange and (iv) knowledge differences between market participants are negligible. These provisions are intended to ensure that under conditions of perfect competition sellers have no influence on market prices and thus take prices as given. Perfect competition, according to mainstream textbooks, ``is the world of price takers'' 
%\label{ref:RNDxIXBN3eF1z}(Samuelson and Nordhaus, 2009, p.150).
\parencite[][p.150]{samuelson_economics_2009}. %
 However, from a~more general point of view, since there is no clear distinction between buyers and sellers, there is no difference in principle between the case in which Dick trades his goat for Tom's sheep and the case in which he trades it for Tom's \$40. Consequently, we cannot say in principle who is the buyer and who is the seller apart from saying that both are both:



The buyer of a~thing is the seller of what he gives in exchange. The seller of a~thing is the buyer of what he receives in exchange for it. In other words, every exchange of two things, one for the other, is composed of a~double purchase and a~double sale. 
%\label{ref:RNDaHrTjQRQwA}(Walras, 2019, p.42 [orig. 1896])
\parencite[][p.42 [orig. 1896]]{walras_lewalras_2019}%




To simplify matters, what we can say is that both Dick and Tom are economic subjects, individual participants in the economy, or, if you will, \textit{traders}. So, the idea of a~world of price-takers has to be formulated more generally. What the perfect competition provisions are really meant to ensure is ``the fundamental competitive assumption that agents cannot influence market prices'' 
%\label{ref:RND7Y3KfXIxK6}(Safra, 1989, p.225; cf. Khan, 2008).
\parencites[][p.225]{safra_strategic_1989}[cf.][]{palgrave_macmillan_perfect_2008}. %
 The economist's basic perspective is thus to ensure that ``the influence of an individual participant on the economy […] be mathematically negligible'' 
%\label{ref:RNDSJdd6Ir68W}(Aumann, 1964, p.39).
\parencite[][p.39]{aumann_markets_1964}. %
 This can best be achieved, as Aumann has shown, by representing the ideal infinity of economic agents as a~single continuum. Since the circumstances in which individual economic agents are economically negligible are precisely the circumstances in which they are numerically negligible 
%\label{ref:RNDdiE0hhlLXI}(Bryant, 2010, p.332),
\parencite[][p.332]{bryant_general_2010}, %
 this formally amounts to the introduction of a~single entity, \textit{the all-trader}, as the single unit of economic exchange. The assumption that traded goods are homogeneous also serves a~similar function. It abstracts from the differences between goods, so it is about product differentiation. It is assumed that under perfect competition it makes no significant difference whether the traded goods are, for example, slightly heavier or smell slightly different: ``A perfectly competitive [trader] sells a~homogeneous product (one identical to the product sold by others in the industry)'' 
%\label{ref:RNDcc8FxLpUuj}(Samuelson and Nordhaus, 2009, p.150).
\parencite[][p.150]{samuelson_economics_2009}. %
 The homogeneity assumption on the side of the goods and the continuum assumption regarding traders are thus two sides of the same coin: both serve the purpose of mathematical integration. They are supported in this by the third stipulation that there are no transaction costs or other obstacles to free and immediate exchange. This ensures the uniqueness of the allocation. Thus, from a~\textit{logical} point of view (and this analysis is \textit{not} anticipated by economists) the following picture emerges: In perfect competition \textit{the all-trader is uniquely mapped onto the all-good}. The fact that the all-trader then also knows everything there is to know is only a~trivial consequence. The triviality of perfect knowledge. So now we are almost in a~position to understand what is deeply problematic about the equilibrium picture of perfect competition. It is not primarily what Friedman intended to defend, namely the lack of realism of the assumptions 
%\label{ref:RNDULGGYHxb8s}(Friedman, 1966),
\parencite[][]{friedman_methodology_1966}, %
 although the assumptions are of course very strong and highly unrealistic. Also, it is not what economists usually criticise from within economic theory 
%\label{ref:RND15H1gooK6T}(cf., e.g., Ackerman and Nadal, 2004; Petri and Hahn, 2003),
\parencites[cf., e.g.,][]{ackerman_flawed_2004}[][]{petri_general_2003}, %
 although these are often points of criticism that very much deserve attention. What really speaks against this picture is ultimately something else.



To see this more clearly, we first need to look at the standard response that is used to dismiss all inconsistencies that arise from the picture of perfect competition. Inconsistencies with real markets and real competition are usually answered by saying that perfect competition is only an ideal. For example, perfect competition is routinely compared to the idea of frictionless surfaces 
%\label{ref:RNDuiB9zgmW3r}(Samuelson, 1947; Friedman, 1966 [1953]; Aumann, 1964; Khan, 2008, \textit{etc.}).
\parencites[][]{samuelson_foundations_1947}[][]{friedman_methodology_1966}[][]{aumann_markets_1964}[][]{palgrave_macmillan_perfect_2008}. %
 The argument goes something like this: Frictionless surfaces cannot exist, but progress towards this ideal helps to reduce friction on real surfaces. This is what makes frictionless surfaces an ideal in the first place. In the case of perfect competition, unfortunately, the opposite is true. Here, every step towards perfection contributes to a~reduction in competition. Take (product) differentiation, for example. Decried in applied equilibrium theory as an unfair barrier to entry to the detriment of pure competition, in real life it is more a~function of consumer acceptance. In an effort to secure business, every supplier or producer will try to attract consumers to his product or service. He will strive to make his product or service as unique from the point of view of his potential customers as they will honour by buying it. As competition increases, we will therefore expect more rather than less differentiation. If need be, not in the product itself, but in the service, in the transaction costs or elsewhere in the economic sphere: ``In a~\textit{free} market individualism is to be expected on the part of the consumers and firms; the goods produced, therefore, will be differentiated to the extent and degree that consumers reward differentiation'' 
%\label{ref:RND7m63Rpm27u}(Armentano, 1972, p.33).
\parencite[][p.33]{armentano_myths_1972}. %
 Differentiation, \textit{i.e.}, making a~difference, is of the very essence of real competition. Remove this feature, abstract from all remaining differences, and what you are looking at is really something else. Seen in the light of day, then, the idea of perfect competition is not at all an ideal that enhances competition or that gives us a~better understanding of it, but quite the opposite. It is a~false, mock or anti-ideal. The pursuit of this ideal leads to a~gradual elimination of competition to the point where there is none at all. The idea of perfect competition thus tempts us to misunderstand the nature of competition. Instead, it paints an irretrievably distorted picture. Perhaps the most charitable thing to say would be that perfect competition is about perfection, not competition. A~perfection that is admittedly neither achievable nor desirable in the real world. A~perfection that is guaranteed by successive steps of logical abstraction. But that is precisely what has got us into trouble.



The logical analysis we have arrived at ultimately reveals the following: we are dealing with a~neat mathematical representation of a~quasi-Parmenidean idea of an almost all-encompassing monism: \textit{The all-trader is uniquely mapped onto the all-good.} No wonder there is neither change nor waste in such a~metaphysical picture. As a~result, there is Pareto optimality and even a~Nash equilibrium, great. But this is merely due to stipulation. A~nice little sleight of hand. And look what it costs: There is no competition either. That is why the immense intellectual effort invested in this idea has always led to resistance. What has not been taken into account, and what could instead help us to better understand competition, is the individual economic agent with all his subjective attitudes. It is to him that we must turn next.



\section{Competition as rivalry}

The idea of pure competition arose in an effort to understand more precisely the ultimate ground of truth of two very popular and plausible classical theses. One was \textit{Adam Smith's} assertion that the greater the number of sellers, the lower the price 
%\label{ref:RND2wfOaQk58A}(Smith, 1776, pp.68–69),
\parencite[][pp.68–69]{smith_inquiry_1776}, %
 the other was \textit{John Stuart Mill's} assumption that there can be only one price in the market 
%\label{ref:RNDncWBEiqSuh}(Mill, 1848, p.291).
\parencite[][p.291]{mill_principles_1848}. %
 The aim of the fathers of general equilibrium theory was to prove these assumptions truisms in a~mathematically convenient way. The imprecise understanding that economists sought to refine (and eventually inadvertently replaced) related to the behaviour of people: ``‘Competition' entered economics from common discourse, and for long it connoted only the independent rivalry of two or more persons'' 
%\label{ref:RNDIpkeBGOokO}(Stigler, 1957, p.1).
\parencite[][p.1]{stigler_perfect_1957}. %
 Today, when the economic mainstream understands competition almost exclusively in terms of perfect competition, the original understanding of competition as rivalry is nevertheless taken for granted. It is consistently implicit in mainstream textbooks 
%\label{ref:RNDfJqUvg9oJD}(cf. Acemoglu, Laibson and List, 2016, p.357; Pindyck and Rubinfeld, 2013, p.281 et passim; Samuelson and Nordhaus, 2009, 172f. Stiglitz and Walsh, 2006, p.241 et passim; among others).
\parencites[cf.][p.357]{acemoglu_microeconomics_2016}[][p.281 et passim]{pindyck_microeconomics_2013}[][p.241 et passim]{}[among others,][]{}. %
 Sometimes it is also stated very clearly: ``Competition exists when two or more firms are rivals for customers'' 
%\label{ref:RNDGhoH3v9Jju}(Mankiw and Taylor, 2014, p.42).
\parencite[][p.42]{mankiw_economics_2014}.%




Underlying all these characterisations is the concession that competition is essentially due to human behaviour. However, the concept of competition as rivalry is then usually explained from equilibrium theory and not the other way around. In contrast to the economic mainstream, the Austrian School of Economics has long recognised that this reverse order of explanation puts the cart before the horse. In his \textit{Rivalry and central planning}, Austrian economist \textit{Don Lavoie} argued that the information function of rivalry is fundamental to understanding the market process. ``Markets are inherently rivalrous, […] they work only as a~consequence of a~competitive struggle among incompatible plans'' 
%\label{ref:RNDHLp0gR5MoM}(Lavoie, 2015, p.180 [orig. 1985]).
\parencite[][p.180 [orig. 1985]]{lavoie_rivalry_2015}. %
 But like other Austrian approaches, Lavoie's account is full of strong assumptions and, more importantly, it does not provide us with an action-theoretic explanation either. Rather, we are offered an inherently economic explanation that invokes assumed ``market forces''. This explanation may or may not be plausible, but it is certainly not fundamental in the sense we are exploring in this paper. So how can we make sense of the idea that competition is essentially rivalry without introducing strong assumptions or economic presuppositions on our part? This is where the minimalist philosophy and the logic of action outlined in the first two sections will make the difference.



We will use (and have already been using) a~simplified, slightly extended variant of first-order predicate logic with logical connectives, variables and the usual quantifiers. Connectives are ``¬'', ``\&'', ``v'', ``\ding{213}'', and ``[27F7?]'', which correspond to their natural language equivalents ``not'', ``and'', ``or'', ``if ... then'', and ``if and only if … then …''. Standard single variables are ``\textit{x}'', ``\textit{y}'', ``\textit{z}'', \textit{etc.}, which can be replaced by proper names (or expressions of the same logical type) such as ``Tom'', ``Dick'' and ``Harry''. Standard variables that take a~predicate position are ``\textit{$\varphi $}'', ``\textit{$\psi $}'', ``\textit{$\chi $}'', \textit{etc.}, which can be replaced by predicates such as ``sleeps'', ``dropped out of high school'' and ``will join the military''. Standard propositional variables are ``\textit{p}'', ``\textit{q}'', ``\textit{r}'', \textit{etc.}, which can be replaced by full declarative sentences such as ``Tom will join the military'', ``Dick is asleep'' and ``Harry dropped out of high school''. The essential point about variables is that they can be bound, thus there are the quantifiers ``${\exists}$'' and ``${\forall}$'', the latter symbol often omitted, which correspond to their natural language equivalents ``at least one (is such that)'' and ``all (are such that)'', so that we can render formulae like ``(${\exists}$\textit{x}) (\textit{x} is asleep)'' as approximately ``Someone is asleep'' or ``(\textit{x}) (${\exists}$\textit{$\varphi $}) (\textit{$\varphi $x})'' as approximately ``Everyone is somehow'' or ``(\textit{p}) (Harry says that \textit{p} \ding{213} \textit{p})'' as approximately ``Everything is as Harry says''. The final step, already introduced in Section 2 above, is the addition of attitude operators ``B\textit{\textsubscript{x}}'', ``W\textit{\textsubscript{x}}'', ``F\textit{\textsubscript{x}}'' and ``H\textit{\textsubscript{x}}'', which correspond to their natural language equivalents ``\textit{x} believes that'', ``\textit{x} wants that'', ``\textit{x} fears that'', and ``\textit{x} hopes that'', so that we can reproduce formulae such as ``B\textit{\textsubscript{x}} \textit{r}'' which can be expanded to ``Tom believes that Harry dropped out of high school'', ``W\textit{\textsubscript{y}} \textit{q}'' to ``Harry wants Dick to sleep'', ``F\textit{\textsubscript{x}} \textit{p}'' to ``Tom fears that nothing is as Harry says'', and ``H\textit{\textsubscript{z}} \textit{r}'' to ``Harry hopes that someone will join the military''. So much for a~brief sketch of the apparatus involved.\footnote{Should readers miss an easily accessible introduction to logic at this point, I~refer them to the classic Lemmon 
%\label{ref:RNDlnC2fBZ9lD}(1965),
\parencite*[][]{lemmon_begining_1965}, %
 for example.}



The next step is to imagine a~simple exchange, such as Dick trading his goat for Tom's sheep. This involves at least the following:



(a) Tom gives Dick his sheep



(b) Dick gives Tom his goat



(c) Tom wants Dick to give him his goat



(d) Dick wants Tom to give him his sheep



(e) Tom thinks that if he gives Dick his sheep, Dick will give Tom his goat



(f) Dick thinks that if he gives Tom his goat, Tom will give him his sheep.



But that is not all. Tom gives Dick his sheep and Dick gives Tom his goat \textit{because} they want what they want and believe what they believe:



(TD*)(a) \& (b) because ((c) \& (d)) \& ((e)\&(f)).



So, we have a~case of intertwined, one could also say \textit{reciprocal}, action, for the above is nothing but a~notational variant for a~plural case of our familiar canonical form of action-reports (A):



(A) \textit{x} \textit{$\varphi $-s} because \textit{x} wants that \textit{p} \& \textit{x} believes that \textit{x~$\varphi $-s} \ding{213} \textit{p}



According to simple formal language described, Tom and Dick's exchange would have to be rendered more perspicuously as follows:



(TD) \textit{$\varphi $xy} \& \textit{$\psi $yx} because W\textit{\textsubscript{x}} \textit{$\psi $yx} \& W\textit{\textsubscript{y}} \textit{$\varphi $xy} \& B\textit{\textsubscript{x}} (\textit{$\varphi $xy} \ding{213} \textit{$\psi $yx}) \& B\textit{\textsubscript{y}} (\textit{$\psi $yx} \ding{213} \textit{$\varphi $xy})



with ``\textit{$\varphi $}'' = ``gives his sheep to'', ``\textit{x}'' = ``Tom'', ``\textit{y}'' = ``Dick'', ``\textit{$\psi $}'' = ``gives his goat to'', ``W\textit{\textsubscript{x}}'' = ``Tom wants that'', ``W\textit{\textsubscript{y}}'' = ``Dick wants that'', ``B\textit{\textsubscript{x}}'' = ``Tom believes that'', and ``B\textit{\textsubscript{y}}'' = ``Dick believes that''. I~will admit that this may look a~bit cryptic indeed. But remember that this is only applying the previously explained and innocuous stipulations. (TD) may be complex, but it is not complicated. Note also that (TD) is just an action-theoretic account of a~reciprocal doing, a~rendering of what sometimes is referred to by the Latin phrase \textit{do ut des}. There is nothing particularly economic about it, or to put it another way, an economic exchange would be nothing but a~special case of (TD).



Now the rivalry only comes into play when we add another participant to the scene. So, let's imagine a~different situation. Tom is still willing to trade with Dick, but now we are counting on another possible trader, Harry. Nothing has happened yet, but in this alternative situation it is conceivable that Tom will trade his sheep for Harry's llama. In strict analogy to (TD), but with suitable substitutions, this would yield (TH):



(TH) \textit{$\varphi $xz} \& \textit{$\psi $zx} because W\textit{\textsubscript{x}} \textit{$\psi $zx} \& W\textit{\textsubscript{z}} \textit{$\varphi $xz} \& B\textit{\textsubscript{x}} (\textit{$\varphi $xz} \ding{213} \textit{$\psi $zx}) \& B\textit{\textsubscript{z}} (\textit{$\psi $zx} \ding{213} \textit{$\varphi $xz})



In order to give an action-theoretic explanation of rivalry, we need to put these parts together in the right way. The essential step we need to add comes from the theory of intentionality: we need to take into account the attitudes Dick and Harry have towards the possibilities (TD) and (TH). This is what makes them rivals in the first place.



The realisation that the introduction of an intentional element is essential to explaining rivalry is almost a~truism. What causes two runners to be in a~race with each other is not that they are moving fast in the same direction. So many people do that every day. Rather, it is the fact that one wants to outdo the other. So, of course, they have to have a~certain attitude towards each other. This introduces an intentional, \textit{i.e.}, subjective, characteristic as an essential element. Since the role of human beings in general equilibrium theory is not really different from the role of ``atoms of the rare gas in my balloon'' 
%\label{ref:RND5O7KkbLpz1}(Samuelson, 1966, p.1411),
\parencite[][p.1411]{samuelson_modern_1966}, %
 we cannot be surprised that this essential element of competition must be absent from the equilibrium picture of perfect competition. However, with the help of the philosophy and logic of human action, it is not difficult to reinsert this element. The essential step is that Dick \textit{hopes} to make the deal but \textit{fears} that Harry might make it instead, and vice versa. This means that they see each other as rivals, and that if they act accordingly, they will be rivals. So, the next step is to establish that if and only if



(PR) H\textit{\textsubscript{y}} (TD) \& F\textit{\textsubscript{y}} (TH) \& H\textit{\textsubscript{z}} (TH) \& F\textit{\textsubscript{z}} (TD)



Dick and Harry \textit{perceive} each other as rivals. They \textit{are} rivals if and only if they act on this perception:



(AR) \textit{$\gamma $y} because W\textit{\textsubscript{y}} ((TD) \& ¬ (TH)) \& B\textit{\textsubscript{y}} \textit{$\gamma $y} \ding{213} ((TD) \& ¬ (TH))



\&



\textit{$\lambda $z} because W\textit{\textsubscript{z}} ((TH) \& ¬ (TD)) \& B\textit{\textsubscript{z}} \textit{$\lambda $z} \ding{213} ((TH) \& ¬ (TD))



where ``\textit{$\gamma $}'' and ``\textit{$\lambda $}'' are representative of what Dick and Harry do to outdo the other. What might that be? Well, Dick might offer Tom a~discount or some other perk, Harry might offer Tom special trade relations or immediate delivery. If this is what they do to secure the deal (and prevent the other from making it), this is their respective rivalrous behaviour. For each agent that involves an individual complex attitude, though. But through simple conjunction elimination in (PR) and (AR) we can uncover the subjective and individual perspective of the respective agent:



(PR\textit{\textsubscript{y}}) H\textit{\textsubscript{y}} (TD) \& F\textit{\textsubscript{y}} (TH)



and



(AR\textit{\textsubscript{y}}) \textit{$\gamma $y} because W\textit{\textsubscript{y}} ((TD) \& ¬ (TH))



such that now we can describe his rivalrous behaviour:



(R)\textit{y} acts \textit{rivalrously} [27F7?] (AR\textit{\textsubscript{y}}) because (PR\textit{\textsubscript{y}})



Rivalry, thus, is when an agent acts rivalrously because he perceives another to be a~rival. And, lo and behold, I~hear some people scoff and say that this is exactly what we had to hear from the philosophers. But anyone who reacts in this way misses an important, indeed crucial, point: in any serious scientific discussion, success is not measured by the conclusion you reach, but by \textit{the way you derive it}. This is precisely the reason why we talk about the scientific \textit{method}. Science without method is not science. It may well be that authors like \textit{Sebastian De Haro} are right and that the interaction between the empirical, natural and social sciences on the one hand and philosophy on the other is characterised by a~kind of ``love-hate relationship'' 
%\label{ref:RNDc0Q1y7MyqS}(De Haro, 2020).
\parencite[][]{de_haro_science_2020}. %
 Nevertheless, the ``analytical function of philosophy'' 
%\label{ref:RNDRLzch5O5FC}(De Haro, 2020, p.304f.)
\parencite[][p.304f.]{de_haro_science_2020} %
 is undeniable in any case. So let us not forget that there are good arguments with a~true conclusion and good arguments with a~false conclusion; there are bad arguments with a~true conclusion and bad arguments with a~false conclusion. Hence, it is not the truth or falsity of a~conclusion that determines whether an argument is good or bad. It must be something else. Philosophers would say: the plausibility of the premisses and the extent to which they lead to the conclusion. But as already mentioned, some economists, most likely under the spell of Friedmann's methodology 
%\label{ref:RNDpgwLITrysJ}(Friedman, 1966, 14f.),
\parencite[][]{},%
care little about the so-called ``reality of assumptions''. This only means, though, that they sometimes and to a~certain extent do not care whether they have a~good or a~bad argument in front of them. Ultimately, however, this cannot stand. And where the foundations of praxeology are at stake, we are well advised not to allow it to.



Let us therefore continue on our chosen path and see that the seemingly trivial (R) leads to our last step, explaining competition to be present when there is rivalrous behaviour, \textit{i.e.}, if and only if there is at least one acting rivalrously:



(C) Competition exists [27F7?] (${\exists}$\textit{x}) (\textit{x} acts rivalrously)



Again, this may be complex when expanded, but it is not complicated. More importantly, we can trace this understanding of competition back to its familiar origins in the theory of action and intentionality, \textit{i.e.}, Sections 2 and 3, so that we are left with nothing but the parsimonious and innocuous assumptions we made there and the assumptions that belong to our variant of first-order predicate logic, which are in any case essential to any reasonable argument.



\section{Conclusion and a~glimpse beyond}

It is sometimes said that ``Austrian economists understand competition better than most economists'' 
%\label{ref:RNDZ0RHWjuQFn}(Nell, 2010, p.142).
\parencite[][p.142]{nell_competition_2010}. %
 Perhaps this is so, but the fact remains that Austrian economists have not traced their understanding of subjectivity to its origin, the theory of intentionality, nor have they traced their sympathy for agent-based modelling of market processes to its foundation, the theory of action. So, they struggled to establish what makes their contribution to economic theory so unique: the philosophy and logic of human action. Looking back at our explanation of competition as rivalry, one might be tempted to say that the conclusion we reached is hardly surprising. And it is true, I~never meant to doubt that economists were aware of the \textit{truth} of this conclusion.\footnote{Remember, however, the lesson from the previous section above.} But what some did not know, or others could not trace back to its root cause, was that there was no need or place in this understanding for anything remotely resembling an equilibrium picture of perfect competition. (C) even makes it clear that explaining competition as rivalry cannot be done within the framework of its market structure approach. Its market structure leaves out what is essential, namely the individual with his subjective attitudes. The explanation of competition as rivalry, on the other hand, avoids the pitfalls of the equilibrium picture. It can also give us a~good idea of what the study of human action can contribute to the study of social sciences in general and economics in particular.



So, what else can the action-theoretic approach contribute to economic theory besides a~solid foundation? For reasons of space, I~can only give an outline here:



\begin{enumerate}

\item \textit{The Coase presumption:} Competition without competitors. \textit{Ronald Coase} 
%\label{ref:RNDKZ0g3nS4yx}(1972)
\parencite*[][]{coase_durability_1972} %
 famously posited that even a~monopolist can only charge competitive prices in the long run. This conjecture helped to explain real phenomena, \textit{e.g.}, why OPEC did not arbitrarily raise oil prices even when it had a~(near) monopoly. Our approach can explain these results without making extravagant assumptions (such as Coase's assumption of competition with a~future self). According to (C), it is sufficient for competitive behaviour that an agent perceives someone as a~rival and acts accordingly. This perception may be erroneous. It may merely be an anticipation of possible future behaviour. Since, in our view, the rival is merely the intentional object of the agent's attitudes, he may or may not be as the agent imagines him, he may even not exist at all (see Section 2);

\item \textit{Risk, Uncertainty and Profit.} In action theory it is a~commonplace that an agent neither strives for what he (really) believes to be impossible, nor for what he (really) believes to be already achieved. Motivation can therefore only be located in the realm of the uncertain. But it is only where the agent acts that the meaning of all competitive behaviour, namely profit, can lie. Thus, we can underline a~result advocated by \textit{Frank Knight} 
%\label{ref:RNDyHIG4LqLPb}(1921),
\parencite*[][]{knight_risk_1921}, %
 and we need only resort to insights gained with the help of the philosophy and logic of human action;

\item \textit{Market failure and antitrust.} Competition does not presuppose the existence of any kind of equilibrium. On the contrary, if there were such an equilibrium, there would be no competition. Consequently, there is also no market failure that manifests itself in competitive behaviour such as (product) differentiation, mergers and acquisitions. This undermines the conceptual basis of most antitrust laws 
%\label{ref:RNDpr6wCvCelM}(cf. Armentano, 1972).
\parencite[cf.][]{armentano_myths_1972}. %
 What drives competition is intentional and therefore subjective: it is the fear of losing business and the hope of somehow still getting it. On action-theoretic grounds, then, it is difficult to find any justification at all for state intervention into the market.

\end{enumerate}

As I~said earlier, Mises held that economic science is based on action theory. This was a~claim that many found too disturbing to defend. He also believed that the theory of human action was ultimately grounded in epistemology, and in his last book he even referred to epistemology as the very foundation of economic science 
%\label{ref:RNDLSWzhOnvon}(Mises, 1962).
\parencite[][]{mises_ultimate_1962}. %
 On this latter point, Mises was mistaken. There is nothing epistemological about action theory or the theory of intentionality. We have proven this by omission. It is more important, however, that we found considerable support for Mises' former point. What has been shown here is evidence for something closely akin to Mises' original claim: The basis of economic science is analytic action theory. To make this very clear: The point here is not to accuse Mises of not having seriously attempted to ground economics in the theory of action. Mises did this like no other. And with considerable success. However, Mises was arguably the only Austrian who was really prepared to go beyond the confines of economic theory---which you have to do if you want to anchor it in another discipline. In this respect, support for Mises within the Austrian community was half-hearted at best. And it did not help that (old) Mises turned on his \textit{alter ego} and endorsed the mistaken claim about epistemology, which so many have repeated ever since. But this unforced error can be corrected, and in part this is what the present paper has done. Thus, as has been suggested elsewhere before 
%\label{ref:RND82RhAudMiD}(cf. Oliva Córdoba, 2017),
\parencite[cf.][]{oliva_cordoba_uneasiness_2017}, %
 praxeology can be well aligned with analytic action theory, retaining the spirit but not the letter of Mises' original approach. The prospects for an integrated approach to Austrian theory as a~fusion of Austrian economics and analytical action theory thus seem good. But even if Austrian economists were to abstain, we should not overlook the fact that in the course of this enquiry we have never had to compromise the rigour, richness and soundness of analytical action theory and the theory of propositional attitudes. If these are decent, well-established and worthwhile fields of study, then recourse to them has most likely added to, rather than detracted from, economic theory. And if this way of studying human action has made a~valuable contribution to explaining competition, it shows not only that the philosophy and logic of human action is useful in the social sciences, but also that it is, or should be, central to economic theory.



\section{Acknowledgments}

Versions of this paper were presented at the one-day conference ``Perspectives of Integrated Austrian Theory'' at the University of Hamburg on Wednesday, 4 October 2017, and at the Libertarian Scholar''s Conference at King's College, New York, on 20 October 2018. I~am indebted to the audiences of these lectures, especially \textit{Joseph T. Salerno} in New York and my commentator \textit{Stefan Kooths} in Hamburg, for helpful comments. Thanks are also due to my students and the participants in \textit{Rolf W. Puster's} research colloquium at the University of Hamburg, which I~have had the honour of co-hosting for twelve years. I~would also like to thank an anonymous reviewer for his valuable suggestions. My special thanks go to \textit{Randall G. Holcombe} and \textit{Rolf W. Puster}, who kindly read earlier versions of the manuscript and improved it in many respects. Any remaining errors are, of course, mine alone.



\section{References}

Acemoglu, D., Laibson, D.I. and List, J.A., 2016. \textit{Microeconomics}. Pearson series in economics. Boston: Pearson.



Ackerman, F. and Nadal, A. eds., 2004. \textit{The Flawed Foundations of General Equilibrium: Critical Essays on Economic Theory}. Routledge frontiers of political economy. London: Routledge.



Ajdukiewicz, K., 1935. Die syntaktische Konnexität. \textit{Studia Philosophica}, 1, pp.1–27.



Anscombe, G.E.M., 1957. \textit{Intention}. 1\textsuperscript{st} ed. Cambridge, MA: Harvard University Press.



Aristotle, 1938. \textit{Categories. On Interpretation. Prior Analytics}. [online] Translated by H.P. Cooke and H. Tredennick Cambridge, MA: Harvard University Press. Available at: {\textless}https://www.loebclassics.com/view/LCL325/1938/volume.xml{\textgreater} [Accessed 14 October 2024].



Armentano, D.T., 1972. \textit{The Myths of Antitrust: Economic Theory and Legal Cases}. New Rochelle, N.Y: Arlington House.



Arrow, K.J. and Debreu, G., 1954. Existence of an Equilibrium for a~Competitive Economy. \textit{Econometrica}, [online] 22(3), pp.265–290. https://doi.org/10.2307/1907353.



Aumann, R.J., 1964. Markets with a~Continuum of Traders. \textit{Econometrica}, [online] 32(1/2), pp.39–50. https://doi.org/10.2307/1913732.



Barrotta, P., 1996. A~Neo-Kantian Critique of Von Mises's Epistemology. \textit{Economics \& Philosophy}, [online] 12(1), pp.51–66. https://doi.org/10.1017/S0266267100003710.



Boettke, P.J., 1990. \textit{The Political Economy of Soviet Socialism: The Formative Years, 1918–1928}. [online] Dordrecht: Springer Netherlands. https://doi.org/10.1007/978-94-017-3433-2.



Brentano, F., 1874. \textit{Psychologie vom empirischen Standpunkte}. [online] Leipzig: Verlag von Duncker \& Humblot. Available at: {\textless}https://archive.org/details/psychologievome02brengoog{\textgreater}.



Brentano, F., 2009. \textit{Psychology from an Empirical Standpoint}. International Library of Philosophy. Translated by A.C. Rancurello, D.B. Terrell and L.L. McAlister Taylor \& Francis.



Bryant, W.D.A. ed., 2010. \textit{General Equilibrium: Theory and Evidence}. Singapore; Hackensack, NJ: World Scientific Pub. Co.



Burge, T., 2010. \textit{Origins of Objectivity}. Oxford: Oxford University Press. https://doi.org/10.1093/acprof:oso/9780199581405.001.0001.



Burton-Roberts, N., 2016. \textit{Analysing Sentences: An Introduction to English Syntax}. 4\textsuperscript{th} ed. Learning about Language. London; New York: Routledge.



Carnap, R., 1963. Intellectual Autobiography. In: \textit{The Philosophy of Rudolf Carnap}, The Library of living philosophers. La Salle, IL: Open Court. pp.3–84.



Coase, R.H., 1972. Durability and Monopoly. \textit{The Journal of Law and Economics}, [online] 15(1), pp.143–149. https://doi.org/10.1086/466731.



Condic, S.B. and Morefield, R., 2021. Hayek on the essential dispersion of market knowledge. \textit{The Review of Austrian Economics}, [online] 34(4), pp.449–463. https://doi.org/10.1007/s11138-019-00487-4.



Crane, T., 1998. Intentionality as the Mark of the Mental. In: A. O'Hear, ed. \textit{Contemporary Issues in the Philosophy of Mind}, Royal Institute of Philosophy Supplements. [online] Cambridge: Cambridge University Press. pp.229–252. https://doi.org/10.1017/CBO9780511563744.013.



Crane, T., 2001. \textit{Elements of Mind: An Introduction to the Philosophy of Mind}. Oxford, New York: Oxford University Press.



Crane, T., 2013. \textit{The Objects of Thought}. Oxford, New York: Oxford University Press.



Davidson, D., 1963. Actions, Reasons, and Causes. \textit{The Journal of Philosophy}, [online] 60(23), pp.685–700. https://doi.org/10.2307/2023177.



Davidson, D., 2001. How is weakness of the Will possible? [1970]. In: \textit{Essays on Actions and Events}, Philosophical Essays of Donald Davidson, 2\textsuperscript{nd} ed. Oxford, New York: Oxford University Press. pp.21–42.



De Haro, S., 2020. Science and Philosophy: A~Love–Hate Relationship. \textit{Foundations of Science}, [online] 25(2), pp.297–314. https://doi.org/10.1007/s10699-019-09619-2.



Ditmarsch, H. van, Halpern, J.Y., Hoek, W. van der and Kooi, B.P. eds., 2015. \textit{Handbook of Epistemic Logic}. London: College publications.



Ebeling, R.M., 1993. Economic Calculation Under Socialism: Ludwig von Mises and His Predecessors. In: J.M. Herbener, ed. \textit{The Meaning of Ludwig von Mises}. [online] Dordrecht: Kluwer Academic Publishers. pp.56–101. https://doi.org/10.1007/978-94-011-2176-7\_3.



Edgeworth, F., Y., 1881. \textit{Mathematical Psychics: An Essay on the Application of Mathematics to the Moral Sciences}. [online] London: Kegan Paul. Available at: {\textless}http://historyofeconomicthought.mcmaster.ca/edgeworth/mathpsychics.pdf{\textgreater} [Accessed 2 September 2019].



Flanigan, J., 2016. Obstetric Autonomy and Informed Consent. \textit{Ethical Theory and Moral Practice}, [online] 19(1), pp.225–244. https://doi.org/10.1007/s10677-015-9610-8.



Frankfurt, H.G., 1978. The Problem of Action. \textit{American Philosophical Quarterly}, [online] 15(2), pp.157–162. Available at: {\textless}https://www.jstor.org/stable/20009708{\textgreater} [Accessed 14 October 2024].



Frege, G., 1891. \textit{Function und Begriff: Vortrag, gehalten in der Sitzung vom 9. Januar 1891 der Jenaischen Gesellschaft für Medicin und Naturwissenschaft}. [online] Jena: Pohle. Available at: {\textless}https://gdz.sub.uni-goettingen.de/id/PPN64299370X{\textgreater} [Accessed 16 October 2024].



Frege, G., 1953. \textit{The Foundations of Arithmetic: A~Logico-Mathematical Enquiry into the Concept of Number}. Rev. ed. New York: Harper and Brothers.



Frege, G., 1960. Function and concept. Translated by P.T. Geach and M. Black. In: \textit{Translations from the Philosophical Writings}, 2\textsuperscript{nd} ed. Oxford: Basil Blackwell. pp.21–41.



Friedman, M., 1966. The Methodology of Positive Economics. In: \textit{Essays in Positive Economics}. Chicago; London: Chicago University Press. pp.3–43.



Fumerton, R., 2017. Epistemology and Science: Some Metaphilosophical Reflections. \textit{Philosophical Topics}, [online] 45(1), pp.1–16. Available at: {\textless}https://www.jstor.org/stable/26529422{\textgreater} [Accessed 15 October 2024].



Hayek, F.A., 1948. Economics and Knowledge. In: \textit{Individualism and Economic Order}. London: Routledge. pp.33–56.



Hayek, F.A. von, 1945. The Use of Knowledge in Society. \textit{The American Economic Review}, [online] 35(4), pp.519–530. Available at: {\textless}https://www.kysq.org/docs/Hayek\_45.pdf{\textgreater} [Accessed 20 September 2024].



Hintikka, J., 1962. \textit{Knowledge and Belief: An Introduction to the Logic of the Two Notions}. Contemporary philosophy. Ithaca, NY: Cornell University Press.



Husserl, E., 1913. \textit{Prolegomena zur reinen Logik}. Halle: Max Niemeyer.



Jennings, B., 2009. Agency and Moral Relationship in Dementia. \textit{Metaphilosophy}, [online] 40(3–4), pp.425–437. https://doi.org/10.1111/j.1467-9973.2009.01591.x.



Khan, M.A., 2008. Perfect Competition. In: Palgrave Macmillan, ed. \textit{The New Palgrave Dictionary of Economics}. [online] London: Palgrave Macmillan UK. pp.1–15. https://doi.org/10.1057/978-1-349-95121-5\_1633-2.



Kirzner, I.M., 1995. The Subjectivism of Austrian Economics. In: G. Meijer, ed. \textit{New Perspectives on Austrian Economics}. London: Routledge. pp.11–22.



Kirzner, I.M., 2016. \textit{The History and Importance of the Austrian Theory of the Market Process}. [online] 2016 Advanced Austrian Seminar, Mercatus Center Academic \& Student Programs. Available at: {\textless}https://www.youtube.com/watch?v=GvE4zEfrv0k{\textgreater} [Accessed 15 October 2024].



Knight, F.H., 1921. \textit{Risk, Uncertainty and Profit}. Boston; New York: Houghton Mifflin Company.



Knudsen, C., 2004. Alfred schutz, Austrian Economists and the Knowledge Problem. \textit{Rationality and Society}, [online] 16(1), pp.45–89. https://doi.org/10.1177/1043463104036622.



Kurrild-Klitgaard, P., 2001. On Rationality, Ideal Types and Economics: Alfred Schüutz and the Austrian School. \textit{The Review of Austrian Economics}, [online] 14(2), pp.119–143. https://doi.org/10.1023/A:1011199831428.



Lachmann, L.M., 1982. Ludwig von Mises and the Extension of Subjectivism. In: I.M. Kirzner, ed. \textit{Method, Process, and Austrian Economics: Essays in Honor of Ludwig Von Mises}. Lexington, MA: Lexington Books. pp.31–40.



Lavoie, D., 2015. \textit{Rivalry and Central Planning: The Socialist Calculation Debate Reconsidered}. Advanced studies in political economy. Arlington, Virginia: Mercatus Center, George Mason University.



Lemmon, E.J., 1965. \textit{Begining Logic}. London: Thomas Nelson and Sons Limited.



Lewis, D., 1970. General semantics. \textit{Synthese}, [online] 22(1), pp.18–67. https://doi.org/10.1007/BF00413598.



Lyons, J., 1968. \textit{Introduction to Theoretical Linguistics}. 1\textsuperscript{st} ed. [online] Cambridge: Cambridge University Press. https://doi.org/10.1017/CBO9781139165570.



Mankiw, N.G., 2020. \textit{Principles of Economics}. 9\textsuperscript{th} ed. Boston, MA: Cengage Learning, Inc.



Mankiw, N.G. and Taylor, M.P., 2014. \textit{Economics}. 3\textsuperscript{rd} ed. Andover: Cengage Learning.



Marshall, A., 1890. \textit{Principles of Economics}. London; New York: Macmillan and Company.



Mas-Colell, A., 1974. An equilibrium existence theorem without complete or transitive preferences. \textit{Journal of Mathematical Economics}, [online] 1(3), pp.237–246. https://doi.org/10.1016/0304-4068(74)90015-9.



McKenzie, L.W., 1954. On Equilibrium in Graham's Model of World Trade and Other Competitive Systems. \textit{Econometrica}, 22(2), pp.147–161. https://doi.org/10.2307/1907539.



McKenzie, L.W., 1981. The Classical Theorem on Existence of Competitive Equilibrium. \textit{Econometrica}, [online] 49(4), pp.819–841. https://doi.org/10.2307/1912505.



Mele, A., 2010. Weakness of will and akrasia. \textit{Philosophical Studies}, [online] 150(3), pp.391–404. https://doi.org/10.1007/s11098-009-9418-2.



Mill, J.S., 1848. \textit{Principles of Political Economy with Some of Their Applications to Social Philosophy}. Boston: Charles C. Little \& John Brown.



Minio-Paluello, L. ed., 1961. \textit{Aristoteles latinus. I:1-5 Categoriae vel Praedicamenta}. Corpus philosophorum medii aevi. Leiden: Brill.



Mises, L. von, 1962. \textit{The Ultimate of Foundation of Economic Science: An Essay on Method}. William Vilker Fund Series in the Humane Studies. Princeton, NJ: Van Nostrand Comp.



Mises, L. von, 1998. \textit{Human Action: A~Treatise on Economics}. Scholar's ed. ed. Auburn AL: Ludwig von Mises Institute.



Motta, M., 2004. \textit{Competition Policy: Theory and Practice}. [online] Cambridge: Cambridge University Press. https://doi.org/10.1017/CBO9780511804038.



Nagel, T., 1979. Subjective and Objective. In: \textit{Mortal Questions}. Cambridge: Cambridge University Press. pp.196–213.



Nell, G.L., 2010. Competition as market progress: An Austrian rationale for agent-based modeling. \textit{The Review of Austrian Economics}, [online] 23(2), pp.127–145. https://doi.org/10.1007/s11138-009-0088-2.



O'Brien, E. by L. and Soteriou, M. eds., 2009. \textit{Mental Actions}. Oxford, New York: Oxford University Press.



Oliva Córdoba, M., 2017. Uneasiness and Scarcity: An Analytic Approach Towards Ludwig von Mises's Praxeology. \textit{Axiomathes}, [online] 27(5), pp.521–529. https://doi.org/10.1007/s10516-017-9352-4.



Petri, F. and Hahn, F. eds., 2003. \textit{General Equilibrium: Problems and Prospects}. Routledge Siena studies in political economy. London: Routledge.



Pindyck, R.S. and Rubinfeld, D.L., 2013. \textit{Microeconomics}. 8\textsuperscript{th} ed. Boston: Pearson.



Plato, 1921. \textit{Plato in Twelve Volumes: Vol. 12: Theaetetus}. Loeb classical library. Translated by H.N. Fowler London; Cambridge: W. Heinemann; Harvard University Press.



Prendergast, C., 1986. Alfred Schutz and the Austrian School of Economics. \textit{American Journal of Sociology}, [online] 92(1), pp.1–26. https://doi.org/10.1086/228461.



Rajagopalan, S. and Rizzo, M.J., 2019. Austrian Perspectives in Law and Economics. In: A. Marciano and G.B. Ramello, eds. \textit{Encyclopedia of Law and Economics}. [online] New York, NY: Springer. pp.92–99. https://doi.org/10.1007/978-1-4614-7753-2\_621.



Richardson, S.S., 2009. The Left Vienna Circle, Part 1. Carnap, Neurath, and the Left Vienna Circle thesis. \textit{Studies in History and Philosophy of Science Part A}, [online] 40(1), pp.14–24. https://doi.org/10.1016/j.shpsa.2008.12.002.



Safra, Z., 1989. Strategic Reallocation of Endowments. In: J. Eatwell, M. Milgate and P. Newman, eds. \textit{Game Theory}. [online] London: Palgrave Macmillan UK. pp.225–231. https://doi.org/10.1007/978-1-349-20181-5\_27.



Samuelson, P.A., 1947. \textit{Foundations of Economic Analysis}. Harvard economic studies. Cambridge: Harvard University Press.



Samuelson, P.A., 1964. Theory and Realism: A~Reply. \textit{The American Economic Review}, [online] 54(5), pp.736–739. Available at: {\textless}https://www.jstor.org/stable/1818572{\textgreater} [Accessed 15 October 2024].



Samuelson, P.A., 1966. Modern Economic Realities and Individualism. In: J.E. Stiglitz, ed. \textit{The Collected Scientific Papers of Paul a. Samuelson. Vol. 2}. Cambridge: MIT Press. pp.1407–1418.



Samuelson, P.A. and Nordhaus, W.D., 2009. \textit{Economics}. 19. ed ed. The McGraw-Hill series economics. Boston, MA: McGraw-Hill.



Schütz, A., 1953. Common-Sense and Scientific Interpretation of Human Action. \textit{Philosophy and Phenomenological Research}, [online] 14(1), pp.1–38. https://doi.org/10.2307/2104013.



Schütz, A., 1996. Political Economy: Human Conduct in Social Life. In: \textit{Collected Papers: Volume IV}. [online] Dordrecht: Springer Netherlands. pp.93–105. https://doi.org/10.1007/978-94-017-1077-0\_10.



Simons, P., 2009. Introduction to the second edition. Translated by A.C. Rancurello, D.B. Terrell and L.L. McAlister. In: O. Kraus and L.L. McAlister, eds. \textit{Psychology from an Empirical Standpoint}, International Library of Philosophy. Taylor \& Francis. pp.xiii–xx.



Smith, A., 1776. \textit{An Inquiry into the Nature and Causes of the Wealth of Nations. Vol. 1}. [online] London: W. Strahan \& T. Cadell. Available at: {\textless}https://books.google.pl/books?id=jRNDAAAAcAAJ{\textgreater} [Accessed 16 October 2024].



Stigler, G.J., 1957. Perfect Competition, Historically Contemplated. \textit{Journal of Political Economy}, [online] 65(1), pp.1–17. https://doi.org/10.1086/257878.



Stiglitz, J.E. and Walsh, C.E., 2006. \textit{Economics}. 4\textsuperscript{th} ed. New York: W.W. Norton.



Storr, V.H., 2019. Ludwig Lachmann's peculiar status within Austrian economics. \textit{The Review of Austrian Economics}, [online] 32(1), pp.63–75. https://doi.org/10.1007/s11138-017-0403-2.



Tallerman, M., 2015. \textit{Understanding Syntax}. 4\textsuperscript{th} ed. London: Routledge. https://doi.org/10.4324/9781315758084.



Wald, A., 1935. Über die eindeutige positive Lösbarkeit der neuen Productionsgleichungen. \textit{Ergebnisse eines Mathematischen Kolloquiums}, 6, pp.12–20.



Walker, A.F., 1989. The Problem of Weakness of Will. \textit{Noûs}, [online] 23(5), pp.653–676. https://doi.org/10.2307/2216006.



Walras, L., 2019. \textit{Léon Walras, Elements of Theoretical Economics: Or the Theory of Social Wealth}. Cambridge: Cambridge University Press.



Weintraub, E.R., 2011. Retrospectives: Lionel W. McKenzie and the Proof of the Existence of a~Competitive Equilibrium. \textit{Journal of Economic Perspectives}, [online] 25(2), pp.199–215. https://doi.org/10.1257/jep.25.2.199.



Wittgenstein, L., 1922. \textit{Tractatus logico-philosophicus}. International library of psychology, philosophy and scientific method. London: Kegan Paul, Trench, Trubner.



Wittgenstein, L., 2013. \textit{Tractatus Logico-Philosophicus}. Routledge great minds. Translated by D. Pears and B. McGuinness London: Routledge.



Wright, G.H. von, 1971. \textit{Explanation and Understanding}. Contemporary philosophy. Ithaca, NY: Cornell University Press.



Yeager, L.B., 1994. Mises and Hayek on calculation and knowledge. \textit{The Review of Austrian Economics}, [online] 7(2), pp.93–109. https://doi.org/10.1007/BF01101944.



Zambrano, A., 2017. Patient Autonomy and the Family Veto Problem in Organ Procurement. \textit{Social Theory and Practice}, [online] 43(1), pp.180–200. Available at: {\textless}https://www.jstor.org/stable/24871373{\textgreater} [Accessed 15 October 2024].

\end{document}

