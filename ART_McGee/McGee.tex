\setcounter{secnumdepth}{2}



\title{Szablon-EN}

\begin{document}

Taxation and the Philosophy of Frédéric Bastiat





Robert W. McGee





Fayetteville State University





Fayetteville, NC 28301 USA





ABSTRACT



Frederic Bastiat (1801-1850) was an economist and journalist. A~member of the French Liberal School, he is best known for his free trade ideas and his philosophy of law. Mark Blaug ranks him as one of the 100 greatest economists before Keynes. Schumpeter called him a~brilliant economic journalist. Haney devoted a~chapter of his History of Economic Thought to Bastiat.



Although Bastiat is known for his work on free trade and the philosophy of law, he also wrote on other topics. To date, almost no one has examined his views on taxation. The purpose of this paper is to fill that gap in the literature.



Key Words: Bastiat; taxation; utilitarian; rights theory; public finance; French Liberalism



\section{Introduction}

Frédéric Bastiat (1801-1850) was born in France and spent most of his life there, although he traveled to England to visit with Cobden and Bright and fully supported their free trade movement. Much of their correspondence was later published as an entire volume of his seven-volume \textit{Oeuvres Complètes} (1864). He died in Rome on December 24, 1850 of tuberculosis.



Bastiat wore many hats. He was an economic journalist and philosopher of law. He was a~gentleman farmer. He was a~justice of the peace and later served in the French Chamber of Deputies in Paris. He was a~husband and father 
%\label{ref:RNDgF5xD6mtga}(Bidet, 1906; Bastiat, 1889; Haney, 1949; Imbert, 1913; Nouvion, 1905; Roche, 1971; 1993; Ronce, 1905; Russell, 1959; 1969).
\parencites[][]{bidet_frederic_1906}[][]{bastiat_f_1889}[][]{haney_history_1949}[][]{imbert_frederic_1913}[][]{}[][]{roche_frederic_1971}[][]{roche_free_1993}[][]{ronce_frederic_1905}[][]{russell_frederic_1959}[][]{russell_frederic_1969}. %
 For six years, starting at age seventeen, he worked in his uncle's counting house, which exposed him to accounting 
%\label{ref:RND1ayKh01aSu}(Hazlitt, 1964, p.xi).
\parencite[][p.xi]{bastiat_introduction_1964}.%




Richman 
%\label{ref:RNDXkS7VtnW7W}(1998, p.ix)
\parencite*[][p.ix]{bastiat_foreward_1998} %
 stated that he was a~legal philosopher of the first rank. Skousen 
%\label{ref:RNDjsccfn9meM}(2001, p.59)
\parencite*[][p.59]{skousen_making_2001} %
 compared him to Benjamin Franklin or Voltaire for his integrity and the purity and elegance of his writing style. Hébert 
%\label{ref:RNDnEQ1zpcgQh}(1987, p.205)
\parencite*[][p.205]{} %
 considers him to be unrivaled in the way he exposed fallacies 
%\label{ref:RNDakURdQpQoB}(Skousen, 2001, p.59).
\parencite[][p.59]{skousen_making_2001}. %
 Schumpeter 
%\label{ref:RNDTZENZLsD9f}(1954, p.500)
\parencite*[][p.500]{schumpeter_history_1954} %
 called him one of the most brilliant economic journalists who ever lived, although he did not consider him to be a~first-rate theorist. Blaug 
%\label{ref:RNDBOLdGCRMkR}(1986)
\parencite*[][]{blaug_great_1986} %
 ranks Bastiat as one of the 100 greatest economists before Keynes. According to some historians of economic thought, political economy reached its apogée with Bastiat. ``After Bastiat, Reybaud could state that work in political economy had almost been exhausted and that there was nothing else to discover.'' 
%\label{ref:RND31otxCAyvR}(Screpanti and Zamagni, 1993, p.2).
\parencite[][p.2]{screpanti_outline_1993}. %
 That turned out not to be true, of course, but that was the thinking at the time.



Although classified as a~member of the French Liberal School or Optimist School 
%\label{ref:RNDOb4ofnxmaQ}(Cossa, 1893, pp.376–382; Gide and Rist, 1948, pp.329–354),
\parencites[][pp.376–382]{cossa_introduction_1893}[][pp.329–354]{gide_history_1948}, %
 he is also considered to be a~forerunner of the Austrian School of Economics 
%\label{ref:RNDGILdpqE6fv}(DiLorenzo, 1999)
\parencite[][]{holcombe_frederic_1999} %
 because of the similarity of his methodology to theirs. Some of his essays applied the theory of opportunity cost, which was unusual at the time (1840s), since the theory of opportunity costs was not fully developed until Carl Menger, the founder of the Austrian School of Economics 
%\label{ref:RND7VqZ5QNcJ3}(Menger, 1871).
\parencite[][]{menger_grundsatze_1871}.%




Although the concept of opportunity cost is attributed to the Austrian School of Economics 
%\label{ref:RNDjjr8PxKcAo}(Buchanan, 1973, p.14; Haney, 1949, p.895; Schumpeter, 1954, p.917),
\parencites[][p.14]{}[][p.895]{haney_history_1949}[][p.917]{schumpeter_history_1954}, %
 its origins can be traced back to Cantillon's \textit{Essai sur la Nature du Commerce en Général} 
%\label{ref:RNDOG85niwr8C}(1755)
\parencite*[][]{} %
 as well as the work of Turgot 
%\label{ref:RNDY6fKBB5xss}(Rothbard, 1995, p.391; 1999, pp.34, 40)
\parencites[][p.391]{rothbard_economic_1995}[][pp.34]{} %
 and Bastiat 
%\label{ref:RND0FWA8W2Ggo}(DiLorenzo, 1999, pp.62–63),
\parencite[][pp.62–63]{holcombe_frederic_1999}, %
 all of whom were French economists.



The classic exposition of Bastiat's application of opportunity cost is in his essay, \textit{What Is Seen and What Is Not Seen} 
%\label{ref:RND7O08FnOYll}(Bastiat, 1850; 1964c, pp.1–50; 2007, U: 1-48).
\parencites[][]{bastiat_ce_1850}[][pp.1–50]{bastiat_selected_1964}[][]{}. %
 In this essay, Bastiat applies the theory of opportunity cost to a~number of issues, including destruction of property, military demobilization, taxes, theater and the fine arts, public works, middlemen, restraints on trade, machinery, credit and several other topics.



Opportunity cost might be defined as ``the sacrifice of the utility of those other things which we could have had from the resources that went into the one we did produce.'' 
%\label{ref:RNDYbz6nKKUhn}(Schumpeter, 1954, p.917).
\parencite[][p.917]{schumpeter_history_1954}. %
 Stated more simply, ``Opportunity cost is income of a~foregone opportunity.'' 
%\label{ref:RND1sysiFarAm}(Magni, 2009).
\parencite[][]{magni_splitting_2009}. %
 Friedrich von Wieser (1851-1926) is credited with inventing the term 
%\label{ref:RNDTXqHiZZ4HX}(Skousen, 2001, p.184),
\parencite[][p.184]{skousen_making_2001}, %
 although Bastiat actually applied the concept before von Wieser was born.



The importance of applying opportunity cost to public policy issues cannot be overstated. Much public policy debate ignores the issue of opportunity cost. Economists, politicians and the media almost uniformly ignore some affected groups when they try to determine public policy positions. Bastiat's methodology makes a~serious effort to include all affected groups. Thus, his essay, \textit{What Is Seen and What Is Not Seen}, remains an important, if neglected, piece of literature.



He was a~vehement opponent of protectionism and socialism and much of his writing attacked one or the other. His book, \textit{The Law} 
%\label{ref:RND7KqnjDq9J4}(1998)
\parencite*[][]{} %
 is required reading in some Tea Party circles 
%\label{ref:RNDj6ymMN9gqr}(Zernike, 2010).
\parencite[][]{zernike_shaping_2010}.%




He equated socialism with a~government that goes beyond its role of protecting life, liberty and property and ventures into the realm of redistribution. He debated the socialists of his time, most notably Proudhon, with whom he exchanged a~series of letters 
%\label{ref:RNDLvnnxp63dC}(Bastiat, 1873b).
\parencite[][]{bastiat_sophismes_1873}. %
 Unfortunately, that debate has not been discussed in the English literature to any great extent, although Imbert 
%\label{ref:RNDa8BYpwRzuE}(1913, pp.57–66)
\parencite*[][pp.57–66]{imbert_frederic_1913} %
 and de Nouvion 
%\label{ref:RNDAZp8LP3T0U}(1905, pp.256–269)
\parencite*[][pp.256–269]{} %
 discussed it in French and Mülberger 
%\label{ref:RNDl2y5LdxKKF}(1896)
\parencite*[][]{} %
 wrote about it extensively in German.



Much of his work, in the original French, is now available on the internet 
%\label{ref:RNDlLvm30yr33}(Bastiat, 1850; 1861; 1862b; 1862a; 1864; 1870; 1873a; 1873b).
\parencites[][]{bastiat_ce_1850}[][]{bastiat_essais_1861}[][]{bastiat_libre-echange_1862}[][]{bastiat_correspondance_1862}[][]{bastiat_cobden_1864}[][]{bastiat_harmonies_1870}[][]{bastiat_ce_1873}[][]{bastiat_sophismes_1873}. %
 About one-third of his works have been translated into English 
%\label{ref:RND0rdHKvAWlE}(Bastiat, 1926; 1964c; 1964b; 1964a; 1991; 1998; 2007).
\parencites[][]{bastiat_bastiat_1926}[][]{bastiat_selected_1964}[][]{bastiat_economic_1964}[][]{bastiat_economic_1964-1}[][]{bastiat_providence_1991}[][]{bastiat_law_1998}[][]{bastiat_bastiat_2007}. %
 The person most responsible for introducing Bastiat to the English speaking world is Dean Russell, who wrote a~dissertation 
%\label{ref:RNDqSGjNLI42a}(Russell, 1959)
\parencite[][]{russell_frederic_1959} %
 and two books 
%\label{ref:RNDTtLQiD9G7W}(Russell, 1969; 1985)
\parencites[][]{russell_frederic_1969}[][]{russell_government_1985} %
 about Bastiat and his work. Hendrick 
%\label{ref:RNDOeWtxbhLkH}(1987)
\parencite*[][]{hendrick_frederic_1987} %
 also wrote a~doctoral dissertation devoted to Bastiat's work, although he did not publish any portion of it. Buccino 
%\label{ref:RND3G8yKECCfw}(1990)
\parencite*[][]{buccino_examination_1990} %
 discussed some of Bastiat's philosophy in her study of other classical political economists in her doctoral dissertation.



George Charles Roche, III, an American historian, wrote two books about Bastiat 
%\label{ref:RNDiZsZB1Uk3m}(Roche, 1971; 1993).
\parencites[][]{roche_frederic_1971}[][]{roche_free_1993}. %
 Bidet 
%\label{ref:RNDl44mvan1N9}(1906),
\parencite*[][]{bidet_frederic_1906}, %
 DeFoville 
%\label{ref:RNDNVpLIQhS4V}(1889),
\parencite*[][]{}, %
 Imbert 
%\label{ref:RNDQTGSchVWka}(1913),
\parencite*[][]{imbert_frederic_1913}, %
 de Nouvion 
%\label{ref:RND2VF7mLIo3y}(1905)
\parencite*[][]{} %
 and Ronce 
%\label{ref:RND7Xfpta7olv}(1905)
\parencite*[][]{} %
 wrote books about him in French. Henry Hazlitt 
%\label{ref:RNDvzMjwg5kfT}(1946; 1979)
\parencites*[][]{hazlitt_economics_1946}[][]{hazlitt_economics_1979} %
 applied Bastiat's methodology to a~number of economic policy issues in the mid- twentieth century. Russell 
%\label{ref:RNDcxTN2E0NDd}(1985)
\parencite*[][]{russell_government_1985} %
 took a~similar approach in the mid-1980s.



Although best known for his work in trade and the philosophy of law, he wrote on other topics as well. To date, no one has examined his views on taxation. The purpose of this paper is to fill that gap in the literature.



\section{Two Philosophical Approaches}

Bastiat was both a~utilitarian 
%\label{ref:RND9IzYgkpHFM}(Bastiat, 1850; 1862b; 1864; 1870; 1873a; 1873b; 1964b; 1964a; 1964c; 2007; 2010)
\parencites[][]{bastiat_ce_1850}[][]{bastiat_libre-echange_1862}[][]{bastiat_cobden_1864}[][]{bastiat_harmonies_1870}[][]{bastiat_ce_1873}[][]{bastiat_sophismes_1873}[][]{bastiat_economic_1964}[][]{bastiat_economic_1964-1}[][]{bastiat_selected_1964}[][]{bastiat_bastiat_2007}[][]{bastiat_collected_2010} %
 and a~rights theorist 
%\label{ref:RNDQ5flWHttis}(Bastiat, 1873a, pp.342–393; 1998; 2007, I: 49-94).
\parencites[][pp.342–393]{bastiat_ce_1873}[][]{bastiat_law_1998}[][]{}. %
 In order to more fully understand his views on taxation it is necessary to take a~few minutes to discuss his two philosophical approaches to various public policy issues.



\subsection{Rights Theory }



The most comprehensive presentation of his rights theory is contained in \textit{The Law} 
%\label{ref:RND2T3VAzAC6i}(Bastiat, 1873a, pp.342–393; 1998; 2007, I: 49-94).
\parencites[][pp.342–393]{bastiat_ce_1873}[][]{bastiat_law_1998}[][]{}. %
 In this essay, first published as a~pamphlet in 1850, Bastiat outlines his basic legal philosophy, which is similar to that of Locke 
%\label{ref:RNDZcjNPFwZHM}(1689)
\parencite*[][]{locke_two_1689} %
 and Nozick 
%\label{ref:RNDwFQBa6BJf4}(1974)
\parencite*[][]{} %
 in many ways. All three believed that government should be limited to the defense of life, liberty and property, which could be labeled a~night watchman state.



Justice reigns when the law is confined to these three functions. When the law goes beyond these three basic functions and into the realm of the redistribution of wealth, the result is injustice.



Bastiat believed that liberty and property existed prior to government. Indeed, the reason governments were formed was to protect life, liberty and property. His position rejects the view of legal positivists, who believe that there is no such thing as inherent rights and that all rights come from government 
%\label{ref:RNDjOtOHRJPAi}(Bentham, 1843; Austin, 1869; Fuller, 1969; Kramer, 1999; Marmor, 2001; Waldron, 1987).
\parencites[][]{bentham_anarchical_1843}[][]{austin_lectures_1869}[][]{fuller_morality_1969}[][]{kramer_defense_1999}[][]{marmor_positive_2001}[][]{waldron_nonsense_1987}.%




In this regard, his view is similar to that of Locke 
%\label{ref:RNDE9eKcDlf3d}(1689)
\parencite*[][]{locke_two_1689} %
 and Nozick 
%\label{ref:RNDFz5IReSonN}(1974),
\parencite*[][]{}, %
 who also reject legal positivism. Although Bastiat mentions ``God'' several times in his writings, he does not take the position that property rights are part of God's plan to make the world a~better place. Such a~position would be closer to that of Locke, who was a~natural law theorist in the Protestant tradition. It could fairly be said that Bastiat was a~secular rights theorist, since his views could be accepted and applied by individuals of any religion, or no religion, although Meredith 
%\label{ref:RNDzsfFbLsrga}(2009)
\parencite*[][]{meredith_taxation_2009} %
 places him in the Christian tradition of normative natural law along the lines of Augustine and Aquinas.



He is against entitlements, such as the right to government health care, government pensions, government subsidies, government enforced minimum wages, government provided welfare, protective tariffs, government education, and so forth. These entitlements are examples of positive law, where the right is not inherent, but rather comes from government. In order for one individual to have a~positive right to something, a~negative right (such as the right to property or the right to contract) of someone else must be violated. In these cases, one person lives at the expense of another.



L'État, c'est la grande fiction à travers laquelle tout le monde s'efforce de vivre aux dépens de tout le monde. 
%\label{ref:RNDOvRPCU5ONu}(Bastiat, 1873a, p.332)
\parencite[][p.332]{bastiat_ce_1873} %
 [The state is that great fiction through which everyone tries to live at the expense of everyone else.]



Governments may not legitimately engage in charity. Engaging in government charity is false philanthropy. True philanthropy involves the giving of one's own property for some worthy cause. False philanthropy involves taking one person's property and giving it to another for what some politician or bureaucrat considers to be a~worthy cause.



Individuals have the right to defend their rights to life, liberty and property. That being the case, it follows logically that groups of individuals can band together to defend these individual rights. Forming defense organizations to defend these rights often makes sense, since it increases efficiency. These defense organizations are often governments, but can also be private defense organizations.



These governments or private defense organizations possess no rights that the individuals who formed them do not possess. Just as an individual has no right to steal, neither does a~government have a~right to steal. If an individual forcibly takes someone else's property, it is theft, which Bastiat refers to as illegal plunder. When a~government takes one person's property and gives it to another person, Bastiat calls it illegal plunder.



Under the pretense of organization, regulation, protection, or encouragement, the law takes property from one person and gives it to another; the law takes the wealth of all and gives it to a~few -- whether farmers, manufacturers, shipowners, artists, or comedians. 
%\label{ref:RNDXBZSHuGBWQ}(Bastiat, 1998, p.13)
\parencite[][p.13]{bastiat_law_1998}%




For Bastiat, government expenditures are just and legitimate only if made for the general welfare, such as the protection of life, liberty and property. Justice, defense and public works expenditures may be justified, but not much else 
%\label{ref:RNDcUGp619bR8}(Braun and Blanco, 2011).
\parencite[][]{braun_bastiat_2011}. %
 Expenditures for special interests such as farmers, manufacturers, artists, students or anyone else are illegitimate because the government takes property from some and gives it to others. All special interest legislation constitutes a~form of legal plunder and any legislation that does not benefit the vast majority constitutes special interest legislation.



But how is this legal plunder to be identified? Quite simply. See if the law takes from some persons what belongs to them, and gives it to other persons to whom it does not belong. See if the law benefits one citizen at the expense of another by doing what the citizen himself cannot do without committing a~crime.



Then abolish this law without delay, for it is not only an evil itself, but also it is a~fertile source for further evils because it invites reprisals. If such a~law -- which may be an isolated case -- is not abolished immediately, it will spread, multiply, and develop into a~system. 
%\label{ref:RNDUMFlokbDKY}(Bastiat, 1998, p.17)
\parencite[][p.17]{bastiat_law_1998}%




When a~portion of wealth is transferred from the person who owns it -- without his consent and without compensation, and whether by force or by fraud -- to anyone who does not own it, then I~say that property is violated; that an act of plunder is committed. 
%\label{ref:RNDWq8TJ2zRot}(Bastiat, 1998, p.22)
\parencite[][p.22]{bastiat_law_1998}%




\subsection{Utilitarianism}



Bastiat was also a~utilitarian. What distinguishes Bastiat's version of utilitarianism from some other versions is that Bastiat made a~sincere attempt to determine the effect a~policy would have on all groups in both the long-run and the short-run. On the first page of his \textit{Selected Essays on Political Economy} 
%\label{ref:RND8PcUAYIFfM}(Bastiat, 1964c, p.1; 2007, I, p.1)
\parencites[][p.1]{bastiat_selected_1964}[][p.1]{} %
 he states that:



There is only one difference between a~bad economist and a~good one: the bad economist confines himself to the \textit{visible} effect; the good economist takes into account both the effect that can be seen and those effects that must be \textit{foreseen}. 
%\label{ref:RNDJMRpVTcWNE}(Bastiat, 1964c, p.1)
\parencite[][p.1]{bastiat_selected_1964}%




He goes on the elaborate on this methodology and provides examples in several of his works 
%\label{ref:RNDW4BCke9c9t}(Bastiat, 1850; 1870; 1873a; 1873b; 1964b; 1964a; 1964c; 2007).
\parencites[][]{bastiat_ce_1850}[][]{bastiat_harmonies_1870}[][]{bastiat_ce_1873}[][]{bastiat_sophismes_1873}[][]{bastiat_economic_1964}[][]{bastiat_economic_1964-1}[][]{bastiat_selected_1964}[][]{bastiat_bastiat_2007}.%




\section{Views on Taxation and Public Finance}

``The state can give nothing to the citizens that it has not first taken from them.'' 
%\label{ref:RNDO7WWnl5Pkl}(Bastiat, 1964c, p.183)
\parencite[][p.183]{bastiat_selected_1964}%




According to one journalist, Bastiat ``argues that governments are essentially stealing when they tax their citizens to spend on welfare, infrastructure or public education 
%\label{ref:RNDVR77wY4A9G}(Zernike, 2010).
\parencite[][]{zernike_shaping_2010}. %
 However, this statement is not quite accurate. Bastiat was not against spending for some public works 
%\label{ref:RNDhoySKuedbl}(Bastiat, 1964b, p.46)
\parencite[][p.46]{bastiat_economic_1964} %
 and he thought that spending for national defense and justice were acceptable uses of tax funds 
%\label{ref:RNDouvGYzHrci}(Bastiat, 1964c, p.184).
\parencite[][p.184]{bastiat_selected_1964}.%




While taxes may be raised for the defense of life, liberty and property, they may not be raised for redistributive purposes. A~redistributive tax system is inherently unjust because it uses force to take property from its rightful owners and distributes it to those who have no just claim on it. Redistributive taxation is a~form of legal plunder.



However, Bastiat was not against all forms of taxation. Taxes were justified if the people whose salaries they paid rendered services to the taxpayers that were equal to what they were paid, in other words, if the people got their moneysworth from their taxes 
%\label{ref:RNDS9qEUqX5PK}(Bastiat, 1964c, p.182).
\parencite[][p.182]{bastiat_selected_1964}. %
 Economists who are familiar with the relative costs and benefits of privatization might be quick to assert that the citizenry seldom, if ever, gets its moneysworth from government, since the private sector can do just about anything faster, cheaper and better than government 
%\label{ref:RNDmwcdkh1Cdz}(Finley, 1989; Ohashi and Roth, 1980; Pirie, 1988; Savas, 1982),
\parencites[][]{finley_public_1989}[][]{}[][]{pirie_privatization_1988}[][]{}, %
 but Bastiat did not raise that question, since privatization was not an issue in the 1840s, probably because the state was relatively small at the time in terms of the institutions and infrastructure that it owned that could be transferred to the private sector.



Then there is the question of whether it can be determined whether the people actually got their moneysworth from government services, since value is a~subjective thing. It's probably true to say that there is no way to determine whether every individual received equal value for government services rendered, but it can be assumed that some individuals received more in services than what they paid in taxes while others received less than what they paid. That being the case, it would be impossible to determine whether taxes could be justified.



\subsection{Progressive Taxation}



Bastiat identified progressive taxation as a~form of plunder 
%\label{ref:RNDMMlIQpes1v}(Bastiat, 1998, pp.18, 27).
\parencite[][pp.18]{}. %
 Presumably, he would approve of a~flat tax, provided the funds spent were limited to the defense of life, liberty and property. He strongly opposed the Marxist concept, ``From each according to his ability, to each according to his needs.'' 
%\label{ref:RNDjDbswalsUx}(Marx, 1875)
\parencite[][]{marx_kritik_1875} %
 Marx and Engels advocated both a~heavy, progressive income tax and a~100 percent inheritance tax 
%\label{ref:RNDem3WxeKsrN}(Marx and Engels, 1848).
\parencite[][]{marx_manifest_1848}. %
 Unless Bastiat could read German, we can be sure that he did not read \textit{The Communist Manifesto} 
%\label{ref:RNDyyv1lKuX9y}(Marx and Engels, 1848)
\parencite[][]{marx_manifest_1848} %
 because the French and English translations did not appear until after his death 
%\label{ref:RND8kmSshTBjS}(see Marx and Engels, 2010),
\parencite[see][]{marx_communist_2010}, %
 but the ideas Marx and Engels espoused in that document were circulating in Europe during Bastiat's lifetime.



\subsection{Using Taxes as a~Means of Equalizing Wealth }



Bastiat viewed the use of the tax system as a~means of equalizing wealth as communism 
%\label{ref:RNDIc1TXkx8IV}(Bastiat, 1964c, p.111).
\parencite[][p.111]{bastiat_selected_1964}. %
 He was against the notion that disparities of wealth should be reduced through the tax system.



\subsection{Gift Taxes }



Bastiat viewed gift taxes as a~violation of property rights:



Exchange, like property, is a~natural right. Every citizen who has produced or acquired a~product should have the option of applying it immediately to his own use or of giving it to whoever on the face of the earth consents to give him in exchange the object of his desires. To deprive him of this faculty, when he has committed no act contrary to public order and good morals, and solely to satisfy the convenience of another citizen, is to legitimize an act of plunder and to violate the law of justice. 
%\label{ref:RNDzjuKu0y4K5}(Bastiat, 1964c, p.112)
\parencite[][p.112]{bastiat_selected_1964}%




\subsection{Inheritance Taxes }



Bastiat was against inheritance taxes, which he regarded as a~violation of property rights. Property comes into existence as the result of labor. It is the fruit of one's labor, which can be passed on to others 
%\label{ref:RNDRITReuo711}(Bastiat, 1964c, pp.188–193).
\parencite[][pp.188–193]{bastiat_selected_1964}.%




The right of inheritance, against which so much has been objected of late, is one of the forms of gift, and assuredly the most natural of all. That which a~man has produced, he may consume, exchange, or give. What can be more natural than that he should give it to his children? It is this power, more than any other, that inspires him with the drive to labor and to save. Do you know why the principle of right of inheritance is thus called in question? Because it is imagined that the property thus transmitted is plundered from the masses. This is a~fatal error. 
%\label{ref:RNDL1Xg5XbOj9}(Bastiat, 2007, I, p.142)
\parencite[][p.142]{bastiat_bastiat_2007}%




\subsection{Using Taxes to Stimulate Economic Activity }



Bastiat opposed the use of tax money to stimulate the economy for two reasons: (1) it was a~form of redistribution of wealth, and therefore legalized plunder, and (2) it did not work. He did not oppose the use of taxes to provide legitimate services, whatever they may be (i.e. services that benefitted the vast majority of the people), but he did oppose using taxes to prime the pump, so to speak, as Keynesian economists advocate. 
%\label{ref:RNDb4zD0FTgTR}(Bastiat, 1964c, pp.8–9, 16).
\parencite[][pp.8–9]{}. %
 Every hundred sous (a French monetary unit at the time) a~Frenchman gives to support the salary of some government bureaucrat is 100 sous that he cannot spend himself 
%\label{ref:RND2E1glbqoo2}(Bastiat, 1964c, p.8).
\parencite[][p.8]{bastiat_selected_1964}. %
 The transfer is merely from one person's pocket to that of another. Total spending and total economic activity do not increase.



While this example may seem to be so obvious that it is hardly worthy of mention, the Keynesian multiplier theory 
%\label{ref:RNDy06zMPLaFN}(Keynes, 1936)
\parencite[][]{keynes_general_1936} %
 is based on the belief that increasing government spending results in a~multiplier effect that increases total economic activity. In fact, increased government spending results in less private sector spending. If the additional funds are raised in the form of borrowing rather than taxes, the result does not change. A~detailed examination of this phenomenon is beyond the scope of the present paper, but this topic has been covered in depth elsewhere 
%\label{ref:RND5ErxIxZb8j}(Ahiakpor, 2000; Dimand, 1997; 2000; Hazlitt, 1946; 1959; 1960; 1979; Hegeland, 1954; Hutt, 1963; 1979; Skousen, 1992; Terborgh, 1968).
\parencites[][]{ahiakpor_hawtrey_2000}[][]{dimand_hawtrey_1997}[][]{dimand_hawtrey_2000}[][]{hazlitt_economics_1946}[][]{hazlitt_failure_1959}[][]{hazlitt_critics_1960}[][]{hazlitt_economics_1979}[][]{hegeland_multiplier_1954}[][]{hutt_keynesianism--retrospect_1963}[][]{hutt_keynesian_1979}[][]{skousen_dissent_1992}[][]{terborgh_new_1968}.%




In Bastiat's time the argument was made that a~troop demobilization would result in increased unemployment. What would happen to the troops if they were demobilized? The reply was that they would become unemployed. As they returned to their home towns they would depress labor rates.



The opposite side of the coin is that French taxpayers would be relieved of paying a~hundred million francs. But the army consumes bread, wine, clothes and weapons, and such purchases spread throughout the economy. All this commercial activity would come to an end if the soldiers went home. Thus, the army must be maintained for economic reasons, even though the soldiers are not needed, or so the argument goes 
%\label{ref:RNDn74wZdSbIx}(Bastiat, 1964c, pp.4–5).
\parencite[][pp.4–5]{bastiat_selected_1964}.%




As Bastiat would say, what is seen is 100,000 soldiers who live well and who provide a~living for their suppliers. What is not seen is the fact that the hundred million francs used to support those soldiers cannot be used to support the taxpayers who are providing the funds.



If the soldiers return to their home towns, what is seen is 100,000 unemployed men being dumped into the labor market, causing wages to become depressed and deepening unemployment. What is not seen is the hundred million francs that are now free to hire those unemployed soldiers. Since the taxpayers are no longer being taxed to support soldiers who create no products or services, they are free to employ those soldiers, who will now be able to produce something. Overall production will increase because the soldiers, who were paid to march back and force, will now be producing something. All of society will benefit 
%\label{ref:RNDoqCIEYBTNp}(Bastiat, 1964c, pp.6–7).
\parencite[][pp.6–7]{bastiat_selected_1964}.%




\subsection{Tariffs }



Tariffs are a~form of taxation, in the sense that they raise revenue for governments. Prior to the adoption of the Sixteenth Amendment to the U.S. Constitution in 1913, tariffs were one of the major sources of revenue in the United States and were the major source of revenue in many other countries prior to the income tax 
%\label{ref:RND1clfVc4Zud}(Webber and Wildavsky, 1986, pp.269–270).
\parencite[][pp.269–270]{webber_history_1986}. %
 However, governments often use tariffs for a~more sinister reason: to protect domestic industry from foreign competition. This use (abuse) of tariffs has been present in history ever since tariffs were first imposed 
%\label{ref:RNDdGkZc37UKS}(Webber and Wildavsky, 1986).
\parencite[][]{webber_history_1986}.%




Tariffs are a~form of subsidy, special interest legislation, since they help one small segment of the domestic population (domestic producers) at the expense of the general public. Bastiat was against all tariffs because he regarded them as legalized plunder 
%\label{ref:RNDhyUtB62p9V}(Bastiat, 1861; 1862b; 1862a; 1864; 1873a; 1873b; 1964b; 1964a; 1964c; 1998; 2007).
\parencites[][]{bastiat_essais_1861}[][]{bastiat_libre-echange_1862}[][]{bastiat_correspondance_1862}[][]{bastiat_cobden_1864}[][]{bastiat_ce_1873}[][]{bastiat_sophismes_1873}[][]{bastiat_economic_1964}[][]{bastiat_economic_1964-1}[][]{bastiat_selected_1964}[][]{bastiat_law_1998}[][]{bastiat_bastiat_2007}.%




There are two potential causes of revolution in the United States: slavery and the high protective tariff…In regard to the tariff question the law says: ``I shall create an armed force, at the citizens' expense, not to make sure that their transactions are free, but to make sure that they are not free, to impair the equivalence of services, so that one citizen may have the liberty of two, and that another may have none at all.'' 
%\label{ref:RND9Bjc9YQet1}(Bastiat, 1964a, p.462)
\parencite[][p.462]{bastiat_economic_1964-1}%




Bastiat's perception was correct. The reason Fort Sumter was attacked by Confederate forces on April 12, 1861, thus starting the American Civil War, was because the fort was being used as a~collection point for the tariff. The fort had no military significance 
%\label{ref:RNDmFxIUMOrZ7}(Adams, 2000, pp.17–33).
\parencite[][pp.17–33]{adams_when_2000}. %
 The reason the southern states wanted to secede from the Union was because of northern hegemony, part of which included the high tariff.



At the time Lincoln was pushing his high tariff through the Congress, the Southerners were doing just the opposite. Their new constitution was adopted … with a~unique provision banning high import taxation… Jefferson Davis, the first president of the Confederacy, justified secession in his inaugural address by making reference to the Declaration of Independence, then emphasizing the import tax issue…With low duties the trade of North America would shift from New York, Boston, and Philadelphia to Savannah, Charleston, and New Orleans…This would spell disaster for the Northern industrialists. Secession offered the South not only freedom from Northern tax bondage but also an opportunity to turn from the oppressed into the oppressor… 
%\label{ref:RNDNcxL4T6tok}(Adams, 1993, p.332).
\parencite[][p.332]{adams_for_1993}.%




Adams 
%\label{ref:RNDzvtbcx1EWl}(Adams, 1993, pp.332–333)
\parencite[][pp.332–333]{adams_for_1993} %
 goes on to state that the main cause of the Civil War was the tariff, not slavery, which was a~secure institution in the South, and which Lincoln promised not to change in the territories where it already existed. Bastiat 
%\label{ref:RNDGeL1m0uVQp}(1964a, p.462)
\parencite*[][p.462]{bastiat_economic_1964-1} %
 was able to see that the high U.S. tariff could lead to war during the 1840s. DiLorenzo 
%\label{ref:RNDa4ppuAQk7U}(2002, p.63)
\parencite*[][p.63]{dilorenzo_real_2002} %
 points out that the high tariff triggered a~constitutional crisis when some South Carolina politicians suggested refusing to collect the tariff at the Charleston, South Carolina port.



As a~general rule, Bastiat viewed tariffs, or customs duties, as a~violation of property rights because the purpose is to protect domestic producers from foreign competition. The tariff constitutes special interest legislation because it benefits a~small group at the expense of the general public. However, if the funds are used for the common expense, the tax is legitimate 
%\label{ref:RNDxODiCazUnT}(Bastiat, 1964c, pp.111–112).
\parencite[][pp.111–112]{bastiat_selected_1964}.%




\subsection{Taxes on Capital }



Bastiat believed that the proletariat can be freed only by increases in capital accumulation. When the amount of capital increases more rapidly than the increase in population, two things happen: lower prices and higher wages. Both of these things improve the lot of the worker. He was against what he referred to as the \textit{war on capital}, the taxing of capital for reasons other than to raise the revenue necessary to perform the legitimate functions of government. Capital that is not secure hides or flees. When that happens there is less money available to employ people. The result is unemployment for some and lower wages for others 
%\label{ref:RNDSAmFwdnUqh}(Bastiat, 1964c, pp.184–185).
\parencite[][pp.184–185]{bastiat_selected_1964}.%




\subsection{Tax Burden }



When a~nation is burdened with taxes, nothing is more impossible than to levy them equally. The tax burden is shifted onto the rich. When government expenditures expand beyond what is needed to pay for its legitimate functions, the state produces more poverty than it cures. When it is an accepted principle that the function of government is to distribute wealth, the tax burden expands beyond its just limits. The amount taken in taxes should be no more than what is needed to protect the people from violence and fraud. Bastiat proposed a~single tax that is proportional to the amount of property owned 
%\label{ref:RNDcF0ePYPYxa}(Bastiat, 1964c, pp.125–126).
\parencite[][pp.125–126]{bastiat_selected_1964}.%




\subsection{School Taxes }



Bastiat opposed forcing some people to pay for the education of other people's children. In Bastiat's time, the government supported the major religions. As a~result, Catholics were forced to support Jewish organizations and Jews were forced to support Catholic organizations. Some religious organizations had their own schools.



Bastiat believed that parents should be responsible for the education of their own children. He also believed that government should not have a~monopoly on education. He disapproved of the top-down, government monopoly on university curriculum, which was based on a~study of the classics. He disapproved of a~classic education because classical scholars glorified plunder and socialism. He did not believe that taxpayers should be forced to pay for indoctrinating the younger generation with such false knowledge 
%\label{ref:RNDRW1O0QjT3Q}(Bastiat, 1964c, pp.278–283).
\parencite[][pp.278–283]{bastiat_selected_1964}.%




\subsection{Subsidizing the Arts}



Bastiat begins his discussion with the question, ``Should the state subsidize the arts?'' 
%\label{ref:RNDCCxtmorCrS}(Bastiat, 1964c, p.11)
\parencite[][p.11]{bastiat_selected_1964} %
 It could be argued that the arts broaden and elevate the soul of the nation. Furthermore, French culture is the envy of the world. Should this modest assessment on the citizens of France be stopped?



He goes on to point out that the issue is really a~question of distributive justice. Do the rights of the legislator allow him to reach into the pocket of the workers to supplement the income of the artist? He also asks whether subsidizing the arts results in the progress of the arts. One might point out that in totalitarian regimes such as those in Nazi Germany 
%\label{ref:RND4kJX8s8Nxs}(Fürstenau, 2020),
\parencite[][]{furstenau_how_2020}, %
 Stalinist Russia 
%\label{ref:RNDU3ezK1zX6G}(Beale, 2019)
\parencite[][]{beale_history_2019} %
 or Maoist China 
%\label{ref:RNDEygDrr4cu9}(Burgess, 2018)
\parencite[][]{burgess_art_2018} %
 the arts were used as propaganda tools. Using art as propaganda has a~long, if undistinguished history 
%\label{ref:RNDxDVwqNLgFZ}(Levy, 2021; Weissman, 2023).
\parencites[][]{}[][]{weissman_how_2023}.%




But getting back to the question of opportunity cost, what is seen is the effect of subsidizing certain arts. What is not seen is what would have happened if those funds had instead been spent by the taxpayers who earned that income. Bastiat believes that the decision as to where the funds should be spent should come from below, not from above.



He later points out that any francs the government spends on the arts creates employment in that field, but only at the expense of employment in the fields where the taxpayers otherwise would have spent their wages. He concludes that the government cannot create jobs but only shift them from one sector of the economy to another.



\subsection{Public Works}



Another example in his essay, \textit{What Is Seen and What Is Not Seen} 
%\label{ref:RND4rWwZ7LFq1}(Bastiat, 1964c)
\parencite[][]{bastiat_selected_1964}%
\textit{,} addresses the question of public works. Whenever the state opens a~road, builds a~palace digs a~canal or repairs a~street it provides jobs for certain workers. That is what is seen. But what is not seen is the workers who are deprived of jobs because the funds that are used for those public works cannot be used to hire their services.



He goes on to say that where the expenditure has utility, such as building a~bridge that is needed, there is not a~problem. Problems result when the state engages in public works projects for the purpose of creating employment. Such a~goal might be used to justify the most prodigal enterprises 
%\label{ref:RNDnxpuZdjmYZ}(Bastiat, 1964c, p.17).
\parencite[][p.17]{bastiat_selected_1964}.%




The great Napoleon, it is said, thought he was doing philanthropic work when he had ditches dug and then filled in. He also said: ``What difference does the result make? All we need is to see wealth spread among the laboring classes.'' 
%\label{ref:RND0k8NGv0ief}(Bastiat, 1964c, p.18)
\parencite[][p.18]{bastiat_selected_1964}%




Later in his essay he goes on to say that public expenditures must be evaluated on their own merits because the effect of any public expenditure is not to create jobs but to divert them. Furthermore, reallocating jobs displaces workers, which disturbs the natural laws that govern the distribution of population over the earth 
%\label{ref:RNDeuVFcwTHlJ}(Bastiat, 1964c, p.41).
\parencite[][p.41]{bastiat_selected_1964}. %
 There is also the danger that the public expenditure will create less useful jobs than the jobs that are prevented from coming into existence, since the latter are created by the wants and demands of the people who have earned the money whereas the former are created by bureaucrats, who are creating jobs just for the sake of expanding employment without regard to the wants and needs of the citizenry.



\section{Conclusion}

The contributions Bastiat has made to the economic and philosophical literature are substantial 
%\label{ref:RNDrAsjj3NeS5}(McGee, 2014c).
\parencite[][]{mcgee_relevance_2014}. %
 He saw the market economy as a~harmony of interests rather than a~struggle between classes 
%\label{ref:RNDHwqgMy5613}(Braun and Blanco, 2011).
\parencite[][]{braun_bastiat_2011}. %
 He opposed government intervention in the economy, since intervention would cause more harm than good 
%\label{ref:RNDywko6BORQM}(Hülsmann, 2001).
\parencite[][]{hulsmann_bastiats_2001}. %
 He anticipated and refuted the Keynesian multiplier theory more than a~generation before Keynes (1883-1946) was born 
%\label{ref:RNDeodsB6C4Fy}(McGee, 2014b).
\parencite[][]{mcgee_keynes_2014}. %
 He is one of the few economic philosophers whose essays have lived on more than 150 years after his death. His view of free trade and protectionism is unsurpassed 
%\label{ref:RNDhwKIapjMx0}(McGee, 2014a).
\parencite[][]{mcgee_economic_2014}.%




His contributions to public finance are discussed in the current paper. His view was that of a~utilitarian classical liberal who believed that taxation could be justified only in cases where the tax funds were spent on projects that benefitted the vast majority of the population. Such expenditures included programs that would protect life, liberty and property. Tax funds spent for any other purpose constituted redistribution, and were therefore illegitimate. His philosophy of public finance is as relevant today as it was in the 1840s, when he wrote on this topic.



\section{References}

Adams, C., 1993. \textit{For Good and Evil: The Impact of Taxes on the Course of Civilization}. London; New York: Madison Books.



Adams, C., 2000. \textit{When in the Course of Human Events: Arguing the Case for Southern Secession}. [online] Lanham, Boulder, New York and Oxford: Rowman \& Littlefield Publishers. Available at: {\textless}https://rowman.com/ISBN/9780847697236/When-in-the-Course-of-Human-Events-Arguing-the-Case-for-Southern-Secession{\textgreater} [Accessed 1 October 2024].



Ahiakpor, J.C.W., 2000. Hawtrey on the Keynesian Multiplier: A~Question of Cognitive Dissonance? \textit{History of Political Economy}, [online] 32(4), pp.889–908. https://doi.org/10.1215/00182702-32-4-889.



Austin, J., 1869. \textit{Lectures on Jurisprudence or the Philosophy of Positive Law}. [online] London: John Murray. Available at: {\textless}http://archive.org/details/in.ernet.dli.2015.275273{\textgreater} [Accessed 1 October 2024].



Bastiat, F., 1850. \textit{Ce qu'on voit et ce qu'on ne voit pas [What Is Seen and What Is Not Seen]}. [online] Available at: {\textless}http://bastiat.org/fr/cqovecqonvp.html{\textgreater} [Accessed 1 October 2024].



Bastiat, F., 1861. Essais, Ébauches, Correspondance. In: P. Paillottet, ed. \textit{Oeuvres complètes de Frédéric Bastiat, 2\textsuperscript{nd} ed. vol. 7 Essais, Ébauches, Correspondance}. [online] Paris: Guillaumin. Available at: {\textless}https://oll.libertyfund.org/titles/paillottet-oeuvres-completes-de-frederic-bastiat-2\textsuperscript{nd}-ed-vol-7-essais-ebauches-correspondance{\textgreater} [Accessed 1 October 2024].



Bastiat, F., 1862a. Correspondance Mélanges. In: P. Paillottet, ed. \textit{Oeuvres complètes de Frédéric Bastiat, 2\textsuperscript{nd} ed. vol. 1 Correspondance et mélanges}. [online] Paris: Guillaumin. Available at: {\textless}https://oll.libertyfund.org/titles/paillottet-oeuvres-completes-de-frederic-bastiat-2\textsuperscript{nd}-ed-vol-1-correspondance-et-melanges{\textgreater} [Accessed 1 October 2024].



Bastiat, F., 1862b. Le Libre-Échange. In: P. Paillottet, ed. \textit{Oeuvres complètes de Frédéric Bastiat, 2\textsuperscript{nd} ed. vol. 2 le Libre-Échange}. [online] Paris: Guillaumin. Available at: {\textless}https://oll.libertyfund.org/titles/paillottet-oeuvres-completes-de-frederic-bastiat-2\textsuperscript{nd}-ed-vol-3-cobden-et-la-ligue{\textgreater} [Accessed 1 October 2024].



Bastiat, F., 1864. Cobden et la Ligue. In: P. Paillottet, ed. \textit{Oeuvres complètes de Frédéric Bastiat, 2\textsuperscript{nd} ed. vol. 3 Cobden et La Ligue}. [online] Paris: Guillaumin. Available at: {\textless}https://oll.libertyfund.org/titles/paillottet-oeuvres-completes-de-frederic-bastiat-2\textsuperscript{nd}-ed-vol-3-cobden-et-la-ligue{\textgreater} [Accessed 1 October 2024].



Bastiat, F., 1870. Harmonies Économiques. In: P. Paillottet, ed. \textit{Oeuvres complètes de Frédéric Bastiat, 3\textsuperscript{rd} ed. vol. 6 Harmonies économiques}. [online] Paris: Guillaumin. Available at: {\textless}https://oll.libertyfund.org/titles/paillottet-oeuvres-completes-de-frederic-bastiat-3\textsuperscript{rd}-ed-vol-6-harmonies-economiques{\textgreater} [Accessed 1 October 2024].



Bastiat, F., 1873a. Sophismes Économiques I. In: P. Paillottet, ed. \textit{Oeuvres complètes de Frédéric Bastiat, 2\textsuperscript{nd} ed. vol. 4 Sophismes Économiques. Petits pamphlets I}. [online] Paris: Guillaumin. Available at: {\textless}https://oll.libertyfund.org/titles/paillottet-oeuvres-completes-de-frederic-bastiat-2\textsuperscript{nd}-ed-vol-4-sophismes-economiques-petits-pamphlets-i{\textgreater} [Accessed 1 October 2024].



Bastiat, F., 1873b. Sophismes Économiques II. In: P. Paillottet, ed. \textit{Oeuvres complètes de Frédéric Bastiat, 1\textsuperscript{st} ed. vol. 5 Sophismes économiques. Petits pamphlets II}. [online] Paris: Guillaumin. Available at: {\textless}https://oll.libertyfund.org/titles/paillottet-oeuvres-completes-de-frederic-bastiat-1\textsuperscript{st}-ed-vol-5-sophismes-economiques-petits-pamphlets-ii{\textgreater} [Accessed 1 October 2024].



Bastiat, F., 1889. \textit{F. Bastiat: oeuvres choisies}. Petite bibliothèque économique française et étrangère. Guillaumin. Paris.



Bastiat, F., 1926. \textit{Bastiat and the Abc of Free Trade, Translated from the Writings of Frédéric Bastiat}. [online] Translated by L. Garreau London: T. Fisher Unwin Ltd. Available at: {\textless}https://catalog.hathitrust.org/Record/009077401{\textgreater} [Accessed 1 October 2024].



Bastiat, F., 1964a. \textit{Economic Harmonies}. Translated by W.H. Boyers Irvington-on-Hudson, N.Y.: Foundation for Economic Education.



Bastiat, F., 1964b. \textit{Economic Sophism}. Irvington-on-Hudson, N.Y.: Foundation For Economic Education.



Bastiat, F., 1964c. \textit{Selected Essays on Political Economy}. Translated by S. Cain Irvington-on-Hudson, N.Y.: Foundation for Economic Education.



Bastiat, F., 1991. \textit{Providence and Liberty: Selected Passages from Frédéric Bastiat}. Translated by R. Audouin Grand Rapids, MI: Acton Institute for the Study of Religion and Liberty.



Bastiat, F., 1998. \textit{The Law}. 2\textsuperscript{nd} ed. Irvington-on-Hudson, N.Y.: Foundation for Economic Education.



Bastiat, F., 2007. \textit{The Bastiat Collection}. [online] Auburn, AL: Ludwig von Mises Institute. Available at: {\textless}https://cdn.mises.org/The\%20Bastiat\%20Collection\_4.pdf{\textgreater} [Accessed 1 October 2024].



Bastiat, F., 2010. \textit{The Collected Works of Frédéric Bastiat: Vol. 4: Controversies and Miscellanies}. Indianapolis: Liberty Fund.



Beale, A.D., 2019. \textit{History of Art in the Soviet Union: Propaganda, Rebellion, and Freedom in Socialist Realism}. [online] History is Now Magazine. Available at: {\textless}http://www.historyisnowmagazine.com/blog/2019/5/19/history-of-art-in-the-soviet-union-propaganda-rebellion-and-freedom-in-socialist-realism{\textgreater} [Accessed 1 October 2024].



Bentham, J., 1843. Anarchical Fallacies. In: J. Bowring, ed. \textit{The Works of Jeremy Bentham}. [online] Edinburgh, London: William Tait and Simkin, Marshall \& Co. pp.489–534. Available at: {\textless}https://oll.libertyfund.org/titles/bowring-the-works-of-jeremy-bentham-vol-2{\textgreater} [Accessed 1 October 2024].



Bidet, F., 1906. \textit{Frédéric Bastiat, l'homme, l'économiste}. Paris: V. Giard et E. Brière.



Blaug, M., 1986. \textit{Great Economists Before Keynes: An Introduction to the Lives \& Works of One Hundred Great Economists of the Past}. Brighton: Wheatsheaf Books.



Braun, C.R. and Blanco, M., 2011. Bastiat as an Economist. \textit{The Independent Review}, [online] 15(3), pp.421–445. Available at: {\textless}https://www.jstor.org/stable/24562518{\textgreater} [Accessed 1 October 2024].



Buccino, J., 1990. \textit{Examination Of Higher Education In The Writings Of The Classical Political Economists}. PhD dissertation. University of South Florida, USA.



Buchanan, J.M., 1973. Introduction. In: J.M. Buchanan and G.F. Thirlby, eds. \textit{L. S. E. Essays on Cost}. [online] New York: London School of Economics and Political Science: Weidenfeld and Nicolson. pp.1–16. Available at: {\textless}https://oll.libertyfund.org/titles/buchanan-l-s-e-essays-on-cost{\textgreater} [Accessed 1 October 2024].



Burgess, A., 2018. \textit{The Art of Chinese Propaganda Posters}. [online] Atlas Obscura. Available at: {\textless}http://www.atlasobscura.com/articles/chinese-propaganda-posters-cultural-revolution-shaomin-li{\textgreater} [Accessed 1 October 2024].



Cantillon, R., 1755. \textit{Essai sur la nature du commerce en général}. Londres: Fletcher Gyles.



Cossa, L., 1893. \textit{An Introduction to the Study of Political Economy}. [online] London; New York: Macmillan. Available at: {\textless}https://katalog.ue.wroc.pl/161800097318/cossa-luigi/an-introduction-to-the-study-of-political-economy?bibFilter=16\&lokFilter=20{\textgreater} [Accessed 1 October 2024].



DiLorenzo, T.J., 1999. Frédéric Bastiat: Between the French and Marginalist Revolutions. In: R.G. Holcombe, ed. \textit{15 Great Austrian Economists}. Auburn, AL: Ludwig von Mises Institute. pp.59–69.



DiLorenzo, T.J., 2002. \textit{The Real Lincoln: A~New Look at Abraham Lincoln, His Agenda, and an Unnecessary War}. Roseville, CA: Prima Publishing.



Dimand, R.W., 1997. Hawtrey and the Multiplier. \textit{History of Political Economy}, [online] 29(3), pp.549–556. https://doi.org/10.1215/00182702-29-3-549.



Dimand, R.W., 2000. Hawtrey and the Keynesian Multiplier: A~Response to Ahiakpor. \textit{History of Political Economy}, [online] 32(4), pp.909–914. https://doi.org/10.1215/00182702-32-4-909.



Finley, L.K. ed., 1989. \textit{Public Sector Privatization: Alternative Approaches to Service Delivery}. New York: Quorum Books.



Fuller, L.L., 1969. \textit{The Morality of Law}. [online] New Haven: Yale University Press. Available at: {\textless}http://www.jstor.org/stable/j.ctt1cc2mds{\textgreater} [Accessed 1 October 2024].



Fürstenau, M., 2020. \textit{How the Nazis Used Poster Art as Propaganda}. [online] dw.com. Available at: {\textless}https://www.dw.com/en/how-the-nazis-used-poster-art-as-propaganda/a-55751640{\textgreater} [Accessed 1 October 2024].



Gide, C. and Rist, C., 1948. \textit{A~History of Economic Doctrines from the Time of the Physiocrats to the Present Day}. 2\textsuperscript{nd} ed. Translated by R. Richards Boston: D.C. Heath.



Haney, L.H., 1949. \textit{History of Economic Thought}. [online] New York: The Macmillan Company. Available at: {\textless}http://archive.org/details/dli.ernet.16872{\textgreater} [Accessed 1 October 2024].



Hazlitt, H., 1946. \textit{Economics in One Lesson}. New York: Harper \& Brothers.



Hazlitt, H., 1959. \textit{The Failure of the ‘New economics'}. Princeton, N.J: van Nostrand.



Hazlitt, H. ed., 1960. \textit{The Critics of Keynesian Economics}. Princeton, N.J.: Van Nostrand.



Hazlitt, H., 1964. Introduction. In: A. Goddard, ed. \textit{Economic Sophisms}. [online] Irvington-on-Hudson, NY: Foundation for Economic Education. pp.xi–xv. Available at: {\textless}https://oll.libertyfund.org/titles/hazlitt-economic-sophisms-fee-ed{\textgreater} [Accessed 1 October 2024].



Hazlitt, H., 1979. \textit{Economics in One Lesson: The Shortest and Surest Way to Understand Basic Economics}. New York: Arlington House.



Hébert, R.F., 1987. Bastiat, Claude Frédéric (1801–1850). In: J. Eatwell, M. Milgate and P. Newman, eds. \textit{The New Palgrave Dictionary of Economics}. [online] London: Palgrave Macmillan UK. https://doi.org/10.1057/978-1-349-95121-5\_30-1.



Hegeland, H., 1954. \textit{The Multiplier Theory}. Lund social science studies. Lund: C.W.K. Gleerup.



Hendrick, R.M., 1987. \textit{Frédéric Bastiat, forgotten liberal: Spokesman for an ideology in crisis}. PhD thesis. Department of History, Graduate School of Arts and Sciences, New York University.



Hülsmann, J.G., 2001. Bastiat's Legacy in Economics. \textit{The Quarterly Journal of Austrian Economics}, [online] 4(4), pp.55–70. https://doi.org/10.1007/BF03184236.



Hutt, W.H., 1963. \textit{Keynesianism-{}-retrospect and prospect: a~critical restatement of basic economic principles}. [online] Chicago: H. Regnery Co. Available at: {\textless}http://archive.org/details/keynesianismretr00hutt{\textgreater} [Accessed 1 October 2024].



Hutt, W.H. (William H., 1979. \textit{The Keynesian Episode: A~Reassessment}. [online] Indianapolis: Liberty Press. Available at: {\textless}http://archive.org/details/keynesianepisode0000hutt{\textgreater} [Accessed 1 October 2024].



Imbert, A., 1913. \textit{Frédéric Bastiat et le socialisme de son temps}. Marseille: Barlatier.



Keynes, J.M., 1936. \textit{The General Theory of Employment, Interest and Money}. London: Macmillan.



Kramer, M.H., 1999. \textit{In Defense of Legal Positivism: Law Without Trimmings}. Oxford; New York: Oxford University Press.



Levy, A., 2021. Idealization and abstraction: refining the distinction. \textit{Synthese}, [online] 198(24), pp.5855–5872. https://doi.org/10.1007/s11229-018-1721-z.



Locke, J., 1689. \textit{Two Treatises of Government in the Former, the False Principles and Foundation of Sir Robert Filmer and His Followers Are Detected and Overthrown, the Latter Is an Essay Concerning the True Original, Extent, and End of Civil Government.} [online] London: Printed for Awnsham Churchill at the Black Swan in Ave-Mary-Lane. Available at: {\textless}https://name.umdl.umich.edu/A48901.0001.001{\textgreater}.



Magni, C.A., 2009. Splitting up Value: A~Critical Review of Residual Income Theories. \textit{European Journal of Operational Research}, [online] 198(1), pp.1–22. https://doi.org/10.1016/j.ejor.2008.09.018.



Marmor, A., 2001. \textit{Positive Law and Objective Values}. Oxford; New York: Clarendon Press.



Marx, K., 1875. \textit{Kritik des Gothaer Programms [Critique of the Gotha Programme]}. [online] Available at: {\textless}https://www.marxists.org/archive/marx/works/1875/gotha/index.htm{\textgreater} [Accessed 2 October 2024].



Marx, K. and Engels, F., 1848. \textit{Manifest der Kommunistischen Partei: Veröffentlicht im Februar 1848 [The Communist Manifesto]}. [online] London: Gedruckt in der Office der ‘Bildungs-Gesellschaft für Arbeiter' von I. E. Burghard. Available at: {\textless}https://oll.libertyfund.org/pages/marx-manifest{\textgreater} [Accessed 2 October 2024].



Marx, K. and Engels, F., 2010. \textit{The Communist Manifesto}. Vintage classics. Vintage Books.



McGee, R.W., 2014a. \textit{Economic Protectionism and the Philosophy of Frédéric Bastiat}. https://doi.org/10.2139/ssrn.2435943.



McGee, R.W., 2014b. \textit{Keynes, Bastiat and the Multiplier}. https://doi.org/10.2139/ssrn.2435914.



McGee, R.W., 2014c. \textit{The Relevance of Frédéric Bastiat to Twenty-First Century Economic Thought}. https://doi.org/10.2139/ssrn.2435731.



Menger, C., 1871. \textit{Grundsätze der Volkswirtschaftslehre}. [online] Wien: Wilhelm Braumüller. Available at: {\textless}https://oll.libertyfund.org/titles/menger-grundsatze-der-volkswirtschaftslehre{\textgreater} [Accessed 2 October 2024].



Meredith, C.T., 2009. Taxation and Legal Plunder in the Thought of Fric Bastiat. \textit{Journal of Markets \& Morality}, [online] 12(2), pp.297–314. Available at: {\textless}https://www.marketsandmorality.com/index.php/mandm/article/view/136{\textgreater} [Accessed 2 October 2024].



Mülberger, A., 1896. \textit{Kapital und Zins: Die Polemik zwischen Bastiat u. Proudhon}. [online] Jena: Verlag von Gustav Fischer. https://doi.org/10.11588/diglit.22431.



Nouvion, G. de, 1905. \textit{Frédéric Bastiat: sa vie, ses oeuvres, ses doctrines: monopole et liberté}. Paris: Guillaumin.



Nozick, R., 1974. \textit{Anarchy, State, and Utopia}. Oxford: Blackwell Publishing.



Ohashi, T.M. and Roth, T.P., 1980. \textit{Privatization, Theory and Practice: Distributing Shares in Private and Public Enterprise}. Vancouver: The Fraser Institute.



Pirie, M., 1988. \textit{Privatization: Theory, Practice and Choice}. Aldershot, UK: Wildwood House.



Richman, S., 1998. Foreward. In: \textit{The Law}, 2\textsuperscript{nd} ed. Irvington-on-Hudson, N.Y.: Foundation for Economic Education. pp.ix–xiv.



Roche, G.C.I., 1971. \textit{Frederic Bastiat; a~Man Alone}. Architects of freedom series. New Rochelle, N.Y: Arlington House.



Roche, G.C.I., 1993. \textit{Free Markets, Free Men: Frederic Bastiat, 1801-1850}. [online] Hillsdale, MI: Hillsdale College Press; The Foundation for Economic Education. Available at: {\textless}http://archive.org/details/freemarketsfreem0000roch{\textgreater} [Accessed 2 October 2024].



Ronce, P., 1905. \textit{Frédéric Bastiat: sa vie, son œuvre}. Paris: Guillaumin.



Rothbard, M.N., 1995. \textit{Economic Thought Before Adam Smith}. Cheltenham, UK \& Northampton, MA, USA: Edward Elgar Publishing.



Rothbard, M.N., 1999. A.R.J. Turgot: Brief, Lucid, and Brilliant. In: R.G. Holcombe, ed. \textit{15 Great Austrian Economists}. Auburn, AL: Ludwig von Mises Institute. pp.29–44.



Russell, D., 1959. \textit{Frédéric Bastiat and the Free Trade Movement in France and England, 1840-1850}. PhD dissertation. Université de Genève.



Russell, D., 1969. \textit{Frederic Bastiat: Ideas and Influence}. Irvington-on-Hudson, NY: Foundation for Economic Education.



Russell, D., 1985. \textit{Government and Legal Plunder: Bastiat Brought up to Date}. Irvington-on-Hudson, NY: Foundation for Economic Education.



Savas, E.S., 1982. \textit{Privatizing the Public Sector: How to Shrink Grovernment}. Chatham, NJ: Chatham House Publ.



Schumpeter, J.A., 1954. \textit{History of Economic Analysis}. Oxford; New York: Oxford University Press.



Screpanti, E. and Zamagni, S., 1993. \textit{An Outline of the History of Economic Thought}. Translated by D. Field Oxford: Clarendon Press.



Skousen, M., 1992. \textit{Dissent on Keynes: A~Critical Appraisal of Keynesian Economics}. New York; Westport, CT and London: Praeger.



Skousen, M., 2001. \textit{The Making of Modern Economics: The Lives and Ideas of the Great Thinkers}. [online] Armonk, NY: M.E. Sharpe. Available at: {\textless}http://archive.org/details/makingofmodernec0000skou{\textgreater} [Accessed 2 October 2024].



Terborgh, G., 1968. \textit{The New Economics}. Washington, DC: Machinery and Allied Products Institute.



Waldron, J. ed., 1987. \textit{Nonsense Upon Stilts: Bentham, Burke and Marx on the Rights of Man}. London: Methuen \& Co. Ltd.



Webber, C. and Wildavsky, A.B., 1986. \textit{A~History of Taxation and Expenditure in the Western World}. New York: Simon and Schuster.



Weissman, S., 2023. \textit{How Has Art Been Used as Propaganda?} [online] Owlcation. Available at: {\textless}https://owlcation.com/humanities/How-has-art-been-used-as-propaganda{\textgreater} [Accessed 2 October 2024].



Zernike, K., 2010. Shaping Tea Party Passion Into Campaign Force. \textit{The New York Times}, [online] 26 Aug., pp.A1, 16. Available at: {\textless}https://www.nytimes.com/2010/08/26/us/politics/26freedom.html{\textgreater} [Accessed 2 October 2024].



\end{document}

